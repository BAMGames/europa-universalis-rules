% -*- mode: LaTeX; -*-

%\definechapterbackground{Exchequer}{}
\chapter{Exchequer test}\label{chapter:Budget}

\section{Overview of the phase}

% RaW: [50]
\aparag[Administration] At the end of the turn, final administrative actions
are resolved and budgets must be completed. First, exceptional taxes that were
scheduled during the administrative phase are resolved. Then comes the
exchequer test. At this point, players roll to determine how well the funds
were collected this turn and to discover their precise income. If the income
is not enough to cover for the expenses, loans must be contracted, either from
the people of your country or from international bankers. Last but not least,
countries may try to improve their \STAB.

\aparag[Sequence of the Exchequer Phase.]
\ExchequerDetails

\section{Exceptional taxes}\label{chBudget:Exceptional taxes}
\aparag[Exceptional taxes] Exceptional taxes are scheduled during the
Administrative phase. See~\ref{chAdministration:Exceptional Taxes} for details
(and examples). They are resolved at this point only. That is, until the end
of the turn (and after most expenses have been planned), players won't know
exactly the amount of collected taxes.
\bparag Note that Exceptional taxes must be planned during Administrative
phase. If a country forfeited the possibility to do so, it is to late now to
decide to raise taxes.

\aparag[Resolution of the taxes]
\bparag Each country which has planned taxes should have written a modifier in
\lignebudget{Exceptional taxes modifier A} (copied from
\lignebudget{Exceptional taxes modifier B}). This modifier was \ADM + 3
$\times$ \STAB (at the time of the Administrative phase).
\bparag Roll 1d10, add the modifier and multiply the result by 10. This is the
amount of taxes (in \ducats).
\bparag Write this amount in \lignebudget{Exceptional taxes}. It may well be
negative if the modifier was negative. In this case, the country will actually
lose money because of the taxes. It is not possible to refuse a ``tax'' once
the amount is known.

\aparag[\RT before Exchequer test]
\bparag Players can know compute their \RT before resolving the Exchequer
test.
\bparag This is the sum of lines \ERSlong{RT after Diplomacy} +
\ERSlong{Pillages, privateers} + \ERSlong{Gold from ROTW and Convoys} +
\ERSlong{Exceptional taxes} of \EcoRS. It is written in \lignebudgetlong{RT
  before Exchequer}.
\bparag Players should also copy \lignebudgetlong{Gross income B} in
\lignebudgetlong{Gross income A} and \lignebudgetlong{Total expenses} in
\lignebudgetlong{Expenses}.

\section{Exchequer test}\label{chBudget:Exchequer test}
\subsection{Gross Income}
\begin{designnote}
  We explain here the technical rules of the economical system. For a
  description of the spirit of these rules, see~\ref{chThePowers:Exchequer}.

  The rules here are quite ``algorithmic'' in order to have them as precise as
  possible and avoid misinterpretations. Thus, there are not well suited to
  understand the whys of the system (only the hows). These rules are meant to
  be closely followed step by step. Check~\ref{chThePowers:Exchequer} in order
  to understand what should happen, as well as read some examples.
\end{designnote}

\aparag[Exchequer test] Each country roll a die on~\ref{table:Administrative
  Actions} modified as follows (cumulative):
\begin{modlist}
\item[+2] If completely at Peace (no war (including civil or overseas wars),
  no intervention (limited or foreign)).
\item[-1 ] per 100\ducats of National Loan.
\item[-1] per ongoing International Loan (whatever the amount, including the
  ones that are partially refunded).
\item[-1 ] per bankruptcy in the last 5 turns.
\item[-1] per loan treaty broken in the last 5 turns.
\end{modlist}
\bparag Find the result by cross-referencing the line of the modified result
with the column equal to the \STAB of the country.
\bparag The result may be either F\textetoile, F, \undemi, \undemi\textetoile,
S or S\textetoile.

\begin{playtip}
  Bankruptcies should be noted by a small \textetoile in \lignebudget{Gross
    income A} for the turns where they affect the Exchequer test.
\end{playtip}

\aparag[Percentages] By cross-referencing this result with the first three
columns of~\ref{table:Exchequer test}, countries obtain three percentages for
``Regular Income'', ``Prestige Income'' and ``National Loan''.
\bparag Add 10 to the ``National Loan'' of countries that are not completely
at peace.
\bparag Add 10 (cumulative) to the ``National Loan'' of \HIS if it has
declared a politic of expulsions (see~\ref{chSpecific:Spain:Expulsion}).
\bparag It is possible and intended that these percentages sum up to more or
less than 100\%.

\aparag[Incomes] Apply each of the three percentages to the whole Gross Income
(\lignebudget{Gross income A}), rounding down, to obtain three incomes.
\bparag Copy these incomes in \lignebudgetlong{Regular income},
\lignebudgetlong{Prestige income} and \lignebudgetlong{Max. national loan}.

\begin{playtip}
  It is often convenient to cut these three boxes in half (diagonally). After
  rolling the exchequer test, immediately copy the percentages in the top-left
  halves, this avoid forgetting the result. Next you can take your time to
  compute the actual value and write it in the bottom-right halves.
\end{playtip}

\GTtable{etatsauvrai}

\subsection{International Loans}\label{chBudget:International loans}
\aparag[Available money] The total amount of available money for international
loans is:
\bparag 50\ducats from the start (unspecified bankers).
\bparag Always add 50\ducats, or 100\ducats for the emperor (German bankers).
\bparag Always add 50\ducats, or 100\ducats for the diplomatic patron of
\paysGenes (Genoese bankers).
\bparag After~\ref{pIII:Amsterdam Stock Exchange} add 50\ducats, or 100\ducats
for \HOL.
\bparag After~\ref{pIV:London Stock Exchange} add 50\ducats, or 100\ducats
for \ANG.
\bparag Thus, the total available money will be between 150 and
350\ducats. Note that it does depend on the country, that is all the countries
have different loan capacities.

\aparag[International Loans test] Each country may roll a die
on~\ref{table:Administrative Actions} modified as follows:
\begin{modlist}
\item[+2] If completely at Peace (no war (including civil or overseas wars),
  no intervention (limited or foreign)).
\item[-1 ] per 100\ducats of National Loan.
\item[-1] per International Loan.
\item[-1 ] per bankruptcy in the last 5 turns.
\item[-1] per loan treaty broken in the last 5 turns.
\item[+1 ] if the country has a Stock Exchange (\HOL after~\ref{pIII:Amsterdam
    Stock Exchange} and \ANG after~\ref{pIV:London Stock Exchange}).
\end{modlist}
\bparag Find the result by cross-referencing the line of the modified result
with the column equal to the \STAB of the country.
\bparag The result may be either F\textetoile, F, \undemi, \undemi\textetoile,
S or S\textetoile.
\bparag Note that this roll is different from the Exchequer test. Do not use
the same roll for both the Exchequer test and the International Loans test as
this would increase the chances of extremely bad results.

\aparag[International Loan] By cross-referencing this result with the last
column of~\ref{table:Exchequer test}, countries obtain one percentages for
``International Loan''.
\bparag Apply this percentage to the total available money and copy the result
in \lignebudgetlong{Max. international loan}.

\begin{playtip}
  Often, International loans are not necessarily and this step may be skipped
  by most countries. It may be useful to start computing your budget (next
  step) before deciding whether to take an international loan or not. Hence,
  it is sometimes more fluent to start computing your budget and then possibly
  come back to looking at international loans. Since there is no new knowledge
  gained between the Exchequer test and the Budget, this does not change
  anything.

  If you wish to follow closely the order of the steps, you should, however,
  always roll for international loan preventively, thus avoiding bad
  surprises.

  Rolling for international loan do not force to take one. It is always
  possible to decline a new international loan after rolling the die and
  seeing the available amount.
\end{playtip}

\section{Budget}\label{chBudget:Budget}
\subsection{Expenses}
\aparag[Regular income] Write in \lignebudget{Remaining expenses} the
difference between \lignebudget{Expenses} and \lignebudget{Regular income}.
\bparag This may be a negative number in the rare case where the Regular
income is larger than the total expenses.

\aparag[Prestige income] Write in \lignebudget{from Prestige} any non-negative
number smaller than both \lignebudget{Prestige income} and
\lignebudget{Remaining expenses}.
\bparag Small value means that more money is spent for prestige \VPs and less
for day-to-day expenses. Those will be covered by loans or debt.

\begin{designnote}
  You cannot spend additional money for prestige (it must be non-negative).
  You cannot take more from prestige than the ``Prestige Income'' (smaller
  than \lignebudget{Prestige income}).  You cannot take more from prestige
  than what is left to pay after the regular income is spent (smaller than
  \lignebudget{Remaining expenses}).
\end{designnote}

\aparag[National Loans] Write in \lignebudget{from N. loan} any non-negative
number smaller than \lignebudget{Max. national loan}.
\bparag Copy this number in \lignebudget{New National loans}.

\begin{designnote}
  National Loans are not limited by expenses. However, you'll have to pay
  interest for them and maybe even refund your people someday.
\end{designnote}

\aparag[National Loans] Write in \lignebudget{from I. loan} any
non-negative number smaller than \lignebudget{Max. international loan}.
\bparag Copy this number in \lignebudget{New International loan}.
\bparag Copy this number in \lignebudget{International loans refunds},
\textbf{three turns} after the current one.
\bparag Copy 10\% of this number (round up) in \lignebudget{International
  loans interests} for the \textbf{next three turns}. If there is already a
number in one of these boxes, add the new value to it.
\bparag That is, you should write 3 interests (for the next three turns), and
one refund (for the same turn as the last interest).

\begin{playtip}
  International loans are usually a bad idea because of the scheduled
  mandatory refund. Use them only when in need.
\end{playtip}

\begin{exemple}
  A correctly filled new international loan (of 100\ducats, at turn $n$) over
  an existing one (of 200\ducats, from turn $n-2$):
  \begin{tabular}{|c|l|r|r|r|r|r|}
    \hline
    & Turn & $n-1$ & $n$ & $n+1$ & $n+2$ & $n+3$\\
    \hline
    1 & New International loan & & 100 & & &\\
    \hline
    2 & I. loan interest & 20 & 20 & 30 & 10 & 10\\
    \hline
    3 & I. loan refunds & & & 200 & & 100\\
    \hline
  \end{tabular}
\end{exemple}

\aparag[New \RT]
\bparag Write in \lignebudget{RT balance} the sum of \lignebudget{from
  Prestige} + \lignebudget{from N. loan} + \lignebudget{from I. loan}
\textbf{minus} \lignebudget{Remaining expenses}. It may be negative if
\lignebudget{Remaining expenses} is too big.
\bparag Write in \lignebudget{RT after Exchequer test} the sum of
\lignebudget{RT before Exchequer} + \lignebudget{RT balance}.

\begin{designnote}
  \lignebudgetlong{Remaining expenses} depict \textbf{expenses} that are left
  to be paid after using the Regular income. Hence it is subtracted from the
  \RT while other lines are added (they are money taken from prestige or loan
  in order to fill the treasury).

  If \lignebudget{Remaining expenses} is \emph{negative}, regular income was
  enough to cover all expenses. Then, the surplus is added to the treasury (as
  subtracting a negative number result in an addition).
\end{designnote}

\begin{designnote}
  All in all, do not try to understand all the steps here while reading the
  rules. After a couple of turns of computing your budget, things will become
  more natural. Note that if you are having a ``teaching session'', you should
  try several ``stupid'' things with your budget to see the consequences.
\end{designnote}

\begin{playtip}
  When planning expenses, it is obviously a good idea to keep an eye on the
  possible income\ldots Too many expenses result in bankruptcy while too few
  result in money ``wasted'' for prestige (instead of being use for buying
  troops or waging war).

  Here are some guidelines in preparing your budget:
  
  First, check in the administrative actions table what are the possible and
  plausible results with respect to your current (and expected) \STAB. You may
  discard very unlikely results (with only 10\% chance of happening) but you
  know you take a risk doing so. It is especially important to take into
  account the worse possible result you may obtain if you want to limit risks.

  Second, check in the Exchequer test table the sum of percentages these
  results produce. Check separately the sum of Regular + Prestige income
  (income without debt) and the sum of the three percentages (income with
  debt). Applying these percentage to your Gross Income will give some amount
  of money.

  Do not spend more than your best income with debt, obviously, doing so
  will result in problems. Spending more than the worse income with debt means
  taking risks. Estimate the risks (Is it a 10\% or 30\% chance of getting the
  worse result?) compared to the situation (Do you have lot of money in your
  \RT to handle the loss?) and the expected gain (Will the extra expense allow
  you to win the war?)

  Spending less than the worse income without debt means that some money will
  necessarily go into Prestige \VPs. Are you sure it won't be better used for
  troops, economical development, \ldots? Spending less that the best income
  without debt means that you may get Prestige \VPs but they are not
  guaranteed either.

  The good cases is when the worst income with debt is roughly equal (or
  larger) to the best income without debt. Spending that amount of money means
  that the worse that can happen is to take a new loan (that can be handled
  later) and that you won't waste too much money on Prestige. Note that you
  have to plan your administrative actions and loan refund before the military
  phase, thus without knowing precisely how long the turn will last and how
  much you'll spend for moving troops (especially if at war). Thus, there is
  often some risk involved\ldots

  \smallskip

  Remember that the economical system works best if you have some loan that
  you refund and recontract immediately (for a net effect of transferring
  Prestige income into the \RT). If you plan to use this loan trick, then the
  amount of loan involved is not really a debt, that is increase you income
  without debt by this amount when planning your expenses.

  \smallskip

  Remember that the worse that can happen is a \RT collapse. But even for that
  you need several turns of bad luck, bad management, or bad wars. Thus, don't
  be afraid of making too big errors with the economical system. You should
  get the hand of it before catastrophic results occur\ldots
\end{playtip}

\begin{exemple}
  If your \STAB is +2 and your are at peace (\bonus{+2} to Exchequer test),
  then you'll likely to get \undemi\textetoile, S or S\textetoile (with only
  10\% chance of \undemi). \undemi\textetoile has 100\% income with debt while
  S\textetoile has 100\% income without debt. Thus, by spending as much as
  your Gross income, you're almost guaranteed to be able to cover your
  expenses, maybe with some new loans. There is a small risk (10\%) of a bad
  result (\undemi) that will leave you with only 80\% income. Estimate the
  risk versus gain for the last 20\% of expenses. On the other hand, a good
  result gives you up to 120\% with debt, hence some choice on whether to
  contract loan in order to get more Prestige.

  \smallskip

  If your \STAB is -2 and you roll at \bonus{-3} due to heavy loans or
  previous bankruptcies, then the likely result are F\textetoile, F or
  \undemi\ (disregarding the unlikely \undemi\textetoile). If you are at war,
  the income with debt for F\textetoile is 80\%, and the income with debt of
  \undemi\ is 90\%. Thus by spending around 80\% of your Gross Income, you're
  sure to be able to fill your budget with some loan. But you're also sure to
  need some new loan\ldots (and a good surprise may arise in the form of
  \undemi\textetoile).

  \smallskip

  Note that the true difference in the table is between \undemi\ (only 50/80\%
  of the total) and \undemi\textetoile (70/100\%). Especially, being at peace
  with a \STAB of +3 guarantees a good result.
\end{exemple}

\subsection{Loan Management}
\aparag Players must then correctly take care of their loans for the next
turn.

\aparag[International loans]
\bparag Since the interests are not changed by partial refund of the capital,
management of the international loans is entirely done during the
administrative phase (when bankrupting or refunding) and the budget segment
(for new loans).

\aparag[National loans]
\bparag Compute in \lignebudget{National loans at end} the difference between
\lignebudget{National loans at start}, minus \lignebudget{National loans
  bankruptcy}, minus \lignebudget{National loans refunds} and add
\lignebudget{New National loans}.
\bparag Report this number in \lignebudget{National loans at start} of the
next turn.

\subsection{Prestige and Wealth}\label{chBudget:Prestige and Wealth}
\aparag[Wealth] During each period, a global wealth is computed for each
country. Wealth represent the overall economical situation of the country, as
well as exceptionally good management (in the form of Prestige).
\bparag At the end of each period, wealth is converted into \VPs. Each country
has a different rate of exchange of wealth for \VPs as each country has
different typical economical situation.
\bparag All in all, each country is expected to score around 100\VPs for
wealth each period, give or take a few dozens if this is supposed to be a
period of glory or decay.

\aparag[Prestige] Write in \lignebudget{Prestige VPs} the difference between
\lignebudgetlong{Prestige income} minus \lignebudgetlong{from Prestige}. That
is the remaining Prestige income that was not spend for covering daily
expenses.

\aparag[Wealth] Turn wealth is the sum of the Gross income and the Prestige
\VPs. Period wealth is the sum of all turn wealth over all the period.
\bparag Write in \lignebudget{Wealth} the sum of \lignebudget{Gross income A}
and \lignebudget{Prestige VPs}.
\bparag Write in \lignebudget{Period wealth} the sum of \lignebudget{Period
  wealth} of the previous turn and \lignebudget{Wealth} of the current turn.
\bparag Exception: If this is the first turn of a period, simply copy
\lignebudget{Wealth} into {Period wealth}. That is, period wealth is reseted
at each period.

\section{Stability Improvement}\label{chBudget:Stability Improvement}
\aparag[Stability] A country may attempt to improve its Stability, but this is
never mandatory. As many actions, Stability improvement requires an investment
and is resolved by a die roll. Beware that in some situations the result may
be negative and cannot be forfeited once the die has been rolled.
\bparag Countries whose monarch was just overthrown due to revolts (see
\ref{chRedep:Execution Monarch by Revolts}) may not do a \STAB improvement
action this turn.

\aparag[Investment] Each player wanting to improve the Stability of his
country first chooses an investment and writes it in \lignebudget{Stability
  improvement}. As for administrative actions, higher investments give bonuses
to the roll.
\bparag The investment are:
\begin{modlist}
  \item Basic Investment: 30 \ducats
  \item Medium (\bonus{+2} to the die-roll): 50 \ducats
  \item Strong (\bonus{+5} to the die-roll): 100 \ducats
\end{modlist}

\aparag[Procedure]
This action is resolved without requiring a table.  The player rolls a die
modified as follows (all modifiers are cumulative):
\begin{modlist}
  \item[+?] \ADM monarch.
  \item[+2/5] if medium/strong investment.
  \item[+2] if the country was victim of a declaration of war this turn
    without having broken an alliance or declared a war itself.
  \item[-3] if the country is at war with at least one major country
    (including overseas wars but excluding interventions).
  \item[-2] if the country is at war with at least one minor country and no
    major country (including overseas wars but excluding interventions).
  \item[-5] if an enemy \ARMY counter is in an owned national province and
    controls the city (not applicable during a Religious/Civil War, do not
    count revolt and rebel troops).
  \item[-3] Exception: for \SPA, the malus for having an enemy \ARMY counter
    controling the city, is -3 only, however it applies for any owned
    territory (not only its national territory). This specificity ends with
    \eventref{pIV:Olivares} (if effects are applied), or with
    \eventref{pV:WoSS} (whatever the choices and outcomes).
  \item[+3] for a Prosperous Power (see below).
  \item[-3] for an Anti-Prosperous Power (see below).
  \item[$\pm$?] by event.
\end{modlist}

\begin{designnote}[Spanish empire]
  The early Spanish empire was more of a multicultural empire including both
  Spain, Italy and the Netherlands than a modern country. Hence, occupying
  any part of the empire will hurt some people (and hamper Stability). There
  is no real notion of national territory to defend at all cost opposed to
  more distant vassals and ``colony''. However, only part of the empire is
  shocked by the war, thus the malus is smaller. Olivares policies recentred
  the empire on Spain, making it more like other European powers of the time.
\end{designnote}

\aparag[Result] If the modified result is equal to:
\begin{modlist}
  \item[5-] the Stability \textbf{decreases} by 1.
  \item[6-10] Nothing changes.
  \item[11-14] the Stability increases by 1.
  \item[15-17] the Stability increases by 2.
  \item[18+] the Stability increases by 3.
\end{modlist}
\bparag Reminder: Stability varies from -3 to +3. It is not possible to
decline the result (especially the loss of Stability) once the die has been
rolled.
\bparag Stability is recorded on the Stability track on the \ROTW map. Move
the Stability marker according to the result of the action.

\aparag[Prosperity]\label{chBudget:Prosperity} tracks the evolution of the
Gross income (as recorded in \lignebudget{Gross income A}). A regular increase
of the Gross income will make people happy and ease Stability improvement, a
regular decrease will make people unhappy.
\bparag[Prosperous Power] A country is \terme{Prosperous} if its Gross income
has not decreased during the last 2 consecutive turns and progressed during at
least one of those turns.
\bparag[Anti-Prosperous Power:] A country is \terme{Anti-Prosperous} if its
Gross Income has decreased 2 consecutive turns.

\begin{exemple}[Prosperity]
  If the Gross income for the last two turns and the current one are:
  \begin{itemize}
  \item 100, 110, 120: the country is prosperous.
  \item 100, 100, 101: the country is prosperous (no decrease, at least one
    increase).
  \item 100, 110, 109: nothing (one decrease prevents prosperity even if the
    final result is higher than two turns earlier)
  \item 100, 99, 98: the country is anti-prosperous.
  \item 100, 99, 99: nothing (one stagnation prevents anti-prosperity).
  \end{itemize}

\end{exemple}

% Local Variables:
% fill-column: 78
% coding: utf-8-unix
% mode-require-final-newline: t
% mode: flyspell
% ispell-local-dictionary: "british"
% End:
