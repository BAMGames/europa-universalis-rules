\definechapterbackground{Incomes}{incomes}
\chapter{Incomes}\label{chapter:Incomes}
\section{Incomes}
\aparag[Overview]
The Income segment of the administrative phase is detailed here. It is
played mostly independently (the only information that one may need from
the other players is their \terme{Gross Land Income} if they are at war
or doing Trade Refusal). Each player will compute the income of their
country, coming from various sources.%, and manage their loans.
% (Jym) Now an admin action resolved later.
% If in a state of war, they also may levy exceptional taxes.
All this is summed up to form their new \RT. It helps to refer to the
\terme{Economic Record Sheet (B)} to better understand this Chapter.

The \terme{Economic Record Sheet (B)} serves as a register for all
financial operations of the country. The \RT and on-going loans are
stored (and computed) on the other ERS (A and C).
%can be computed there  after each phase.

Computation of income is only the first segment of the administrative
phase. However, its rules are separated from the rest of the phase for
clarity.

Part of the income is differed (gold form \ROTW, Convoys, Exceptional
taxes) and only perceived at the end of the turn (usually with some
hazards on the way). They are however briefly described here in order to
have all the income sources together. Check the corresponding Chapters
for a complete definition of these incomes.

\section{Land income}\label{chIncomes:LandIncome}
\aparag The player registers three kinds of land income:
\begin{itemize}
\item The income of all his provinces in \lignebudget{Provinces income};
\item The income of all the provinces of his vassal minor countries in
  \lignebudget{Vassal provinces income};
\item In negative, all the provinces (either his of his vassals') that
  are inexploitable (due to revolts, military occupation, looting, enemy
  control, or corrupted pashas) are recorded in
  \lignebudget{Occupation, Pillages, Revolts}.
\item Some random events (e.g.~\eventref{eco:Agricultural Crisis}) mark their
gains or losses in \lignebudget{Event}.
\end{itemize}
\bparag The sum of all this is the \terme{Land Income}, and goes in
\lignebudget{Land income}.

\begin{todo}
  Add partial ERS to the examples.
\end{todo}

\begin{exemple}[Land income]
  All along, the examples will details the income phase of the first
  turn for \POR.

  At the beginning of turn 1, \POR owns the provinces of
  \province{Tras-os-Montes} (income 5), \provinceBeira (3),
  \provinceTejo (6), \provinceAlentejo (6), \provinceAlgarve (5),
  \provinceTanger (2) and \provinceAcores (2) (counted as an European
  province even if located on the \ROTW map). Thus, its \terme{Provinces
    incomes} is 29\ducats, written in line \ERSshort{Provinces
    income}.

  At the beginning of the game, \POR has no minor ally, especially no
  \VASSAL and no \terme{Vassal provinces income}. However, since the
  Diplomatic phase occurs before the income segment, it is possible that
  a lucky Diplomatic action succeeded in getting \POR a \VASSAL. This is
  unlikely and we'll suppose it did not happen. So \POR has no
  \terme{Vassal provinces income} and can leave line \ERSshort{Vassal
    provinces income} empty (or write 0 in it, but leaving it empty is
  usually more readable).

  Let's suppose that the events of turn 1 resulted in a revolt in
  \provinceBeira. Then \POR write -3 in line \ERSshort{Pillages,
    Revolts, Pashas}. Notice that it is easier to do this count in
  negative because the content of line \ERSshort{Provinces income}
  will usually be the same every turn (except when one annexes or loses
  provinces) while the revolts change almost every turn. This avoid
  tedious recomputation of incomes each turn.

  Thus, the \terme{Land income} of \POR is 29+0-3=26\ducats, written in
  line \ERSshort{Land income}.
\end{exemple}

% \NPincludeA4{accounting-income}{Economic Record sheet B}{economic-record-sheet}

\section{Industrial income}\label{chIncomes:IndustrialIncome}
\subsection{Manufactures}\label{chIncomes:Manufactures}
\aparag Manufactures represent industries of all types developed by the
player (triangular-shaped counters).
%\bparag These manufactures are created by the administrative operation
%of \terme{Manufacture Creation}. The front side of the counter is used
%to represent 1 level of manufacture of the given type, the back
%indicates that there are 2 levels of the same type.
\aparag The manufactures all have a \terme{fixed income}.
\bparag The sum of all the incomes of the manufactures goes in
\lignebudgetlong{Manufactures}
\bparag The income of resources exploited by various manufactures
(fishing, salt) is not recorded here.
\bparag A \MNU brings no income if the province it is located also
brings no income (due to military occupation, revolt, pillage,
corrupted pashas, \ldots)

\aparag All manufactures bring other advantages than their incomes.
\bparag If there is a \textetoile on the counter (\RES{Metal},
\RES{Instruments} and \RES{Art} manufactures), only one such \MNU per
country provides the bonus (that is, only consider the \MNU of this type
of higher level).
\bparag Otherwise, the advantage is summed for all manufactures of this
type.
\bparag Quick summary of the advantages: \RES{Metal} \MNU enhance land
technology research ; \RES{Instruments} \MNU enhance naval technology
research ; \RES{Art} \MNU help increase \STAB ; \RES{Cloth} and \RES{Wine}
\MNU increase the income from foreign trade ; \RES{Cereals} \MNU increase
the income from domestic trade ; \RES{Salt}, \RES{Fish} and \RES{Wood}
\MNU produce resources of Salt, Fish or Wood.

\begin{exemple}[\MNU income]
  At turn 1, \POR has two \MNU. The \RES{Instruments} \MNU in
  \provinceTejo has an income of 7\ducats while the \RES{Wine} \MNU in
  \province{Tras-os-Montes} has an of 4\ducats.

  Since none of them are in the revolted province of \provinceBeira,
  both provide their income. Thus, the income is 7+4=11\ducats
  (written in line \ERSshort{Manufactures}).
\end{exemple}

\subsection{European Gold}
\aparag Each European mine brings a fixed income of 20\ducats as long as
that mine is not depleted (this may happen following~\eventref{eco:Mine
  Depletion}). The European mine income is registered on
\lignebudget{European mines}.
\bparag A mine brings no income if the province it is located also
brings no income (due to military occupation, revolt, pillage,
corrupted pashas, \ldots)
\aparag The income of mines located outside of Europe is processed
separately, because it has to be repatriated to Europe before it can be
credited to the player's treasury.
\bparag[Exception: \construction{Elmina}.] If \POR is a major country
and owns a \TP in \granderegion{Cotedor}, it exploits two gold mines
as European mines (for a total of 40\ducats). Destruction or loss of
ownership of this \TP definitely cancels this effect.

\begin{exemple}[European gold]
  As per Specific rules, \POR can exploit two gold mines in
  \construction{Elmina} as if it were European gold (see above
  and~\ref{chSpecific:Portugal:African Gold}). Thus, its \terme{European
    mines income} is 40\ducats (20\ducats per mine), recorded in line
  \ERSshort{European mines}.
\end{exemple}

\subsection{Industrial income}
\aparag The sum of all these incomes is put in \lignebudget{Industrial
  income}.

\begin{exemple}[Industrial income]
  The \terme{Industrial income} of \POR is 11+40=51\ducats.
\end{exemple}

\section{Trade income}\label{chIncomes:Trade Income}
\subsection{Domestic trade}
\aparag The Domestic Trade is computed by cross-indexing the
\terme{Total provinces Income} (the sum of \lignebudget{Provinces
  income} and \lignebudget{Vassal provinces income}, that is the
province income of both the country and its vassals without considering
pillages, revolts, military occupation or other hazards)
added to bonuses provided by \MNU against the \DTI of the country
in~\ref{table:Domestic Trade Income}.
%\bparag The \terme{Land Income} to use there is the sum of line~7
%and \lignebudget{9}.
\bparag Each level of \RES{Cereals} \MNU owned by the country adds
20\ducats to the \terme{Total provinces Income} for this computation
only.
% PB: TBD -
% \bparag A power having less than 6 (currently active) levels of
% commercial fleet of any nationality in its \CTZ (if any) has its \DTI
% reduced by one for this calculus (minimum is 1).
\bparag The result is put in \lignebudget{Domestic trade income}.

\GTtable{domestictrade}

\begin{exemple}[Domestic trade income]
  The \terme{Total provinces income} of \POR is 29+0=29\ducats (the
  revolt in \provinceBeira does not change it). Since \POR has no
  \RES{Cereals} \MNU, it stays unchanged. Thus, \POR will look in line
  1-40.

  The \DTI of \POR is 3, so \POR looks in column 3, the result is
  3\ducats written in line \ERSshort{Domestic trade income}.

  Notice that if \POR decide to build a \RES{Cereals} \MNU (and
  succeed), its Land income would become 29+20=49\ducats (for this
  computation) allowing it to look in the second line (41-80) for a
  total of 9\ducats. So, in addition to its fixed and variable incomes,
  this \MNU would bring 6\ducats of \terme{Domestic trade income} each
  turn and will quickly refund itself\ldots (it may, however, not be the
  best strategical choice to do immediately).
\end{exemple}

\subsection{Foreign trade}\label{chIncomes:Foreign Trade}
\aparag The Foreign Trade is computed according to the \terme{Blocked
  trade} and the \FTI of the country.

\aparag Each country has a \terme{Basic blocked trade} (corresponding to
its Domestic market), expressed below~\ref{table:Foreign Trade
  Income}.
\bparag For some countries, this is fixed and for some other it depends
on the provinces owned.
\bparag Note: countries not mentioned (\POL, \POR, \PRU, \VEN) have a
\terme{Basic blocked trade} of 0\ducats. \RUS also has a \terme{Basic
  blocked trade} of 0\ducats before it fulfils the condition indicated
in the table.

\aparag Each country has an \terme{Extra blocked trade} which is the sum
of its vassals income, trade refusal (including due to wars) and some
other events.
\bparag The \terme{Extra blocked trade} \textbf{only} is reduced by
50\ducats for each side of \RES{Wine} or \RES{Cloth} \MNU owned by the
country (thus 100\ducats for a level 2 \MNU).
\bparag The \terme{Extra blocked trade} can never be reduced below 0.

\aparag The sum of the \terme{Basic blocked trade} and the \terme{Extra
  blocked trade} is the \terme{Blocked trade}.
\bparag Note that the \terme{Basic blocked trade} is never
reduced. Thus, the \terme{Blocked trade} will always be at least equal
to the \terme{Basic blocked trade}.
\bparag Locate the line corresponding to the \terme{Blocked trade}
in~\ref{table:Foreign Trade Income}.
\bparag For each \pays{usa} that exists and is at peace, go up one
line in the table (several \pays{usa} may be created as result of
revolts in the colonies of other countries than \ANG).
\bparag Cross-referencing the line for the \terme{Blocked trade} with
the column corresponding to the \FTI of the country gives the
\terme{Foreign trade income}, to be put in \lignebudget{Foreign trade
  income}.

\GTtable{foreigntrade}

\begin{exemple}[Foreign trade income]
  \POR has a \terme{Basic blocked trade} of 0. Let's suppose it declared
  war on \paysMaroc on turn 1 and \TUR chose to defend \paysMaroc\ldots
  Then \paysMaroc is refusing trade to \POR, creating an \terme{Extra
    blocked trade} of 12\ducats (the sum of the incomes of its
  provinces). Similarly, \TUR refuses trade to \POR creating an
  \terme{Extra blocked trade} of 88\ducats. The total \terme{Extra
    blocked trade} of \POR is 12+88=100\ducats.

  However, \POR has a \RES{Wine} \MNU of level 1. This allows it to
  reduce its \terme{Extra blocked trade} by 50\ducats for a final value
  of 100-50=50\ducats.

  The \terme{Blocked trade} of \POR is 0+50=50\ducats, so it will look
  its \terme{Foreign trade income} in the second line of the table
  (50-99). Its \FTI is 2 (never use special \FTI here), so \POR looks in
  the second column and find the result of 54\ducats for its
  \terme{Foreign trade income}, written in line \ERSshort{Foreign
    trade income}.

  What happens with \TUR? At turn 1, it has a \terme{Basic blocked
    trade} of 100\ducats. \POR creates an \terme{Extra blocked trade} of
  29\ducats. Even if \TUR had a \RES{Cloth} \MNU (this is not the case
  at turn 1), it would only reduce the \textbf{Extra} blocked trade by
  50\ducats. So it will actually reduce it by 29\ducats and the extra
  21\ducats of bonus would be lost.
\end{exemple}

\subsection{Commercial fleets}
\subsubsection{Trading zones}
\aparag Each \TradeFLEET bears a letter for identification and has a
level between 1 and 6 recorded in the \terme{Commercial fleet table}
situated on bottom left of the \terme{Colonial record sheet}.
\bparag The counter is to be placed on its \Faceplus side if the level
is between 4 and 6, on its \Facemoins side otherwise.
\bparag Counters are placed in a \CTZ or a \STZ. Each country can have
only one \TradeFLEET per \CTZ or \STZ.
\bparag For easy reference, a global Trade fleet sheet is also provided
to record the level of each country in each sea zone. It is best kept by
a player with heavy commercial activity (\POR, \HOL or \ANG, usually).
\aparag A commercial fleet in a \STZ brings an income of 1\ducats per
level.
\bparag A commercial fleet in a \CTZ brings an income of 2\ducats per
level
\bparag[Exception: Baltic] Each level of \TradeFLEET in \stz{Baltique}
brings an income of 2\ducats, as if it were a \CTZ.
\bparag These incomes are reported in \lignebudget{STZ+CTZ level
  income}.

\begin{designnote}
  There is no CTZ for Sweden, Poland and Prussia, that went through the
  Baltic Sea for their trade. However, there was an intense flux of
  merchandise going through this area.Hence, \stz{Baltique} brings the
  same income as a \CTZ without being one (no country has a bonus for
  \TFI there).
\end{designnote}

%\bparag The economic value of the Baltic sea is 40\ducats. The
%\stz{Baltique} is represented as a \CTZ, and gives 2\ducats for each
%fleet level. Furthermore, it is bound to specific rules bound to the
%Sund taxes, see \ruleref{chSpecific:Sund Levies}.

\begin{exemple}[\TradeFLEET level income]
  At turn 1, \POR has a \TradeFLEET of level 3 in \stz{Canarias},
  bringing 3\ducats of \terme{Level income} (1\ducats per level in \STZ)
  and a \TradeFLEET of level 1 in \stz{Guinee} for another 1\ducats of
  \terme{Level income}.

  So, its total \terme{STZ+CTZ level income} is 3+1=4\ducats written in
  line \ERSshort{STZ+CTZ level income}.
\end{exemple}

\subsubsection{Monopolies}\label{chIncomes:Commercial Monopoly}
\aparag If in any \CTZ or \STZ, a country has a \TradeFLEET of level 6,
it has a \terme{total monopoly} and can register the sum inscribed in
large print in the \CTZ or \STZ symbol in \lignebudget{STZ+CTZ monopoly
  income}.
\bparag Note that in this case, no other country may have a \TradeFLEET
in this \CTZ or \STZ. See~\ref{chAdministration:Competition}.

\aparag Otherwise, if it has a \TradeFLEET\faceplus, it has a
\terme{partial monopoly} and can register half the sum inscribed in
large print in the \CTZ or \STZ symbol in \lignebudget{STZ+CTZ monopoly
  income} (round down).
\bparag Only one country may have a \TradeFLEET\faceplus in a given \CTZ
or \STZ. See~\ref{chAdministration:Competition}.

\aparag For each \CTZ or \STZ where a country has a
\TradeFLEET\facemoins, it can register the sum inscribed in small print
in the \CTZ or \STZ symbol in \lignebudget{STZ+CTZ monopoly income}
(presence bonus, it is thus usually more beneficial to open new
markets than reinforce old ones).
\bparag This sum is usually 1/10\th of the large sum, or 1/5\th on the
\ROTW map.

\aparag Players also register the \textbf{number} of partial and total
monopolies they have in Trade Zones as these bring \VPs. This is
recorded in \lignebudget{Partial/Total monopolies (trade)}. This is used
for \VPs computation.

\begin{exemple}[Monopolies income]
  \POR has no monopoly. However, it has \TradeFLEET in two \STZ and
  still gets a presence bonus equal to the number in small print in
  these \STZ. For \stz{Canarias}, this is 4\ducats while for
  \stz{Guinee}, this is 3\ducats. Thus, its \terme{STZ+CTZ monopoly
    income} is 4+3=7\ducats written in line \ERSshort{STZ+CTZ
    monopoly income}.

  Since it has no monopolies, it has nothing to write in line
  \ERSshort{Partial/Total monopolies (trade)}
\end{exemple}

\subsection{Trade centres and convoys}
\subsubsection{Trade centres}
\aparag Trade centres represent the main hubs of trade in selected areas
of the World. They are given to the country dominating the trade in
these areas (usually by having more \TradeFLEET) and bring a substantial
income. Trade centres must be located in a province of the owning
country.

\aparag There are four \terme{Trade Centres}, marked by counters:
\emph{Great Orient}, \emph{Atlantic}, \emph{Mediterranean} and
\emph{Indian}.
\bparag Position of the Trade Centres change during the Interphase (at
the end of turn). However, we remind here how they are attributed.
See~\ruleref{chInter:Trade Centres} for details.

\aparag The initial positions are \provinceNil for the \emph{Great
  Orient} centre, \provinceVeneto for the \emph{Mediterranean} centre,
\provinceVlaandern for the \emph{Atlantic} centre and \villeDiu for
the \emph{Indian} centre.

\aparag If the province in which the Trade Centre is located is
militarily occupied, the trade centre does not bring any income this
turn.
% (Jym), added to avoid too harsh situations
\bparag Revolts, pashas, pillages or other hazards do not impact the
Trade Centres income.

\subsubsection{The Great Orient centre}
\aparag The \emph{Great Orient} centre is initially located in
\provinceNil and moves to \provinceIzmir as soon as \provinceNil is not
own by \paysEgypte (usually at the time of conquest by \TUR).
\aparag The income of the \emph{Great Orient} centre is 100\ducats, plus
modifiers.
\bparag 10\ducats are added for every complete group of 3 non-European
\COL or \TP counters (any side).
\bparag Exception: If a minor is giving its colonial income to a
Christian major other than \VEN due to
\ref{chDiplo:AdenOmanExoticResources}, do not count its establishments
as non-European ones (don't count them as European ones either).
\bparag 10\ducats are subtracted per complete set of 5 \COL or \TP
counters (any side) in \continent{Asia} (except \continent{Siberia}) of
a Christian player (all but \TUR).
\bparag 10\ducats are added for Muslim control of \province{Ormuz};
10\ducats for \province{Socotra}; and 10\ducats for either
\province{Malacca S} or \province{Sumatra C}. For this rule, these
provinces are considered controlled, in decreasing order of precedence
by (i) a major (other than \VEN) having a \dipAT with a country with a
\COL/\TP in the province ; (ii) any country (major or minor) having a
\COL/\TP in the province ; (iii) Muslim controlled if the province is
empty.
% (control is: having a \TP or \COL or \dipAT of a power owning it).
\bparag 10\ducats are subtracted per Christian \TradeFLEET counter (any
side) in \stz{Arabie} and \stz{Indien}, with maximum of -50\ducats.
\bparag 50\ducats are subtracted if \TUR and \pays{perse} are at war.
\bparag It can never be negative. At worse, it becomes 0\ducats.
\bparag Therefore, the income at turn 1 of the \emph{Great Orient}
centre is 170\ducats, since there are 10 \TP of \pays{gujarat}, 1 \TP
and 1 \COL of \pays{aden} and 1 \COL of \pays{oman} and the 3 straits
are controlled by Muslims (\paysGujarat for \province{Malacca S} and
\provinceOrmuz and empty for \provinceSocotra).
\bparag Notice that if \POR signs an \dipAT with \paysAden, then its
\COL and \TP are not counted as ``non-European'' anymore (without being
counted as ``Christian''), thus there will only be 11 non-European
establishment instead of 13 and the income of the centre will drop to
160\ducats.
\bparag This income is registered in \lignebudget{Trade centres income}
of the country owning the centre.

\begin{histoire}
  All along the 16th century, there was a commercial (and sometime
  military) fight to bring the goods from Orient (pepper and other
  spices, silk, \ldots) to Europe. The historical road went mostly
  through land with the Silk road from China and the indo-arab trade in
  the Red sea. Portuguese opened a sea road going round Africa.

  The land road went through Egypt and Turkey, both of them raising
  heavy taxes on this trade represented by the Trade centre
  income. Venice was principally in charge of distributing it in the
  Mediterranean, again with heavy profit, thus building its commercial
  empire (by a long chain of consequences, the Indian monsoon actually
  dictated when the big fairs happened in Germany: the time it took
  for the spices to cross the Indian Ocean, the Red sea, be sold in
  Egypt, carried to Venice where the German merchants could buy them
  and cross the Alps back home\ldots)

  As soon as Christians found a new road for these good, they try to get
  rid of the Turks and Venetian in between. Portuguese, then Dutch and
  English merchants seized the spice trade.

  Thus, the ``land road'' income decrease with the number of Christian
  establishment buying or producing goods and sending them to Europe
  along the ``sea road''. Venice is an exception to this because it
  still wanted to use the ``land road'' rather than the ``sea
  road''. Minor countries tend to keep the old (land) road except if
  they have specific agreement with majors. Controlling the straits also
  allows to control the trade.

  The growth of the ``sea road'' is reflected in the appearance of the
  East Indies convoy described after the Trade centres.
\end{histoire}

\begin{designnote}
  Note that both \TUR and \VEN want the trade to use the land road
  (which they can tax as they are the end points). Thus, their control
  is treated in the way that make the income of the centre increase
  (\TUR because it's Muslim, \VEN because it's always an exception).

  Other countries want the trade to use the sea road (that they create
  and control) to avoid Turkish and Venetian taxes. Thus their control
  is treated in the way that make the income decrease. Their
  exploitation goes back home by sea and if they control the straits,
  they may close them to force the trade through the sea road.

  Large \ROTW countries (\paysMogols) are not Muslim in game. If they
  seize control of part of the trade, they will divert it for
  themselves rather than send it to Europa\ldots
\end{designnote}

\aparag As long as the \emph{Great Orient} centre is in \province{Nil}:
\bparag One half of its income is gained by \paysEgypte if at war ; or \VEN
otherwise.
\bparag The other half is gained by \paysDamas if at war ; or \TUR if
\paysDamas is either conquered or not at war and on the turkish
diplomatic track.

\aparag The Great Orient centre moves to \TUR when \paysEgypte is
conquered.

\subsubsection{Atlantic, Mediterranean and Indian Ocean centres}\label{chIncomes:Trade Centres}
\aparag The Atlantic and the Mediterranean trade centres bring a fixed
income to their owner of 100\ducats.
\bparag The Indian Ocean trade centre brings a fixed income to its owner
of 50\ducats.
%TBD: or 30\ducats if the \emph{Great Orient} centre has income 100\ducats or higher.
\bparag They are given to the country having the most levels of
\TradeFLEET in the corresponding \STZ and \CTZ.
See~\ruleref{chInter:Trade Centres} for details.
\bparag This income is registered in \lignebudget{Trade centres
  income}.

\subsubsection{Trade centres losses}
\aparag ``\terme{Guerre de course}'' is not reliable for a country which
is dominating trade in a region as privateers are likely to turn against
the most numerous vessels rather than against the sparser enemy ships.

\aparag If a \corsaire belonging to a country allied with the owner of
a Trade centre (including the owner himself) causes permanent losses on
an enemy \TradeFLEET in a \STZ or \CTZ belonging to this centre, then
the income of the centre is decreased by 10\ducats per permanent loss
caused for the next turn.
\bparag See~\ref{chRedep:Corsair Attack} for details.
\bparag Exception: \corsaire of \paysChevaliers or any \Barbaresques
never cause Trade centres losses.

\aparag This loss is recorded (in negative) in \lignebudget{Trade centre
  losses}.

\subsubsection{Convoys}
\aparag Convoys represent heavy trade of specific resources (gold or
spices). They are given to the country dominating trade in this resource
(usually by exploiting most of it). They do not bring income per se but
must be brought back to Europe where the gold they carry can be
unloaded. However, the journey can be dangerous and convoys can be
attacked and seized by pirates, privateers or enemy fleet.
\bparag Convoys are not taken into account during the Income
segment. However, they do bring income at the end of turn (especially
the Spanish gold fleets) so we remind here how they are
attributed.

\aparag There are four possible convoys: the \terme{Levant} fleet of
\bazar{Izmir}, the \terme{East Indies} convoy, the \terme{Flota de Oro},
and the \terme{Flota del Per\'u}.
\bparag Convoys are attributed during the Interphase.
\bparag Each convoy represents a certain number of ships carrying
gold. During the Military phase they must be moved toward Europe and can
be attacked by \corsaire or enemy \FLEET.
\bparag When a convoy safely reaches Europe, its gold is unloaded into
the country's \RT (\lignebudget{Gold from ROTW and Convoys}).

\aparag The \terme{Flota de Oro} and \terme{Flota del Per\'u} convoys are
given to \SPA when it exploits sufficiently many gold in
\continent{America}.

\aparag The \terme{Levant} convoy is given to \TUR as soon as it owns
the Great Orient trade centre. \TUR must send it to another player each
turn.

\aparag The \terme{East Indies} convoy is given to the country who
exploits the most of the following resources: \RES{Silk}, \RES{Product
  of Orient} and \RES{Spices} if it exploits at least 10 of them.

\subsection{Trade income}
\aparag The sum of all these incomes is put in \lignebudget{Trade
  income}.

\begin{exemple}[Trade income]
  \POR does not own any Trade centre at the beginning of the game, so it
  has nothing to write in lines \ERSshort{Trade centres income} and
  \ERSshort{Trade centre losses}.

  Thus, its \terme{Trade income} is 3+54+4+7+0-0=68\ducats written in
  line \ERSshort{Trade income}.
\end{exemple}

\section{Colonial income}\label{chIncomes:ColonialIncome}
\subsection{Colonies \& Trading-posts}
%\aparag Trading-posts are small outposts that allow the contact with
%local merchants, whereas colonies are large establishment with European
%implantation. All of those have a level between 1 and 6, recorded in the
%\terme{Colonial Record Sheet}.
%\bparag The counter is to be placed on its \Faceplus side if the level
%is between 4 and 6, on its \Facemoins side else.

\aparag Each \COL brings an income registered in \lignebudget{Colonies}:
\bparag A \COL\facemoins brings 1\ducats per level, and as many \ducats
as the income value of the \Area (first number, see
\ruleref{chBasics:ROTW Areas}).
\bparag A \COL\faceplus brings 1\ducats per level, and twice the income
value of the Area.
\bparag However, \COL exploiting gold do not provide income. See
\ruleref{chIncomes:Gold Colonies}.

\aparag Each \TP brings an income registered in \lignebudget{Trading
  posts}:
\bparag A \TP\facemoins brings 1\ducats.
\bparag A \TP\faceplus brings 2\ducats.

\aparag \COL and \TP do not bring any income if they are pillaged,
revolted or military occupied.

\aparag[] [BLP] If the path between a \COL or \TP and Europe goes
through a Fortified strait, the controller of the Strait may choose to
close it.
\bparag This has to be announced in the Diplomatic phase. This
immediately gives an \OCB to the owner of the establishment.
\bparag Establishment behind closed Straits do not bring any income
this turn. Neither regular income nor resource income.

\begin{exemple}[Colonial income]
  At turn 1, \POR has a \COL of level 3 in \province{Cabo Verde}
  bringing an income of 3 (level) + 1 (one time the income of the
  \granderegion{Cabo Verde} area) = 4\ducats written in line
  \ERSshort{Colonies}.

  It has a \TP of level 3 in \construction{Elmina}, bringing an income
  of 1\ducats (it is side \Facemoins and the income of
  \granderegionCotedor is not taken into account for \TP) written in
  line \ERSshort{Trading posts}.
\end{exemple}

\subsection{Exotic resources}
\label{chIncomes:Exotic ressources}
\aparag \COL and \TP, as well as certain \MNU in Europe can exploit a
limited number of exotic resources.
\bparag The income of the exotic resources is the product of the number
of exploited resources of each kind, multiplied by the price of each
resource (computed as per~\ref{chAdministration:ExoticResourcesPrices})
\bparag This is recorded in \lignebudget{Exotic resources}.
\bparag Players exploiting sufficiently many resources can speculate
to try and increase the price. See~\ref{chAdministration:Speculation}.

\begin{exemple}[Exotic resources income]
  At turn 1, the \TP in \construction{Elmina} can exploit 3 \RES{Slaves}
  as it is level 3 (and does so even if it produces gold as per
  Portuguese special rule). The initial price of \RES{Slaves} is
  2\ducats, so \POR gains 3$\times$2=6\ducats, written in
  \ERSshort{Exotic resources}.
\end{exemple}

\aparag[Manufactures] \MNU can have only 2 levels per counter.
\bparag A Fishery (\RES{Fish} \MNU) may only be built in a coastal
province. It exploits as many \RES{Fish} as its level (1 or 2).
\bparag A Salter (\RES{Salt} \MNU) may only be built in a province with
\RES{Salt} resource (the number indicating the quantity).
\bparag A Salter of level 1 exploits 1 \RES{Salt}.
\bparag A Salter of level 2 exploits all \RES{Salt} from the province (up to
3).
\bparag \textbf{Exception:} A Venetian Salter of level 2 in \provinceVeneto
exploits all the \RES{Salt} from Venetian coastal provinces touching
\regionMediterranee.

\aparag[Square resources.]
\bparag In \continent{Asia}, it requires 3 \TP levels or 2 \COL levels to
exploit 1 unit of \RES{Sugar}, \RES{Cotton} or \RES{Products of America},
without need for Slaves.
\bparag Outside of \continent{Asia}, it requires 2 \COL levels to exploit 1
unit of \RES{Sugar}, \RES{Cotton} or \RES{Products of America}. Furthermore,
those resources are subject to the need of Slaves in \continent{America} (see
\ruleref{chIncomes:NeedSlaves}).

\aparag[Fish.] It requires 1 \COL level to exploit 1 unit of \RES{Fish}.

\aparag[Fur.]
\bparag Each \TP level can exploit up to 2 units of \RES{Fur}.
\bparag One level of \COL can exploit all the \RES{Fur} in an
\Area. However, each \COL\faceplus in the \Area reduces the number of
available \RES{Fur} by 1 unit.

\aparag[Circled resources.] For all other resources (\RES{Products of Orient},
\RES{Salt}, \RES{Silk}, \RES{Slaves}, \RES{Spices}), one level of \COL or of
\TP exploits up to 1 unit of the resource.

\aparag[Gold.] See \ruleref{chIncomes:Gold Colonies} if there is a gold mine in
the province where a \COL is.

\aparag[Wood.] \RES{Wood} does not bring income as other resources, but
it brings advantages in naval constructions and can be sold to other
countries. See~\ref{chIncomes:Wood} for exploiting and selling
\RES{Wood} and~\ref{chLogistic:Effect of Wood Maintenance}
and~\ref{chLogistic:Effect of Wood Purchase} for the effects of
\RES{Wood}.

\aparag A \ROTW minor country will exploit resources if and only if it has \TP
or \COL to exploit them, following the same rules as major countries.

\aparag Exotic resources on the \ROTW map are shared for a whole \Area (see
\ruleref{chBasics:ROTW Areas}).; if there is only two resources of Spices in
an \Area, two outposts in different provinces of the same \Area will have to
share the exploitation.
\bparag If there is disagreement for the exploitation of resources, this
is settled through the \terme{competition mechanism}
(\ruleref{chAdministration:Competition}), that eliminates levels until
there can no more be disagreement.
\bparag Note, however, that the right to exploit a resource may change
only if there is a change of situation in the \Area (new level of
establishment, a country announce that it stop its exploitation,
\ldots)

\aparag[New exploitation] There are only a few cases where there might be a
disagreement for the exploitation of a resource: a new resource appeared
(through events, because of competition or military intervention in the
previous turn, because it is one of the resources that appears late), or the
number of levels available to exploit the resource changed in this turn.
\bparag Those two cases will lead to \terme{automatic competition}. At the end
of the administrative phase, a \terme{competition} will take place until there
are enough resources (or no more enough levels) for everybody to be satisfied
with the current attribution of resources.

\bparag If the whole market in an \Area is already attributed, there is
no \terme{automatic competition}. The players have to spend
\terme{competition actions} (see \ruleref{chAdministration:Competition})
to change the market repartition, or do it through the use of diplomacy
(exotic resources exploitation can change in the Diplomatic phase by a
simple announce).

\aparag[Slaves and plantations]\label{chIncomes:NeedSlaves}
Some resources require \RES{Slaves} to be exploited in plantations in
\continent{America} (the square ones: \RES{Sugar}, \RES{Products of
  America} and \RES{Cotton}). At least 1 unit of exploited Slaves is
required for each unit of exploited resource requiring slavery.
\bparag[Triangular trade] These units of \RES{Slaves} can come either
from the same country exploiting it (and gaining income both for the
\RES{Slave} and the other resource), from another major selling its
\RES{Slaves}, from minors allies or from contraband.
\bparag[Reselling of slaves] If a major country wants too sell part of
its \RES{Slaves} production, it is free to do so at any price. He must
perceive the sum during the diplomatic phase (written in
\lignebudget{Wood and Slaves}). However, the sold \RES{Slave} units
cannot be used in his own plantations (obviously). The country selling
its slaves still gets income from it. The country buying them can use
them for any square resource it chooses.
\bparag[Contraband of slaves] If a country has at least one level of
\TradeFLEET in \stz{Arabie} or \stz{Guinee}, or if another \MAJ in this
position gives him this right, he can use the contraband of
\RES{Slaves} for his colonies. He receives only half the usual income
(round down for each unit of the resource) for the exotic resources
exploited with contraband \RES{Slaves}.
%[TBD : Rajouter Aiguilles en zone de contrebande d'esclaves ? Jym: bof,
%pas vraiment dans les ``regions productrices'' historiques.]
% PB: Non
\bparag[International contraband] If a country needs \RES{Slaves} but
does not fill the conditions above, it still can use the contraband of
\RES{Slaves}, but he will receive no income for the exotic resources
exploited that way. The resources, however, are considered exploited for
price variation purpose (it is not possible to ``hold back''
exploitation).
\bparag[Slaves of allies] Minor countries can sell for free the
\RES{Slaves} they exploit to their Diplomatic patron
(e.g. \pays{portugal} during \eventref{pIII:Portuguese Annexation}).
\bparag Note that \RES{Slaves} always bring income to the country that
produce them. The sale of \RES{Slaves} happens independently of this
income and does not change it.

\aparag Players also register the \textbf{number} of partial and total
monopolies they have in Trade Zones as these bring \VPs. This is
recorded in \lignebudget{Partial/Total monopolies (resources)}.

\begin{exemple}[Using Slaves]
  In the late 17th century, \ANG exploits a total of 6 \RES{Sugar}
  (price 6) and 3 \RES{Product of America} (price 5). Thus, it should
  theoretically bring an income of 6$\times$6+3$\times$5=51\ducats.

  However, since all this is done in \continent{America}, \RES{Slaves}
  are needed to work in the plantations (\continent{Asia} had much more
  local population that was used as workers in the plantations). \ANG
  only exploits 4 \RES{Slaves} (price 7).

  If \ANG do not find anyone wanting to sell \RES{Slaves} and has no
  \TradeFLEET close to \continent{Africa}, it must use international
  contraband. Thus, its 4 \RES{Slaves} allow to exploit 4 \RES{Sugar}
  but the rest (2 \RES{Sugar} and 3 \RES{Product of America}) is
  lost. The income is thus 4$\times$7 (for the \RES{Slaves}) +
  4$\times$6 (for the \RES{Sugar} exploited with them) + 0 (for the
  other resources exploited with contraband \RES{Slaves}) = 52.

  Note that (i) \RES{Slaves} both brings income per se and allows other
  resource to bring income, making it a very valuable resource ; and
  (ii) \ANG cannot choose not to use contraband \RES{Slaves} and not to
  exploit the remaining resources (lower exploitation has a better
  chance of raising prices). If it has sufficient levels of \COL to
  exploit it, it must do so.

  Now, suppose that the treaty of Methuen has been signed. \paysPortugal
  is on the Diplomatic track of \ANG and gives its 2 \RES{Slaves} for
  free, for a total of 6 \RES{Slaves}. Only 3 resources still require
  \RES{Slaves}. \HIS agrees to sell 1 \RES{Slave} for 3\ducats. \ANG
  cannot find the last ones but has a \TradeFLEET in \stz{Guinee}
  allowing for direct contraband.

  Thus, the situation is now:\\
  For \HIS, the \RES{Slave} both brings an income of 7\ducats during
  incomes and a ``gift'' of 3\ducats during the Diplomatic phase.\\
  For \ANG, 3\ducats are payed to \HIS for 1 \RES{Slave} during the
  Diplomatic phase. So \ANG has a total of 7 \RES{Slaves} (4 of its
  owns, 2 of its minor ally and 1 brought to \HIS) and must use two from
  contraband for the last resources. It get to choose which resources
  use contraband \RES{Slave} and only brings half income, it is better
  to choose the cheapest one, in this case \RES{Product of America}.\\
  The final income for \ANG is 4$\times$7 (its \RES{Slaves}) +
  6$\times$6 (all the \RES{Sugar} is exploited normally) + 1$\times$5 (1
  \RES{Product of America} can be exploited) + 2$\times$2 (2
  \RES{Product of America} is exploited at half price, round down) =
  73\ducats.
\end{exemple}

\subsection{ROTW gold}
\begin{note} Gold may be produced by gold mines located in the \ROTW
  map. This gold is not registered in the country's income immediately,
  because it has first to be repatriated to Europe. But it is
  nevertheless produced during the income phase. The repatriation of the
  gold takes place during the Military phase.
\end{note}
\aparag To exploit a gold mine, a \COL (with any number of levels) has
to be in the province containing the mine symbol. If a country wishes to
exploit gold, it simply has to announce it during the Diplomatic phase.
\bparag A country is never obliged to exploit a mine even if it has a
Colony in that province.
\bparag Once announced, the exploitation is definitive until depletion
of the mine (by \eventref{eco:Mine Depletion}) and cannot be voluntarily
stopped.
\bparag A \ROTW gold mine produce 20\ducats worth of gold (or silver,
jewels, \ldots) each turn, except for the mine in \villeTenochtitlan
(\granderegionAzteca), producing 40\ducats, and the mine in
\bazarPotosi (\granderegionInca East), producing 50\ducats. Both
these incomes are recalled on the map.
\bparag\label{chIncomes:Gold Colonies} A \COL exploiting a mine produces no
other income, nor does it exploit Exotic Resources. The gold produced is
reported on the \terme{Colonial Record Sheet}.
\bparag No slaves are needed to exploit a gold mine.
\bparag Gold has a major influence on the variation of inflation,
see~\ref{chPeace:Increase Inflation}.

\aparag \label{chIncomes:Gold Production} Gold can be transported by
earth during the income phase or during the redeployment phase (see
\ruleref{chRedep:Gold Repatriation}) (or both).
\bparag During the income phase, Gold can be stored in any port \COL in
the \Area where it was produced or an adjacent one.
\bparag From the ports, the gold has to be repatriated to Europe using
\NTD, the \emph{Flota de Oro} convoy or the \emph{Flota del
  Per\'u}. Each \NTD can carry up to 15\ducats worth of gold (each
Transport point is worth 5\ducats).

\begin{exemple}[\ROTW income]
  Since \POR has a specific rule for its gold in \constructionElmina,
  there is no \ROTW gold for it.

  So, it's \terme{\ROTW Income} is 4+1+6 = 11\ducats.
\end{exemple}

\section{Other incomes}\label{chIncomes:OtherIncomes}
\subsection{Events \& diplomatic incomes}
\aparag Economic events may change the \RT. The economical events all
tell exactly at which point their effect goes in the \EcoRS. The
political events usually act between line \ERS{RT at start of turn} and
\ERS{RT after Events} while the economical ones usually act on lines
\ERS{RT after Events}, \ERS{Event} and \ERS{Events}.
\aparag Diplomatic events modify the \RT in two ways: expenses for the
diplomatic actions, subsidies and gifts or loans between major players.
The latter go in \lignebudget{Gifts and loans between players}, the
first and second in \lignebudget{Diplomatic actions},
\lignebudget{Diplomatic reactions} and \lignebudget{Subsidies and
  dowries}. Reimbursement of loans between major players also comes at
this point.

%\subsection{Loans}\label{chIncomes:LoansIncomes}
%\aparag[Rules for v1 of the economical system]
%Ignore the paragraph when using v2 comptability.
%\aparag There are two kind of loans that can be done by the player, that
%are not loans (and as such submitted to negotiations as specified in
%\ruleref{chDiplo:Alliance:Loan Treaty}. They are the national loans, and the
%international loans. The player can only subscribe to one single type of
%loan for one given turn. He must first determine for each one, in
%\tableref{table:Loans}, the offer that is made by the ``market''
%(capital, interest, duration).
%\bparag A loan may be refunded by a player at any time before the normal
%number of turns.
%\bparag Loans subscribed are recorded in \lignebudget, interests and
%capital reimbursements are recorded in \lignebudget{30}.
%\bparag The player is entitled to only one from the two loans and may
%see all possibilities before choosing.
%\bparag The duration of a loan entails the current turn. A 1 turn-long
%loan must therefore be reimbursed on the turn following its subscription.
%\GTtable{loans}
% TODO : delete this. Any way, the loan table disappears in accountingv2
%\begin{tablehere}
%\loantable
%\end{tablehere}
%\aparag[National Loan] The player borrows the money mostly from the
%medium class of his home country. A National loan need not be refunded
%by the player that has subscribed it. If he does not refund the capital,
%he immediately loses 1 \STAB level and will receive the following penalties:
%\bparag No new national loan may be obtained until the end of the period.
%\bparag All new International loans receive a -3 die-roll penalty until
%the end of the current period.
%\bparag He has to pay one last time the interest on the loan.
%\aparag[International Loan] This type of loan must to be refunded in its
%entirety at the end of its length. An international loan can be refunded
%by another loan (of any type).
%\bparag If the player cannot refund the loan at the expiration of the
%last turn of the loan, or if he cannot pay interest at any turn, he
%suffers an immediate bankruptcy (\ruleref{chThePowers:Bankruptcy}).
%\aparag[Interest of a Loan] Interest is paid during each turn until the
%capital has been refunded entirely. Interest is based on the interest
%rate obtained for the loan. Every turn, the amount of interest is always
%based on l00\% of the subscribed capital, even if the principal has been
%partially refunded in previous turns.
%\bparag No interest is paid the turn the loan is received by the
%player. But he has to pay interest each following turn, even the turn he
%pays back all the capital (and even if he chooses to not refund a
%national loan).
%\aparag[End of game and Transfers] Any player that has not refunded any
%current International or National loan at the end of the game or at the
%moment he has to be transferred loses a number of \VP equal to the
%double of the \Ducats not refunded (principal and interest combined).
\subsection{Exceptional taxes}\label{chIncomes:Exceptional Taxes}
\aparag Exceptional taxes are an administrative operation. However,
since it brings money, the computation is recalled here.
\bparag Exceptional taxes being a domestic action, it cannot be
performed at the same time as another domestic action, and it might be
forbidden by bankruptcy.
\bparag Check~\ref{chAdministration:Exceptional Taxes} for details.

\aparag Exceptional taxes may be raised only if at war and if \STAB is
not -3.

\aparag[Summary.] To compute the exceptional taxes modifier:
\bparag First, lower \STAB by 1 (except if an enemy stack besiege or
occupy a province during a non-civil war).
\bparag Then, add 3 times the \STAB to the \ADM of the monarch.

\aparag The modifier is written in \lignebudget{Exceptional taxes
  modifier B} and copied in \lignebudget{Exceptional taxes modifier A}.
\bparag At the end of turn (only), roll 1d10, add the modifier and
multiply the result by 10.
\bparag This is the amount of \ducats gained (or lost in case of a
negative number) by the taxes.

\section{Income computation}\label{chIncomes:Summary}
\subsection{Gross income}
\aparag The \terme{Gross income} is the sum of the \terme{Land Income},
\terme{Industrial Income}, \terme{Trade Income} and \terme{Colonial
  Income}. It is written in \lignebudget{Gross income B} and copied in
\lignebudget{Gross income A}.
\bparag The \terme{Land Income} was defined in
\ruleref{chIncomes:LandIncome} and is the income of owned provinces.
\bparag The \terme{Industrial Income} is the sum of the various incomes
of \ruleref{chIncomes:IndustrialIncome}, i.e. Manufactures income and European
Gold.
\bparag The \terme{Trade Income} is the sum of the various incomes of
\ruleref{chIncomes:Trade Income}, i.e. Commercial fleets, Domestic Trade,
Foreign Trade and Trade Centres (but not Convoys).
\bparag The \terme{Colonial Income} is the sum of the various incomes of
\ruleref{chIncomes:ColonialIncome}, i.e. Colonies \& Trading-Posts income and
Exotic Resources exploitation, but not \ROTW gold.
\bparag[Stability] The Gross Income has an effect on \STAB at the end of
the turn (see \ruleref{chBudget:Prosperity}).

\begin{exemple}[Gross income]
  Thus, for \POR at turn 1 (with a \REVOLT in \provinceBeira and a war
  against \paysMaroc and \TUR), the results were :
  \begin{itemize}
  \item 26\ducats of \terme{Land Income} in \lignebudget{Land income} ;
  \item 51\ducats of \terme{Industrial Income} in
    \lignebudget{Industrial income} ;
  \item 68\ducats of \terme{Trade Income} in \lignebudget{Trade income} ;
  \item 11\ducats of \terme{\ROTW Income} in \lignebudget{ROTW income}.
  \end{itemize}
  For a grand total of 156\ducats written in \lignebudget{Gross income
    B}. This is not much and should increase quickly as the colonial
  empire expands. Maybe that war against \paysMaroc and \TUR is not such
  a great way to start the game\ldots
\end{exemple}

% Local Variables:
% fill-column: 78
% coding: utf-8-unix
% mode-require-final-newline: t
% mode: flyspell
% ispell-local-dictionary: "british"
% End:

% LocalWords:  addtolist compte pdf inexploitable AgriculturalCrisis ROTW
% LocalWords:  MineDepletion interphase