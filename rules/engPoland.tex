\sectionJ{\anchorpaysmajeur{Pologne}}{\blason{pologne}}

\subsection{The Polish Crown}
\subsubsection{Elective Monarchy}
\aparag[General modifiers.]
\bparag A modifier of {\bf -2} is applied to the die-roll to determine
the length of reign of a new Polish Monarch. This also increases the
probability of Dynastic Crisis.\label{chSpecific:Poland:King Duration}
\bparag A bonus of {\bf +1} is applied to the die-rolls to determine the
capacities of \Man and Shock as a general of a Polish Monarch.
\aparag[Generals as Monarch.]
\bparag Some generals can be elected as Kings: \leader{Bathory},
\leader{Sobieski}, \leader{Patkul}. Whenever \POL has to roll for a new
Monarch and one of these generals is in play, he can decide that the
general is elected as Monarch. This changes the Polish Dynasty.
\bparag When this happens, the new Monarch will last for the number of
turns remaining to the general (but he will test for survival now, from
the following turn on). The Monarch keeps his abilities as general.
\bparag If ever this Monarch is replaced due to an event, he goes back
to his normal general status (and is not killed).

\subsubsection{Particular Monarchs}
\aparag[\anchormonarque{John and Alexander}] are two successive kings
(considered as one for game purposes) in 1492.  He has values 4/5/4 and
is supposed to die at the beginning of turn 5. His heir is
\monarque{Zygmunt I}.
\aparag[\anchormonarque{Zygmunt I}] also called Sigismund I the Old has
values 8/7/8 and is supposed to last 8 turns. He will not test survival
for the first 5 turns. He cannot be used as a general.
\aparag[\anchormonarque{Bathory}] is a general \leaderwithdata{Bathory}
that can be elected as a Monarch. He has values 8/7/9.
\bparag Before he is elected, he can command only \LD, Ukrainian \ARMY
or Polish vassals.
\bparag After he is elected, these restrictions are removed.
\aparag[\anchormonarque{Sobiesky}] is a general
\leaderwithdata{Sobieski} can be elected as a Monarch. He has values
6/6/8.
\aparag[\anchormonarque{Zygmunt III}] is put on the Polish throne by
\eventref{pIII:Union Poland Sweden} with a new dynasty (the Wasa). He
has values 5/5/6 and is also general \leaderwithdata{Zygmunt III}. He is
supposed to last 9 turns.
\aparag[\anchormonarque{August II}] is put on the Polish throne by
\eventref{pV:Saxon King Poland}, with a new Dynasty. He will last 7
turns, but his values are to be determined randomly on the last column
of the table.
\aparag[\anchormonarque{Patkul}] is a general \leaderwithdata{Patkul}
that can be elected as a Monarch. He has values 5/9/4, and lasts from
turn 42 to 46. During his reign, \POL can not make any alliance with
\SUE, nor be in the same alliance as \SUE during a war.
\bparag[\anchorministre{Patkul}] Even if \leader{Patkul} is only a
general, he serves as a Minister giving a Diplomacy of 9. \POL may also
sign offensive limited alliances to enter a war against \SUE when
\leader{Patkul} is a Minister or a Monarch.
\aparag[\anchormonarque{Stanislas}] may be put on the Polish throne by
\subeventref{pVI:GNW:Polish Civil War}, with a new dynasty. He has
values 6/5/6 and will last for 1 to 6 turns.

\subsection{Political Disunity}
\subsubsection{Economical difficulties}
\aparag \POL has a malus of {\bf -1} to the die-rolls to implant
Manufactures or raise \FTI and \DTI.
\aparag \POL has no own \CTZ.
\aparag \POL may use only 3 counters of \TradeFLEET, excepted if it is
\terme{Protestant} in which case it can use all its 6 \TradeFLEET.
\aparag \POL is an Orthodox country regarding military technologies.
\subsubsection{The Union of Lublin}
\aparag\label{chSpecific:Poland:Before Lublin} In 1492, the player of \POL
controls forces of \paysmajeur{Pologne}, \anchorpaysmajeur{Lithuanie} and
\pays{Ukraine}. Those forces are distinct.
\bparag \POL has distinct free maintenance for each country.
\bparag \POL can raise forces only in their national provinces, or at
doubled cost in other provinces.
\bparag \POL has a no \CB if some national territory of \pays{Lithuanie}
is owned by other countries. But for other effects, provinces of both
countries of national provinces.
\bparag Generals may depend from one of these countries, and can lead
only forces of their country (or a multi-national stack). Other generals
(with no country specified) are only constrained by the Hierarchy rules.
\aparag[Union of Lublin.]\label{chSpecific:Poland:Union Lublin} It is
established by \eventref{pII:Union Lublin}.
\bparag \POL is now one country: every national provinces of
\pays{Pologne} and \pays{Lithuanie} are national provinces of \POL.
\bparag Units of \pays{Pologne} and \pays{Lithuanie} are no more
differentiated, and their \terme{basic forces} and their leaders are
associated.
\bparag Add {\bf +2} to the die-rolls for determining the length of
reign of a new Monarch (this effectively cancels
\ruleref{chSpecific:Poland:King Duration}).
\bparag Some limits of \POL are raised.
\bparag The Union of Lublin can be broken if a \MAJ imposes a peace of
level 3 or higher against \POL, and forfeits all conditions of peace in
order to break the Union. It is also broken if \POL is not Catholic.
\subsubsection{The two Polish Capitals}\label{chSpecific:Poland:Mazowia}
\aparag The Polish capital is initially set to \ville{Cracovie} in
\province{Malopolska}.
\aparag \province{Mazowia} is owned by \paysmajeur{Pologne} in 1492 as a
permanent Vassal but is not a national province of \paysmajeur{Pologne}.
\aparag During the reign of \monarque{Zygmunt I}, \POL can annex
\province{Mazowia} by using one Diplomatic actions and spending
100\ducats.
\bparag Alternatively, \POL can make a war against \pays{mazovie} (it
has no \CB). It annexes it by imposing an unconditional surrender on
it. In both cases, the province then becomes a national province of
\pays{Pologne}, and \pays{mazovie} disappears.
\aparag At any point after annexation of \province{Mazowia}, during
diplomacy phase, \POL can decide that \ville{Varsovie} is its new
capital. \POL gains {\bf 2} \STAB immediately. From now on,
\ville{Varsovie} and \ville{Cracovie} are both capitals of \POL for the
rules of peace, except that \province{Malopolska} can be annexed as the
result of Peace by another country, at which point it ceases to have a
Capital in there.
\subsubsection{Liberum Veto or Absolutism}
\aparag[Liberum Veto.]\label{chSpecific:Poland:Liberum Veto} Event
\eventref{pIV:Liberum Veto} has the following consequences:
\bparag It nullifies the {\bf +2} modifier to the die-roll for
determining the length of reign of a new Monarch given by the Union of
Lublin (the {\bf -2} malus thus resumes).
\bparag Declaration of war by \POL costs {\bf 3} \STAB without \CB, and
{\bf 2} \STAB with a \CB (and 0 with a free \CB).
\bparag \POL has a malus of {\bf -5} instead of {\bf -3} to raise its
\STAB if at war against a \MAJ at the end of a turn.
\bparag \POL can not maintain fortresses of level higher than 3 if at
peace.
\bparag If \RUS is at war, and \POL not, \RUS can cross provinces in
\POL (no siege, no pillage, no supply into or through). It it does so,
\POL has a free \CB against \RUS the very next turn the trespassing
happened.
\aparag[War for Absolutism.]\label{chSpecific:Poland:Absolutism} After
\ref{pIV:Liberum Veto}, each time there is a new Monarch in \POL
(before the events), \POL can begin a war to establish Absolutism in
the country. Event \eventref{pIV:Polish Civil War} occurs this turn
as one of the 4 events.
\bparag If the war is successful in establishing Absolutism, all the
effects of the Liberum Veto are nullified.
\bparag Absolutism can end as a result of a war against \POL. If a \MAJ
imposes a peace of level 3 or higher against \POL, and forfeits all
conditions of peace, this ends Absolutism. It cannot be imposed anew.
\subsubsection{Polish Ukraine}
\label{chSpecific:Poland:Polish Ukraine}
\aparag Provinces in \pays{Ukraine} are not national provinces of \POL
(neither \pays{Lithuanie} nor \pays{Pologne}). They have their own army
of Cossacks.
\bparag One \ARMY and 4 \LD can be used by \POL and raised in
\pays{Ukraine} as long as it owns at least one province in the country.
Those forces are identical to Polish forces.
\aparag[Agitations of Cossacks.]
\bparag When there is an Ukrainian \ARMY controlled by \POL, it can let
it cause some Agitations by its own in adjacent countries. This has to
be decided at the beginning of the Military Phase. This is not possible
if \POL is at war against \TUR or \pays{Crimee}.
\bparag The army is taken over from the map and \POL chooses the target
of the Agitations: \RUS or \TUR.  It then rolls 1d10, and add {\bf +2}
if the \ARMY is \Faceplus, and a further {\bf +2} if the target is
\TUR. A result of 10 or higher causes a revolt that is rolled on the
table of the target country.  If this revolt is not north of
\province{Alep}, \province{Kordistan} and \province{Azarbayadjan} (not
included), it does not happen.  The army is unavailable for the whole
turn and is replaced in \pays{Ukraine} at the end of the turn (if there
is no province available, it is destroyed).
\aparag Event \eventref{pIV:Revolt Cossacks} separates \pays{Ukraine}
from \POL, and so its forces can no more be used by \POL.
\aparag Religious attitude regarding Orthodoxy may affect the use of the
forces of \pays{Ukraine} by \POL.

\subsubsection{Polish Annexations and Crusades}
\aparag \POL may annexe completely the following countries: \pays{Don},
\pays{Moldavie}, \pays{Valachie} and \pays{Transylvanie}.  This is only
possible if the country is adjacent to \POL and \POL is not Protestant.

\subsubsection{\sectionpays{Pologne} as a minor country}
\aparag See~\ruleref{chSpecific:Campaign:Transfer Poland} for the conditions of
the transfer to \paysmajeur{Prusse}.
\bparag The events \eventref{pVI:WoPS}, \eventref{pVI:WoAS} and
\eventref{pVII:Seven Years War} trigger the change to
\paysmajeur{Prusse} if in period VI. If none of those happen, the
transfer happens at the beginning of turn 51.
\aparag \pays{pologne} immediately becomes a \MIN. The ongoing wars
continue.
\bparag \pays{Pologne} never uses the \CB proposed by events unless if
they are mandatory.
\bparag If a dynastic union with \pays{Saxe} is effective due to
\eventref{pV:Saxon King Poland}, the union is kept: the two \MIN are as
one for diplomacy purposes. The only way to propose a separate peace is
through an unconditional peace. The union is kept as long as there is no
change of dynasty (which can happen only by events such as
\eventref{pVI:WoPS} or \eventref{pVII:First Partition Poland}.
\aparag \pays{pologne} never uses the \CB offered by events, unless they
are mandatory.
\aparag There is a permanent malus of {\bf -3} to have \pays{pologne}
enter a war (unless Absolutism was established, see below) and to
diplomacy on it.
\aparag \terme{Land Technology} and \terme{Naval Technology} of
\pays{Pologne} is linked to \terme{Orthodox} counters, and raises at the
same time (but stays where it is if ahead of the \terme{Orthodox}
counter).
\aparag If \RUS is at war, and \POL not, \RUS can cross provinces in
\POL (no siege, no pillage, no supply into or through). It it does so,
the Patron of \POL has a \CB against \RUS the very next turn the
trespassing happened, and if used, \POL enters fully in the war with no
test (and is place in \EG).
\aparag If Absolutism is established in \pays{pologne} (because \POL as
a Major power did it, or because of events), neither the Russian
trespassing nor the {\bf -3} on Diplomacy or entry in war apply.
Additionally, \pays{pologne} has a bonus of {\bf +2} to its
reinforcements die-rolls.

\subsection{Religious attitudes}
\subsubsection{Regarding Orthodoxy}\label{chSpecific:Poland:Orthodoxy}
\aparag \POL has to choose an attitude regarding Orthodoxy at the time
of event \eventref{pI:Reformation}.
\aparag[Conversion of Orthodoxes.] This is the historic choice. No
changes.
\aparag[Tolerance of Orthodoxes.]
\bparag \POL can use no more forces of \pays{Ukraine} (and loses the
associated free maintenance for these forces).
\bparag The Cossacks won't revolt per \eventref{pIV:Revolt Cossacks}.
\aparag[Support of Orthodoxy.]
\textbf{Warning: this option is experimental, not tested and 
should be used with care.}
\bparag Main religion of \POL is now Orthodoxy. It gains actions for
colonisation and some conquistadors.
\bparag \POL can annexe and destroy any Khanate country for the
remainder of the game, if the destroyed minor is adjacent to a province
of \POL.
\bparag As long as \POL owns a province in \pays{Ukraine}, it raises one
free \LD of \pays{Ukraine} each turn, and has 2 \LD added to the free
maintenance of the forces of \pays{Ukraine}.
\bparag Forces of \pays{Lithuanie} are not adversely affected by the
restrictions of \ruleref{chMilitary:Movement:Wasteland} (as well as the forces
of \pays{Ukraine}). This is not true for forces of \pays{Pologne}.
\bparag Many events are modified. The Union of Lublin and the Absolutism
in \POL will not be possible. \POL is Conciliatory when the second
Reformation occurs.
\subsubsection{Regarding Protestantism}
\aparag \POL has to choose an attitude regarding Protestantism at the
time of \eventref{pI:Reformation2}.
\aparag[Catholic/Conciliatory.] This is the historic choice. No changes.
\aparag[Catholic/Counter-Reformation.]
\bparag Provinces in the \region{Duche de Prusse} quit \POL and are
annexed by \pays{Brandebourg}.
\bparag \POL has a \CB against any Protestant country until the end of
period IV.
\bparag It can abandon all peace conditions when obtaining a
unconditional surrender over a Protestant country and ask restoration of
Catholicism in this country.  It gains 20 \PV if it as a \MAJ, and 10\PV
if it is a \MIN power.
\aparag[Protestantism.] \POL becomes Protestant.
\bparag The Union of Lublin is broken and will not be possible.
\bparag Various limits per turn/period are modified. \POL gains actions
for \TP/\COL and increased commercial capacities.
\bparag The free maintenance of the Ukrainian army is reduced to \LD in
periods II and III, and none afterwards.
\bparag \POL can annexe any capital province of \pays{Hanse} (and
possibly destroy this country) if the province is adjacent to \POL.


\subsection{\sectionpaysmajeur{Pologne} in play}
\subsubsection{Available counters}
\aparag[Military]
\bparag[\sectionpaysmajeur{Pologne}] 2\ARMY, 1\FLEET, 3\LDND, 6\LD, 2\NTD, 2\LDENDE, 
2 fortresses 1/2, 4 fortresses 2/3, 4 fortresses 3/4, 1 fortress 4/5, 2 forts.
\bparag[\sectionpays{Lithuanie}] 2\ARMY, 6\LD.
\bparag[\sectionpays{Ukraine}] 1\ARMY, 2\LD.
\bparag[\sectionpays{Mazovie}] No forces (just minor fortresses and leaders).
\aparag[Economical] 5\COL, 5\TP, 6\MNU, 6\TradeFLEET (3 usable at
start), 2 \ROTW treaty counters.


% LocalWords:  Sobieski Patkul Bathory Zygmunt Stary Wasa Lublin Lithuanie pIII
% LocalWords:  Pologne Batory pV pI pII pIV pVI malus Mazowia multi Varsovie de
% LocalWords:  Cracovie Malopolska Liberum Crimee Alep Kordistan Azarbayadjan
% LocalWords:  Moldavie Valachie Transylvanie Khanate Brandebourg Duche Prusse
% LocalWords:  Hanse Prypec Baltarusija Ukrainya Boheme Wielkopolska Danzig
% LocalWords:  Preussen Severia Zemaitija Podolie Wolyn Smolenska Polacak pVII
% LocalWords:  Mazovie Hongrie Lietuva WoPS WoAS mazovie pologne
