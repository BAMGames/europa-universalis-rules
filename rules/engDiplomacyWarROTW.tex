
\section{Conflicts against non-European}
\subsection{Generalities}
\aparag[Areas owned by minor countries.]
The Natives in areas owned by minor countries in the \ROTW, and the
cities, can not be attacked by a power if it is not at war against the
minor country. Exception: a reaction during the turn by Natives may
cause battles in such a province without involvement of the minor
country; in this case the power can continue to attack the Natives in
this province until the end of the turn, but not the cities.

\aparag[Wars in the \ROTW.]
An overseas war is sufficient to make a war against a country in \ROTW,
by definition of this kind of war.
\bparag Forces of a country in the \ROTW may never go on the European
map. They are deployed in any province they own (even if there is \COL
or \TP or enemy forces; in the last case, an immediate battle happens
before the first military round).
\bparag A country in the \ROTW always receives fixed reinforcements each
turn of limited or full war, as described in the Annexes. Those can only
raise their force to the basic forces of the country.
\bparag If a minor country is at peace during one whole turn, its basic
forces come back entirely.
\bparag The forces of a minor country are always in full supply in the
provinces of owned areas, and use those provinces as supply sources if
outside the area. A province where there is a \TP/\COL or a fort
controlled by an enemy can not be used as supply source to go outside
(but minor troops are still supplied within the province).
\bparag A country in the \ROTW uses all the Natives that are in the
areas that it controls. Natives are of moral ``conscript'' (exception:
Natives in \granderegion{Japan} are ``veteran'') and are added to
regular forces if there is any in the province. They never move.  They
will attack \TP and \COL in their provinces if they are at war against
the owning country.
\bparag Natives and regular forces of minor countries can do "Native
attack" in owned areas at the end of the turn to destroy \COL or \TP.
Additionally, regular forces can burn down controlled \TP as per normal
rules (Natives cannot).

\aparag[Areas with no minor countries.]
Some areas are less organised: no minor country owns them. A European
country can decide to attack Natives or cities in the corresponding
provinces without being at war, with no declaration beforehand.
\bparag If Natives are attacked in a given province, they will continue
to react (as defined afterwards) against the aggressor until the end of
the turn.
\bparag To assault or besiege a city, a power has first to attack the
Natives of the province (or they have to be already active).
\begin{designnote}
  By ``less organised'', we do not mean, of course, that areas such as
  South-East Asia or Indonesia were lacking states. Dai Viet,
  Ayutthaya, the sultanate of Borneo and other countries clearly
  exists. However, these countries were of a rather local importance and
  their relative strength and tolerance to the Europeans is directly
  represented by the values of the corresponding area. \ROTW countries
  correspond to large empires such as China or the Mogols, with a large
  territorial base or a powerful army.
\end{designnote}

\subsection{Reactions by countries in the \ROTW}
\aparag At the end of the phase of event, a test of reaction is made in
a country from the \ROTW where one of the conditions is met:
\bparag there is a military force in one of its province (excepted if
this force is in a foreign \COL settled in the province, or if allowed
by a \dipFR or \dipAT);
\bparag there is a European \COL or \TP that is not allowed by
diplomatic status (or a special rule).

\aparag The test is 1d10, compared to the Activation level of the
country. If it is strictly lower, the minor country declares an Overseas
war against any and all powers that satisfy one of the previous
conditions.
\bparag List of the Activation levels:
\begin{modlist}
\item[9/3] \pays{Moghol} before/after \eventref{pVI:Last Great Mughals}
\item[9/11] \pays{Chine} and \pays{Japon} before/after
  \subeventrefshort{pIII:CCA:Closure China} and
  \subeventrefshort{pIV:JCA:Closure Japan}, except in newly conquered areas
  (6)
\item[9] \pays{gujarat}
\item[8] \pays{Iroquois}, \pays{Soudan}
\item[4] \pays{Inca}, \pays{Azteque}, \pays{Vijayanagar}
\item[6] All others: \pays{Siberie}, \pays{Oman}, \pays{Aden},
  \pays{Mysore}, \pays{Hyderabad}, \pays{afghans}, \pays{ormus}
\end{modlist}

\subsection{Reactions by Natives during the rounds}
\aparag At the end of each military round, before the sieges, a test of
reaction is made in every province in the \ROTW where there is a
European military force that is
\bparag Neither in a \COL of a European power;
\bparag Nor allowed by some \dipFR or \dipAT in this province by a minor
country owning the area.
\bparag When a land stack moves also through a province where none of
the two previous conditions hold, a test of reaction is also made before
it leaves the province.
\bparag Finally some attempts of putting \TP or \COL in a province may
cause an automatic reaction of the Natives,
see~\ruleref{chAdministration:Colony:Critical failure} 
and~\ruleref{chAdministration:TP:Critical failure}.

\aparag The test of reaction is resolved by rolling 1d10. If it is
strictly inferior to the Tolerance level in the area, the Natives react.
When the Tolerance is "-", no reaction can happen.

\aparag[Effect of a reaction.]
\bparag The reaction is an attack of the Natives against the units that
caused the reaction, and all units of the same country in the province
(not area).
\bparag The attack is revolved immediately (as an interception if it is
caused by a movement, or a regular battle if it is at the end of the
round or due to botched \TP/\COL action).
\bparag
The reaction last until the end of the turn and the Natives will attack
any other force of the power causing the reaction that is in the
province. Only one battle is possible each round (at the time of the
first interception by reaction, or at the end of the round). Natives
will then attack \COL/\TP owned by the power at the end of the turn.
Note that if A has activated the Natives against him, and controls a
fort of fortress of the side B who has not, the Natives would attack A
and besiege its forces (attrition if A is withdrawn in the fortress) but
would not attack a \COL/\TP owned by B (even if controlled by A) at the
end of the turn.
\bparag If units of another player enter the province later in the turn, 
they can also provoke a reaction of the Natives against them. 
