%% *-* latex-mode *-*



\event{pIII:FWR Detailed}{III-D}{Religious Wars in France}{5}{PBNew}

\history{1562-1598}{The wars are fragmented in 5 parts. \\
  (1) First, Second and Third wars (1562-1570) with many truces broken by one
  side or the other. \\
  (2) Fourth and Fifth wars (1570-1575), where the Massacre
  of the Saint-Barth\'el\'emy heightens the intensity of the war.\\
  (3) Sixth and Seventh wars (1575-1580) where the Catholic League and the
  Duke of Guise seem almighty, and a background announced Dynastic Crisis. \\
  (4) Eighth war (1585-1598) that is the war of Succession for the French
  Crown. \\
  (5) Alternative history: more troubles if France is not Conciliant (mainly
  with foreign support).  }

\dure{until the end of \ref{pIII:FWR Last Stand} or \ref{pIII:FWR Succession}
  (as specified in these events) or at the end of period III.}

\activation{This event is composed by many sections describing first the
  general conditions under which the wars are fought, then specifics of the
  evolution of the Wars: from a set of strictly Religious Wars that go harder
  and harder to a War of Succession. The passage from one event to another is
  described hereafter.}
\aparag This event can not happen before turn 11 (1540). If the turn if 10 or
before, re-roll and do not mark off.
\aparag Only one \ref{pIII:FWR} can be rolled and marked off each turn. If a
second one is obtained, do not mark off and re-roll.
\aparag After the end of this event, \ref{pIII:FWR} triggers an event \RD, and
the box is marked.
\bparag If \FRA is \CATHCO, its Monarch will have a malus of \bonus{+2} to his
Survival Test next turn.
\bparag If \FRA is \CATHCR or Protestant, the \REVOLT is rolled on the table
of \FRA.
\aparag From the first to the end of the last event, \FRA is in religious
Civil War and is limited in many aspects.

\phevnt
\aparag[The states within the State] Two minor countries, \paysHuguenots and
\paysLigue are created for this event. No diplomacy is authorised on them;
they have the same technology and military features as \FRA.
% \bparag When at peace with \FRA, all their provinces and counters (armies,
% leaders,...) belong to \FRA and are used as such.
% \bparag When at war against \FRA, \FRA still earns the income of their
% provinces (except where there are \REVOLT !)  and those minor countries
% never pillage the provinces in \FRA.
\aparag[Les Huguenots]
\bparag \hug has the following provinces (if in \FRA): \provinceCaux,
\provinceTouraine, \provincePoitou, \provinceQuercy, \provinceGuyenne,
\provinceLanguedoc, \provinceBearn, \provinceDauphine, \provinceCevennes
(those provinces have a white shield border).
\bparag \hug is protestant.
\bparag Its main controller is \ENG (if Protestant) or \HOL (if it exists) or
\SUE (if Protestant), else \MAJHOL. This major power will be noted \HUG (and
the minor \hug); it may change at each turn (depending on the changes of
religion).
\aparag[La Ligue]
\bparag \lig has the following provinces (if in \FRA):
% \provinceArmor, \provinceFinistere, \provinceMorbihan,
\provinceNormandie, \provinceMaine, \province{Ile-de-France},
\provinceOrleanais, \provincePicardie, \provinceChampagne, \provinceBerry,
\provinceBourgogne, \provinceLyonnais, \provinceProvence (those provinces have
a yellow shield border).
\bparag \lig is \CATHCR.
\bparag Its main controller is the \SDCF (if it is not \FRA), \SPA (if
\CATHCR), \ENG (if Catholic), or \SPA (\CATHCO) in the last possibility.  This
major power will be noted \LIG (and the minor \lig); it may change at each
turn (depending on the changes of religion).
\aparag The Loyalists are \FRA and its allies. The Rebels are the revolted
minor country (\lig or \hug) and its allies. \REB is the Major Power that
controls the Rebels (\LIG or \HUG).
\aparag The Catholic side is the one of \lig else of Catholic \FRA.
\aparag The Protestant side is the one of \hug else of Protestant \FRA.

\aparag[Military units]
\bparag Basic forces of \FRA drops to \ARMY \facemoins (or \ARMY \facemoins,
\LD if in period \period{II}). Counters limit for \FRA drops to 3 \ARMY (and 2
\ARMY for each minor).
\bparag Basic forces of the new minors is \ARMY \facemoins, \LD (or \ARMY
\faceplus if in period \period{II}) if it has not the same religion than \FRA
and \ARMY \facemoins (\ARMY \facemoins, \LD if in period \period{II}) if it
has the same religion than \FRA.
\bparag If the minor is at war against \FRA, then it is controlled by its main
controller (either \HUG or \LIG). Else, if \FRA is at war (even civil war
against the other minor) then \FRA may use its troops as if they were french
troops.
\bparag If \FRA is at peace, the main controller of each minor may declare a
limited intervention (following usual rules) of this minor in any existing war
during the diplomatic phase. If the minor has the same religion than \FRA,
this can only be done if \FRA agrees to.  The main controller plays the troops
of the minor and pay for its campaign or reinforcements.
\bparag If \FRA is at peace, and the main controller doesn't want to use the
troops of the minor (or can't), then \FRA may use them as if they were its own
troops.
\bparag If \FRA is at peace, it may build troops of any of the two minors at
regular cost. This counts toward purchase limit of the turn.
\bparag If the minor is not used by somebody else, \FRA has to pay the
maintenance of any troops in addition to the basic maintenance of the minor.
\bparag If \FRA is at peace and the minor has less than its basic forces and
is not used in another war by its main controller, then \FRA has to build
troops of the minor. It is not complied to buy more than the turn limit or to
go bankruptcy, but it must build troops for the minor prior to any other
administrative action. If both minors lack troops, \FRA must start building
troops of the minor having a different religion than its own.
\bparag If \FRA is at peace with the minor, it cannot voluntary dismiss
(i.e. by not paying upkeep) troops of the minor below what was left at the end
of the last civil war. Yet, if the loss is due to any other reason (such as
being used in another war or by its main controller in a foreign
intervention), \FRA is not complied to buy new troops up to this value (just
up to the basic maintenance of the minor).

\aparag[Incomes]
\bparag If \FRA is at war against the minor, then it get no land income from
the provinces of the minor (this also may change the industrial and commercial
incomes of \FRA). Manufactures in these provinces do not provide income
either.
\bparag If \FRA is at peace, the provinces of the minor having the same
religion as \FRA are treated exactly like french provinces: they provide full
land income, manufactures and gold mines provide also full income.
\bparag If \FRA is at peace, the provinces of the minor having different
religion than \FRA only provide half their regular income: land income is
halved (this also change industrial and commercial income), manufactures
provide only half their facial value and half their percentage, gold mines
provide only 10\ducats, \ldots
\bparag If \FRA is in civil war (but not against the minor), provinces of the
minor only provide half their regular income (as above).
\bparag The (land) income not perceived by \FRA does not increase its foreign
trade.
\bparag If \FRA is at peace, it only gets 75\% of its colonial income if its
catholic.

\aparag[Military control]
\bparag If \FRA is not at war against the minor, then both may use provinces
belonging to both of them as supply sources.
\bparag If \FRA is at war against the minor, then supply may go through any
province not containing an unbesieged hostile troop or \REVOLT .

\effetlong
\aparag[Fragile Health of the Valois]
\bparag From the beginning of the event, and as long as the French Monarch is
a Valois, it adds \bonus{+3} to its Survival Test.
\aparag[Lack of Heirs]
\bparag An additional test of Dynastic Crisis is made at the beginning of each
turn (at the Monarch Survival Phase). A malus of \bonus{-1} is applied for
each \ref{pIII:FWR} rolled since the beginning of the game.
\bparag If a Dynastic Crisis occurs (because of the previous test or of a
normal test after the death of the Monarch), apply directly \ref{pIII:FWR
  Succession} as the first event of the turn.  If a Dynastic Crisis occurs
without the death of the Monarch, the rules of the event use the historical
name \anchormonarque{Henri III} to designate the current Monarch of \FRA.

\aparag[Mandatory Change of Religious Attitude] \FRA can be complied to change
its Religious choice during the war because of a Coup (\ref{pIII:FWR
  Succession}), or an unconditional surrender caused by foreign powers. The
following points occur (but not if the change is voluntary when designating an
Heir of the Valois).
\bparag \FRA goes down to {\bf -3} in \STAB, loses \bonus{-1} in \FTI, and
loses 30 \PV.
\bparag The controller of the side imposing its Heir by a Coup, or the
countries that force a unconditional surrender gain 30 \PV each time a
mandatory change is made.

\begin{digressions}[General troubles in France each time an event happens]


  \digression[pIII:FWR:Politic Crisis]{Politic crisis}

  \phevnt
  \aparag \FRA may not be part of a new loan treaty until~\ref{pIII:FWR Final}
  \aparag \FRA loses {\bf 2} \STAB.
  \aparag The diplomacy of \FRA is lowered by \bonus{-2} (minimum of 3).
  \aparag \FRA and its adversaries make a mandatory white peace (exception:
  see \ref{pIII:FWR Last Stand}).
  \aparag \FRA is involved in religious civil war when at war against
  Rebels. No-one can declare a war to \FRA at those times, but \MAJ may do
  \terme{Foreign Intervention} in the war each time the war resumes (new event
  or broken Truce) excepted if explicitly forbidden.


  \digression[pIII:FWR:Economic Crisis]{Economic crisis}

  \phevnt
  \aparag On the first event, the Royal Treasury of \FRA is diminished by half
  and loses at least 100\ducats.  On subsequent events, the Royal Treasury of
  \FRA is halved with a minimum loss of 50\ducats.
  % diminished by 50\ducats if greater than 50\ducats, goes to 0\ducats
  % if greater than 20\ducats, and be diminished by 20\ducats if less than
  % 20\ducats.

  \bparag If \FRA makes a bankruptcy while at war against the rebels, they
  will receive \ARMY\facemoins extra reinforcement (\LD each if there are two
  rebels).
  % On the first event, the Royal Treasury of \FRA is halved and loses at
  % least 50\ducats if \FRA has not enough in its royal treasury, then it lose
  % everything, lose and 1 \STAB and rebels will receive \ARMY \facemoins
  % extra reinforcement (or \DT each if there are two rebels). On the
  % following ones, the Royal treasury of \FRA is halved with a maximal lose
  % of 50\ducats.
  % \aparag It receives no commercial income, no industrial income, and
  % no colonial income during the wars; however if a war resumes at the
  % end of a Truce, \FRA receives half of those industrial, commercial
  % and colonial income during the first turn after the Truce, as the
  % Truce is broken after some months (or years) of relative peace.
  \aparag \FRA (and also \hug and \lig) makes a mandatory trade refusal
  against all other countries. This does not provide CB or entail loss of
  stability and only last while \FRA is in civil war.
  % \aparag Industrial income is halved (after a reduction due to \hug
  % or \lig being at war with \FRA).
  \bparag \FRA only gets 75\% of its colonial income if protestant, 50\% if
  \CATHCO and 25\% if \CATHCR.
  \aparag \FRA can make no economic action (\COL, \TP, \TFI, \CONC) during the
  wars (even if the Truce was broken this turn), except as a reaction to
  concurrence.
  \aparag A \PIRATE\faceplus is placed in \CTZ of \FRA; at most one \PIRATE
  can be here due to this event.
  \aparag \FRA has to pay separate campaigns for any troop going in the \ROTW
  or whose movement end on the \ROTW map (so, it can bring back troops from
  the \ROTW without penalty).


  \digression[pIII:FWR:Uprisings]{Uprisings in France}

  \phevnt
  \aparag If \FRA is \CATHCR or \CATHCO, the Rebels are \hug.  If \FRA is
  Protestant, the Rebels are \lig. \FRA is at war against the Rebels (it is
  not a declaration of war by the Rebels).
  \aparag If \FRA is \CATHCR or \CATHCO, roll 1d10 and place \REVOLT
  \facemoins in the following provinces, excepted in the first province where
  the \REVOLT is \faceplus:
  \bparag result odd: \provincePoitou, \provinceQuercy, \provinceGuyenne,
  \provinceLanguedoc, \provinceAuvergne;
  \bparag result even: \provinceCaux, \provincePoitou, \provinceGuyenne,
  \provinceTouraine, \provinceVendee.
  \aparag If \FRA is \CATHCR, add a \REVOLT \faceplus in \provinceDauphine and
  a \REVOLT \facemoins in \provinceArmor.
  \aparag If the die-roll was 9 or 10 (between 7 and 10 if \FRA is \CATHCR),
  place a \REVOLT \facemoins on a randomly chosen colony (or \TP if no colony
  is available).
  \aparag If \FRA is Protestant, place a \REVOLT \faceplus in
  \province{Ile-de-France}, a \REVOLT \facemoins in \provinceLyonnais and roll
  1d10 for the other ones (the \REVOLT is \faceplus in the first province of
  the list and \facemoins in the others):
  % \bparag result even: \provinceQuercy, \provincePoitou,
  % \provinceDauphine, \provinceTouraine, \provinceCaux;
  \bparag result even: \provinceProvence, \provinceNormandie, \provinceMaine,
  \provinceTroyes, \provinceVendee;
  \bparag result odd: \provinceOrleanais, \provinceChampagne,
  \provinceTouraine, \provinceCaux, \provincePicardie.
  \aparag If the die-roll was 10, place a \REVOLT \facemoins on a randomly
  chosen colony (or \TP if no colony is available).
  \aparag The Rebels receive 2 minor unnamed generals to be placed on \REVOLT
  (they can only lead \REVOLT , not forces of the Rebels, and are eliminated
  when the \REVOLT is finally suppressed).
  \aparag The Rebels own its provinces and control those where there is a
  \REVOLT\faceplus %and any province of \FRA where
  % there is a \REVOLT .


  \digression[pIII:FWR:Military Troubles]{Military Troubles}

  \phevnt
  \aparag On the first event, only the basic forces of \FRA are kept (\ARMY
  \faceplus, \ARMY \facemoins, \DT), in veteran status.  If \FRA has less than
  this, it will receive less troops than stated. The rebels takes their forces
  first, then the non-rebelled minors and lastly \FRA.
  \aparag Roll 1d10:
  \bparag result even: \FRA keeps \ARMY \facemoins and \DT; the Rebels have
  \ARMY \facemoins; the minor of the same religion as \FRA has \ARMY
  \facemoins;
  \bparag result odd; \FRA keeps \DT, the Rebels have \ARMY \faceplus; the
  minor of the same religion as \FRA has \ARMY \facemoins.
  % \aparag The Rebels may already have some forces (remaining after a
  % favourable Truce) that are added to this forces. All those forces are
  % veterans.
  \aparag If the current turn is in period II, \FRA adds \ARMY \facemoins to
  its forces and the minor sharing its religion add \DT.
  \aparag If \FRA is Emperor of the \HRE, it can use the \ARMY of \HRE as a
  help in this war.
  \aparag Minor country \paysLorraine is activated and allied of the Catholic
  side. It gives 1 \DT, both sides can pass or stop in its provinces but the
  \REVOLT never extend in those.
  \aparag The forces of the Rebels are deployed in their provinces that are in
  \REVOLT .  The forces of \FRA are placed in any province of \FRA that does
  not belong to the Rebels.
  \aparag The naval forces of \FRA may defect as follows. Roll 1d10.
  \bparag result 1-8: \FRA keeps all the naval forces.
  \bparag result of 9: 1 \DN is given to the Major Power controlling the
  Rebels and the rest are Rebel forces.
  \bparag result of 0: 1 \DN is given to the first Protestant country of the
  list: \ANG, \HOL, \SUE, \POL, or to the Major Power controlling the rebels
  if there is none, and the rest are Rebel forces.
  \bparag Naval forces of the Rebels have to go in a port of Rebels.  When, at
  the end of a round, there is no port left to Rebels, the navy comes back in
  the ownership of \FRA.

  \phadm
  \aparag \FRA can build reinforcements as usual and deploys them in provinces
  not owned by the Rebels.
  \aparag The Rebels gain reinforcements in offensive mode on the minor table,
  with a bonus of \bonus{+2} and some other modifiers (see the various steps
  of the events). It gains only the \LD written in the table, not the F, CM or
  leaders.
  \bparag If \FRA is not \CATHCO, add \bonus{+1} to the roll.
  \bparag The Rebels receive 1\fortress if the result is even, or 2\fortress
  if the result is equal to 11 or higher.
  \bparag The reinforcements of the Rebels are deployed in provinces in
  \REVOLT , and the fortresses can only be deployed in provinces with \REVOLT
  \faceplus.
  \aparag[Leaders] After the building of forces, the loyalty of the leaders is
  tested.
  \bparag \leaderMontmorency is always loyal to \FRA.
  \bparag \lig receives \leader{Henri de Guise}.
  \bparag \hug receives \leaderColigny, \leaderConde and, beginning with
  \ref{pIII:FWR Barthelemy}, \leaderNavarre.
  \bparag Every other named leader is checked by rolling 1d10: used by the
  Catholic side if result 1-7; used by the Protestant side if the result is
  8-10.
  \bparag Each side should have at least two leaders. If one has less, it
  receives an unnamed general from those of \FRA.
  \bparag Neither the Loyalists nor the Rebels can use mercenary generals.
  \bparag This repartition is made once for all the following wars; but \FRA
  can use all its leaders (whether from \lig or \hug) during Truces.


  \digression[pIII:FWR:Military Operations]{Military operations during the
    wars}

  \phmil
  \aparag The Rebels control all cities of provinces with \REVOLT at start.
  It draws supply from all provinces of the rebel minor country and from
  cities it controls.
  \aparag \FRA controls all cities of provinces not in \REVOLT .  It draws
  supply from any such provinces.
  \aparag French Leaders of both side are only killed in battles if the
  die-roll was a natural 1. Else, if they would be killed (due to modifiers),
  they are Captured instead and are freed when a Truce happens.
  \aparag The Rebels and the minor countries that are involved in the war have
  a simple campaign each turn. Their controller may pay for a more important
  campaign (by spending the cost of the campaign minus 20\ducats).
  \aparag A city owned by the rebel minor country makes an immediate voluntary
  surrender if besieged by a land stack that is commanded by a named rebel
  general and that sets a siege with at least one \ARMY \faceplus.
  % \aparag If \FRA proposes a Truce favourable to the Rebels at the end
  % of any military round, it is signed immediately and the military
  % operations for this war ceases for the rest of the turn.  This is
  % not possible during events \ref{pIII:FWR Succession} and
  % \ref{pIII:FWR Last Stand}.
  % (Jym) (4) and (5) (no Truce). TODO deal with (1)


  \digression[pIII:FWR:Truces]{Truces during the Wars of Religion}

  \phpaix
  % \aparag Each war from \ref{pIII:FWR Beginning} to
  % \ref{pIII:FWR League} last only one turn, then ends with a
  % Truce favouring one side of the other. The war can resume the next
  % turn or a following one. \ref{pIII:FWR Succession} and
  % \ref{pIII:FWR Last Stand} are full Civil War were no Truces are
  % admitted.
  % \aparag[Favoured side in the Truce]
  % \bparag Excepted if a Truce favourable to the Rebels was proposed by
  % \FRA, determine the side favoured by the Truce according to the
  % following calculus of their respective positions.
  % \bparag A side obtains 1 point per victory in land battle, 2 per
  % major victory, 1 per siege made (but none for automatic surrenders),
  % and the Rebels gain 1 per \REVOLT remaining in France at the end of
  % turn (before any extension).
  % \bparag Whomever has the highest total obtains a favourable Truce;
  % The Rebels are favoured if it is a draw.
  \aparag At the end of any turn, \FRA may propose peace to the rebelled
  minor. This is treated as a regular peace with minor. This can not be done
  during \ref{pIII:FWR Succession} and \ref{pIII:FWR Last Stand} who have
  specific ending conditions.
  % \bparag The initial situation is the one at the beginning of the
  % military phase. Any \REVOLT crushed adds a +1 bonus to the french
  % die roll. Automatic surrenders of cities do not provide any peace
  % differential. (Jym) rather do the opposite => no differential for
  % \REVOLT , but differential for sieges, even automatic ? would be more
  % in line with the rest of the game...
  \bparag The initial situation is the one at the beginning of the military
  phase. \REVOLT do not count toward the peace differential, but provinces
  taken (including automatic surrender) count.
  \bparag Money may not be asked/given as a peace condition.
  \bparag A valid peace condition is the establishment or demolition of a
  safety place. If a safety place is granted, the minor may put a level 3
  fortress in an owned province. If possible it must be put in a province
  initially in \REVOLT \faceplus.
  \bparag The first peace condition must be a safety place (if possible).
  \bparag Any colonial establishment still having a \REVOLT when peace is
  signed immediately lose one level (and may thus be destroyed).
  \aparag If no truce is granted, \REVOLT do not extend as normal but \FRA
  loses stability for both the \REVOLT and the duration of war.
  \aparag Two white peaces count as a losing truce toward french objectives
  (but a single white peace has no effect).
  % \aparag If \FRA removes all the \REVOLT counters and retakes all rebel
  % fortresses, it may ask for an (automatic) unconditional surrender. It
  % immediately gains 20 \VP and 2 stability. The next \ref{pIII:FWR} rolled
  % will be played as \ref{pIII:FWR Succession} (even if \FRA is \CATHCO) after
  % which the civil wars will permanently stop.
  \aparag If \FRA is for two consecutive turns of the same war at -3 in
  stability and does not manage to sign a peace, it must surrender
  unconditionally and suffer a mandatory change of religion.
  \aparag If \FRA sign a unfavourable peace, it loses 1 stability.
%  \aparag If \FRA sign a favourable peace, it gains 1 stability.

  \aparag[Effect of a Truce] All \REVOLT are suppressed in \FRA; the naval
  forces are back in the ownership of \FRA (except the \DN that might have
  been seized by foreigners).
  % \bparag If the Truce favours \FRA, \FRA annexes one province of the
  % rebel minor country; all fortresses of the Rebels are withdrawn; all
  % remaining Rebel land forces are given to \FRA; \FRA gains {\bf 1}
  % \STAB.
  % \bparag If the Truce favours the Rebels, they annexe one province
  % (from \FRA or the opposed minor country; if possible a province they
  % once owned); one of its provinces becomes a Place of Safety and
  % gains a level 3 fortress (if possible, a province having a \REVOLT
  % \faceplus before the Truce); the Rebels keep half of its land forces
  % (round up), the rest is dismantled and \FRA will have to build
  % forces anew to its basic forces before the next war; all fortresses
  % in the provinces of the rebel minor country remain.
  % \aparag During Truces, \FRA earn half of its colonial, industrial
  % and commercial income. This includes turns were a Truce is broken by
  % the following mechanism.

  \phdipl
  \aparag During Truces, \FRA is not limited in its diplomatic and
  administrative actions, and can also be involved in external wars (using its
  forces as well as those of \lig and \hug). This does not include turns where
  a Truce breaks down. Remember that both \lig and \hug may be used by their
  controllers.
  \aparag The Truce can be questioned at the beginning of any phase of
  Diplomacy:
  \bparag If the Rebels have at least their basic forces, Roll 1d10
  + %the difference between the position value for
  % Rebels at the end of last war and the value for \FRA + \STAB of \FRA
  the level of the peace (in favour of the rebel) \bonus{-1} per turn since
  the beginning of the Truce. If the result is 4 or below, the Rebels will
  break the Truce.
  \bparag Else, if \FRA did not have a favourable Truce and wants to break it,
  it can do it after one full turn of peace at least.
  \bparag If the Truce is broken, apply \ref{pIII:FWR:Politic Crisis},
  \ref{pIII:FWR:Economic Crisis}, \ref{pIII:FWR:Uprisings},
  \ref{pIII:FWR:Military Troubles} at the end of the Diplomatic phase.
\end{digressions}



\event{pIII:FWR Beginning}{III-D (1)}{The first 3 Wars of Religion}{1}{PB}

\tour{Turn 1}

\phevnt
\aparag[Michel de l'Hospital]\label{pIII:FWR:Hospital} If \FRA is \CATHCR, it
can now decide to play the rest of the event as \CATHCO.  Its religion changes
immediately, using only the lasting effects of the \ref{pI:Reformation}; the
initial \REVOLT are played as \CATHCR though.
\aparag The Wars of Religion begin; apply the general conditions and the
lasting effects on the Valois as found in \numberref{pIII:FWR Detailed}.
\aparag Apply the full effects of \ref{pIII:FWR:Politic Crisis},
\ref{pIII:FWR:Economic Crisis}, \ref{pIII:FWR:Uprisings} and
\ref{pIII:FWR:Military Troubles}.
\bparag For each \REVOLT that should be placed, roll a die: the \REVOLT
actually happens only if the result if 6 or higher. Add 1 to the die roll if
\FRA is not \CATHCO (do not add if \FRA just changed its attitude due to
Michel de l'Hospital, but still use the \CATHCR line for placing \REVOLT ).

\phdipl
\aparag No Foreign intervention allowed on the first turn.
\aparag \REB can make a very limited intervention in the war, only with naval
forces (in order to install or break a blockade; no naval movement of Rebel
land forces), that costs no \STAB.
\aparag If \FRA is \CATHCR, \LIG can make a foreign intervention as an ally of
\FRA.

\begin{digressions}[Specific conditions of the first event]


  \digression[pIII:FWR:War1 Military]{Military operations during the first
    event}

  \phmil
  \aparag Use the general rules of \ref{pIII:FWR:Military Operations}.
  \aparag If all the leaders of on side are captured, wounded or killed, this
  side signs a level 1 peace in favour of its enemy at the end of the round.

  \aparag At the beginning of each military round (except the first), a new
  \REVOLT is rolled for in France.
  \bparag This revolt is always rolled on the table for \FRA in period III,
  even if this is not the current period. Moreover, if \FRA is catholic,
  \textbf{subtract} its \STAB from the localisation die roll rather than
  adding it.
  \bparag If this \REVOLT is in the rebel minor country and has no \REVOLT nor
  Loyalist land force in it, place a new \REVOLT \facemoins which takes the
  city.
  \aparag A city in \FRA that had not a \REVOLT \faceplus at the beginning of
  the current war nor is a safety place, makes an immediate voluntary
  surrender if besieged by a land stack of \FRA (or its allies) that sets a
  siege with at least one \ARMY \faceplus and there is no more \REVOLT in the
  province (including if the \REVOLT was just crushed this round).


  \digression[pIII:FWR:Peace1]{Peace during the first event}

  \phpaix
  % \aparag A Truce is necessarily signed, and the favoured side is determined
  % as explained in \ref{pIII:FWR:Truces}. All the effects
  % explained here are applied (so the \REVOLT are withdrawn before extension
  % or Stability loss).
  \aparag No peace of level higher than 2 can be signed during this first war,
  especially no unconditional surrender can happen.
  \aparag If \LIG was in foreign intervention, allied to a \CATHCR\ \FRA, it
  wins 15 \PV if the Truce is in favour of \FRA and \LIG had forces in at
  least one battle or one siege (including voluntary surrender) against the
  Rebels.
  \aparag \FRA may choose to commit \xnameref{pIII:FWR Barthelemy} on any
  later turn. Consider that \numberref{pIII:FWR Barthelemy} is one of the four
  events rolled this turn and apply all the relevant effects.

  \tour{Turn 2 and following: Extension of the War}


  \digression[pIII:FWR:Continuation1 War]{Extension of the war}

  \phevnt
  \aparag[\REVOLT extension]
  \bparag For each two \REVOLT still existing in France (including colonial
  empire), roll die on the \REVOLT table for \FRA. If the province is neither
  occupied by loyalist troops or part of the non-rebelling minor, place a
  \REVOLT \facemoins which takes the city there.
  \bparag Roll a die. Add 2 if \FRA is \CATHCR, subtract 2 if \FRA is
  protestant. On a roll of 6 or more, place a \REVOLT \facemoins in a randomly
  chosen french colony (if there is no french colony or all have 2 \REVOLT
  \faceplus, in a randomly chosen \TP).

  \phadm
  \aparag Rebel will receive reinforcement as on turn 1.

  \phdipl
  \aparag Foreign interventions are now permitted.
  \aparag \REB can make a limited intervention as an ally on the Rebels (and
  it is not limited to naval forces only from now on).
  \aparag \HOL can make a limited intervention as an ally of a rebel \hug.
  \aparag \SPA can make a limited intervention as an ally of a rebel \lig.

  \phmil
  \aparag[Intervention of \paysPalatinat]\label{pIII:FWR:Palatinate} If
  inactive, \paysPalatinat makes a limited intervention as an ally of the
  Rebels (it is a mercenary army). It is played by \REB. The intervention
  force is \leader{Jean-Casimir}, one \ARMY \faceplus and 1 \DT.  If the
  \nameref{pII:Schmalkaldic League} or the \nameref{pIII:League Nassau}
  exists, and the Rebels are \hug, this intervention is made with 2 \ARMY
  \faceplus.  \leader{Jean-Casimir} is a general of \paysPalatinat (and serves
  this country if it is at war elsewhere) that will stay as long as
  \ref{pIII:FWR Barthelemy} is not finished.  After that, \paysPalatinat is
  without leader (for intervention) or has normal generals (for other wars).

  \tour{Turn 2 and following: Breaking of Truces}


  \digression[pIII:FWR:Continuation1 Truce]{Breaking of Truces}

  \phevnt
  \aparag If a Dynastic Crisis occurs, \ref{pIII:FWR Succession} will happen
  at this turn. If \numberref{pIII:FWR} is rolled for at this turn, mark off
  the box and consider that it triggers \numberref{pIII:FWR Succession}.
  \aparag As long as a new \numberref{pIII:FWR} is not rolled for, the Truce
  can be broken as explained in \ref{pIII:FWR:Truces}. A war begins anew, as
  explained there.
  \aparag If a new \shortref{pIII:FWR} is rolled for in the Political Event
  Phase, the next phase of \shortref{pIII:FWR} begins (\numberref{pIII:FWR
    Barthelemy}, \numberref{pIII:FWR League} or \numberref{pIII:FWR
    Succession}). Go to this event.
  \aparag If none of this happens, \FRA is in civil peace, and has its
  activity limited by \ref{pIII:FWR:Truces} only.

  \phadm
  \aparag If the Truce has been broken, apply the full effects of
  \ref{pIII:FWR:Politic Crisis}, \ref{pIII:FWR:Economic Crisis},
  \ref{pIII:FWR:Uprisings} and \ref{pIII:FWR:Military Troubles}, and the
  following points.

  \phdipl
  \aparag Foreign interventions are now permitted.
  \aparag \REB can make a limited intervention as an ally on the Rebels (and
  it is not limited to naval forces only from now on).
  \aparag \HOL can make a limited intervention as an ally of a rebel \hug.
  \aparag \SPA can make a limited intervention as an ally of a rebel \lig.

  \phmil
  \aparag The war is prosecuted according to \ref{pIII:FWR:Military
    Operations}, and \ref{pIII:FWR:War1 Military}.
  \aparag[Intervention of \paysPalatinat] If inactive, \paysPalatinat makes a
  limited intervention as an ally of the Rebels (it is a mercenary army). It
  is played by \REB. The intervention force is \leader{Jean-Casimir}, one
  \ARMY \faceplus and 1 \DT.  If the \nameref{pII:Schmalkaldic League} or the
  \nameref{pIII:League Nassau} exists, and the Rebels are \hug, this
  intervention is made with 2 \ARMY \faceplus.  \leader{Jean-Casimir} is a
  general of \paysPalatinat (and serves this country if it is at war
  elsewhere) that will stay as long as the \ref{pIII:FWR Barthelemy} is not
  finished. Beginning with next event, \paysPalatinat is back to normal (no
  leader for intervention or normal generals for other wars).

  \phpaix
  % \aparag A Truce is necessarily signed, and the favoured side is determined
  % as explained in \ref{pIII:FWR:Truces}. All the effects
  % explained here are applied (so the \REVOLT are withdrawn before extension
  % or Stability loss).
  \aparag If a Major Power makes a limited intervention and the side it helps
  obtains a Truce in its favour, the Major Power gains 10 \PV if it had land
  forces in at least one battle or one siege (including voluntary surrender)
  against the enemy side.

\end{digressions}



\event{pIII:FWR Barthelemy}{III-D (2)}{The Saint-Barthelemy}{1}{PB}

\tour{Turn 1}

\phevnt
\aparag A new war breaks out. Apply the full effects of \ref{pIII:FWR:Politic
  Crisis}, \ref{pIII:FWR:Economic Crisis}, \ref{pIII:FWR:Uprisings} and
\ref{pIII:FWR:Military Troubles}.
\aparag \leaderNavarre is available as a \paysHuguenots general.

\phdipl
\aparag No Foreign intervention is allowed.
\aparag \REB can make a somewhat limited intervention in the war, only with
naval forces (in order to make or break blockade; no naval movement of Rebel
land forces) or with land forces in coastal besieged provinces of the Rebels,
in order to stop the siege; afterwards it can withdraw or remain in this
province only.
\aparag The Rebels control all cities in the rebel minor country (and not only
those with a \REVOLT in there).
\aparag \FRA can then announce an attempt of
\xnameref{pIII:FWR:Saint-Barthelemy}, and resolves this odious deed. This is
of course mandatory if this event happen due to \FRA's choice during
\ref{pIII:FWR Beginning}.
\aparag If \FRA is \CATHCR, \LIG can make a limited intervention as an ally of
\FRA.

\begin{digressions}[Specific conditions of the second event]


  \digression[pIII:FWR:Saint-Barthelemy]{Massacre of the Saint-Barth\'el\'emy}

  \phdipl
  \aparag 1d10 is rolled for every rebel leader, excepted \leader{Henri de
    Guise} and \leaderNavarre. An even result means that the leader was killed
  in the Massacre.
  \aparag Each city in the rebel minor country is taken by \FRA by rolling
  1d10 higher than the level of the fortress; one die is rolled for each
  city. The cities taken this way are military controlled by \FRA but still
  owned by the rebel minor country.
  \aparag The Rebels will have a malus of \bonus{-1} to receive its
  reinforcements at this turn.
  \aparag The Rebels can no longer make a limited intervention in
  \ref{pIII:Dutch Revolt}.
  \aparag \FRA loses {\bf 1} \STAB.
  \aparag The Survival roll of the French Monarch is modified by an additional
  \bonus{+1} until the end of the Wars of Religion.


  \digression[pIII:FWR:War2 Military]{Military operations after the
    Saint-Barth\'el\'emy}

  \phmil
  \aparag Use the general rules of \ref{pIII:FWR:Military Operations}.
  \aparag If all the leaders of on side are captured, wounded or killed, this
  side signs a level 1 peace in favour of its enemy at the end of the round.
  \aparag At the beginning of each military round (except the first), a new
  \REVOLT is rolled for in France. If this \REVOLT is in the rebel minor
  country and has no \REVOLT nor Loyalist land force in it, place a new
  \REVOLT \facemoins which takes the city.
  \aparag \FRA (and its allies) %have a bonus of \bonus{+1} to suppress \REVOLT
  % in France and
  perform automatic surrenders of rebel fortresses as in the previous war.
\end{digressions}

\phpaix
% \aparag A Truce is necessarily signed at the end of the turn, and the
% favoured side is determined as explained in \ref{pIII:FWR:Truces}. All the
% effects explained here are applied (so the \REVOLT are withdrawn before
% extension or Stability loss).
\aparag If \LIG was in intervention, allied to a \CATHCR\ \FRA, it wins 15 \PV
if the Truce is in favour of \FRA and \LIG had forces in at least one battle
or one siege (including voluntary surrender) against the Rebels.

\tour{Turn 2 and following}

\phevnt
\aparag The event goes on as described in \ref{pIII:FWR:Continuation1 Truce},
except that the military operations follow the rules of \ref{pIII:FWR:War2
  Military}, or as in \ref{pIII:FWR:Continuation1 War} if no peace was signed.



\event{pIII:FWR League}{III-D (3)}{The Rise and Fall of the League}{1}{PB}

\tour{Turn 1}

\phevnt
\aparag A new war breaks out. Apply the full effects of \ref{pIII:FWR:Politic
  Crisis}, \ref{pIII:FWR:Economic Crisis}, \ref{pIII:FWR:Uprisings} and
\ref{pIII:FWR:Military Troubles}.
\aparag If \REB spends 50\ducats, the Rebels will have a bonus of \bonus{+1}
to their reinforcement roll.
\aparag If \FRA is \CATHCR or \CATHCO, \LIG may give finances to \lig.  It
spends 100\ducats and takes the control of the stack commanded by
\leader{Henri de Guise} (he can take new forces during the military rounds as
long as the hierarchy is respected). One purpose of this is to attempt a Coup
by the League (as explained in \ref{pIII:FWR:League Coup}).

\phdipl
\aparag Usual Foreign interventions are permitted (even during the first
turn).

\begin{digressions}[Specific conditions of the third event]


  \digression[pIII:FWR:War3 Military]{Military operations during the League}

  \phmil
  \aparag Use the general rules of \ref{pIII:FWR:Military Operations}.
  \aparag At the beginning of each military round (except the first), a new
  \REVOLT is rolled for in France. If this \REVOLT is in the rebel minor
  country and has no \REVOLT nor Loyalist land force in it, place a new
  \REVOLT \facemoins which takes the city.
  \aparag \FRA (and its allies) %have a bonus of \bonus{+2} to suppress \REVOLT
  % in France and
  perform automatic surrenders of rebel fortresses as in the previous wars.


  \digression[pIII:FWR:League Coup]{Guise Coup and assassination}

  \phpaix
  % \aparag A Truce is necessarily signed at the end of the turn, and the
  % favoured side is determined as explained in
  % \ref{pIII:FWR:Truces}. All the effects explained here are
  % applied (so the \REVOLT are withdrawn before extension or Stability loss).
  \aparag If \LIG has taken control of \leader{Henri de Guise} and this
  general is not Captured, it may attempt a Coup that will make \leader{Henri
    de Guise} the Heir of the kingdom, by spending 100\ducats more.
  \aparag If \FRA is \CATHCO, or if \LIG has taken control of \leader{Henri de
    Guise}, \FRA may attempt to murder this pretender, even if \LIG does not
  attempt a Coup.
  \aparag Both those operations are described in the following event,
  \ref{pIII:FWR Succession} and are resolved as described in
  \xnameref{pIII:FWR:Coup Murder Pretender}.
  \bparag If the Coup is successful, \ref{pIII:FWR Succession} begins the very
  next turn, with \monarque{Henri de Guise} as the mandatory Heir (see
  afterwards).
  \bparag If \leader{Henri de Guise} was murdered and no event
  \shortref{pIII:FWR} happens (by Dynastic Crisis or rolled event), the Truce
  is broken by the \lig who is the Rebel for one particular war. Apply the
  procedure for a Truce broken, with \lig as the Rebels.
\end{digressions}

\tour{Turn 2 and following}

\phevnt
\aparag The event goes on as described in \ref{pIII:FWR:Continuation1 Truce},
except that the military operations follow the rules of \ref{pIII:FWR:War3
  Military}, or as in \ref{pIII:FWR:Continuation1 War} if no peace was signed.
\bparag If \leader{Henri de Guise} was murdered the previous turn and no
\shortref{pIII:FWR} happens (either by Dynastic Crisis or rolled event), the
Truce is now broken by the \lig who is the Rebel for this particular war.
Apply the procedure for the breaking of a Truce, with \lig as the Rebels.
\lig receives the general \leaderwithdata{Mayenne}.
\bparag Else, the Rebels are those of the previous war if the Truce is broken.
\aparag[Foreign limited interventions] (added to those already allowed).
\bparag Some limited interventions are allowed here; a country can help only
the first at-war country listed, or none at all.
\bparag \HOL can help \hug else a non \CATHCR\ \FRA.
\bparag \ENG Protestant or \CATHCR can help \hug else a non \CATHCR\ \FRA.
\bparag \ENG \CATHCR can help \lig, else a non Protestant \FRA.
\bparag \SPA can help \lig, else a non Protestant \FRA.



\event{pIII:FWR Succession}{III-D (4)}{War of Succession}{1}{PB}

\activation{This events is activated by a Dynastic Crisis during the Wars of
  Religion, or as the fourth event of \numberref{pIII:FWR}, or after a
  successful Coup by \leader{Henri de Guise}.}

\tour{Turn 1}

\phevnt
\aparag \hug and \lig revolt and will fight to impose their pretender on the
French Crown. Every one is sure now that there is no direct Heir of the last
Valois Monarch, \monarque{Henri III}.
\aparag If the French Monarch \monarque{Henri III} died at the beginning of
this turn, \FRA has to choose its Heir. Apply now the effects of
\xnameref{pIII:FWR:Designation Heir}, followed by the effect of the new
Religious attitude.
\aparag If a Coup was successful at the previous turn, the designated Heir is
now the one of the side having made this Coup. Apply his choice of Religious
Attitude.
\aparag Otherwise, apply only the event corresponding to the current Religious
attitude of \FRA; \FRA will have the opportunity to modify the would-be Heir
at the time of the death of the last Valois Monarch.
\aparag Only a Coup or a mandatory change of religion can change the Heir once
he is appointed.
\aparag Apply the full effects of \ref{pIII:FWR:Politic Crisis},
\ref{pIII:FWR:Economic Crisis}, \ref{pIII:FWR:Uprisings} and
\ref{pIII:FWR:Military Troubles}. Also apply \ref{pIII:FWR:War4 Military} and
\ref{pIII:FWR:War4 Peace}.
\begin{digressions}[The choice of the Heir]


  \digression[pIII:FWR:Designation Heir]{Designation of the Heir}

  \phevnt
  \aparag There are three possible Heirs.  Each one is linked to the choice of
  a Religious attitude, and \FRA can not change completely its attitude on its
  own: \CATHCR can not choose Protestant and a Protestant \FRA can not choose
  \CATHCR.  Any other choice is permitted.  \FRA can be forced to change its
  attitude because of a Coup.
  \aparag[\CATHCR] The Heir is \monarque{Henri de Guise}.  If \leader{Henri de
    Guise} is alive, the general is also the Heir; if not it's a cousin with
  random military capacities. The Heir has values 6/9/7.  When the Monarch is
  \monarque{Henri de Guise}, \FRA gains a free maintenance for one \ARMY
  \faceplus, event if it is still in Civil War. \FRA immediately annexes
  \provinceLorraine.
  \aparag[\CATHCO] The Heir would be \monarque{Henri IV}, that is a converted
  \monarque{Henri de Navarre}. If \leaderNavarre is alive, the general is also
  the Heir; if not it's a cousin with random military capacities. The Heir has
  values 9/9/9. When the Monarch is \monarque{Henri IV}, \FRA gains a free
  maintenance for one \ARMY \faceplus, event if it is still in Civil War.
  \aparag[Protestant] The Heir is \monarque{Henri de Navarre} who remains
  Protestant. If \leaderNavarre is alive, the general is also the Heir; if not
  it's a cousin with random military capacities. The Heir has values 9/9/9.
  \aparag[A new religious attitude] The designation of an Heir changes
  immediately the Religious Stand of \FRA.
  \bparag The Heir is Crowned now if the king is dead, or assists the king and
  will be crowned at the time of its death.
  \bparag If the Heir dies, another of the same family (and same
  characteristics) will stand forward.
  \bparag An Heir does not make Survival Test before its crowning; it will
  last 5 turns beginning with the turn of its crowning.
  \aparag
  Apply one of \xnameref{pIII:FWR:France is Protestant},
  \xnameref{pIII:FWR:France is Counter-Reformation} or
  \xnameref{pIII:FWR:France is Conciliant}.


  \digression[pIII:FWR:France is Protestant]{France is Protestant}

  \phevnt
  \aparag \lig rebels, following the general rules.
  \aparag If \monarque{Henri III} is dead and the Heir is crowned, \LIG can
  make a limited intervention from the first turn of the war.  Moreover, \lig
  will have a bonus of \bonus{+2} to its reinforcement roll.
  \aparag If \leader{Henri de Guise} is dead, \lig receive the general
  \leaderMayenne (B.2.2.2).
  \aparag \LIG can always make a limited intervention from the second turn of
  the war onward.
  \aparag \hug is immediately annexed by \FRA: its provinces become french
  provinces (and provide income as such) and its units (armies, leaders)
  become french units. Both the counter limits and free maintenance of \FRA
  resumes their regular values.


  \digression[pIII:FWR:France is Counter-Reformation]{France is \CATHCR}

  \phevnt
  \aparag \hug rebels, following the general rules.
  \aparag If \monarque{Henri III} is dead and the Heir is crowned, \HUG and
  \HOL can make a limited intervention from the first turn of the war.
  \aparag \HUG and \HOL can always make a limited intervention from the second
  turn of the war onward.
  \aparag \lig is immediately annexed by \FRA: its provinces become french
  provinces and its units become french units. Both the counter limit and
  maintenance of \FRA resume their regular values.


  \digression[pIII:FWR:France is Conciliant]{France is \CATHCO}

  \phevnt
  \aparag[If the king is \monarque{Henri III}, a Valois]
  \bparag Both \lig and \hug rebel, and a three-sided war begins between \FRA
  and the two Rebels.
  % \bparag The initial repartition of French forces is: \FRA has \ARMY
  % \faceplus, \lig has \ARMY \facemoins and \hug has \DT.
  % \bparag If the naval forces desert, decide at random if it is to
  % join the \hug or the \lig.
  \bparag \leaderNavarre is a possible Heir but is hesitant.  He is used as a
  general by \FRA, excepted if \hug controls or besieges \ville{Paris}.  He
  will go the side of \FRA as soon as he is chosen as Heir at the death of
  \monarque{Henri III}, or could go back to the Protestant side if
  \monarque{Henri de Navarre} is the chosen Heir, or if a Protestant Coup is
  made.
  \aparag Notice that as soon as \monarque{Henri III} die, one of the minor
  (the one having the chosen heir) will sign peace with \FRA and be
  immediately annexed.
  % Leader \leaderNavarre is a possible Heir but is hesitant.  Neither
  % the \hug nor \FRA can use it. He will go the side of \FRA as soon as
  % he is chosen as Heir at the death of \monarque{Henri III}, or could
  % go back to the Protestant side if \monarque{Henri de Navarre} is the
  % chosen Heir, or if a Protestant Coup is made.
  \aparag[If the king is the Heir,] (brand-new catholic \monarque{Henri IV}).
  \bparag \lig rebels, following the general rules.
  \aparag If \leader{Henri de Guise} is dead, \lig receive the general
  \leaderMayenne (B.2.2.2).
  \aparag If \leader{Henri de Guise} is alive, \lig will have a bonus of {\bf
    +2} to its reinforcement roll.
  \aparag \hug is immediately annexed by \FRA.
\end{digressions}

\phdipl
\aparag Foreign intervention are allowed.

\phadm
\aparag \FRA gets full income of all non-revolted, controlled provinces,
including those belonging to a revolted rebel or in the \ROTW.
\aparag As soon as the last Valois dies, \FRA is no more restricted in
administrative actions.
\aparag[Reinforcements of Rebels]
\bparag If \LIG spends 50\ducats, the \lig will have a bonus of \bonus{+1} to
their reinforcement roll.
\bparag If \HUG spends 50\ducats, the \hug will have a bonus of \bonus{+1} to
their reinforcement roll.

\tour{Turn 2 and afterwards}

\phevnt
\aparag Except for what follows, use the same rules as turn 1.
\aparag If the French Monarch \monarque{Henri III} died at the beginning of
some turn, \FRA has to choose its Heir (if no Coup has imposed an Heir). Apply
the effect of \ref{pIII:FWR:Designation Heir}, and then the effect of the
(possibly new) Religious attitude that follows. The revolted side receives new
\REVOLT according to \ref{pIII:FWR:Uprisings}.
\aparag Else, if a Coup was successful, apply \ref{pIII:FWR:Uprisings} to roll
for new \REVOLT of the now rebel side. The war resumes with rebels depending
on the new religious attitude.
\aparag If a pretender was murdered on the previous turn, new \REVOLT are
rolled for according to \ref{pIII:FWR:Uprisings} for this side only.
\aparag \paysSavoie will make (or continue) a limited intervention as an ally
of \lig (or \FRA if \CATHCR), with an \ARMY \faceplus and one unnamed minor
general.

\phadm
\aparag[Reinforcements of Rebels]
\bparag Reinforcements will be received for the rebel side(s) according to
\ref{pIII:FWR:Military Troubles} but the initial repartition of forces is not
made anew (it has already been done).
\bparag If \LIG spends 50\ducats, the \lig will have a bonus of \bonus{+1} to
their reinforcement roll.
\bparag If \HUG spends 50\ducats, the \hug will have a bonus of \bonus{+1} to
their reinforcement roll.

\begin{digressions}[Specific conditions of the War of Succession]


  \digression[pIII:FWR:War4 Military]{Military operations during the War of
    Succession}

  \phmil
  \aparag Use the general rules of \ref{pIII:FWR:Military Operations}.
  % \aparag \FRA and its allies have a bonus of \bonus{+3} to suppress \REVOLT
  % in France.
  \aparag \paysPalatinat makes (or continues) a limited intervention as an
  ally of the side of \leaderNavarre or \monarque{Henri de Navarre} with
  \ARMY\faceplus, \LD and a random general. If the Monarch is \monarque{Henri
    III} with \monarque{Henri IV} as the chosen Heir, \paysPalatinat makes no
  intervention.
  \aparag \FRA draws supply from any province in France (including those of
  \lig and \hug), except those in \REVOLT
  \aparag \lig and \hug draw supply only from the provinces they control.
  \aparag[Voluntary surrender]
  \bparag A city besieged by \FRA with at least one \ARMY \faceplus,
  voluntarily surrenders if there was no \REVOLT \faceplus in it at the
  beginning of the turn, nor is it a Place of Safety and there is no more
  \REVOLT in the province (including if the \REVOLT was just crushed this
  round).
  \bparag A city besieged by \lig with at least one \ARMY \faceplus,
  voluntarily surrenders if it is in the territory owned by \lig.
  \bparag A city besieged by \hug with at least one \ARMY \faceplus,
  voluntarily surrenders if it is in the territory owned by \hug.


  \digression[pIII:FWR:War4 Peace]{How to end the War of Succession?}

  \phpaix
  \aparag If there are only 2 sides in this war, the War of Succession ends if
  \FRA control \villeParis and all the places of safety and the fortresses
  where there was a \REVOLT\faceplus at some point during the war and has won
  a Major Victory over Rebel forces (at least 3 \DT of Rebels), or if all
  Rebel forces and \REVOLT have been eliminated.
  \bparag \FRA has to spend 100\ducats to stop the war; no Coup or
  Assassination can happen. Apply \ref{pIII:FWR:End of the War of Succession}.
  \aparag If there are only 2 sides in this war, the War of Succession ends if
  \FRA has no land forces left and the Rebel controls the city of Paris. A
  Coup in favour of the Rebels is automatically made with no possible murder
  attempt by \FRA. A mandatory change of Religious attitude is imposed on \FRA
  and the new Monarch is the Heir of the winning side. Apply \ref{pIII:FWR:End
    of the War of Succession}.
  \aparag \FRA ends as barely victorious if this is the end of the first turn
  of period IV (then no Coup is permitted). Apply now \ref{pIII:FWR:End of the
    War of Succession} and \ref{pIII:FWR Final}.
  \aparag If \lig is in rebellion, controls the city of Paris, and
  \leader{Henri de Guise} is alive, then \LIG can spend 100\ducats for an
  attempt of Counter-Reformation Coup.
  \aparag If \hug is in rebellion, controls the city of Paris, and
  \leaderNavarre is alive, then \HUG can spend 100\ducats for an attempt of
  Protestant Coup.
  \aparag If a Coup is attempted, \FRA can try to murder the pretender
  (\leader{Henri de Guise} or \leaderNavarre).
  \aparag If no Coup is attempted, \FRA can try to murder one pretender of
  revolted \lig or \hug (\leader{Henri de Guise} or \leaderNavarre).
  % \aparag \FRA loses no \STAB because of the \REVOLT but loses \STAB as in an
  % usual war (1 the first turn, 2 the second, \dots)
  \aparag The war keeps on until one side is victorious; there is no Truce.


  \digression[pIII:FWR:Coup Murder Pretender]{Coup and Murder of the
    Pretender}

  \phpaix
  \aparag The side attempting the Coup (\LIG or \HUG) has to spend 100\ducats
  then rolls 1d10 and adds \bonus{+2} if \FRA is \CATHCO; \bonus{+2} if the
  \ref{pIII:FWR:Saint-Barthelemy} was not perpetrated; \bonus{+3} if the
  \ref{pIII:FWR:Saint-Barthelemy} was made against the religious faction of
  the coup's side; \bonus{+2} if \FRA makes no Murder attempt; \bonus{+1} per
  victory of the pretender's minor country with at least one \ARMY \faceplus.
  \aparag[Failure of the Coup] If the result of the Rebels is 9 or lower, the
  Coup is failed. It may succeed if the result is 10 or higher.
  \aparag If \FRA attempts to murder the pretender, it rolls 1d10, and add
  \bonus{+2} for each point of \STAB that it spends (it has to have those
  points); and \bonus{+3} is no Coup attempt was made.
  \aparag[Result of Assassination] If the result of \FRA is 9 or lower, the
  murder is failed. It may succeed if the roll is 10 or higher.  \FRA loses
  {\bf 1} \STAB, and the Valois \monarque{Henri III} will have
  an additional permanent malus of \bonus{+3} to its Survival Test until his
  death. \\
  \centerline{\textit{"Il est encore plus grand mort que vivant."}}
  \aparag[If both a Coup and a Murder succeed]
  \bparag If the result of \FRA is higher of equals to the result of the Coup,
  the Coup actually fails; the Pretender is murdered.
  \bparag Else (if the result of Rebels is higher than the result of \FRA),
  the Coup succeeds. \FRA makes a mandatory change of Religious attitude and
  of designated Heir. The pretender is not killed (miraculously saved!) and
  becomes the new Heir.
  \aparag[Successful Coup]
  \bparag The new mandatory Heir is the one (\monarque{Henri de Guise} or
  \monarque{Henri de Navarre}) of the side doing the Coup and the Religious
  attitude of \FRA is changed according to this new Heir.
  \bparag When \monarque{Henri III} dies, the Heir is crowned as the French
  King.
  \bparag If this case, on the next turn, a Civil War with the new sides
  depending of the new Religious attitude continues, or begins if the Coup was
  during event (3).
  \aparag In addition, \FRA has a mandatory defensive alliance with the
  controller of the side having done the Coup, and this power can now make
  full intervention in the war until the end of \nameref{pIII:FWR}.


  \digression[pIII:FWR:End of the War of Succession]{End of the War of
    Succession}

  \phinter
%  \aparag \STAB of \FRA is raised by {\bf 2}.
  \aparag The new Monarch is the last designated Heir (\monarque{Henri III} is
  pushed aside if he is still alive...)
  \aparag All \REVOLT and forces of minor countries \hug and \lig are
  removed. But they continue to exist (they can rebel one more time if \FRA is
  not \CATHCO).
  \aparag[Intervention of Foreign countries]
  \bparag Minor countries having forces left in \FRA propose an immediate
  white peace to \FRA. If it is accepted, they withdraw and are at peace with
  \FRA. Else, they are now in a regular war with \FRA (but no one is victim of
  a declaration of war).
  \bparag Any Major power having forces left in \FRA has to sign a white
  peace, or are from now on in regular war with \FRA.  Their military activity
  is no more limited; nobody is victim of a declaration of war (but \FRA and
  its enemies are at war), and regular call to allies will be possible on the
  next turn.  This war causes normal loss of \STAB, beginning with a loss of
  {\bf 1} \STAB this turn.
  \bparag The only specificity of this war is that, if a unconditional peace
  is forced on \FRA, the winning power must change the Monarch of \FRA to the
  Heir of its Religious Attitude.  In this case this is the only condition of
  the peace, and \FRA has a mandatory defensive alliance with the winners
  during the reign of the new Monarch.
  \aparag As soon as \FRA is at peace at an end-of-turn and \CATHCO,
  \ref{pIII:FWR Final} is applied.
\end{digressions}



\event{pIII:FWR Last Stand}{III-D (5)}{Last Stand of the Heretics}{1}{PB}

\history{Alternate history}

\condition{}
\aparag If \ref{pIII:FWR Succession} is not finished, do not mark off and
reroll.
\aparag If \FRA is \CATHCO and no unconditional surrender was obtained by \FRA
against \hug in a previous war, mark off the event, play \RD instead and the
French king will have a malus of \bonus{+2} to his Survival test for the next
turn.
\aparag If \FRA is \CATHCO but did force an unconditional surrender of \hug in
a previous war, \hug rebels itself.
\aparag If \FRA is Protestant or \CATHCR at the end of \ref{pIII:FWR
  Succession} and \ref{pIII:FWR Final} was not applied, the rest of the event
happens.
\aparag If \FRA is Protestant or \CATHCR but \ref{pIII:FWR Final} already
occurred, play \RD instead with the \REVOLT on the table of \FRA.

\phevnt
\aparag One of \lig or \hug rebels itself depending on the religion of
\FRA. Apply the full effects of \ref{pIII:FWR:Politic Crisis},
\ref{pIII:FWR:Economic Crisis}, \ref{pIII:FWR:Uprisings} and
\ref{pIII:FWR:Military Troubles}.  Also apply \ref{pIII:FWR:War5 Military} and
\ref{pIII:FWR:War5 Peace}.
\aparag If the revolting minor was already annexed by \FRA (this may happen if
a mandatory religious change is then forced on \FRA), recreate it
immediately. It will get no troops at beginning.
\aparag If the non-rebelling minor still exists, it is immediately annexed by
\FRA: its provinces become regular French provinces and its units become
french units.
\aparag \REB is not obliged to do a white peace with \FRA.
\bparag If it chooses to continue a war, it can make a full military
intervention in the Civil War. But it will continue to suffer a normal loss of
\STAB at the end of turns, whereas \FRA will lose at most {\bf 2} \STAB each
turn during the Civil War.
\bparag If it chooses to sign a white peace, or if it was at peace, \REB can
make a limited intervention in the war.
\aparag \LIG can make a limited intervention as an ally of a \CATHCR\ \FRA.
\aparag \HOL can make a limited intervention as an ally of Protestant
\FRA. Else it can make a limited intervention as an ally of \hug.

\phdipl
\aparag Usual foreign interventions are allowed.

\begin{digressions}[Specific conditions of the last event]


  \digression[pIII:FWR:War5 Military]{Military operations during the fifth
    event}

  \phmil
  \aparag Use the rules of \ref{pIII:FWR:Military Operations}.
  % \aparag \FRA and its allies have a bonus of \bonus{+4} to suppress \REVOLT
  % in France.
  \aparag A city in \FRA that had not a \REVOLT \faceplus at the beginning of
  the turn, makes an immediate voluntary surrender if besieged by a land stack
  of \FRA (or its allies) that sets a siege with at least one \ARMY \faceplus
  and there is no more \REVOLT in the province (including if the \REVOLT was
  just crushed this round).


  \digression[pIII:FWR:War5 Peace]{How to end the Last Stand?}

  \phpaix
  \aparag \FRA loses at most {\bf 2} \STAB per turn because of the war.
  \aparag No Truce happens ever in this civil war. It keeps going until one
  side wins.
  \aparag The War ends if \FRA controls Paris, all the places of safety and
  the fortresses in provinces where there was a \REVOLT\faceplus at some point
  in the war and wins a Major Victory over
  Rebel forces (at least 3 \DT of Rebels) or if all Rebel forces and \REVOLT
  have been eliminated.
  \aparag The War ends if \FRA has no land forces left and the Rebel controls
  the city of Paris. An change of Heir in favour of the Rebels is
  automatically made (with no possible murder attempt by \FRA) that causes a
  mandatory change of Religious attitude. The new Monarch of \FRA is the Heir
  of the winning side.
  \aparag \FRA ends as barely victorious if the last turn of period III has
  ended (now or previously).


  \digression[pIII:FWR:End of the Last Stand]{End of the Last Stand}

  \phpaix
  % \aparag \STAB of \FRA is raised by {\bf 2}.
  \bparag The new Monarch is the last designated Heir (if it did change; the
  former one is pushed aside)
  \bparag All \REVOLT %and forces of minor countries \hug and \lig
  are removed.
  \bparag Minor countries having forces left in \FRA propose an immediate
  white peace to \FRA. If it is accepted, they withdraw and are at peace with
  \FRA. Else, they are now in a regular war with \FRA (but no one is victim of
  a declaration of war).
  % \bparag Minor countries having forces left in \FRA withdraw and are at
  % peace with \FRA.
  \bparag Any Major power having forces left in \FRA has to sign a white
  peace, or are from now on in regular war with \FRA.  Their military activity
  is no more limited; nobody is victim of a declaration of war (but \FRA and
  its enemies are at war), and regular call to allies will be possible on the
  next turn.  This war causes normal loss of \STAB, beginning with a loss of
  {\bf 2} \STAB this turn for everyone. The \SDCF could impose a change of
  Religion, but by normal rules and not by specific rules of this event.
  \bparag When this War ends, apply \ref{pIII:FWR Final}.
\end{digressions}



\event{pIII:FWR Final}{III-D (Final)}{End of the Wars of Religion}{1}{PB}

\activation{}
\aparag This event is applied when the fifth event \xref{pIII:FWR} is at last
resolved.
\aparag This event is applied also as soon as \FRA is at peace and \CATHCO
after the end of the fourth event.
\aparag At the end of the last turn of the period III (or the first turn of
period IV if \ref{pIII:FWR Succession} is happening), this event is applied
regardless of other conditions.

\phinter
\aparag The Wars of Religion are ended. Further events \numberref{pIII:FWR}
cause \RD with the \REVOLT in \FRA.
\aparag The Monarch should be the designated Heir, or the Heir is crowned
right now.
\aparag Minor countries \hug and \lig are immediately annexed by \FRA. All
their provinces are now regular provinces of \FRA. All their land forces
become french land forces. \FRA gets back its regular counter limit and
maintenance.  The navy is given back to \FRA. If alive, \leaderConde,
\leaderColigny, \leaderMayenne, \leaderNavarre and \leader{Henri de Guise}
retire (excepted the now Monarch); all other french leaders are now regular
french leaders.
\aparag If the king is \monarque{Henri de Guise} or \monarque{Henri IV}, \FRA
gains a free maintenance of one \ARMY \faceplus until the end of his
reign. This is not the case if the Monarch is \monarque{Henri de Navarre}.
\aparag[Victory Points]
\ENG, \HOL and \SPA win each 25 \PV if they have been allied at least once to
the side of the Heir that won finally the wars. They lose 25 \PV if they have
fought against this winning side.
\aparag[Economic consequences] Roll 1d10 and add \bonus{+1} for each
favourable truce conceded to the rebels, \bonus{+1} if \FRA has been complied
to change its Religious attitude, and \bonus{+1} is \FRA is \CATHCR.
\bparag Result 1-3: 1 level of French \TradeFLEET is lost to \HOL;
\bparag Result 4-5: 1 level of French \TradeFLEET is lost to \HOL, and 1 to
\ENG;
\bparag Result 6-10: 2 levels of French \TradeFLEET are lost to \HOL, and 1 to
\ENG; the \FTI of \FRA is diminished by \bonus{-1};
\bparag Result 11+: 2 levels level of French \TradeFLEET are lost to \HOL, and
2 to \ENG; both \FTI and \DTI of \FRA are diminished by \bonus{-1};
\bparag \HOL chooses first the \TradeFLEET it takes, then \ENG chooses.
\bparag If \FRA is \CATHCR, the level chosen by \HOL are lost but not received
by \HOL; \ENG gains the levels if it is Catholic, if not those levels are lost
for everyone.
\bparag if \ENG is \CATHCR and \FRA is not, \SUE chooses and gains the levels
instead of \ENG.
% \aparag Roll 1d10. Add \bonus{+2} if \CATHCR, \bonus{-2} if Protestant. On a
% result of 6+, one french \COL or \TP (decide at random between all those
% that exist) is lost.
\aparag[Undesired policy]
\bparag If the chosen Heir was Protestant but \FRA is no more Protestant at
the end of the Wars of Religion, \FRA has a malus of \bonus{-2} to all its
colonial actions during the period IV and its \FTI and \DTI are diminished by
a further 1.
\bparag If the chosen Heir was \CATHCO but \FRA is \CATHCR at the end of the
Wars of Religion, \FRA has a malus of \bonus{-1} to all its colonial actions
during the period IV.  Each event \RD obtained in period IV has a chance to
make appear a second \REVOLT \faceplus in \FRA. Roll 1d10: 1-3
\provincePoitou, 4-6: \provinceGuyenne, 7-10: none.
\bparag If the chosen Heir was \CATHCR but \FRA is no more \CATHCR, \FRA has a
malus of \bonus{-2} to all its Technological actions during the period IV.
Each event \RD obtained in period IV has a chance to make appear a second
\REVOLT \faceplus in \FRA. Roll 1d10: 1-3 \provinceArmor, 4-6:
\provinceOrleanais, 7-10: none.

% Local Variables:
% fill-column: 78
% coding: utf-8-unix
% mode-require-final-newline: t
% mode: flyspell
% ispell-local-dictionary: "british"
% End:

% LocalWords: pIII FWR PBNew malus Ligue Ile de unbesieged l'Hospital pII
% LocalWords: Barthelemy Casimir Schmalkaldic Mayenne Conciliant reroll Jym
% LocalWords: TODO
