\section{On Specific Minor Powers}
The minor powers that can also be major powers are mentioned in their
own chapters (\paysmajeur{Autriche}, \pays{Hollande},
\paysmajeur{Prusse}, \paysmajeur{Suede}, \paysmajeur{Venise}).

\subsection{Italian and Mediterranean countries}\label{chSpecific:Italy}

\subsubsectionJ{\sectionpays{Papaute}}{\blason{papaute}}
\label{chSpecific:Papacy}
\aparag If any power has a control of a province of \pays{Papaute}, the
\SDCF and the power that has the \pays{Papaute} on its diplomatic track
both have a \CB against the controller of the province as long as it has
one.
\bparag Remark that this is modified in case of \terme{Crusade} (until
the end of period III).
\bparag If \province{Lazio} is under control of \TUR at any time, all
Catholic players have a permanent free \CB against \TUR (until
\ville{Roma} is released).
\aparag[Diplomacy]
\bparag A Catholic country has a bonus due to having the same religion
on \pays{Papaute} only if it is Counter-Reformation.
\bparag[The Papal treasury]\label{chSpecific:Papacy:Gold} The Holy See provides
a financial help of 50\ducats to its controller if it is a Catholic and
at least in \MA (to be recorded in \lignebudget{Subsidies and dowries}).
\bparag \TUR may make no diplomacy on \pays{Papaute}.
\bparag Any Catholic player that declares war to \pays{Papaute} has to
spend double the usual cost of \STAB.
\aparag[The Pope in Venice.] If \ville{Roma} is conquered by \TUR, or if
\pays{Papaute} is annexed by \VEN, the Pope is taken in
\ville{Venezia}. \VEN gains a bonus of {\bf +1} to diplomacy attempts on
all catholic minor countries.

\subsubsectionJ{\sectionpays{chevaliers}}{\blason{chevaliers}}\label{chSpecific:Knights}
The \pays{chevaliers} minor country represents the Knights of the Order
Of Saint-John of Jerusalem. This minor country starts in the province of
\province{Rhodos} at the beginning of game in AD 1492.
\aparag[Diplomacy] Any Christian player declaring war to the
\pays{chevaliers} loses immediately {\bf 4} \STAB levels.
\bparag \TUR can make no Diplomacy on the \pays{chevaliers}.
\aparag[Relations with Turkey] The \pays{chevaliers} are always in a
state of restricted Overseas War against \TUR.
\bparag It allows them to use their Privateer and naval forces (no land
forces) to attack \TUR. \TUR can use its own naval forces to fight
against them.
\bparag The diplomatic patron of the \pays{chevaliers} play this forces,
or \SPA if they are neutral.
\bparag The annexes specify the reinforcements gained by the
\pays{chevaliers} each turn: a \corsaire\facemoins (or \Faceplus if in
\province{Rhodos}), and a \NGD or a \NDE.
\bparag This state of war does not cause automatic \STAB loss at the end
of turn.  But, each turn that the pirate of The Knights inflicts losses
on Turkish commercial fleets, \TUR loses 1 \STAB level if at peace and
not anti-prosperous.

\aparag[Transfer to Malta]\label{chSpecific:Knights:Transfer}
Whenever \province{Rhodos} is conquered by the Turkish player, the
Spanish player may cede the province of \province{Malta} to The
\pays{chevaliers} (if this province is still Spanish).
\bparag If \province{Malta} is not owned by \SPA when this happens, the
\SDCF may ask to the owner of \province{Corfu} or \province{Kreta} (if
Catholic) to transfer the \pays{chevaliers} on one of these islands. The
province is ceded in the same way and the \SDCF receives the benefits of
the operation (instead of \SPA). If no \MAJ accepts a transferal, the
\pays{chevaliers} is definitively destroyed.
\bparag In counterpart of the cession, \SPA receives the diplomatic
marker of the \pays{chevaliers}, placed directly in his \VASSAL box. It
will remain there until the disappearance of the \pays{chevaliers}. No
diplomacy is then allowed anymore on this minor, except for \SPA (for
possible diplomatic annexation).
\bparag Units of the \pays{chevaliers} are transferred and refilled
automatically to their new province upon ceding of this province by
\SPA.
\bparag If it is conquered by the Turkish player after this transfer,
all \pays{chevaliers} units are destroyed definitively, even if the
province is subsequently recaptured by a Christian player.

\aparag[Military forces of the Knights]
\bparag
Units of the \pays{chevaliers} are always \terme{Veteran}. Their
maintenance is free, including that of all reinforcements received.
\bparag Their \corsaire may not go out of \region{Mediterranee}. If
there is a port under Christian control (it doesn't matter which
Christian nation, player or minor country) touching sea zones in the
\seazone{Mediterranee E} or \seazone{Egee}, the privateer's dice rolls
are modified by {\bf -2}.
\bparag The rest of their forces can only be used against Turkish forces,
or forces of countries allied to \TUR in a current war.

\bparag[The Grand Master]\label{chSpecific:Grand Master}
The Knights have a permanent military leader named \leader{Grand
  Maitre}. It is never eliminated (a new \leader{Grand Maitre} is
automatically and immediately elected if it is killed, injured or
captured).  He may be used either as an Admiral (privateer Admiral
included) or a General, at the discretion of the player controlling that
minor.
\bparag[La Valette]\label{chSpecific:La Valette}
The \leader{Grand Maitre} is replaced by \leader{La Valette} if,
beginning with period III or the transfer of the \pays{chevaliers}
outside of \province{Rhodos}, a roll of 4 or more is obtained on
1d10. This roll is made in the first battle or siege were the
\leader{Grand Maitre} is used (except in naval combats because of the
limited Overseas War against \TUR). \leader{La Valette} remains for 4
turns in the game (including the current turn), and will be replaced
back by the usual \leader{Grand Maitre} at the end of this time (or if
killed or captured, or temporarily if injured). He may enter the game
only once.

\subsubsectionJ{Barbaresque countries}{\blason{cyrenaique}\blason{tripoli}\blason{tunisie}\blason{algerie}\blason{maroc}}\label{chSpecific:Barbaresques}
\aparag[The Barbaresques.] Barbaresque countries are \pays{Cyrenaique},
\pays{Tripoli}, \pays{Tunisie}, \pays{Algerie} and \pays{Maroc}. They
are always in a state of restricted Overseas War against every Christian
countries.
\bparag It allows them to use \corsaire and naval forces (no land
forces) to attack Christian countries. Christian countries can use their
own naval forces or \Presidios to fight against the Barbaresques.
\bparag As an exception, \corsaire of the Barbaresques may loot European
provinces adjacent to the \STZ they attack, even if they are European
provinces usually outside the scope of Overseas Wars.
\bparag \TUR plays the Barbaresques that are neutral, and the diplomatic
patrons play those that are not.
\bparag This state of war causes no loss of \STAB.
\bparag[Reinforcements] They receive some reinforcements each turn:
\pays{Algerie} gains a \corsaire\facemoins each turn; in periods I to
III it receives also a \ND or 2 \NGD (player's choice) and in periods IV
and after, only one \NGD or a \NDE. Other countries gain only a
\corsaire\facemoins 2 turns after their Privateer has been destroyed.
\bparag \textit{Exception.} Whenever \leader{Dragut} is in play and if
it used in its Privateer leader role, a \corsaire\facemoins of
\pays{Tunisie} is raised (even if eliminated at previous turn).
\bparag[Mandatory Sea Sortie] The Privateers usually have to go out at
sea each turn, except if their Patron decides against it: a test is made
at the beginning of the 2nd round if the Privateer is not at sea, by
rolling 1d10 for each country the Patron wants to keep the Privateer at
port.  This is permitted if the result is lower or equal to the number
of the current period plus the Diplomatic status bonus and the
geopolitical and bonus/malus (but not the religious one).
\aparag[Which seas are attacked]
\bparag \pays{Algerie} may send its Privateer in the
\seazone{Mediterranee W}, to attack both \ctz{Espagne} and \stz{Lion},
or in \stz{Lion}.
\bparag \pays{Tunisie} may send its Privateer in \stz{Lion} or
\stz{Ionienne}.
\bparag \pays{Tripoli} and \pays{Cyrenaique} send their Privateer in
\stz{Ionienne} or \ctz{Venise}.
\aparag All Christian countries have a permanent Overseas \CB against
the Barbaresque countries.
\aparag \Presidios may be installed in coastal provinces of Barbaresque
countries.

\aparag[Relations between \TUR and the Barbaresques]
Depending on several events, \TUR may have geopolitical malus to all
diplomacy attempts against all Barbaresque countries.
\bparag Initially (before event \eventref{pII:Alignment of
  Barbaresques}, or \eventref{pII:Algeria Vassalisation} at the end of
\leader{Barbaros}), \TUR has a {\bf -3} malus to all diplomacy attempts
against all Barbaresque countries.  This malus is cancelled afterwards.
\bparag Event \eventref{pIV:Morocco} puts back a {\bf -3} malus to all
diplomacy attempts against \pays{Maroc}.
\bparag Event \eventref{pVI:Barbaresques} sets a uniform {\bf -3} malus
to all diplomacy attempts against all Barbaresque countries (including
\pays{Maroc}).
\\
PD 07/20078: MORE TO DO
\begin{designnote}
These rules simulate both the clear trend toward inpendence of those regions,
the occasional in-fighting that are not expliciteley dealt with, but also leave
open the historical window of Turkish domination over those countries. 
\end{designnote}

\aparag[Pirates and Ottoman admirals]
\leader{Barbaros} and \leader{Dragut} may be used as Turkish leader if
their country is a \VASSAL of \TUR. They can then lead both Turkish
units and units from their own country.
\bparag[\sectionleader{Barbaros}] The first time \leader{Barbaros} is
reputed dead due to battle loss or attrition, he is in fact unavailable
for the rest of the turn but returns back in play at the beginning of
the following turn.

\subsubsectionJ{The Mamluks: \sectionpays{Egypte}
  and \sectionpays{damas}}{\blason{mamelouks}\blason{damas}}\label{chSpecific:Mamluks}
\aparag The two countries \pays{Egypte} and \pays{damas} are ruled by
the Mamluks. They are allied in all wars and will do a limited intervention
on the behalf of the other if involved in war.
\bparag In full war, \pays{Egypte} in naval or defensive
still send its forces freely in \pays{damas}. The converse is not true.
\aparag[Trade of Grand Orient.] In 1492, the \CCs{Grand Orient} is in
\ville{Alexandrie}. As long as it is the case:
\bparag \VEN earns half of the income of the \CCs{Grand Orient} if
\pays{Egypte} is not at war.
\bparag \TUR receives half of its income if it owns \ville{Damas}, or if
it has \pays{damas} its diplomatic chart.
\bparag In 1492, \pays{Egypte} knows
\seazone{Rouge}. \subeventref{pI:WRS:War Indian Sea} gives more
discoveries.
\aparag[Conquest by Turkey.] If, at a phase of peace, one Mamluk state
has no \ARMY counter left in any of its provinces and its capital is
controlled by \TUR, then the \MIN is destroyed and all its provinces are
annexed by \TUR.
\bparag When \pays{Egypte} disappears, the \CCs{Grand Orient} is
permanently displaced to \ville{Smyrna} and \TUR receives from now on
its full income. From now on, the convoy of \ville{Smyrna} is available.
\TUR gains all the discoveries of \pays{Egypte} (thanks to Piri Reis).

\subsubsectionJ{\sectionpays{Genes}}{\blason{genes}}
\aparag[Enmity with Venice] \VEN can make no regular diplomacy upon
\pays{Genes}.
\aparag \pays{Genes} has a commercial fleet and a base \FTI of 3, or 4
in periods IV to VII.
\aparag As long as \pays{Genes} as a commercial fleet in \ctz{Espagne},
\SPA has a diplomatic bonus of {\bf +2} on \pays{Genes}.
\aparag[Andrea Doria] The first time \leader{A Doria} is reputed dead
due to battle loss or attrition, he is in fact unavailable for the rest
of the turn but returns back in play at the beginning of the following
turn.

\subsection{German countries}

\subsubsectionJ{The Holy Roman Empire
  (\sectionpays{saint-empire})}{\blason{saint-empire}}\label{chSpecific:HRE}
\aparag \pays{saint-empire} is a political entity regrouping the German
minor countries of the \HRE:
\FEforeachlist{listofminorsHRE}{\pays{\loopitem}}{, }. The Emperor
(usually \HAB) has the following advantages.
\bparag The Emperor receives 50\ducats as subsides each turn.
\bparag The Emperor has a free \CB if any country of the \HRE is
attacked. This may change after \eventref{pIV:TYW}.
\bparag The Emperor may not be \HAB (or \SPA) due to
\eventref{pI:Emperor Election} or \eventref{pII:Emperor Election}.
\aparag[Imperial Army] 
The Emperor may use the counters of \pays{saint-empire} under
some conditions.
If the Emperor declares war following the previous \CB, or due to some 
events, he may use the counters of the \pays{saint-empire}. The Emperor 
has at its disposal 1\ARMY and 2\LD counters.
The maintenance of the imperial units is free.
\bparag 1 \terme{Veteran} \LD is obtained for free on the first turn.
\bparag Forces brought by the \HRE allies that are at least in \EW of
the Emperor can also be put directly in the imperial units.
\bparag[Placement of the units] The initial imperial units may be placed
in \province{Franken}, in the Emperor's own capital, or in the country
of the \HRE that triggered the intervention of the Emperor.
\bparag If the imperial units are placed in the Emperor's capital, some
of the Emperor's own units may also be transformed in imperial
units. However, they will not be returned at the end of the war.

\aparag[Reinforcements] During each Logistics phase of the intervention
(including the first), the Emperor may pay 50\ducats to roll a
reinforcement die (no modifiers, under the \terme{Defensive} attitude). The units
obtained there are imperial units (but \terme{Conscripts}).
\bparag Reinforcements may be placed in any province of the \HRE.
\bparag Fortresses levels may be put in either the attacked \HRE member,
or in the Emperor's territory if not possible.
\bparag Campaigns obtained there may only serve to move imperial units,
or units of a \HRE minor country (not \pays{Habsbourg}).

\aparag[Geographic limits] Imperial units may only be used in the \HRE,
in \region{Italie} or in any province of the Emperor (including Habsburg
autonomous states if \HAB or \SPA is Emperor,
see~\ruleref{chSpecific:Spain:Autonomous States}).
\bparag The provinces annexed by \FRA are no longer part of the \HRE
after their annexation.

\aparag[End of intervention] When the war that triggered the
intervention stops, the imperial units disappear (even if other wars are
ongoing).

\aparag[Capital of the \HRE] \ville{Frankfurt} is the capital of the
\HRE (in \province{Franken}). It may be annexed normally, but the
Emperor has a free \CB against the \MAJ that owns it.

\subsubsection{Alliances in the \HRE}
\aparag Some alliances in the \HRE may lead to the appearance of some
local alliances of minor countries with special rules: see
\eventref{pII:Schmalkaldic League}, \eventref{pIII:League Nassau},
\subeventref{pIV:TYW:Northern HRE Alliance},
\subeventref{pIV:TYW:Southern HRE Alliance} or even
\subeventref{pIV:TYW:German Empire}.

\subsubsectionJ{The \sectionprovince{OberPfalz}: \sectionpays{Baviere}
  and \sectionpays{Palatinat}}{\blason{baviere}\blason{palatinat}}
\aparag In 1492, \pays{Palatinat} has 2 \ARMY counters and
\pays{Baviere} 1 \ARMY counter. This changes during \eventref{pIV:TYW},
and may change permanently following \subeventref{pIV:TYW:Peace Prague}.
\bparag \pays{Baviere} may obtain a permanent bonus of {\bf +1} in
reinforcements following \subeventref{pIV:TYW:Peace Prague} or
\eventref{pVII:Bavarian Succession}.

\subsubsectionJ{\sectionpays{Suisse}}{\blason{suisse}}
\aparag[Restriction of Intervention]
\bparag A limited intervention by \pays{Suisse} is restricted to one
\ARMY\faceplus, that can only go in \region{Italie}.
\bparag Its is not possible to involve fully \pays{Suisse} in a war
except by declaring a new war against it. Thus full implication because
the country is in \EW or doing a limited intervention is not allowed.
\aparag[Military Specifics]
The forces of \pays{Suisse} before \TBAR are always \terme{Veteran} and
cancel the cavalry modifier of enemies.
\aparag[The Perpetual Peace]
During periods I to III, if the \pays{Suisse} army suffers a major
defeat during a battle (even if they were not alone), it may sign a
\terme{Perpetual Peace} with the winning player's country at the
conclusion of the upcoming Peace phase (player's choice).
\bparag When the peace is signed, \pays{Suisse} may no longer attack (or
have its units used by a player to attack) the winning country and
reciprocally.
\bparag This peace brings 10 additional \VP to the player that obtains
it, and a gain of {\bf 1} additional level in \STAB.
\bparag If \pays{Suisse} signs a perpetual peace, its diplomatic counter
can no longer ever exceed the \MA box of any player, including the
victorious player. The military specifics of \pays{Suisse} are cancelled
and \pays{Suisse} is now forbidden to make limited intervention in wars:
it can only be in wars if attacked.
\bparag The player receiving the benefit of the perpetual peace may,
until the end of period V, buy at the normal cost one Veteran \LD per
turn that is not counted in his turn limit.


\subsection{Northern and Western countries}
\subsubsectionJ{The Low Countries: \sectionpays{provincesne}}{\blason{hollande}}\label{chSpecific:Belgium}
\begin{designnote}
  The trading countries of the North-East were only step by step
  integrated in the empire of Charles V, between 1520 and 1543, either
  by military action or diplomatic weddings. They are assembled in a
  minor country called \pays{provincesne} (Low Countries).
\end{designnote}
\aparag The provinces \province{Holland}, \province{Utrecht},
\province{Gelderland}, \province{Overijssel}, \province{Friesland}
% (jym) Zeeland in provinces NE makes more sense
and \provinceZeeland 
are assembled in 1492 in the \pays{provincesne} minor country. See also
\ruleref{chSpecific:Spain:Spanish Holland}.

\aparag[Disappearance] After \eventref{pI:Habsburg Alliance}, \SPA may
annexe these provinces, either by military action (in which case a
special exception is granted to allow the capital to be taken as a
normal province) or through dynastic actions.
\bparag\label{chSpecific:Belgium:Diplomatic Annexation} A dynastic action may
be made to attempt to annex a province through diplomacy if \SPA and
\pays{provincesne} are not at war. The difficulty of this action is the
income value of the province, divided by two and rounded
down. \province{Gelderland} may not be annexed in this way.
% (jym) Zeeland in provinces NE
\bparag When \eventref{pI:Burgundy Inheritance} happens,
\provinceZeeland, if still owned by \pays{provincesne}, is immediately
annexed by \HIS.


\subsubsectionJ{\sectionpays{bourgogne}}{\blason{bourgogne}}\label{chSpecific:Burgundy}
\aparag[The status of Burgundy in 1492] There is a minor
\pays{bourgogne} in 1492. It is placed in \EW of \HAB. No diplomacy can
be attempted on it.
\bparag A declaration of war on \pays{bourgogne} is in fact a
declaration of war against \HAB.
\bparag \HAB may test normally for entry in war of \pays{bourgogne}.

\aparag[Spanish Low Countries] These are the provinces of
\province{Vlaanderen}, \province{Flandre}, \province{Hainaut},
\province{Brabant}, \province{Limburg}, \province{Luxemburg} and
\province{Artois}. They are annexed by \SPA as soon as
\dynasticaction{A}{2} (and thus \eventref{pI:Burgundy Inheritance}) is
played. They form the \terme{Spanish Low Countries}, who can be annexed
in one block (the parts that \SPA still owns) during \eventref{pV:WoSS}
by either \AUS, \ENG, \FRA or \SPA.
\bparag \province{Franche-Comte} is also inherited by \SPA, but is not
part of the \terme{Spanish Low Countries}.
% (jym) Zeeland in provinces NE
% \bparag \province{Zeeland} joins the \terme{Spanish Holland} or
% \pays{Hollande} as soon as either one exists.
\bparag The \CCs{Atlantic} is initially both in \pays{bourgogne} and
\pays{hollande}. It gives its incomes to \SPA after either
\eventref{pI:Habsburg Alliance} or \eventref{pI:Burgundy
  Inheritance}.

\subsubsectionJ{\sectionpays{Liege}}{\blason{liege}}\label{chSpecific:Liege}
\aparag[\sectionpays{Liege}]
It can only be a vassal or annexed by the owner of the
provinces of the \terme{Spanish Low Countries}, \SPA, \AUS, \ENG or \FRA
(initially, it would be \SPA but that owner may change depending on the 
consequences of \eventref{pV:WoSS}).

\subsubsectionJ{\sectionpays{Danemark}}{\blason{danemark}}\label{chSpecific:Denmark}
\aparag \anchorpaysmajeur{Danemark} may be played as a major country in
some setting.
\aparag \pays{Danemark} has a commercial fleet and a base \FTI of 3, or
4 in periods IV to VII.
\aparag[The Sund and Danemark]
\bparag In 1492, \pays{Danemark} has the Rights on the Levies on the
Sund (see \ruleref{chSpecific:Sund Levies}).
\bparag If \pays{Danemark} levies the taxes, it adds one \LD to its
reinforcements this turn. If it has the Rights on the Levies on the
Sund, \pays{Danemark} will take them if it is fully at war.  It may take
them if it makes a limited intervention (controller's choice).
\bparag Whenever \pays{Danemark} signs a victorious peace, it takes back
the Rights on the Sund, even if this condition is not part of the Peace
Treaty. In this case, the previous owner of those Rights has a free \CB
against \pays{Danemark} on the following turn if it was not on the
losing side of the peace.
\bparag The country having the Rights on the Sund can give them back to
\pays{Danemark} as a diplomatic announcement. The country gains a {\bf
  +2} on diplomatic actions on \pays{Danemark} this turn.
\bparag \pays{Danemark} is the only minor country that considers taking
the Rights on the Sund as a valid condition of peace.
\aparag[Relations with \sectionpaysmajeur{Suede}]
\bparag \SUE can not achieve a status better than \MA. See also
\ruleref{chSpecific:Sweden:Denmark} for the claim of \pays{Danemark} to the
Swedish Crown.
\bparag Lower the European Market by 75\ducats when \SUE (or
\pays{Suede}) and \pays{Danemark} are at war against each other. This
effect is not applied to any country that is involved in this war.

\subsubsectionJ{The Hansa}{\blason{hanse}}\label{chSpecific:Hansa}
\aparag The \pays{Hanse} has many capital cities in its provinces (the
country is an union of independent cities). It may be destroyed due to
event \eventref{pIV:TYW}.
\aparag The \pays{Hanse} has commercial fleets and a base \FTI of 2, or
3 in periods IV to VII.

\subsubsectionJ{The United States of America}{\blason{usa}}\label{chSpecific:USA}
\aparag The United States of America (\pays{USA}) is a new minor country
created by a defeat of the Colonial power during the event
\eventref{pVII:Independence War}.
\aparag[Forces of the USA]
The \pays{USA} have a basic force made of one \ARMY\faceplus.  This army
is of class \CAIV.
\aparag The \pays{USA} controls all rebel Colonies that have
victoriously seceded from the \MAJ.
\aparag
Any player and minor countries can now place commercial fleets in \STZ
located on sea zones adjacent to \pays{USA} territories, without
restriction.
\aparag The basic value of the European foreign market increases after
the creation of the \pays{USA}: read this income one line above the
usual line (except if at war against the \pays{USA}).
\aparag The \pays{USA} is a stoutly neutral country. Therefore, no
diplomacy is possible with the \pays{USA}.
\aparag[\sectionpays{USA} and War]
In case of declaration of war on the \pays{USA}, this minor rolls on
\tableref{table:Minor Reinforcements} both during the Logistic purchase
sub-phase and at the end of every winter round in the Military phase.
% Muf. Remplacer "tous les 3 rounds" par "après chaque hiver ?"
\bparag Also consider that American colonies of \pays{USA} have 6
levels each for militia and fortification, and for movement and supply
purposes of their own units (not applicable to foreign units).
\aparag[Qu\'ebec and South America]
Because the event may happen more than one time, there may exists more
than one country sharing the same characteristics of the \pays{USA}.

\subsection{Eastern countries}

\subsubsection{The Khanates and Cossacks}
\begin{histoire}
  In 1492, the khanate of the Golden Horde, heir of the mongol
  conquests, is but a shadow of its former glory. It still exists,
  however, and claims sovereignty over the other khanates. It is only in
  1502 that the khanate of Crimea destroyed the Golden Horde.
\end{histoire}
\aparag[The Golden Horde] \hfill \blason{steppes}
\bparag In 1492, the \pays{steppes} is the khanate of the Golden
Horde. Its basic forces is \ARMY\faceplus and \LD.
\bparag If a country declares war to either \pays{kazan},
\pays{astrakhan} or \pays{cosaquesdon}, the Golden Horde makes a full
intervention in war at the side of the \MIN. This does not apply if the
\MIN declares the war.
\bparag \eventref{pI:End Golden Horde} destroys the Golden
Horde. Henceforth, \pays{steppes} can no more use the \ARMY counter, and
its basic forces are reduced to \LD (and the basic reinforcements to
nothing). The defensive alliance is also broken and does not apply any
more.

\aparag[The Wastelands of the Khanates] The territories of the Khanates
are subject to \ruleref{chBasics:Wasteland}.

\aparag[Cossacks of \sectionpays{Ukraine}] This \MIN can be created by
\eventref{pIV:Revolt Cossacks}, where the specifics are described.

\subsubsection{The Nordic Orders, \sectionpays{Pskov}, \sectionpays{Ryazan}, \sectionpays{siberie}}
\aparag[Nordic Orders] \hfill \blason{teutoniques1} \blason{teutoniques2}
\bparag \pays{teutoniques1} and \pays{teutoniques2} may be destroyed by
\eventref{pI:Fall Teutonic} and \eventref{pIII:Northern Secularisation}.

\aparag[Russian Principalities] \hfill \blason{pskov} \blason{ryazan}
\bparag \pays{Pskov} or \pays{Ryazan} may be destroyed by
\eventref{pI:Pskov Ryazan}.
\aparag[\sectionpays{siberie}]\label{chSpecific:Siberia} \hfill
\blason{siberie}
\bparag Settlements of \TP/\COL are not allowed east of
\granderegion{Siberie} as long as \pays{siberie} exists. \TP and \COL
can be placed in \granderegion{Siberie} though.
\bparag \pays{siberie} is destroyed when there is at least 10 levels of
\COL (not \TP) in its territory, or when it is defeated in unconditional
surrender by any power.

\subsubsection{\sectionpays{Boheme}, \sectionpays{Hongrie}
  and \sectionpays{Transylvanie}}\label{chSpecific:Hungary}
  
\aparag[\sectionpays{Hongrie}] \hfill \blason{hongrie}
\bparag If during Period I or II, \TUR wins a Major Battle against a
stack with at least one \ARMY of \pays{Hongrie}, or takes military
control of \province{Hongrie}, the event \eventref{pI:Fall Hungary} is
activated next turn.
\bparag During the remaining of the turn, \HAB can make a limited
intervention as an ally of collapsing \pays{Hongrie}, with no
declaration of war by \HAB and no cost in \STAB. This intervention can
be made with all the available forces of \HAB at that time (no
reinforcements), without the usual limit for limited intervention.
\bparag If \eventref{pI:Hungarian Freedom} has been played, a victory
obeying the same conditions by \HAB against \pays{Hongrie} activates the
same event, in any Period.
\bparag Alternatively, \pays{Hongrie} may be annexed by the \hab at the
conditions described in \eventref{pI:Habsburg Hungary}.

\aparag[\sectionpays{Transylvanie}] \hfill \blason{transylvanie}
\bparag This minor country is created after \eventref{pI:Fall
  Hungary}.When it does not exist, military leaders of
\pays{Transylvanie} are leaders of \pays{Hongrie}.
\bparag The power that caused the creation of \pays{Transylvanie}
(because of \eventref{pI:Fall Hungary}) cannot own any province that
is initially in \pays{Transylvanie} and should always give it back at the 
next opportunity (even if it recreates the minor country).

\aparag[\sectionpays{Boheme}] \hfill \blason{boheme}
\bparag \paysBoheme may disappear because of event \eventref{pI:Habsburg
  Bohemia}, and be recreated (and destroyed also) in
\eventref{pIV:Bohemian Revolt}.
\aparag \pays{Boheme} and \pays{Hongrie} can be recreated as a Habsburg
autonomous state (see~\ruleref{chSpecific:Spain:Autonomous States}).
\aparag[\sectionpays{mazovie}] This minor country with no military
forces is a vassal of \POL and may be annexed by \POL under certain
conditions (see~\ruleref{chSpecific:Poland:Mazowia}). A declaration of war
against \pays{mazovie} by anyone but \POL is a declaration of war
against \POL.

% TODO: some precisions are due about the fall of Hungary if it goes to
% \HAB first.

\subsubsectionJ{\sectionpays{Perse}}{\blason{perse}}\label{chSpecific:Persia}
\aparag[\sectionpays{Perse} in the \ROTW]
\bparag See \ruleref{chDiplo:Diplo-ormus} for the rules about \pays{Ormus}
and its interactions with \pays{perse}. In the absence of \TP in
\pays{Ormus}, the resources in \pays{perse} are not exploited.
\bparag Because of some events, \pays{perse} may annex
\granderegion{Afghanistan}, and its units gain the right to go in the
\ROTW (as a country from the \ROTW, spending 4 \MP in rough terrain).

\aparag[Persian Uprising]\label{chSpecific:Persia:uprising}
\bparag If \pays{Perse} does not own all the provinces of the \region{Perse}
%(\province{Pars}, \province{Isfahan}, \province{Bam}, \province{Meshhed})
and a revolt happens %in one of those provinces, then:
in a country owning such a province (usually \TUR, sometimes \RUS) an uprising
may occur.
\bparag If the modified die roll to determine the revolted
province is 0 or less, the uprising occurs.
\bparag Othewise, roll 1d10, add 1 for each province of the \region{Perse}
currently owned by the country in which the revolt occured. The uprising
occurs on a result of 11 or more.
\aparag Choose at random one province of the \region{Perse} owned by the
country in which the revolt occured and place a Revolt\faceplus there. This
is the initial province of the uprising.
\bparag Place a Revolt\facemoins in all other provinces of the \region{Perse}
not owned by \pays{perse} (even if these are not owned by the same country as
the initial province).
\bparag These revolts are friendly to \pays{perse}.
\aparag If existing, \pays{Perse} declares a war against the owner of the initial
province of the uprising, taking reinforcements in offensive status ;
\bparag at the beginning of the war, \pays{perse} takes control of all the
fortresses in the provinces of the \region{Perse} currently owned by its
opponent.
\aparag If it does not exist, \pays{Perse} is re-created immediately and
declares war against the owner of the initial province of the uprising, taking
reinforcements in defensive status ;
\bparag it owns all the provinces of the \region{Perse} that were owned by its
opponent before the uprising.

\subsubsectionJ{\sectionpays{Gujarat}}{\blason{gujarat}}
\begin{histoire}
  After the naval victory in 1509 of Almeida before Diu, the city was
  finally taken only in 1534 (and conceded to Portugal in 1535) by the
  local sultan, which signed the end of Arab dominance in the area.
\end{histoire}
\aparag[Arab trade in India] \pays{Gujarat} has \TP in various areas of
the \ROTW.
\bparag In the areas owned by itself, the natives do react to foreign
presence.
\bparag A \TP is protected by the intrinsic fort if there is no city in
the province.
\bparag If there is a city in the province, it has to be taken. In the
case of \province{Mumbai}, \province{Goa} and \province{Kolikot}, this
means an Overseas War must be declared upon \pays{Vijayanagar}.
\bparag \ruleref{chSpecific:Portugal:Goa Colony} may apply.
\bparag The \TP in \continent{Africa} have 1\LD stacked with them.

% LocalWords: Espagne Venise Papaute Lazio Venezia Mamluks Egypte damas pI Piri
% LocalWords: Barbaresque Barbaresques Mediterranee Alexandrie Reis Mamluk Egee
% LocalWords:  Italie Habsburg Suisse Prusse Autriche Hollande Rhodos Kreta pII
% LocalWords:  Maitre Valette Cyrenaique Tunisie Algerie Maroc Ionienne Dragut
% LocalWords:  Barbaros Franken OberPfalz Palatinat Baviere pIII HRE pVII Sund
% LocalWords:  provincesne Danemark bourgogne Gelderland khanate Khanates WoSS
% LocalWords:  khanates pV Hansa Hanse mazovie Mazowia siberie Hongrie Boheme
% LocalWords:  Transylvanie pIV TYW perse Almeida Diu
