% -*- mode: LaTeX; -*-

\chapter{Military Concepts}\label{chapter:MilitaryConcepts}

\begin{designnote}
  This Chapter describes the main concepts and common rules used
  during the Military phase, such as stacking limits, handling of
  multi-national stacks, supply, \ldots

  The proper rules are presented in Segment order in the previous
  Chapter. Most concepts presented here are common with others wargames and a
  rough idea about them is enough to both understand the flow of the military
  phase and play the most frequent situations. Thus, the main rules are
  presented first. Of course, even if concepts are common with other games,
  the precise details are not the same and you should refer to these rules
  when the need arise.

  You may probably skip this Chapter during a first reading, as it sometimes
  becomes ``precisions for the insanes''.
\end{designnote}

\begin{todo}
  This Chapter is under heavy work. The absence of detailed numbering of rules
  reflects this.

  It is currently mostly reminders of what I need to write here.
\end{todo}

\section{Initiative}
Initiative of a country both in intervention and at war.

Initiative of a minor alone controlled by a country otherwise at war.

\section{Stacking}

\section{Command}
Leading multi-national stacks.

\section{Hierarchy}
Fear this!

\section{Supply}
Source of Supply, Line of Supply, Supply by naval stacks, ports, arsenals.

\section{Campaigns}
Especially, detailled stuff about the extended campaigns\ldots

% Local Variables:
% fill-column: 78
% coding: utf-8-unix
% mode-require-final-newline: t
% mode: flyspell
% ispell-local-dictionary: "british"
% End:
