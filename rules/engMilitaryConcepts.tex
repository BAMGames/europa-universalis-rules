% -*- mode: LaTeX; -*-

\chapter{Military Concepts}\label{chapter:MilitaryConcepts}

\begin{designnote}
  This Chapter describes the main concepts and common rules used
  during the Military phase, such as stacking limits, handling of
  multi-national stacks, supply, \ldots

  The proper rules are presented in Segment order in the previous
  Chapter.

  You may probably skip this Chapter during a first reading, it sometimes
  becomes ``precisions for the insanes''.
\end{designnote}

\begin{todo}
  This Chapter is under heavy work. The absence of detailed numbering of rules
  reflects this.

  It is currently mostly reminders of what I need to write here.
\end{todo}

\section{Rounds and Impulses}
The military phase is split into several \emph{rounds}. During each round,
alliances all get an \emph{impulse} to play (in decreasing order of
initiative). During its impulse, each alliance plays a number of segments,
similar to the segments of other phases.

\section{Initiative}
Initiative of a country both in intervention and at war.

Initiative of a minor alone controlled by a country otherwise at war.

\section{Command}
Leading multi-national stacks.

\section{Hierarchy}
Fear this!

% Local Variables:
% fill-column: 78
% coding: utf-8-unix
% mode-require-final-newline: t
% mode: flyspell
% ispell-local-dictionary: "british"
% End:
