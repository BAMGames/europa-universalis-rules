% -*- mode: LaTeX; -*-

\chapter{Military Concepts}\label{chapter:MilitaryConcepts}

\begin{designnote}
  This Chapter describes the main concepts and common rules used
  during the Military phase, such as stacking limits, handling of
  multi-national stacks, supply, \ldots

  The proper rules are presented in Segment order in the previous
  Chapter. Most concepts presented here are common with others wargames and a
  rough idea about them is enough to both understand the flow of the military
  phase and play the most frequent situations. Thus, the main rules are
  presented first. Of course, even if concepts are common with other games,
  the precise details are not the same and you should refer to these rules
  when the need arise.

  You may probably skip this Chapter during a first reading, as it sometimes
  becomes ``precisions for the insanes''.
\end{designnote}

\begin{todo}
  This Chapter is under heavy work. The random presence of detailed numbering
  of rules reflects this.
\end{todo}

\section{Description of Military forces}

We describe here the different kinds of military forces: troops, navies and
fortifications. Troops and navies work in similar ways, especially with the
notion of \terme{detachment}, but with small differences.

All these forces are in limited amount. The number of counters provided in the
game is an absolute limit on what is usable. Different countries have
different number of counters of each kind.

Exception: \pays{pirates} \corsaire, \pays{natives} troops, neutral fortresses
and revolted/rebelled troops are not limited. If you need more than provided
by the game, you may use whatever you wish to represent them.



\subsection{Land forces}


\subsubsection{Troops}\label{chLogistic:Troops definition}
\aparag[Troops] are represented by three different kinds of counters
corresponding to various size of land forces: \terme{Army} (\ARMY),
\terme{Land Detachment} (\LD) and \terme{Land Detachment of Exploration}
(\LDE).
\bparag The basic count unit is one \LD.

\aparag[Detachments.] One \LD represent some infantry and cavalry. The precise
number of them depends on the country and the period. \LD are abstract
representation of small field forces and consists roughly in 1000 to 10000
soldiers.

\aparag[Armies.] \ARMY counters have two sides. An \ARMY\facemoins is two \LD
plus some field and siege artillery. An \ARMY\faceplus is four \LD plus more
artillery.

\aparag[Breaking armies.] An \ARMY counter can be broken into an equivalent
number of \LD (2 or 4) of the same country at any time in the game. Note that
artillery is lost in the process.
\bparag Similarly, an \ARMY\faceplus can be turn into one \ARMY\facemoins and
two \LD at any time.
\bparag However, an \ARMY\faceplus may not be broken into two
\ARMY\facemoins as this would create an \ARMY counter (see below).
\bparag Especially, \ARMY can be broken during movement or to satisfy losses
(whether combat or attrition). If one \ARMY\faceplus suffers a 1 \LD loss,
there is one \ARMY\facemoins and 1 \LD remaining.
\bparag However, if there are not enough \LD counters to satisfy the loss,
heavier loss are suffered. If one \ARMY\facemoins suffers a 1 \LD loss but
there are no more unused \LD of the same nationality available, then the
entire \ARMY\facemoins is annihilated.

\aparag[Creating and reinforcing armies.] The only way to create a new \ARMY
counter is to buy it during the Administrative phase (logistic segment).
\bparag Especially, it is never possible to ``merge'' two \LD into an
\ARMY\facemoins nor to break an \ARMY\faceplus into two \ARMY\facemoins.
\bparag On the other hand, it is possible to reinforce an \ARMY\facemoins with
two \LD (in one stack) and turn it into an \ARMY\faceplus. This can be done at
any time in the game.
\bparag It is also possible to merge two \ARMY\facemoins into one
\ARMY\faceplus.

\aparag[Special armies.] The armies of \pays{saint-empire} and \pays{croises}
act as containers. Each may contain up to 4\LD of some nationality and can be
created at any point during the turn. The precise contents of these armies
must be written done in order to give back the \LD to their owners when the
army is broken.
\bparag As an exception to normal rules, these \ARMY can be created during the
military rounds.

\aparag[Detachments of Exploration.] One \LDE represents roughly one third of
a \LD.
\bparag \LDE can only exists on the \ROTW map (including European provinces on
the \ROTW map). As soon as one \LDE enters the European map, it is immediately
destroyed.
\bparag One \LD can be split in 3 \LDE at any time (especially to satisfy
losses in the \ROTW). 3 \LDE stacked together must be turned into 1\LD after
movement.
\bparag For maintenance and purchase, 1\LDE costs as much as half a \LD.
\bparag \LDE are never counted in stacking and supply limits.

\aparag[Natives.] Each \ROTW \Area holds a certain number of natives per
province. They are written on the \ROTW map in number of \LD or \LDE.
\bparag Counters (from the \pays{natives} ``country'') are provided to
remember losses of natives in each province. You may use
\ARMY\faceplus/\facemoins to represent 4/2 \LD of natives but this is for
convenience only: they are not considered as \ARMY for game purposes. These
counters are in unlimited quantity.


\subsubsection{Military doctrine}
\aparag Each country has an \terme{Army Class} written in roman numerals on
its counters.
\bparag The class of a country determines three factors: its \terme{Size}, its
\terme{Cavalry} and its \terme{Artillery}.
\bparag Some countries (mostly majors) belong to one army class but have
special cases for artillery and cavalry.
\bparag The army class of each country can be read in~\ref{table:Army
  Classes}. There is one line per class with its number and name on the left
and the list of countries belonging to it on the right.
\bparag Most minor countries are grouped according to their cultural groups.
\bparag The army class of minor countries can also be found in their
description in the appendices under the name ``Military doctrine''.
\bparag Note that when these values change (with a new period), each existing
land counter is automatically updated to the new values.

\GTtable{armyclasses}

\aparag[Size.] The army size of each country, per period, can be read
in~\ref{table:Army Classes} by cross-referencing the army class of the country
(or its name) with the current period.
\bparag The result is a number between 0 and 7 representing an abstract
measure of the typical size of forces fielded by this country during that
period.
\bparag A larger size means that the country usually fielded more men in
battles. However, this is an abstract measure and there is no direct
correspondence between the size and an actual number of soldiers. Moreover,
these numbers are relative (to other countries). A decreasing size does not
mean that the country had smaller armies, but rather than its neighbours
started having larger ones.
\bparag Countries with larger size do more damage in battle when facing
countries with smaller size.

\aparag[Cavalry] is abstractly represented by giving a small bonus in battle
to certain classes of armies during certain periods of the game.

\aparag[Artillery.] Each \ARMY\facemoins and \ARMY\faceplus contains a certain
number of artillery. This number is an abstract representation (rather than an
actual number of guns and howitzers) of the amount and efficiency of field and
siege artillery.
\bparag The number of artillery per \ARMY\faceplus can be read
in~\ref{table:Artillery Per Army} by cross-referencing the country (or class)
with the current period.
\bparag An \ARMY\facemoins always contains half the number of artillery of an
\ARMY\faceplus (rounded down).

\GTtable{artilleryvalue}

\aparag[Artillery of stacks.]\label{chMilitary:Stacks:Artillery}
When two (or more) \ARMY are stacked together,
their artillery numbers do not simply add. Instead, use the following
computation:
\bparag Take the artillery value of one \ARMY in the stack (the larger the
better); add \bonus{+2} if there is another \ARMY with 2 or more artillery
otherwise, add \bonus{+1} if there is another \ARMY with 1 artillery.

\begin{exemple}
  \FRA is of class \CAIV (``majors''). In periods \period{I} to \period{IV},
  it has a size of 2, then 3 in period \period{V} and 4 afterwards.

  In period \period{II}, \FRA has 3 artillery per \ARMY\faceplus. Thus, it has
  only 1 artillery per \ARMY\facemoins (3/2, rounded down). A stack with
  2\ARMY\faceplus of \FRA is thus considered to have 3 (first \ARMY) + 2
  (second \ARMY with 2 or more artillery) = 5 artillery for all game purposes
  (battles and sieges). A stack of \ARMY\faceplus \ARMY\facemoins has 3 + 1
  (second \ARMY with only 1 artillery) = 4 artillery. Lastly, a stack of 3
  \ARMY\facemoins of \FRA only has 1 (first \ARMY) + 1 (second \ARMY with 1
  artillery) = 2 artillery (\emph{i.e.} the third \ARMY counter does not add
  any artillery to the stack).
\end{exemple}



\subsection{The Navy}

\aparag[Naval forces] are represented by three different kinds of counters
corresponding to various sizes of naval forces: \terme{Fleet} (\FLEET),
\terme{Naval Detachment} (\ND) and \terme{Naval Detachment of Exploration}
(\NDE).
\bparag The basic count unit is one \ND. However, there are several kind of
\ND corresponding to various type of ships.

\aparag[Warships and Galleys.] \ND can represent different kinds of
ships. Mostly warships, galleys or transports.
\bparag Thus, there are several kind of \ND: \terme{Naval Warship Detachment}
(\NWD), \terme{Naval Galley Detachment} (\NGD) (also the \terme{Galleass},
written \VGD because they are first used by \VEN) and \terme{Naval Transport
  Detachment} (\NTD).
\bparag All those naval detachments are treated differently, but some rules
apply to all. In this case, the generic term \ND will be used.
\bparag \VGD are considered \NGD when the case apply (\emph{i.e.} whenever
there are rules for \NGD without special cases for \VGD, these rules apply).
\bparag \NGD can only exists in the \regionMediterrannee and the
\regionBaltique.

\aparag[Detachments.] One \ND represents roughly 2 to 6 first category ships
(galleys, galleons, man-o-war, \ldots) plus various second category ships. The
precise number depends on the country, the period and the kind of ships
involved.
\bparag \NTD only contains transport ships. They do not participate in battles
but can be used to transport troops or gold.

\aparag[Fleets.] \FLEET counters are containers. They may hold a certain
number of \NWD (or \NGD) plus some \NTD.
\bparag Unlike \ARMY, the exact content of a \FLEET counter depends both on
countries and period (representing evolution of the naval doctrines).
\bparag The (maximal) content of the fleets is detailed in the
\tableref{table:Countenance of Fleets}. It can contain a number of \NWD (a
\NGD counts for half a \NWD) and a number of \NTD. This number depends on the
period and the country involved.
\bparag There is one line per country (or class) and one column per
period. Each box contains four numbers as ``$x/y:x'/y'$''. The first two
($x/y$) are the maximum number of \NWD/\NTD in a \FLEET\facemoins and the last
two ($x'/y'$) are the maximum number of \NWD/\NTD in a \FLEET\faceplus.
\bparag A \FLEET is put on the side \Faceplus only if there is not enough room
in a \FLEET\facemoins to accommodate all the \ND. The counter is turned as
necessary.
\bparag Since the exact content of \FLEET counters is not fixed, it must be
written down. There is space for this on the colonial record sheet of each
country.
\bparag Note that when the capacity change (with a new period), existing
\FLEET counters are not automatically ``topped up'' to their new
capacity. They keep the same number of \ND and nez ones must be brought. It is
however possible that the content of a not full \FLEET\Faceplus suddenly fits
into a new size \FLEET\Facemoins\ldots

\GTtable{fleetsize}

\aparag[Creating and breaking fleets.] A \FLEET may be broken into several \ND
(depending of its content) at any time.
\bparag Similarly, several \ND can be merged into a \FLEET (or incorporated
into an existing one) at any time. \FLEET counters may be created this way.
\bparag Even if it does not provides a direct military advantage (such as the
artillery for \ARMY), using \FLEET rather than \ND usually decrease
maintenance cost and allows for more concentration of forces (because of
stacking limits).

\aparag[Detachments of Exploration.] One \NDE represents roughly one or two
warships (one third of a \NWD).
\bparag \NDE can exists both on the \ROTW and European maps.
\bparag One \NWD (only) can be split in 3 \NDE at any time (especially to
satisfy losses). 3 \NDE stacked together must be turned into 1\NWD after
movement.
\bparag For maintenance and purchase, 1\NDE costs as much as half a \NWD.
\bparag \NDE are never counted in stacking and supply limits.

\aparag[Pirates.] The last naval forces are pirates and privateers. They
represent independent sailors that attack trade fleets. Privateers (\corsaire)
work for one country; pirates are represented by the (abstract) minor country
\pays{pirates} (who mostly has \corsaire units).

\aparag[Trade fleets] (\TradeFLEET) are not naval forces. They only represent
trade activity (not specific ships), do not move and can only be attacked by
\corsaire.

\begin{exemple}
  In period \period{III}, the size of English fleets is ``2/1:5/1''. Thus, a
  \FLEET\Facemoins of \ANG may contain up to 2\NWD and 1\NTD while a
  \FLEET\Faceplus may contain up to 5\NWD and 1\NTD.

  If \ANG wishes to group together 3\NWD (and no \NTD), it must use a
  \FLEET\Faceplus (and pay the maintenance cost for one) because this cannot
  fit within one \FLEET\Facemoins.

  In period \period{III}, \TUR has also a fleet size of ``2/1:5/1''. However,
  since \NGD only count as half a \ND in fleet countenance, one
  \FLEET\Faceplus of \TUR may hold up to 10\NGD and 1\NTD.
\end{exemple}



\subsection{Fortifications}

\aparag Fortifications are immobile forces used to defend provinces. There are
two kinds of fortifications: fortresses and forts. In Europe, fortifications
represent the whole defence system of the province thus including several
actual fortresses, citadels, fortified towns, \ldots
\bparag Fortifications are also supply sources for both land and naval troops.

\subsubsection{Fortification counters}
\aparag[Fortresses] have a level between 1 and 5.
\bparag Each European province, as well as some \ROTW provinces, has a basic
fortress of level either 1 or 2 drawn of the map.
\bparag Fortresses of higher level may be built provided the country has a
sufficiently high technology.
\bparag Fortresses may lose levels due to sieges. If this puts the fortress
below its basic level, use the white level 1 counters to denote it. In no case
can the fortress of a province with a basic fortress go below 1.
\bparag Note that fortresses counters are double-sided. Thus, building a
fortress prevents a country from building the one on the back of the
counter. It is always possible to switch one fortress counter for another (of
the same level (and country)) if the need arise.

\aparag[Forts] are sometimes referred to as ``level 0 fortresses''. They may
only exist in the \ROTW.
\bparag All colonial establishments (\COL and \TP), as well as missions
automatically have a fort.
\bparag Other forts may be built during the military phase by land forces.
\bparag A \COL of level 6 is considered to be an European province. Thus, it
gains for free a basic fortress of level 1. Use white level 1 counters to
denote it. Since this is a basic fortress, there is no need to pay for its
maintenance.

\aparag[\Presidios] are small fortifications built in enemy territory to try
and control access to the sea rather than the land itself.
\bparag In European provinces where there is a circled anchor (whatever its
colour), a foreign country may build a
\Presidio. See~\ref{chRedep:Presidios} for building it
and~\ref{chMilitary:Presidios} for its effects.

\aparag[Arsenals.] Some countries have fortresses counters with a gold anchor
on them. These are \terme{arsenals}.
\bparag Arsenals can only be built in the \ROTW (exception:
\construction{Gibraltar}, \construction{Sebastopol} and
\construction{Saint-Petersbourg}).

\subsubsection{Fortifications as Supply sources}
\aparag[Land Supply.]
\bparag Forts may only supply detachments (\LD or \LDE).
\bparag Other fortresses can supply any number of land troops, whatever the
level of the fortress.
\bparag \COL and \TP, although they only have a fort, are supply sources for
any number of land troops (that is, the establishment has more supply capacity
that its intrinsic fort).

\aparag[Naval Supply.]
\bparag Forts may only supply detachments (\ND or \NDE).
\bparag Regular (non-arsenal) ports can supply any number of naval stacks
containing at most one \FLEET counter (each), whatever the level of the
fortress.
\bparag Arsenal can supply any number of naval forces of any size.
\bparag \COL and \TP, although they only have a fort, are supply sources as
regular ports: each can supply any number of naval stacks containing at most
one \FLEET counter (that is, the establishment has more supply capacity that
its intrinsic fort).

\begin{designnote}
  Note that supply limits are cumulative. That is, a single fortress may
  supply as many stacks (land or naval) as wanted, as long as it can supply
  each of them individually. There is no ``using up'' of the supply capacity.

  The ``extra supply capacity'' of \COL or \TP (with respect to their
  fortification level) is reminded in the size of the counter: they use big
  counters because they have a lot of food.
\end{designnote}



\subsection{Veteran and Conscripts}
\label{chMilitary:Veteran Conscripts}

\aparag[Veterans and Conscripts.] All land forces can be either
\terme{Veteran} or \terme{Conscripts}. A \terme{Veteran} army has seen more
battles than conscripts, is better trained, and less likely to flee in the
presence of the enemy. A \terme{Conscript} army is formed of newer soldiers
and paid less.
\bparag \terme{Veteran} have a bonus in battle (better moral). However, their
maintenance cost is also higher.
% forces have a morale higher by 1 point compared to
% \terme{Conscripts}. However, the maintenance cost of \terme{Conscripts}
% units is lesser than the \terme{Veteran} ones.

\aparag[Who is Veteran?] If the country is at peace (being only engaged in
\terme{Overseas Wars} (see \ref{chDiplo:Overseas Wars}) and limited
interventions (see \ref{chDiplo:InterventionLimitee}) counts as being at
peace), all land forces are maintained as \terme{Veteran} forces, using the
\terme{Peace maintenance} price.
\bparag If the country is at war, then all land forces already existing at the
beginning of turn can be maintained (unit by unit) as either \terme{Veteran}
or \terme{Conscript}. Newly recruited units are \terme{Conscripts}.

\aparag[Mixed stacks.] A force formed by stacking or merging several units is
\terme{Veteran} only if more than half of the \LD composing the units are
\terme{Veteran}.
\begin{exemple}
  An \ARMY\faceplus composed by the merging of 1 \terme{Veteran} \LD and 1
  \terme{Veteran} \ARMY\facemoins and 1 \terme{Conscript} \LD is considered to
  be a \terme{Veteran} unit. But if this \ARMY\faceplus is stacked with an
  \ARMY\facemoins of \terme{Conscripts}, this stack is considered
  \terme{Conscripts} (since there are as many \terme{Conscripts} as there are
  \terme{Veteran}). However, if one \LD of this stack is destroyed (due to
  battle or attrition), one \LD of \terme{Conscripts} will be removed (leaving
  either 2\ARMY\facemoins and 1\LD or 1\ARMY\faceplus and 1\LD), and the stack
  as a whole will be \terme{Veteran}.
\end{exemple}

\begin{playtip}
  Think twice before upkeeping troops as \terme{Conscripts}. The extra moral
  will make a huge difference in battles and is usually worth your money.
\end{playtip}

\aparag[Navy] Naval forces are \terme{Veterans} if they are maintained from a
previous turn, or \terme{Conscripts} if there are newly recruited this turn.
\bparag The difference only occurs for Naval technologies \terme{Vessel} and
\terme{Three-decker}.% \terme{Veteran} naval forces have morale 4 instead of 3.

%%%%%%%%%%%%%%%%%%%%%%%%%%%%%%%%%%%%%%%%%%%%%%%%%%%%%%%%%%%%%%%%%%%%%%

\section{Initiative}
Initiative of a country both in intervention and at war.

Initiative of a minor alone controlled by a country otherwise at war.

\section{Stacking}\label{chMilitary:Stacking}
\aparag[Empilement]
\bparag sur terre, [3 pions et 8 DT] + 2 pachas
\bparag  sur mer [3 pions] + 2 Tr
\bparag de et dc ne sont pas comptés, mais au plus 2 dans une pile.
\bparag[Arsenaux et ports] Ports réguliers ne peuvent contenir plus d'un pion
\FLEET.  Seuls les arsenaux (en Europe: indiqués sur la carte; \ROTW (et cas
particuliers) : indiqué sur le pion forteresse) peuvent accueillir 2 ou 3
\FLEET.  Les forts ne peuvent accueillir que des \DN ou \NDE.

La \textbf{taille} d'une pile militaire est donnée par la moyenne,
arrondie à l'inférieure, des tailles de chaque unité. Cette moyenne
est comptabilisée en équivalents détachements (donc une A+
compte pour 4 fois plus qu'un DT) et arrondie strictement à l'inférieur.

\textbf{Exception:} Si des pachas accompagnent l'armée, le moral est forcément
conscrit. Le restant (taille et technologie) est déterminé selon la règle
normale.

+ techno d'une pile (médiane). ATTENTION, besoin d'une valeur numérique pour
l'interception (arg).

+ bouger ici moral d'une pile ? artilleries d'une pile ? taille d'une pile.

+ bouger ici commandement d'une pile multinationale ? (implique de bouger
avant la section commandement pour def de ``commandement immédiat'')

\section{Friendly/Enemy}
When is a province friendly or enemy?

\section{Command}
Leading multi-national stacks.

\subsection{Leadership}\label{chMilitary:Leadership}
\subsubsection{Double-sided Leaders}\label{chMilitary:Double Sided Leaders}
\aparag Some leaders have two different sides for the same country and the
same turns but with two different roles, and can be used as either one or the
other of their roles.
\bparag The counter has a {\textetoile} written on one of the sides,
indicating in which limit the counter counts (independently of the role
for which it is used).
\bparag Kings that can also be something else do not count in the limits
as soon as they are kings.
\bparag For most countries, the role has to be determined at the
beginning of the round, effective for the whole round. \POR is an
exception (see \ruleref{chSpecific:Portugal:Explorers}).

\aparag Some leaders have two sides but no {\textetoile}.
\bparag The side they're used (and the category they're considered) has
to be determined at the beginning of each turn and is effective for the
whole turn.
\bparag In most cases, the choice is restricted because each side
denotes a change of state of the leader (e.g. change of nationality,
crowning, \ldots) and thus only one of them is available at a given
time.
\bparag Especially, generic monarchs of minor countries
(e.g. \leaderShah) are chosen at random when the country goes at war and
cannot be changed before total peace.

\aparag Beware! Some double-sided leaders do not have the same turns of
activity on each side. Thus during certain turns only one side will be
usable.

\subsubsection{Leaders of Multi-national stacks}\label{chMilitary:multi national}
\aparag[On land]
\bparag If there is a Leader with a monarch symbol (Monarch, Turkish Vizier or
heir allowed to lead troops) in the stack, he must takes command.
\bparag Otherwise, the leader with the most troop of its country (or troops he
is allowed to command) takes command (this may be a replacement leader if the
country with the most troops has no leader in the stack).
\bparag In case of tie, highest ranking tied leader takes command.
\bparag If tied again, players choice (at random in case of disagreement).
\bparag The country of the commanding leader pays for campaigns and win/loss
\STAB in case of Major Battle.

\aparag[At sea]
\bparag If there is a Monarch, he take command.
\bparag Otherwise, the highest ranking leader among those which can command at
least one \FLEET counter takes command.
\bparag Otherwise, highest ranking leader takes command.
\bparag In case of tie, players choice (at random in case of disagreement).
\bparag The country of the commanding leader pays for campaigns and win/loss
\STAB in case of Major Battle.

\subsubsection{Deployment of leaders}
\aparag[Replacement of unammed leaders]
\bparag If, during military rounds, one player falls below the minimum limit
of leaders for one category (due to death or injury) he gets as many random
\anonyme\ as necessary to reach the limit again.
\bparag The new leader arrive at the beginning of next round, in the same
place (\LeaderE and \LeaderC may also be placed in Europe or in any
\COL/\TP).
\bparag This may break the hierarchy in which case the player must try to
restore it.
\bparag When a wounded leader comes back, the lowest ranking \anonyme\ leader
of the same category is removed and the wounded may take command of any stack
without breaking the hierarchy.

\begin{exemple}
  At the beginning of turn 1, \FRA has two \LeaderG : \leader{Foix} (rank A)
  and \anonyme 2 (rank F). During the military campaign in Italy \leader{Foix}
  get ambushed by perfidious Spaniards near Napoli. He barely escaped with
  several sword wounds and must rest for many months.

  Since \FRA has now only one \LeaderG, he pick at random a \anonyme\LeaderG
  and gets \anonyme 1 (rank E).

  A few rounds later, \leader{Foix} comes back and can take command of any
  stack. Since \anonyme 2 has the lowest rank (F), he is relieved from command
  and removed from the game.

  \smallskip

  During turn 2, \POR has two \LeaderE (\leader{Dias}, if not dead during turn
  1, and \leader{Cabral}). \leader{Dias} boldly tries to circumnavigate
  America but dies, his ship crushed in the ice at Cape Horn. Since \POR still
  has 1 \LeaderE, which is larger or equal to his limit for period I, he does
  not get any \anonyme\LeaderE in replacement.
\end{exemple}

\aparag[Admirals in the \ROTW]
An admiral temporary gets the possibility to go in the \region{Atlantic} if
both the following conditions are fulfilled:
\bparag The country has no naval leader allowed in the \ROTW (either \LeaderE
or a \LeaderA with the \ROTW, '\$' or '@' capacity).
\bparag It is period V or later, or the country as at least 3 \COL/\TP in
\continent{America}.
\bparag The \LeaderA allowed to go in the \region{Atlantic} (for this turn) is
the lowest ranking \LeaderA. He may not go in seazones with a malus. If he
arrives this turn, he must be placed in Europe.

\section{Hierarchy}
Fear this!

\section{Supply}
Source of Supply, Line of Supply, Supply by naval stacks, ports, arsenals.

\subsubsection{Sources of Supply, Lines of Supply}

\aparag[Source of Supply - Land]
\bparag Source of Supply on Land: any controled city; \TP or \COL.  Exception:
neither owned nor allied: gives weak supply.  Fortresses in desert: gives full
supply in the province, only weak supply further.
\bparag Forts: are Sources of Supply on Land for \LD or \LDE only.
\bparag \Presidios are Sources of Supply only for forces inside the fortress.

\aparag[Supply by naval forces]
Naval forces may provide SoS to Land forces in coastal provinces.
\bparag \de only: can supply up to 1\LD (and 2\LDE) and blocade only fort.
\bparag \ND counters: supply up to 3\LD (without \ARMY) and blocade up to
\fortress level 1.
\bparag One \FLEET counter and at least 2 \ND in the stack: may supply up to
5\LD (including \ARMY) and blocade up to \fortress 3.
\bparag \FLEET\faceplus with at least 3\ND in the stack: may supply any stack
and blocade any \fortress.
\bparag Convoys and are never taken into account for supply and blocade.

\aparag[Source of Supply - Sea]
Arsenals are SoS for all naval forces; other ports of city, \COL or \TP are
SoS for stacks with at most one \FLEET;
\bparag Forts (not of \TP) are SoS only for stack with at most one \ND
(and possibly \NDE).
\bparag \Presidios are SoS for naval forces without \FLEET;
however, a naval force containing up to one \FLEET may enter a
\Presidio to supply it (if besieged) or bring forces.
\bparag[Stacking:] Arsenals contain any size of force;
Normal ports can have at most one \FLEET inside;
forts may contain only \ND, \NDE (no \FLEET).

\aparag[Line of Supply - Land] LoS goes from SoS to troops.
\bparag In Wasteland, any non Wasteland native country double the cost in \MP
for LoS until construction of \ville{Saint-Petersbourg} or \eventref{pVI:Great
  Northern War} (whichever occurs first). TODO: not via fleet
\bparag In non-nationnal desert, double the cost in \MP for LoS.
\bparag {\bf When supplied by naval forces} Length of LoS is 3\MP (6\MP in
Wasteland or Desert) plus 1 per sea crossed from a SoS able to supply the
naval stack.
\bparag Note that the seazone with the fleet is \textbf{not} crossed by the
LoS (only entered to turn the fleet into a SoS for the troops), hence troops
supplied by ships adjacent to a port have a LoS of length 3\MP only.
\bparag Note also that only the 3\MP of ``supply by sea'' is doubled if
required, not the extra \MP for extra seas.

\aparag[Ravitaillement des flottes et ports d'attache]
Arsenals are SoS for all naval forces; other ports of city, \COL or \TP: for
stacks with at most one \FLEET; forts: only for stack with at most one \ND
(and possibliy \NDE).

\aparag[Taille des forces navales] pour le ravitaillement terrestre, des
forteresses et le blocus.
\bparag \de seuls : ravitaille jusqu'à 1\DT (+ \LDE) et blocus ou ravitaillement de fort (f0) seulement
\bparag pions \DN : ravitaille jusqu'à 3\DT (sans \ARMY) et blocus ou ravitaillement de f0 ou f1
\bparag un pion \FLEET et au moins 2 \ND ravitaille jusqu'à 5\DT (avec \ARMY possible) et blocus ou ravitaillement jusqu'à f3
\bparag un pion \FLEET\faceplus et au moins 3 \ND (Convoy ne comptent pas): ravitaille pile
de taille quelconque et blocus ou ravitaillement jusqu'à f5
\bparag Les pions Convoys ne comptent pas.

\section{Blockade}
blockade vs being a port, a \SoS, \ldots

\Presidio and \USURE as blockade or not

\section{Campaigns}
Especially, detailled stuff about the extended campaigns\ldots

\section{Attrition}
\subsubsection{When does Attrition occur?}
\aparag[Supply Segment] (Before movement). Land stacks (only) roll for
attrition if at least one of the following case occurs. If several cases
occur, each above the first gives a \textbf{+2} malus to the roll (``Double
cause'').
\bparag No LoS ;
\bparag weak Supply, namely:
\begin{modlist}
\item LoS of 6 or more \MP (except \LD/\LDE in \ROTW)
\item LoS through non-national desert (including last province)
\item SoS not owned by alliance (only controlled)
\item Supplied by a fleet not adjacent to its own SoS (except for
  \LD/\LDE in the \ROTW).
\item Besieged (siege attition)
\end{modlist}
\bparag Force in \terme{Cold area} in an uncontrolled province after Winter
round (including in case of Summer/Summer transition and end of turn) (in the
\ROTW, add the malus of the area) ;
\bparag \Timar after Winter round (as above) (Special,
see~\ruleref{chSpecific:Turkey:Yearly Campaigning})

\aparag[Movement Segment, land] Land stacks roll for attrition at the end of
movement (before battle) if at least one of the following case occurs. If
several cases occur, each above the first gives a \textbf{+2} malus to the
roll (``Double cause'').
\bparag Large stack ($>$ 5 \LD + 1 \Pasha, or $\geq$ 3\LD if no logistic) ;
\bparag moving 6\MP or more ;
\bparag moving 3\MP or more during \terme{bad weather} ;
\bparag if embarking \textbf{or} disembarking not in friendly port.

\aparag[Attrition at sea] Naval stacks always roll for attrition except when
staying at port the whole round.

\aparag[Siege Attrition] (during Supply or Siege Segment)
\bparag Besieged during Supply Segment.
\bparag Besieger if the siege is impossible (not enough troops or no LoS) or
if requested by the siege roll.

\aparag[After battle]
\bparag On land, any non-winning troop (use specific table).
\bparag At sea, any moving stack (retreat or following to port).

\aparag[End of round (or turn)/Redeployement] In the following cases, a stack
must move and roll for attrition at the end of round or turn. Usual causes of
attrition for movement occur and cause maluses.
\bparag If no LoS during Supply Segment and still no LoS at end of round:
forced redeployement (and attrition). If no way out (naval not allowed), the
stack is destroyed.
\bparag Siege not maintained at end of turn (no Siegework\faceplus).
\bparag Fleet staying at sea at end of turn.
\bparag Fleet going to port at end of turn.
\bparag Peace evacuation

\subsubsection{Attrition results}
\aparag The effect of the result \textbf{P} in the attrition table depends on
the technology of the stack. In case of mixed stacks, take the worst technology.
\bparag Until \TARQ: 1\LD lost during movement \textbf{and} one side of
\PILLAGE in any non-neutral province entered or left.
\bparag \TMUS, \TBAR, \TMAN: either 1\LD lost during movement or both
\terme{foraging} (-1drm during 1st day of battle) and one side of \PILLAGE in
any non-neutral province entered or left.
\bparag \TL: either 1\LD lost during movement or one side of \PILLAGE in any
non-neutral province entered or left.
\bparag Besieged troops cannot pillage and thus must lose 1\LD.

\section{Movements}
\aparag[Mouvement le long d'un rivière en \ROTW]
se qualifie si un même fleuve ou lac est adjacent
aux deux provinces. Ajouter le coût de traversée du
fleuve le cas échéant.
\bparag Ne sert pas au mvt de pions A; sert pour mvt de LD, LDE
et au ravitaillement.

\subsection{Special Movements}\label{chMilitary:Movement:Special Movements}
\aparag[Provinces with several coasts]\label{chMilitary:Movement:Port Multiple
  Coasts} Movements that imply entering a port and going out of a port
may allow a naval stack to go out through a different sea zone than the
one used to enter.
\bparag It is not possible if this means to go through land (if the
province has multiple coasts as defined in \ruleref{chBasics:Multiple
  Coasts}).
\bparag This is possible only if the naval stack owns the port. A \COL
or \TP is required for a \FLEET, a fort is sufficient otherwise
(including convoys).
\bparag[Portugal] It is possible for \SPA to go through \province{Cabo
  Verde} (or any other portuguese settlement) if it has \pays{Portugal}
as a special vassal.
\bparag[Cape Horn] As a special exception, it is not possible to go out
through a different sea zone if it avoids \seazone{Horn}, unless the
naval stack ends its movement there (and goes out at the next round).

\begin{exemple}It is possible, with a \TP placed in \ville{Kyoto}, to
  enter in the same move with 1\NWD coming from \seazone{Japon} and
  going out from \seazone{Philippines}, without going through
  \seazone{Jaune} nor \seazone{Pacifique NW}. However, it is not
  possible with a \FLEET, nor if the \TP is not owned by the naval
  stack.
\end{exemple}

\aparag[Blockading with several coasts]\label{chMilitary:Movement:Blockading
  Multiple Coasts} A naval stack may blockade a port from any sea zone
adjacent to the port, unless there are multiple coasts as defined in
\ruleref{chBasics:Multiple Coasts}.
\bparag In this special case, there is a \terme{main coast} which is the
one that must be blockaded (usually where the anchor is drawn).
%\bparag Galleys cannot blockade ports that are adjacent to sea zones
%that they cannot reach (\province{Gibraltar}, \province{Tanger},
%\province{Jylland}, \province{Ostlandet}).

\aparag[Wasteland]\label{chMilitary:Movement:Wasteland} Movement in the
Wasteland area (see~\ruleref{chBasics:Wasteland}) (for all purposes,
including \LoS length computation) is doubled until the end of the
Wasteland (see~\ruleref{chSpecific:End Wasteland}).

\section{Sieges}
\subsection{Ports and terrain modifiers}
\label{chMilitary:Concepts:Siege:Terrain}
\aparag[Terrain malus.] Depending on the terrain of the province (Plain or
not), the presence of a port (and its blockade status) and whether the siege
happens in Europe (including European provinces in the \ROTW, including \COL
of level 6) or in the \ROTW, a terrain malus is applied to the roll.
\bparag This malus is \bonus{-2} if either the terrain is not Plain or there
is a non-blockaded port in the province.
\bparag It is \bonus{-3} (only) if both the conditions are true (non-Plain,
non-blockaded port).
\bparag If undermining a \ROTW province with a port (blockaded or not), the
terrain is always considered to be Plain. Hence the malus cannot be
\bonus{-3}.
\bparag If undermining a fort, the malus is only \bonus{-1} whenever
applicable.
\bparag If a blockaded port received supply during this round (this is an
active naval action), then it is considered to be non-blockaded for this round
(only).
\bparag Explicit detailed lists of all possible cases follows

\aparag[Terrain malus in Europe.] When undermining an European province,
including an European province in the \ROTW and including \COL of level 6:
\bparag if the terrain is not Plain and there is a non-blockaded or resupplied
port in the province, the malus is \bonus{-3};
\bparag if the terrain is not Plain and there is no port in the province, the
malus is \bonus{-2};
\bparag if the terrain is not Plain and there is a blockaded port in the
province, the malus is \bonus{-2};
\bparag if the terrain is Plain and there is a non-blockaded or resupplied
port in the province, the malus is \bonus{-2};
\bparag if the terrain is Plain and there is no port in the province, the
malus is \bonus{-0};
\bparag if the terrain is Plain and there is a blockaded port in the
province, the malus is \bonus{-0}.

\aparag[Terrain malus in the \ROTW.] When undermining a non-fort fortress in
the \ROTW (including cities in the \ROTW but excluding European provinces in
the \ROTW):
\bparag if there is a non-blockaded or resupplied port in the province, the
malus is \bonus{-2};
\bparag if there is a blockaded port in the province, the malus is \bonus{-0};
\bparag if there is no port in the province and the terrain is not Plain, the
malus is \bonus{-2};
\bparag if there is no port in the province and the terrain is Plain, the
malus is \bonus{-0}.
\bparag Remember that any coastal \COL or \TP automatically includes a port.

\aparag[Terrain malus for forts.] When besieging a fort (including missions or
\COL/\TP that have not been fortified further):
\bparag if there is a non-blockaded or resupplied port in the province, the
malus is \bonus{-1};
\bparag if there is a blockaded port in the province, the malus is \bonus{-0};
\bparag if there is no port in the province and the terrain is not Plain, the
malus is \bonus{-1};
\bparag if there is no port in the province and the terrain is Plain, the
malus is \bonus{-0}.

\begin{designnote}[Ports and terrain]
  Difficult terrain makes the siege harder. It is not easy to bring the
  desired amount of weapons and supplies (howitzers, guns, gunpowder,
  cannonballs, \ldots) to the siege site through mountains or
  swamps. Unblockaded ports don't make the siege more difficult \emph{per se}
  but they allow for an easier supply of the besieged garrison and population,
  thus reinforcing their fighting spirit. However, both malus are not
  cumulative as the same city can hardly be hidden in the mountain and a large
  sea port receiving supply\ldots

  \smallskip

  In Europe, even if each province only contains one city (hence one fortress)
  in game terms, this actually represents several cities spread in it. On the
  other hand, in the \ROTW, a \COL or \TP often represent a single small
  establishment. Thus, it is assumed that a coastal \COL or \TP is actually
  built really close to the sea, probably in a natural harbour, and does not
  really take advantage of the inland terrain to protect itself. For this
  reason, it is always considered to be in Plain, while taking a port in
  Europe require both to capture the actual port cities and the ones hidden
  further away in the mountain, hence both port and terrain act. Note that
  blockading a \ROTW port does not remove this, the fortress is still
  considered to be in Plain, hence there is no malus.

  Once a coastal \COL reaches level 6, it becomes an European province and
  thus its terrain matters again. The \COL is now large enough to have spread
  inland and to actually contains more than a handful of villages. On the
  other hand, \ROTW cities are still \ROTW fortresses because European control
  in these areas was usually very focused on controlling only the port and the
  cost and did not extend further (typically, the Portuguese
  \construction{Goa} did not really extend out of the actual city boundaries
  and in any cases, never controlled the whole in-game province).

  Forts are very small fortifications, hence the malus for port or terrain is
  smaller.
\end{designnote}

\section{Unsorted rules}
\begin{designnote}
  This Section consists in a bunch of unrelated rules relevant to the Military
  Phase. These rules should be properly grouped and dispatched in the correct
  place of this Chapter. This will be done when the military rules will be
  written (aka in a long time\ldots)

  Rules presented here are sometimes barely more than a summary rather than a
  proper rule written in a proper way.
\end{designnote}

\aparag[Damaged ships] \terme{Damaged} \ND are written down globally by naval zones:
Mediterranean Sea, Atlantic in Europe, Atlantic in \ROTW, Indian, Asian and East Pacific.
They are refitted for usage:
\bparag cost = 0.5*coût achat DN à un round suivant pour les remettre en état.
               Effet = remet tout de suite en jeu les DN voulus.
\bparag gratuit au début du tour suivant si on entretient la flotte;
\bparag on peut la garder \terme{Damaged} pour un coût d'entretien divisé par 2 ;
\bparag On les remet en priorité dans un Arsenal de la zone, sinon dans un
port capable de les contenir.

\subsection{Effet d'un presidio}\label{chMilitary:Presidios}

\aparag[Presidios and Blockade]
\bparag The port is considered as blockaded by this fortress, even if the
country that thus exerts the blockade is not at war with the owner of
the blockaded port.

\bparag Any exit from or entry into this port by units (privateers, Dn
or F) may trigger an reaction by the fortress. This reaction is decided
by the owner of the Presidio. This a declaration of war (with the usual
CB cost) if the interception is against any unit except privateers.

\bparag The reaction is resolved as a fire by the Presidio on the following table:
\interceptionb

\aparag[As Source of Supply]
\bparag \Presidios are Sources of Supply only for forces inside the fortress.
\bparag \Presidios are SoS Sources of Supply for naval forces without \FLEET;
however, a naval force containing up to one \FLEET may enter temporarily a
\Presidio to supply it (if besieged) or bring forces.

%%% détroits fortifiés par Jym. Modifiez à volonté.
\aparag[Strait fortifications]\label{chMilitary:Strait Fortifications}Certain
straits are marked with a red naval frontier and a tower symbol near
the province controlling them. These are the strait between Italy and
Sicily (controlled by \ville{Messine}), the entrance to
\seazone{Adriatique} (controlled by \province{Corfou}), the
Dardanelles (\province{Dardanelles}) and the Bosporus
(\province{Trakya}) in Europe; and the entrance to Saint-Laurent river
(Louisbourg, on Cape Breton Island), entrance to \seazone{Rouge}
(\province{Socotra}), entrance to \seazone{Persique} (\province{Ormus}),
the Malacca strait (\ville{Malacca}) and the Sunda strait
(\ville{Jakarta}) in the \ROTW.
\bparag In Europe, they act as a \Presidio of level 2 against any fleet
trying to cross the red lines. Using them against any unit but \corsaire
gives a free \CB to the owner of the intercepted stack for the next
turn.
\bparag If a power has a \Presidio on the \province{Dardanelles}, it
negates the effect of the Strait Fortifications for this power.
\bparag In the \ROTW, they act as a \Presidio of level half the level of
the fortress in the province (rounded down). Using them against any unit
but \corsaire give a free \CB (normal or oversea, offended player's
choice)) to the owner of the intercepted stack.
\bparag For the Sunda strait, the city of \ville{Jakarta} must also by
owned, usually by placing a \COL there.
\bparag Minor countries (usually \pays{venise} in Europe and
\pays{Gujerat} for Malacca (sometimes \pays{Chine})) will always use
them against power at war with them. If they are at peace, their
controller chose whether to use it or not. If they are neutral, they
will always use them against \corsaire and never against other naval
units.



\subsection{Terrains}

\aparag[Effet du terrain sur mouvement et combats]
\bparag Plaine: 1 PM si ami, 2 PM sinon (2 et 4 si hors-Europe) ;
\bparag Accidenté en Europe : 2 PM, sauf 3 PM en Montagne ennemi ;
\bparag Accident en ROTW: 4 PM si ou mvt de forces d'un pays mineur de
ROTW ; 6 pm si ennemi ;
\bparag Rivière, passe, détroit, arrivée ou départ en marais: +1 PM (et
+2 PM HE)
\bparag Déplacement naval: 3 PM (indépendamment du terrain de départ ou
d'arrivée, y compris marais), sauf si de port ami à port ami 2 PM. 6/3
PM en rotw.

\aparag[Les différentes zones de forêts]
\bparag  forêts nordiques : suède+Finlande+côte baltique actuellement orientale
\bparag forêts orientales : celles actuelles (sauf dessus) et Prussia et
adjacent, Lovonie, Podolie.


\aparag[Effet sur le combat]\label{chMilitary:Battle:Forests} REVOIR : tables à jour
\bparag  Modificateurs feu et choc \\
	en marais, forêt ou désert -1 \\
	en montagne pour l'attaquant (sauf s'il a intercepté) -1 \\
	force traversant un fleuve ou une passe de montagne -1 \\
		(1er round, et sauf si il a intercepté)
\bparag Modificateurs feu \\
	force débarque ou traverse un détroit -2 [1er round]
\bparag Modificateurs choc \\
	force débarque ou traverse un détroit -3 [1er round]
\bparag Modificateurs poursuite\\
	en marais, forêt, désert ou montagne -1 \\
	vainqueur a traversé fleuve, passe, ou détroit ou débarque -1 \\
	retraite du perdant à travers passe, fleuve, détroit ou réembarquement +1

\aparag Si plusieurs piles se rejoignent dans une même province pour
une bataille (2 forces qui convergent ou interception), on prend le plus
défavorable effet de terrain de frontière.

\aparag[Finlande-Suède] Un mouvement de retraite (après bataille ou
redéploiment forcé) est autorisé entre les provinces au
nord de ces deux zones. Le mouvement prend toute la capacité de mouvement
restante (donc il faut faire un test d'attrition car les 12 MP sont dépensés).
C'est la seule forme de mouvement autorisée par ce chemin.

\subsection{Occupations [BLP]}
\aparag Occupation markers are in limited amount.

\subsubsection{Placement of occupations}
\aparag Occupations markers may be placed instead of Controls in any
of the following cases.
\bparag Any country may place occupations in \continentCaraibes.
\bparag[] [TBD] Any country may place occupation on \TP of a major or
former major country.
\bparag \POR may place occupations on \paysOman and \paysAden.
\bparag \RUS may place occupations on any province adjacent to its
national territory.
\bparag \HIS may place occupations on Dutch provinces
during~\ref{pIII:Dutch Revolt}. Similarly, \HOL may place occupation
on Spanish provinces that were part of \paysBourgogne during the same
Event.
\bparag \TUR, \AUS and \POL may place occupations in any province that
was part of \paysHongrie, after~\ref{pI:Fall Hungary}.
\bparag[] [TBD] \VEN may place occupations in \regionItalie if
\emph{Itali e San Marco} has been declared.
\bparag[] [TBD] Before~\ref{pII:Act Supremacy}, \ANG may place
occupations in former territory of the 100 years war:
\provinceGuyenne, \provinceQuercy, \provincePoitou, \provincePicardie.
\bparag \FRA may place occupations in provinces of the HRE that are
adjacent to owned territory.
\bparag \paysSavoie may place occupation in any province with its
shield (blurred or not).

\aparag Contrary to Controls, occupations are not removed when peace
is signed and may stay in place.
\bparag However, the peace treaty may include removal of certain
occupation. This does not change the rest of the peace (\emph{i.e.} it
is purely an agreement between players and do not change the level of
the peace or the number of conditions exchanged).

\aparag Occupations may be voluntarily removed by a diplomatic
declaration. In that case, the control of the province is immediately
given back to its owner.

\aparag If a country annex an occupied province, the occupation is
removed.

\aparag As control markers, if an occupied province is besieged and
taken by another power, the occupation is removed.

\aparag In order to maintain occupation, a country must keep at least
1\LD in each occupied province. If, at any moment, this garrison is
not present, immediately remove the occupation marker and
\bparag If the occupant and the owner are at peace, return
control of the province to its rightful owner;
\bparag If they are at war, replace the occupation by a control of the
same country.

\subsubsection{Effect of occupation}
\aparag In addition to giving control of the province or
establishment, occupations also give income (including exploited
resources) to the occupant rather than the owner.

\aparag Each country with an occupied owned province has a free \CB
(\OCB if this is a \COL or \TP) against the occupant.

\aparag The province is still owned by the rightful owner for all \VPs
purpose (especially for period objectives).

\aparag The occupant must maintain a garrison of at least 1\LD.
\bparag If there are less troops and the occupant and occupied are not
at war (interventions do not count), remove the occupation and control
of the province is immediately returned to its owner. Any remaining
occupant troops (\de in the \ROTW) are immediately destroyed.
\bparag If there are less troops and the two countries are at war,
immediately replace the occupation by a control.

\aparag The Portuguese occupations also enforce an \dipAT with
\paysOman or \paysAden.

\subsection{En ROTW}

\aparag[Indigènes et combat]
Ils attaquent des forces à chaque round normalement ; en cas
de défaite avec déroute, ils n'attaquent pas ce joueur au
round suivant, seulement celui d'après. En cas de victoire
ou de défaite normale, ils attaquent dès le round suivant.
Ils font le siège des forts/forteresses, mais jamais d'assaut
[l'assaut est représenté en fin de round par l'attaque des indigènes]

\aparag Une ville dans une région qui n'est à aucun pays mineur peut
être attaquée sans déclaration préalable de guerre. La déclaration
doit se faire à la phase des combats, les indigènes de la zone
forment l'armée qui défend la ville et le pays européen peut
ensuite, si il les défait, mettre le siège ou faire l'assaut.
On ne peut installer une COL dans une telle zone qu'en ayant
pris la ville au tour d'avant.

\aparag Un TP établi ou une Mission a un fort. En revanche, une présence militaire
autre qu'une forteresse peut causer l'activation des indigènes.

\aparag Une COL n'a pas cet avantage (mais on peut en construire) ;
cependant dans une COL établie, la présence de forces armées
n'entraine plus de réaction des indigènes, mais
seulement d'un pays mineur ayant la région, ou lors des
résultats E* à une colonisation.

\aparag[Pillages]
\bparag Sans A, les pillages en \ROTW sont au plus \Facemoins.
\bparag L'or à terre est capturé à moitié si pillage \Facemoins et en
entier si pillage \Faceplus.

% Local Variables:
% fill-column: 78
% coding: utf-8-unix
% mode-require-final-newline: t
% mode: flyspell
% ispell-local-dictionary: "british"
% End:
