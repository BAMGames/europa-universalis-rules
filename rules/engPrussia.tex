\sectionJ{\anchorpaysmajeur{Prusse}}{\blasonJ{prusse}}\label{chSpecific:Prussia}
\subsection{From \sectionpays{Brandebourg} to \sectionpaysmajeur{Prusse}}
\subsubsection{\pays{brandebourg} as a Minor Country}
\aparag Before the transfer from \paysmajeur{Pologne}, this country is
named \pays{brandebourg}, name of the \HRE Electorate it was in 1492. Its armies
are normal occidental armies, of class \CAIII.
\aparag \pays{brandebourg} (or the \region{Duche de Prusse}, see below)
grows through the following events:
\bparag \eventref{pI:Fall Teutonic} gives \province{Preussen} to
\pays{brandebourg} if \POL is not \terme{Catholic}, else it joins the
\region{Duche de Prusse}.
\bparag \eventref{pIII:Northern Secularisation} adds \province{Memel} to
\region{Duche de Prusse}.
\bparag \eventref{pIV:TYW} may add \province{Ost Pommern} to
\pays{brandebourg}.
\bparag \eventref{pIV:Great Elector} changes the basic forces of
\pays{brandebourg} and gives it a claim to \region{Duche de
  Prusse}. \POL may cede these provinces specially.
\bparag \eventref{pV:Kingdom Prussia} changes again the basic forces of
\pays{brandebourg}, annexes the \region{Duche de Prusse} and
\province{Berg}. It may become a kingdom.

\aparag[\sectionregion{Duche de Prusse}] This is the name of the
belongings of the Elector of Brandenburg that were under the Polish
crown authority until after the Thirty Years War. It can be ceded by
\POL following \eventref{pIV:Great Elector} advantageously and must be
ceded during \eventref{pV:Kingdom Prussia}.

\subsubsection{\sectionpaysmajeur{Prusse} as a Major Country}
\aparag See~\ruleref{chSpecific:Campaign:Transfer Poland} for the conditions of
the transfer from \paysmajeur{Pologne}.
\bparag The events \eventref{pVI:WoPS} or \eventref{pVI:WoAS}
trigger the change to \paysmajeur{Prusse} if in period VI. 
If none of those happen, the transfer happens at the beginning of turn 51.
\aparag \PRU has no \CTZ. 
\aparag If an event (of a previous period) makes \pays{brandebourg}
declare war, the declaration is transformed into a mandatory \CB against
the country that should have been the subject of the declaration of
war. The \CB can be refused at the cost of 3\STAB.
\subsubsection{Silesia: \sectionprovince{Silesie}
  and \sectionprovince{Lausitz}}
\aparag \PRU wins immediately a special \MNU\facemoins at the first
event phase where the two provinces \province{Silesie} and
\province{Lausitz} are in its possession, to be placed in one of these
two provinces.
\bparag This \MNU will be lost if the provinces are lost. It can then be
rebuilt by an administrative action if it regains at least one.
\bparag This \MNU can be raised to \Faceplus normally.
\subsubsection{Military Means}\label{chSpecific:Prussia:Military Means}
\aparag Troops bought under the recruitment limit by \PRU are directly
\terme{Veteran}. Its armies are of class \CAIV.
\aparag \PRU can proceed to exceptional levies
(see~\ruleref{chExpenses:Exceptional Levies}) with no loss of \STAB,
or with a loss in \STAB after a normal (not major) defeat in a land battle.
\aparag \PRU has each turn a free multiple campaign. It is upgraded to
two free multiple campaigns under \monarque{Friedrich II}.

%\subsection{Avantages militaires}
% La technologie militaire de la Prusse avance automatiquement de 3
% cases par tour sous les conditions suivantes~:
% \begin{itemize}
% \item le souverain a 9 en militaire (exemple : Frederic-Guillaume I ou
%   Frederic II) ;
% \item un evenement parmi VI-1, VI-4, VI-9 ou VI-11 est en cours (tire
%   a ce tour ou avant et la guerre induite n'est pas finie pour la
%   Prusse) ;
% \item cette augmentation se fait apres les ajustements des autres pays
%   majeurs par action, en m^eme temps que celle des pays mineurs (et
%   avant le replacement des palliers) ;
% \item cette augmentation ne peut pas amener la Prusse au-dela de une
%   case devant la Guerre en Dentelles ; des que ce point est atteint
%   (ou quand la date de la GD n'est pas atteinte et que le pion de
%   technologie terrestre de la Prusse est bloquee), l'augmentation ne
%   se fait pas a tour en cours ;
% \item cette augmentation, quand elle a lieu, est gratuite et remplace
%   l'action administrative d'augmentation de la technologie de la
%   Prusse (qui ne peut en faire d'autre).
% \end{itemize}


\subsection{\sectionpaysmajeur{Prusse} in play}
\subsubsection{Prussian Monarchs}
\begin{histoire}
  Prussia, a country with few resources, managed to carve out its
  greatness thanks to the energy of a few bright sovereigns. First,
  there was the Great Elector Frederick-William, that managed to pull
  his territories out of the Thirty Years War in a good state. Then
  Frederick~I that obtained the royal dignity and then
  Frederick-William~I, the Soldier-King, that built a modern army for
  Prussia. And finally Frederick~II, that led the country to brilliant
  victories but also to the verge of destruction.
\end{histoire}
\aparag[\anchormonarque{Friedrich-Wilhelm}.] If \PRU becomes a \MAJ,
before turn 51, the monarch is \monarque{Friedrich-Wilhelm}, with values
8/5/9. He is scheduled to survive until the beginning of turn 51. He is
not a general (the \leader{Friedrich-Wilhelm} general is the Great
Elector).
\aparag[\anchormonarque{Friedrich II}.] At the end of the reign of
\longmonarque{Friedrich-Wilhelm} (usually beginning of turn 51), or
immediately if the event \eventref{pVII:Seven Years War} is rolled for
and activated, \monarque{Friedrich II} takes the throne of
\paysmajeur{Prusse}. He has values 9/9/9, is a general
\leaderwithdata{Friedrich II}. He is scheduled to last 9 turns, and does
not roll for survival for the first 6 turns.
\bparag He makes survival tests at the end of battles normally (no {\bf
  -1} due to his 6 characteristics).
\bparag[The hay stack escape] The first time \leader{Friedrich II}
should die in battle, he escapes unharmed.
\bparag During his Reign, \paysmajeur{Prusse} may break any Alliance for
the cost of 1 \STAB (instead of the usual 2).

\subsubsection{Available counters}
\aparag[Military] 4\ARMY, 5\FLEET, 2\LDND, 8\LD, 2\NTD, 3\LDENDE, 2
fortresses 1/2, 4 fortresses 2/3, 4 fortresses 3/4, 1 fortresses 4/5, 2
forts.
\aparag[Economical] 2\COL, 2\TP, 5\MNU, 1 special \MNU, 2\TradeFLEET
counters.

% LocalWords:  Prusse pVII Pologne brandebourg Duche de Preussen pI pIII HRE pV
% LocalWords:  pIV Ost Pommern Memel pVI WoPS WoAS Silesie Lausitz
