% Copyright 2010 by Jean-Christophe Dubacq 
%           2007 by Till Tantau
%
% This file may be distributed and/or modified
%
% 1. under the LaTeX Project Public License and/or
% 2. under the GNU Public License.
%
% See the file doc/generic/pgf/licenses/LICENSE for more details.

\usetikzlibrary{fit}
\pgfkeys{/tikz/xsize/.code=\tikz@lib@xsize{#1}}
\tikzoption{xsize offset}{\pgfmathsetlength\tikz@lib@xsize@length{#1}}
\tikzoption{xsize scale}{\def\tikz@lib@xsize@scale{#1}}
\newdimen\tikz@lib@xsize@length
\tikz@lib@xsize@length=0pt
\def\tikz@lib@xsize@scale{1}

\def\tikz@lib@xsize#1{%
  \pgf@xb=-16000pt\relax%
  \pgf@xa=16000pt\relax%
  \pgf@yb=-16000pt\relax%
  \pgf@ya=16000pt\relax%
  % Now iterate over the coordinates
  \tikz@lib@fit@scan#1\pgf@stop%
  % Now, let's see what has happend
  \ifdim\pgf@xa>\pgf@xa%
    % Nothing... Ok, let's just ignore this.
  \else%
    % Ok, compute center and width and height
    \pgf@x=\pgf@xb%
    \advance\pgf@x by-\pgf@xa%
    \advance\pgf@x by \tikz@lib@xsize@length%
    \pgf@x=\tikz@lib@xsize@scale\pgf@x
    \pgfkeysalso{/tikz/text width/.expanded=\the\pgf@x}%
    \pgfkeysalso{every fit/.try}%
  \fi%
}

\endinput
