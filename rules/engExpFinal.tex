% -*- mode: LaTeX; -*-

\section{Administrative expenses}
\aparag Write in \lignebudget{Adm. total} the sum of \lignebudget{Loan
  interests}, \lignebudget{Mandatory loan refund} and all lines between
\lignebudget{Optional loan refunds} and \lignebudget{Other expenses}
included.

\section{Exotic resources price variation, Trade centres and convoys}
\label{chAdministration:Variation}



\subsection{Price of exotic resources}
\label{chAdministration:ExoticResourcesPrices}

\subsubsection{Exploitation level}
\aparag[Exploited quantities.] The ``exploited quantities'' markers must
always be adjusted to their correct values (in practice, it is usually
sufficient to do it at the end of each administrative and peace phases).
\bparag The following things may require an adjustment of the counters: new
colonies or trading posts, competition, burned trading posts, peace conditions
implying the assignment of resources exploitation to other trading-posts,
events that may have changed the number and level of trading-posts or
colonies, and any other event that may change the exploitation of exotic
resources.
\bparag Minor \ROTW countries do exploit resources where they have a \TP or a
\COL. They do not exploit in other provinces (typically in provinces they own
but without an establishment).
% (Jym) ???
% \bparag The production of Silk, Products of Orient and Spices is always
% considered to be at least 1/10\th of the income of the \emph{Great Orient}
% trade centre; this is for the exotic resources prices variations roll only,
% not for any other consideration.
\bparag Even if they do not appear at the same time, the \RES{Sugar} in
\continent{Bresil} and elsewhere is still considered to be the same
resource. Same thing goes for the \RES{Cotton} in \continent{America} and
\continent{Asia}.

\aparag[Bookkeeping quantities.] Each resource has an \terme{exploited
  quantity} marker to denote on the Exotic resource tracks the amount which is
exploited.
\bparag There are 3 Exotic resource tracks, depending on whether the maximum
exploitable quantity is 20 (\RES{Product of America}, \RES{Cotton}, \RES{Salt}
or \RES{Silk}), 30 (\RES{Fish}) or 40 (\RES{Products of Orient}, \RES{Spice},
\RES{Fur}, \RES{Sugar} or \RES{Slaves}). There are reminders near them to
recall which track to use for which resource.
\bparag There is also an \terme{Exotic resources sheet} with one box per
exploitable resource. This is a global sheet and whenever a player exploits
(or stops to exploit) a resource, he should note it on this sheet. This allows
for easier counting of the exploited quantities and avoid some errors where
players accidentally exploit the same resource twice.

\aparag Depending on the amount of resource exploited and the maximum amount
exploitable for this resource (20, 30 or 40), each resource has an
exploitation level which is either \terme{rare} (green), \terme{low} (blue),
\terme{medium} (white), \terme{large} (purple), \terme{high} (yellow) or
\terme{excess} (red).
\bparag These levels are colour-coded on the tracks for easy reference.
\bparag The colour of the box in which each \terme{exploited quantity} marker
is indicates the exploitation level for this resource.


\subsubsection{Variation of price}
\aparag[Economic situation.]
\bparag The economic situation die roll gives an \terme{economic situation}
on~\ref{table:Random Piracy and Economy Roll}. The economic situation can be
one of \terme{Crisis}, \terme{Normal} or \terme{Boom}.
\bparag Remember that this roll is done during the Event phase as it also
controls apparition of piracy. See~\ref{chEvents:Economic situation}. The
situation is recorded on the \ROTW map.
\bparag Crossing the economic situation with the production level of each
resource gives a basic variation of the prices, from -2 to +2 boxes (see
\tableref{table:Exotic resources variations}).
\bparag \tableref{table:Exotic resources variations} is recalled on the \ROTW
map, near the exotic resources tracks, using the same colour-code as them for
exploitation levels.

\GTtable{ecopiracyprices}

\aparag On each price marker of each resource, there are two numbers: left is
for the low price threshold, right is for high price threshold.

\aparag[Market variation.] Roll a die for each resource, and add it to the
current price of the resource. If the result is less or equal to the low
threshold, then the market variation will be +1. If the result is greater or
equal to the high threshold, the market variation will be -1. Otherwise, the
market variation is 0.
\bparag The combination of the basic variation and of the market variation
gives the number of boxes the price marker will move to the right (positive
value) or to the left (negative value).
\bparag The price variation cannot exceed 2 boxes to the left or to the
right. If the two variations sum up to +3 or -3, use +2 or -2 instead.

\aparag There is a minimal value and maximal value for all resources. Under no
circumstances shall the price marker go out of those bounds.
\bparag The price marker stops when reaching the maximum or minimum value.
\bparag These values are written in the exotic resources prices track, on the
top line for the minimal price and on the bottom line for the maximum price.

\aparag[Speculation]\label{chAdministration:Speculation}
A \MAJ may speculate on a product to have a better chance of increase of the
price of this product. He must announce it before the price adjustment, and he
will gain only half of his normal income for this resource.
\bparag Speculation on price must be announced during the Diplomatic phase, as
a diplomatic announcement.
\bparag If he has either a total or partial monopoly
(see~\ref{chThePowers:MonopolyExoticResources}), he puts the production
counter upside-down. The resource is considered to have a \terme{rare}
production level (green) for the computation of its basic variation.
\bparag If he has no monopoly but at least 4,6 or 8 units exploited (according
to the maximal production of the product that could be 20,30 or 40), the
counter is put left-to-right to show that a \bonus{-1} will be added to the
roll for market regulation of this resource.

\aparag Price for all resources is computed independently, but there is only
one economic situation roll, common to all resources.
\bparag Price variation comes from the global market dynamism, and individual
adjustments depending on threshold prices.

\begin{exemple}[Prices of exotic resources]
  The production of \RES{Spice} is 32, current price is 7, and the production
  of \RES{Salt} is 3, current price is 6 (rightmost box). The die-roll of
  economic situation gives a 10: a Boom. First thing, the inflation marker is
  moved one box to the right. The thresholds for \RES{Spice} are 8 and 16, the
  thresholds for \RES{Salt} are 7 and 15.

  The basic variation of the \RES{Spice} price is +1 (since there already is
  high production). A dice is rolled for a result of 9. Adding the price (7)
  to it yields a total of 16, larger than the high threshold, thus giving a
  market variation of -1. Therefore, the price variation of \RES{Spice} is
  +1-1=0, the price does not change at all.

  The basic variation of the \RES{Salt} price is +2, because there is only a
  rare production. Adding a die-roll of 1 to the price (6) yields a result of
  7, smaller than the low threshold, thus giving a market variation of +1.
  The total variation is thus +3, which is capped at +2. The price marker
  should be moved 2 boxes to the right. However, the maximum price for
  \RES{Salt} is the leftmost 7 box, so the marker stops there.
\end{exemple}



\subsection{Attribution of centres of trade and convoys [to move in Interphase]}

\aparag Convoys represent heavy trade of specific resources (usually
gold). They are given to the country dominating trade in this resource
(usually exploiting most of it). They do not bring income per se but must be
brought back to Europe where the gold they carry can be unloaded. However, the
journey can be dangerous and convoys can be attacked and seized by pirates,
privateers or enemy fleet.

\aparag There are four possible convoys: the \terme{Levant} fleet of
\bazar{Izmir}, the \terme{East Indies} convoy, the \terme{Flota de Oro}, and
the \terme{Flota del Per\'u}.


\subsubsection{The convoys}\label{chIncomes:Convoys}
\aparag The convoys are represented as a special \ND, and may be moved during
the military phase (see \ruleref{chMilitary:Convoys})
\bparag They represent a certain number of \NTD, each carrying 15\ducats that
is credited only upon arrival in a port during the military or redeployment
phases, to the owner of the port (or its Patron, if it is a minor vassal).
\bparag The sum is reported in \lignebudget{Gold from ROTW and Convoys}.
\bparag They can be attacked by either enemy fleets, pirates or privateers
(see \ruleref{chMilitary:Convoys}).

\aparag[Levant Convoy]\label{chIncomes:Levant Convoy} This
convoy contains 2 to 4 \NTD, each carrying an income of 15\ducats.  It appears
every turn in \province{Izmir}, at the administrative phase, if the \CCs{Grand
  Orient} is owned by \TUR.
% TODO: variable number of \NTD (2 to 4) ??
\bparag If \province{Izmir} does no more belong to \TUR, any Turkish port on
\region{Mediterranee} will do instead.
\bparag The actual content of the Convoy depends on the current income of the
\CCs{Grand Orient}: 4\NTD if 100\ducats or more, 3 \NTD if 50\ducats or more
and 2 \NTD if less than 50\ducats.
\bparag The convoy is attributed by \TUR to any other player of his choice,
provided that this player either has a port in \region{Mediterranee} (owned or
vassal) or controls the \CCs{Mediterranee}, and it accepts the convoy. This
attribution is done as a diplomatic announcement (and thus can be part of a
larger agreement such as ``buying'' the convoy).
\bparag If \TUR attributes the convoy to another player, it receives 20\ducats
as soon as the convoy reaches the other player (in \lignebudget{Gold from ROTW
  and Convoys}).
\bparag The Turkish player may refuse to grant the convoy or possibly no one
accepts it. In such a case, \TUR immediately loses 20\ducats from its \RT and
1 \STAB.
\bparag If the convoy is not attributed to the player that controls the
\CCs{Mediterranee}, this player receives an immediate and temporary \CB
against \TUR (overseas or normal) at this turn only.
\bparag The convoy can be escorted by \TUR or by the receiving \MAJ. \TUR
decides when the convoy leaves the port (and thence, who escorts it).
\bparag Income of the convoy is credited in the player's treasury immediately
upon arrival in any home port or vassal port of that player in Europe,
i.e. during the Military or Redeployment phase (and not the Logistics
sub-phase).
\bparag Any player at war with either \TUR or the receiver of the convoy
(except \TUR and the receiver) may try to intercept the convoy. The escorting
force may always participate in the defence of the convoy, whatever the status
of war between the interceptor and the escort
(see~\ruleref{chMilitary:Convoys}).
\bparag If the convoy never arrives at its final destination (in a port of the
designed \MAJ), the penalty for not attributing the convoy is applied
immediately to \TUR.

\aparag[East Indies convoy] It is attributed to the player that does exploit
the most of the following resources: Silk, Products of Orient, Spices, and at
least 10 of them.
\bparag In case of ties, the convoy is attributed, among the tied countries,
in the first one in the following order: the country controlling
the \emph{Atlantic} Trade Centre ; \HOL (before 1700, turn 42); \ANG (after
1700, turn 43). If the tie cannot be broken that way, the convoy is not
attributed.
\bparag The East Indies convoy must start from a \COL\faceplus or a
\TP\faceplus in \continent{Asia}, and must reach a port of its owner on the
Europe map. It is worth 4 \NTD carrying 15\ducats each.
\bparag The gold obtained when reaching a port on the Europe map with a convoy
is accounted for in \lignebudget{Gold from ROTW and Convoys} of the receiving
player (either the original owner or someone else seizing it during the trip).

\aparag[Spanish convoys]\label{chIncomes:SpanishConvoys} The Spanish gold
fleet can only transport gold stored in Spanish ports. The \terme{Flota de
  Oro} and the \terme{Flota del Per\'u} are each equivalent to 5 \NTD.
\bparag The \terme{Flota de Oro} appears in a port on the Atlantic coast, the
\terme{Flota del Per\'u} appears in a port on the Pacific coast.
\bparag Spanish convoys can load and unload gold in ports, thus transporting
gold from port to port, and finally to Europe (see
\ruleref{chMilitary:FlotaDeOroMovement}). They are initially empty, as opposed
to the other convoys.



% Local Variables:
% fill-column: 78
% coding: utf-8-unix
% mode-require-final-newline: t
% mode: flyspell
% ispell-local-dictionary: "british"
% End:

% LocalWords: Carrack Tercios Galleass Boyars