% -*- mode: LaTeX; -*-
\chapter{Political Events of Period VI}
%\section{Period VI}
\label{events:pVI}



\subsection*{Event Table of Period VI}

\begin{eventstable}[Period VI events table]
  \tabcolsep=5pt\centering%
  \begin{tabular}{|l|*{5}{c}|l|}
    \hline
    1\up{st}\textarrow& 1-4 & 5-6 & 7 & 8 & 9 & 10 \\ \hline
    1 & 1  & 4  & 4  & R16 & 3   & \textbullet~1--2 \\
    2 & 2  & 5  & 18 & 17  & 4   &  +1 then \\
    3 & 3  & 9  & R1 & 18  & 5   & \nameref{events:pV}\\
    4 & 6  & 10 & 2  & 19  & 6   & \textbullet~3--10: \\
    5 & 7  & 15 & 11 & 8   & R7  & \nameref{events:pV}\\
    6 & 8  & 16 & R12& R11 & 15  & \\
    7 & 11 & 17 & 13 & 12  & 9   & \\
    8 & 12 & 1  & 14 & R13 & R10 & \\
    9 & 13 & R2 & 7  & 1   & R18 & \\ \hline
    10 & \multicolumn{6}{l|}{1--6 \nameref{events:pVII}, 7--10 \nameref{events:pV}} \\ \hline
  \end{tabular}
\end{eventstable}

\eventssummary{%
  pVI:Great Northern War|,%
  pVI:WoSS|O{pV:WoSS},%
  pVI:Kingdom Prussia|O{pV:Kingdom Prussia},%
  pVI:Jacobite Rebellion|S{pVI:Jacobite:First Revolt}/%
  S{pVI:Jacobite:Bonny Prince Charlie},%
  pVI:Act Establishment|,%
  pVI:Vassalisation Hanover|,%
  pVI:Treaty Methuen|,%
  pVI:Act Union|,%
  pVI:Bill Test|,%
  pVI:Heinsius|,%
  pVI:WoPS|,%
} \eventssummary{%
  pVI:War Turkey|E/E,%
  pVI:WoAS|,%
  pVI:Kurland|,%
  pVI:Slave Revolts WI|E/E,%
  pVI:Bantu Raids|E/E,%
  pVI:Last Great Mughals|,%
  pVI:Wars India|T{A}/T{B}/T{C},%
  pVI:Mazepa|,%
  pVI:WoJE|,%
  pVI:Comuneros|,%
  pVI:WoQA|,%
  pVI:Alberoni|,%
  pVI:Bulavin Rebellion|,%
  pVI:Africa|E/E,%
  pVI:Camisards|,%
  pVI:Barbaresques|,%
}

\newpage\startevents



\event{pVI:Great Northern War}{VI-1}{The Great Northern War}{1}{PBNew}

\history{1700-1721}
\dure{until the end of the war caused by the event.}
% (Jym) All \CB switched from \phevnt -> \phdipl. It looks cleaner to manage
% reactions at diplomacy time due to all \CB... the only difference is
% preventing the signing of alliances, but this does not really matter here
% since there are two turns to use the \CB... (JCD) I am not sure to follow
% this line of reasoning, but in fact, I don't think this matters much.

\phdipl
\aparag[Russian aggression of \SUE] \RUS has a free \CB against \SUE if they
have a common frontier.
\bparag This \CB can be used at this turn or the next one.
\bparag If \RUS does not use this \CB, it loses 2 \STAB at the end of the
diplomacy phase of the next turn. This becomes a loss of 3 \STAB during and
after the reign of \monarque{Peter the Great}.
\aparag[Polish aggression of \SUE] \POLpol has a normal \CB against \SUE if
they have a common frontier.
\bparag This \CB can be used at this turn or the next one.
\bparag \POLpol is affected by \xnameref{pVI:GNW:Polish Civil War}).
\bparag If \POL does not use this \CB, it loses 2 \STAB at the end of the
diplomacy phase of the next turn. This becomes a loss of 3 \STAB if either
\ref{pIV:Liberum Veto} never happened, or Absolutism has been established
(\ref{chSpecific:Poland:Absolutism}) or the dynasty of \payssaxe currently
rules \POL per \ref{pV:Saxon King Poland}.
\bparag This \CB can be used as a reaction to the \CB of \RUS above, or as a
regular \CB.
\bparag If there is a \POLmin (special or normal), apply \ref{pVI:GNW:Minor
  Poland}.
\aparag[Forfeit] If neither \RUS nor \POL use their \CB by the end of next
turn, consider the event played and \SUE is considered to have won the war for
all purposes (especially for the lasting effects).
\bparag If either \RUS or \POL are already at war against \SUE, either can
declare that they transform the war into this event. This is considered as
using the \CB provided by the event (with no \STAB cost in the case of \POL)
and triggers everything triggered by the use of the \CB.
\aparag[Swedish generalisation of the war] If one of \RUS or \POL uses its \CB
to declare war on \SUE, then \SUE has a free \CB against the other one.
\bparag This \CB is used as a reaction to the \CB of \RUS or \POL.
\bparag[Surprise aggression] As an exception, this \CB can be used at the
beginning of any military round of any turn of the war. In this case, the
country enters war without a call for allies.
\aparag[Prussian involvement] If \PRU is a major country, it has a \CB against
either \POL or \SUE (its choice).
\bparag This \CB can be used at the turn of the event or at the next
one. There is no penalty for not using it.
\aparag[Danish aggression] \paysDanemark may enter the war against \SUE (see
\xnameref{pVI:GNW:War Denmark}).
\aparag[Alliances] \RUS, \POL, \PRU or \SUE are not necessarily allied in the
war. They have to sign a formal alliance if they want to be allied.

\tour{After the war begins}

\phadm
\aparag At the first turn of the war (only), \SUE receive reinforcements as a
minor country. It makes one roll in offensive attitude \textbf{and} one in
defensive attitude.
\bparag These reinforcements are \Veteran. They do not count toward this turn
purchase limit.

\phmil
\aparag If the dynasty of \payssaxe rules in \POL, troops of \POL and \SUE can
cross the \HRE in order to wage war in \payssaxe.
\bparag No side may besiege or pillage provinces of the \HRE belonging to
countries not at war.
\aparag Troops of \SUE may enter provinces of \regionUkraine even if they
belong to a country not at war (they may thus trigger \ref{pVI:Mazepa}).
\bparag This gives a free \CB against \SUE to both the owner of the province
and the protector of \paysukraine to be used during the next turn.
% (Jym) In \PPI, the \CB was against enemies of \SUE if \TUR used it. Here,
% \TUR already has a \CB because of the revolt of \leaderMazepa and it gives a
% double \CB to \TUR (against either \RUS or \SUE). Easier to write like that

% (Jym), notes Pierre, small civil war even if absolutism:
\aparag Fortress owned by \POL and controlled by \SUE gives full supply to
\SUE.

\phpaix
\aparag[Starting the Revolt of Mazepa] If there is any \ARMY counter of \SUE
in any province of \regionUkraine at the beginning of the peace phase then
\ref{pVI:Mazepa} will occur next turn. Consider it as the first event rolled
for during the next event phase.
\bparag This revolt will occur even if the peace is signed at this turn. In
this case, the revolt is considered to have occurred at the very end of the
turn, before signing the peace.
\aparag If \SUE signs no unfavourable peace for this war (including if the war
does not occur), it immediately wins 50 \VP.

\effetlong
% (Jym) Err, how about the overseas choice ?  (JCD) Does not matter, the event
% is more specific. Let the third army become a totally normal army.
\aparag If \SUE signs no unfavourable peace for this war (including if the war
does not occur), then \SUE may use up to 3 \ARMY counters in Europe with no
condition on the number of provinces and even if the politics of \ROTW
expansion was chosen earlier.

\begin{digressions}[\payspologne and \paysDanemark in the Great Northern War]


  \digression[pVI:GNW:War Denmark]{War in \paysdanemark}
  % (Jym) : was: \DAN in \EC of \RUS + entry in war.  Reformulation to avoid
  % \minmaxing...  We might need diagrams to know which cases should be
  % needed...
  \aparag If \paysdanemark is inactive:
  \bparag If \RUS declares war on \SUE, then \paysdanemark is put in \CE of
  \RUS and fully enters war against \SUE.
  \bparag If \RUS does not declares war on \SUE, but \POL does, then
  \paysdanemark is put in \CE of \POL and fully enters war against \SUE.
  \aparag If \paysdanemark is already at war against \SUE:
  \bparag If its controller is \RUS or \POL and uses its \CB, then it is
  raised in \EG of its controller.
  \bparag If it is not allied to any \MAJ, it is put in \EC of the first \MAJ
  to use its \CB against \SUE (\RUS first, then \POL).
  \bparag If its controller is \RUS or \POL and does not use its \CB, the war
  goes on but \SUE can now obtain the truce (see \ref{pVI:GNW:Danish Truce}).
  \bparag If \paysdanemark is at war against a \MAJ declaring war to \SUE, it
  immediately proposes a white peace. If another \MAJ declares war to \SUE,
  \paysdanemark is then put in \CE of this \MAJ and enters war against \SUE.
  \aparag Otherwise (\paysdanemark at war against someone not part of the
  Great Northern War), \paysdanemark does not partake to the Great Northern
  War.
  % (Jym) Now included in event \leaderMazepa.
  % \aparag If \paysukraine is at war against \RUS or if
  % \ref{pVI:Mazepa} is occurring, \TUR has a free \CB against \RUS at
  % the next diplomacy phase.
  % \bparag This \CB exists at each turn were one of the conditions is true.

  \phpaix
  \aparag\label{pVI:GNW:Danish Truce} If the capital of \paysdanemark is
  controlled by \SUE at the beginning of a peace phase, or if \paysdanemark
  loses a major defeat (on land or on sea) against \SUE (not its allies), it
  proposes a truce to \SUE.
  \bparag If \SUE accepts the truce, \SUE evacuates the capital of
  \paysdanemark but keeps other controlled provinces.
  \bparag If the peace is signed during this truce, provinces of \paysdanemark
  controlled by \SUE must be taken into account when computing peace
  differential.
  \bparag The truce lasts for 3 turns after which \paysdanemark automatically
  enters back in the war.
  % (Jym) added:
  \bparag During the truce, \paysdanemark stays on the diplomatic track of its
  patron and is still considered at war for all purposes.
  % (Jym) hum, not too sure about that. Especially because of the
  % \seazoneOresund levies


  \digression[pVI:GNW:Polish Civil War]{Polish Civil War}
  \begin{histoire}[Tumult in Poland]
    Multiples candidates losing the Polish crown when Augustus II of Saxony
    was elected in 1697 were still trying to influence the Polish
    politics. They all played a complex political game for the crown during
    this war. Even if he was military forced to abdicate at the treaty of
    Altranst\"{a}dt, Augustus was soon back in the war and got his throne
    back. Sweden did not manage to impose a lasting king, even if Stanislas
    Leszinski was elected for a short and contested reign in 1706. Stanislas
    tried to come back at the death of Augustus, this time with the help of
    France, yielding to the War of Polish Succession.
  \end{histoire}
  % (Jym) More references to IV-A for Absolutism that may also come with the
  % the National Revival in pVII (associated to event of Kosciusko).

  \condition{}
  \aparag If Absolutism has been established in \POL, ignore this sub-event.
  % (Jym) WoPS:Polish Victory establishes Absolutism... What?
  \aparag If \payspologne is a special \EW of either \FRA or \SUE per
  \ref{pVI:WoPS:Polish Victory} or a regular \MIN (without Absolutism), see
  the modifications of the Civil War in \ref{pVI:GNW:Minor Poland}.

  \phmil
  \aparag If a Swedish \ARMY first enters a province owned by \POL and no
  battle (except overrun) occurs, the fortress may surrender to \SUE.
  \bparag Roll 1d10, add the current \STAB of \POL (0 if it is a \MIN), add
  the level of the fortress. If the result is 5 or less, the fortress
  immediately surrenders to \SUE.
  % (Jym) added:
  \bparag \SUE has to stop movement in the province in order to try this
  surrender, but it occurs during its movement segment and not during the
  siege segment.
  % (Jym) added:
  \bparag Troops inside the fortress are redeployed as if \emph{Honor of war}
  had been granted. The fortress does not lose one level for being taken.
  % (Jym) I do hesitate about next point. Possibly too tough to manage with a
  % war that lasts 4 turns. But without that, I am afraid that it would be
  % too powerful to reattempt everywhere until this succeeds. In doubt, I
  % comment, since this was not there before.
  %
  % \bparag This can only occurs once per province during the war.
  %
  % (Jym) Saxony is reliable:
  \bparag Provinces of \payssaxe are not subject to automatic surrender to
  \SUE.

  \phpaix
  \aparag If, at the beginning of a peace phase, \SUE controls \villeVarsovie
  or the \STAB of \POL is 0 or lower, \SUE manages to impose its pretender as
  a king for (part of) \POL.
  \bparag If \payspologne is a \MIN, this can only occurs if \SUE controls
  \villeVarsovie.
  \bparag \SUE receives \leaderwithdata{Poniatowski2}. Remove Polish
  \leaderPoniatowski if in play. If he was not in play (even if already dead),
  he will stay with \SUE for 2 turns.
  \bparag Starting with next turn, \SUE can raise up to one \ARMY\faceplus in
  any controlled or owned national province of \POL. This \ARMY has the class,
  technology and cost of Polish troops. It does not decrease the number of
  Polish (or regular Swedish) counters available. It does not count toward
  purchase limits for \SUE nor for \POL.
  % (Jym) to settle the matter of change if the \ARMY is broken
  \bparag \SUE may not have more than 4\LD worth of ``Polish'' troops and may
  not split them. It may, however, use one \LD counter if needed.
  \bparag This is a Swedish \ARMY and can thus trigger \ref{pVI:Mazepa}.
  \aparag If at the beginning of a peace phase, \SUE controls both
  \villeVarsovie and either \villeDresden (if the \payssaxe dynasty rules
  \POL) or \villeCracovie (otherwise), \POL propose a mandatory truce to \SUE.
  \bparag If \SUE accepts the truce, it may immediately annex one province of
  \POL (\SUE chooses which).
  \bparag This truce can only be imposed once during the war.
  \bparag During the truce, \SUE keeps control of the fortresses it controls
  at the beginning of the truce.
  % (Jym) :
  \bparag However, \POL gives back to \SUE the provinces of \SUE it controls
  at the beginning of the truce.
  \bparag As long as the truce lasts, \SUE can freely cross provinces
  controlled by \POL. They count as enemy provinces for movement purpose and
  \SUE cannot stop in them or pillage them. Supply may cross these provinces.
  \bparag During the truce, \POL do not lose \STAB because of the war (as if
  in armistice).
  \bparag The truce can be broken by \POL either after 3 turns of truce or
  during a turn following a major defeat of \SUE
  \aparag\label{pVI:GNW:Stanislas Victory} If \POL signs an unfavourable
  peace after a truce was imposed (even if broken), then \SUE manage to impose
  its pretender on the throne.
  \bparag The new king of \POL is \monarqueStanislas with values 6/5/6. He
  will last as long as a random length for Minister, see \ref{eco:Excellent
    Minister}. This is a new dynasty.
  \bparag \label{pVI:GNW:Stanislas} As long as \monarqueStanislas rules, \POL
  and \SUE are in defensive alliance and \POL must answer any call for ally
  made by \SUE.


  \digression[pVI:GNW:Minor Poland]{Minor Poland}
  % (Jym) Only WoPS may impose durably the \EG of \POLmin. GNW can do that
  % only for the duration of Stanislas reign (by breaking the \EG of WoPS if
  % needed)

  \activation{These effects modify and overrules the effects of
    \ref{pVI:GNW:Polish Civil War} if \payspologne is already a minor
    country.}

  \phdipl
  \aparag If \payspologne is a regular minor country, it makes a mandatory
  white peace with all its enemies (except \SUE and allies) and uses its \CB
  to declare war on \SUE. It will call for allies as per regular rules.
  \aparag If \payspologne is a regular minor country, apply all the effects of
  \ref{pVI:GNW:Polish Civil War} except \ref{pVI:GNW:Stanislas}. Use the
  following instead: For the reign of \monarqueStanislas, \payspologne is put
  in \EG of \SUE and no diplomacy is allowed on it, after which \payspologne
  becomes a normal minor country.
  \bparag For all purposes except incomes (declarations of war, victory
  conditions, \ldots) consider that special \EG as if \payspologne were a
  \VASSAL of \SUE.
  % (Jym) taken from \PPI. Not sure this is useful...
  \bparag As an exception to the normal rules, the order of preference for
  controlling \payspologne during this war is: \PRU, \FRA, \AUS, \HOL, \ENG,
  \RUS.
  \bparag If \payspologne signs no unfavourable peace during this war, it is
  put in \EG of the country that controlled it. Otherwise, it becomes neutral.
  \aparag[\payspologne special minor of \SUE] Due to \ref{pVI:WoPS:Polish
    Victory}, any declaration of war against \SUE also includes
  \payspologne. Apply \ref{pVI:GNW:Polish Civil War} substituting \RUS for
  \SUE (including the benefits of \leaderPoniatowski and his \ARMY). \RUS can
  impose its pretenders on the throne.
  \bparag If \RUS imposes its pretender on the Polish throne, \payspologne it
  put in \EG of \RUS, with no diplomacy possible, for the reign of
  \monarqueStanislas after which \payspologne becomes a normal minor country.
  \aparag[\payspologne special minor of \FRA] Due to \ref{pVI:WoPS:Polish
    Victory}, \FRA decides whether \payspologne uses its \CB against \SUE or
  not.
  \bparag If \payspologne is at war, it is played by \FRA
  % (Jym) added:
  but \FRA does not have to enter war against \SUE (it \emph{may} choose to do
  so, using the normal \CB of \POL).
  \bparag If \SUE manages to impose its pretender, this breaks the special
  status of \payspologne. It becomes a special \EG of \SUE (as above) for the
  reign of \monarqueStanislas and after that a regular minor country.
  \bparag If \SUE does not manage to impose its pretender, \payspologne stays
  a special \EG of \FRA.

\end{digressions}



\event{pVI:WoSS}{VI-2}{The War of Spanish Succession}{1}{PBMod}

\begin{todo}
  Add possibility to gives ``compensations'' to some minors to ``buy'' them in
  the war and make them change side. Historically: Sicily for \paysSavoie and
  bid on the imperial throne for \paysBaviere.
\end{todo}

\condition{This event is the same as \ref{pV:WoSS} which happens now if it did
  not occur yet. Else, treat as \RD and mark off.}



\event{pVI:Kingdom Prussia}{VI-3}{Creation of the Kingdom of
  Prussia}{1}{RistoMod}

\condition{This event is the same as \ref{pV:Kingdom Prussia} which happens
  now if it did not occur yet. Else, treat as \RD and mark off.}



\event{pVI:Jacobite Rebellion}{VI-4}{Jacobite Rebellion}{2}{RistoMod}

\history{1715/1745-46}

\condition{}
% (JCD) Added the mark off part. Obviously, no Jacobite rebellion for Catholic
% \ANG.
\aparag If \ANG is \CATHCR or \CATHCO, roll for two \REVOLT in \ANG, then mark
off and consider as played.
\aparag This event can only happen if \paysecosse is on the diplomatic track
of \ANG or if \ANG owns at least four provinces of \paysecosse. Otherwise, do
not mark off and re-roll.
\aparag There are two rebellions with slightly different initial
conditions. Apply the rules hereafter, but read initial placement in
\xnameref{pVI:Jacobite:First Revolt} or \xnameref{pVI:Jacobite:Bonny Prince
  Charlie}.

\phdipl
\aparag The rebellion is controlled by \FRA if Catholic, otherwise by \HIS.
\aparag If \FRA is \CATHCR, it has a \CB to make a full intervention at the
side of \paysecosse.
\bparag If \FRA is Catholic, it can make a limited intervention at the side of
\paysecosse.
\bparag If \FRA is Protestant, it can make a limited intervention at the side
of \ANG.
\aparag \HOL can make a limited intervention at the side of \ANG.
\aparag \HIS can make a limited intervention at the side of \paysecosse.
\begin{todo}
  Intervention only if Alberoni is or was minister. Need to write Alberoni
  before enforcing this condition.
\end{todo}
\aparag Other countries can make foreign intervention as per normal religious
wars rules (see \ref{chDiplo:Religious Civil War}). \paysecosse is considered
to be Catholic during this war.

\phadm
\aparag Rebels roll for reinforcements in offensive attitude for the duration
of the war.
\bparag Rebels can use the counters of both \paysecosse and \paysroyalistes.
\bparag reinforcements must be put in provinces where there are already rebels
or allied troops (not just \REVOLT ). If none exist, the rebels receive no
reinforcements.

\phmil
\aparag The \REVOLT are supply sources for the rebels and limited supply
sources for their allies.

\phpaix
\aparag \ANG wins if there are no more \REVOLT and either there is no more
rebel \ARMY or the rebels and their allies have suffered one more major defeat
that \ANG this turn.
\bparag In this case, remove all rebel counters from the map.
\bparag \paysecosse get back to the diplomatic position it had before the war
on the English track.
\bparag If \FRA was fully at war, a normal peace has still to be signed.
\aparag The rebels win if the king is overthrown by the \REVOLT or if they
control \villeLondon and there is at least one \REVOLT still in play or if a
fully allied \FRA manages to impose an unconditional surrender to \ANG.
\bparag If the rebels win and were not allied to any \CATHCR country, \ANG
becomes \CATHCO.
\bparag If the rebels win and were allied to a \CATHCR country, \ANG becomes
\CATHCR.
\bparag At the beginning of the next turn, the king of \ANG dies and an
automatic \terme{Dynastic Crisis} occurs in \ANG. This overrules \ref{pVI:Act
  Establishment}.
\aparag Apply the following additional effects if \FRA was fully at war and
manages to impose an unconditional surrender to \ANG:
\bparag \ANG loses 50\VP.
\bparag Events \shortref{pIV:Act Navigation} and \shortref{pVI:Act Union} are
cancelled.
\bparag \ANG makes an enforced dynastic alliance with \FRA and must give a
\COL or \TP of its choice as a dowry.
\bparag \ANG makes an enforced offensive alliance with \FRA for two turns and
must respect it when \FRA calls it as ally.
\bparag \ANG cannot declare war to \FRA for the duration of the new king and
his successor.


\subevent[pVI:Jacobite:First Revolt]{First Jacobite Rebellion}
\history{1715}

\phevnt
\aparag If \paysecosse was allied to \ANG, remove all its troops from the map.
\bparag \paysecosse is not considered to be \VASSAL or \ANNEXION by \ANG as
long as the war lasts (for incomes or victory condition purpose).
\aparag Place a \REVOLT \facemoins in each of the following provinces:
\provinceHighlands, \provinceMoray and \provinceAlba.
%\bparag Place a \ARMY\facemoins of \paysecosse in one of the revolted
%provinces.


\subevent[pVI:Jacobite:Bonny Prince Charlie]{Bonny Prince Charlie}
\history{1745-1746}

\phevnt
\aparag If \paysecosse was allied to \ANG, remove all its troops from the map.
\bparag \paysecosse is not considered to be \VASSAL or \ANNEXION by \ANG as
long as the war lasts (for incomes or victory condition purpose).
\aparag Place a \REVOLT \facemoins in each of the following provinces:
\provinceHighlands, \provinceMoray and \provinceAlba.
\bparag Place a \ARMY\faceplus of \paysecosse and general \leader{Prince
  Charles} in one of the revolted provinces.



\event{pVI:Act Establishment}{VI-5}{Act of Establishment}{1}{Risto}

\history{1701}

\effetlong
\aparag From now on \ENG can no longer suffer dynastic crisis due to a roll on
the Monarch Reign table.
\aparag However, it can still suffer dynastic crisis due to events.



\event{pVI:Vassalisation Hanover}{VI-6}{Vassalisation of
  \payshanovre}{1}{Risto}

\history{1714}

\condition{}
\aparag Cannot occur if \ENG is not Protestant. In that case mark as played.
\aparag Cannot occur if \ref{pVI:Act Union} and \ref{pVI:Act Establishment}
have not already occurred both. In that case re-roll and do not mark off.

\phevnt
\aparag If \payshanovre is currently in a war against \ENG, it offers
immediately a white peace.
\aparag \payshanovre becomes a permanent \VASSAL of \ENG for the rest of the
game. No diplomacy is allowed on \payshanovre.

\effetlong
\ephase Revolts in \pays{hanovre} are no more automatically suppressed if
inactive.  \fphase \ANG may now use the troops of \pays{hanovre} to fight
revolts inside its territory and use its troops to fight revolts inside
\pays{hanovre}.



\event{pVI:Treaty Methuen}{VI-7}{Treaty of Methuen}{1}{RistoMod}

\history{1703}

\condition{}
% (Jym) let's be explicit about the rare cases that would lead to a double
% application of this event
\aparag This event can normally only happen once, either triggered by
\ref{pV:WoSS} (at the beginning of the war or at peace time) or by rolling for
it in the table.
\bparag If the event has already been rolled for when \ref{pV:WoSS} occurs,
then \emph{Dynastic link and alliance with Portugal} is not at stake in the
war, except if \HIS managed to re-annex \paysportugal after the event.
\bparag If \emph{Dynastic link and alliance with Portugal} was chosen by a
\MAJ during \ref{pV:WoSS}, then consider the event as already played, mark off
and play \RD instead as per normal rules.
\aparag If this event was triggered by \ref{pV:WoSS}, apply
\xnameref{pVI:Methuen:WoSS}, else apply \xnameref{pVI:Methuen:Normal}


\subevent[pVI:Methuen:Normal]{Treaty of Methuen}
\history{1703}

\condition{}
\aparag If \paysportugal is annexed by \HIS as per \ref{pIII:POR Ann.:Portugal
  Annexed}, play \ref{pIV:Portuguese Revolt} in addition to this event (even
if \numberref{pIV:Portuguese Revolt} already occurred and was won by \HIS).
% (Jym) Choice of \PORmin if at war breaking its alliance vs event waiting the
% peace, I chose the former because it's funnier.
\aparag If \ENG is at war against \paysportugal allied to a \MAJ, \paysportugal
breaks its alliance, sign a white peace with \ENG, becomes neutral and the
event occurs.
\bparag Allies of \ENG have the choice to either sign a white peace with
\paysportugal or break their alliance with \ENG and stay at war with
\paysportugal.
% (Jym) event several times in the table, so giving a penalty.
\bparag If \ENG is at war against \paysportugal (not allied to a \MAJ), then
the event cannot occur. Mark-off and play \RD instead.

\phdipl
\aparag \ENG receives a bonus of \bonus{+5} for its diplomacy on \paysportugal
for this turn only.

\effetlong
\aparag From now on \paysportugal always gives rights to trade to \ENG as per
\ref{chAdministration:Limited Access:Giving Rights}, even if it is not on the
English diplomatic track.


\subevent[pVI:Methuen:WoSS]{Dynastic link and alliance with Portugal}
\history{not historic}

\condition{}
\aparag If this event is triggered by \HIS, \paysportugal is annexed by
\HIS. Apply all the effects of \ref{pIII:POR Ann.:Portugal Annexed}.
\bparag Otherwise, apply this event.

\phevnt
\aparag If \paysportugal was annexed by \HIS as per \ref{pIII:POR
  Ann.:Portugal Annexed}, it breaks its annexation and becomes a regular minor
country.
\aparag \paysportugal signs a white peace with the \MAJ triggering the event.
\bparag Allies of the \MAJ triggering the event have the choice to either sign
a white peace with \paysportugal or break their alliance with the \MAJ and
stay at war with \paysportugal.
\aparag If it was not on the diplomatic track of the \MAJ triggering the
event, \paysportugal becomes neutral.

\phdipl
\aparag The \MAJ triggering the event receives a bonus of \bonus{+5} for its
diplomacy on \paysportugal for this turn only.

\effetlong
\aparag From now on \paysportugal always gives rights to trade to the \MAJ
triggering the event as per \ref{chAdministration:Limited Access:Giving
  Rights}, even if it is not on its diplomatic track.



\event{pVI:Act Union}{VI-8}{Act of Union}{1}{RistoMod}

\history{1704}

\condition{}
\aparag Cannot occur if \ENG has been defeated in a Jacobite rebellion
(\ref{pV:Glorious Revolution} or \ref{pVI:Jacobite Rebellion}). In that case
mark off as played.
% (Jym) and play \RD with the R in \ENG instead ?
\aparag Cannot occur if a Jacobite rebellion is still active.
\bparag In that case, mark off but re-roll another event.
\bparag During the first event phase after the end of the rebellion,
\ref{pVI:Act Union} will automatically be the first event rolled this turn.
\aparag Cannot occur if \paysecosse is not \VASSAL of \ENG.
\bparag In that case, mark off but re-roll another event.
\bparag During the first event phase where \paysecosse is \VASSAL of \ENG,
\ref{pVI:Act Union} will automatically be the first event rolled this turn.
% (Jym) Above, my interpretation of conditions of Risto + remark of Pierre
% below: (Risto) 1. Can occur only if Scotland is currently in \VASSAL of \ENG
% (whether by event 6:IV or not). Otherwise mark off as played.  3. Cannot
% occur if Jacobite rebellion (12:V or 10:VI) is still active.  In that case
% re-roll, but do not mark off as played.  (Pierre) Changes: if conditions 1
% or 3 : triggered as soon as Scotland is \VASSAL

\phevnt
\aparag \paysecosse is annexed by \ENG.
\bparag All \TradeFLEET levels of \paysecosse are immediately added to the
\TradeFLEET levels of \ENG in the same zone. This may cause automatic
concurrence to be solved immediately. If after that \ENG has more than 6
levels of \TradeFLEET in any zone, reduce to 6 levels.
% \bparag Units of \paysecosse currently in play are disbanded.  (Jym) Or not?
% Giving forces for free during a war that might have a big reinforcement roll
% may look very juicy but opposite to that, removing troops helping \ANG to
% sustain a siege looks absurd. Deleting.

\effetlong
\aparag All provinces belonging to \paysecosse in 1492 are now considered as
national provinces of \ENG.
\aparag From now on, \ENG can raise, upkeep and use military counters of
\paysecosse (not \TradeFLEET) as if it were its own counters.



\event{pVI:Bill Test}{VI-9}{Bill of Test}{1}{Risto}
\begin{todo}
  Change!
\end{todo}
\history{1673}

\effetlong
\aparag From now on \ENG can no longer be forced to change religion by foreign
conquest.
\bparag However, it can still be forced to change religion as a result of
\ref{pV:Glorious Revolution} or \ref{pVI:Jacobite Rebellion}.



\event{pVI:Heinsius}{VI-10}{Heinsius}{1}{Risto}

\history{1689-1720}
\dure{as long as \strongministre{Heinsius} remains the excellent minister}

\condition{\HOL can refuse this event if it so wishes. In that case mark off
  as played.}
\aparag \HOL can freely dismiss \ministreHeinsius at the end of any following
monarch survival phase and the event terminates.

\phevnt
\aparag \HOL receives an excellent minister \ministreHeinsius, with values
9/8/7.  He will last for a random length for Minister, see \ref{eco:Excellent
  Minister}.

\phdipl
\aparag \HOL can once ignore a call for help by an ally without the loss of
stability for such a treachery.



\event{pVI:WoPS}{VI-11}{War of Polish Succession}{1}{PB}

\history{1733-1735}
\dure{Until the end of the war caused by the event.}

\condition{}
\aparag The event is pending. It will be activated as soon as the year is 1700
or more and the king of \POL dies.
\bparag If the event is pending while \POL becomes a minor, continue to roll
for survival of the king every turn until his death (either scheduled or
premature) activate the event.

\phevnt
\aparag If this was not already the case, \POL becomes the \POLmin. \PRUMin
becomes the major \PRU. See \ref{chSpecific:Campaign:Becoming Prussia} for
details.
\aparag The crown of \payspologne is proposed to the step-father of a foreign
king and \payspologne looks for the protection of this foreign king.
\bparag If \ref{pVI:Great Northern War} happened and \SUE managed to impose
its candidate on the throne of \payspologne, then the potential protectors
are, in order, \SUE then \FRA.
\bparag In all other cases (\numberref{pVI:Great Northern War} did not happen
or wasn't won by \SUE), the potential protectors are, in order, \FRA then
\SUE.
\bparag The first potential protector must immediately accept or refuse the
crown. If it refuses, then the second one must either accept or refuse.
\aparag \payspologne immediately signs a white peace with its protector and is
put in \EG of its protector.
\bparag If both protectors refuse, \payspologne will fight alone in the
upcoming war. Apply only the first point of the diplomatic phase (\CB for
\RUS) as well as the effects of the peace phase on the future of \payspologne
% (Jym) Added the peace conditions in the case where there are no
% protectors. Most likely peace -3 for \POLmin and in this case beginning of
% the open bar.

\phdipl
\aparag \RUS has a free \CB against \payspologne this turn.
\aparag \AUS has a free \CB against \payspologne this turn.
\aparag \SUE (if not protector) and \PRU both have a (normal) \CB against
\payspologne this turn.
\aparag If \payssaxe was ruling \payspologne due to \ref{pV:Saxon King Poland}
and a war against \payspologne is declared due to this event, \payssaxe also
declares war on \payspologne and is put in \EG of the first country at war
against \payspologne in the following list: \RUS, \AUS, \SUE, \PRU.
% (Jym) Added \PRU at the end of the list since it is technically possible
% that it may be the sole country at war...
\aparag All countries entering war against \payspologne due to this event are
considered allied for the duration of the war without need to sign a formal
alliance.
\aparag If nobody declares war on \payspologne, it becomes a permanent \EG of
its protector as if there has been a Polish victory. Apply all the effects of
\xnameref{pVI:WoPS:Polish Victory}.

\phpaix
\aparag An extra malus of \bonus{-4} is applied for all separate peace against
\payspologne or \payssaxe (if it entered war due to being allied with
\payspologne by \ref{pV:Saxon King Poland}) for this war.
% (Jym) Formulation a bit complex for Saxony because the malus has no reason
% to be if the Polish Dynasty is no more in place or never was but Saxony was
% just a normal ally of say \AUS.
\begin{digressions}[Conditions of Victory]


  \digression[pVI:WoPS:Polish Victory]{Polish Victory}
  \aparag If \payspologne (and its side) signs a favourable peace of level 3
  or more, \payspologne becomes a permanent \EG of its protector.
  \bparag For all purposes except incomes (declarations of war, victory
  conditions, \ldots) consider that \payspologne is a \VASSAL of its
  protector.
  \bparag No diplomacy is allowed on \payspologne anymore.
  \bparag The protector immediately wins 50 \VP.
  % (Jym) Added: (mostly if \POL wins without protector)
  \aparag Absolutism is established in \payspologne.
  % (Jym) See below the trip about Lorraine annexation
  \aparag At the peace, the protector can annex any province, even the
  capital, of one minor country.
  \bparag This province must be adjacent to the territory of the protector.
  \bparag This can destroy the country.
  \bparag The minor must be either on the diplomatic track of the protector or
  on the diplomatic track of one of its enemies (even if not at war).
  \bparag This count as one peace condition if the province is occupied by the
  protector (or its allies) or as all peace conditions (for the protector and
  its allies) otherwise (minor not at war, or even allied with the protector).
  \bparag If \SUE is the protector, it can annexe this way the whole
  \regionNorvege whatever the current diplomatic status of \paysdanemark (or
  \paysVnorvege). This always count as all the peace conditions for the
  alliance of \SUE.


  \digression[pVI:WoPS:Polish Defeat]{Polish Defeat}
  \aparag If \payspologne (and its side) signs an unfavourable peace of level
  3 or more, the protector loses 15\VP (even if it was not at war).
  \bparag \payspologne becomes neutral. From now on, it will never be able to
  go above \SUB on the diplomatic track.
  \bparag Absolutism is abolished in \payspologne.
  \aparag From now on, \RUS, \AUS, \PRU and all countries of the \HRE can
  freely cross provinces of \payspologne. The provinces are considered enemy
  and don't give supply, it is not allowed to stop in \payspologne or pillage
  its provinces because of attrition.
  \aparag If they were still at war against \payspologne when the peace is
  signed, both \RUS and \AUS win 50\VP.


  \digression[pVI:WoPS:Status Quo]{Status Quo}\label{pVI:WoPS:Status quo}
  \aparag If neither side gets a full victory as per the previous cases, apply
  these effects.
  \aparag \payspologne is put in \EG of its protector. It is a normal minor.
  % (Jym) Technically, the protector may have let \POL down, so \POL may no
  % more be in \EG. But the loss may be not that high.
  \aparag The protector loses 15\VP (even if not at war).
  \aparag If they were still at war against \payspologne when the peace is
  signed, both \RUS and \AUS win 30\VP.
  \aparag Absolutism is abolished in \payspologne.
  \aparag From now on, \RUS, \AUS, \PRU and all countries or the \HRE can
  freely cross provinces of \payspologne. The provinces are considered enemy
  and don't give supply, it is not allowed to stop in \payspologne or pillage
  its provinces because of attrition.
  % (Jym) What ? Give a \CB to a \MIN ?
  \bparag Crossing polish provinces gives a \CB to \payspologne for the next
  diplomacy phase.
  % (Jym) See below the trip about Lorraine annexation
  \aparag At the peace, the protector can annex the last province of one minor
  country who only has one province left, even if this is a capital.
  \bparag This province must be adjacent to the territory of the protector.
  \bparag This destroys the country.
  \bparag The minor must be either on the diplomatic track of the protector or
  on the diplomatic track of one of its enemies (even if not at war).
  \bparag This does not count as a peace condition and is done in addition to
  the normal peace.
  \bparag If the protector chooses to annexe a province of a minor country not
  on its track (but on the track of one of its enemies), it must gives to its
  diplomatic patron the diplomatic control of a minor from its own track which
  is at least at the same level of diplomatic control. The enemy of the
  protector chose which diplomatic compensation he takes.
  \bparag If \SUE is the protector, it can annex this way the whole
  \regionNorvege as if it was only one province. \paysdanemark (or
  \paysVnorvege) must be on its track, or on the track of one enemy (in which
  case diplomatic compensation apply as above).
  % (Jym) Trip Lorraine (hist.) \FRA supports Stanislas and obtains a losing
  % Status Quo. Some escape for Stanislas must be found for the international
  % standing of Louis XV. Francois I of Lorraine is the Emperor (and husband
  % to Maria-Theresa). He exchanges Lorraine against \paysParme and Lorraine
  % goes to Stanislas. (\PPI) Lorraine could be annexed during the war, but
  % it usually did not work, either France won or \SUE looked stupid. (game
  % problem) 1. Give \SUE the occasion of a good deal. 2. Allow a bit more
  % choice to the protector especially if Lorraine was already annexed.
  % 3. Avoid the search for Status Quo for the protector. Answers: 1. Norway!
  % 3. Thing a bit strange with an annexation in case of win. More interesting
  % because no compensation to give. Choice about the minor. Case study: SUE
  % may annex Norway, Kurland. Or some other things in the \HRE, such as bits
  % of \paysHanse, Munster, Oldenburg, Bremen. The bad thing would be to annex
  % bits of Hanover, but let's see \SUE try to anger \ENG.
  % For \FRA, Lorraine, Alsace, Liege, Savoy, Switzerland, Palatinate,
  % Cologne, Trier, Baden, \paysMonferrato and \paysgenes.
  % It looks fine, seen like that.

\end{digressions}



\event{pVI:War Turkey}{VI-12}{War against Turkey}{2}{RistoMod}

% (Jym) I wonder about the 2 dates and occurrences. But in EU9, it is a
% problem because \AUS is major all the time.
\history{1716-18/1737-39}

\condition{The first eligible in the following list occurs, each case can only
  happen once per game}
\aparag \AUS receives a free \CB against \TUR for this turn. It can choose to
decline this offer, in which case proceed with the list.
% (Jym) Not possible in EU9. Removing: b. Inactive \HAB declares war against
% \TUR.  (JCD) We should keep it. In EU9, c. will be used; in EU8, possibly
% too.
\aparag If inactive, \AUSmin declares war against \TUR. It calls for allies as
usual.
\aparag If inactive, \paysVenise declares war against \TUR. It calls for
allies as usual and will have \bonus{+2} to all the reinforcements check made
during this war.
\aparag If none of the conditions apply, nothing happens.



\event{pVI:WoAS}{VI-13}{War of Austrian Succession}{1}{PB}

\history{1740-1748}
\dure{Until the end of the war}

\condition{}
\aparag Cannot happen if there is a \GE. In this case, mark off and play \RD
instead.
\aparag Cannot happen before period VI (thus, \AUSmin has become \AUS
anyway). In this case, do not mark off and re-roll.
\aparag Cannot happen before the start of the war caused by \ref{pV:WoSS}. In
this case, do not mark off and re-roll.

\phevnt
\aparag[The Pragmatic Sanction]
\bparag The king of \AUS dies. The new queen is \monarque{Maria Theresia}
(values 8/8/7, lasts 8 turns, does not roll for survival during 5 turns, adds
\ARMY\faceplus as basic forces).
% (Jym) I am not sure about how the dissociation is managed when \AUS becomes
% a \MAJ in pIV. By default:
\bparag Mandatory dynastic dies between \HIS and \AUS are voided (if still
existent).
\bparag If \paysbaviere won the electorate during \ref{pIV:TYW}, it opposes
the Sanction and pretends to the throne of \AUS. Otherwise, \payspalatinat
does.
\aparag \AUS loses control of the pretending country.
\aparag The pretending country proposes a white peace to its current enemies
and then declares war to \AUS.
\aparag If this is not already the case, \POL becomes the \POLmin. \PRUMin
becomes the major country \PRU.
\bparag See \ruleref{chSpecific:Campaign:Becoming Prussia} for details on how
to handle this.
% (Jym) What? this was attached to the point above (PPI)

% \PRU offers an immediate white peace to its enemies. \PRU then has a free
% \CB against \POL.

% (Jym) I imagine the white peace would rather be proposed by \POL rather than
% \PRU. And why this \CB?

\phdipl
\aparag \PRU has a free \CB against \AUS at this turn (only).
\bparag If it uses it, \PRU and the pretending country are allied for the
duration of the war.
\aparag \FRA has a \CB against \AUS during every turn of the war caused by the
event.
\bparag If it uses it, place the pretending country in \EG of \FRA.
\bparag If \PRU and \FRA use it, they are allied for the war without need for
signing a formal alliance.
\bparag If \FRA does not use this \CB at the first turn of war, the pretending
country will call for allies as per normal rules.
\aparag \ANG has a free \CB against \FRA as a reaction of the previous \CB
(only).
\bparag This \CB can only be used in reaction to \FRA declaring war to \AUS.
\bparag If it uses it, \ANG and \AUS are allied for the war, without need for
signing a formal alliance.

\phadm
\aparag At the first turn of the war (only), \PRU rolls for reinforcements as
a minor country (in offensive attitude).
% (Jym)
\bparag These reinforcements are \Veteran. They do not count toward this turn
purchase limit.
\aparag At the first turn of the war (only), \AUS rolls for reinforcements as
a minor country (in defensive attitude).
% (Jym)
\bparag These reinforcements are \Conscripts. They do not count toward this
turn purchase limit.

\phpaix
\aparag If \AUS signs an unconditional surrender, it loses the imperial
throne. The pretending country becomes Emperor for the rest of the game.
\bparag In that case, \PRU automatically gets the royal dignity as per
\ref{pV:Kingdom Prussia}. If that event didn't happen yet, consider it to be
the first event rolled next turn with any mention to \paysbrandebourg
referring to \PRU instead (in that case, \AUS \textbf{must} give the royal
crown to \PRU in the following diplomacy phase). (JCD) TODO there is probably
a problem with that, since \AUS will no more be Emperor...
\aparag Extra \VP are granted for the control of certain provinces at the end
of the war.
\bparag \PRU gains 25\VP per province annexed from \AUS. It loses 20\VP if it
annexes none.
\bparag \AUS gains 20\VP per province annexed from \PRU. It loses 25\VP if it
annexes none.
\bparag The player controlling the pretending country gains 30\VP per province
annexed from \AUS and loses 15\VP if the pretending country annexes no
province. These \VP are also lost (or won) by \AUS.



\event{pVI:Kurland}{VI-14}{War of Succession in Kurland}{1}{PBnew}

\history{1730-1731}
\dure{As long as \payscourlande exists.}

\phevnt
\aparag The provinces \provinceCourlande and \provinceLivonija declare
independence from their current owner and form the minor country
\payscourlande.
\aparag \leader{von Sachsen}, or, if he's not alive, a random mercenary
general lasting 4 turns, takes command in the new duchy and look for a
protector.
\bparag The following countries must immediately accept or refuse to become
protector of the duchy (in order): \FRA, \AUS, \PRU, \HOL.
% (Jym) Suppression of references to the country that Sachsen serves
\bparag If all of them refuse, then the general wisely chooses to stand
back. \payscourlande doesn't get a general and won't get reinforcements in any
war.
\bparag If there is a protector, then \payscourlande becomes a permanent
\VASSAL of its protector and no diplomacy is allowed on it.

\phdipl
\aparag Any country owning one province or more of the minor when the event
happens gets a free \CB against \payscourlande.
% (Jym):
\bparag A minor country uses this \CB only if there is already a major country
using this \CB (for the other province).

\phadm
\aparag The general of \payscourlande can lead troops of its protector.

\phpaix
\aparag \payscourlande has no capital and can thus be annexed by anybody.

\effetlong
\aparag The protector loses 30 \VP at the end of the game if \payscourlande
does not exist.



\event{pVI:Slave Revolts WI}{VI-15}{Slave Revolts in the West
  Indies}{*}{Risto}

\history{No precise date}

\phevnt
\aparag Roll 1d10 for each power having \COL in areas \granderegionCuba,
\granderegionHaiti and/or \granderegionAntilles. On a result of 7 or more, a
\REVOLT\faceplus is placed in one randomly chosen \COL of the power.



\event{pVI:Bantu Raids}{VI-16}{Bantu Raids}{*}{Risto}

\history{No precise date}

\begin{todo}
  May represent the early Xhosa wars starting in 1779 but should then be
  pushed in VII. Otherwise, could be removed.
\end{todo}

\phevnt
\aparag Natives of area \granderegionNatal and the two coastal provinces
bordering it are activated for this turn and shall attack all \COL/\TP in
these provinces.

\phadm
\aparag The strength of the natives activated by this event is always 6\LD
(whatever the printed value) and they automatically receive a native leader.



\event{pVI:Last Great Mughals}{VI-17}{The Last of the Great Mughals}{1}{PBnew}

\history{1707 (Death of Aurangzeb)}

\phevnt
\aparag The general \leader{Great Mughal} is removed from the game.
\aparag \xnameref{pII:Mughal Expansions} cannot happen anymore.
\aparag The basic forces of \paysmogol becomes \ARMY\faceplus.
\aparag Reaction of country \paysmogol becomes 3.
\aparag \paysmogol loses 1d10/3 (round to closest) areas (the ones with the
largest numbers).

\effetlong
\aparag \paysmysore and \payshyderabad are created as soon as their respective
province does not belongs to \paysmogol anymore.
\bparag This can happen either at the start of this event, due to the
provinces lost by this event or at some other point in the game if \paysmogol
loses provinces.
\bparag Both countries are not necessarily created at the same time.
\aparag Colonial powers may now raise Indian troops (``Sepoy'') as per their
respective specific rules.



\event{pVI:Wars India}{VI-18}{Wars in India}{3}{PBnew}

\condition{}
\aparag Roll 1d10 and apply the correct subevents.
\bparag 1-4 = A) War between \paysmogol and \paysperse. Apply
\xnameref{pVI:India:Mughal Persian War}.
\bparag 5-8 = B) War between \paysafghans and \paysperse. Apply both
\xnameref{pVI:India:Afghan Empire} and \xnameref{pVI:India:Fall Persian
  Safavids}.
\bparag 9-10 = C) War between \paysafghans and \paysmogol. Apply both
\xnameref{pVI:India:Afghan Empire} and \xnameref{pVI:India:Rise Marathi}. This
case may not happen before either~\ref{pVI:Last Great Mughals}, re-roll
another case if needed.

\aparag Each of the three previous cases can only happen once. If it already
happen, re-roll another case.

\aparag Each of the following sub-event can only happen
once. \xnameref{pVI:India:Afghan Empire} may occur due to two different cases
(B and C). The second time, ignore it and only plays the other sub-event.

\aparag In each of the three case, natives in one random province in
\continentIndia are activated.

\subevent[pVI:India:Mughal Persian War]{\paysmogol-\paysperse War}
\history{1739}

\phevnt
\aparag \paysmogol loses all provinces except the areas \granderegionDelhi,
\granderegionAoudh, \granderegionBengale, \granderegionGondwana and
\granderegionOrissa.
\aparag Lower the difficulty and tolerance (for \COL and \TP implantation) by
2 in every province controlled by \paysmogol
\aparag \paysperse gets the general \leaderwithdata{Nadir Shah} for 5 turns.

\phdipl
\aparag Test fidelity of \paysPerse and \paysOrmus.


\subevent[pVI:India:Afghan Empire]{Afghan Empire}
\history{1747}

\phevnt
\aparag The minor country \paysafghans is created and owns area
\granderegionAfghanistan except \provinceHerat if owned by \paysPerse.


\subevent[pVI:India:Fall Persian Safavids]{Fall of the Persian Safavids}
\history{1749}

\phevnt
\aparag The lasting effect of \ref{pIII:Persian Safavids} are cancelled.
\aparag \provinceHerat is annexed by \paysafghans

\phdipl
\aparag Test fidelity of \paysPerse and \paysOrmus.

\subevent[pVI:India:Rise Marathi]{Rise of the Marathi}
\history{1746-1761}

\phevnt
\aparag \paysmogol only loses all provinces except the areas
\granderegionDelhi, \granderegionAoudh, \granderegionBengale and
\granderegionGondwana.
\aparag Lower the difficulty and tolerance (for \COL and \TP implantation) by
2 in every province in \continentIndia.
\bparag This is not cumulative with the decrease caused inside \paysmogol by
\xnameref{pVI:India:Mughal Persian War}.



\event{pVI:Mazepa}{VI-19}{Revolt of Mazepa}{1}{PBnew}

\history{1708-1709}

\condition{}
\aparag \paysukraine is looking for a new protector.
\bparag If this event is triggered during \ref{pVI:Great Northern War}, either
by troops of \SUE entering \regionUkraine or by rolling for it on the table,
then the new protector is \SUE.
% (Jym) added:
\bparag If the current protector of \paysukraine is at war against another
\MAJ, then the new protector is chosen among the countries at war against the
current protector in the following list: \RUS, \POL, \TUR, \AUS, \SUE, \PRU.
\bparag If the current protector of \paysukraine is not at war against any
other \MAJ, then the new protector is chosen in the following list: \POL (if
Orthodox), \RUS, \TUR, \POL, \AUS, \SUE, \PRU.
% (Jym) First neighbours as protector then further away

\phevnt
\aparag The potentials protectors are asked in order if they accept or not to
protect \paysukraine.
\bparag If all refuse, \paysukraine will not have a protector for the duration
of the war.
\aparag \paysukraine declares war on its former protector and the new
protector must immediately join this war with no cost in \STAB.
\aparag Counters of \paysukraine are immediately removed from play.
\aparag Place a \REVOLT \faceplus in a province of \regionUkraine.
\bparag If the event is triggered by Swedish presence, then the \REVOLT is put
in the province where the Swedish \ARMY is. Otherwise, a random province is
chosen in \paysukraine.
% (Jym) general Mazepa with extravagant values !
\aparag Place general \leaderwithdata{Mazepa} with the \REVOLT, scheduled to
last 4 turns.
\aparag Place a \LD of \paysukraine in the revolted province.
\bparag If the new protector either has a common border with \paysukraine or a
``king ranked'' general in a province adjacent to \paysukraine, place an
\ARMY\facemoins instead.
\bparag ``King ranked'' generals are those bearing the king symbol, namely
monarchs, \leader{Carl XII} as an heir to the throne
(\ref{chSpecific:Sweden:Charles XII}) or \leader{Sadri Azam} and other Viziers
(\ref{chSpecific:Turkey:Vizier}).
\aparag The revolt is considered active as long as \leaderMazepa is alive and
at least one \REVOLT exists in one of the provinces of \regionUkraine.

\phdipl
\aparag Any country possessing a province of \regionUkraine with a \REVOLT in
it has a free \CB against either the former or the current protector (its
choice).
\bparag Minor countries use this \CB against the new protector.
\aparag As long as the revolt is active, \TUR as a free \CB against either the
former or the new protector (its choice).

\phadm
\aparag If the revolt is active \paysukraine roll for reinforcements in
offensive attitude, base on the income of the provinces with a \REVOLT in
them.
\bparag The reinforcement roll has a malus of \bonus{-2} unless a ``king
ranked'' leader of the new protector is in or adjacent to \paysukraine.

\phmil
% (Jym) Add a limited supply for \paysukraine else it may botch if it falls
% out of war and Mazepa has zero supply
\aparag \REVOLT are limited supply sources for the troops of \paysukraine but
are not supply source for the protector.
% (Jym) added, so that \REVOLT may spawn as long as the \ARMY is there
\aparag If a stack containing troops of \paysukraine takes a fortress, place a
\REVOLT \facemoins in the province.

\phpaix
\aparag The \REVOLT can extend in any province of \regionUkraine.
\bparag \REVOLT in \paysUkraine cause loss of \STAB to the \textbf{former}
protector. Other \REVOLT in \regionUkraine cause loss of \STAB to the owner of
the province as per normal rules.
\aparag If the new protector signs a white or favourable peace while the
revolt is still active, all the provinces of \regionUkraine belonging to
countries that were at war against the new protector during this war are
annexed by the \MIN \paysukraine. The new protector gain all the benefits of
\ref{pIV:Revolt Cossacks}.
\aparag Otherwise, the former protector stays protector of \paysukraine (with
the provinces still belonging to the minor after the peace is signed).

% (Jym), Placeholders

\event{pVI:WoJE}{VI-s}{War of Jenkins' ear}{1}{PBNotEvenWritten}
\history{1739-1748}
\begin{todo}
  \ANG vs \HIS in America. Later part of~\ref{pVI:WoAS}.
\end{todo}


\event{pVI:Comuneros}{VI-t}{Revolt of the Comuneros}{1}{PBNotEvenWritten}
\history{1721-1735}
\begin{todo}
  Revolt in Paraguay. Maybe doable via revolt tables only.
\end{todo}

\event{pVI:WoQA}{VI-u}{War of the Quadruple Alliance}{1}{PBNotEvenWritten}
\history{1718-1720}
\begin{todo}
  \SPA vs \paysNaples.
\end{todo}

\event{pVI:Alberoni}{VI-v}{Alberoni}{1}{PBNotEvenWritten}
\history{1711-1719}
\begin{todo}
  Excellent (?) minister for \HIS. Should be VI-2(2). Could be related
  to~\ref{pVI:WoQA}.
\end{todo}

\event{pVI:Bulavin Rebellion}{VI-w}{Bulavin's Rebellion}{1}{PBNotEvenWritten}
\history{1707-1708}
\begin{todo}
  Revolt in \paysAstrakhan.
\end{todo}


\event{pVI:Africa}{VI-x}{Troubles in Africa}{*}{JymNew}
\history{No precise date. Hypothetical clashes with inland African empires.}

\begin{todo}
  Should replace~\ref{pVI:Bantu Raids}.
\end{todo}

\phevnt
\aparag Roll one die on the following table: 1. \granderegionSenegal ;
2. \granderegionCotedivoire ; 3. \granderegionCotedor ;
4. \granderegionCameroun (except \province{Fernando Po}; 5. \granderegionGabon
; 6. \granderegionCongo ; 7. \granderegionAngola ; 8. \granderegionNyasa (two
Southern provinces) ; 9. \granderegionNyasa (two Northern provinces) ;
10. \granderegionKenya.
\bparag The natives in the two provinces designed are activated. They have a
strength of 4\LD and one \LeaderG, whatever the printed value.

\event{pVI:Camisards}{VI-y}{Revolt of the Camisards}{1}{JymNew}
\begin{todo}
  Maybe should be V-6 (2).
\end{todo}
\history{1702-1711}

\condition{If \ref{pV:Expulsion French Protestants} %
  % Edit de Fontainebleau, revocation de l'Edit de Nantes
  did not occur yet, apply it now in addition to this event.}
% Gros des combats de 1702 a 1704.  Je me souviens plus de ce que j'avais
% dit...  De mémoire, R- en Languedoc, R+ (+general (Roland/Cavalier)) + A-
% huguenot (+general (l'autre)) en Cevennes.  l'A a des renforts mais a un
% moment FRA peut payer pour virer les généraux.  J'avais du envoyer un mail
% avec tout ca mais je trouve plus.



\event{pVI:Barbaresques}{VI-z}{End of the Ottoman rule in North
  Africa}{1}{PBNotEvenWritten}
\history{17??}
\dure{Until the end of the game}

\effetlong
\aparag If \ref{pIV:Morocco} did not happen yet, apply it immediately in
addition to this event.
\aparag \TUR has a malus of \bonus{-3} to diplomacy with all \Barbaresques
(\paysCyrenaique, \paysTripoli, \paysTunisie, \paysAlgerie and \paysMaroc).
\bparag This malus supersedes the malus on \paysmaroc given by
\numberref{pIV:Morocco} and is not cumulative with it.

\stopevents

% Local Variables:
% fill-column: 78
% coding: utf-8-unix
% mode-require-final-newline: t
% mode: flyspell
% ispell-local-dictionary: "british"
% End:

% LocalWords: pV pVII pVI WoSS Vassalisation Methuen Heinsius WoPS WoAS dt
% LocalWords: Kurland Mughals Mazepa Camisards Barbaresques PBNew Altranst
% LocalWords: Leszinski Poniatowski PBMod RistoMod offensive Jym malus Quo
% LocalWords: pIV Risto Theresia PBnew JCD TODO von Sachsen Mughal pII Sadri
% LocalWords: subevents Safavids Azam JymNew PBNotEvenWritten GNW PPI
%  LocalWords:  Comuneros
