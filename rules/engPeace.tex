% -*- mode: LaTeX; -*-

\definechapterbackground{Budget and Peaces}{victories}
\chapter{Budget and Peaces}\label{chapter:Peace}

\section{Overview of the phase}

% RaW: [50]
\aparag[Administration] At the end of the turn, final administrative actions
are resolved and budgets must be completed. First, exceptional taxes that were
scheduled during the administrative phase are resolved. Then comes the
exchequer test. At this point, players roll to determine how well the taxes
where collected this turn and to discover their precise income. If the income
is not enough to cover for the expenses, loans must be contracted, either from
the people of your country or from international bankers. Last but not least,
countries may try to improve their \STAB.

\aparag[Peace] Wars can be ended only by a Peace. There are several types of
Peace, from the white peace (return to statu quo) to the unconditional
surrender. The type depends mostly on the difference between the \STAB of the
belligerents, slightly modified by the military situation. In some cases,
countries must accept any peace proposed by the opponents, but usually some
discussion occurs between the players.
\bparag[Crusade] If \TUR conquers too many Christian provinces, the pope may
try to launch a Crusade. 

% \aparag[Events and special peace conditions] A small number of political
% events can initiate wars where the peace conditions are constrained or
% modified by specific conditions. These events do not change the rules below
% for other simultaneous conflicts where one side of the war is not one of those
% described in the event.
% \bparag Remark that the other side must enter war through another means than
% the event or the alliances network called at the time of the event (else, it
% is considered part of the event).

\aparag[Sequence of the Peace Phase.]
\PeaceDetails

\section{Exceptional taxes}\label{chPeace:Exceptional taxes}
\aparag[Exceptional taxes] Exceptional taxes are scheduled during the
Administrative phase. See~\ref{chExpenses:Exceptional Taxes} for details (and
examples). They are resolved at this point only. That is, until the end of the
turn (and after most expenses have been planned), players won't know exactly
the amount of collected taxes.
\bparag Note that Exceptional taxes must be planned during Administrative
phase. If a country forfeited the possibility to do so, it is to late now to
decide to raise taxes.

\aparag[Resolution of the taxes]
\bparag Each country which has planned taxes should have written a modifier in
\lignebudget{Exceptional taxes modifier A} (copied from
\lignebudget{Exceptional taxes modifier B}). This modifier was \ADM + 3
$\times$ \STAB (at the time of the Administrative phase).
\bparag Roll 1d10, add the modifier and multiply the result by 10. This is the
amount of taxes (in \ducats).
\bparag Write this amount in \lignebudget{Exceptional taxes}. It may well be
negative if the modifier was negative. In this case, the country will actually
lose money because of the taxes. It is not possible to refuse a ``tax'' once
the amount is known.

\aparag[\RT before Exchequer test]
\bparag Players can know compute their \RT before resolving the Exchequer
test.
\bparag This is the sum of lines \ERSlong{RT after Diplomacy} +
\ERSlong{Pillages, privateers} + \ERSlong{Gold from ROTW and Convoys} +
\ERSlong{Exceptional taxes} of \EcoRS. It is written in \lignebudgetlong{RT
  before Exchequer}.
\bparag Players should also copy \lignebudgetlong{Gross income B} in
\lignebudgetlong{Gross income A} and \lignebudgetlong{Total expenses} in
\lignebudgetlong{Expenses}.

\section{Exchequer test}\label{chPeace:Exchequer test}
\subsection{Gross Income}
\begin{designnote}
  We explain here the technical rules of the economical system. For a
  description of the spirit of these rules, see~\ref{chThePowers:Exchequer}.

  The rules here are quite ``algorithmic'' in order to have them as precise as
  possible and avoid misinterpretations. Thus, there are not well suited to
  understand the whys of the system (only the hows). These rules are meant to
  be closely followed step by step. Check~\ref{chThePowers:Exchequer} in order
  to understand what should happen, as well as read some examples.
\end{designnote}

\aparag[Exchequer test] Each country roll a die on~\ref{table:Administrative
  Actions} modified as follows:
\begin{modlist}
\item[+2] If completely at Peace (no war (including civil or overseas wars),
  no intervention (limited or foreign)).
\item[-1 ] per 100\ducats of National Loan, or per International Loan.
\item[-1 ] per bankruptcy (or broken loan treaty) in the last 5 turns.
\end{modlist}
\bparag The result may be either F\textetoile, F, \undemi, \undemi\textetoile,
S or S\textetoile.

\begin{playtip}
  Bankruptcies should be noted by a small \textetoile in \lignebudget{Gross
    income A} for the turns where they affect the Exchequer test.
\end{playtip}

\aparag[Percentages] By cross-referencing this result with the first three
columns of~\ref{table:Exchequer test}, countries obtain three percentages for
``Regular Income'', ``Prestige Income'' and ``National Loan''.
\bparag Add 10 to the ``National Loan'' of countries that are not completely
at peace.
\bparag Add 10 (cumulative) to the ``National Loan'' of \HIS if it has
declared a politic of expulsions (see~\ref{chSpecific:Spain:Expulsion}).
\bparag It is possible and intended that these percentages sum up to more or
less than 100\%.

\aparag[Incomes] Apply each of the three percentages to the whole Gross Income
(\lignebudget{Gross income A}), rounding down, to obtain three incomes.
\bparag Copy these incomes in \lignebudgetlong{Regular income},
\lignebudgetlong{Prestige income} and \lignebudgetlong{Max. national loan}.

\begin{playtip}
  It is often convenient to cut these three boxes in half (diagonally). After
  rolling the exchequer test, immediately copy the percentages in the top-left
  halves, this avoid forgetting the result. Next you can take your time to
  compute the actual value and write it in the bottom-right halves.
\end{playtip}

\GTtable{etatsauvrai}

\subsection{International Loans}\label{chPeace:International loans}
\aparag[Available money] The total amount of available money for international
loans is:
\bparag 50\ducats from the start (unspecified bankers).
\bparag Always add 50\ducats, or 100\ducats for the emperor (German bankers).
\bparag Always add 50\ducats, or 100\ducats for the diplomatic patron of
\paysGenes (Genoese bankers).
\bparag After~\ref{pIII:Amsterdam Stock Exchange} add 50\ducats, or 100\ducats
for \HOL.
\bparag After~\ref{pIV:London Stock Exchange} add 50\ducats, or 100\ducats
for \ANG.
\bparag Thus, the total available money will be between 150 and
350\ducats. Note that it does depend on the country, that is all the countries
have different loan capacities.

\aparag[International Loans test] Each country may roll a die
on~\ref{table:Administrative Actions} modified as follows:
\begin{modlist}
\item[+2] If completely at Peace (no war (including civil or overseas wars),
  no intervention (limited or foreign)).
\item[-1 ] per 100\ducats of National Loan, or per International Loan.
\item[-1 ] per bankruptcy (or broken loan treaty) in the last 5 turns.
\item[+1 ] if the country has a Stock Exchange (\HOL after~\ref{pIII:Amsterdam
    Stock Exchange} and \ANG after~\ref{pIV:London Stock Exchange}).
\end{modlist}
\bparag The result may be either F\textetoile, F, \undemi, \undemi\textetoile,
S or S\textetoile.
\bparag Note that this roll is different from the Exchequer test. Do not use
the same roll for both the Exchequer test and the International Loans test as
this would increase the chances of extremely bad results.

\aparag[International Loan] By cross-referencing this result with the last
column of~\ref{table:Exchequer test}, countries obtain one percentages for
``International Loan''.
\bparag Apply this percentage to the total available money and copy the result
in \lignebudgetlong{Max. international loan}.

\begin{playtip}
  Often, International loans are not necessarily and this step may be skipped
  by most countries. It may be useful to start computing your budget (next
  step) before deciding whether to take an international loan or not. Hence,
  it is sometimes more fluent to start computing your budget and then possibly
  come back to looking at international loans. Since there is no new knowledge
  gained between the Exchequer test and the Budget, this does not change
  anything.

  If you wish to follow closely the order of the steps, you should, however,
  always roll for international loan preventively, thus avoiding bad
  surprises.
\end{playtip}

\section{Budget}\label{chPeace:Budget}
\subsection{Expenses}
\aparag[Regular income] Write in \lignebudget{Remaining expenses} the
difference between \lignebudget{Expenses} and \lignebudget{Regular income}.
\bparag This may be a negative number in the rare case where the Regular
income is larger than the total expenses.

\aparag[Prestige income] Write in \lignebudget{from Prestige} any non-negative
number smaller than both \lignebudget{Prestige income} and
\lignebudget{Remaining expenses}.
\bparag Small value means that more money is spent for prestige \VPs and less
for day-to-day expenses. Those will be covered by loans or debt.

\begin{designnote}
  You cannot spend additional money for prestige (it must be non-negative).
  You cannot take more from prestige than the ``Prestige Income'' (smaller
  than \lignebudget{Prestige income}).  You cannot take more from prestige
  than what is left to pay after the regular income is spent (smaller than
  \lignebudget{Remaining expenses}).
\end{designnote}

\aparag[National Loans] Write in \lignebudget{from N. loan} any non-negative
number smaller than \lignebudget{Max. national loan}.
\bparag Copy this number in \lignebudget{New National loans}.

\begin{designnote}
  National Loans are not limited by expenses. However, you'll have to pay
  interest for them and maybe even refund your people someday.
\end{designnote}

\aparag[National Loans] Write in \lignebudget{from I. loan} any
non-negative number smaller than \lignebudget{Max. international loan}.
\bparag Copy this number in \lignebudget{New International loans}.
\bparag Copy this number in \lignebudget{International loans refunds},
\textbf{three turns} after the current one.
\bparag Copy 10\% of this number (round up) in \lignebudget{International
  loans interests} for the \textbf{next three turns}. If there is already a
number in one of these boxes, add the new value to it.
\bparag That is, you should write 3 interests (for the next three turns), and
one refund (for the same turn as the last interest).

\begin{playtip}
  International loans are usually a bad idea because of the scheduled
  mandatory refund. Use them only when in need.
\end{playtip}

\begin{exemple}
  A correctly filled new international loan (of 100\ducats, at turn $n$) over
  an existing one (of 200\ducats, from turn $n-2$):
  \begin{tabular}{|c|l|r|r|r|r|r|}
    \hline
    & Turn & $n-1$ & $n$ & $n+1$ & $n+2$ & $n+3$\\
    \hline
    1 & New International loan & & 100 & & &\\
    \hline
    2 & I. loan interest & 20 & 20 & 30 & 10 & 10\\
    \hline
    3 & I. loan refunds & & & 200 & & 100\\
    \hline
  \end{tabular}
\end{exemple}

\aparag[New \RT]
\bparag Write in \lignebudget{RT balance} the sum of \lignebudget{from
  Prestige} + \lignebudget{from N. loan} + \lignebudget{from I. loan}
\textbf{minus} \lignebudget{Remaining expenses}. It may be negative if
\lignebudget{Remaining expenses} is too big.
\bparag Write in \lignebudget{RT after Exchequer test} the sum of
\lignebudget{RT before Exchequer} + \lignebudget{RT balance}.

\begin{designnote}
  \lignebudgetlong{Remaining expenses} depict \textbf{expenses} that are left
  to be paid after using the Regular income. Hence it is subtracted from the
  \RT while other lines are added (they are money taken from prestige or loan
  in order to fill the treasury).

  If \lignebudget{Remaining expenses} is \emph{negative}, regular income was
  enough to cover all expenses. Then, the surplus is added to the treasury (as
  subtracting a negative number result in an addition).
\end{designnote}

\begin{designnote}
  All in all, do not try to understand all the steps here while reading the
  rules. After a couple of turns of computing your budget, things will become
  more natural. Note that if you are having a ``teaching session'', you should
  try several ``stupid'' things with your budget to see the consequences.
\end{designnote}

\begin{playtip}
  When planning expenses, it is obviously a good idea to keep an eye on the
  possible income\ldots Too many expenses result in bankruptcy while too few
  result in money ``wasted'' for prestige (instead of being use for buying
  troops or waging war).

  Here are some guidelines in preparing your budget:
  
  First, check in the administrative actions table what are the possible and
  plausible results with respect to your current (and expected) \STAB. You may
  discard very unlikely results (with only 10\% chance of happening) but you
  know you take a risk doing so. It is especially important to take into
  account the worse possible result you may obtain if you want to limit risks.

  Second, check in the Exchequer test table the sum of percentages these
  results produce. Check separately the sum of Regular + Prestige income
  (income without debt) and the sum of the three percentages (income with
  debt). Applying these percentage to your Gross Income will give some amount
  of money.

  Do not spend more than your best income with debt, obviously, doing so
  will result in problems. Spending more than the worse income with debt means
  taking risks. Estimate the risks (Is it a 10\% or 30\% chance of getting the
  worse result?) compared to the situation (Do you have lot of money in your
  \RT to handle the loss?) and the expected gain (Will the extra expense allow
  you to win the war?)

  Spending less than the worse income without debt means that some money will
  necessarily go into Prestige \VPs. Are you sure it won't be better used for
  troops, economical development, \ldots? Spending less that the best income
  without debt means that you may get Prestige \VPs but they are not
  guaranteed either.

  The good cases is when the worst income with debt is roughly equal (or
  larger) to the best income without debt. Spending that amount of money means
  that the worse that can happen is to take a new loan (that can be handled
  later) and that you won't waste too much money on Prestige. Note that you
  have to plan your administrative actions and loan refund before the military
  phase, thus without knowing precisely how long the turn will last and how
  much you'll spend for moving troops (especially if at war). Thus, there is
  often some risk involved\ldots

  \smallskip

  Remember that the economical system works best if you have some loan that
  you refund and recontract immediately (for a net effect of transferring
  Prestige income into the \RT). If you plan to use this loan trick, then the
  amount of loan involved is not really a debt, that is increase you income
  without debt by this amount when planning your expenses.

  \smallskip

  Remember that the worse that can happen is a \RT collapse. But even for that
  you need several turns of bad luck, bad management, or bad wars. Thus, don't
  be afraid of making too big errors with the economical system. You should
  get the hand of it before catastrophic results occur\ldots
\end{playtip}

\begin{exemple}
  If your \STAB is +2 and your are at peace (\bonus{+2} to Exchequer test),
  then you'll likely to get \undemi\textetoile, S or S\textetoile (with only
  10\% chance of \undemi). \undemi\textetoile has 100\% income with debt while
  S\textetoile has 100\% income without debt. Thus, by spending as much as
  your Gross income, you're almost guaranteed to be able to cover your
  expenses, maybe with some new loans. There is a small risk (10\%) of a bad
  result (\undemi) that will leave you with only 80\% income. Estimate the
  risk versus gain for the last 20\% of expenses. On the other hand, a good
  result gives you up to 120\% with debt, hence some choice on whether to
  contract loan in order to get more Prestige.

  \smallskip

  If your \STAB is -2 and you roll at \bonus{-3} due to heavy loans or
  previous bankruptcies, then the likely result are F\textetoile, F or
  \undemi\ (disregarding the unlikely \undemi\textetoile). If you are at war,
  the income with debt for F\textetoile is 80\%, and the income with debt of
  \undemi\ is 90\%. Thus by spending around 80\% of your Gross Income, you're
  sure to be able to fill your budget with some loan. But you're also sure to
  need some new loan\ldots (and a good surprise may arise in the form of
  \undemi\textetoile).

  \smallskip

  Note that the true difference in the table is between \undemi\ (only 50/80\%
  of the total) and \undemi\textetoile (70/100\%). Especially, being at peace
  with a \STAB of +3 guarantees a good result.
\end{exemple}

\subsection{Loan Management}
\aparag Players must then correctly take care of their loans for the next
turn.

\aparag[International loans]
\bparag Since the interests are not changed by partial refund of the capital,
management of the international loans is entirely done during the
administrative phase (when bankrupting or refunding) and the budget segment
(for new loans).

\aparag[National loans]
\bparag Compute in \lignebudget{National loans at end} the difference between
\lignebudget{National loans at start}, minus \lignebudget{National loans
  bankruptcy}, minus \lignebudget{National loans refunds} and add
\lignebudget{New National loans}.
\bparag Report this number in \lignebudget{National loans at start} of the
next turn.

\subsection{Prestige and Wealth}\label{chPeace:Prestige and Wealth}
\aparag[Wealth] During each period, a global wealth is computed for each
country. Wealth represent the overall economical situation of the country, as
well as exceptionally good management (in the form of Prestige).
\bparag At the end of each period, wealth is converted into \VPs. Each country
has a different rate of exchange of wealth for \VPs as each country has
different typical economical situation.
\bparag All in all, each country is expected to score around 100\VPs for
wealth each period, give or take a few dozens if this is supposed to be a
period of glory or decay.

\aparag[Prestige] Write in \lignebudget{Prestige VPs} the difference between
\lignebudgetlong{Prestige income} minus \lignebudgetlong{from Prestige}. That
is the remaining Prestige income that was not spend for covering daily
expenses.

\aparag[Wealth] Turn wealth is the sum of the Gross income and the Prestige
\VPs. Period wealth is the sum of all turn wealth over all the period.
\bparag Write in \lignebudget{Wealth} the sum of \lignebudget{Gross income A}
and \lignebudget{Prestige VPs}.
\bparag Write in \lignebudget{Period wealth} the sum of \lignebudget{Period
  wealth} of the previous turn and \lignebudget{Wealth} of the current turn.
\bparag Exception: If this is the first turn of a period, simply copy
\lignebudget{Wealth} into {Period wealth}. That is, period wealth is reseted
at each period.

\section{Stability Improvement}\label{chPeace:Stability Improvement}

\label{chPeace:Stability Improvement}
\aparag[Stability] A country may attempt to improve its Stability, but this is
never mandatory. As many actions, Stability improvement requires an investment
and is resolved by a die roll. Beware that in some situations the result may
be negative and cannot be forfeited once the die has been rolled.

\aparag[Investment] Each player wanting to improve the Stability of his
country first chooses an investment and writes it in \lignebudget{Stability
  improvement}. As for administrative actions, higher investments give bonuses
to the roll.
\bparag The investment are:
\begin{modlist}
  \item Basic Investment: 30 \ducats
  \item Medium (\bonus{+2} to the die-roll): 50 \ducats
  \item Strong (\bonus{+5} to the die-roll): 100 \ducats
\end{modlist}

\aparag[Procedure]
This action is resolved without requiring a table.  The player rolls a die
modified as follows (all modifiers are cumulative):
\begin{modlist}
  \item[+?] \ADM monarch.
  \item[+2/5] if medium/strong investment.
  \item[+2] if the country was victim of a declaration of war this turn
    without having broken an alliance or declared a war itself.
  \item[-3] if the country is at war with at least one major country
    (including overseas wars but excluding interventions).
  \item[-2] if the country is at war with at least one minor country and no
    major country (including overseas wars but excluding interventions).
  \item[-5] if an enemy \ARMY counter is in an owned national province and
    controls the city (not applicable during a Religious/Civil War, do not
    count revolt and rebel troops).
  \item[-3] Exception: for \SPA, the malus for having an enemy \ARMY counter
    controling the city, is -3 only, however it applies for any owned
    territory (not only its national territory). This specificity ends with
    \eventref{pIV:Olivares} (if effects are applied), or with
    \eventref{pV:WoSS} (whatever the choices and outcomes).
  \item[+3] for a Prosperous Power (see below).
  \item[-3] for an Anti-Prosperous Power (see below).
  \item[$\pm$?] by event.
\end{modlist}

\begin{designnote}[Spanish empire]
  The early Spanish empire was more of a multicultural empire including both
  Spain, Italy and the Netherlands than a modern country. Hence, occupying
  any part of the empire will hurt some people (and hamper Stability). There
  is no real notion of national territory to defend at all cost opposed to
  more distant vassals and ``colony''. However, only part of the empire is
  shocked by the war, thus the malus is smaller. Olivares policies recentred
  the empire on Spain, making it more like other European powers of the time.
\end{designnote}

\aparag[Result] If the modified result is equal to:
\begin{modlist}
  \item[5-] the Stability \textbf{decreases} by 1.
  \item[6-10] Nothing changes.
  \item[11-14] the Stability increases by 1.
  \item[15-17] the Stability increases by 2.
  \item[18+] the Stability increases by 3.
\end{modlist}
\bparag Reminder: Stability varies from -3 to +3. It is not possible to
decline the result (especially the loss of Stability) once the die has been
rolled.
\bparag Stability is recorded on the Stability track on the \ROTW map. Move
the Stability marker according to the result of the action.

\aparag[Prosperity]\label{chPeace:Prosperity} tracks the evolution of the
Gross income (as recorded in \lignebudget{Gross income A}). A regular increase
of the Gross income will make people happy and ease Stability improvement, a
regular decrease will make people unhappy.
\bparag[Prosperous Power] A country is \terme{Prosperous} if its Gross income
has not decreased during the last 2 consecutive turns and progressed during at
least one of those turns.
\bparag[Anti-Prosperous Power:] A country is \terme{Anti-Prosperous} if its
Gross Income has decreased 2 consecutive turns.

\begin{exemple}{Prosperity}
  If the Gross income for the last two turns and the current one are:
  \begin{itemize}
  \item 100, 110, 120: the country is prosperous.
  \item 100, 100, 101: the country is prosperous (no decrease, at least one
    increase).
  \item 100, 110, 109: nothing (one decrease prevents prosperity even if the
    final result is higher than two turns earlier)
  \item 100, 99, 98: the country is anti-prosperous.
  \item 100, 99, 99: nothing (one stagnation prevents anti-prosperity).
  \end{itemize}

\end{exemple}

\section{Peace offers and discussions}\label{chPeace:Peace offers}
\subsection{Signing Peaces}
Countries at war (either major or minor) may sign peaces. Peaces are usually
done between two alliances and not between single countries. Separate peaces
are possible but usually harder. Peace between major countries (and their
minor allies) are the result of an agreement between players. However, the
Stability of the countries and the military situation creates a \terme{Peace
  differential} and strongly constrain the peace. This represent the overall
opinion of the countries toward the current war and prevent players from
signing unrealistic peaces. Peaces when one side only consist in minor
countries (most of the time, a single one) are resolved by a die roll
depending mostly on the military situation.

\aparag[Global peace] If two alliances are at war, they may sign a global
peace between them.

\aparag[Separate peace between majors] If two alliances are at war, some
powers may sign peace with the whole enemy alliance.
\bparag Powers signing separate peaces are considered as breaking their
alliance (loosing 2\STAB and giving a \CB to former allies as
per~\ref{chDiplo:Alliance:Defensive Alliance}).
% avoid taking 3 provinces to 2 different enemies and stay at war with the
% third
\bparag If several members of the same alliance want to sign a separate peace
with the same enemy alliance at a given turn, they must sign one single
separate peace.

\aparag[Minor allies] always sign peace at the same time as their diplomatic
patron.

\begin{exemple}[Separate peace]
  \TUR is at war against \VEN, \HIS (and \AUS) and \POL. After an incursion in
  the Balkan, \provinceVeneto itself is threatened, thus \VEN would like to
  sign peace before it's too late. On the East side, \RUS is massing troops
  along the Polish frontier and \POL would also like to get out of here in
  order to defend its border. On the other hand, \HIS and \AUS have not
  suffered much and want to stay at war.

  \TUR may choose to accept the separate peace either with \VEN alone, or with
  \POL alone, or with both \VEN and \POL together (treating this as a peace
  with an alliance). In any case, the powers signing the peace (\VEN or \POL)
  are breaking their alliance with \HIS and thus lose 2 \STAB and give a \CB
  to \HIS for the next turn.

  Any minor allies of \VEN or \POL (signing the peace) is also included in the
  peace. Minors allies of \TUR are also part of the peace.
\end{exemple}

\aparag[Proposing separate peace with minor] An alliance may propose separate
peace with minor allies of an opposing alliance at the following conditions:
\bparag An alliance may propose a separate peace to minor in \VASSAL or
\ANNEXION of one enemy if the alliance controls the capital of the minor.
\bparag An alliance may propose a separate peace to minor in \VASSAL or
\ANNEXION of one enemy if the minor controls the capital of one major of the
alliance. In this case, it must be a winning peace (level 1 or more) in favour
of the minor.
\bparag An alliance may propose a separate peace to minor in \VASSAL or
\ANNEXION of one enemy if it has captured the monarch of the minor and choose
to ransom it for a separate peace.
\bparag An alliance may propose a separate peace to minor \textbf{not} in
\VASSAL or \ANNEXION of one enemy if it controls any province of the minor.
\bparag An alliance may propose a separate peace to minor \textbf{not} in
\VASSAL or \ANNEXION of one enemy if the minor controls any province of one
major of the alliance. In this case, it must be a winning peace (level 1 or
more) in favour of the minor.
% should be in each event description to be sure.
% \bparag An alliance may sign a unconditional peace (peace of level 5) in
% favour of any minor that declared war by event.
\bparag Otherwise, each alliance may propose a separate peace to one and only
one minor ally of each opposing alliance.

\aparag[Signing separate peace with minors]
\bparag As all peaces with minors, separate peaces with minors are resolved by
a die roll.
\bparag Contrarily to separate peaces with majors, each separate peace with
minors is resolved independently.
\bparag However, in case of a war against an alliance composed only of minors,
it is not possible to sign a separate peace at the same turn as the global
peace.

\aparag[Mandatory peaces between majors] It is usually not mandatory to sign a
peace, however:
\bparag If a country is a -3\STAB for two consecutive turns at the beginning
of the peace segment, it must \textbf{propose} a peace to all alliances
against which it was at war during these two turns.
\bparag If several members of the same alliance must propose a mandatory
peace, they must propose it together (as usual with separate peaces).
\bparag The peace proposal is made based on the \terme{Peace differential} as
any regular peace. That is, the country is forced to proposed a peace but the
other regular rules for peaces are still enforced. This is not necessarily a
surrender, and in some cases it is even possible to be forced to proposed a
winning peace\ldots
\bparag The opposing alliance is not forced to accept the peace. It the peace
is refused, their is no penalty.
\bparag If two powers at war against one another must propose a mandatory
peace, then the peace must be signed.
\bparag In the rare case where two alliances are at war and one (or more)
member of each alliance must propose a mandatory peace, proceed as follows:
(i) each power first propose a peace to the whole enemy alliance (ii) if both
are refused, then the mandatory peace is signed only between the concerned
powers.

\aparag[Mandatory peace and alliances]
\bparag If a power is forced to propose a peace and that peace is accepted,
that power is not considered to have broken alliance.
\bparag Especially, this does not give a \CB to its former allies.

\aparag[Mandatory peaces with minors]
\bparag If all provinces of a minor are controlled by enemies (not necessarily
the same alliance), then the minor automatically signs a mandatory
unconditional surrender (peace of level 5) with all its enemies together. That
is, this is one global peace and not one surrender against each enemy.
\bparag If an alliance of minors is at war with no major ally, it
automatically accept an unconditional surrender (peace of level 5) in its
favour if the enemy propose it.

\aparag[Tri-partite wars]
\bparag If three (or more) alliance are at war against one another, each peace
signed is only signed between two alliances. The others stay at war.
\bparag It is of course possible that all alliances at war decide to sign
peace at the same moment.

\aparag[Event and peaces]
\bparag Many events create wars with specific conditions with regard to peace,
including:
\begin{modlist}
  \item Specific way to end a war, that is, specific conditions enforcing
    mandatory peaces.
  \item Specific peace conditions that may be taken, in addition to the
    regular one (described below).
  \item Specific peace proposal that will automatically be accepted by some
    minor countries.
\end{modlist}

\aparag[Peace discussions] may last for a long time, especially for big wars
including many countries.
\bparag It is advised to try and minimise the time involved for peace
discussion and keep the negotiations for the Diplomatic phase. However, some
complex transactions and evaluation of the new situation are not uncommon.


%% No possibility to make a white peace at any time.
% \subsection{The Informal Peace}

% \aparag The informal peace is concluded following a mutual agreement between 2
% players or more, and has to be announced to all other players.
% \aparag[Consequences]
% None of the players earns any VP for an informal peace. The war stops
% immediately.
% \bparag This type of peace can comprise any clause for which there was mutual
% agreement between the involved players, in the limit authorized by rules on
% Agreement. The Agreement is applied now.
% \bparag This type of peace can not be concluded with a minor.

\subsection{Peace differential}


\subsection{The Formal Peace}

\aparag A formal peace can be of different types: white-peace, negotiated,
conditional or unconditional.
\bparag It is also the only type of peace authorized with a minor country (or
alliances of minor countries with no major power), but with a different
mechanism.
\bparag Once the principle of such a peace is accepted by the involved
alliances, the type of a formal peace can be negotiated or not and depends on
differential of the involved alliances's Stability.
\aparag[The Peace Differential]
For the detail of peace, we will call "Peace Differential" (PD) the
differential between the Stability values of the 2 alliances at war and
modified by the military situation as described below.  The PD is considered
between the dominant alliance with the highest score (presumed victor) and the
other one with the lowest (presumed loser). The Stability value of an alliance
is that of the Major power in the alliance.
\bparag[Case of Allies]
if more than one Major power is involved in a side when signing a peace, the
Stability used is the average of the allied powers, rounded on the nearest
number (and down if one-half is obtained).
\bparag[Military Situation modifiers]
The Stability of a Major power is modified if it controls more provinces in
the enemy alliance than the ennemies control in itself's territory.  If the
difference is at least 2, the bonus is {\bf +1}, then {\bf +2} if at least 4,
and {\bf +3} is at least 6. Control of a capital city of a Major power counts
as 2 provinces in this comparison; \COL and \TP count for only one-half
province.
\bparag[Military Situation modifiers in Overseas Wars]\label{chPeace:Privateer Effect}
The same comparison is made but controlled \COL and \TP count as full
provinces.  \textit{Privateer Effect:} Moreover, each Commercial fleet with
level 4 or higher that was reduced to 0 or 1 because of Privateers count as
one province for the looting power at this turn (or 2 if in the \CTZ of the
enemy).

\aparag[Peaces that are permitted]
\bparag If the PD is at most +2 in favour of one alliance, a Peace of level 0
(White Peace) or Negotiated Peace of level 1 is permitted in favour of any
alliance (even the one with the lowest modified Stability).
\bparag In any case, a Conditional Peace of level equal to the PD in favour of
the dominant alliance is allowed. The maximum level is 5, even if the Peace
modifier is higher.
\bparag If a Major power (not an alliance) has its capital (or both its
capitals) and half of its national provinces (round up) controlled by enemies,
an Unconditional Peace is allowed against it (but not mandatory); if there is
more than one Major Power in an alliance and the condition is true over one of
them, an Unconditional Peace is possible.

\aparag[Peace constrained] If a Major power is for two consecutive at -3 in
Stability, vis-a-vis the same enemy power or alliance, he is complied to make
a losing White, Conditional or Unconditional peace with this power. The level
of that peace is the worst one that is authorised depending on their Peace
Differential or on the military situations for UP.
\bparag The enemy is not forced to accept this peace. The power is just forced
to \emph{propose} it to its enemy. Exception: if the peace proposed is of
level 4 or 5, then the enemy is forced to accept it (it's basically an
unconditional surrender).
\bparag If both alliances are contrained to sign a peace at the same time, the
Peace is mandatory at the current PD level.
\bparag Under-developped Russia signs a constrained peace only if at -3 in
\STAB during three consecutive turns.
\bparag If the Major powers allied to this power do not want to sign a Peace,
the power makes a mandatory Separate Peace that entails neither loss of \STAB,
nor \CB given to the former allies.

\aparag[Definition of the conditions]
The exact terms of the peace have to be agreed by everyone in the alliances,
following the rules below for the condition. This is true especially for the
number of provinces to be ceded or of indemnities given instead, or if
specific conditions (such that nullifies some events) are asked by the
victorious alliance. If no agreement is possible, the peace will not be
signed.

\aparag[White Peace, or Peace of level 0]
The two alliances have to evacuate all conquests made during the course of
this war and return to the situation of province control existing at the start
of the war, except by express agreement between players.

\aparag[Negotiated or Conditional Peace of level 1]
\bparag The victorious alliance receives one peace condition of its choice
among the following ones.
\bparag[Cession of Territory] One province of the loser's choice is ceded.  It
is selected by the loser following the rule \ruleref{chSpecific:Tranfer
  Provinces Peace}.
\bparag[Indemnities]
The losing alliance gives 50 \ducats of war indemnities to the victorious
alliance.
\bparag[Diplomatic concessions]
One country is removed from one loser's Diplomatic Track and placed back into
the \Neutral box. This can not be a country that is \VASSAL or in \ANNEXION.
If the loser is a \MIN power, one can alternatively chooses to gain it in \MR
status.

\aparag[Conditional Peace of level 2]
\bparag The victorious alliance receives one peace condition of its choice
among the following ones.
\bparag[Cession of Territory] The victorious alliance receives one province of
the its choice. It is selected following the rule \ruleref{chSpecific:Tranfer
  Provinces Peace}.
\bparag[Indemnities]
The losing alliance gives 75 \ducats of war indemnities to the victorious
alliance.
\bparag[Diplomatic concessions]
One country is removed from one loser's Diplomatic Track (any status, excepted
if blocked by other rules of events) and placed back into the \Neutral box.
If the loser is a \MIN power, one can alternatively chooses to gain it in \MR
status.

\aparag[Conditional Peace of level 3]
\bparag The victorious alliance receives two peace conditions of its choice
among the following ones. Each of them can be selected twice.
\bparag[Cession of Territory] The victorious alliance receives one province.
The first province selected is its choice, but the second is of the loser's
choice.  Territories are selected following the rule
\ruleref{chSpecific:Tranfer Provinces Peace}.
\bparag[Indemnities]
The losing alliance gives 75 \ducats of war indemnities to the victorious
alliance.
\bparag[Diplomatic concessions]
One country is removed from one loser's Diplomatic Track (any status, excepted
if blocked by other rules of events) and placed back into the \Neutral box, or
in \MR status of one winner.  If the loser is a \MIN power, one can
alternatively chooses to gain a Diplomatic status, in \MR for one condition,
or \AM for two peace conditions.

\aparag[Conditional Peace of level 4]
\bparag The victorious alliance receives three peace conditions of its choice
among the following ones. Each of them can be selected ore than one.
\bparag[Cession of Territory] The victorious alliance receives one province.
The first and third provinces selected are of its choice, but the second is of
the loser's choice.  Territories are selected following the rule
\ruleref{chSpecific:Tranfer Provinces Peace}.
\bparag[Indemnities]
The losing alliance gives 100 \ducats of war indemnities to the victorious
alliance.
\bparag[Diplomatic concessions]
One country is removed from one loser's Diplomatic Track (any status, excepted
if blocked by other rules of events) and placed back into the \Neutral box, or
in \MR status of one winner.  If the loser is a \MIN power, one can
alternatively chooses to gain a Diplomatic status, in \MR for one condition,
or \AM for two peace conditions, or \VASSAL for three peace conditions (if
status possible).

\aparag[Conditional Peace of level 5 and Unconditional Peace]
\bparag The victorious alliance receives three peace conditions of its choice
among the following ones. Each of them can be selected ore than one.
\bparag[Cession of Territory] The victorious alliance receives one province.
All provinces are the choice of the victorious alliance.  Territories are
selected following the rule \ruleref{chSpecific:Tranfer Provinces Peace}.
\bparag[Indemnities]
The losing alliance gives 150 \ducats of war indemnities to the victorious
alliance.
\bparag[Diplomatic concessions]
One country is removed from one loser's Diplomatic Track (any status, excepted
if blocked by other rules of events) and placed back into the \Neutral box, or
in \MR status of one winner.  If the loser is a \MIN power, one can
alternatively chooses to gain a Diplomatic status, in \MR for one condition,
or \AM for two peace conditions, or \VASSAL or \ANNEXION for three peace
conditions (if status possible).



\subsection{Transfers of Provinces by Peaces}\label{chSpecific:Tranfer
  Provinces Peace}

\aparag Provinces are transferred as an implementation of the peace just
concluded, immediately upon the conclusion of the Peace phase.

\aparag[Choice of Provinces]
The provinces that can be annexed are:
\bparag provinces militarily conquered in the course of the just concluded
conflict are still controlled (except, usually, the capital province);
\bparag or provinces that used to belong to the victor and were annexed by the
loser during a preceding war (including any National provinces of the loser).
\bparag In priority, the National provinces of the victorious alliance are
chosen and annexed.
\bparag Minor Powers that are fully at war are included in the alliance like
any other power, so that they may gain provinces also.

\aparag[Transfer of Colony or Trading Post]
A player can choose a Colony or Trading Post instead of a province, on the
condition that he has effectively occupied this Colony during the course of
just ended conflict, or that the he has or used to have a \COL in the same
\terme{area}.
\bparag This \COL or \TP counts as half a province if controlled at the end of
the conflict.  Only \COL at levels 6 count as a full province.
\bparag If it is not controlled at the end of the the conflict, it may still
be chosen but then counts a full province.

\aparag[Overseas Wars]
A peace treaty ending an Overseas War may not involved change of ownership of
any province on the European mainland. Only transfers of \TP/\COL, % provinces
%in "Barbaresques" countries, \pays{Egypte} and \pays{Irak}, 
or Indemnities are valid conditions of peace.
% \bparag [TBD: pour annexer en Afrique, il faut une guerre normale, guerre
% outremer permet juste les présidios + indemnités + diplomatie] [\textbf{PB:
%   actuellement je suis en faveur de cette restriction}]

\aparag[Transfer of provinces of minor countries]
If a player, at war against an alliance, accepts to make peace, all his minors
do the same, except those at war by event.
\bparag If an alliance receives provinces by a favorable peace, provinces of a
minor ally fully involved in the war, or of a Vassal or Annexed minor power,
may be given to the victor, in place of provinces of the loser
alliance. Howevern these provinces should obey the conditions of rule
\ruleref{chSpecific:Tranfer Provinces Peace}.
\bparag All minor countries provinces ceded by the controlling player count as
his own for victory points.
\bparag If minor cedes provinces by this way, its diplomatic marker is
immediately move to the box \Neutral.



\subsection{Alliances of Major Powers}


\subsubsection{Case of Victorious Allies}
\aparag The number of 3 provinces transferred is a maximum. The sharing of the
spoils of war among allies is left to their free choice.
\aparag In case of disagreement, the power with the Monarch having the best
Diplomacy rating in the alliance (decide at random in case of ties) and whose
country presently has units in the loser's territory, is to take the final
decision of the proposition of Peace. Peace conditions are fixed on at a time
by each power in order of descending Diplomacy rating (so, at most 3 powers
decide).
\aparag However, any ally deprived of any provinces/indemnites when he is
effectively occupying some of the loser's provinces may ask his faulty
ally(ies) to break the alliance (with the loss of 2 \STAB for them) and
immediately receives a \CB against him/them.


\subsubsection{Separate Peace}
\aparag If several players are allied and at war together, some may want to
make separate peace, for instance if the conditions does not please them, or
to stop the war early. A Separate Peace is only allowed if one Major power in
the alliance wants to remain at war (it is not possible to sign Separate
Peaces in order to multiply the Treaties of Peace).
\aparag[Separate Peace and Casus Belli]
In this case, any other Allied player at war on the same side that the one
making a separate peace receives an immediate temporary \CB against this
player.
\bparag This \CB may only be used the turn immediately following the signature
of the separate peace or it is lost.
\bparag The Allied player making a separate peace breaks his military
Alliance, and suffers a loss of 2 levels of Stability.
\bparag Exception: If the Ally must make peace because of his Stability (Peace
Constrained, see above), he does not break his Alliance (and there is no \CB
against him).
\aparag Separate Peace can also be attempted to be signed with Minor
countries, using the mechanism for Peaces with Minor powers. For each alliance
that is at war against it, an alliance may offer a Separate Peace only to one
Minor power in the enemy alliance (and that can not be a Vassal, excepted if
the Capital of the Vassal is controled).



\subsection{General Consequences of the Peace}

\aparag Peace brings the conflict opposing the players to an end.

\aparag In Conditional or Unconditional Peace (not in Negotiated or White
Peace), the loser has not the right to declare a war against the victor for
the whole next turn, excepted if using a new \CB obtained after the Peace
treaty.  Existing \CB at the time of the Peace (even permanent ones) are
negated for both powers for one turn.

\aparag[Peace and Casus Belli]
Permanent \CB are canceled if their cause disappears with the war (e.g.
return of a national Province previously annexed). Temporary \CB obtained
before the end of the war disappears, excepted after a White Peace.

\aparag[Peace and Evacuation]
All units present on the territory of a former enemy are repatriated by their
owner to the closest friendly province of his choice, without having to pay
the cost of activation and rolling for attrition at {\bf -2}.
\bparag Established \Presidios are not given back during Evacuation.
\bparag However, a power may decide at any Peace Phase to dismantle one of its
\Presidios at no cost.

\aparag[Peace and Stability]
Finally, during the conclusion of a peace with a player or an attacking minor
(at war by event), the Stability increases by 1 level for each signatory
player that is now fully at peace.
\bparag In case of peace with multiple enemies, the Stability increase is set
anyway at a maximum of +1 even if peace with different players or minors is
made during the same peace phase.
\bparag A peace with a minor does not increase the player's Stability (unless
this minor has declared war by an event).

\aparag[Indemnities] Indemnites agreed in the Peace are to be paid by one of
the losing Major powers. He may pay them now or at a segment of Announcement
of the two following turns. They can be paid in fractions during the 3 turns
allowed.
\bparag Failure to pay all the due indemnities give a temporary free \CB to
the power that was wronged against the power that should have paid.

\aparag[Peace and Minors]
All active minors of players now at peace immediately sign the peace at the
same time.
\bparag
Exception: a minor at war declared by an event makes peace only through a
specific peace for this war against the ennemy alliance, usinf the procedure
for minor countries.  This is not considered as a separate peace (even if
involved in war alongside its controlling player).



\subsection{Peace with Minor powers}

\aparag To make the peace with a minor, the enemy player has to indicate that
he commits himself to peace negotiations with the minor.  The controlling
player of the \MIN may be at war (against the player) or not; in the former
case it is a "separate peace", attempted and signed before any peace between
players.
\bparag An alliance may usually offer peace with only one minor country per
alliance at war against it and per turn; and this minor may not be a vassal ir
annexed.
\bparag Exception: An alliance can always propose a specific peace to a minor
at war declared by an event .
\bparag Exception: An alliance can offer separate peace to any or all minor
vassal to the alliance or annexed, whose capital city is controls.
\bparag Exception: An alliance can offer separate peace to any or all
non-vassal, and non-annexed minors if controlling at least one of their
provinces (each one), or if each one of them controls individually at least
one province in the alliance.
\bparag If allied powers do not agree, the decision is taken by the Power
having the higher value in DIP (decide ties randomly).

\aparag[Method]
To sign peace with the minor, the player rolls 1d10, taking into account the
sought-after peace level and all applicable modifiers. The level of peace is
chosen by the player, without taking into account his own Stability.
\bparag[Result]
The peace is signed if the modified die-roll result is 6 or more.
\bparag[Peace level Modifier]
This modifier is the triple of the value of the peace level chosen. It is used
as a positive modifier if the offered peace is favorable to the minor, or as a
negative modifier if the offered peace is favorable to the player.
\bparag[Nationality Modifier]
It is applied for the case of conflicts with specific minor countries
which are: \\
-4: \pays{Perse}, \pays{egypte}, \pays{damas}, \pays{Chine},
\pays{Japon} \\
-3: \pays{USA}, \pays{Mogol}, \pays{Venise}, \pays{Pologne},
\pays{Habsbourg},  \pays{Brandebourg} after IV-11\\
-2: \pays{Portugal}, \pays{danemark}
\bparag[Modifiers of Situation]
These are is applied cumulatively according to what happened during the
current turn (and to what happened during previous turns for the
lost/conquered provinces only):\\
+2:	per province/\TP\faceplus/\COL lost by the minor (+4 if capital)\\
-2:	per province/\TP\faceplus/\COL conquered by the minor (-4 if capital) \\
+1:	per \TP\facemoins  lost by the minor (round down) \\
-1:	per \TP\facemoins conquered by the minor \\
+2: if the capital province of the \MIN was conquered this turn, or if it was
captured then lost since \\
-4: if the \MIN has captured a capital province of a \MAJ this turn, -or if it
was
captured then lost since \\
-2:	per major battle won by the minor  \\
+2:	per major battle won by the player \\
-1:	per battle won by the minor \\
+1:	per battler won by the player\\
+1:	per minor military leader killed or captured \\
+2: if the Monarch of the minor country is captured and its Ransom is
used for Peace \\
-1:	per military leader of the player killed or captured\\
+1:	per siege won by the player \\
-1:	per siege won by the minor \\
-2: If minor is heretic (Catholic vs. Protestant, before the end of
the \terme{Religious Dissension}) \\
-2: if it is an attempt to negotiate a separate peace $\pm$?: the PD of the
controller regarding the alliance attempting the peace (if the controller is
at war against the alliance), max. -3/+3
\bparag
All these modifiers are cumulative in one single turn.
\bparag \COL and \TP\faceplus controlled counts as a full province (modifier
of $\pm 2$); the same is true for control of cities of minor countries in the
\ROTW;
\bparag \TP\facemoins controlled gives a modifier of $\pm 1$ only; the same
modifier holds for occupying a province without city of a minor country in the
\ROTW;
%% was \pm 1, remis à 1.5 pour être comme les tables ou la liste de modifieurs
%% ci-dessus...
\aparag[Overseas Wars]
\bparag A Minor country always accepts to sign a White Peace in Overseas War
(if it is not a Separate Peace).

\aparag[Consequences of Peace]
If the peace is signed, no Stability level is gained (exception: if this minor
declared war to the player by an event). The player that controlled the minor
does not earn anything.
\bparag The conditions of Peace are the same as for a Peace between Major
powers.
\bparag A Minor country will at most indemnities up to 4 times its income,
immediately at the conclusion of the Peace. Any other indemnities are void.
\bparag If the minor country is the victor, the player that controls the minor
country chooses the ceded provinces (if any). He must do so in priority among
those located the closest from the minor country's territory, in terms of
movement points (a sea zone is equivalent to 2 MP for this calculation).
\bparag A minor country nevers takes Diplomatic concessions, only provinces
and indemnities.

\aparag[Multiple and Separate Peace]
If a player signs a separate peace with a minor country (he is still at war
against the controlling player); this minor may not be again involved in a war
against him next turn (unless by an event or a Crusade).
\bparag A minor country at war by event may only make a separate peace.

\aparag[Unconditional Peace]
A Minor country will sign a mandatory Unconditional Peace if all of its
provinces are controlled by the ennemy. This peace is one global peace against
all the powers controlling its provinces (so it can lose only 3 provinces).
\bparag If a minor country loses and signs an unconditional peace its
political marker is moved automatically to the box \Neutral.

\aparag[Automatic Peace]
An alliance that proposes a victorious Unconditional Peace to a Minor power,
and that Minor power was the attacking one (caused by event), or is not allied
to an alliance (so this is not a Separate Peace), the Peace is automatically
accepted by the Minor power.

\aparag[Failure of Peace negotiations]
If the die-roll is inferior or equal to 5 the war with the minor continues
normally for the following turn. Another peace attempt with that minor will be
allowed during the peace phase of next turn.

\section{Test for crusade}\label{chPeace:Crusade}

% Local Variables:
% fill-column: 78
% coding: utf-8-unix
% mode-require-final-newline: t
% mode: flyspell
% ispell-local-dictionary: "british"
% End:
