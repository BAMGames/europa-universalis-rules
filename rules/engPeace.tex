% -*- mode: LaTeX; -*-

\definechapterbackground{Peaces}{victories}
\chapter{Peaces}\label{chapter:Peace}

\section{Overview of the phase}

% RaW: [50]
\aparag[Peace] Wars can be ended only by a Peace. There are several types of
Peace, from the white peace (return to statu quo) to the unconditional
surrender. The type depends mostly on the difference between the \STAB of the
belligerents, slightly modified by the military situation. In some cases,
countries must propose peace to their opponents, but usually some discussion
occurs between the players.

\aparag[Crusade] In the early game, if \TUR conquers too many Christian
provinces, the pope may try to launch a Crusade.

\aparag[Sequence.]
\PeaceDetails

\section{Ransoms}\label{chPeace:Ransoms}
\aparag[Majors] If a \MAJ has its monarch (or Swedish heir) captured (due to
battle), it \textbf{must} pay a ransom. The monarch is immediately liberated.
\bparag The ransomed country loses 2 \STAB and pay 200\ducats to the ransoming
country.
\bparag If the monarch was captured by a minor country, the money is lost
(it is payed but nobody gains it).
\bparag It is not possible to avoid ransom in any way. No keeping prisoners,
no execution, \ldots even if both the ransoming and ransomed players agree.
\bparag Thus, ransom may cause a later bankrupt or an immediate mandatory
peace. Do not risk your monarch if you cannot afford the price.

\aparag[Minors] If a \MIN has its monarch captured by a major country, he
\textbf{must} be ransomed.
\bparag The major holding the prisoners chooses one (and only one) ransom
among:
\begin{modlist}
\item 50\ducats.
\item[OR] \bonus{+2} to a peace proposal.
\item[OR] possibility to do a separate peace proposal.
\end{modlist}
\bparag If a minor monarch is captured by another minor country, he is
automatically ransomed for free (some money transfer between minors, not
represented).

\aparag Money gained or lost due to ransoms is written in \lignebudget{Ransom,
  peace}.

\aparag[The return of the king]
\bparag Ransomed monarchs will be available again during the next Interphase.
\bparag Especially, Ransomed major monarchs may not use their values for the
rest of the Peace phase.

\section{Peace offers and discussions}\label{chPeace:Peace offers}
\subsection{Signing Peaces}
Countries at war (either major or minor) may sign peaces. Peaces are usually
done between two alliances and not between single countries (each alliance may
contain one or more country). Separate peaces are possible but usually
harder. Peace between major countries (and their minor allies) are the result
of an agreement between players. However, the Stability of the countries and
the military situation creates a \terme{Peace Differential} and strongly
constrains the peace. This represents the overall opinion of the countries
toward the current war and prevents players from signing unrealistic
peaces. Peaces when one side consists only of minor countries (most of the
time, a single one) are resolved by a die roll depending mostly on the
military situation.

\subsubsection{Regular cases}
\aparag[Global peace] If two alliances are at war, they may sign a global
peace between them.

\aparag[Separate peace between majors] If two alliances are at war, some
powers may sign peace with the whole enemy alliance.
\bparag Powers signing separate peaces are considered as breaking their
alliance (loosing 2 \STAB and giving a \CB to former allies as
per~\ref{chDiplo:Alliance:Defensive Alliance}).
% avoid taking 3 provinces to 2 different enemies and stay at war with the
% third
\bparag If several members of the same alliance want to sign a separate peace
with the same enemy alliance at a given turn, they must sign one single
separate peace.
\bparag Note that this also prevent signing a separate peace with one member
of the alliance and, at the same turn, a global peace with the rest of the
alliance. All members of the alliance who want to sign peace (in this case,
everybody) must do so together.

\aparag[Minor allies] usually sign peace when their diplomatic patron does.
\bparag However, the diplomatic patron may choose to do a separate peace
without some of its minor allies. In this case, the major loses 2 \STAB for
the separate peace and the diplomatic control of the minors staying at war.

\begin{exemple}[Separate peace]
  \TUR is at war against \VEN, \HIS (and \AUS) and \POL. After an incursion in
  Hungary, \provinceVeneto itself is threatened, thus \VEN would like to sign
  peace before it's too late. On the East side, \RUS is massing troops along
  the Polish frontier and \POL would also like to get out of here in order to
  defend its border. On the other hand, \HIS and \AUS have not suffered much
  and want to stay at war.

  \TUR may choose to accept the separate peace either with \VEN alone, or with
  \POL alone, or with both \VEN and \POL together (treating this as a peace
  with an alliance). In any case, the powers signing the peace (\VEN or \POL)
  are breaking their alliance with allies staying at war (\HIS) and thus lose
  2 \STAB and give a \CB to these allies for the next turn.

  Any minor allies of \VEN or \POL (signing the peace) is also included in the
  peace. Minors allies of \TUR are also part of the peace.
\end{exemple}

\begin{designnote}
  It is not possible to sign several separate peace (in one war) in a single
  turn, and it is not possible to sign both separate and global peace on the
  same turn, in order to limit the number of peace conditions that may be
  exchanged each turn.
\end{designnote}

\aparag[Proposing separate peace with minor]\label{chPeace:Separate peace minor}
An alliance may propose separate peace with minor allies of an opposing
alliance at the following conditions:
\bparag An alliance may propose a separate peace to any minor in \VASSAL or
\ANNEXION of one enemy if the alliance controls the capital of the minor.
\bparag An alliance may propose a separate peace to any minor in \VASSAL or
\ANNEXION of one enemy if the minor controls the capital of one major of the
alliance. In this case, it must be a winning peace (level 1 or more) in favour
of the minor.
\bparag An alliance may propose a separate peace to any minor of one enemy if
it has captured the monarch of the minor and chooses to ransom it for a
separate peace.
\bparag An alliance may propose a separate peace to any minor \textbf{not} in
\VASSAL or \ANNEXION of one enemy if it controls any province of the minor.
\bparag An alliance may propose a separate peace to any minor \textbf{not} in
\VASSAL or \ANNEXION of one enemy if the minor controls any province of one
major of the alliance. In this case, it must be a winning peace (level 1 or
more) in favour of the minor.
% should be in each event description to be sure.
% \bparag An alliance may sign a unconditional peace (peace of level 5) in
% favour of any minor that declared war by event.
\bparag In addition, each alliance may propose a separate peace to one and
only one minor ally of each opposing alliance, \textbf{not} in \VASSAL or
\ANNEXION.

\aparag[Signing separate peace with minors]
\bparag As all peaces with minors, separate peaces with minors are resolved by
a die roll.
\bparag Contrarily to separate peaces with majors, each separate peace with
minors is resolved independently.
\bparag However, it is not possible to sign a separate peace at the same turn
as the global peace.

\begin{exemple}[Separate peaces with minors]
  \TUR, allied to \paysMaroc and \paysTripoli, with \VASSAL\ \paysAlgerie and
  \paysTunisie is at war against \HIS, allied to \paysVenise with \VASSAL\
  \paysChevaliers. \HIS controls \provinceJebelTubqal (in \paysMaroc),
  \provinceOran (in \paysAlgerie) and \provinceIfriqiya (capital of
  \paysTunisie). \TUR does not control any Christian provinces.

  \TUR may not propose peace to \paysChevaliers as it is a \VASSAL. It may
  propose peace to \paysVenise.

  \HIS may propose peace to \paysMaroc because it controls one of its
  provinces. \HIS may propose a peace to \paysTunisie, even through it is a
  \VASSAL, because it controls its capital. \HIS may not propose peace to
  \paysAlgerie because it is a \VASSAL and even if it controls one province,
  it does not controls the capital. It may, in addition, propose peace to
  \paysTripoli as each alliance is always entitled to one separate peace with
  one enemy minor at no condition.

  Thus, \HIS may propose up to three separate peaces with minors. If it does,
  each of these peaces is resolved separately.
\end{exemple}

\subsubsection{Mandatory peaces}\label{chPeace:Mandatory peaces}
\aparag[Mandatory peaces between majors] It is usually not mandatory to sign a
peace, however:
\bparag If a country is at -3 \STAB for two consecutive turns at the beginning
of the peace offer segment, it must \textbf{propose} a peace to each alliance
(containing at least one major) against which it was at war during these two
turns. Note that the check happens \textbf{after} \STAB improvement, thus
mandatory peace usually occur because of a failed improvement (or a ransom).
\bparag Exception: \RUS, before its military reform, is only forced to propose
peace if it is at -3 \STAB for 3 consecutive turns.
\bparag The opposing alliance is not forced to accept the peace. It the peace
is refused, there is no penalty.
\bparag Exception: if the level of the proposed peace (see below) is 4 or 5 in
favour of the enemy, then the enemy is forced to accept it (this is basically
an unconditional surrender). In this case (only), the winning alliance chooses
the nature of the conditions for the peace.
\bparag If two powers at war against one another must both propose a mandatory
peace, then the peace must be signed.
\bparag The peace proposal is made based on the \terme{Peace Differential} as
any regular peace. That is, the country is forced to proposed a peace but the
other regular rules for peaces are still enforced. This is not necessarily a
surrender, and in some cases it is even possible to be forced to proposed a
winning peace\ldots

\aparag[Mandatory peace and alliances]
\bparag Since this condition is checked for each country (and not for each
alliance), it may be a separate peace proposal (with only some members of the
alliance forced to propose peace).
\bparag If several members of the same alliance must propose a mandatory
peace, they must propose it together (as usual with separate peaces).
\bparag If a power is forced to propose a peace and that peace is accepted,
that power is not considered to have broken alliance.
\bparag Especially, this does not give a \CB to its former allies.

\aparag[Mandatory peaces and global peaces] Note that if a global peace is
signed, no separate peace may be signed first. Thus, mandatory peace proposals
only happen if the global peace is not signed.

\aparag[Mandatory peaces with minors]\label{chPeace:Mandatory peace minors}
\bparag If all provinces of a minor are controlled by enemies (not necessarily
the same alliance), then the minor automatically signs a mandatory
unconditional surrender (peace of level 5) with all its enemies together. That
is, this is one global peace and not one surrender against each enemy.
\bparag It is not possible to refuse that peace. In case of disagreement
between the winners, they are considered allied for the resolution of the
peace only.
\bparag If the minor was at war allied to a major, it immediately goes to
\Neutral before resolving the peace (the minor consider that its patron should
have protected it).
\bparag If an alliance of minors is at war with no major ally, it
automatically accepts an unconditional surrender (peace of level 5) in its
favour if any enemy proposes it. See~\ref{chPeace:automatic peace minor} for
more on this.

\subsubsection{Other specific cases}
\aparag[Tri-partite wars]
\bparag If three (or more) alliances are at war against one another, each
peace signed is only signed between two alliances. The others stay at war.
\bparag It is of course possible that all alliances at war decide to sign
peace at the same moment.

\aparag[Events and peaces] Many events create wars with specific conditions
with regard to peace, including:
\bparag Specific way to end a war, that is, specific conditions enforcing
mandatory peaces.
\bparag Specific peace conditions that may be taken, in addition to the
regular ones.
\bparag Specific peace proposal that will automatically be accepted by some
minor countries.

\aparag[Disagreements]
\bparag If members of an alliance do not agree toward signing a peace, all
decisions concerning the proposal and acceptation of the peace are taken by
the country whose monarch has the higher \DIP (resolve ties at random) among
those (of that alliance) involved in the proposal (that is, you have nothing
to say about a separate peace made by your ally, except threatening it of
later reprisals, but threats have no in-game effect).
\bparag Note that effectively, the monarch with higher \DIP takes all the
decisions alone and is in no way forced to listen to his allies (however, do
not complain that nobody wants you as an ally if you keep ignoring them).
\bparag Only countries that are fully at war are considered. That is,
countries in limited or foreign intervention may not impose their will to
their allies and have a purely consultative say in the peace discussion.

\begin{exemple}[Disagreements]
  \FRA and \SPA are at war against \HOL and \ANG. \FRA and  \HOL both have
  higher \DIP than their ally.
  \begin{itemize}
  \item If \HOL wants to sign a global peace (\emph{e.g.} because
    \villeAmsterdam is besieged) while \ANG wants to stay at war (because it
    think situation in the \ROTW will become better), \HOL may impose its
    decision to \ANG and sign the peace. \HOL may also, obviously, decide to
    sign a separate peace.
  \item If \HOL proposes a separate peace that \FRA wants to accept but \HIS
    would like to refuse, \FRA may impose its decision.
  \item If \ANG wants to sign a separate peace, \HOL has nothing to say about
    it and may not force it to stay at war.
  \end{itemize}
\end{exemple}

\aparag[Timing for the insanes]
Separate peaces between two alliances are considered simultaneous. Especially,
a power signing a separate peace with an enemy alliance is still allowed to
discuss any separate peace proposal from this alliance. Peace agreement may be
global (as in ``I sign this separate peace only is this one is only
signed''). Remember that in case of disagreement, the countries stay at war
and that's all.
\bparag Precise peace timing:
\begin{enumerate}
\item Global peace proposals and discussions between majors. All proposals and
  agreement are simultaneous and it is not possible to wait for a peace before
  signing another.
\item Separate peace proposals and discussions between majors, including
  mandatory separate peaces. All proposals and agreements are simultaneous.
\item Peace with minors, including separate peaces with minor allies. All
  proposals are simultaneous before any die is rolled.
\end{enumerate}

\begin{exemple}[Disagreements (continued)]
  \begin{itemize}
  \item If both \HOL and \HIS want to sign a separate peace with their
    enemies, that \FRA and \HIS are ready to accept the Dutch peace but \ANG
    would like to stay at war against \HIS, then \HOL is still part of the
    peace discussion and may force \ANG to accept the Spanish peace at the
    same time that it itself sign peace with \FRA.
  \item In the same situation, \HIS may decide that its separate peace is
    valid if and only if the Dutch peace is accepted. Typically if \ANG and
    \HOL try to buy \HIS out of the war by offering it an advantageous peace,
    \HIS may link it to the peace with \HOL in order to avoid leaving \FRA
    alone against two enemies.
  \end{itemize}
\end{exemple}

\aparag[Cultural agreement]
\bparag Peace agreements may include promises for future actions or agreements
on future Diplomatic phases.
\bparag It is, however, not possible to immediately sign any agreement (loan,
dynastic alliance, military alliance, \ldots) Hence, it is always possible to
``forget'' about these between the signature of the peace and the next
Diplomatic phase. Again, do not complain that nobody loves you if you keep
forgetting your agreements (Europa Universalis is a long term game and treason
is often a bad strategy).
\bparag Such promises do not have to be publicly announced and may be kept
secret between players (even from allies). Thus, they are often jokingly
referred as ``cultural agreements'' as they have no in-game effect (only a
promise between players). Players sometimes get out of a secret discussion
announcing they are signing peace with ``transfer of one province and some
cultural agreements''\ldots

\begin{playtip}[Peace discussions]
  Peace discussions may last for a long time, especially for big wars
  including many countries. It is advised to try and minimise the time
  involved for peace discussions and keep the negotiations for the Diplomatic
  phase. However, evaluation of the new situation is required and some complex
  transactions are not uncommon (nor unrealistic given what historically
  happened during the time frame of the game). Discussions should be kept
  focused on the current peace and not diverge toward long term agreements
  (these are best suited for the Diplomatic phase).

  Players may isolate themselves from other players in order to discuss
  peaces. Either allies wanting to prepare a common proposal or enemies
  wanting to discuss secret clauses without third party players interfering in
  the discussion. Private discussions do not need to include all members of a
  given alliance\ldots As a rule of thumb, peace discussions between enemies
  is faster if there are no other players listening and commenting the
  proposals, trying to influence it. However, do not hesitate to ask advice
  from other players to check if some proposal is as balanced as it
  seems. Especially, inexperimented players may have hard time to grasp all
  the consequences of some agreements and may want to consult an experimented
  neutral player\ldots
\end{playtip}

%% No possibility to make a white peace at will.
% \subsection{The Informal Peace}

% \aparag The informal peace is concluded following a mutual agreement between 2
% players or more, and has to be announced to all other players.
% \aparag[Consequences]
% None of the players earns any VP for an informal peace. The war stops
% immediately.
% \bparag This type of peace can comprise any clause for which there was mutual
% agreement between the involved players, in the limit authorized by rules on
% Agreement. The Agreement is applied now.
% \bparag This type of peace can not be concluded with a minor.

\subsection{Interventions}
\aparag Countries in limited or foreign intervention in a war that goes on may
choose to either continue the intervention or withdraw.
\bparag The choice is made for each intervention separately (each country in
decreasing order of initiative indicates for each intervention whether it
stays or withdraws).
\bparag The choice is made by the country doing the intervention. The
initiative taken into account forthe order being the one of the country (not
the one of the alliance).

\aparag Continuing a limited intervention will cost some \STAB. Continuing a
foreign intervention costs nothing but prevent reinforcing the
stack. See~\ref{chPeace:Stability}.
\bparag Withdrawing requires evacuation of the intervening units as
per~\ref{chPeace:Evacuation}.

\subsection{Peace differential}
\aparag[The \terme{Peace Differential}] is an abstract way of determining the
winner of any war between majors. It is mostly based on the \STAB of the
countries involved, representing the people support for the war, slightly
modified by the military situation.
\bparag In case of separate peace, the \terme{Peace Differential} is computed
only between the countries involved in the proposal.
\bparag \terme{Peace Differential} strongly constrains the possibility of
peace.

\aparag[The basic Peace Differential] is the difference between the \STAB of
the enemies.
\bparag In case of alliance, take the average \STAB of all members of the
alliance. \textbf{Do not round} numbers at this point.
\bparag Note that the basic PD is symmetrical, that is if an alliance has a
basic PD of +1.5 versus another alliance, then the second alliance has a basic
PD of -1.5 versus the first.

\aparag[The modified Peace Differential] is obtained from the basic PD by
checking the military situation.
\bparag The alliance that controls more enemies provinces adds (and the other
subtracts) to its basic PD:
\begin{modlist}
  \item[+1] if it controls 2 or 3 more provinces.
  \item[+2] if it controls 4 or 5 more provinces.
  \item[+3] if it controls at least 6 more provinces.
\end{modlist}
\bparag Count capitals as 2 provinces.
\bparag Do count provinces of minor allies (or provinces controlled by minor
allies) together with those of its diplomatic patron.
\bparag Count \COL and \TP as \undemi\ province. Exception: \COL of level
6 are considered as European provinces and count as a full province.

\begin{exemple}[Modified Peace Differential]
  \RUS is at war against allied \TUR and \SUE. The \STAB are 1 for \RUS, 0 for
  \SUE and 1 for \TUR. Thus, the basic PD is 0.5 (1 - (1+0)/2) in favour of
  \RUS.

  \RUS occupy Swedish \provinceNeva and \provinceKarelen but \TUR occupy both
  \provinceAstragan and \provinceTerek (annexed by \RUS a long time
  ago). Both side thus controls as many enemy provinces and the PD is not
  modified.
\end{exemple}

\begin{exemple}[PD and separate peaces]
  In the same situation, if \RUS wants to sign a separate peace with \SUE,
  then its basic PD is 1 (1-0, the \STAB of \TUR does not count). Since this
  peace is only with \SUE, provinces controlled by \TUR are not taken into
  account. \RUS controls 2 more provinces than \SUE, and the PD in its favour
  is increased by 1 to 2.

  On the other hand, if \RUS wants to sign a separate peace with \TUR, the
  basic PD is 0 (they both have 1 \STAB) modified to -1 as \TUR controls two
  more provinces.
\end{exemple}

\begin{exemple}[PD and minors]
  If \paysCrimee was at war allied to \TUR and \RUS controls \provinceCrimee,
  this province has to be taken into account for modified peace differential
  in any peace that include \TUR. Since it is a capital, it counts as 2
  provinces. Thus, the modified PD of \RUS against the alliance is now +1.5,
  and against \TUR (in case of separate peace), 0.
\end{exemple}

\aparag[Military situation in overseas war]\label{chPeace:Privateer Effect}
\bparag During overseas wars, count occupied \COL and \TP as one province
each.
\bparag[Privateer effect] In addition, each \TradeFLEET\Faceplus which was
reduced to current level 0 or 1 counts as 1 province (2 in the country own
\CTZ).
\bparag Do count all \CTZ/\STZ where \TradeFLEET have been reduced without
remembering who caused the losses.

\begin{designnote}[Privateer effect]
  is triggered even if the losses were caused by \pays{pirates} or a third
  party \corsaire (typically, one of \Barbaresques), which may seem
  illogical. However, \corsaire are only a partial and abstract representation
  of the actual privateer activity. It is assumed that the real activity is
  more widespread, including in zones where no counter was send. Moreover, the
  target country probably doesn't know for sure who attacked each of its
  merchants. Or doesn't make a real difference between pirates and enemy
  privateers\ldots
\end{designnote}

\begin{exemple}[Privateer effect]
  \FRA and \ANG are entangled in a commercial war. A \TradeFLEET of \ANG of
  level 6 in \ctz{Angleterre} was reduced to current level 0 due to attacks by
  \leaderBart. Another \TradeFLEET of level 4 in \stz{Amerique} was
  reduced to level 1 due to combined attack of a \pays{pirates} \corsaire and
  a French \corsaire. A third \TradeFLEET of level 5 was reduced to level 1 in
  \stz{Lion}. Meanwhile, \ANG manages to take a \COL of level 4 of \FRA in
  \granderegionQuebec as well as a \TP in \continentIndia. This counts as 4
  provinces occupied by \FRA and 2 by \ANG, thus a +1 to PD in favour of \FRA.
\end{exemple}

\aparag[The net Peace Differential] is obtained by rounding the modified PD to
the nearest integer. In case of halves, round down in disfavour of the winning
side (that is, round toward 0). Then cap to \bonus{+5} (and \bonus{-5}) if
needed.
\bparag Note that fractions in the PD may only come from the \STAB
difference. However, the military situation may change the winner, thus the
direction of the final rounding.
\bparag The net peace differential is also symmetrical. Thus, it is always
sufficient to compute the PD from the point of view of one of the alliances.

\begin{exemple}[Rounding PD]
  \SUE, \POL and \TUR are at war against \RUS. The \STAB of \RUS and \SUE is
  1, while the \STAB of \POL and \TUR is 0. No side controls enemy
  provinces. Thus, the basic (and the modified) PD is 1 - (1+0+0)/3 = \td\ in
  favour of \RUS, rounded to +1 in favour of \RUS.

  \smallskip

  \SUE and \TUR are at war against \RUS. The \STAB of \RUS is 0, the \STAB of
  \TUR is 1 and the \STAB of \SUE is 2. Thus, the basic PD is +1.5 in favour
  of the alliance (or -1.5 in ``favour'' of \RUS). If the military situation
  does not modify this, it is rounded to +1 in favour of the alliance.

  \smallskip

  \SUE and \TUR are at war against \RUS. The \STAB of \RUS and \TUR are 1
  while the \STAB of \SUE is 0. Thus, the basic PD is +0.5 in favour of
  \RUS. However, the alliance controls four Russian provinces while \RUS
  controls no enemy province. Thus, the PD is modified by 2 in favour of the
  alliance, for a result of +1.5, rounded down to +1 in favour of the
  alliance. Note that if rounding had occured before modification, the PD
  would have been rounded to 0 and then modified to +2 in favour of the
  alliance. Hence, it is important not to round at the wrong time.
\end{exemple}

\subsection{The Peace levels}
The \terme{peace level} represents in an abstract way the amount of
``winning'' the winner has. It varies between 0 (white peace) and 5
(unconditional surrender). The peace level is strongly constrained by the
\terme{Peace Differential}. In turn, the peace level indicate how many
\terme{conditions} the loser has to give to the winner.

\subsubsection{Peace levels and conditions}
\aparag[Peaces that are permitted]
\bparag In any case, a \terme{Conditional Peace} of level equal to the PD in
favour of the dominant alliance is
allowed. % The maximum level is 5, even if the Peace
% Differential is higher.
\bparag If the \terme{Peace Differential} is at most +2 in favour of one
alliance, a \terme{Negotiated Peace} of level 0 (White Peace) or 1 is
permitted in favour of any alliance (even the one with the lowest modified PD,
that is, the apparent loser).
\bparag Exception: if a power if forced to proposed a Mandatory peace (as
in~\ref{chPeace:Mandatory peaces}), it must propose a Conditional Peace and
may not propose a Negotiated one.
\bparag If at least one Major member of an alliance has its capital (or both
if it has two) and at least half of its other national provinces controlled by
enemies (not necessarily allied), then a Conditional Peace of level 5 is
allowed against that alliance.

\aparag[Peace conditions] The level of the peace determine both the number of
conditions that the losing alliance must give to the winning one and some
details on these conditions, as described below.
\bparag Only countries that are fully at war may give or take peace
conditions. That is, countries in limited or foreign intervention do not risk
to lose anything at peace time, but they may not either have any
gain. Obviously, there may be some promises to be fulfilled at a later
Diplomatic phase, but as always promises are not binding.
\bparag There are 4 types of conditions that may be given at peace:
\begin{modlist}
\item[Territorial concessions:] The losing alliance gives ownership of one
  province to the winning alliance. See~\ref{chPeace:Transfer Provinces Peace}
  to know which province may be annexed by who. The province may belong to any
  member of the loosing alliance (including minor allies). The province can be
  given to any member of the winning alliance (including minor allies). The
  choice of the province is made either by the losing or winning alliance,
  depending on the level of the peace.
\item[Indemnities:] The losing alliance must give some money to the winning
  alliance. The money must come from the \RT of one or more majors of the
  loosing alliance (minor allies may not pay the indemnities) and can be given
  to one or more members of the winning alliance (minor allies may receive the
  indemnities). The amount is written in \lignebudget{Ransom, peace} of the
  concerned countries (negative for the losers, positive for the winners). If
  the losing alliance is composed solely of minor countries, they may pay
  indemnities.

  The losing alliance always choose who pay, while the winning alliance always
  choose who gets the money.
\item[Diplomatic concessions:] Either of the choice below. The precise choice
  is only decided when implementing the condition and is always made by the
  winning alliance. The minor involved must not necessarily be part of the war
  to be chosen (drastic changes of alliances and distant weddings were not
  uncommon). The minor involved may however not be at war elsewhere (it may be
  part of the just finishing war).
  \begin{itemize}
  \item (Europe) The loosing alliance must give diplomatic control of one of
    its European minor allies to the winning alliance. If the loosing alliance
    is solely composed of minors, then the winning alliance may gain
    diplomatic control of one of them.
  \item (\ROTW) One \ROTW minor breaks its diplomatic status with some member
    of the losing alliance and may increase its status with some member of the
    winning alliance.
  \end{itemize}
\item[Special conditions:] Events and other specific rules sometimes create
  specific concessions that may (or must) be used as peace conditions for some
  wars. Sometimes, a minimum level of the peace is required in order to ask
  for this concession. Sometimes, a concession is automatically added to other
  peace conditions as soon as the peace reaches a certain level.
\end{modlist}

\aparag[Terms of the peace]
When a peace is agreed between majors, the terms must specify both the level
of the peace and the nature of the conditions. For example, two countries may
sign ``a peace of level 3 with one territorial concession first and then one
diplomatic concession''.
\bparag Once the peace is agreed, players may choose the precise conditions
(which province to annex, who is going to pay the indemnities, \ldots)
\bparag The order of the concessions is important only in case of disagreement
between players.

\aparag[Deciding details]\label{chPeace:Implementing conditions}
In each alliance, the country whose monarch has the higher \DIP has all power
to decide which peace to sign.
\bparag However, for the precise choice of the conditions, the choice is made
in decreasing order of \DIP in each alliance. That is, the monarch with higher
\DIP chooses the first condition, the second one chooses the second, and so
one (looping back to the monarch with higher \DIP if needed).
\bparag Not that choices are made sometime by the losing alliance and sometime
by the winning one. The choice order is followed by each alliance separately.
\bparag For Territorial concessions only, the alliance who choose depends on
the level of the peace and the number of territorial concessions (only). That
is, if the only territorial concession is the second condition of the peace,
it is still the first territorial concession.

\begin{exemple}[Disagreement]
  \HIS and \HOL are losing a war against \FRA, allied to \paysPortugal. \HOL
  has higher \DIP than \HIS. The peace differential is 4, so the only peace
  that may be signed is of level 4, hence three conditions. After some
  discussions, \FRA and \HOL agree on indemnities as first conditions and then
  two territorial concessions after that. Note that having lower \DIP, \HIS
  may take part in the discussion but in the end, the decision is made by
  \HOL, however, if \HOL wanted to stay at war, \HIS could have signed a
  separate peace.

  Since \HOL has the higher \DIP, it chooses how to implement the losing side
  of the first condition and decides that \HIS is going to pay all the
  indemnities (they could have been split in any way between the
  losers). Being the only Major, \FRA chooses who receive the money. Even if
  it could have given some to \paysPortugal, it prefers to keep all of
  it\ldots (note that as part of the discussion, it could have been agreed
  that this money goes to \paysPortugal (and is lost) but this would have been
  only a verbal non binding agreement and in the end \FRA decides who gets the
  money). Even if the first territorial concession is the second condition, it
  is the first territorial one, hence chosen by the winning alliance (for a
  peace of level 4). \FRA chooses to annex a Spanish province. Lastly, the
  second territorial concession is implemented. It is chosen by the losing
  alliance. Since \HOL already has its turn in choosing a condition (for the
  indemnities) and \HIS did not has its, \HIS chooses and decides to give a
  Dutch \COL to \paysPortugal.
\end{exemple}

\begin{playtip}
  Note that letting an unwilling ally support all the weight of the peace is
  probably not a good long term strategy if you still need allies for future
  wars. Usually, the precise implementation of the peace conditions is agreed
  upon between players before signing the peace. The precise order of choice
  is rarely needed (but sometimes when two countries fight over a single
  province, the difference between a level 1 and level 2 peace can be pretty
  big).

  \smallskip

  A lot of rules are written to handle disagreements because we need how to
  resolve the situation for the rare cases where players really have divergent
  opinions. In most cases, the players go out of the negotiation room already
  knowing all the details of the peace, and the high \DIP country discuss with
  its allies rather than imposing a peace.

  Especially, \emph{Europa Universalis} is a (very) long term
  game. Backstabbing people during peace negotiation is probably not a good
  long term strategy. You may do it sometime when you really have a huge gain
  or a big opposing goal with someone, but be careful.
\end{playtip}

\aparag[Disagreement] Any power in the winning alliance who is currently
controlling at least one province of the losing alliance and does not receive
a full peace condition may denounce the peace (receiving some money but less
than the value of indemnities for the peace level is the same as receiving
nothing from this point of view: it is not a full peace condition).
\bparag In this case, all the majors of the winning alliance that received at
least one full peace condition immediately break their alliance with all the
powers denouncing the peace.
\bparag As usual, powers breaking alliance lose 2 \STAB and give a \CB to
their former allies.
\bparag Powers breaking alliance that way stay allied together. Powers
denouncing the peace stay allied together.
\bparag Powers that neither denounce the peace nor received a full peace
condition must immediately chose either to denounce the peace or to accept it.
\bparag If they accept the peace, they are breaking their alliance with the
power denouncing it (and stay allied with the others), at usual cost.
\bparag If they denounce the peace, they stay allied with the other powers
denouncing it.
\bparag Note that the powers breaking the alliance are the ones that did
receive something (or stick with them), not the ones that choose to denounce
the peace. That is, the powers denouncing the peace have been wronged and ask
their former allies for compensations, and it is the refusal of giving such
compensations (unrepresented in game) which causes the breaking of the
alliance.
\bparag Note also that denouncing the peace does not automatically create a
war between the former allies. It only breaks the alliance and give a \CB to
some of them.

\subsubsection{Description of peace levels}
\aparag[Peace of level 0 (White peace)] No conditions are given or taken.
% The two alliances have to evacuate all conquests made during the
% course of this war and return to the situation of province control existing at
% the start of the war, except by express agreement between players.

\aparag[Peace of level 1] The winning alliance receives one peace condition.
\bparag[Territorial concession] The province is selected by the losing
alliance.
\bparag[Indemnities] The losing alliance gives 50 \ducats of war indemnities
to the winning alliance.
\bparag[European Diplomatic concession] One European country, neither in
\VASSAL nor \ANNEXION, is removed from one loser's Diplomatic Track and placed
back into the \Neutral box. If the losing alliance is composed solely of \MIN
powers, the winning alliance may gain one of them in \MR status.
\bparag[\ROTW Diplomatic concession] One \ROTW country breaks \dipFR status
with one member of the losing alliance. If the losing alliance is composed
solely of \MIN, then one of them is forced to sign a \dipFR with one member of
the winning alliance.

\aparag[Peace of level 2] The winning alliance receives one peace
condition.
\bparag[Territorial concession] The province is selected by the winning
alliance.
\bparag[Indemnities] The losing alliance gives 75 \ducats of war indemnities
to the winning alliance.
\bparag[European Diplomatic concession] One country (any status, excepted if
blocked by other rules of events) is removed from one loser's Diplomatic Track
and placed back into the \Neutral box. If the losing alliance is composed
solely of \MIN powers, the winning alliance may gain one of them in \MR
status.
\bparag[\ROTW Diplomatic concession] One \ROTW country decreases one level
(from \dipAT to \dipFR or from \dipFR to neutral) with one member of the losing
alliance. If the losing alliance is composed solely of \MIN, then one of them
is forced to sign a \dipFR with one member of the winning alliance.

\aparag[Peace of level 3] The winning alliance receives two peace conditions.
\bparag[Territorial concession] The first territorial concession is chosen by
the winning alliance, the second (if there are two) is chosen by the losing
alliance.
\bparag[Indemnities] The losing alliance gives 75 \ducats of war indemnities
to the winning alliance.
\bparag[European Diplomatic concession] One country (any status, excepted if
blocked by other rules of events) is removed from one loser's Diplomatic Track
and placed back into the \Neutral box, or in \MR status of one winner. If the
losing alliance is composed solely of \MIN powers, the winning alliance may
gain one of them in \MR status for one peace condition or in \AM for two peace
conditions.
\bparag[\ROTW Diplomatic concession] One \ROTW country either breaks \dipAT
with one member of the losing alliance or both breaks \dipFR with one member
of the losing alliance and signs \dipFR with one member of the winning
alliance. If the losing alliance is composed solely of \MIN, then one of them
is forced to sign a \dipFR with one member of the winning alliance, or a
\dipAT for two conditions.

\aparag[Peace of level 4] The winning alliance receives three peace
conditions.
\bparag[Territorial concession] The first and third territorial concessions
are chosen by the winning alliance. The second one is chosen by the losing
alliance.
\bparag[Indemnities] The losing alliance gives 100 \ducats of war indemnities
to the winning alliance.
\bparag[European Diplomatic concession] One country (any status, excepted if
blocked by other rules of events) is removed from one loser's Diplomatic Track
and placed back into the \Neutral box, or in \MR status of one winner. If the
losing alliance is composed solely of \MIN powers, the winning alliance may
gain one of them in \MR status for one peace condition or in \AM for two
peace, or in either \EG or \VASSAL (if this status is possible) for three
peace conditions.
\bparag[\ROTW Diplomatic concession] One \ROTW country breaks any status
(\dipFR or \dipAT) with one member of the losing alliance and signs \dipFR
with one member of the winning alliance. If the losing alliance is composed
solely of \MIN, then one of them is forced to sign a \dipFR with one member of
the winning alliance, or a \dipAT for two conditions.

\aparag[Peace of level 5 (Unconditional Peace)] The winning alliance receives
three peace conditions.
\bparag[Territorial concession] All provinces are chosen by the winning
alliance.
\bparag[Indemnities] The losing alliance gives 150 \ducats of war indemnities
to the winning alliance.
\bparag[European Diplomatic concession] One country (any status, excepted if
blocked by other rules of events) is removed from one loser's Diplomatic Track
and placed back into the \Neutral box, or in \MR status of one winner. If the
losing alliance is composed solely of \MIN powers, the winning alliance may
gain one of them in \MR status for one peace condition or in \AM for two
peace, or in either \EG or \VASSAL or \ANNEXION (if these status are possible)
for three peace conditions.
\bparag[\ROTW Diplomatic concession] One \ROTW country breaks status with one
member of the losing alliance and signs \dipFR with one member of the winning
alliance. If the losing alliance is composed solely of \MIN, then one of them
is forced to sign a \dipFR with one member of the winning alliance, or an
\dipAT for two conditions.

\aparag[Indemnities] Note that the amount given for indemnities is the amount
\emph{per condition}. That is, if a peace of level 5 is signed with three
indemnities as the three conditions, the total amount is 3 $\times$ 150 =
450\ducats !

\subsection{Transfers of Provinces by Peaces}\label{chPeace:Transfer
  Provinces Peace}
\aparag If a peace includes territorial concessions, some provinces owned by
the loosing alliance (including minors) immediately change ownership and now
belong to one member of the winning alliance (possibly a minor).
\bparag Not all powers may annex all provinces. If there is not enough
provinces to annex in order to fulfil all the territorial concessions, the
peace may not be signed under these terms. That is, some other conditions must
be chosen rather than territorial ones.

\aparag[Choice of Provinces] The provinces that may (or may not) be annexed
are:
\bparag Capitals may never be annexed unless explicitly specified elsewhere.
\bparag Any power may annex provinces it controls at the time of the peace.
\bparag Any power may annex any of its national provinces, whoever controls it
(even if still controlled by the enemy alliance).
\bparag Any power may annex any province it previously owned during the game,
whoever controls it (even if still controlled by the enemy alliance).
\bparag Any power may annex any province with its blurred shield in it,
whoever controls it (even if still controlled by the enemy alliance).
\bparag Any power may annex a \TP or \COL (including of level 6) if it was
controlled during some point of the war by any member of its alliance.
\bparag Any power may annex a \TP or \COL (including of level 6) if it owned
an establishment in the same \Area at some point during the game.
\bparag Exception: if a province, \TP or \COL is currently controlled by a
third party power (not member of any of the alliances signing peace), it may
only be annexed if the controlling power agrees. In that case, the controlling
power must evacuate the province as per~\ref{chPeace:Evacuation}.

\aparag[Priority] If any national province of the winning alliance is
currently owned by any member of the losing alliance and controlled by a
member of the winning alliance, it must be chosen as territorial concession
(if the peace includes some territorial concession).
\bparag If several exists, the choice is made by the power choosing how to
implement the condition (\ref{chPeace:Implementing conditions}).
\bparag Note that this priority does not prevent any other peace condition
(indemnities, diplomatic concessions, \ldots) to be obtained at peace instead
of territorial concessions.
\bparag Note that provinces with blurred shield are (usually) not national
provinces and thus don't have priority.

\begin{exemple}[Allowed peace conditions]
  \SUE and \RUS are at war. \SUE owns \provinceNeva (a Russian national
  province) from a previous war. They decide to sign a peace of level 1
  favouring \RUS.
  \begin{itemize}
  \item If \RUS controls both \provinceNeva and \provinceKarelen, and the
    belligerent agree on a territorial concession, then \SUE must choose to
    give \provinceNeva as occupied national provinces have priority.
  \item Whether \RUS controls \provinceNeva or not, they may agree on
    Indemnities or a Diplomatic concession as the sole peace condition. The
    national province does not prevent other conditions but simply constrains
    territorial concessions.
  \item If \RUS controls \provinceKarelen but not \provinceNeva and they agree
    on a territorial concession, \SUE can choose to give either \provinceNeva
    or \provinceKarelen to \RUS. Non occupied national provinces are eligible
    as territorial concessions but have no priority.
  \item If \RUS does not control \provinceKarelen, then it cannot annex it at
    peace (whatever the level of the peace). Non-national provinces must be
    occupied to be annexed.
  \end{itemize}
\end{exemple}

\aparag[Transfer of Colony or Trading Post]
\bparag One territorial concession (whatever the level of the peace) allows to
annex two \COL or \TP if (i) both are controlled by the winning alliance at
the end of the war and (ii) none of them is a \COL of level 6.
\bparag \COL of level 6 or establishments that are not controlled at the end
of the war are annexed for a full condition each.
\bparag The two establishments may be annexed by different winners and from
different losers.
\bparag The power choosing how to implement the peace condition does chose
both establishments and their new owners.

\aparag[Overseas Wars]
A peace treaty ending an Overseas War may not involved change of ownership of
any province on the European map, \province{Islas Canarias} or \province{Cabo
  Verde}.
\bparag Note that \COL of level 6 may still be annexed and that the ``two for
one'' rule above still applies.

\aparag[Transfer of provinces of minor countries]
\bparag Minors signing peace at the same time as their Diplomatic patron are
involved in the peace as any power and may thus cede or annex provinces.
\bparag For this purpose, provinces with a non-blurred shield, as well as
provinces formerly owned by the minor, count as ``national provinces'' of the
minor (especially for the priority of annexation rule).
% (Jym):
\bparag Additionally, provinces of \regionBalkans are considered as national
provinces of \paysVenise.
% Replaces:
% (Pierre) major allies of minor Venise must give Moree to it even if it was
% never owned by \VEN. This represents the result of the Peloponese war
% (Morosini) where the \AUS-Venise alliance wins and Venise annex Moree.
\bparag Provinces gained or lost by minors count as if gained or lost by their
Diplomatic patrons for \VPs.
\bparag If the losing alliance chooses to give a province of a minor when it
may have chosen a province of a major from the European map, this minor goes
to Neutral after the peace is signed.
% Only in case of choice made by the patron (and allies). Thus avoid \RUS
% taking 1 province from Crimea (only) and having it drop from \TUR track as a
% free bonus.
% Possible \COL/\TP annexation are ignored as this probably would create too
% strange stuff.

\begin{designnote}
  The last case only occurs when the minor thinks that its patron ``sold'' its
  territory. Especially, it is not triggered if the annexed province is chosen
  by the winners (then  the loosing alliance could not have done it better),
  nor if the priority of annexation forces the loser to give a province of the
  minor (\emph{e.g.} the winner do not occupy other provinces, or the minor
  owns a national province of a winner, \ldots)
\end{designnote}

\subsection{Peace with Minor powers}\label{chPeace:Peace with Minors}
\aparag Peaces with minor powers are handled by a die roll.

\aparag[Global peace] An alliance at war against an alliance composed solely
of minors (often a single major against a single minor) may propose peace to
the whole alliance of minors.
\bparag As usual, minors allied to majors in the proposing alliance are
included in the peace treaty.

\aparag[Separate peace] An alliance at war may propose a separate peace to
some minors allied to an opposing alliance.
\bparag Check~\ref{chPeace:Separate peace minor} to see at which conditions an
alliance may propose a separate peace to minor enemies (1 per alliance per
turn + specific situations).
\bparag Note that this include proposing separate peace to members of an
alliance composed solely of minors.
\bparag Contrary to separate peace with majors, each separate peace with minor
is signed with a single minor. However, a war may not end the same turn one or
more separate peace with minors involved in it is attempted.

\begin{exemple}[Separate and global peace]
  At turn 7, \FRA is at war against \HIS and \paysSavoie. \FRA is crushing
  \paysSavoie but is loosing on the Spanish side of the war. Thus, the global
  peace differential would only allow a white peace. However, \FRA would like
  to sign a favourable separate peace with \paysSavoie in order to annex
  \provinceBresse, even if this forces an unfavourable peace with \HIS.

  If \FRA attempts to sign a separate peace with \paysSavoie at turn 7, it may
  not, at the same turn, sign a global peace with \HIS (whatever the result of
  the attempted peace with \paysSavoie). \FRA may (attempt to) sign a peace
  with \paysSavoie at turn 7, stay at war and sign a global peace with \HIS
  (and \paysSavoie if the separate peace failed) at turn 8. Obviously, there
  is a risk in doing so that \HIS overruns \FRA during this extra turn of
  war\ldots
\end{exemple}

\aparag[Disagreement] As usual in case of disagreement inside an alliance, any
decision on which peace to (try to) sign is made by the country whose monarch
has the higher \DIP.

\aparag[Method]
\bparag The alliance proposing peace choose the level of the peace, between -5
(unconditional peace favouring the proposing alliance) and +5 (unconditional
peace favouring the target alliance) (that is, the level of the peace is seen
from the minor's perspective) as well as the nature of the conditions.
\bparag Unless this is a special case of automatic peace
(see~\ref{chPeace:automatic peace minor}), the proposing alliance rolls a die,
modified as in~\ref{chPeace:peace minors modifiers}.
\bparag If the result is 6 or more, the peace is signed.
\bparag If the result is less than 6, the peace is not signed and the
countries stay at war.
\bparag Note that \STAB of the majors is not taken into account. The military
situation, however, plays a huge role in the DRM.


\aparag[Peace modifiers]\label{chPeace:peace minors modifiers}
The DRMs to the peace roll are all cumulative.
\bparag[Nature of the Peace] These modifiers take into account the level of
the peace as well as the nature of the conditions given or taken.
\begin{modlist}
\item[$\pm$ 3] per level of the peace (positive if the target alliance wins
  the war, negative if the proposing alliance wins);
\item[+ ?] per peace condition given to the target alliance (depending on the
  nature of the condition);
\item[- ?] per peace condition taken from the target alliance (depending on
  the nature of the condition).
\end{modlist}
\bparag The modifiers for the nature of the conditions are:
\begin{modlist}
\item[+1] per territorial condition;
\item[-1] per indemnities;
\item[0] per diplomatic condition;
\item[$\pm$ ?] per specific condition (\bonus{0} if not specified).
\end{modlist}
See example below for details on these modifiers.
\bparag[Nationality Modifier] is applied when signing peace with specific
minors. In case of peace with an alliance of minors, apply the sum of the
nationality modifiers of all the minors in the alliance.
\begin{modlist}
\item[-4] peace with either \paysPerse, \paysEgypte, \paysDamas, \paysChine or
  \paysJapon;
\item[-3] peace with either \paysUSA, \paysMogol, \paysVenise, \paysPologne,
  \paysHabsbourg or, after \ref{pIV:Great Elector}, \paysBrandebourg;
\item[-2] peace with \paysPortugal or \paysDanemark.
\end{modlist}
\bparag[Modifiers for military Situation]~
\begin{modlist}
% \item[+4] per capital owned by the minor and controlled by the major;
% \item[-4] per capital owned by the major and controlled by the minor;
\item[+2] per province, \TP\faceplus, \COL or city in the \ROTW owned by the
  target alliance and controlled by the proposing alliance;
\item[-2] per province, \TP\faceplus, \COL or city in the \ROTW owned by the
  proposing alliance and controlled by the target alliance;
\item[+1.5] per \TP\facemoins owned by the target alliance and controlled by
  the proposing alliance;
\item[-1.5] per \TP\facemoins owned by the proposing alliance and controlled
  by the target alliance;
\item[+2] per capital province of the target alliance that was conquered at
  any point of the war (even if liberated since);
\item[-2] per capital province of the proposing alliance that was conquered at
  any point of the war (even if liberated since);
\item[-2] If at least one member of the target alliance is heretic toward at
  least one member of the proposing alliance (Catholic vs. Protestant, before
  the end of the \terme{Religious Dissension}).
\end{modlist}
\bparag[Modifiers for military action] These modifiers are only valid the turn
they happen and are reseted at each turn.
\begin{modlist}
\item[+2] per major battle won by the proposing alliance;
\item[-2] per major battle won by the target alliance;
\item[+1] per battle won by the proposing alliance;
\item[-1] per battle won by the target alliance;
\item[+1] per military leader of the target alliance (including Monarchs)
  killed or captured;
\item[+2] per Monarch of the target alliance captured and whose Ransom is used
  for Peace modifier;
\item[-1] per military leader of the proposing alliance killed or captured;
\item[+1] per siege won by the proposing alliance (\bonus{+2} if this is a
  capital);
\item[-1] per siege won by the target alliance (\bonus{-2} if this is a
  capital).
\end{modlist}
\bparag[Separate peace]~
\begin{modlist}
\item[-2] if it is an attempt to negotiate a separate peace;
\item[$\pm$?] the peace differential of the proposing alliance versus the
  alliance of the minor (maximum, \bonus{-3}/\bonus{+3}).
\end{modlist}

\aparag[Automatic peaces]\label{chPeace:automatic peace minor}
\bparag If all provinces of a minor are controlled by enemies it automatically
signs a mandatory unconditional surrender with all its enemies together.
See~\ref{chPeace:Mandatory peace minors}
\bparag If an alliance of minors is at war with no major ally, it
automatically accepts an unconditional surrender (peace of level 5) in its
favour if any enemy proposes it. In this case (only), the controllers of the
minors select the nature of the conditions and must choose territorial or
specific conditions if possible (and indemnities last), resolving disagreement
as if they were allied. Note that it is still possible to attempt a regular
peace of level +5 (favouring the minor) and rolling die.
\bparag Minor countries always accept to sign a global White Peace in Overseas
War. Note that the major may choose not to propose the peace and stay at war,
or roll to try and get a winning peace.

\aparag[Consequences of Peace]
\bparag The conditions of Peace are the same as for a Peace between Major
powers.
\bparag A Minor country will at most pay indemnities up to 4 times its income
(total), immediately before the conclusion of the Peace (before changing
ownership of provinces) but disregarding enemy control, \REVOLT or
\PILLAGE. Any other indemnities are void but are still a valid peace
condition.
\bparag The controller of the minor makes all decisions concerning peace if
needed (normally, only the choice of annexed provinces, depending on the level
of the peace and the number of territorial concessions). If the controller has
to choose which provinces to annex, he must choose provinces adjacent to the
minor's territory if possible.
\bparag A minor country nevers takes Diplomatic concessions, only provinces,
indemnities and special conditions. It may, however, give Diplomatic
concessions.

\begin{designnote}[Fair play]
  When making decision on behalf of a minor, players should always take the
  minor's interest into account. Neutral minors do have a controller only as a
  game artefact because it's impossible to play them otherwise. However, the
  controller should not take advantage of this to poorly play the minor, and
  especially not as an asset to be negotiated (\emph{i.e.} ``I choose to have
  the minor annex a province you don't need for your victory objectives if you
  give me 50\ducats.'' is definitely not the way this game is intended to be
  played\ldots)

  As much as possible, we tried to have neutral minors controlled by powers
  who should have interest in letting the minor do well (or at least no
  interest in having it do poorly). But this is not always the case due to
  circumstances. Don't abuse your position as controller of a minor's
  country. Play it for the best interest of the minor, not yours. Don't
  hesitate to ask advice to other players on the decisions the minor should
  make if you have doubts.

  Minors who have a Diplomatic patron are another matter. They are basically
  part of one country's empire and more or less obey the orders\ldots
\end{designnote}

\aparag[Multiple and Separate Peace]
If one or more major sign a separate peace with a minor country (and stay at
war against the controlling country); this minor may not be again involved in
a war against these majors next turn (unless by an event).

\begin{exemple}[Nature of the peace]
  Proposing to a minor a peace of level -1 (the minor loses) with territorial
  concessions (\emph{i.e.} annexing one province of the minor), creates a
  modifier of 3 \textmultiply (-1) (level of the peace) - (+1) (territorial
  concession) = -4. With indemnities, the modifier would be -3 - (-1) =
  -2. Minors are more eager to sign peace giving money than territory (one
  needs to roll high to obtain peace).

  Proposing a peace of level +3 (the minor wins) with two indemnities (for a
  total of 150\ducats) creates a modifier of 3 \textmultiply (+3) (level of,
  the peace) + (-1) + (-1) (two indemnities) = +7. With two territorial
  concessions, the modifier would be +9 + (+1) + (+1) = +11. It's easier to
  sign peace if you give territory than money.
\end{exemple}

\begin{exemple}[Military situation]
  In 1563 (turn 15), \ref{pIII:War Sweden Denmark} erupts and \paysDanemark
  attacks \SUE.

  During turn 15, \SUE wins two naval battles (Bornholm and second \"{O}land)
  but loses one (first \"{O}land) where its admiral is captured (in game,
  killed) and loses one land battle (Mared), while trying to invade
  \provinceSkane. Meanwhile, the Swedish Northern army manages to take control
  of \provinceTrondelag. Thus, the situation modifier is +2 (one province of
  the minor is occupied) while the action modifier is +2 (two battles won by
  \SUE) -2 (two battles won by \paysDanemark) -1 (one Swedish leader killed)
  +1 (one successful siege) = 0. Since there is a -2 nationality modifier for
  \paysDanemark and an additional -2 due to the Danish claim on the Swedish
  crown (\ref{chSpecific:Sweden:Denmark}), the global modifier is -4. Not
  wanting to sign a white or losing peace, \SUE decides to stay at war.

  \smallskip

  During turn 16, \SUE wins three naval battles (R\"{u}gen, Bukow and another
  time near Bornholm), killing the Danish admiral once and on land both win
  once (the Danes at Axtorna, the Swedes at Brobacka). \provinceTrondelag
  stays in Swedish hands and no other siege succeed but the Danish general
  \leaderRantzau is nonetheless killed in an unsuccessful assault (at
  Varberg). Thus, the situation modifier is still +2. The action modifier,
  however, is now +4 (four Swedish victories) -1 (one Danish victory) +2 (two
  Danish leaders killed) = +5. Note that the actions of turn 15 are now
  forgotten and only what happened recently is taken into account. With the
  nationality modifier and the special modifier, this results in a +1
  favouring \SUE.

  Being also involved in the Livonian war, \SUE wants to sign peace. Being in
  a not too bad situation (especially with \leaderRantzau dead), \SUE wants to
  attempt a winning peace. Depending on the peace, the modifier will be:
  \begin{itemize}
  \item Peace of level +1 (Danish victory), giving indemnities (no province
    may be given as \paysDanemark controls none): +1 (situation, action and
    nationality modifiers) + 3 \textmultiply (+1) (level of the peace) + (-1)
    (giving indeminties) = +3 and will have 80\% chances of success (3 or
    more).
  \item Peace of level 0 (back to \emph{status quo}): +1 + 3 \textmultiply 0
    = +1 (60\% success).
  \item Peace of level -1 (Swedish victory), annexing a province (either
    \provinceTrondelag or one of the three national provinces of \SUE owned by
    \paysDanemark): +1 + 3 \textmultiply (-1) (level of the peace) - (+1)
    (taking one territorial concession) = -3 (20\% success only, and the
    special modifier stays for future peaces).
  \item Peace of level -1, taking indemnities: +1 + 3 \textmultiply (-1) -
    (-1) (taking indemnities) = -1 (40\% success, but very little gain with
    only 50\ducats.)
  \item Peace of level -1, asking \paysDanemark to abandon claims on the crown
    (special peace condition provided by~\ref{chSpecific:Sweden:Denmark}): +1
    + 3 \textmultiply (-1) - 0 (no modifier for this condition) = -2 (30\%
    success and future peaces will be much easier).
  \item Peace of level -2, asking heavier indemnities: +1 + 3 \textmultiply
    (-2) - (-1) = -6 (impossible). In this not really decisive situation, it
    is simply impossible to ask for a peace of level 2.
  \end{itemize}

  \SUE decides to ask \paysDanemark to abandon its claims on the Swedish
  crown. Diplomats of both countries meet at Stettin where \SUE rolls 9-2=7,
  this is 6 or more, hence successful. \paysDanemark agrees to sign the treaty
  of Stettin.
\end{exemple}

\begin{exemple}[Separate peace]
  At turn 7, \FRA is at war against \HIS and \paysSavoie. \FRA controls
  \provinceSavoia (from a previous turn) and \provinceBresse (from this turn)
  but \HIS controls \provinceArtois (annexed earlier by \FRA) and
  \provinceLanguedoc. \paysSavoie was not involved in any battles (its last
  troops heroically defended \provinceNice against a french siege). The \STAB
  of \FRA is +0 and the \STAB of \HIS is +1.

  The Peace Differential between the alliances, seen from the French side, is
  -1 (\STAB differential) not modified (\FRA controls 3 provinces, the capital
  counting as two, but \HIS controls 2, so there is only 1 extra province, not
  enough to modify the PD). \HIS is not willing to sign an unfavourable
  negotiated peace (level 1 favouring \FRA) and \FRA also refuse the
  conditional peace of level 1 (favouring \HIS). However \FRA wants to try and
  get \paysSavoie out of the war in order to free its occupation troops and
  repulse the Spaniards.  The situation modifier is +4 (two provinces
  occupied, the Spanish occupation does not count for a peace with
  \paysSavoie) +2 (capital was conquered) = +6. The action modifier is +1 (one
  siege) and there is a separate peace modifier of -2 (Separate peace) -1
  (Peace differential versus the alliance of the minor) = -3. Thus, the global
  modifier is +4. \FRA is almost guaranteed to have a white peace (90\%
  success) and may even try to annex \provinceBresse (50\% success).
\end{exemple}

\begin{designnote}[Peace differential]
  If the minor is on the winning side of the war, it does not want to betray
  its allies to get out (or only with heavy spoils), thus the PD acts
  negatively in the peace roll. On the other hand, if the minor is on the
  loosing side of the war, it may want to try and cut its loses before being
  involved in a dramatic peace, thus the PD acts positively. Always take the
  PD as seen from the alliance proposing peace. Always consider the PD versus
  the whole alliance of the minor (not only its controller, this is
  \textbf{not} a separate peace with the controller).
\end{designnote}

\begin{exemple}[Max indemnities]
  \FRA is at war against the lone \paysLorraine and controls
  \provinceLorraine. Since it is the only province of the minor, it
  automatically accepts an unconditional surrender leaving \FRA with three
  conditions to choose. Since its only province is a capital, \FRA may not
  annex it. \FRA choose to take 3 indemnities. Since it is a peace of level 5,
  each indemnity is worth 150\ducats for a total of 450\ducats. However, the
  income of \paysLorraine is only 7\ducats, thus the maximum indemnities it
  may pay is 7 \textmultiply 4 = 28\ducats total. The 422 other \ducats are
  lost (but still represent a valid peace condition, that is \FRA may sign the
  peace for 28\ducats).

  Remark: in this situation, it would probably be wiser for \FRA to ask for
  diplomatic concessions, representing in this case the long-term French
  occupation of \paysLorraine in the 16th and 17th centuries.
\end{exemple}

\subsection{General Consequences of the Peace}
\aparag Peace brings the conflict opposing the belligerent countries to an
end.
\bparag Unless involved in another war, the countries are now considered at
peace for all game purposes.

\aparag[Resolving peaces] Peace conditions must be transferred immediately
upon signing the peace.
\bparag Provinces given as territorial concession change ownership. Mark with
the correct ownership counters. If there is a fortress in the province, the
new owner may immediately replace it with one of its fortress of the same or
lower level or destroy it (special European arsenals may be replaced by a
fortress of the same level or an European arsenal if one is allowed here).
\bparag \COL and \TP given as territorial concessions also change
ownership. Replace the counter by a counter of the same nature of the new
owner. Level and exploited resources stay the same, update the corresponding
record sheets. Any fortress or arsenal may be replaced by a counter of the
same nature and same or lower level of the new owner (a fortress may also
replace an arsenal, an arsenal may not replace a fortress), or destroyed.
\bparag If the new owner does not have available counters, it may immediately
destroy (or reduce levels of) existing ones as needed.
\bparag If not enough fortress, \COL or \TP counters are available (\COL and
\TP limit is usually smaller than counter mix), the owner may destroy one of
its existing one. If not enough ownership counters are available, make new or
use whatever mean you wish to denote ownership.
\bparag Any minor given as diplomatic concession changes patron. Place its
diplomatic counter at the right position on the diplomatic track.
\bparag Indemnities must be payed immediately in full, even if this leads to a
future bankrupt.
\bparag Other specific conditions are also implemented immediately, marking
any changes as possible.

\aparag[Returning control] Remove any control markers of countries signing
peace that is located inside a country signing peace with it. Control of these
provinces is returned to their rightful owner.
\bparag Any fortress of a country signing peace located in a province owned by
another country signing peace with it may be immediately replaced by a counter
of the same or lower level of the owner, or destroyed. Owner of the province
chooses.
\bparag Exception: \Presidios are kept. They do not change ownership and are
not removed.

\aparag[Peace Evacuation]\label{chPeace:Evacuation}
Any land unit in a not-owned province must evacuate unless the owner of the
unit is either at war or intervention with (ally) or against (enemy) the owner
of the province.
\bparag Evacuating units must move to owned and controlled territory.
\bparag Evacuating units may move through any country that was part of any
just ended war with (ally) or against (enemy) them, including former enemies,
regardless of the presence of any unit (even those that were not part of a
just ended war, \emph{e.g.} third party units involved in another war).
\bparag Exception: they may not enter a province with an unbesieged enemy unit
or fortress (from another war).
\bparag Evacuating units may not, however, move through provinces of countries
that were not part of the war.
\bparag Evacuating units may move by sea, even if there is no fleet to
transport them.
\bparag Evacuating units have unlimited movement capacity (\emph{i.e.} they
are not limited to 12\MP). Evacuation is not necessarily done toward the
closest province, however, evacuating units may not move more than 12\MP if
they can evacuate in 12\MP or less.
\bparag Evacuating units may not be intercepted.
\bparag Evacuating units roll for attrition as usual, with a \bonus{-2} to the
roll, and considering all provinces as friendly. Ignore any bad weather. Each
full set of 6\MP expanded is one extra cause of attrition.

\aparag[Evacuation at sea] Naval units of a country signing at least one peace
may either evacuate to any owned and controlled port or stay at sea.
\bparag If they return to port, they must roll for attrition with a bonus of
\bonus{-2}.
\bparag However, if they stay at sea, they do not need to roll for attrition.

\begin{designnote}[Control and evacuation]
  Returning control and evacuating only happens between former belligerents
  (including allies). If a country is involved in another war, it does not
  have to return control and evacuate from this war (if it is still going
  on).

  When, evacuating, you must also evacuate from your ally, except if your are
  still together fighting in another war.

  If ending an intervention, you must also evacuate any unit that was part of
  the intervention.

  Land unit in non-controlled provinces of countries at war are handled by
  \ref{chRedep:Redeployment}.
\end{designnote}

\aparag[Evacuation and redeployment] If any stack is out of supply after
evacuation (this may happen because of separate peace), it may chose to also
evacuate or stay where it is.
\bparag If it evacuate, it does not get the \bonus{-2} to attrition roll. In
addition it is considered to have entered at least one enemy province (the one
where it starts its evacuation).

\aparag[Memento]
\bparag At this point of the turn, land units should be either:
\begin{modlist}
\item In a controlled province.
\item[OR] In a controlled \Presidio.
\item[OR] Besieging a province where they could maintain siege.
\item[OR] In a province controlled by a member of the same alliance, together
  at war (or intervention).
\item[OR] Besieged in a fortress.
\item[OR] In the \ROTW, in a province without any establishment.
\end{modlist}
\bparag Any other land unit must either have redeploy or evacuate [or I did
overlook a tricky special case].

\begin{exemple}[Returning control and Evacuation]
  \FRA and \HIS sign peace and are now fully at peace. After any change in
  ownership of provinces due to the treaty, \FRA must give back control of any
  province owned by \HIS it currently controls and reciprocally \HIS must give
  back control of French provinces. Next, any French troop in Spanish province
  must evacuate to a French province. During evacuation, it may cross any
  French or Spanish province and ignore the presence of any other unit
  (stacking limit is still enforced at the end of evacuation). Similarly,
  Spanish troops must evacuate French territory.
\end{exemple}

\begin{exemple}[Neutral provinces]
  \FRA and \HOL sign peace and are now fully at peace. When evacuating Dutch
  provinces, French troops may not cross Spanish provinces (typically in the
  Spanish Netherlands) because \HIS was not part of the just ended war.

  \smallskip

  \FRA is involved in two separate wars, one with \HIS and the other with
  \HOL. Both end the same turn. Now French unit evacuating Dutch territory may
  cross Spanish provinces (it would be too complicated and not that much
  realistic to try and track which troops were in which war).
\end{exemple}

\begin{exemple}[Neutral and enemy troops]
  \FRA is at war against \HIS and \HOL (separately) and \HIS is also at war
  against \HOL. \FRA signs peace with \HOL but stays at war against \HIS. \HIS
  stays at war against \HOL and currently control or besiege several Dutch
  provinces.

  \FRA must evacuate from Dutch territory as it is now at peace with
  \HOL. \FRA do not need to evacuate from Spanish territory as it is still at
  war against \HIS. When evacuating from Dutch territory, French troops may
  not cross provinces owned by \HIS (they are not part of a just ended war)
  nor Dutch provinces with Spanish control or siege (military presence of an
  ongoing enemy).

  If, on the other hand, \FRA signs peace with both \HIS and \HOL, then its
  troops evacuating from Dutch territory may move through Spanish units
  (ignore any non-enemy unit). Even if \FRA was not at war against \HIS (but
  only against \HOL), its evacuating troops may move through Dutch province
  controlled by Spanish troops.
\end{exemple}

\begin{exemple}[Alliance]
  \FRA is at war against allied \HIS and \HOL, they sign peace and are now
  fully at peace. \HOL must not only evacuate from French provinces but also
  from Spanish ones (not-owned, not at war allied with). If the war goes on,
  however, Dutch troops may stay in Spanish or French provinces. Similarly, if
  \HOL and \HIS are still involved together in another war, Dutch troops in
  Spanish provinces should not be evacuated and must stay here (evacuation is
  not an option, if you don't need to evacuate, you may not evacuate).

  If \ANG is in limited intervention allied to \HIS and \HOL and stops its
  intervention (either because the war ends or because it wants to do
  something else), then it must evacuate from the territory of \FRA, \HIS and
  \HOL and must go back to England. If the intervention continues, however,
  the English stack may stay on the continent.
\end{exemple}

\aparag[Pacification] Unless this is a Negotiated Peace, or a Conditional
white peace:
\bparag All existing \CB at the time of the peace are negated for 1 turn, even
permanent ones.
\bparag Additionally, each loser is forbidden to declare war without \CB
against any victor next turn.
\bparag For this purpose, all peaces of level 1 are considered Conditional
peace if possible. Only the peaces that could not have been signed as
Conditional peace are Negotiated peaces.

\begin{designnote}
  This effectively prevents the losers from attacking the winners next turn,
  unless an new \CB appears, usually by event. The winners, however, may
  attack the losers but at high cost (no \CB).
\end{designnote}

\aparag[Peace and Casus Belli]
\bparag Any permanent \CB whose cause does not exists any more is cancelled
(\emph{e.g.} return of the last national province, conversion of an heretic,
\ldots)
\bparag Unless this is a white peace, all temporary \CB from all belligerents
(not only the attacker) obtained before the end of the war are considered to
have been used.

\begin{designnote}[Temporary \CB]
  Most temporary \CB are one time. In case of war, all of them are considered
  used, that is, the war is waged over all former causes of resentment not
  just over the single border dispute that made it erupt.

  Some temporary \CB are multiple use (\emph{e.g.} once per period). In this
  case, the war ``uses'' one of these.
\end{designnote}

\aparag[Peace and \STAB]\label{chPeace:Peace and Stability}
Any major country that both
\begin{modlist}
\item was fully at war against at least another major country or was victim of
  a declaration of war by a minor (either by political event or \RD) during
  one of the previous turns ;
\item[AND] is now completely at peace (no intervention either) for the first
  time since these wars ;
\end{modlist}
immediately gains 1 \STAB.
\bparag If the country is not completely at peace now, the \STAB will be
gained when it will be at peace, even if the last peace treaty should not be
enough to gain it.
\bparag This gain is limited to 1 \STAB per country per turn, no matter how
many peaces are signed.

\begin{exemple}[Standard case]
  At turn 46, at the end of \ref{pV:WoSS}, \FRA and \HIS sign peace with \ANG,
  \HOL and \AUS. \AUS is still involved in a war in \paysHongrie against
  \TUR. Since \FRA, \HIS, \ANG and \HOL are now fully at peace they each gain
  1 \STAB, and only 1, no matter how many enemies they signed peace with. \AUS
  is prevented from gaining it by still being at war against \TUR. At turn 47,
  \AUS and \TUR sign peace. They are now fully at peace and both gain 1 \STAB.
\end{exemple}

\begin{exemple}[Peace with minors and \STAB]
  At turn 5, \TUR attacks \paysDamas, a minor. If it signs peace at the end of
  turn 5, it does not gain \STAB as this is a minor and \TUR was the
  attacker.

  At turn 5, \TUR attacks \paysDamas. At turn 6, \ref{pII:War Persia Turkey}
  happens early and \paysPerse attacks \TUR. At the end of turn 6, \TUR manage
  to sign peace with \paysPerse. Since it was victim of a declaration of war
  by a minor, it should gain \STAB. However, it is still at war against
  \paysDamas and may not gain it, but the fact that it got out of a ``big''
  war is remembered. At turn 7, \TUR signs peace with \paysDamas. Since it is
  now completely at peace, it gains 1 \STAB.
\end{exemple}

\begin{exemple}[Separate peaces and \STAB]
  At turn 10, \TUR is at war against allied \HIS and \VEN. It signs a separate
  peace with \VEN. Since it is a major country, it should gain 1 \STAB but is
  prevented to do so by still being at war against \HIS. \VEN, however, is now
  fully at peace and gain 1 \STAB (thus mitigating the 2 \STAB loss of
  breaking an alliance for separate peace).

  At turn 8, \TUR and \HIS sign peace. \TUR is now fully at peace and has two
  reasons to gain \STAB: the former treaty with \VEN and the current with
  \HIS. However, the max gain is 1 per turn, so it gains only 1
  \STAB. Similarly, \HIS is now fully at peace and gains 1 \STAB.
\end{exemple}

\begin{exemple}[Peace and interventions]
  At turn 28, \POL is both at war against \RUS and in foreign intervention in
  \ref{pIV:TYW}. It signs peace with \RUS. Since it is a peace with a major,
  it should gain 1 \STAB, but being in intervention prevents this. At turn 29,
  \POL ends its intervention. At the end of the turn, since it is now fully at
  peace and was previously at war against a major, it gains 1 \STAB.
\end{exemple}

\begin{playtip}
  This \STAB gain may only occur when a country becomes fully at peace. So
  most of the time you don't need to figure out whether you gain \STAB or
  not. The question should only arise when becoming fully at peace.
\end{playtip}

% Useless with negative treasure.

% \aparag[Indemnities] Indemnites agreed in the Peace are to be paid by one of
% the losing Major powers. He may pay them now or at a segment of Announcement
% of the two following turns. They can be paid in fractions during the 3 turns
% allowed.
% \bparag Failure to pay all the due indemnities give a temporary free \CB to
% the power that was wronged against the power that should have paid.



\section{Stability adjustment}\label{chPeace:Stability}
\subsection{Wars}
\aparag The \STAB of each country that is not fully at peace (and some other
cases) decreases.

\aparag[Full war]
\bparag Each country fully at war (either against a major or minor) loses as
many \STAB as the duration of the war (in turns).
\bparag Thus, the loss is 1 \STAB on the first turn of the war, 2 on the
second and so on.
\bparag The loss is limited to a maximum of 4 \STAB per turn.
\bparag Note that this loss is applied \textbf{after} peaces have been signed,
thus countries signing peace are not affected by it.

\aparag[Overseas Wars]
\bparag The same loss of \STAB is applied for Overseas Wars
\bparag The loss for an overseas war is limited to 2 \STAB per turn, however.

\aparag[Multiple Wars] Loses for wars are not cumulative. Only apply the
bigger loss.

\begin{exemple}[\STAB adjustment: Thirty years war]
  At turn 26, \ref{pIV:Bohemian Revolt} occurs and \AUS enters war against
  \paysBoheme. At the end of the turn, \AUS loses 1 \STAB for this war.

  At turn 27, the war degenerate in \ref{pIV:TYW} and both \HIS and \HOL enter
  the war. At the end of the turn, \HIS and \HOL both lose 1 \STAB, as they
  have been at war for one turn, while \AUS loses 2 \STAB, as it has been at
  war for two turns.

  At turn 28, \SUE takes the defence of the protestant cause and enter the
  war. At the end of the turn, \SUE loses 1 \STAB, \HIS and \HOL lose 2 each
  and \AUS loses 3\ldots

  At turn 29, \FRA enter the war against the Habsburg empire. At the end of
  the turn, \FRA loses 1 \STAB, \SUE loses 2, \HIS and \HOL lose 3 each and
  \AUS loses 4.

  At the end of turn 30, \FRA loses 2 \STAB, \SUE loses 3, \HOL and \HIS lose
  4 each and \AUS should lose 5 but the loss is limited to
  4. \ministreRichelieu starts asking his opponents if they're ready to accept
  his terms\ldots
\end{exemple}

\subsection{Interventions}
\aparag[Limited Intervention]
\bparag For each continued limited intervention, the intervening country loses
1 \STAB.
\bparag This loss is cumulative with the loss for full wars.

\aparag[Foreign intervention]
\bparag There is no cost for continuing a foreign intervention. However, the
intervening stack may not be reinforced.
\bparag Remember that it is always possible to withdraw from a foreign
intervention and re-intervene next turn (while limited intervention may only
be declared at the beginning of the war). This does cost 1 \STAB and de facto
allows to reinforce the stack.

\begin{exemple}[Interventions]
  \FRA is at was against \HIS and enters a limited intervention against
  \ANG. At the end of the turn, \FRA loses 1 \STAB for the war and 1 for the
  intervention, for a total of 2 \STAB. At the next turn, \FRA will loses
  2 \STAB for the war and 1 for the intervention.

  \ANG is in intervention both against \FRA and against \AUS (in two separate
  wars). At the end of the turn, it loses 2 \STAB, one for each intervention.
\end{exemple}

\subsection{Other cases}
\aparag Some other rules or events cause lose of \STAB at this point.
\bparag Sometime, the loss is cumulative with others losses, sometimes it's
not.

\aparag[Turkey and the Knights]
\bparag If the \corsaire of \paysChevaliers caused the loss of at least one
Turkish \TradeFLEET level, \TUR loses 1 \STAB.
\bparag This loss is not cumulative with any other. Thus it happens if and
only if \TUR did not lose \STAB at this segment yet.

\aparag[\villeVienne]
\bparag If \TUR took control of \villeVienne this turn and still controls it
at the end of the turn, each Catholic country among \HIS, \AUS, \POL, \FRA and
\ANG loses 1 \STAB.
\bparag If \TUR controls \villeVienne without owning the province (either from
this turn or a previous one), \HAB loses 1 \STAB.
\bparag These losses are cumulative with other losses.

\aparag[\villeRoma]
\bparag If \TUR took control of \villeRoma this turn and still controls it at
the end of the turn, each Catholic country loses 1 \STAB.
\bparag If \TUR controls \villeRoma without owning the province (either from
this turn or a previous one), the \SDCF loses 1 \STAB.
\bparag These losses are cumulative with other losses.

\section{Inflation}\label{chPeace:Inflation}
\aparag Each turn, the \RT of each country is decreased as a way to represent
Inflation (increase of prices is equivalent to decrease of stockpiled money).
\bparag Countries with negative \RT still lose money from Inflation (as the
debt owners adjust their requests and old obligations are refunded while new
ones are ĉontracted, even if not represented in game).

\begin{designnote}
  Inflation increases quickly as the gold and silver flow from
  \continentAmerica becomes high and regular. Do not hope to go back to the
  good old days of low inflation\ldots
\end{designnote}

\subsection{Increase of Inflation}\label{chPeace:Increase Inflation}
\aparag Inflation varies between 5\% and 33\%. The Inflation counter is placed
on the bottom line of the Resources prices track (on the \ROTW map) in the box
corresponding to the current Inflation.
\bparag Place the counter with the ``$\geq 3$'' side up if there is 100\ducats
or more of gold exploited in the \ROTW this turn, and with its ``$\geq 7$''
side up otherwise.
\bparag Count all the gold exploited in the \ROTW, no matter who exploited it
where (\emph{i.e} Russian gold from \continentSiberia counts) and no matter
whether it was repatriated in Europe, kept in the \ROTW or sunk.
\bparag Also count gold exploited from new \COL even those placed this turn.

\aparag[Increase of Inflation] Roll 1d10. If it is larger than the
threshold on the counter ($\geq 3$ or $\geq 7$ depending on the side up), move
the Inflation marker one box to the right.
\bparag Exception: the counter may never go beyond the 33\% box.

\aparag[Other variations]
\bparag Economic situation may increase Inflation as
per~\ref{chEvents:Inflation}.
\bparag Economic events~\ref{eco:Inflation} or~\ref{eco:Deflation} may
increase of decrease Inflation.

\subsection{Inflation}
\aparag[Inflation value]
\bparag Countries that do exploit gold in \continentAmerica
have an \terme{Inflation value} equal to the percentage written in the box
where the counter is located (between 5\% and 33\%).
\bparag Other countries have an \terme{Inflation value} equal to the
percentage written on the box on the left of the marker (between 5\% and 25\%,
also use 5\% when the marker is on the leftmost box).
\bparag Exception: \TUR before its reform use Inflation as if it exploited
gold in \continentAmerica (\ref{chSpecific:Turkey:Cost Pashas}).
\bparag Countries that exploit gold in the \ROTW out of \continentAmerica
(usually \RUS in \continentSiberia) do not suffer from higher Inflation. Only
gold from \continentAmerica counts.

\begin{designnote}[Gold flow]
  The Russian gold flow was way smaller than the Spanish Silver flow from
  America. Moreover, higher Inflation for \RUS causes some non-historical
  gamey tactics.
\end{designnote}

\begin{playtip}[Spanish World]
  Due to the increased Inflation, exploiting only one mine in
  \continentAmerica is probably not worth the effort. That is, if you start
  going for gold, go and grab as much as possible. Usually, only \HIS manage
  this due to its early arrival in the New World.
\end{playtip}

\aparag[Minimal inflation]
\bparag Each country has a \terme{Minimal inflation} which is equal to its
\terme{Inflation value}, in \ducats (\emph{e.g.} a country with an Inflation
value of 10\% has a Minimal inflation of 10\ducats).
\bparag Exception: \POL (always), \RUS (during periods \period{I}-\period{V})
and \SUE (during periods \period{III}-\period{V}) % and \DAN always
have a \terme{Minimal inflation} equal to half their \terme{Inflation value}
(round up).

\aparag[Computed Inflation, Actual Inflation]
\bparag Each country has a \terme{Computed Inflation} which is its Inflation
value (percentage) applied to the absolute value of its \RT (drop the minus
sign if any).
\bparag Each country has a \terme{Actual Inflation} which is the maximum
between its Minimal inflation and its computed inflation.

\aparag[Inflation]
\bparag Each country loses an amount of money equal to its Actual Inflation.
\bparag This loss is written in \lignebudget{Inflation}.

\begin{designnote}
  In practice, if your \RT is between -100\ducats and 100\ducats (or between
  -50\ducats and 50\ducats for ``poor'' countries), your Actual Inflation is
  equal to your Inflation value (no need to compute). Otherwise, it's equal to
  your Computed Inflation. Thus, the actual computation is easier than what
  the rules suggest\ldots
\end{designnote}

\begin{exemple}[Inflation value]
  If the Inflation markers is in the seventh box (leftmost 20\%), then
  countries that exploit gold in \continentAmerica (usually, only \HIS, plus
  \TUR because of its special rule) have an Inflation value of 20\% while all
  other countries have an Inflation value of 10\% (the box on the left of the
  marker).

  If the marker is in the eighth box, then all countries have an inflation
  value of 20\%.
\end{exemple}

\begin{exemple}[Computed and Actual Inflation]
  Suppose that the Inflation value for \FRA is 25\%.

  \FRA has thus a Minimal Inflation of 25\ducats. If its \RT is 60\ducats,
  then its Computed Inflation is 15\ducats (25\% of 60\ducats) and its Actual
  Inflation is 25\ducats (maximum between the Minimal Inflation of 25\ducats
  and the Computed Inflation of 15\ducats).

  If its \RT is 160\ducats, the Computed Inflation is 40\ducats and thus the
  Actual Inflation is also 40\ducats.

  If its \RT is -60\ducats (debts), the Computed Inflation is still 15\ducats
  (computed on the absolute value) and the Actual Inflation is 25\ducats. If
  the \RT is -160\ducats the Computed Inflation is 40\ducats and so is the
  Actual Inflation.
\end{exemple}

\begin{exemple}[Poor countries]
  Suppose that the Inflation value for \POL is 25\%.

  \POL has thus a Minimal Inflation of 13\ducats (25/2, round up). If its \RT
  is 40\ducats, the Computed Inflation is 10\ducats and the Actual Inflation
  is 13\ducats. If its \RT is -60\ducats, the Computed Inflation is 15\ducats
  and so is the Actual Inflation.
\end{exemple}

\begin{playtip}[Happy budget]
  Keeping your \RT after peace between -100\ducats and 100\ducats is the key
  to avoid too many loses due to Inflation. In most cases, this is relatively
  easy because you can always take more or less loans to adjust your \RT. Note
  that since loans interest is only 10\%, it is usually more interesting to
  get loans than a highly negative \RT (once the Spanish silver starts
  flowing, inflation will likely be stuck on the 25\%/33\% box with only a
  handful of turns in the 20\%/25\% box). Loans, however, have other
  disadvantages. Because of peaces indemnities, the Spanish Gold flow or too
  many military expenses, it is sometimes tricky to achieve this.

  Each turn, you will need to pay for your Inflation, that is most of the time
  25\ducats (or 13\ducats for poor countries). This means that each turn you
  must manage to get a budget positive by 25\ducats. Or you'll start getting
  into negative \RT which is not a good idea. Getting this positive each turn
  is not as easy at it seems and will require clever use of National
  loans\ldots Obviously, during big wars you may not manage it and start
  getting into debt. Hopefully you'll stay at peace long enough to stabilise
  your budget before the next war.
\end{playtip}

\section{Test for crusade}\label{chPeace:Crusade}
\aparag During periods \period{I}-\period{III}, each turn \TUR annexes a
Christian province as a result of a peace, a test for Crusade is made.
\bparag These tests end in period \period{IV}.
\bparag The test only occurs on the turns where \TUR signs a peace resulting
in the annexation of one or more province belonging to a Christian country. It
does not occurs if \TUR gains ownership of provinces via Dynastic ties with
another power, diplomatic annexation, special \regionBalkans annexation
(\ref{chSpecific:Balkans}) or any other mean.

% google battle:
% "call to crusade" -> 9,320,00
% "call for crusade" -> 19,200,000    WINNER!
\aparag[Call for Crusade]
\bparag Roll 1d10, modified as follows.
\bparag[Modifiers] (cumulative, up to a maximum of +5):
\begin{modlist}
\item[+1] for each province belonging to a Christian country annexed by \TUR
  (whatever the mean (count provinces from \regionBalkans annexed from a
  Christian country, not those that where Neutral before annexation) during
  the last 5 turns;
% replaced "current period" by "last 5 turns". It makes no sense to have the
% pope forget everything suddenly (or we implement pope election...)
\item[-2] for each Catholic major at war against a Christian country
% Replaced "against all but \TUR" by "against a Christian". A HIS-Algeria war
% should not prevent the call...
\item[-3] if the event \ref{pI:Reformation} has occurred.
\item[+5] if \villeVienne or \villeRoma is controlled by \TUR (whether owned
  or not).
\end{modlist}
\bparag If the unmodified result is 10 (whatever the modifiers) or if the
modified result is 10 or more, the Pope calls for Crusade.

\aparag[\villeRoma] During periods \period{I}-\period{II}, if \TUR takes
control of \villeRoma and still holds it at the end of the turn (whether the
peace has been signed or not), an automatic call for Crusade is made by the
Pope.
\bparag This only occurs on the turn where \villeRoma is captured by \TUR. If
it keeps control of it (including annexation of the province), there is no
automatic call for Crusade. If \TUR loses control of \villeRoma and retakes it
during another turn, another automatic call for Crusade may happen.

\aparag[Crusade] If the pope calls for Crusade, consider that
\eventref{pII:Crusade} is rolled as a fifth event on the following turn. There
may be only 4 other political events (no extra event, consider that a ``+1''
result was already obtained).

% Local Variables:
% fill-column: 78
% coding: utf-8-unix
% mode-require-final-newline: t
% mode: flyspell
% ispell-local-dictionary: "british"
% End:
