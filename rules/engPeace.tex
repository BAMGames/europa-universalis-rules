% -*- mode: LaTeX; -*-

\definechapterbackground{Peaces}{victories}
\chapter{Peaces}\label{chapter:Peace}

\aparag After all military affairs, many operations remain to be resolved:
checking economic consequences of the military operations, checking the
positions of ongoing conflicts, signing Peaces, redeployment of forces and so
on.
\bparag Redeployment Phase
\bparag Peace Phase
\bparag Interphase and End-of-turn




\section{Redeployment Phase}

% RaW: [40]

\aparag[Sequence.]
% The redeployment phase unfolds with different segments, resolved in
% this order, and by order of initiative in each segment:
\RedepDetails



\subsection{Attacks by Natives}\label{chPeace:Native Attack}

\aparag At the end of the military round, either activated Natives (activated
because of Military presence or an ongoing war) or forces of a \ROTW minor
country may attack offending settlements \COL or \TP in each province. Natives
and minor countries in their own area always attack; minor countries in areas
that they do not own will attack on choice of their controller.
\bparag Natives and minor countries add their force to do one attack only.
\bparag If the Natives have been defeated this turn and a minor country
attacks, the controller may choose to let only the forces of the minor country
do the attack alone.
\aparag The attack is resolved immediately by one die-roll on
\tableref{table:Pirates Natives Raids}. This die-roll is modified by:
\begin{modlist}
\item[+1] for each \LD in defence (even besieged).
\item[+2/+4] for each \ARMY\facemoins/\Faceplus in defence (even besieged).
\item[+N] level of the fortress.
\item[+M] Manoeuvre value of a \terme{land leader} in defence.
\item[-1] For each full \LD of the Natives (or attacking forces).
\item[-1] For each foreign \COL or \TP in the same \Area.
\item[-M] Manoeuvre value of a Native (or attacking) leader.
\item[+3]If the Natives were defeated at least once in the province this turn
  without being routed.
\item[+6]If the Natives were defeated at least once in the province this turn
  and were routed, not cumulative with above.
\end{modlist}
\aparag Losses on settlements is given in the table. Each lost level step is
applied to the Colony/Trading Post present. If a Colony, or a Trading Post is
reduced to the level 0, the establishment is destroyed.
\aparag Losses on land forces in defense are also given. The losses are
applied to fortifications and land forces (one level of fortifiction equals
one \LD here), divided as the player prefers.
\aparag Natives that were activated due to Administrative actions or Military
presence, cease now to be active. Those that are active because of a war, stay
as they are (they will behave in the same way as the minor country).



\subsection{Attacks by Pirates \& Privateers}\label{chPeace:CorsairAttack}

\aparag Pirates and privateers attack commercial fleets to attempt to decrease
their levels, and possibly to capture gold repatriated to Europe by these
fleets.
\bparag[Pirates] Pirates appear as explained in \ruleref{chEvents:Piracy} and
they remain until completely eliminated. They are active every turn.
\bparag[Privateers]
\label{chPeace:CorsairAttack Privateer}
Privateers are raised by Major Powers (see \ruleref{chExpenses:Recruiting
  Privateers}), or are in the basic forces of some minor powers (the
\pays{chevaliers} and the "Barbaresques" countries). They must go out at sea
on the first or second round or they will have no effect.
\bparag Beginning with the third round, they stay in the sea they were placed
in, and will be able to attack one \STZ or \CTZ in this sea or an adjacent
sea. The specific \STZ or \CTZ has to be annouced at that time.

\aparag[Raiding Fleets with Privateer Admirals]\label{chSpecific:Raiding
  Privateer admirals}
Privateer, or an Admiral with Privateer capacity, may lead one \corsaire he
starts the turn with.  He may lead it in the same stack as naval forces not
containing a \FLEET.  The \corsaire does not count for attrition, nor in
battle.  The stack acts both as regular naval force (and can attack, blockade,
and so on), and a Privateer stack (other players may attempt to suppress the
\corsaire counter).  The \corsaire does not count for attrition, nor in battle
(nor is affected by battles).  The stack may split at any time (for instance
if the naval force has to retreat in a port), and the leader chose which stack
he stays with.
\bparag As an exception to \ruleref{chPeace:CorsairAttack Privateer}, a
\corsaire led by Privateer or Privateer-Admiral may move after the second
round, and has to remain in place only on the last round (the player telling
at the beginning of this round which \CTZ\\STZ will be attacked if there are
several of them). However, it still has to be at sea at the end of every round
after the first, else (if at port), it cannot leave again for the rest of the
turn and will not attack commercial fleets (or loot) this turn.
\bparag Note that the leader may move as he prefers but can only lead the one
\corsaire he starts the turn with (even leaving it then coming back), or naval
forces.
%% PB 07/2008 : TBD ??  J'ai essaye de simplifier cette partie avec la version
%% ci-dessus, pour mieux
% représenter aussi ce qu'on veut.
%
% \corsaire be moved every round to another \corsaire or naval stack, excepted
% on the last round where it can not join a new \corsaire counter.
% \aparag[Raiding Fleets with Privateer Admirals]\label{chSpecific:Raiding
% Privateer admirals}
% \textbf{[To be deplaced in military]} Naval forces not containing a \FLEET
% that are led by a Privateer, or and Admiral with Privateer capacity may move
% with one \corsaire counter in their stack if it begins even after the 2nd
% round.  Naval forces not containing a \FLEET that are led by a Privateer, or
% and Admiral with Privateer capacity, may be announced to be conducting
% privateer attack of trading fleets (and potentially looting) at the latest
% at the end of the second round if at sea.
% \bparag In that case, replace one \ND for a \corsaire\facemoins, or two \ND
% for a \corsaire\faceplus.  AUTRE VERSION: sans transformation de pions
% \bparag In this case, they may not leave anymore the sea zones adjacent to
% the \CTZ or \CTZ they are attacking, for the remaining of the Military
% Phase.
% \bparag The force is hereafter for the rest of the Military Phase dealed
% with as being \corsaire\facemoins if they have between 2\de and 1\ND2\de,
% and \corsaire\faceplus if they have 2\ND (or \NGD) or more, with no
% capacity as a naval force. It tests for attrition only if decides to move to
% another sea zone; it can not attack, intercept of be attacked or intercepted
% (except to attack Convoys as per Privateer rules)).
% \bparag If a roll to reduce Privateer succeed, the stack suffers 50\%
% attrition.  If it losed forces so as to have less than 2\de, the stack
% becomes a regular naval force at the beginning of its next movement segment.

\aparag[Looting by Pirates or Privateers] Pirates and Privateers may try to
loot Trading Posts or Colonies, and also enemy provinces for privateers, that
are a province bordered by the sea they are in.
\bparag Looted provinces, Colonies or Trading Posts may belong to minor
countries or to players. For privateers to be allowed to loot, it is necessary
that a state of war exists between the owner of the privateer unit and the
owner of the looted province. Overseas Wars are enough to loot \TP or \COL,
but not European provinces.
\bparag \textit{Exceptions:} Looting of European provinces by the
"Barbaresques" is permitted, as well as looting in their provinces. Sea Hounds
may loot European provinces also, see \ruleref{chSpecific:England:Sea Hounds}.
\bparag Pirate may loot following \ruleref{chEvents:PiracyTarget}. After a
turn of looting, non-eliminated pirates go back to the \STZ they belong to.
\bparag The privateer intending on looting is placed in the concerned
province, Trading Post or Colony. They have to disembark during any round
except the last from the sea zone they are operating in.
\bparag If privateer/pirate unit is still present at the Redeployment phase,
it loots. Looting privateer/pirate are unaffected by forces or battles (except
that those forces may attempt to destroy them during the military phases).
\bparag A maximum of 1 privateer/pirate unit (any side up) can loot the same
Colony/province in the same turn. Privateers/Pirates looting a province or
\COL/\TP can not attack at the same turn the \CTZ/\STZ.

\aparag[Actions of Pirates and Privateers] In \STZ/\CTZ where
pirates/privateers are active, one die is rolled on \tableref{table:Pirates
  Natives Raids}.
\bparag Pirates attack all fleets in the zone and are resolved first as a
separate atack in each zone.  countries.
\bparag Privateers of \terme{Barabaresque} countries make joint attacks after
Pirates.  Turkish privateers may be added to the same attack if at war against
all the aimed christian countries, or if taking advantage of rule
\ruleref{chSpecific:Balkans}.
\bparag Then one attack is resolved for all the Privateers of the same
alliance, in the order of Initiative.  Privateers may target only fleets of
countries against which their owner is at war (or overseas
war). \emph{Exception:} see the restricted Overseas War of the
\pays{chevaliers} or the "Barbaresques" countries.

\bparag The die is modified as follows:
\begin{modlist}
\item[+2] if the \corsaire is not exactly in the sea zone of the \STZ.
\item[+3] if only one \corsaire\facemoins
\item[+1] per side of target \TradeFLEET or \FLEET (a Convoy counts as 2
  sides)
\item[+1] If one or more \ND in defence and no \FLEET
\item[+2/+4] per \FLEET\facemoins/\Faceplus defending
\item[+M] \Man of a defending \LeaderA
\item[-1] per round at sea (max. -3) (NA on Convoy attacks)
\item[-M] \Man of a \corsaire admiral %(\textonehalf for land raids in Europe)
\item[+1] if a naval battle occurred in the sea %(not for land raids)
\item[-2] \pays{chevaliers} with Christian port on \seazone{Egee} or
  \seazone{Mediterranee E}
\end{modlist}
\bparag Levels losses are taken on the target players' fleets present in the
\STZ/\CTZ (including minors).

\aparag[Defense against Pirates and Privateers]
\bparag[Defending naval force] A defending naval force is a stack of
fleet/detachment markers present at the start of the Redeployment phase in a
sea zone that is part of the concerned \STZ/\CTZ, that decides to defend (and
can do so because of war status).
\bparag The player can choose to suffer his losses from his warships in the
\STZ/\CTZ (if any). In such a case, one eliminated \ND equals the loss of 1
level of commercial fleet.

\aparag[Return of lost levels] All levels eliminated by privateers and pirates
are losses to the \terme{current level}, not the \terme{maximum level}. See
\ruleref{chExpenses:Commercial Fleet Adjustment}.
\bparag Each \textetoile result destroys permanently one level of commercial
fleet (it reduces by one the \terme{maximum level}).  The largest fleet is
affected by this permanent loss. In case of equality, the owner of the
Privateer decide the fleet to be reduced; for Pirates, this is done at random.

\aparag[Income of Privateers] Each level eliminated by a Privateer brings an
income to the player that controls the privateer unit equal to the small
number printed in the \STZ. Privateers of minor countries give no income (even
if Vassals).
\bparag This ``privateers income'' is to be placed in \lignebudget{Pillages,
  privateers} of the privateering player.
\bparag In case of stacks with \corsaire from several powers, this income is
divided between the powers (including minor ones).

\aparag[Results of Pirates \& Privateers Looting] These lootings are resolved
exactly in the same manner: one die-roll on \tableref{table:Pirates Natives
  Raids}, modified as follows:
\begin{modlist}
\item[+3] if only one \corsaire\facemoins
\item[+2/+4] per \ARMY\facemoins/\Faceplus defending
\item[+1] Per \LD (including militia) %against land raids
\item[+M] \Man of a defending \LeaderG/\LeaderC/\LeaderGov
\item[-1] per round in province (max. -3) %(NA on Convoy attacks)
\item[-M] \Man of a \corsaire admiral (\textonehalf for land raids in Europe)
  % \item[+1] if a naval battle occurred in the sea (not for land raids)
\item[-2] \pays{chevaliers} with Christian port on \seazone{Egee} or
  \seazone{Mediterranee E}
\item[+N] Twice the level of the fortress, +1 for fort
\end{modlist}
\bparag If a result \textddag\ is obtained on the table, the province or
\TP/\COL is looted: place a marker \PILLAGE\faceplus. The privateer's owner
receives the total income of the province/settlement (gold stored here and
resources exploited included).
\bparag If a result \textdag\ is obtained on the table, the province or
\TP/\COL is weakly looted: place a marker \PILLAGE\facemoins. The privateer's
owner receives half the total income of the province/settlement (resources
exploited included), but Gold stored here is entirely captured.
\bparag Pirates receive no income for looting.
\bparag These incomes are placed in \lignebudget{Pillages, privateers}, except
stored gold in \lignebudget{Gold from ROTW & Convoys}.
\bparag There is neither loss of land forces due to the looting (opposite to
Natives attack) nor protection by sacrificing forces in the province.

\aparag[Minor countries against Piracy]
\bparag Minor countries at war can use their naval forces against Privateers
and Pirates in \STZ or \CTZ where they have a \TradeFLEET of their own.
\bparag Christian Minor countries whose \TradeFLEET are attacked by privateers
of \terme{Barbarques} may also use their naval forces to fight against those
privateers (Patron's choice to deplace their forces), even if at peace.
\bparag Against Pirates, minor countries at peace fight in an abstract way.
Roll 1d10 for each \STZ or \CTZ with Pirates and add 1 for each commercial
fleet of a minor country (+2 if the fleet is \Faceplus).  If the result is 8
or higher, one \corsaire\facemoins is eliminated.



\subsection{Extension of Revolts}

\aparag \REVOLT in inactive minor countries are automatically removed without
any roll.


\subsubsection{Loss of \STAB due to Revolts}
\label{chPeace:Revolts Stability}
\aparag [TBD] \REVOLT in minor countries cause loss of \STAB to their
diplomatic patron as if they were in it.

\aparag If one or more \REVOLT still exist in a country, the country loses
\STAB
\bparag for each \REVOLT\faceplus, it loses 1 \STAB level
\bparag for all \REVOLT\facemoins, it loses only one additional \STAB level
\aparag However, the maximum a player can lose from revolts of all types is 3
\STAB levels.


\subsubsection{Extension of \REVOLT}
\label{chPeace:Extension Revolts}
\aparag Then, after \STAB losses, \REVOLT extend. Adjust the \REVOLT markers
simultaneously:
\bparag A previously \REVOLT\facemoins becomes a \REVOLT\faceplus
\bparag A previously \REVOLT\faceplus generates a \REVOLT\facemoins in the
same or adjacent province (belonging to the victim country, or in the region
allowed if the \REVOLT was created by an event), of the choice of the player
controlling the \REVOLT.
% Jym, following JC.
\bparag \textbf{[TBD]} \REVOLT in \regionIrlande may extend this way across
the \seazoneMan into \ANG.
\bparag If the extension of a \REVOLT\faceplus is not possible, a \REVOLT \LD
is placed in the same province (or an \ARMY\facemoins if there was already 1
\REVOLT \LD present, the \LD being absorbed in the \ARMY).
\bparag Remember that there can be at most 2 \REVOLT markers stacked together.
\bparag Unbesieged cities in revolt and revolted troops generate a
\REVOLT\facemoins in their province if there is no \REVOLT counter in it.
% Jym, yes this was in EU6...  removed, 04/2011.
% \bparag When all provinces bordering all the sea zones attached to a
% particular \STZ or \CTZ are in revolt (unusual but possible), place one
% pirate \corsaire\facemoins in this \STZ/\CTZ at the beginning of the Peace
% phase (during the extension of revolts).

% Jym: si PB pas sur, je vire.
% \bparag \textbf{[TBD]} Revolt \Faceplus in provinces with at least an \ARMY
% counter of the owner is controlled: it does not extend.  (PB 07/2008: I
% favor that; 09/2009: I am not so sure now....)

% \begin{todo}
%   Check if this is useful. Jym believes the events allowing this explicitly
%   state the provinces allowed for proliferation. Anyway, this is too fuzzy
%   to be used as a rule without crosschecking every event that generates
%   revolts. Plus it may have a bad effect (eg Pugatchev revolt spreading in
%   still independant Khanates...)
% \end{todo}
% \aparag[Ethnic Revolts] Revolts appearing through the play of political
% events different from the R result on the event are considered as ethnic
% revolts.
% \bparag Ethnic revolts always proliferate in all provinces described in the
% political event, even if the owner of that province is not the country
% victim of the event.


\subsubsection{Independence of Revolted Principalities}\label{chPeace:Peace:Independence Revolt}
A \MAJ may give the independence to some groups of provinces (usually
separated from the mainland of the power) if all the provinces of the group he
owns (except at most one) have a revolt \Facemoins.  This announce is made
during the diplomatic phase.  See the rule
\ruleref{chSpecific:Peace:Independence Revolt}.


\subsubsection{Execution of the
  Monarch}\label{chPeace:ExecutionMonarchByRevolts}
\aparag If, at the moment of the Peace phase, half of all owned national
provinces (rounded up; they are shown on the map by a coat of arms of the
power) are in revolt, the regime of the country is overthrown and the monarch
is executed.
\aparag[Consequences]
All revolts present in the country are removed. A new monarch is
determined. The new monarch is considering as suffering a "Dynastic Crisis".
% Jym: useless, c'est deja l'effet DC...
% \bparag All the values of the new Monarch are obtained disregarding the
% values of the former ruler.
\bparag The \STAB is reduced by \textbf{2} levels%.
% Jym, memories of ruling during WoSS when I was FRA and HOL was overthrown.
and no \STAB improvement action is allowed this turn.
\bparag The \DTI is reduced by one (1 is the minimum).
\bparag 3 levels of Commercial fleets are reduced in the \CTZ box of the
country (chosen at random, even on other players' commercial fleets - NB: it
represents pillage and lost properties due to this really unstable situation!)
% \bparag The Stability is reduced by 4 levels.



\subsection{Land Military Looting}

Powers may receive supplementary income at this phase by looting provinces,
Colonies or Trading Posts of the other players (or minor countries) that their
units occupy.
\aparag A power can loot a province (or Colony/Trading Post) if he possesses
land units there (i.e. armies or detachments, or Turkish pashas) during this
phase (and so has been able to besiege the fortress).  Looting is allowed even
if the player does not in control the fortress of the province, Colony or
Trading Post.
\aparag[Adjustment of Already Existing Looting Markers]
Remove all \Facemoins looting markers, then adjust all \Faceplus looting
markers to their side \Facemoins, before to proceed to looting of the current
turn.
\aparag[Looting Income]
Unless there already was a looting marker in the province, the player that
loots receives immediately in his Royal Treasury a sum equal to the total
cumulated income of all looted provinces, not including exotic resources for
Colonies/Trading Posts.
\aparag[Placement of New Looting Markers]
When a looting takes place, one looting marker \Faceplus is placed on the map,
in the looted province, Colony or Trading Post.
\aparag[Looting in \ROTW]
If a land force has less than 2 \LD, it can only loot weakly a \COL/\TP.  Put
a \Facemoins looting marker in the province and the power gains only half of
the regular income.  If the power has at least 2 \LD, it may loot completely
the province.
\aparag[Consequences of Looting]
A side \Faceplus or side \Facemoins looting marker in a province (Colony,
Trading Post) cancels all land income of this province (or Colony, Trading
Post) and the exploitation of income from exotic resources.
\aparag[Elimination of Trading Posts]
A force occupying an enemy \TP at this phase may, instead of looting, destroy
it (player's choice).



\subsection{Mandatory Retreat of Sieges, \Presidios}

\aparag[Lack of Supply] Forces that have no LOS makes a mandatory redeployment
and have to test for attrition at {\bf +2} for this movement.
\aparag[Inadequate Siegeworks]
If a besieged fortress has no Siegework\faceplus placed over it, or has not
suffered a Breach or HW result during the course of the turn, the besieger has
to retreat to the closest friendly unbesieged city or port. This redeployment
causes an attrition roll for movement.
\aparag[Continued Sieges]
If a besieged fortress has a Siegework\faceplus, or a suffered from a Breach
or HW result during the course of the turn, the besieging player may either:
\bparag Decide to raise the siege and make a redeployement movement, with
attrition at {\bf -2};
\bparag Continue Siege, keeping only a Siegework\facemoins marker.

\aparag[Construction of \Presidios]\label{chPeace:BuildPresidios}
In some provinces, a Power may build or increase \Presidios if the province is
not owned but where its has military control, or is besieging the fortress
with adequate Siegeworks to continue the siege (even if the power decides to
raise the siege).
\bparag Provinces where \Presidios can be built are shown by a circled anchor
on the map.
\bparag A \Presidio is marked by a fortress counter. Its price is the same as
the fortress. The maximum level of a \Presidio is 3.  A power has to pay for
the Maintenance of \Presidio.  \Presidios may never be built or increased
during the Logistic Phase.
\bparag A power may only increase the fortress level of an existing \Presidios
at the exact same conditions for building it in the first place.
\bparag Ifever the power gains ownership of a province containing its own
\Presidio, the \Presidio is converted as a regular fortress for the province
(if higher than the current level), or dismantled.



\subsection{Return to Port}

\aparag Naval units being located in a sea zone have the choice to
\bparag retreat to a friendly, unblockaded port of their choice (not
necessarily the closest one). They roll for attrition in this movement.
\bparag stay at sea, by doing an attrition with a modifier of {\bf +2}, the
difficulty of the sea they are in, and the modifier due to being at sea of +3
or +6 depending of the distance to the nearest Sea Supply Source
(port/arsenal).
\aparag Then all privateers have to be repatriated to a port of their owner's
choice.
\aparag Pirates are never repatriated and remain permanently in their
\STZ/\CTZ of appearance until their destruction by a player.
\aparag No interception may ever occur during this phase, not even by
\Presidios or \StraitFort.



\subsection{Gold repatriation}\label{chPeace:GoldRepatriation}

\aparag[Before redeployment] For the income phase, see
\ruleref{chIncomes:GoldProduction}.
\bparag For the transportation for convoys, see
\ruleref{chIncomes:SpanishConvoys} and \ruleref{chMilitary:Convoys}.
\bparag Remark that gold that was recuperated during the military rounds
(through convoy reception or attack) is available as soon as it arrives in a
province on the European map of the power. Still, usually it is recorded
during this segment.
\aparag[Gold transportation]\label{chPeace:GoldTransportation} During the
redeployment phase, gold can be transported by land as long as it can find a
way of \TP, \COL or forts at most 12\MP apart crossing only friendly land. If
there is a land path to National Territory, it can not be stocked in \ROTW.
\bparag This is especially the case for \RUS exploiting mines in
\continent{Siberia}.
\bparag Gold stored in a \COL bordering the \seazone{Caspienne} (on the \ROTW
map) before the redeployment phase can be moved to a province on the other
side during this segment (and probably to their home country, either \POL or
\TUR).
\bparag Gold repatriated by land or stolen cannot be intercepted in any
way. If it reaches the main country, it is added to the royal treasury.




\section{Peaces and ongoing wars}

% RaW: [50]
\aparag Wars can be ended only by a Peace. There are several types of
Peace. These types depend on the difference between the Stabilities and
military situation of the loser and of the winner of the peace. However
players must be extremely careful not to be at a -3 level of Stability for two
consecutive turns. In such a case, they would be forced to accept any
peace. However, players still have the possibility to act on their Stability
level before peace is concluded
\aparag[Events and special peace conditions] A small number of political
events can initiate wars where the peace conditions are constrained or
modified by specific conditions. These events do not change the rules below
for other simultaneous conflicts where one side of the war is not one of those
described in the event.
\bparag Remark that the other side must enter war through another means than
the event or the alliances network called at the time of the event (else, it
is considered part of the event).

\aparag[Sequence of the Peace Phase.]
\PeaceDetails



\subsection{Stability Improvement action}

\label{chPeace:Stability Improvement}
\aparag A player can attempt to improve his Stability, but this is never
compulsory. If he desires to improve it, he must write down the investment he
wishes to pay for such an action and then roll for the Test of Stability
Improvement. The Stability is then adjusted in function of the obtained
result. This result may even be negative!
\aparag[Investment]
\bparag Basic Investment: 30 \ducats
\bparag Medium (+2 to the die-roll): 50 \ducats
\bparag Strong (+5 to the die-roll): 100 \ducats

\aparag[Procedure]
This action is resolved without requiring a table.  The player rolls the die
of conjuncture. This die-roll is modified as follows:
\bparag +? \ADM monarch,
% [TBD: or \MIL if at war ?]
\bparag +? if medium or strong investment
\bparag +2 if victim of a declaration of war (solely if the player has not
transgressed himself an alliance, and has not declared war to anybody during
this turn)
\bparag -2 if at war with a minor country
\bparag -3 if at war with one or more players (non-cumulative with the -2
above)
\bparag -5 if an enemy \ARMY counter is on a national province and controls
the province city (not applicable during a Religious/Civil War or to rebel
stacks appeared during or as a consequence of such an event)
\bparag -3 Exception: for \SPA, the malus for having an enemy \ARMY counter
controling the city, is -3 only, however it applies for any owned territory
(not only its national territory).  This specificity ends
with \eventref{pIV:Olivares} (if effects are applied), or with
\eventref{pV:WoSS} (whatever the choices and outcomes).
\bparag +3 for a Prosperous Power
\bparag -3 for an Anti-Prosperous Power
\bparag $\pm$? by event
\aparag[Result]
If the modified result is equal to:
\bparag 11-14: the Stability increases 1 level
\bparag 15-17: the Stability increases 2 levels
\bparag 18+: the Stability increases 3 levels
\bparag 5 or less: the Stability decreases 1 level
\bparag Reminder: Stability varies from -3 to +3.

\aparag[Prosperity]\label{chPeace:Prosperity}
The player checks on his economic form the evolution of his \terme{Gross
  Income} on the last two consecutive turns (including the one just
played). This evolution affects the Stability Improvement Action, as seen
above.
\bparag[Prosperous Power:]
If the gross income has not decreased during the last 2 consecutive turns and
progressed during at least one of those turns.
\bparag[Anti-ProsperousPpower:]
If it has decreased 2 consecutive turns.



\subsection{The Informal Peace}

\aparag The informal peace is concluded following a mutual agreement between 2
players or more, and has to be announced to all other players.
\aparag[Consequences]
None of the players earns any VP for an informal peace. The war stops
immediately.
\bparag This type of peace can comprise any clause for which there was mutual
agreement between the involved players, in the limit authorized by rules on
Agreement. The Agreement is applied now.
\bparag This type of peace can not be concluded with a minor.



\subsection{The Formal Peace}

\aparag A formal peace can be of different types: white-peace, negotiated,
conditional or unconditional.
\bparag It is also the only type of peace authorized with a minor country (or
alliances of minor countries with no major power), but with a different
mechanism.
\bparag Once the principle of such a peace is accepted by the involved
alliances, the type of a formal peace can be negotiated or not and depends on
differential of the involved alliances's Stability.
\aparag[The Peace Differential]
For the detail of peace, we will call "Peace Differential" (PD) the
differential between the Stability values of the 2 alliances at war and
modified by the military situation as described below.  The PD is considered
between the dominant alliance with the highest score (presumed victor) and the
other one with the lowest (presumed loser). The Stability value of an alliance
is that of the Major power in the alliance.
\bparag[Case of Allies]
if more than one Major power is involved in a side when signing a peace, the
Stability used is the average of the allied powers, rounded on the nearest
number (and down if one-half is obtained).
\bparag[Military Situation modifiers]
The Stability of a Major power is modified if it controls more provinces in
the enemy alliance than the ennemies control in itself's territory.  If the
difference is at least 2, the bonus is {\bf +1}, then {\bf +2} if at least 4,
and {\bf +3} is at least 6. Control of a capital city of a Major power counts
as 2 provinces in this comparison; \COL and \TP count for only one-half
province.
\bparag[Military Situation modifiers in Overseas Wars]
The same comparison is made but controlled \COL and \TP count as full
provinces.  \textit{Privateer Effect:} Moreover, each Commercial fleet with
level 4 or higher that was reduced to 0 or 1 because of Privateers count as
one province for the looting power at this turn (or 2 if in the \CTZ of the
enemy).

\aparag[Peaces that are permitted]
\bparag If the PD is at most +2 in favour of one alliance, a Peace of level 0
(White Peace) or Negotiated Peace of level 1 is permitted in favour of any
alliance (even the one with the lowest modified Stability).
\bparag In any case, a Conditional Peace of level equal to the PD in favour of
the dominant alliance is allowed. The maximum level is 5, even if the Peace
modifier is higher.
\bparag If a Major power (not an alliance) has its capital (or both its
capitals) and half of its national provinces (round up) controlled by enemies,
an Unconditional Peace is allowed against it (but not mandatory); if there is
more than one Major Power in an alliance and the condition is true over one of
them, an Unconditional Peace is possible.

\aparag[Peace constrained] If a Major power is for two consecutive at -3 in
Stability, vis-a-vis the same enemy power or alliance, he is complied to make
a losing White, Conditional or Unconditional peace with this power. The level
of that peace is the worst one that is authorised depending on their Peace
Differential or on the military situations for UP.
\bparag The enemy is not forced to accept this peace. The power is just forced
to \emph{propose} it to its enemy.
\bparag If both alliances are contrained to sign a peace at the same time, the
Peace is mandatory at the current PD level.
\bparag Under-developped Russia signs a constrained peace only if at -3 in
\STAB during three consecutive turns.
\bparag If the Major powers allied to this power do not want to sign a Peace,
the power makes a mandatory Separate Peace that entails neither loss of \STAB,
nor \CB given to the former allies.

\aparag[Definition of the conditions]
The exact terms of the peace have to be agreed by everyone in the alliances,
following the rules below for the condition. This is true especially for the
number of provinces to be ceded or of indemnities given instead, or if
specific conditions (such that nullifies some events) are asked by the
victorious alliance. If no agreement is possible, the peace will not be
signed.

\aparag[White Peace, or Peace of level 0]
The two alliances have to evacuate all conquests made during the course of
this war and return to the situation of province control existing at the start
of the war, except by express agreement between players.

\aparag[Negotiated or Conditional Peace of level 1]
\bparag The victorious alliance receives one peace condition of its choice
among the following ones.
\bparag[Cession of Territory] One province of the loser's choice is ceded.  It
is selected by the loser following the rule \ruleref{chSpecific:Tranfer
  Provinces Peace}.
\bparag[Indemnities]
The losing alliance gives 50 \ducats of war indemnities to the victorious
alliance.
\bparag[Diplomatic concessions]
One country is removed from one loser's Diplomatic Track and placed back into
the \Neutral box. This can not be a country that is \VASSAL or in \ANNEXION.
If the loser is a \MIN power, one can alternatively chooses to gain it in \MR
status.

\aparag[Conditional Peace of level 2]
\bparag The victorious alliance receives one peace condition of its choice
among the following ones.
\bparag[Cession of Territory] The victorious alliance receives one province of
the its choice. It is selected following the rule \ruleref{chSpecific:Tranfer
  Provinces Peace}.
\bparag[Indemnities]
The losing alliance gives 75 \ducats of war indemnities to the victorious
alliance.
\bparag[Diplomatic concessions]
One country is removed from one loser's Diplomatic Track (any status, excepted
if blocked by other rules of events) and placed back into the \Neutral box.
If the loser is a \MIN power, one can alternatively chooses to gain it in \MR
status.

\aparag[Conditional Peace of level 3]
\bparag The victorious alliance receives two peace conditions of its choice
among the following ones. Each of them can be selected twice.
\bparag[Cession of Territory] The victorious alliance receives one province.
The first province selected is its choice, but the second is of the loser's
choice.  Territories are selected following the rule
\ruleref{chSpecific:Tranfer Provinces Peace}.
\bparag[Indemnities]
The losing alliance gives 75 \ducats of war indemnities to the victorious
alliance.
\bparag[Diplomatic concessions]
One country is removed from one loser's Diplomatic Track (any status, excepted
if blocked by other rules of events) and placed back into the \Neutral box, or
in \MR status of one winner.  If the loser is a \MIN power, one can
alternatively chooses to gain a Diplomatic status, in \MR for one condition,
or \AM for two peace conditions.

\aparag[Conditional Peace of level 4]
\bparag The victorious alliance receives three peace conditions of its choice
among the following ones. Each of them can be selected ore than one.
\bparag[Cession of Territory] The victorious alliance receives one province.
The first and third provinces selected are of its choice, but the second is of
the loser's choice.  Territories are selected following the rule
\ruleref{chSpecific:Tranfer Provinces Peace}.
\bparag[Indemnities]
The losing alliance gives 100 \ducats of war indemnities to the victorious
alliance.
\bparag[Diplomatic concessions]
One country is removed from one loser's Diplomatic Track (any status, excepted
if blocked by other rules of events) and placed back into the \Neutral box, or
in \MR status of one winner.  If the loser is a \MIN power, one can
alternatively chooses to gain a Diplomatic status, in \MR for one condition,
or \AM for two peace conditions, or \VASSAL for three peace conditions (if
status possible).

\aparag[Conditional Peace of level 5 and Unconditional Peace]
\bparag The victorious alliance receives three peace conditions of its choice
among the following ones. Each of them can be selected ore than one.
\bparag[Cession of Territory] The victorious alliance receives one province.
All provinces are the choice of the victorious alliance.  Territories are
selected following the rule \ruleref{chSpecific:Tranfer Provinces Peace}.
\bparag[Indemnities]
The losing alliance gives 150 \ducats of war indemnities to the victorious
alliance.
\bparag[Diplomatic concessions]
One country is removed from one loser's Diplomatic Track (any status, excepted
if blocked by other rules of events) and placed back into the \Neutral box, or
in \MR status of one winner.  If the loser is a \MIN power, one can
alternatively chooses to gain a Diplomatic status, in \MR for one condition,
or \AM for two peace conditions, or \VASSAL or \ANNEXION for three peace
conditions (if status possible).



\subsection{Transfers of Provinces by Peaces}\label{chSpecific:Tranfer
  Provinces Peace}

\aparag Provinces are transferred as an implementation of the peace just
concluded, immediately upon the conclusion of the Peace phase.

\aparag[Choice of Provinces]
The provinces that can be annexed are:
\bparag provinces militarily conquered in the course of the just concluded
conflict are still controlled (except, usually, the capital province);
\bparag or provinces that used to belong to the victor and were annexed by the
loser during a preceding war (including any National provinces of the loser).
\bparag In priority, the National provinces of the victorious alliance are
chosen and annexed.
\bparag Minor Powers that are fully at war are included in the alliance like
any other power, so that they may gain provinces also.

\aparag[Transfer of Colony or Trading Post]
A player can choose a Colony or Trading Post instead of a province, on the
condition that he has effectively occupied this Colony during the course of
just ended conflict, or that the he has or used to have a \COL in the same
\terme{area}.
\bparag This \COL or \TP counts as half a province if controlled at the end of
the conflict.  Only \COL at levels 6 count as a full province.
\bparag If it is not controlled at the end of the the conflict, it may still
be chosen but then counts a full province.

\aparag[Overseas Wars]
A peace treaty ending an Overseas War may not involved change of ownership of
any province on the European mainland. Only transfers of \TP/\COL, provinces
in "Barbaresques" countries, \pays{Egypte} and \pays{Irak}, or Indemnities are
valid conditions of peace.
\bparag [TBD: pour annexer en Afrique, il faut une guerre normale, guerre
outremer permet juste les présidios + indemnités + diplomatie] [\textbf{PB:
  actuellement je suis en faveur de cette restriction}]

\aparag[Transfer of provinces of minor countries]
If a player, at war against an alliance, accepts to make peace, all his minors
do the same, except those at war by event.
\bparag If an alliance receives provinces by a favorable peace, provinces of a
minor ally fully involved in the war, or of a Vassal or Annexed minor power,
may be given to the victor, in place of provinces of the loser
alliance. Howevern these provinces should obey the conditions of rule
\ruleref{chSpecific:Tranfer Provinces Peace}.
\bparag All minor countries provinces ceded by the controlling player count as
his own for victory points.
\bparag If minor cedes provinces by this way, its diplomatic marker is
immediately move to the box \Neutral.



\subsection{Alliances of Major Powers}


\subsubsection{Case of Victorious Allies}
\aparag The number of 3 provinces transferred is a maximum. The sharing of the
spoils of war among allies is left to their free choice.
\aparag In case of disagreement, the power with the Monarch having the best
Diplomacy rating in the alliance (decide at random in case of ties) and whose
country presently has units in the loser's territory, is to take the final
decision of the proposition of Peace. Peace conditions are fixed on at a time
by each power in order of descending Diplomacy rating (so, at most 3 powers
decide).
\aparag However, any ally deprived of any provinces/indemnites when he is
effectively occupying some of the loser's provinces may ask his faulty
ally(ies) to break the alliance (with the loss of 2 \STAB for them) and
immediately receives a \CB against him/them.


\subsubsection{Separate Peace}
\aparag If several players are allied and at war together, some may want to
make separate peace, for instance if the conditions does not please them, or
to stop the war early. A Separate Peace is only allowed if one Major power in
the alliance wants to remain at war (it is not possible to sign Separate
Peaces in order to multiply the Treaties of Peace).
\aparag[Separate Peace and Casus Belli]
In this case, any other Allied player at war on the same side that the one
making a separate peace receives an immediate temporary \CB against this
player.
\bparag This \CB may only be used the turn immediately following the signature
of the separate peace or it is lost.
\bparag The Allied player making a separate peace breaks his military
Alliance, and suffers a loss of 2 levels of Stability.
\bparag Exception: If the Ally must make peace because of his Stability (Peace
Constrained, see above), he does not break his Alliance (and there is no \CB
against him).
\aparag Separate Peace can also be attempted to be signed with Minor
countries, using the mechanism for Peaces with Minor powers. For each alliance
that is at war against it, an alliance may offer a Separate Peace only to one
Minor power in the enemy alliance (and that can not be a Vassal, excepted if
the Capital of the Vassal is controled).



\subsection{General Consequences of the Peace}

\aparag Peace brings the conflict opposing the players to an end.
\aparag In Conditional or Unconditional Peace (not in Negotiated or White
Peace), the loser has not the right to declare a war against the victor for
the whole next turn, excepted if using a new \CB obtained after the Peace
treaty.  Existing \CB at the time of the Peace (even permanent ones) are
negated for both powers for one turn.
\aparag[Peace and Casus Belli]
Permanent \CB are canceled if their cause disappears with the war (e.g.
return of a national Province previously annexed). Temporary \CB obtained
before the end of the war disappears, excepted after a White Peace.
\aparag[Peace and Evacuation]
All units present on the territory of a former enemy are repatriated by their
owner to the closest friendly province of his choice, without having to pay
the cost of activation and rolling for attrition at {\bf -2}.
\bparag Established \Presidios are not given back during Evacuation.
\bparag However, a power may decide at any Peace Phase to dismantle one of its
\Presidios at no cost.

\aparag[Peace and Stability]
Finally, during the conclusion of a peace with a player or an attacking minor
(at war by event), the Stability increases by 1 level for each signatory
player that is now fully at peace.
\bparag In case of peace with multiple enemies, the Stability increase is set
anyway at a maximum of +1 even if peace with different players or minors is
made during the same peace phase.
\bparag A peace with a minor does not increase the player's Stability (unless
this minor has declared war by an event).

\aparag[Indemnities] Indemnites agreed in the Peace are to be paid by one of
the losing Major powers. He may pay them now or at a segment of Announcement
of the two following turns. They can be paid in fractions during the 3 turns
allowed.
\bparag Failure to pay all the due indemnities give a temporary free \CB to
the power that was wronged against the power that should have paid.

\aparag[Peace and Minors]
All active minors of players now at peace immediately sign the peace at the
same time.
\bparag
Exception: a minor at war declared by an event makes peace only through a
specific peace for this war against the ennemy alliance, usinf the procedure
for minor countries.  This is not considered as a separate peace (even if
involved in war alongside its controlling player).



\subsection{Peace with Minor powers}

\aparag To make the peace with a minor, the enemy player has to indicate that
he commits himself to peace negotiations with the minor.  The controlling
player of the \MIN may be at war (against the player) or not; in the former
case it is a "separate peace", attempted and signed before any peace between
players.
\bparag An alliance may usually offer peace with only one minor country per
alliance at war against it and per turn; and this minor may not be a vassal ir
annexed.
\bparag Exception: An alliance can always propose a specific peace to a minor
at war declared by an event .
\bparag Exception: An alliance can offer separate peace to any or all minor
vassal to the alliance or annexed, whose capital city is controls.
\bparag Exception: An alliance can offer separate peace to any or all
non-vassal, and non-annexed minors if controlling at least one of their
provinces (each one), or if each one of them controls individually at least
one province in the alliance.
\bparag If allied powers do not agree, the decision is taken by the Power
having the higher value in DIP (decide ties randomly).

\aparag[Method]
To sign peace with the minor, the player rolls 1d10, taking into account the
sought-after peace level and all applicable modifiers. The level of peace is
chosen by the player, without taking into account his own Stability.
\bparag[Result]
The peace is signed if the modified die-roll result is 6 or more.
\bparag[Peace level Modifier]
This modifier is the triple of the value of the peace level chosen. It is used
as a positive modifier if the offered peace is favorable to the minor, or as a
negative modifier if the offered peace is favorable to the player.
\bparag[Nationality Modifier]
It is applied for the case of conflicts with specific minor countries
which are: \\
-4: \pays{Perse}, \pays{egypte}, \pays{damas}, \pays{Chine},
\pays{Japon} \\
-3: \pays{USA}, \pays{Mogol}, \pays{Venise}, \pays{Pologne},
\pays{Habsbourg},  \pays{Brandebourg} after IV-11\\
-2: \pays{Portugal}, \pays{danemark}
\bparag[Modifiers of Situation]
These are is applied cumulatively according to what happened during the
current turn (and to what happened during previous turns for the
lost/conquered provinces only):\\
+2:	per province/\TP\faceplus/\COL lost by the minor (+4 if capital)\\
-2:	per province/\TP\faceplus/\COL conquered by the minor (-4 if capital) \\
+1:	per \TP\facemoins  lost by the minor (round down) \\
-1:	per \TP\facemoins conquered by the minor \\
+2: if the capital province of the \MIN was conquered this turn, or if it was
captured then lost since \\
-4: if the \MIN has captured a capital province of a \MAJ this turn, -or if it
was
captured then lost since \\
-2:	per major battle won by the minor  \\
+2:	per major battle won by the player \\
-1:	per battle won by the minor \\
+1:	per battler won by the player\\
+1:	per minor military leader killed or captured \\
+2: if the Monarch of the minor country is captured and its Ransom is
used for Peace \\
-1:	per military leader of the player killed or captured\\
+1:	per siege won by the player \\
-1:	per siege won by the minor \\
-2: If minor is heretic (Catholic vs. Protestant, before the end of
the \terme{Religious Dissension}) \\
-2: if it is an attempt to negotiate a separate peace $\pm$?: the PD of the
controller regarding the alliance attempting the peace (if the controller is
at war against the alliance), max. -3/+3
\bparag
All these modifiers are cumulative in one single turn.
\bparag \COL and \TP\faceplus controlled counts as a full province (modifier
of $\pm 2$); the same is true for control of cities of minor countries in the
\ROTW;
\bparag \TP\facemoins controlled gives a modifier of $\pm 1$ only; the same
modifier holds for occupying a province without city of a minor country in the
\ROTW;
%% was \pm 1, remis à 1.5 pour être comme les tables ou la liste de modifieurs
%% ci-dessus...
\aparag[Overseas Wars]
\bparag A Minor country always accepts to sign a White Peace in Overseas War
(if it is not a Separate Peace).

\aparag[Consequences of Peace]
If the peace is signed, no Stability level is gained (exception: if this minor
declared war to the player by an event). The player that controlled the minor
does not earn anything.
\bparag The conditions of Peace are the same as for a Peace between Major
powers.
\bparag A Minor country will at most indemnities up to 4 times its income,
immediately at the conclusion of the Peace. Any other indemnities are void.
\bparag If the minor country is the victor, the player that controls the minor
country chooses the ceded provinces (if any). He must do so in priority among
those located the closest from the minor country's territory, in terms of
movement points (a sea zone is equivalent to 2 MP for this calculation).
\bparag A minor country nevers takes Diplomatic concessions, only provinces
and indemnities.

\aparag[Multiple and Separate Peace]
If a player signs a separate peace with a minor country (he is still at war
against the controlling player); this minor may not be again involved in a war
against him next turn (unless by an event or a Crusade).
\bparag A minor country at war by event may only make a separate peace.

\aparag[Unconditional Peace]
A Minor country will sign a mandatory Unconditional Peace if all of its
provinces are controlled by the ennemy. This peace is one global peace against
all the powers controlling its provinces (so it can lose only 3 provinces).
\bparag If a minor country loses and signs an unconditional peace its
political marker is moved automatically to the box \Neutral.

\aparag[Automatic Peace]
An alliance that proposes a victorious Unconditional Peace to a Minor power,
and that Minor power was the attacking one (caused by event), or is not allied
to an alliance (so this is not a Separate Peace), the Peace is automatically
accepted by the Minor power.

\aparag[Failure of Peace negotiations]
If the die-roll is inferior or equal to 5 the war with the minor continues
normally for the following turn. Another peace attempt with that minor will be
allowed during the peace phase of next turn.




\section{Interphase and End-of-turn}

% RaW: [21, 51]

This is the last phase of the turn and is played simultaneously.  The Players'
State Prosperity is verified because it can influence positively or negatively
their Stability; the Stability is modified by ongoing wars or limited
interventions.  Remove military leaders scheduled to leave the game and take
those scheduled to arrive on the following turn.  Remove all un-named leaders
of major powers, and those of minor powers that are now at peace.  In this
phase, Inflation is determined and affects all players' treasuries. Inflation
increases according to the quantity of gold repatriated during this turn from
the \ROTW.
\aparag[Sequence.]
\InterDetails



\subsection{Prosperity and Stability adjustment}

\aparag[Effects of ongoing wars]
During a war, at the end of each turn, the \STAB of each participating player
is reduced 1 cumulative level each turn (i.e.. the player loses 1 level on the
first turn, 2 others on the second, 3 on the third and so
on... etc.). However, after 4 consecutive war turns, the loss is limited to -4
levels per turn, and this until the peace is made.
\aparag[Effects of Overseas Wars]
The same reduction of \STAB is applied in Overseas Wars excepted that the loss
is limited to -2 levels per turn.
\aparag[Multiple Wars] If a power is involved in more than one war, only the
hardest lost is applied.
\aparag[Continuing Limited Intervention]
A limited intervention in a war may be continued from one turn to another, but
this costs -1 level in \STAB for the intervening power.
\aparag[Turkey and the Knights]
If \TUR is neither at war, nor Anti-Prosperous, it loses -1 level in \STAB if
the pirate of The \pays{chevaliers} managed to inflict losses on Turkish
commercial fleets.
\aparag[Vienna]
% \aparag[\ville{Vienne}]
If \TUR controls \ville{Vienne}, \HAB (\SPA or \AUS) loses 1 additional level
of \STAB.

% \aparag[Prosperity]\label{chPeace:Prosperity}
% The player checks on his economic form the evolution of his \terme{Gross
% Income} on the last two consecutive turns (including the one just
% played). This evolution affects the player's Stability.
% \bparag[Prosperous power]
% If the gross income has progressed 2 consecutive turns, the Stability
% increases 1 level.
% \bparag[Anti-Prosperous power]
% If it has decreased 2 consecutive turns, the Stability declines 1 level.  PB
% 07/2008: changed to a modifier to Stab Improvement



\subsection{Placement of leaders}

\aparag[Ransom of Leaders]
Captured Monarchs are given back at this time.
\bparag For Major Powers, it costs them 2 \STAB and they give 200\ducats to
the ransoming power (this is mandatory).
\bparag For Minor Powers, their Monarch is ransomed against 50\ducats.
\bparag Alternativeley, a ransome Monarch of a minor power can be used at the
Peace Segment, to increases the modifier for Peace by a +2.

\begin{todo}
  DEPLACER la majorité de ça !!!!
\end{todo}
\aparag Generals and Admirals to be received (those stated as available from
the start of the following turn) are placed normally.
\bparag[Dates on Military Leader counters]
The date of arrival (turn number) is the figure located on the upper left-hand
side of the leader silhouette.  The date of departure (turn number) figure is
located on the lower left-hand side of the leader silhouette.
\bparag Placement of Newly arrived Leaders These leaders are placed on
military stacks of their owner, located in friendly unbesieged cities or
friendly province.
\bparag Leaders Removal from Game Leaders are removed from the game if their
date of departure corresponds to the turn just played (i.e. they leave the
game at the end of their scheduled turn of departure).
\bparag Remove all un-named leaders of Major Powers; leave those of Minor
Powers at war or those on Revolts (but keep all military forces of Minor
powers in play, even if at peace, for usage on next turn if they are again
involved in a war).

\aparag[Conquistadors and Explorers]
The Conquistadors and Explorers may either be placed in a national province of
the player, or on the \ROTW map, in a \COL/\TP of the player or a province
were a players' leader of the same type was at the end of the turn.
\bparag Explorers may use their full values when traveling from the European
map to the \ROTW map but cannot stay on the European map at the end of a
round, unless as the result of an interception.

\aparag[Governors, Overseas Generals and Admirals]
Generals and Admirals with \$ or @ symbol, or Governors, may only be placed in
existing \COL or \TP of the power. If there is none, they are delayed until
they can be placed.
\bparag Leaders bearing a symbol \$ may only be place in \continent{America}.
\bparag Leaders bearing a symbol @ may only be place in \continent{Asia}.



\subsection{Inflation}

\aparag The Gold counter placed on the ``Resources and Prices'' track
indicates the percentage of inflation currently in force. The maximum
inflation rate is 33\%.
\bparag This percentage of inflation is the proportion of the Royal Treasure
that is lost by each power at the end of the turn. It is reported in
\lignebudget{Inflation}.
\aparag[Increase of Inflation] The increase of inflation is controlled by one
die-roll. If successful (7 or more), this die-roll will increase the inflation
by 1 box to the right (unless inflation is already at its maximum).
\bparag Inflation can also increase due to specific economic conditions (see
\ruleref{chIncomes:EconomicInflation}) or various events.
\bparag \label{chPeace:InflationGoldROTW} If the quantity of gold in the \ROTW
that was produced this turn exceeds 100\ducats, the inflation counter is
turned on its other side before the die-roll (success on 3 or more) for the
current turn.
\bparag Inflation increase is done before discounting inflation from the \RT.
\aparag[Limited Inflation] \label{chPeace:InflationGold} Players that do not
produce gold in \continent{America} are affected by the inflation as if the
inflation marker was found one box to the left of the box it currently
occupies.
\bparag However, the minimum inflation is at least 5\% for all players.



\subsection{Trade centres}\label{chPeace:TradeCentres}

\aparag The Trade Centres may be moved during the interphase.
\aparag The \emph{Great Orient} centre does not move from \shortprovince{Nil}
as long as both \pays{damas} and \pays{egypte} are not conquered by \TUR.
\bparag At this point, the convoy of \bazar{Izmir} appears, and the
\emph{Great Orient} centre is moved to any coastal province of
\paysmajeur{Turquie} bordering the Mediterranean sea.
\bparag If the province is ceded by \TUR, the \emph{Great Orient} centre has
to be moved to another such province.
\aparag The \emph{Mediterranean} centre is given to the country that has the
greatest total number of fleets in the following zones: \stz{Caspienne},
\stz{Noire}, \stz{Lion}, \stz{Ionienne}, \ctz{Turquie}, \ctz{Venise}.
\bparag If the owner changes or if the province in which the Trade Centre
resides is ceded, the Trade Centre has to be moved to a national port of the
controller (bordering the Mediterranean Sea if there is any).
\bparag If no such province is eligible, the Trade Centre is temporarily not
available.
\aparag The \emph{Indian Ocean} centre is given to the country that has the
greatest total number of fleets in the following zones: \stz{Tempetes},
\stz{Oman}, \stz{Indien}, \stz{Formose}.
\bparag It follows the same rule as above (placed in a National port of the
controller).
\aparag The \emph{Atlantic} centre is given to the country that has the
greatest total number of fleets in all other \STZ and \CTZ.
\bparag It follows the same rule as above, but with the Atlantic Ocean instead
of Mediterranean sea.




% Local Variables:
% fill-column: 78
% coding: utf-8-unix
% mode-require-final-newline: t
% mode: flyspell
% ispell-local-dictionary: "british"
% End:
