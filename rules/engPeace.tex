% -*- mode: LaTeX; -*-

\definechapterbackground{Budget and Peaces}{victories}
\chapter{Budget and Peaces}\label{chapter:Peace}

\section{Overview of the phase}

% RaW: [50]
\aparag[Administration] At the end of the turn, final administrative actions
are resolved and budgets must be completed. First, exceptional taxes that were
scheduled during the administrative phase are resolved. Then comes the
exchequer test. At this point, players roll to determine how well the funds
were collected this turn and to discover their precise income. If the income
is not enough to cover for the expenses, loans must be contracted, either from
the people of your country or from international bankers. Last but not least,
countries may try to improve their \STAB.

\aparag[Peace] Wars can be ended only by a Peace. There are several types of
Peace, from the white peace (return to statu quo) to the unconditional
surrender. The type depends mostly on the difference between the \STAB of the
belligerents, slightly modified by the military situation. In some cases,
countries must propose peace to their opponents, but usually some discussion
occurs between the players.

\aparag[Crusade] In the early game, if \TUR conquers too many Christian
provinces, the pope may try to launch a Crusade.

% \aparag[Events and special peace conditions] A small number of political
% events can initiate wars where the peace conditions are constrained or
% modified by specific conditions. These events do not change the rules below
% for other simultaneous conflicts where one side of the war is not one of those
% described in the event.
% \bparag Remark that the other side must enter war through another means than
% the event or the alliances network called at the time of the event (else, it
% is considered part of the event).

\aparag[Sequence of the Peace Phase.]
\PeaceDetails

\section{Exceptional taxes}\label{chPeace:Exceptional taxes}
\aparag[Exceptional taxes] Exceptional taxes are scheduled during the
Administrative phase. See~\ref{chExpenses:Exceptional Taxes} for details (and
examples). They are resolved at this point only. That is, until the end of the
turn (and after most expenses have been planned), players won't know exactly
the amount of collected taxes.
\bparag Note that Exceptional taxes must be planned during Administrative
phase. If a country forfeited the possibility to do so, it is to late now to
decide to raise taxes.

\aparag[Resolution of the taxes]
\bparag Each country which has planned taxes should have written a modifier in
\lignebudget{Exceptional taxes modifier A} (copied from
\lignebudget{Exceptional taxes modifier B}). This modifier was \ADM + 3
$\times$ \STAB (at the time of the Administrative phase).
\bparag Roll 1d10, add the modifier and multiply the result by 10. This is the
amount of taxes (in \ducats).
\bparag Write this amount in \lignebudget{Exceptional taxes}. It may well be
negative if the modifier was negative. In this case, the country will actually
lose money because of the taxes. It is not possible to refuse a ``tax'' once
the amount is known.

\aparag[\RT before Exchequer test]
\bparag Players can know compute their \RT before resolving the Exchequer
test.
\bparag This is the sum of lines \ERSlong{RT after Diplomacy} +
\ERSlong{Pillages, privateers} + \ERSlong{Gold from ROTW and Convoys} +
\ERSlong{Exceptional taxes} of \EcoRS. It is written in \lignebudgetlong{RT
  before Exchequer}.
\bparag Players should also copy \lignebudgetlong{Gross income B} in
\lignebudgetlong{Gross income A} and \lignebudgetlong{Total expenses} in
\lignebudgetlong{Expenses}.

\section{Exchequer test}\label{chPeace:Exchequer test}
\subsection{Gross Income}
\begin{designnote}
  We explain here the technical rules of the economical system. For a
  description of the spirit of these rules, see~\ref{chThePowers:Exchequer}.

  The rules here are quite ``algorithmic'' in order to have them as precise as
  possible and avoid misinterpretations. Thus, there are not well suited to
  understand the whys of the system (only the hows). These rules are meant to
  be closely followed step by step. Check~\ref{chThePowers:Exchequer} in order
  to understand what should happen, as well as read some examples.
\end{designnote}

\aparag[Exchequer test] Each country roll a die on~\ref{table:Administrative
  Actions} modified as follows (cumulative):
\begin{modlist}
\item[+2] If completely at Peace (no war (including civil or overseas wars),
  no intervention (limited or foreign)).
\item[-1 ] per 100\ducats of National Loan.
\item[-1] per ongoing International Loan (whatever the amount, including the
  ones that are partially refunded).
\item[-1 ] per bankruptcy in the last 5 turns.
\item[-1] per loan treaty broken in the last 5 turns.
\end{modlist}
\bparag Find the result by cross-referencing the line of the modified result
with the column equal to the \STAB of the country.
\bparag The result may be either F\textetoile, F, \undemi, \undemi\textetoile,
S or S\textetoile.

\begin{playtip}
  Bankruptcies should be noted by a small \textetoile in \lignebudget{Gross
    income A} for the turns where they affect the Exchequer test.
\end{playtip}

\aparag[Percentages] By cross-referencing this result with the first three
columns of~\ref{table:Exchequer test}, countries obtain three percentages for
``Regular Income'', ``Prestige Income'' and ``National Loan''.
\bparag Add 10 to the ``National Loan'' of countries that are not completely
at peace.
\bparag Add 10 (cumulative) to the ``National Loan'' of \HIS if it has
declared a politic of expulsions (see~\ref{chSpecific:Spain:Expulsion}).
\bparag It is possible and intended that these percentages sum up to more or
less than 100\%.

\aparag[Incomes] Apply each of the three percentages to the whole Gross Income
(\lignebudget{Gross income A}), rounding down, to obtain three incomes.
\bparag Copy these incomes in \lignebudgetlong{Regular income},
\lignebudgetlong{Prestige income} and \lignebudgetlong{Max. national loan}.

\begin{playtip}
  It is often convenient to cut these three boxes in half (diagonally). After
  rolling the exchequer test, immediately copy the percentages in the top-left
  halves, this avoid forgetting the result. Next you can take your time to
  compute the actual value and write it in the bottom-right halves.
\end{playtip}

\GTtable{etatsauvrai}

\subsection{International Loans}\label{chPeace:International loans}
\aparag[Available money] The total amount of available money for international
loans is:
\bparag 50\ducats from the start (unspecified bankers).
\bparag Always add 50\ducats, or 100\ducats for the emperor (German bankers).
\bparag Always add 50\ducats, or 100\ducats for the diplomatic patron of
\paysGenes (Genoese bankers).
\bparag After~\ref{pIII:Amsterdam Stock Exchange} add 50\ducats, or 100\ducats
for \HOL.
\bparag After~\ref{pIV:London Stock Exchange} add 50\ducats, or 100\ducats
for \ANG.
\bparag Thus, the total available money will be between 150 and
350\ducats. Note that it does depend on the country, that is all the countries
have different loan capacities.

\aparag[International Loans test] Each country may roll a die
on~\ref{table:Administrative Actions} modified as follows:
\begin{modlist}
\item[+2] If completely at Peace (no war (including civil or overseas wars),
  no intervention (limited or foreign)).
\item[-1 ] per 100\ducats of National Loan.
\item[-1] per International Loan.
\item[-1 ] per bankruptcy in the last 5 turns.
\item[-1] per loan treaty broken in the last 5 turns.
\item[+1 ] if the country has a Stock Exchange (\HOL after~\ref{pIII:Amsterdam
    Stock Exchange} and \ANG after~\ref{pIV:London Stock Exchange}).
\end{modlist}
\bparag Find the result by cross-referencing the line of the modified result
with the column equal to the \STAB of the country.
\bparag The result may be either F\textetoile, F, \undemi, \undemi\textetoile,
S or S\textetoile.
\bparag Note that this roll is different from the Exchequer test. Do not use
the same roll for both the Exchequer test and the International Loans test as
this would increase the chances of extremely bad results.

\aparag[International Loan] By cross-referencing this result with the last
column of~\ref{table:Exchequer test}, countries obtain one percentages for
``International Loan''.
\bparag Apply this percentage to the total available money and copy the result
in \lignebudgetlong{Max. international loan}.

\begin{playtip}
  Often, International loans are not necessarily and this step may be skipped
  by most countries. It may be useful to start computing your budget (next
  step) before deciding whether to take an international loan or not. Hence,
  it is sometimes more fluent to start computing your budget and then possibly
  come back to looking at international loans. Since there is no new knowledge
  gained between the Exchequer test and the Budget, this does not change
  anything.

  If you wish to follow closely the order of the steps, you should, however,
  always roll for international loan preventively, thus avoiding bad
  surprises.

  Rolling for international loan do not force to take one. It is always
  possible to decline a new international loan after rolling the die and
  seeing the available amount.
\end{playtip}

\section{Budget}\label{chPeace:Budget}
\subsection{Expenses}
\aparag[Regular income] Write in \lignebudget{Remaining expenses} the
difference between \lignebudget{Expenses} and \lignebudget{Regular income}.
\bparag This may be a negative number in the rare case where the Regular
income is larger than the total expenses.

\aparag[Prestige income] Write in \lignebudget{from Prestige} any non-negative
number smaller than both \lignebudget{Prestige income} and
\lignebudget{Remaining expenses}.
\bparag Small value means that more money is spent for prestige \VPs and less
for day-to-day expenses. Those will be covered by loans or debt.

\begin{designnote}
  You cannot spend additional money for prestige (it must be non-negative).
  You cannot take more from prestige than the ``Prestige Income'' (smaller
  than \lignebudget{Prestige income}).  You cannot take more from prestige
  than what is left to pay after the regular income is spent (smaller than
  \lignebudget{Remaining expenses}).
\end{designnote}

\aparag[National Loans] Write in \lignebudget{from N. loan} any non-negative
number smaller than \lignebudget{Max. national loan}.
\bparag Copy this number in \lignebudget{New National loans}.

\begin{designnote}
  National Loans are not limited by expenses. However, you'll have to pay
  interest for them and maybe even refund your people someday.
\end{designnote}

\aparag[National Loans] Write in \lignebudget{from I. loan} any
non-negative number smaller than \lignebudget{Max. international loan}.
\bparag Copy this number in \lignebudget{New International loan}.
\bparag Copy this number in \lignebudget{International loans refunds},
\textbf{three turns} after the current one.
\bparag Copy 10\% of this number (round up) in \lignebudget{International
  loans interests} for the \textbf{next three turns}. If there is already a
number in one of these boxes, add the new value to it.
\bparag That is, you should write 3 interests (for the next three turns), and
one refund (for the same turn as the last interest).

\begin{playtip}
  International loans are usually a bad idea because of the scheduled
  mandatory refund. Use them only when in need.
\end{playtip}

\begin{exemple}
  A correctly filled new international loan (of 100\ducats, at turn $n$) over
  an existing one (of 200\ducats, from turn $n-2$):
  \begin{tabular}{|c|l|r|r|r|r|r|}
    \hline
    & Turn & $n-1$ & $n$ & $n+1$ & $n+2$ & $n+3$\\
    \hline
    1 & New International loan & & 100 & & &\\
    \hline
    2 & I. loan interest & 20 & 20 & 30 & 10 & 10\\
    \hline
    3 & I. loan refunds & & & 200 & & 100\\
    \hline
  \end{tabular}
\end{exemple}

\aparag[New \RT]
\bparag Write in \lignebudget{RT balance} the sum of \lignebudget{from
  Prestige} + \lignebudget{from N. loan} + \lignebudget{from I. loan}
\textbf{minus} \lignebudget{Remaining expenses}. It may be negative if
\lignebudget{Remaining expenses} is too big.
\bparag Write in \lignebudget{RT after Exchequer test} the sum of
\lignebudget{RT before Exchequer} + \lignebudget{RT balance}.

\begin{designnote}
  \lignebudgetlong{Remaining expenses} depict \textbf{expenses} that are left
  to be paid after using the Regular income. Hence it is subtracted from the
  \RT while other lines are added (they are money taken from prestige or loan
  in order to fill the treasury).

  If \lignebudget{Remaining expenses} is \emph{negative}, regular income was
  enough to cover all expenses. Then, the surplus is added to the treasury (as
  subtracting a negative number result in an addition).
\end{designnote}

\begin{designnote}
  All in all, do not try to understand all the steps here while reading the
  rules. After a couple of turns of computing your budget, things will become
  more natural. Note that if you are having a ``teaching session'', you should
  try several ``stupid'' things with your budget to see the consequences.
\end{designnote}

\begin{playtip}
  When planning expenses, it is obviously a good idea to keep an eye on the
  possible income\ldots Too many expenses result in bankruptcy while too few
  result in money ``wasted'' for prestige (instead of being use for buying
  troops or waging war).

  Here are some guidelines in preparing your budget:
  
  First, check in the administrative actions table what are the possible and
  plausible results with respect to your current (and expected) \STAB. You may
  discard very unlikely results (with only 10\% chance of happening) but you
  know you take a risk doing so. It is especially important to take into
  account the worse possible result you may obtain if you want to limit risks.

  Second, check in the Exchequer test table the sum of percentages these
  results produce. Check separately the sum of Regular + Prestige income
  (income without debt) and the sum of the three percentages (income with
  debt). Applying these percentage to your Gross Income will give some amount
  of money.

  Do not spend more than your best income with debt, obviously, doing so
  will result in problems. Spending more than the worse income with debt means
  taking risks. Estimate the risks (Is it a 10\% or 30\% chance of getting the
  worse result?) compared to the situation (Do you have lot of money in your
  \RT to handle the loss?) and the expected gain (Will the extra expense allow
  you to win the war?)

  Spending less than the worse income without debt means that some money will
  necessarily go into Prestige \VPs. Are you sure it won't be better used for
  troops, economical development, \ldots? Spending less that the best income
  without debt means that you may get Prestige \VPs but they are not
  guaranteed either.

  The good cases is when the worst income with debt is roughly equal (or
  larger) to the best income without debt. Spending that amount of money means
  that the worse that can happen is to take a new loan (that can be handled
  later) and that you won't waste too much money on Prestige. Note that you
  have to plan your administrative actions and loan refund before the military
  phase, thus without knowing precisely how long the turn will last and how
  much you'll spend for moving troops (especially if at war). Thus, there is
  often some risk involved\ldots

  \smallskip

  Remember that the economical system works best if you have some loan that
  you refund and recontract immediately (for a net effect of transferring
  Prestige income into the \RT). If you plan to use this loan trick, then the
  amount of loan involved is not really a debt, that is increase you income
  without debt by this amount when planning your expenses.

  \smallskip

  Remember that the worse that can happen is a \RT collapse. But even for that
  you need several turns of bad luck, bad management, or bad wars. Thus, don't
  be afraid of making too big errors with the economical system. You should
  get the hand of it before catastrophic results occur\ldots
\end{playtip}

\begin{exemple}
  If your \STAB is +2 and your are at peace (\bonus{+2} to Exchequer test),
  then you'll likely to get \undemi\textetoile, S or S\textetoile (with only
  10\% chance of \undemi). \undemi\textetoile has 100\% income with debt while
  S\textetoile has 100\% income without debt. Thus, by spending as much as
  your Gross income, you're almost guaranteed to be able to cover your
  expenses, maybe with some new loans. There is a small risk (10\%) of a bad
  result (\undemi) that will leave you with only 80\% income. Estimate the
  risk versus gain for the last 20\% of expenses. On the other hand, a good
  result gives you up to 120\% with debt, hence some choice on whether to
  contract loan in order to get more Prestige.

  \smallskip

  If your \STAB is -2 and you roll at \bonus{-3} due to heavy loans or
  previous bankruptcies, then the likely result are F\textetoile, F or
  \undemi\ (disregarding the unlikely \undemi\textetoile). If you are at war,
  the income with debt for F\textetoile is 80\%, and the income with debt of
  \undemi\ is 90\%. Thus by spending around 80\% of your Gross Income, you're
  sure to be able to fill your budget with some loan. But you're also sure to
  need some new loan\ldots (and a good surprise may arise in the form of
  \undemi\textetoile).

  \smallskip

  Note that the true difference in the table is between \undemi\ (only 50/80\%
  of the total) and \undemi\textetoile (70/100\%). Especially, being at peace
  with a \STAB of +3 guarantees a good result.
\end{exemple}

\subsection{Loan Management}
\aparag Players must then correctly take care of their loans for the next
turn.

\aparag[International loans]
\bparag Since the interests are not changed by partial refund of the capital,
management of the international loans is entirely done during the
administrative phase (when bankrupting or refunding) and the budget segment
(for new loans).

\aparag[National loans]
\bparag Compute in \lignebudget{National loans at end} the difference between
\lignebudget{National loans at start}, minus \lignebudget{National loans
  bankruptcy}, minus \lignebudget{National loans refunds} and add
\lignebudget{New National loans}.
\bparag Report this number in \lignebudget{National loans at start} of the
next turn.

\subsection{Prestige and Wealth}\label{chPeace:Prestige and Wealth}
\aparag[Wealth] During each period, a global wealth is computed for each
country. Wealth represent the overall economical situation of the country, as
well as exceptionally good management (in the form of Prestige).
\bparag At the end of each period, wealth is converted into \VPs. Each country
has a different rate of exchange of wealth for \VPs as each country has
different typical economical situation.
\bparag All in all, each country is expected to score around 100\VPs for
wealth each period, give or take a few dozens if this is supposed to be a
period of glory or decay.

\aparag[Prestige] Write in \lignebudget{Prestige VPs} the difference between
\lignebudgetlong{Prestige income} minus \lignebudgetlong{from Prestige}. That
is the remaining Prestige income that was not spend for covering daily
expenses.

\aparag[Wealth] Turn wealth is the sum of the Gross income and the Prestige
\VPs. Period wealth is the sum of all turn wealth over all the period.
\bparag Write in \lignebudget{Wealth} the sum of \lignebudget{Gross income A}
and \lignebudget{Prestige VPs}.
\bparag Write in \lignebudget{Period wealth} the sum of \lignebudget{Period
  wealth} of the previous turn and \lignebudget{Wealth} of the current turn.
\bparag Exception: If this is the first turn of a period, simply copy
\lignebudget{Wealth} into {Period wealth}. That is, period wealth is reseted
at each period.

\section{Stability Improvement}\label{chPeace:Stability Improvement}

\label{chPeace:Stability Improvement}
\aparag[Stability] A country may attempt to improve its Stability, but this is
never mandatory. As many actions, Stability improvement requires an investment
and is resolved by a die roll. Beware that in some situations the result may
be negative and cannot be forfeited once the die has been rolled.
\bparag Countries whose monarch was just overthrown due to revolts (see
\ref{chRedep:Execution Monarch by Revolts}) may not do a \STAB improvement
action this turn.

\aparag[Investment] Each player wanting to improve the Stability of his
country first chooses an investment and writes it in \lignebudget{Stability
  improvement}. As for administrative actions, higher investments give bonuses
to the roll.
\bparag The investment are:
\begin{modlist}
  \item Basic Investment: 30 \ducats
  \item Medium (\bonus{+2} to the die-roll): 50 \ducats
  \item Strong (\bonus{+5} to the die-roll): 100 \ducats
\end{modlist}

\aparag[Procedure]
This action is resolved without requiring a table.  The player rolls a die
modified as follows (all modifiers are cumulative):
\begin{modlist}
  \item[+?] \ADM monarch.
  \item[+2/5] if medium/strong investment.
  \item[+2] if the country was victim of a declaration of war this turn
    without having broken an alliance or declared a war itself.
  \item[-3] if the country is at war with at least one major country
    (including overseas wars but excluding interventions).
  \item[-2] if the country is at war with at least one minor country and no
    major country (including overseas wars but excluding interventions).
  \item[-5] if an enemy \ARMY counter is in an owned national province and
    controls the city (not applicable during a Religious/Civil War, do not
    count revolt and rebel troops).
  \item[-3] Exception: for \SPA, the malus for having an enemy \ARMY counter
    controling the city, is -3 only, however it applies for any owned
    territory (not only its national territory). This specificity ends with
    \eventref{pIV:Olivares} (if effects are applied), or with
    \eventref{pV:WoSS} (whatever the choices and outcomes).
  \item[+3] for a Prosperous Power (see below).
  \item[-3] for an Anti-Prosperous Power (see below).
  \item[$\pm$?] by event.
\end{modlist}

\begin{designnote}[Spanish empire]
  The early Spanish empire was more of a multicultural empire including both
  Spain, Italy and the Netherlands than a modern country. Hence, occupying
  any part of the empire will hurt some people (and hamper Stability). There
  is no real notion of national territory to defend at all cost opposed to
  more distant vassals and ``colony''. However, only part of the empire is
  shocked by the war, thus the malus is smaller. Olivares policies recentred
  the empire on Spain, making it more like other European powers of the time.
\end{designnote}

\aparag[Result] If the modified result is equal to:
\begin{modlist}
  \item[5-] the Stability \textbf{decreases} by 1.
  \item[6-10] Nothing changes.
  \item[11-14] the Stability increases by 1.
  \item[15-17] the Stability increases by 2.
  \item[18+] the Stability increases by 3.
\end{modlist}
\bparag Reminder: Stability varies from -3 to +3. It is not possible to
decline the result (especially the loss of Stability) once the die has been
rolled.
\bparag Stability is recorded on the Stability track on the \ROTW map. Move
the Stability marker according to the result of the action.

\aparag[Prosperity]\label{chPeace:Prosperity} tracks the evolution of the
Gross income (as recorded in \lignebudget{Gross income A}). A regular increase
of the Gross income will make people happy and ease Stability improvement, a
regular decrease will make people unhappy.
\bparag[Prosperous Power] A country is \terme{Prosperous} if its Gross income
has not decreased during the last 2 consecutive turns and progressed during at
least one of those turns.
\bparag[Anti-Prosperous Power:] A country is \terme{Anti-Prosperous} if its
Gross Income has decreased 2 consecutive turns.

\begin{exemple}[Prosperity]
  If the Gross income for the last two turns and the current one are:
  \begin{itemize}
  \item 100, 110, 120: the country is prosperous.
  \item 100, 100, 101: the country is prosperous (no decrease, at least one
    increase).
  \item 100, 110, 109: nothing (one decrease prevents prosperity even if the
    final result is higher than two turns earlier)
  \item 100, 99, 98: the country is anti-prosperous.
  \item 100, 99, 99: nothing (one stagnation prevents anti-prosperity).
  \end{itemize}

\end{exemple}

\section{Ransoms}\label{chPeace:Ransoms}
\aparag[Majors] If a \MAJ has its monarch (or Swedish heir) captured (due to
battle), it \textbf{must} pay a ransom. The monarch is immediately liberated.
\bparag The ransomed country loses 2 \STAB and pay 200\ducats to the ransoming
country.
\bparag If the monarch was captured by a minor country, the money is lost
(minor countries do not have treasures).
\bparag It is not possible to avoid ransom in any way. No keeping prisoners,
no execution, \ldots even if both the ransoming and ransomed countries agree.
\bparag Thus, ransom may cause a later bankrupt or an immediate mandatory
peace. Do not risk your monarch if you cannot afford the price.

\aparag[Minors] If a \MIN has its monarch captured by a major country, he
\textbf{must} be ransomed.
\bparag The major holding the prisoners chooses one (and only one) ransom
among:
\begin{modlist}
\item 50\ducats.
\item \bonus{+2} to a peace proposal.
\item possibility to do a separate peace proposal.
\end{modlist}
\bparag If a minor monarch is captured by another minor country, he is
automatically ransomed for free (some money transfer between minors, not
represented).

\aparag Money gained or lost due to ransoms is written in \lignebudget{Ransom,
  peace}.

\aparag[The return of the king]
\bparag Ransomed major monarchs are immediately available again as per usual
rules.
\bparag Ransomed minor monarchs are placed back in play during the next
Interphase together with new leaders.

\section{Peace offers and discussions}\label{chPeace:Peace offers}
\subsection{Signing Peaces}
Countries at war (either major or minor) may sign peaces. Peaces are usually
done between two alliances and not between single countries (each alliance may
contain one or more country). Separate peaces are possible but usually
harder. Peace between major countries (and their minor allies) are the result
of an agreement between players. However, the Stability of the countries and
the military situation creates a \terme{Peace Differential} and strongly
constrains the peace. This represents the overall opinion of the countries
toward the current war and prevents players from signing unrealistic
peaces. Peaces when one side only consist in minor countries (most of the
time, a single one) are resolved by a die roll depending mostly on the
military situation.

\subsubsection{Regular cases}
\aparag[Global peace] If two alliances are at war, they may sign a global
peace between them.

\aparag[Separate peace between majors] If two alliances are at war, some
powers may sign peace with the whole enemy alliance.
\bparag Powers signing separate peaces are considered as breaking their
alliance (loosing 2 \STAB and giving a \CB to former allies as
per~\ref{chDiplo:Alliance:Defensive Alliance}).
% avoid taking 3 provinces to 2 different enemies and stay at war with the
% third
\bparag If several members of the same alliance want to sign a separate peace
with the same enemy alliance at a given turn, they must sign one single
separate peace.

\aparag[Minor allies] usually sign peace when their diplomatic patron does.
\bparag However, the diplomatic patron may choose to do a separate peace
without some of its minor allies. In this case, the major loses 2 \STAB for
the separate peace and the diplomatic control of the minors staying at war.

\begin{exemple}[Separate peace]
  \TUR is at war against \VEN, \HIS (and \AUS) and \POL. After an incursion in
  Hungary, \provinceVeneto itself is threatened, thus \VEN would like to sign
  peace before it's too late. On the East side, \RUS is massing troops along
  the Polish frontier and \POL would also like to get out of here in order to
  defend its border. On the other hand, \HIS and \AUS have not suffered much
  and want to stay at war.

  \TUR may choose to accept the separate peace either with \VEN alone, or with
  \POL alone, or with both \VEN and \POL together (treating this as a peace
  with an alliance). In any case, the powers signing the peace (\VEN or \POL)
  are breaking their alliance with allies staying at war (\HIS) and thus lose
  2 \STAB and give a \CB to these allies for the next turn.

  Any minor allies of \VEN or \POL (signing the peace) is also included in the
  peace. Minors allies of \TUR are also part of the peace.
\end{exemple}

\aparag[Proposing separate peace with minor] An alliance may propose separate
peace with minor allies of an opposing alliance at the following conditions:
\bparag An alliance may propose a separate peace to any minor in \VASSAL or
\ANNEXION of one enemy if the alliance controls the capital of the minor.
\bparag An alliance may propose a separate peace to any minor in \VASSAL or
\ANNEXION of one enemy if the minor controls the capital of one major of the
alliance. In this case, it must be a winning peace (level 1 or more) in favour
of the minor.
\bparag An alliance may propose a separate peace to any minor of one enemy if
it has captured the monarch of the minor and chooses to ransom it for a
separate peace.
\bparag An alliance may propose a separate peace to any minor \textbf{not} in
\VASSAL or \ANNEXION of one enemy if it controls any province of the minor.
\bparag An alliance may propose a separate peace to any minor \textbf{not} in
\VASSAL or \ANNEXION of one enemy if the minor controls any province of one
major of the alliance. In this case, it must be a winning peace (level 1 or
more) in favour of the minor.
% should be in each event description to be sure.
% \bparag An alliance may sign a unconditional peace (peace of level 5) in
% favour of any minor that declared war by event.
\bparag In addition, each alliance may propose a separate peace to one and
only one minor ally of each opposing alliance, \textbf{not} in \VASSAL or
\ANNEXION.

\aparag[Signing separate peace with minors]
\bparag As all peaces with minors, separate peaces with minors are resolved by
a die roll.
\bparag Contrarily to separate peaces with majors, each separate peace with
minors is resolved independently.
\bparag However, in case of a war against an alliance composed only of minors,
it is not possible to sign a separate peace at the same turn as the global
peace.

\begin{exemple}[Separate peaces with minors]
  \TUR, allied to \paysMaroc and \paysTripoli, with \VASSAL\ \paysAlgerie and
  \paysTunisie is at war against \HIS, allied to \paysVenise with \VASSAL\
  \paysChevaliers. \HIS controls \provinceJebelTubqal (in \paysMaroc),
  \provinceOran (in \paysAlgerie) and \provinceIfriqiya (capital of
  \paysTunisie). \TUR does not control any Christian provinces.

  \TUR may not propose peace to \paysChevaliers as it is a \VASSAL. It may
  propose peace to \paysVenise.

  \HIS may propose peace to \paysMaroc because it controls one of its
  provinces. \HIS may propose a peace to \paysTunisie, even through it is a
  \VASSAL, because it controls its capital. \HIS may not propose peace to
  \paysAlgerie because it is a \VASSAL and even if it controls one province,
  it does not controls the capital. It may, in addition, propose peace to
  \paysTripoli as each alliance is always entitled to one separate peace with
  one enemy minor at no condition. 

  Thus, \HIS may propose up to three separate peaces with minors. If it does,
  each of these peaces is resolved separately.
\end{exemple}

\subsubsection{Mandatory peaces}\label{chPeace:Mandatory peaces}
\aparag[Mandatory peaces between majors] It is usually not mandatory to sign a
peace, however:
\bparag If a country is at -3 \STAB for two consecutive turns at the beginning
of the peace segment, it must \textbf{propose} a peace to each alliance
(containing at least one major) against which it was at war during these two
turns. Note that the check happens \textbf{after} \STAB improvement, thus
mandatory peace usually occur because of a failed improvement (or a ransom).
\bparag Exception: \RUS, before its military reform, is only forced to propose
peace if it is at -3 \STAB for 3 consecutive turns.
\bparag The opposing alliance is not forced to accept the peace. It the peace
is refused, there is no penalty.
\bparag Exception: if the level of the proposed peace (see below) is 4 or 5 in
favour of the enemy, then the enemy is forced to accept it (this is basically
an unconditional surrender).
\bparag If two powers at war against one another must both propose a mandatory
peace, then the peace must be signed.
\bparag The peace proposal is made based on the \terme{Peace Differential} as
any regular peace. That is, the country is forced to proposed a peace but the
other regular rules for peaces are still enforced. This is not necessarily a
surrender, and in some cases it is even possible to be forced to proposed a
winning peace\ldots

\aparag[Mandatory peace and alliances]
\bparag If several members of the same alliance must propose a mandatory
peace, they must propose it together (as usual with separate peaces).

\aparag[Mandatory peaces and global peaces] Note that if a global peace is
signed, no separate peace may be signed first. Thus, mandatory peace proposals
only happen if the global peace is not signed.

\aparag[Mandatory peace and alliances]
\bparag If a power is forced to propose a peace and that peace is accepted,
that power is not considered to have broken alliance.
\bparag Especially, this does not give a \CB to its former allies.

\aparag[Mandatory peaces with minors]
\bparag If all provinces of a minor are controlled by enemies (not necessarily
the same alliance), then the minor automatically signs a mandatory
unconditional surrender (peace of level 5) with all its enemies together. That
is, this is one global peace and not one surrender against each enemy.
\bparag It is not possible to refuse that peace. In case of disagreement
between the winners, they are considered allied for the resolution of the
peace only.
\bparag If the minor was at war allied to a major, it immediately goes to
\Neutral before resolving the peace (the minor consider that its patron should
have protected it).
\bparag If an alliance of minors is at war with no major ally, it
automatically accepts an unconditional surrender (peace of level 5) in its
favour if any enemy proposes it.

\subsubsection{Other specific cases}
\aparag[Tri-partite wars]
\bparag If three (or more) alliances are at war against one another, each
peace signed is only signed between two alliances. The others stay at war.
\bparag It is of course possible that all alliances at war decide to sign
peace at the same moment.

\aparag[Events and peaces]
\bparag Many events create wars with specific conditions with regard to peace,
including:
\begin{modlist}
  \item Specific way to end a war, that is, specific conditions enforcing
    mandatory peaces.
  \item Specific peace conditions that may be taken, in addition to the
    regular one (described below).
  \item Specific peace proposal that will automatically be accepted by some
    minor countries.
\end{modlist}

\aparag[Disagreements]
\bparag If members of an alliance do not agree toward signing a peace, all
decisions concerning the proposal and acceptation of the peace are taken by
the country whose monarch has the higher \DIP (resolve ties at random) among
those (of that alliance) involved in the proposal (that is, you have nothing
to say about a separate peace made by your ally, except threatening it of
later reprisals, but threats have no in-game effect).
\bparag Note that effectively, the monarch with higher \DIP takes all the
decisions alone and is in no way forced to listen to his allies (however, do
not complain that nobody wants you as an ally if you keep ignoring them).
\bparag Only countries that are fully at war are considered. That is,
countries in limited or foreign intervention may not impose their will to
their allies and have a purely consultative say in the peace discussion. 

\begin{exemple}[Disagreements]
  \FRA and \SPA are at war against \HOL and \ANG. \FRA and  \HOL both have
  higher \DIP than their ally.
  \begin{itemize}
  \item If \HOL wants to sign a global peace (\emph{e.g.} because
    \villeAmsterdam is besieged) while \ANG wants to stay at war (because it
    think situation in the \ROTW will become better), \HOL may impose its
    decision to \ANG and sign the peace.
  \item If \HOL proposes a separate peace that \FRA wants to accept but \HIS
    would like to refuse, \FRA may impose its decision.
  \item If \ANG wants to sign a separate peace, \HOL has nothing to say about
    it and may not force it to stay at war.
  \end{itemize}
\end{exemple}

\aparag[Timing for the insanes]
Separate peaces between two alliances are considered simultaneous. Especially,
a power signing a separate peace with an enemy alliance is still allowed to
discuss any separate peace proposal from this alliance. Peace agreement may be
global (as in ``I sign this separate peace only is this one is only
signed''). Remember that in case of disagreement, the countries stay at war
and that's all.
\bparag Precise peace timing:
\begin{enumerate}
\item Global peace proposals and discussions between majors. All proposal and
  agreement are simultaneous and it is not possible to wait for a peace before
  signing another.
\item Separate peace proposals and discussions between majors, including
  mandatory separate peaces. All proposal and agreements are simultaneous.
\item Peace with minors, including separate peaces with minor allies. All
  proposal are simultaneous before any die is rolled.
\end{enumerate}

\begin{exemple}[Continued]
  \begin{itemize}
  \item If both \HOL and \HIS want to sign a separate peace with their
    enemies, that \FRA and \HIS are ready to accept the Dutch peace but \ANG
    would like to stay at war against \HIS, then \HOL is still part of the
    peace discussion and may force \ANG to accept the Spanish peace at the
    same time that it itself sign peace with \FRA.
  \item In the same situation, \HIS may decide that is separate peace its
    valid if and only if the Dutch peace is accepted. Typically if \ANG and
    \HOL try to buy \HIS out of the war by offering it an advantageous peace,
    \HIS may link it to the peace with \HOL in order to avoid leaving \FRA
    alone against two enemies.
  \end{itemize}
\end{exemple}

\aparag[Cultural agreement]
\bparag Peace agreements may include promise for future actions or agreements
on future Diplomatic phases.
\bparag It is, however, not possible to immediately sign any agreement (loan,
dynastic alliance with exchange of provinces, military alliance, \ldots)
Hence, it is always possible to ``forget'' about these between the signature
of the peace and the next Diplomatic phase. Again, do not complain that nobody
loves you if you keep forgetting your agreements (Europa Universalis is a long
term game and treason is often a bad strategy).
\bparag Such promises do not have to be publicly announced and may be kept
secret between players (even from allies). Thus, they are often jokingly
referred as ``cultural agreements'' as they have no in-game effect (only a
promise between players).

\begin{playtip}[Peace discussions]
  Peace discussions may last for a long time, especially for big wars
  including many countries. It is advised to try and minimise the time
  involved for peace discussions and keep the negotiations for the Diplomatic
  phase. However, evaluation of the new situation is required and some complex
  transactions are not uncommon. Discussions should be kept focused on the
  current peace and not diverge toward long term agreements (these are best
  suited for the Diplomatic phase).

  Players may isolate themselves from other players in order to discuss
  peaces. Either allies wanting to prepare a common proposal or enemies
  wanting to discuss secret clauses without third party players interfering in
  the discussion. Private discussions do not need to include all members of a
  given alliance\ldots As a rule of thumb, peace discussions between enemies
  is faster if there are no other players listening and commenting the
  proposals, trying to influence it. However, do not hesitate to ask advice
  from other players to check if some proposal is as balanced as it
  seems. Especially, inexperimented players may have hard time to grasp all
  the consequences of some agreement and may want to consult an experimented
  neutral player\ldots
\end{playtip}

%% No possibility to make a white peace at will.
% \subsection{The Informal Peace}

% \aparag The informal peace is concluded following a mutual agreement between 2
% players or more, and has to be announced to all other players.
% \aparag[Consequences]
% None of the players earns any VP for an informal peace. The war stops
% immediately.
% \bparag This type of peace can comprise any clause for which there was mutual
% agreement between the involved players, in the limit authorized by rules on
% Agreement. The Agreement is applied now.
% \bparag This type of peace can not be concluded with a minor.

\subsection{Peace differential}
\aparag[The \terme{Peace Differential}] is an abstract way of determining the
winner of any war between majors. It is mostly based on the \STAB of the
countries involved, representing the people support for the war, slightly
modified by the military situation.
\bparag In case of separate peace, the \terme{Peace Differential} is computed
only between the countries involved in the proposal.
\bparag \terme{Peace Differential} strongly constrains the possibility of
peace.

\aparag[The basic Peace Differential] is the difference between the \STAB of
the enemies.
\bparag In case of alliance, take the mean \STAB of all members of the
alliance. Do not round numbers at this point.
\bparag Note that the basic PD is symmetrical, that is if an alliance has a
basic PD of +1.5 versus another alliance, then the second alliance has a basic
PD of -1.5 versus the first.

\aparag[The modified Peace Differential] is obtained from the basic PD by
checking the military situation.
\bparag The alliance that controls more enemies provinces adds (and the other
subtracts) to its basic PD:
\begin{modlist}
  \item[+1] if it controls 2 or 3 more provinces.
  \item[+2] if it controls 4 or 5 more provinces.
  \item[+3] if it controls at least 6 more provinces.
\end{modlist}
\bparag Count capitals as 2 provinces.
\bparag Do count provinces of minor allies (or provinces controlled by minor
allies) together with those of its diplomatic patron.
\bparag Count \COL and \TP as \undemi\ province. Exception: \COL of level
6 are considered as European provinces and count as a full province.

\begin{exemple}[Modified Peace Differential]
  \RUS is at war against allied \TUR and \SUE. The \STAB are 1 for \RUS, 0 for
  \SUE and 1 for \TUR. Thus, the basic PD is 0.5 (1 - (1+0)/2) in favour of
  \RUS.

  \RUS occupy Swedish \provinceNeva and \provinceKarelen but \TUR occupy both
  \provinceAstragan and \provinceTerek (annexed by \RUS a long time
  ago). Both side thus controls as many enemy provinces and the PD is not
  modified.
\end{exemple}

\begin{exemple}[PD and separate peaces]
  In the same situation, if \RUS wants to sign a separate peace with \SUE,
  then its basic PD is 1 (1-0, the \STAB of \TUR does not count). Since this
  peace is only with \SUE, provinces controlled by \TUR are not taken into
  account. \RUS controls 2 more provinces than \SUE, and the PD in its favour
  is increased by 1 to 2.

  On the other hand, if \RUS wants to sign a separate peace with \TUR, the
  basic PD is 0 (they both have 1 \STAB) modified to -1 as \TUR controls two
  more provinces.
\end{exemple}

\begin{exemple}[PD and minors]
  If \paysCrimee was at war allied to \TUR and \RUS controls \provinceCrimee,
  this province has to be taken into account for modified peace differential
  in any peace that include \TUR. Since it is a capital, it counts as 2
  provinces. Thus, the modified PD of \RUS against the alliance is now +1.5,
  and against \TUR (in case of separate peace), 0.
\end{exemple}

\aparag[Military situation in overseas war]\label{chPeace:Privateer Effect}
\bparag During overseas wars, count occupied \COL and \TP as one province
each.
\bparag[Privateer effect] In addition, each \TradeFLEET of maximum level 4 or
more which was reduced to current level 0 or 1 counts as 1 province (2 in the
country own \CTZ).
\bparag Do count all \CTZ/\STZ where \TradeFLEET have been reduced without
remembering who caused the losses.

\begin{designnote}[Privateer effect]
  Privateer effect is triggered even if the losses were caused by
  \pays{pirates} or a third party \corsaire (typically, one of
  \Barbaresques), which may seem illogical. However, \corsaire are only a
  partial and abstract representation of the actual privateer activity. It is
  assumed that the real activity is more widespread, including in zones where
  no counter was send. Moreover, the target country probably doesn't know for
  sure who attacked each of its merchants. Or doesn't make a real difference
  between pirates and enemy privateers\ldots
\end{designnote}

\begin{exemple}[Privateer effect]
  \FRA and \ANG are entangled in a commercial war. A \TradeFLEET of \ANG of
  maximum level 6 in \ctz{Angleterre} was reduced to current level 0 due to
  attacks by \leaderBart. Another \TradeFLEET of level 4 in \stz{Atlantique W}
  was reduced to level 1 due to combined attack of a \pays{pirates} \corsaire
  and a French \corsaire. A third \TradeFLEET of level 5 was reduced to level
  1 in \stz{Lion}. Meanwhile, \ANG manages to take a \COL of level 4 of \FRA
  in \granderegionQuebec as well as a \TP in \continentIndia. This counts as 4
  provinces occupied by \FRA and 2 by \ANG, thus a +1 to PD in favour of \FRA.
\end{exemple}

\aparag[The net Peace Differential] is obtained by rounding the modified PD to
the nearest integer. In case of halves, round down in disfavour of the winning
side (that is, round toward 0). Then cap to \bonus{+5} (and \bonus{-5}) if
needed.
\bparag Note that fractions in the PD may only come from the \STAB
difference. However, the military situation may change the winner, thus the
direction of the final rounding.
\bparag The net peace differential is also symmetrical. Thus, it is always
sufficient to compute the PD from the point of view of one of the alliance.

\begin{exemple}[Rounding PD]
  \SUE, \POL and \TUR are at war against \RUS. The \STAB of \RUS and \SUE is
  1, while the \STAB of \POL and \TUR is 0. No side controls enemy
  provinces. Thus, the basic (and the modified) PD is 1 - (1+0+0)/3 = \td\ in
  favour of \RUS, rounded to +1 in favour of \RUS.
  
  \smallskip

  \SUE and \TUR are at war against \RUS. The \STAB of \RUS is 0, the \STAB of
  \TUR is 1 and the \STAB of \SUE is 2. Thus, the basic PD is +1.5 in favour
  of the alliance (or -1 in ``favour'' of \RUS). If the military situation
  does not modify this, it is rounded to +1 in favour of the alliance.

  \smallskip

  \SUE and \TUR are at war against \RUS. The \STAB of \RUS and \TUR are 1
  while the \STAB of \SUE is 0. Thus, the basic PD is +0.5 in favour of
  \RUS. However, the alliance controls four Russian provinces while \RUS
  controls no enemy province. Thus, the PD is modified by 2 in favour of the
  alliance, for a result of +1.5, rounded down to +1 in favour of the
  alliance. Note that if rounding had occured before modification, the PD
  would have been rounded to 0 and then modified to +2 in favour of the
  alliance. Hence, it is important not to round at the wrong time.
\end{exemple}

\subsection{The Peace levels}
The \terme{peace level} represents in an abstract way the amount of
``winning'' the winner has. It varies between 0 (white peace) and 5
(unconditional surrender). The peace level is strongly constrained by the
\terme{Peace Differential}. In turn, the peace level indicate how many
\terme{conditions} the loser has to give to the winner.

\subsubsection{Peace levels and conditions}
\aparag[Peaces that are permitted]
\bparag In any case, a \terme{Conditional Peace} of level equal to the PD in
favour of the dominant alliance is
allowed. % The maximum level is 5, even if the Peace
% Differential is higher.
\bparag If the \terme{Peace Differential} is at most +2 in favour of one
alliance, a \terme{Negotiated Peace} of level 0 (White Peace) or 1 is
permitted in favour of any alliance (even the one with the lowest modified PD,
that is the apparent loser).
\bparag Exception: if a power if forced to proposed a Mandatory peace (as
in~\ref{chPeace:Mandatory peaces}), it must propose a Conditional Peace and
may not propose a Negotiated one.
\bparag If at least one Major member of an alliance has its capital (or both
if it has two) and at least half of its national provinces controlled by
enemies (not necessarily allied), then a Conditional Peace of level 5 is
allowed against that alliance.

\aparag[Peace conditions] The level of the peace determine both the number of
conditions that the losing alliance must give to the winning one and some
details on these conditions, as described below.
\bparag Only countries that are fully at war may give or take peace
conditions. That is, countries in limited or foreign intervention do not risk
to lose anything at peace time, but they may not either have any
gain. Obviously, there may be some promises to be fulfilled at a later
Diplomatic phase, but as always promises are not binding.
\bparag There are 4 types of conditions that may be given at peace:
\begin{modlist}
\item[Territorial concessions:] The losing alliance gives ownership of one
  province to the winning alliance. See~\ref{chPeace:Transfer Provinces Peace}
  to know which province may be annexed by who. The province may belong to any
  member of the loosing alliance (including minor allies). The province can be
  given to any member of the winning alliance (including minor allies). The
  choice of the province is made either by the losing or winning alliance,
  depending on the level of the peace.
\item[Indemnities:] The losing alliance must give some money to the winning
  alliance. The money must come from the \RT of one or more majors of the
  loosing alliance (minor allies may not pay the indemnities) and can be given
  to one or more members of the winning alliance (minor allies may receive the
  indemnities). The amount is written in \lignebudget{Ransom, peace} of the
  concerned countries (negative for the losers, positive for the winners). If
  the losing alliance is composed solely of minor countries, they may pay
  indemnities.

  The losing alliance always choose who pay, while the winning alliance always
  choose who gets the money.
\item[Diplomatic concessions:] Either of the choice below. The precise choice
  is only decided when implementing the condition and is always made by the
  winning alliance. The minor involved must not necessarily be part of the war
  to be chosen (drastic changes of alliances and distant weddings were not
  uncommon). The minor involved may however not be at war elsewhere (it may be
  part of the just finishing war).
  \begin{itemize}
  \item (Europe) The loosing alliance must give diplomatic control of one of
    its European minor allies to the winning alliance. If the loosing alliance
    is solely composed of minors, then the winning alliance may gain
    diplomatic control of one of them.
  \item (\ROTW) One \ROTW minor breaks its diplomatic status with some member
    of the losing alliance and may increase its status with some member of the
    winning alliance.
  \end{itemize}
\item[Special conditions:] Events and other specific rules sometimes create
  specific concessions that may (or must) be used as peace conditions for some
  wars. Sometimes, a minimum level of the peace is required in order to ask
  for this concession. Sometimes, a concession is automatically added to other
  peace conditions as soon as the peace reaches a certain level.
\end{modlist}

\aparag[Terms of the peace]
When a peace is agreed between majors, the terms must specify both the level
of the peace and the nature of the conditions. For example, two countries may
sign ``a peace of level 3 with one territorial concession first and then one
diplomatic concession''.
\bparag Once the peace is agreed, players may choose the precise conditions
(which province to annex, who is going to pay the indemnities, \ldots)
\bparag The order of the concessions is important only in case of disagreement
between players.

\aparag[Deciding details]\label{chPeace:Implementing conditions}
In each alliance, the country whose monarch has the higher \DIP has all power
to decide which peace to sign.
\bparag However, for the precise choice of the conditions, the choice is made
in decreasing order of \DIP in each alliance. That is, the monarch with higher
\DIP chooses the first condition, the second one chooses the second, and so
one (looping back to the monarch with higher \DIP if needed).
\bparag Not that choices are made sometime by the losing alliance and sometime
by the winning one. The choice order is followed by each alliance separately.
\bparag For Territorial concessions only, the alliance who choose depends on
the level of the peace and the number of territorial concessions (only). That
is, if the only territorial concession is the second condition of the peace,
it is still the first territorial concession.

\aparag[Disagreement] Any power in the winning alliance who is currently
controlling at least one province of the losing alliance and does not receive
a full peace condition may denounce the peace (receiving part of some
indemnities is not enough to prevent a country from denouncing the peace).
\bparag In this case, all the majors of the winning alliance that received at
least one full peace condition immediately break their alliance with all the
powers denouncing the peace.
\bparag As usual, powers breaking alliance lose 2 \STAB and give a \CB to
their former allies.
\bparag Powers breaking alliance that way stay allied together. Power
denouncing the peace stay allied together.

\aparag Power that neither denounce the peace nor received a full peace
condition must immediately chose either to denounce the peace or to accept
it.
\bparag If they accept the peace, they are breaking their alliance with the
power denouncing it (and stay allied with the others), at usual cost.
\bparag If they denounce the peace, they stay allied with the other powers
denouncing it.

However, any ally deprived of any provinces/indemnites
when he is effectively occupying some of the loser's provinces may ask his
faulty ally(ies) to break the alliance (with the loss of 2 \STAB for them) and
immediately receives a \CB against him/them.

\begin{exemple}[Disagreement]
  \HIS and \HOL are losing a war against \FRA, allied to \paysPortugal. \HOL
  has higher \DIP than \HIS. The peace differential is 4, so the only peace
  that may be signed is of level 4, hence three conditions. After some
  discussions, \FRA and \HOL agree on indemnities as first conditions and then
  two territorial concessions after that. Note that having lower \DIP, \HIS
  may take part in the discussion but in the end, the decision is made by
  \HOL, however, if \HOL wanted to stay at war, \HIS could have signed a
  separate peace.

  Since \HOL has the higher \DIP, he choose how to implement the losing side
  of the first condition and decides that \HIS is going to pay all the
  indemnities (they could have been split in any way between the
  losers). Being the only Major, \FRA chooses who receive the money. Even if
  it could have given some to \paysPortugal, it prefer to keep all of it\ldots
  Even if the first territorial concession is the second condition, it is the
  first territorial one, hence chosen by the winning alliance. \FRA chooses to
  annex a Spanish province. Lastly, the second territorial concession is
  implemented. It is chosen by the losing alliance. Since \HOL already has its
  turn in choosing a condition (for the indemnities) and \HIS did not has its,
  \HIS chooses and decides to give a Dutch \COL to \paysPortugal.

  Note that letting an unwilling ally support all the weight of the peace is
  probably not a good long term strategy if you still need allies for future
  wars. Usually, the precise implementation of the peace conditions is agreed
  upon between players before signing the peace. The precise order of choice
  is rarely needed.
\end{exemple}

\subsubsection{Description of peace levels}
\aparag[Peace of level 0 (White peace)] No conditions are given or taken.
% The two alliances have to evacuate all conquests made during the
% course of this war and return to the situation of province control existing at
% the start of the war, except by express agreement between players.

\aparag[Peace of level 1] The winning alliance receives one peace condition.
\bparag[Territorial concession] The province is selected by the losing
alliance.
\bparag[Indemnities] The losing alliance gives 50 \ducats of war indemnities
to the winning alliance.
\bparag[European Diplomatic concession] One European country, neither in
\VASSAL nor \ANNEXION, is removed from one loser's Diplomatic Track and placed
back into the \Neutral box. If the losing alliance is composed solely of \MIN
powers, the winning alliance may gain one of them in \MR status.
\bparag[\ROTW Diplomatic concession] One \ROTW country breaks \dipFR status with
one member of the losing alliance. If the losing alliance is composed solely
of \MIN, then one of them is forced to sign a \dipFR with one member of the
winning alliance.

\aparag[Peace of level 2] The winning alliance receives one peace
condition.
\bparag[Territorial concession] The province is selected by the winning
alliance. 
\bparag[Indemnities] The losing alliance gives 75 \ducats of war indemnities
to the winning alliance.
\bparag[European Diplomatic concession] One country (any status, excepted if
blocked by other rules of events) is removed from one loser's Diplomatic Track
and placed back into the \Neutral box. If the losing alliance is composed
solely of \MIN powers, the winning alliance may gain one of them in \MR
status.
\bparag[\ROTW Diplomatic concession] One \ROTW country decreases one level
(from \dipAT to \dipFR or from \dipFR to neutral) with one member of the losing
alliance. If the losing alliance is composed solely of \MIN, then one of them
is forced to sign a \dipFR with one member of the winning alliance.

\aparag[Peace of level 3] The winning alliance receives two peace conditions.
\bparag[Territorial concession] The first territorial concession is chosen by
the winning alliance, the second (if there are two) is chosen by the losing
alliance.
\bparag[Indemnities] The losing alliance gives 75 \ducats of war indemnities
to the winning alliance.
\bparag[Diplomatic concession] One country (any status, excepted if blocked
by other rules of events) is removed from one loser's Diplomatic Track and
placed back into the \Neutral box, or in \MR status of one winner. If the
losing alliance is composed solely of \MIN powers, the winning alliance may
gain one of them in \MR status for one peace condition or in \AM for two peace
conditions.
\bparag[\ROTW Diplomatic concession] One \ROTW country either breaks \dipAT with
one member of the losing alliance or both breaks \dipFR with one member of the
losing alliance and signs \dipFR with one member of the winning alliance. If the
losing alliance is composed solely of \MIN, then one of them is forced to sign
a \dipFR with one member of the winning alliance, or a \dipAT for two conditions.

\aparag[Peace of level 4] The winning alliance receives three peace
conditions.
\bparag[Territorial concession] The first and third territorial concessions
are chosen by the winning alliance. The second one is chosen by the losing,
alliance.
\bparag[Indemnities] The losing alliance gives 100 \ducats of war indemnities
to the winning alliance.
\bparag[Diplomatic concession] One country (any status, excepted if blocked
by other rules of events) is removed from one loser's Diplomatic Track and
placed back into the \Neutral box, or in \MR status of one winner. If the
losing alliance is composed solely of \MIN powers, the winning alliance may
gain one of them in \MR status for one peace condition or in \AM for two
peace, or in either \EG or \VASSAL (if this status is possible) for three
peace conditions.
\bparag[\ROTW Diplomatic concession] One \ROTW country breaks status with one
member of the losing alliance and signs \dipFR with one member of the winning
alliance. If the losing alliance is composed solely of \MIN, then one of them
is forced to sign a \dipFR with one member of the winning alliance, or a
\dipAT for two conditions.

\aparag[Peace of level 5 (Unconditional Peace)] The winning alliance receives
three peace conditions.
\bparag[Territorial concession] All provinces are chosen by the winning
alliance.
\bparag[Indemnities] The losing alliance gives 150 \ducats of war indemnities
to the winning alliance.
\bparag[Diplomatic concession] One country (any status, excepted if blocked
by other rules of events) is removed from one loser's Diplomatic Track and
placed back into the \Neutral box, or in \MR status of one winner. If the
losing alliance is composed solely of \MIN powers, the winning alliance may
gain one of them in \MR status for one peace condition or in \AM for two
peace, or in either \EG or \VASSAL or \ANNEXION (if these status are possible)
for three peace conditions.
\bparag[\ROTW Diplomatic concession] One \ROTW country breaks status with one
member of the losing alliance and signs \dipFR with one member of the winning
alliance. If the losing alliance is composed solely of \MIN, then one of them
is forced to sign a \dipFR with one member of the winning alliance, or a
\dipAT for two conditions.

\aparag[Indemnities] Note that the amount given for indemnities is the amount
\emph{per condition}. That is, if a peace of level 5 is signed with three
indemnities as the three conditions, the total amount is 3 $\times$ 150 =
450\ducats !



\subsection{Transfers of Provinces by Peaces}\label{chPeace:Transfer
  Provinces Peace}
\aparag If a peace includes territorial concessions, some provinces owned by
the loosing alliance (including minors) immediately change ownership and now
belong to one member of the winning alliance (possibly a minor).
\bparag Not all powers may annex all provinces. If there is not enough
provinces to annex in order to fulfil all the territorial concessions, the
peace may not be signed under these terms.

\aparag[Choice of Provinces] The provinces that may be annexed are:
\bparag Any power may annex provinces it controls at the time of the peace.
\bparag Any power may annex any of its national provinces, whoever controls it
(even if still controlled by the enemy alliance).
\bparag Any power may annex any province it previously owned, whoever controls
it (even if still controlled by the enemy alliance).
\bparag Any power may annex any province with its blurred shield in it,
whoever controls it (even if still controlled by the enemy alliance).
\bparag Exception: capitals may never be annexed unless explicitly specified
elsewhere.
\bparag Any power may annex a \TP or \COL (including of level 6) if it was
controlled during some point of the war by any member of its alliance.
\bparag Any power may annex a \TP or \COL (including of level 6) if it owned
an establishment in the same \Area at some point during the game.
\bparag Exception: if a province, \TP or \COL is currently controlled by a
third party power (not member of any of the alliances signing peace), it may
only be annexed if the controlling power agrees. In that case, the controlling
power must evacuate the province as per~\ref{chPeace:Evacuation}.

\aparag[Priority] If any national province of the winning alliance is
currently owned by any member of the losing alliance, it must be chosen as
territorial concession.
\bparag If several exists, a controlled one must be chosen first.
\bparag If tied, choice is made by the power choosing how to implement the
condition (\ref{chPeace:Implementing conditions}).
\bparag Note that this priority does not prevent any other peace condition
(indemnities, diplomatic concessions, \ldots) to be obtained at peace.
\bparag Note that provinces with blurred shield are (usually) not national
provinces and thus don't have priority.

\begin{designnote}
  Ownership of national provinces provides a \CB. Leaving one non-important
  national province owned by someone else could lead to repeated wars against
  it with big territorial gain each time (especially against minor). Thus,
  this ``\CB-farming'' is restricted.
\end{designnote}

\aparag[Transfer of Colony or Trading Post]
\bparag One territorial concession (whatever the level of the peace) allow to
annex two \COL or \TP if (i) both are controlled by the winning alliance at
the end of the war and (ii) none of them is a \COL of level 6.
\bparag \COL of level 6 or establishment that are not controlled at the end of
the war are annexed for a full condition each.
\bparag The two establishments may be annexed by different winners.
\bparag The power choosing how to implement the peace condition does chose
both establishments and their new owners.

\aparag[Overseas Wars]
A peace treaty ending an Overseas War may not involved change of ownership of
any province on the European map, \province{Islas Canarias} or \province{Cabo
  Verde}.
\bparag Note that \COL of level 6 may still be annexed and that the ``two for
one'' rule above still applies.
% \bparag [TBD: pour annexer en Afrique, il faut une guerre normale, guerre
% outremer permet juste les présidios + indemnités + diplomatie] [\textbf{PB:
%   actuellement je suis en faveur de cette restriction}]

\aparag[Transfer of provinces of minor countries]
\bparag Minors signing peace at the same time as their Diplomatic patron are
involved in the peace as any power and may thus cede or annex provinces.
\bparag For this purpose, provinces with a non-blurred shield, as well as
provinces formerly owned by the minor, count as ``national provinces'' of the
minor (especially for the priority of annexation rule).
% major allies of minor Venise must give Moree to it even if it was never
% owned by \VEN. This represents the result of the Peloponese war (Morosini)
% where the \AUS-Venise alliance wins and Venise annex Moree.
\bparag Additionally, provinces of \regionBalkans are considered as national
provinces of \paysVenise.
\bparag Provinces gained or lost by minors count as if gained or lost by their
Diplomatic patrons for \VPs.
% Only in case of choice made by the patron (and allies). Thus avoid \RUS
% taking 1 province from Crimea (only) and having it drop from \TUR track as a
% free bonus.
% Possible \COL/\TP annexation are ignored as this probably would create too
% strange stuff.
\bparag If the losing alliance chooses to give a province of a minor when it
may have chosen a province of a major from the European map, this minor goes
to Neutral after the peace is signed.

\subsection{General Consequences of the Peace}
\aparag Peace brings the conflict opposing the belligerent countries to an
end.
\bparag Unless involved in another war, the countries are now considered at
peace for all game purposes.

\aparag[Resolving peaces] Peace conditions must be transferred immediately
upon signing the peace.
\bparag Provinces given as territorial concession change ownership. Mark with
the correct ownership counters. If there is a fortress in the province, the new
owner may immediately replace it with one of its fortress of the same level or
destroy it (special European arsenals may be replaced by a fortress of the
same level or an European arsenal if one is allowed here).
\bparag \COL and \TP given as territorial concessions also change
ownership. Replace the counter by a counter of the same nature of the new
owner. Level and exploited resources stay the same, update the corresponding
record sheets. Any fortress or arsenal may be replaced by a counter of the
same nature and level of the new owner, or destroyed.
\bparag If not enough fortress, \COL or \TP counters are available (\COL and
\TP limit is usually smaller than counter mix), the owner may destroy one of
its existing one. If not enough ownership counters are available, make new or
use whatever mean you wish to denote ownership.
\bparag Any minor given as diplomatic concession changes patron. Place its
diplomatic counter at the right position on the diplomatic track.
\bparag Indemnities must be payed immediately in full, even if this leads to a
future bankrupt.
\bparag Other specific conditions are also implemented immediately, marking
any changes as possible.

\aparag[Returning control] Remove all control markers of country signing peace
that is located inside a country signing peace with it. Control of these
provinces is returned to their rightful owner.
\bparag Any fortress of a country signing peace located in a province owned by
another country signing peace with it may be immediately replaced by a counter
of the same level of the owner, or destroyed. Owner of the province chooses.
\bparag Exception: \Presidios are kept. They do not change ownership and are
not removed.

\aparag[Evacuation at land]\label{chPeace:Evacuation}
Any land unit in a non-controlled province owned by a country not at war with
(ally) or against (enemy) the owner of the unit must be evacuated by movement
and go into any owned and controlled province.
\bparag Evacuating units may move through any country that was part of the
war, including former enemies, regardless of the presence of any unit.
\bparag Exception: they may not enter a province with an unbesieged enemy unit
or fortress (from another war).
\bparag Evacuating units may not, however, move through provinces of countries
that were not part of the war.
\bparag Evacuating units may move by sea, even if there is no fleet to
transport them.
\bparag Evacuating units have unlimited movement capacity (\emph{i.e.} they
are not limited to 12\MP), however, they may not move more than 12\MP if they
can evacuate in 12\MP or less.
\bparag Evacuating units may not be intercepted.
\bparag Evacuating units roll for attrition as usual with a \bonus{-2} to the
roll, and considering all provinces as friendly. Ignore any bad weather. Each
set of 6\MP expanded is one cause of attrition.

\aparag[Evacuation at sea] Naval units of a country signing at least one peace
may either evacuate to any owned and controlled port or stay at sea.
\bparag If they return to port, they must roll for attrition with a bonus of
\bonus{-2}.
\bparag However, if they stay at sea, they do not need to roll for attrition.

\begin{designnote}[Control and evacuation]
  Returning control and evacuating only happens between former belligerents
  (including allies). If a country is involved in another war, it does not
  have to return control and evacuate from this war (if it is still going
  on).

  When, evacuating, you must also evacuate from your ally, except if your are
  still together fighting in another war.
  
  Land unit in non-controlled provinces of countries at war are handled by
  \ref{chRedep:Redeployment}.
\end{designnote}

\aparag[Evacuation and redeployment] If any stack is out of supply after
evacuation (this may happen because of separate peace), it may chose to also
evacuate or stay where it is.
\bparag If it evacuate, it does not get the \bonus{-2} to attrition roll. In
addition it is considered to have entered at least one enemy province (the one
where it starts its evacuation).

\aparag[Memento]
\bparag At this point of the turn, land units should be either:
\begin{modlist}
\item In a controlled province.
\item In a controlled \Presidio.
\item Besieging a province where they could maintain siege.
\item In a province controlled by a member of the same alliance, together at
war. This includes troops in foreign or limited intervention.
\item Besieged in a fortress.
\item In the \ROTW, in a province without any establishment.
\end{modlist}
\bparag Any other land unit must either have redeploy or evacuate [or I did
overlook an obvious special case].

\aparag[Pacification] Unless this is a Negotiated Peace, or a Conditional
white peace:
\bparag All existing \CB at the time of the peace are negated for 1 turn, even
permanent ones.
\bparag Additionally, each loser is forbidden to declare war without \CB
against any victor next turn.

\begin{designnote}
  This effectively prevents the losers from attacking the winners next turn,
  unless an new \CB appears, usually by event. The winners, however, may
  attack the losers but at high cost (no \CB).
\end{designnote}

\aparag[Peace and Casus Belli]
\bparag Any permanent \CB whose cause does not exists any more is cancelled
(\emph{e.g.} return of the last national province).
\bparag Unless this is a white peace, all temporary \CB from all belligerents
(not only the attacker) obtained before the end of the war are considered to
have been used.

\begin{designnote}[Temporary \CB]
  Most temporary \CB are one time. In case of war, all of them are considered
  used, that is, the war is waged over all former causes of resentment not
  just over the single border dispute that made it erupt.

  Some temporary \CB are multiple use (\emph{e.g.} once per period). In this
  case, the war ``consomme'' one of these use.
\end{designnote}

\aparag[Peace and \STAB]\label{chPeace:Peace and Stability}
Any major country that both
\begin{modlist}
\item was fully at war against at least another major country or was victim of
  a declaration of war by a minor (either by political event or \RD, the major
  must be the first victim of the war, not dragged in by alliances) during one
  the previous turns ;
\item AND is now completely at peace (no intervention either) for the first
  time since these wars ;
\end{modlist}
immediately gain 1 \STAB.
\bparag If the country is not completely at peace now, the \STAB will be
gained when it will be at peace, even if the last peace treaties should not be
enough to gain it.
\bparag This gain is limited to 1 \STAB per country per turn, no matter how
many peaces are signed.

\begin{exemple}[Standard case]
  At turn 46, at the end of \ref{pV:WoSS}, \FRA and \HIS sign peace with \ANG,
  \HOL and \AUS. \AUS is still involved in a war in \paysHongrie against
  \TUR. \FRA, \HIS, \ANG and \HOL are now fully at peace and gain 1 \STAB and
  only 1, no matter how many enemies they signed peace with. \AUS is prevented
  from gaining it by still being at war against \TUR. At turn 47, \AUS and
  \TUR sign peace. They both gain 1 \STAB.
\end{exemple}

\begin{exemple}[Peace with minors and \STAB]
  At turn 5, \TUR attacks \paysDamas, a minor. If it signs peace at the end of
  turn 5, it does not gain \STAB as this is a minor and \TUR was the
  attacker.

  At turn 5, \TUR attacks \paysDamas. At turn 6, \ref{pII:War Persia Turkey}
  happens early and \paysPerse attacks \TUR. At the end of turn 6, \TUR manage
  to sign peace with \paysPerse. Since it was victim of a declaration of war
  by event, it should gain \STAB. However, it is still at war against
  \paysDamas and may not gain it, but the fact that it got out of a ``big''
  war is remembered. At turn 7, \TUR signs peace with \paysDamas. Since it is
  now completely at peace, it gains 1 \STAB.
\end{exemple}

\begin{exemple}[Separate peaces and \STAB]
  At turn 10, \TUR is at war against allied \HIS and \VEN. It signs a separate
  peace with \VEN. Since it is a major country, it should gain 1 \STAB but is
  prevented to do so by still being at war against \HIS. \VEN, however, is now
  fully at peace (supposing there is no other war) and gain 1 \STAB (thus
  mitigating the 2 \STAB loss of breaking an alliance for separate peace). 

  At turn 8, \TUR and \HIS sign peace. \TUR is now fully at peace and has two
  reasons to gain \STAB: the former treaty with \VEN and the current with
  \HIS. However, the max gain is 1 per turn, so it gains only 1
  \STAB. Similarly, \HIS is now fully at peace and gains 1 \STAB.
\end{exemple}

\begin{exemple}[Peace and interventions]
  At turn 28, \POL is both at war against \RUS and in foreign intervention in
  \ref{pIV:TYW}. It signs peace with \RUS. Since it is a peace with a major,
  it should gain 1 \STAB, but being in intervention prevents this. At turn 29,
  \POL ends its intervention. At the end of the turn, since it is now fully at
  peace and was previously at war against a major, it gains 1 \STAB.
\end{exemple}

% Useless with negative treasure.

% \aparag[Indemnities] Indemnites agreed in the Peace are to be paid by one of
% the losing Major powers. He may pay them now or at a segment of Announcement
% of the two following turns. They can be paid in fractions during the 3 turns
% allowed.
% \bparag Failure to pay all the due indemnities give a temporary free \CB to
% the power that was wronged against the power that should have paid.

\subsection{Peace with Minor powers}

\aparag To make the peace with a minor, the enemy player has to indicate that
he commits himself to peace negotiations with the minor.  The controlling
player of the \MIN may be at war (against the player) or not; in the former
case it is a "separate peace", attempted and signed before any peace between
players.
\bparag An alliance may usually offer peace with only one minor country per
alliance at war against it and per turn; and this minor may not be a vassal ir
annexed.
\bparag Exception: An alliance can always propose a specific peace to a minor
at war declared by an event .
\bparag Exception: An alliance can offer separate peace to any or all minor
vassal to the alliance or annexed, whose capital city is controls.
\bparag Exception: An alliance can offer separate peace to any or all
non-vassal, and non-annexed minors if controlling at least one of their
provinces (each one), or if each one of them controls individually at least
one province in the alliance.
\bparag If allied powers do not agree, the decision is taken by the Power
having the higher value in DIP (decide ties randomly).

\aparag[Method]
To sign peace with the minor, the player rolls 1d10, taking into account the
sought-after peace level and all applicable modifiers. The level of peace is
chosen by the player, without taking into account his own Stability.
\bparag[Result]
The peace is signed if the modified die-roll result is 6 or more.
\bparag[Peace level Modifier]
This modifier is the triple of the value of the peace level chosen. It is used
as a positive modifier if the offered peace is favorable to the minor, or as a
negative modifier if the offered peace is favorable to the player.
\bparag[Nationality Modifier]
It is applied for the case of conflicts with specific minor countries
which are:
\begin{modlist}
\item[-4] \pays{Perse}, \pays{egypte}, \pays{damas}, \pays{Chine},
\pays{Japon}
\item[-3] \pays{USA}, \pays{Mogol}, \pays{Venise}, \pays{Pologne},
\pays{Habsbourg},  \pays{Brandebourg} after IV-11
\item[-2] \pays{Portugal}, \pays{danemark}
\end{modlist}
\bparag[Modifiers of Situation]
These are is applied cumulatively according to what happened during the
current turn (and to what happened during previous turns for the
lost/conquered provinces only):
\begin{modlist}
\item[+2] per province/\TP\faceplus/\COL lost by the minor (+4 if capital)
\item[-2] per province/\TP\faceplus/\COL conquered by the minor (-4 if
  capital) 
\item[+1] per \TP\facemoins  lost by the minor (round down) 
\item[-1] per \TP\facemoins conquered by the minor 
\item[+2] if the capital province of the \MIN was conquered this turn, or if
  it was captured then lost since
\item[-4] if the \MIN has captured a capital province of a \MAJ this turn, -or
  if it was captured then lost since
\item[-2] per major battle won by the minor  
\item[+2] per major battle won by the player 
\item[-1] per battle won by the minor 
\item[+1] per battler won by the player
\item[+1] per minor military leader killed or captured 
\item[+2] if the Monarch of the minor country is captured and its Ransom is
  used for Peace
\item[-1] per military leader of the player killed or captured
\item[+1] per siege won by the player 
\item[-1] per siege won by the minor 
\item[-2] If minor is heretic (Catholic vs. Protestant, before the end of the
  \terme{Religious Dissension})
\item[-2] if it is an attempt to negotiate a separate peace $\pm$?: the PD of
  the controller regarding the alliance attempting the peace (if the
  controller is at war against the alliance), max. -3/+3
\end{modlist}
\bparag All these modifiers are cumulative in one single turn.
\bparag \COL and \TP\faceplus controlled counts as a full province (modifier
of $\pm 2$); the same is true for control of cities of minor countries in the
\ROTW;
\bparag \TP\facemoins controlled gives a modifier of $\pm 1$ only; the same
modifier holds for occupying a province without city of a minor country in the
\ROTW;
%% was \pm 1, remis à 1.5 pour être comme les tables ou la liste de modifieurs
%% ci-dessus...

\aparag[Overseas Wars]
\bparag A Minor country always accepts to sign a White Peace in Overseas War
(if it is not a Separate Peace).

\aparag[Consequences of Peace]
If the peace is signed, no Stability level is gained (exception: if this minor
declared war to the player by an event). The player that controlled the minor
does not earn anything.
\bparag The conditions of Peace are the same as for a Peace between Major
powers.
\bparag A Minor country will at most indemnities up to 4 times its income,
immediately at the conclusion of the Peace. Any other indemnities are void.
\bparag If the minor country is the victor, the player that controls the minor
country chooses the ceded provinces (if any). He must do so in priority among
those located the closest from the minor country's territory, in terms of
movement points (a sea zone is equivalent to 2 MP for this calculation).
\bparag A minor country nevers takes Diplomatic concessions, only provinces
and indemnities.

\aparag[Multiple and Separate Peace]
If a player signs a separate peace with a minor country (he is still at war
against the controlling player); this minor may not be again involved in a war
against him next turn (unless by an event or a Crusade).
\bparag A minor country at war by event may only make a separate peace.

\aparag[Unconditional Peace]
A Minor country will sign a mandatory Unconditional Peace if all of its
provinces are controlled by the ennemy. This peace is one global peace against
all the powers controlling its provinces (so it can lose only 3 provinces).
\bparag If a minor country loses and signs an unconditional peace its
political marker is moved automatically to the box \Neutral.

\aparag[Automatic Peace]
An alliance that proposes a victorious Unconditional Peace to a Minor power,
and that Minor power was the attacking one (caused by event), or is not allied
to an alliance (so this is not a Separate Peace), the Peace is automatically
accepted by the Minor power.

\aparag[Failure of Peace negotiations]
If the die-roll is inferior or equal to 5 the war with the minor continues
normally for the following turn. Another peace attempt with that minor will be
allowed during the peace phase of next turn.

\section{Test for crusade}\label{chPeace:Crusade}

% Local Variables:
% fill-column: 78
% coding: utf-8-unix
% mode-require-final-newline: t
% mode: flyspell
% ispell-local-dictionary: "british"
% End:
