% -*- mode: LaTeX; -*-

\definechapterbackground{Budget and Peaces}{victories}
\chapter{Budget and Peaces}\label{chapter:Peace}

\section{Overview of the phase}

% RaW: [50]
\aparag[Administration] At the end of the turn, final administrative actions
are resolved and budgets must be completed. First, exceptional taxes that were
scheduled during the administrative phase are resolved. Then comes the
exchequer test. At this point, players roll to determine how well the taxes
where collected this turn and to discover their precise income. If the income
is not enough to cover for the expenses, loans must be contracted, either from
the people of your country or from international bankers. Last but not least,
countries may try to improve their \STAB.

\aparag[Peace] Wars can be ended only by a Peace. There are several types of
Peace, from the white peace (return to statu quo) to the unconditional
surrender. The type depends mostly on the difference between the \STAB of the
belligerents, slightly modified by the military situation. In some cases,
countries must accept any peace proposed by the opponents, but usually some
discussion occurs between the players.
\bparag[Crusade] If \TUR conquers too many Christian provinces, the pope may
try to launch a Crusade. 

% \aparag[Events and special peace conditions] A small number of political
% events can initiate wars where the peace conditions are constrained or
% modified by specific conditions. These events do not change the rules below
% for other simultaneous conflicts where one side of the war is not one of those
% described in the event.
% \bparag Remark that the other side must enter war through another means than
% the event or the alliances network called at the time of the event (else, it
% is considered part of the event).

\aparag[Sequence of the Peace Phase.]
\PeaceDetails

\section{Economical and Administrative affairs}
\subsection{Exceptional taxes}\label{chPeace:Exceptional taxes}
\aparag[Exceptional taxes] Exceptional taxes are scheduled during the
Administrative phase. See~\ref{chExpenses:Exceptional Taxes} for details (and
examples). They are resolved at this point only. That is, until the end of the
turn (and after most expenses have been planned), players won't know exactly
the amount of collected taxes.
\bparag Note that Exceptional taxers must be planned during Administrative
phase. If a country forfeited the possibility to do so, it is to late now to
decide to raise taxes.

\aparag[Resolution of the taxes]
\bparag Each country which has planned taxes should have written a modifier in
\lignebudget{Exceptional taxes modifier A} (copied from
\lignebudget{Exceptional taxes modifier B}). This modifier was \ADM + 3
$\times$ \STAB (at the time of the Administrative phase).
\bparag Roll 1d10, add the modifier and multiply the result by 10. This is the
amount of taxes (in \ducats).
\bparag Write this amount in \lignebudget{Exceptional taxes}. It may well be
negative if the modifier was negative. In this case, the country will actually
lose money because of the taxes. It is not possible to refuse a ``tax'' once
the amount is known.

\aparag[\RT before Exchequer test]
\bparag Players can know compute their \RT before resolving the Exchequer
test.
\bparag This is the sum of lines \ERSlong{RT after Diplomacy} +
\ERSlong{Pillages, privateers} + \ERSlong{Gold from ROTW and Convoys} +
\ERSlong{Exceptional taxes} of \EcoRS. It is written in \lignebudgetlong{RT
  before Exchequer}.
\bparag Players should also copy \lignebudgetlong{Gross income B} in
\lignebudgetlong{Gross income A} and \lignebudgetlong{Total expenses} in
\lignebudgetlong{Expenses}.

\subsection{Exchequer test}\label{chPeace:Exchequer test}
\begin{designnote}
  We explain here the technical rules of the economical system. For a
  description of the spirit of these rules, see~\ref{chThePowers:Exchequer}.

  The rules here are quite ``algorithmic'' in order to have them as precise as
  possible and avoid misinterpretations. Thus, there are not well suited to
  understand the whys of the system (only the
  hows). Check~\ref{chThePowers:Exchequer} in order to understand what should
  happen.
\end{designnote}
\subsection{International loans}\label{chPeace:International loans}
\subsection{Stability Improvement}\label{chPeace:Stability Improvement}

\label{chPeace:Stability Improvement}
\aparag A player can attempt to improve his Stability, but this is never
compulsory. If he desires to improve it, he must write down the investment he
wishes to pay for such an action and then roll for the Test of Stability
Improvement. The Stability is then adjusted in function of the obtained
result. This result may even be negative!
\aparag[Investment]
\bparag Basic Investment: 30 \ducats
\bparag Medium (+2 to the die-roll): 50 \ducats
\bparag Strong (+5 to the die-roll): 100 \ducats

\aparag[Procedure]
This action is resolved without requiring a table.  The player rolls the die
of conjuncture. This die-roll is modified as follows:
\bparag +? \ADM monarch,
% [TBD: or \MIL if at war ?]
\bparag +? if medium or strong investment
\bparag +2 if victim of a declaration of war (solely if the player has not
transgressed himself an alliance, and has not declared war to anybody during
this turn)
\bparag -2 if at war with a minor country
\bparag -3 if at war with one or more players (non-cumulative with the -2
above)
\bparag -5 if an enemy \ARMY counter is on a national province and controls
the province city (not applicable during a Religious/Civil War or to rebel
stacks appeared during or as a consequence of such an event)
\bparag -3 Exception: for \SPA, the malus for having an enemy \ARMY counter
controling the city, is -3 only, however it applies for any owned territory
(not only its national territory).  This specificity ends
with \eventref{pIV:Olivares} (if effects are applied), or with
\eventref{pV:WoSS} (whatever the choices and outcomes).
\bparag +3 for a Prosperous Power
\bparag -3 for an Anti-Prosperous Power
\bparag $\pm$? by event
\aparag[Result]
If the modified result is equal to:
\bparag 11-14: the Stability increases 1 level
\bparag 15-17: the Stability increases 2 levels
\bparag 18+: the Stability increases 3 levels
\bparag 5 or less: the Stability decreases 1 level
\bparag Reminder: Stability varies from -3 to +3.

\aparag[Prosperity]\label{chPeace:Prosperity}
The player checks on his economic form the evolution of his \terme{Gross
  Income} on the last two consecutive turns (including the one just
played). This evolution affects the Stability Improvement Action, as seen
above.
\bparag[Prosperous Power:]
If the gross income has not decreased during the last 2 consecutive turns and
progressed during at least one of those turns.
\bparag[Anti-ProsperousPpower:]
If it has decreased 2 consecutive turns.

\section{Peace offers and discussions}\label{chPeace:Peace offers}
\subsection{The Informal Peace}

\aparag The informal peace is concluded following a mutual agreement between 2
players or more, and has to be announced to all other players.
\aparag[Consequences]
None of the players earns any VP for an informal peace. The war stops
immediately.
\bparag This type of peace can comprise any clause for which there was mutual
agreement between the involved players, in the limit authorized by rules on
Agreement. The Agreement is applied now.
\bparag This type of peace can not be concluded with a minor.



\subsection{The Formal Peace}

\aparag A formal peace can be of different types: white-peace, negotiated,
conditional or unconditional.
\bparag It is also the only type of peace authorized with a minor country (or
alliances of minor countries with no major power), but with a different
mechanism.
\bparag Once the principle of such a peace is accepted by the involved
alliances, the type of a formal peace can be negotiated or not and depends on
differential of the involved alliances's Stability.
\aparag[The Peace Differential]
For the detail of peace, we will call "Peace Differential" (PD) the
differential between the Stability values of the 2 alliances at war and
modified by the military situation as described below.  The PD is considered
between the dominant alliance with the highest score (presumed victor) and the
other one with the lowest (presumed loser). The Stability value of an alliance
is that of the Major power in the alliance.
\bparag[Case of Allies]
if more than one Major power is involved in a side when signing a peace, the
Stability used is the average of the allied powers, rounded on the nearest
number (and down if one-half is obtained).
\bparag[Military Situation modifiers]
The Stability of a Major power is modified if it controls more provinces in
the enemy alliance than the ennemies control in itself's territory.  If the
difference is at least 2, the bonus is {\bf +1}, then {\bf +2} if at least 4,
and {\bf +3} is at least 6. Control of a capital city of a Major power counts
as 2 provinces in this comparison; \COL and \TP count for only one-half
province.
\bparag[Military Situation modifiers in Overseas Wars]\label{chPeace:Privateer Effect}
The same comparison is made but controlled \COL and \TP count as full
provinces.  \textit{Privateer Effect:} Moreover, each Commercial fleet with
level 4 or higher that was reduced to 0 or 1 because of Privateers count as
one province for the looting power at this turn (or 2 if in the \CTZ of the
enemy).

\aparag[Peaces that are permitted]
\bparag If the PD is at most +2 in favour of one alliance, a Peace of level 0
(White Peace) or Negotiated Peace of level 1 is permitted in favour of any
alliance (even the one with the lowest modified Stability).
\bparag In any case, a Conditional Peace of level equal to the PD in favour of
the dominant alliance is allowed. The maximum level is 5, even if the Peace
modifier is higher.
\bparag If a Major power (not an alliance) has its capital (or both its
capitals) and half of its national provinces (round up) controlled by enemies,
an Unconditional Peace is allowed against it (but not mandatory); if there is
more than one Major Power in an alliance and the condition is true over one of
them, an Unconditional Peace is possible.

\aparag[Peace constrained] If a Major power is for two consecutive at -3 in
Stability, vis-a-vis the same enemy power or alliance, he is complied to make
a losing White, Conditional or Unconditional peace with this power. The level
of that peace is the worst one that is authorised depending on their Peace
Differential or on the military situations for UP.
\bparag The enemy is not forced to accept this peace. The power is just forced
to \emph{propose} it to its enemy. Exception: if the peace proposed is of
level 4 or 5, then the enemy is forced to accept it (it's basically an
unconditional surrender).
\bparag If both alliances are contrained to sign a peace at the same time, the
Peace is mandatory at the current PD level.
\bparag Under-developped Russia signs a constrained peace only if at -3 in
\STAB during three consecutive turns.
\bparag If the Major powers allied to this power do not want to sign a Peace,
the power makes a mandatory Separate Peace that entails neither loss of \STAB,
nor \CB given to the former allies.

\aparag[Definition of the conditions]
The exact terms of the peace have to be agreed by everyone in the alliances,
following the rules below for the condition. This is true especially for the
number of provinces to be ceded or of indemnities given instead, or if
specific conditions (such that nullifies some events) are asked by the
victorious alliance. If no agreement is possible, the peace will not be
signed.

\aparag[White Peace, or Peace of level 0]
The two alliances have to evacuate all conquests made during the course of
this war and return to the situation of province control existing at the start
of the war, except by express agreement between players.

\aparag[Negotiated or Conditional Peace of level 1]
\bparag The victorious alliance receives one peace condition of its choice
among the following ones.
\bparag[Cession of Territory] One province of the loser's choice is ceded.  It
is selected by the loser following the rule \ruleref{chSpecific:Tranfer
  Provinces Peace}.
\bparag[Indemnities]
The losing alliance gives 50 \ducats of war indemnities to the victorious
alliance.
\bparag[Diplomatic concessions]
One country is removed from one loser's Diplomatic Track and placed back into
the \Neutral box. This can not be a country that is \VASSAL or in \ANNEXION.
If the loser is a \MIN power, one can alternatively chooses to gain it in \MR
status.

\aparag[Conditional Peace of level 2]
\bparag The victorious alliance receives one peace condition of its choice
among the following ones.
\bparag[Cession of Territory] The victorious alliance receives one province of
the its choice. It is selected following the rule \ruleref{chSpecific:Tranfer
  Provinces Peace}.
\bparag[Indemnities]
The losing alliance gives 75 \ducats of war indemnities to the victorious
alliance.
\bparag[Diplomatic concessions]
One country is removed from one loser's Diplomatic Track (any status, excepted
if blocked by other rules of events) and placed back into the \Neutral box.
If the loser is a \MIN power, one can alternatively chooses to gain it in \MR
status.

\aparag[Conditional Peace of level 3]
\bparag The victorious alliance receives two peace conditions of its choice
among the following ones. Each of them can be selected twice.
\bparag[Cession of Territory] The victorious alliance receives one province.
The first province selected is its choice, but the second is of the loser's
choice.  Territories are selected following the rule
\ruleref{chSpecific:Tranfer Provinces Peace}.
\bparag[Indemnities]
The losing alliance gives 75 \ducats of war indemnities to the victorious
alliance.
\bparag[Diplomatic concessions]
One country is removed from one loser's Diplomatic Track (any status, excepted
if blocked by other rules of events) and placed back into the \Neutral box, or
in \MR status of one winner.  If the loser is a \MIN power, one can
alternatively chooses to gain a Diplomatic status, in \MR for one condition,
or \AM for two peace conditions.

\aparag[Conditional Peace of level 4]
\bparag The victorious alliance receives three peace conditions of its choice
among the following ones. Each of them can be selected ore than one.
\bparag[Cession of Territory] The victorious alliance receives one province.
The first and third provinces selected are of its choice, but the second is of
the loser's choice.  Territories are selected following the rule
\ruleref{chSpecific:Tranfer Provinces Peace}.
\bparag[Indemnities]
The losing alliance gives 100 \ducats of war indemnities to the victorious
alliance.
\bparag[Diplomatic concessions]
One country is removed from one loser's Diplomatic Track (any status, excepted
if blocked by other rules of events) and placed back into the \Neutral box, or
in \MR status of one winner.  If the loser is a \MIN power, one can
alternatively chooses to gain a Diplomatic status, in \MR for one condition,
or \AM for two peace conditions, or \VASSAL for three peace conditions (if
status possible).

\aparag[Conditional Peace of level 5 and Unconditional Peace]
\bparag The victorious alliance receives three peace conditions of its choice
among the following ones. Each of them can be selected ore than one.
\bparag[Cession of Territory] The victorious alliance receives one province.
All provinces are the choice of the victorious alliance.  Territories are
selected following the rule \ruleref{chSpecific:Tranfer Provinces Peace}.
\bparag[Indemnities]
The losing alliance gives 150 \ducats of war indemnities to the victorious
alliance.
\bparag[Diplomatic concessions]
One country is removed from one loser's Diplomatic Track (any status, excepted
if blocked by other rules of events) and placed back into the \Neutral box, or
in \MR status of one winner.  If the loser is a \MIN power, one can
alternatively chooses to gain a Diplomatic status, in \MR for one condition,
or \AM for two peace conditions, or \VASSAL or \ANNEXION for three peace
conditions (if status possible).



\subsection{Transfers of Provinces by Peaces}\label{chSpecific:Tranfer
  Provinces Peace}

\aparag Provinces are transferred as an implementation of the peace just
concluded, immediately upon the conclusion of the Peace phase.

\aparag[Choice of Provinces]
The provinces that can be annexed are:
\bparag provinces militarily conquered in the course of the just concluded
conflict are still controlled (except, usually, the capital province);
\bparag or provinces that used to belong to the victor and were annexed by the
loser during a preceding war (including any National provinces of the loser).
\bparag In priority, the National provinces of the victorious alliance are
chosen and annexed.
\bparag Minor Powers that are fully at war are included in the alliance like
any other power, so that they may gain provinces also.

\aparag[Transfer of Colony or Trading Post]
A player can choose a Colony or Trading Post instead of a province, on the
condition that he has effectively occupied this Colony during the course of
just ended conflict, or that the he has or used to have a \COL in the same
\terme{area}.
\bparag This \COL or \TP counts as half a province if controlled at the end of
the conflict.  Only \COL at levels 6 count as a full province.
\bparag If it is not controlled at the end of the the conflict, it may still
be chosen but then counts a full province.

\aparag[Overseas Wars]
A peace treaty ending an Overseas War may not involved change of ownership of
any province on the European mainland. Only transfers of \TP/\COL, % provinces
%in "Barbaresques" countries, \pays{Egypte} and \pays{Irak}, 
or Indemnities are valid conditions of peace.
% \bparag [TBD: pour annexer en Afrique, il faut une guerre normale, guerre
% outremer permet juste les présidios + indemnités + diplomatie] [\textbf{PB:
%   actuellement je suis en faveur de cette restriction}]

\aparag[Transfer of provinces of minor countries]
If a player, at war against an alliance, accepts to make peace, all his minors
do the same, except those at war by event.
\bparag If an alliance receives provinces by a favorable peace, provinces of a
minor ally fully involved in the war, or of a Vassal or Annexed minor power,
may be given to the victor, in place of provinces of the loser
alliance. Howevern these provinces should obey the conditions of rule
\ruleref{chSpecific:Tranfer Provinces Peace}.
\bparag All minor countries provinces ceded by the controlling player count as
his own for victory points.
\bparag If minor cedes provinces by this way, its diplomatic marker is
immediately move to the box \Neutral.



\subsection{Alliances of Major Powers}


\subsubsection{Case of Victorious Allies}
\aparag The number of 3 provinces transferred is a maximum. The sharing of the
spoils of war among allies is left to their free choice.
\aparag In case of disagreement, the power with the Monarch having the best
Diplomacy rating in the alliance (decide at random in case of ties) and whose
country presently has units in the loser's territory, is to take the final
decision of the proposition of Peace. Peace conditions are fixed on at a time
by each power in order of descending Diplomacy rating (so, at most 3 powers
decide).
\aparag However, any ally deprived of any provinces/indemnites when he is
effectively occupying some of the loser's provinces may ask his faulty
ally(ies) to break the alliance (with the loss of 2 \STAB for them) and
immediately receives a \CB against him/them.


\subsubsection{Separate Peace}
\aparag If several players are allied and at war together, some may want to
make separate peace, for instance if the conditions does not please them, or
to stop the war early. A Separate Peace is only allowed if one Major power in
the alliance wants to remain at war (it is not possible to sign Separate
Peaces in order to multiply the Treaties of Peace).
\aparag[Separate Peace and Casus Belli]
In this case, any other Allied player at war on the same side that the one
making a separate peace receives an immediate temporary \CB against this
player.
\bparag This \CB may only be used the turn immediately following the signature
of the separate peace or it is lost.
\bparag The Allied player making a separate peace breaks his military
Alliance, and suffers a loss of 2 levels of Stability.
\bparag Exception: If the Ally must make peace because of his Stability (Peace
Constrained, see above), he does not break his Alliance (and there is no \CB
against him).
\aparag Separate Peace can also be attempted to be signed with Minor
countries, using the mechanism for Peaces with Minor powers. For each alliance
that is at war against it, an alliance may offer a Separate Peace only to one
Minor power in the enemy alliance (and that can not be a Vassal, excepted if
the Capital of the Vassal is controled).



\subsection{General Consequences of the Peace}

\aparag Peace brings the conflict opposing the players to an end.

\aparag In Conditional or Unconditional Peace (not in Negotiated or White
Peace), the loser has not the right to declare a war against the victor for
the whole next turn, excepted if using a new \CB obtained after the Peace
treaty.  Existing \CB at the time of the Peace (even permanent ones) are
negated for both powers for one turn.

\aparag[Peace and Casus Belli]
Permanent \CB are canceled if their cause disappears with the war (e.g.
return of a national Province previously annexed). Temporary \CB obtained
before the end of the war disappears, excepted after a White Peace.

\aparag[Peace and Evacuation]
All units present on the territory of a former enemy are repatriated by their
owner to the closest friendly province of his choice, without having to pay
the cost of activation and rolling for attrition at {\bf -2}.
\bparag Established \Presidios are not given back during Evacuation.
\bparag However, a power may decide at any Peace Phase to dismantle one of its
\Presidios at no cost.

\aparag[Peace and Stability]
Finally, during the conclusion of a peace with a player or an attacking minor
(at war by event), the Stability increases by 1 level for each signatory
player that is now fully at peace.
\bparag In case of peace with multiple enemies, the Stability increase is set
anyway at a maximum of +1 even if peace with different players or minors is
made during the same peace phase.
\bparag A peace with a minor does not increase the player's Stability (unless
this minor has declared war by an event).

\aparag[Indemnities] Indemnites agreed in the Peace are to be paid by one of
the losing Major powers. He may pay them now or at a segment of Announcement
of the two following turns. They can be paid in fractions during the 3 turns
allowed.
\bparag Failure to pay all the due indemnities give a temporary free \CB to
the power that was wronged against the power that should have paid.

\aparag[Peace and Minors]
All active minors of players now at peace immediately sign the peace at the
same time.
\bparag
Exception: a minor at war declared by an event makes peace only through a
specific peace for this war against the ennemy alliance, usinf the procedure
for minor countries.  This is not considered as a separate peace (even if
involved in war alongside its controlling player).



\subsection{Peace with Minor powers}

\aparag To make the peace with a minor, the enemy player has to indicate that
he commits himself to peace negotiations with the minor.  The controlling
player of the \MIN may be at war (against the player) or not; in the former
case it is a "separate peace", attempted and signed before any peace between
players.
\bparag An alliance may usually offer peace with only one minor country per
alliance at war against it and per turn; and this minor may not be a vassal ir
annexed.
\bparag Exception: An alliance can always propose a specific peace to a minor
at war declared by an event .
\bparag Exception: An alliance can offer separate peace to any or all minor
vassal to the alliance or annexed, whose capital city is controls.
\bparag Exception: An alliance can offer separate peace to any or all
non-vassal, and non-annexed minors if controlling at least one of their
provinces (each one), or if each one of them controls individually at least
one province in the alliance.
\bparag If allied powers do not agree, the decision is taken by the Power
having the higher value in DIP (decide ties randomly).

\aparag[Method]
To sign peace with the minor, the player rolls 1d10, taking into account the
sought-after peace level and all applicable modifiers. The level of peace is
chosen by the player, without taking into account his own Stability.
\bparag[Result]
The peace is signed if the modified die-roll result is 6 or more.
\bparag[Peace level Modifier]
This modifier is the triple of the value of the peace level chosen. It is used
as a positive modifier if the offered peace is favorable to the minor, or as a
negative modifier if the offered peace is favorable to the player.
\bparag[Nationality Modifier]
It is applied for the case of conflicts with specific minor countries
which are: \\
-4: \pays{Perse}, \pays{egypte}, \pays{damas}, \pays{Chine},
\pays{Japon} \\
-3: \pays{USA}, \pays{Mogol}, \pays{Venise}, \pays{Pologne},
\pays{Habsbourg},  \pays{Brandebourg} after IV-11\\
-2: \pays{Portugal}, \pays{danemark}
\bparag[Modifiers of Situation]
These are is applied cumulatively according to what happened during the
current turn (and to what happened during previous turns for the
lost/conquered provinces only):\\
+2:	per province/\TP\faceplus/\COL lost by the minor (+4 if capital)\\
-2:	per province/\TP\faceplus/\COL conquered by the minor (-4 if capital) \\
+1:	per \TP\facemoins  lost by the minor (round down) \\
-1:	per \TP\facemoins conquered by the minor \\
+2: if the capital province of the \MIN was conquered this turn, or if it was
captured then lost since \\
-4: if the \MIN has captured a capital province of a \MAJ this turn, -or if it
was
captured then lost since \\
-2:	per major battle won by the minor  \\
+2:	per major battle won by the player \\
-1:	per battle won by the minor \\
+1:	per battler won by the player\\
+1:	per minor military leader killed or captured \\
+2: if the Monarch of the minor country is captured and its Ransom is
used for Peace \\
-1:	per military leader of the player killed or captured\\
+1:	per siege won by the player \\
-1:	per siege won by the minor \\
-2: If minor is heretic (Catholic vs. Protestant, before the end of
the \terme{Religious Dissension}) \\
-2: if it is an attempt to negotiate a separate peace $\pm$?: the PD of the
controller regarding the alliance attempting the peace (if the controller is
at war against the alliance), max. -3/+3
\bparag
All these modifiers are cumulative in one single turn.
\bparag \COL and \TP\faceplus controlled counts as a full province (modifier
of $\pm 2$); the same is true for control of cities of minor countries in the
\ROTW;
\bparag \TP\facemoins controlled gives a modifier of $\pm 1$ only; the same
modifier holds for occupying a province without city of a minor country in the
\ROTW;
%% was \pm 1, remis à 1.5 pour être comme les tables ou la liste de modifieurs
%% ci-dessus...
\aparag[Overseas Wars]
\bparag A Minor country always accepts to sign a White Peace in Overseas War
(if it is not a Separate Peace).

\aparag[Consequences of Peace]
If the peace is signed, no Stability level is gained (exception: if this minor
declared war to the player by an event). The player that controlled the minor
does not earn anything.
\bparag The conditions of Peace are the same as for a Peace between Major
powers.
\bparag A Minor country will at most indemnities up to 4 times its income,
immediately at the conclusion of the Peace. Any other indemnities are void.
\bparag If the minor country is the victor, the player that controls the minor
country chooses the ceded provinces (if any). He must do so in priority among
those located the closest from the minor country's territory, in terms of
movement points (a sea zone is equivalent to 2 MP for this calculation).
\bparag A minor country nevers takes Diplomatic concessions, only provinces
and indemnities.

\aparag[Multiple and Separate Peace]
If a player signs a separate peace with a minor country (he is still at war
against the controlling player); this minor may not be again involved in a war
against him next turn (unless by an event or a Crusade).
\bparag A minor country at war by event may only make a separate peace.

\aparag[Unconditional Peace]
A Minor country will sign a mandatory Unconditional Peace if all of its
provinces are controlled by the ennemy. This peace is one global peace against
all the powers controlling its provinces (so it can lose only 3 provinces).
\bparag If a minor country loses and signs an unconditional peace its
political marker is moved automatically to the box \Neutral.

\aparag[Automatic Peace]
An alliance that proposes a victorious Unconditional Peace to a Minor power,
and that Minor power was the attacking one (caused by event), or is not allied
to an alliance (so this is not a Separate Peace), the Peace is automatically
accepted by the Minor power.

\aparag[Failure of Peace negotiations]
If the die-roll is inferior or equal to 5 the war with the minor continues
normally for the following turn. Another peace attempt with that minor will be
allowed during the peace phase of next turn.

\subsection{Test for crusade}\label{chPeace:Crusade}

% Local Variables:
% fill-column: 78
% coding: utf-8-unix
% mode-require-final-newline: t
% mode: flyspell
% ispell-local-dictionary: "british"
% End:
