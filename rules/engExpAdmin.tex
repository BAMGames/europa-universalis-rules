% -*- mode: LaTeX; -*-

\section{Overview of Administrative actions}

\subsection{General mechanism and list of actions}

\aparag[Mechanism] All administrative actions are solved according to the
following scheme: all administrative actions are written down (including all
details); a column and die-roll modifier is determined for each administrative
action; the column mostly depends on the \terme{investment}, that is the
amount of money put into the action, and the characteristics of the monarch
(for domestic operations) or of the country (for external ones); one die is
rolled for each action in \tableref{table:Administrative Actions}; the result
qualifies the success or failure of the action.
\bparag The Technological roll reads the table in a slightly different way
(see \ruleref{chExpenses:Technology Improvement}).
\bparag If an action has become impossible, the cost is still to be paid. This
is especially the case for competition (because the target of the competition
can be eliminated by somebody else), or any operation that would raise the
level of a \COL, a \TP or a commercial fleet beyond 6 (such as two identical
actions on a level 5 \COL to diminish the chances of failure).

\aparag[Administrative operations]\label{chExpenses:Administrative Actions}
The following operations are available:
\bparag[Domestic operations:] \terme{Creation of \MNU}, \terme{\FTI
  improvement}, \terme{\DTI improvement}, \terme{Exceptional taxes}.
\bparag[External operations:] \terme{Commercial fleet implantation},
\terme{Colonisation}, \terme{Trading-post establishment}.
\bparag[Technological operations:] \terme{Land Technology Improvement},
\terme{Naval Technology Improvement}.
\bparag[Competitions:] \terme{Normal competition} (often called simply
\terme{Competition}) and \terme{Automatic competition} (used to resolve
abnormal situations resulting from simultaneous actions of different
countries).
\bparag[Administration for minors:] Some minors countries have administrative
actions (usually \TradeFLEET implementation, sometimes colonisation) that are
handled by their diplomatic patron.

\aparag[Administrative limits] \label{chExpenses:Administrative Limits} Each
player is entitled to a certain number of actions:
% \bparag Domestic operations cannot be done if \terme{Exceptional taxes} were
% levied this turn or during Military Phase of the previous one.  Only one
% such operation can be done per turn.
\bparag Domestic operations are mutually exclusive. Each country can only
attempt one each turn.
\bparag The limitations for the external operations and (normal) competition
are given in the player-specific tables. They form an upper bound on the
number of operations and a player can choose to do less external operations
than this limit. See~\ref{chThePowers:Turn Limits} for details.
\bparag Both technological operations (naval and land) can be done each turn
but only one may have an investment higher than a \terme{Basic investment}.
\bparag If a player has a limit greater than 1 for a given type of operation
(e.g. Colonisation or Competition), he can choose between either several
separate attempts or a multiple attempt on the same objective (or any
combination).
\begin{exemple}
  With 2 actions of Colonisation per turn in period \period{VI} (1700-1759),
  the English player can make 2 attempts on the same Colony or 1 attempt each
  on 2 different Colonies, in the same turn.
\end{exemple}
\bparag Some players are entitled to actions with specific restrictions
(e.g. \SPA may have actions restricted to \POR administration while
\eventref{pIII:Portuguese Annexation} is in effect).
\bparag Bankruptcies may change the limits of a country on the turn they
occur.

\aparag[Investment] All administrative actions have a general mechanism called
investment: each action can be made with a \terme{Basic investment}, a
\terme{Medium investment} or a \terme{Strong investment}.
\bparag The values of the investments are 10\ducats, 30\ducats and 50\ducats
for the following operations: \terme{Commercial Fleet Implantation},
\terme{Trading-Post Establishment}, \terme{Normal Competition}.
\bparag The values of the investments are 30\ducats, 50\ducats and 100\ducats
for the following operations: \terme{\MNU creation}, \terme{\FTI or \DTI
  improvement}, \terme{Colonisation}, \terme{Technology improvement}.
\bparag The investment changes the column used in~\ref{table:Administrative
  Actions} for the die-roll. Each action is done with one (and only one)
investment, but two similar actions (such as two \terme{Colonisations}) can be
done with different investments during the same turn.
% \bparag The Investment is added last, after apply a threshold of -4 to all
% the sums for the columns.
% \bparag The complete procedure for \terme{Technology improvement} is
% described in \ruleref{chExpenses:Technology}.
\bparag The sum of all investments goes in \lignebudget{Administrative
  actions}.

\aparag Each action (except Exceptional taxes) is resolved
using~\ref{table:Administrative Actions}.
\bparag In each case, a column of the table is determined as explained for
each action.
\bparag In each case, the investment adds 0, 1 or 3 columns to this
computation. In each case, the column is first thresholded between -4 and +4
and then investment is added. Thus, attempting to improve technology with only
3 in \MIL (without \MNU) and a strong investment results in a base column of
-6 (3-9) thresholded at -4 and then switched to -1 (+3 columns for a strong
investment).
\bparag Once the column is determined, a modifier is also determined.
\bparag The result is read by rolling 1d10 plus the modifier above and cross
referencing this with the column used for the action.

\aparag Results of administrative actions is either S, \undemi\ or F
(sometimes with a \textetoile).
\bparag Usual meaning of these are:
\begin{modlist}
\item[S] The action is a Success.
\item[\undemi] The action may be successful. Roll 1d10 and compare with the
  \FTI of the country (special \FTI may apply). If the roll is less or equal
  than the \FTI, treat as S, otherwise, treat as F.
\item[F] The action is a Failure. The money for the investment is lost but
  nothing happens.
\end{modlist}
\bparag Check each specific action for the precise explanation of the
results. They may differ from the general case explained here. Especially,
\terme{Technology improvement} uses a different mechanism to read its result.

\GTtable{admintbl}

%%%%%%%%%%%%%%%%%%%%%%%%%%%%%%%%%%%%%%%%%%%%%%%%%%%%%%%%%%%%%%%%%%%%%%



\subsection{Counters limitation}

\aparag The number of \COL, \TP, \MNU and \TradeFLEET counters that a country
may have in play at a given time is limited.
\bparag This is a limit on the number of counters, each of them may have any
number of level in it.
\bparag The limit of \MNU (only) may be exceeded as
per~\ref{chThePowers:Exceeding Limits}.
\bparag The limit of \TradeFLEET is usually the number of counters provided by
the game (exception: \SUE), while the limit of \COL and \TP evolves as the
game goes and can be found in the country tables (see~\ref{chThePowers:Period
  Limits}).

\aparag If, for some reason, a country has more counters of one type that
allowed, it must immediately remove the exceeding ones (at controlling
player's choice).

\aparag A country may not attempt an action that would create a counter of a
kind whose limit is reached. For example, a country having reached its maximum
number of \COL for the period may try to increase the level of existing ones
but it may not attempt to create a new \COL.
\bparag However, it is possible at the beginning of the administrative phase
to voluntarily destroy \COL, \TP, \MNU or \TradeFLEET in order to free
counters and use them elsewhere. It must be done before actions are planned
(and resolved).




\section{Mandatory actions and bankruptcies}\label{chExpenses:Mandatory}



\subsection{Commercial fleet adjustment}\label{chExpenses:Commercial Fleet
  Adjustment}

\aparag[Temporary losses] \TradeFLEET suffer temporary losses from
piracy. This is handled by having a \terme{maximum level} and a \terme{current
  level}.
\bparag The \terme{current level} represents the current amount of trade a
country has in a trade zone.
\bparag The \terme{maximum level} represents the potential trade that a
country will have once the turmoils caused by piracy will be tamed and
repaired.
\bparag Both the current and maximum levels must be kept for each \TradeFLEET
(by its owner and on the general \TradeFLEET sheet).
\bparag Both these levels are between 0 and 6.
\bparag The current level may never be larger than the maximum level. If this
somehow happens, decrease the current level to the value of the maximum level.
\bparag A commercial fleet is destroyed when its \terme{maximum level} reaches
0, not its \terme{current level}.

\aparag[Current level] Unless specified, when the level of a \TradeFLEET is
mentioned in the rules, use the \terme{current level}.
\bparag Specifically, use the \terme{current level} for deciding which side
(\Facemoins or \Faceplus) the counter should be, computing incomes, deciding
monopolies, allocating Trade Centres and modifying \TFI actions.
\bparag If the \emph{current} level is 0 but not the \emph{maximum} level then
the \TradeFLEET still exists: the counter is still here and cannot be used
elsewhere and the \TradeFLEET is considered as present for all effects where
presence only (ie whatever the level) affects game (eg as modifier for \TFI or
concurrence done by other countries).

\aparag[Maximum level] The maximum level is used only to determine monopolies
for end-of-period \VPs.
% \bparag Any loss or gain of the \terme{maximum level} is also a loss or gain
% of the \terme{current level}

\aparag[Changing levels] Unless specified, any change of level (whether gain
or loss) changes both the \terme{current} and \terme{maximum level}.
\bparag Especially, bankruptcies, \TFI and competitions affect both the
current and maximum levels.
\bparag If this would cause the current level to go below 0, or the maximum
level to go above 6, then only the level that can be affected is modified (eg,
if your opponent has a \TradeFLEET with a current level of 0, you can still do
competition on it to decrease its maximum level ; conversely, if one of your
\TradeFLEET has a maximum level of 6 and a current level of 2, you can spend
money to do \TFI on it and speed up the recovery process).
\bparag Only piracy and automatic adjustment (recovery from piracy) may affect
the current level without affecting the maximum level.

\aparag[Automatic adjustment] Each \TradeFLEET whose \terme{current level} is
smaller than its \terme{maximum level} increases its \terme{current level} by
1 (that is, recovers from previous temporary losses).
\bparag This gain is of 2 levels for a \TradeFLEET that has a \terme{maximum
  level} of 5 or 6. % TBD: or one ?
\bparag Notice that this adjustment automatically happens for each \TradeFLEET
of each country in each \CTZ/\STZ. That is, a player does not have to choose
which \TradeFLEET is adjusted and may not transfer adjustment from one
\TradeFLEET to another or save it for a further turn.

% \aparag This is the segment where non-vassal minor countries increase the
% levels of commercial fleets (\ruleref{chExpenses:Minor Commercial Fleets}).



\subsection{Loan management}

\aparag[Interests] Each country must pay a 10\% interests on all ongoing
loans.
\bparag For international loans, it is 10\% of the original amount, even if it
was partly refunded. Note that once the loan is totally refunded (even before
term), it is no more ongoing and does not require paying interests anymore.
\bparag For national loans, it is 10\% of the current amount (round up).
\bparag Interests must be payed the turn the loan is refunded (ie refunding
loans happens after paying interests in turn order). This prevents one turn
interest-free loans.

\aparag Interests are written on the loan ERS, in \lignebudget{International
  loans interests} and \lignebudget{National loans interests}.
\bparag \lignebudget{International loans interests} is filled when an
international loan is contracted.
\bparag \lignebudget{National loans interests} is filled at this segment, it
is 10\% (round up) of \lignebudget{National loans at start}.
\bparag The sum of \lignebudget{National loans interests} and
\lignebudget{International loans interests} is copied in \lignebudget{Loan
  interests}.

\aparag[Mandatory refund] International loans must be refunded at most 3 turns
after they are contracted. If an international loan ends this turn write in
\lignebudget{Mandatory loan refund} the amount to pay.
\bparag National loans don't need to be refunded\ldots

\begin{exemple}
  At turn 1, \POR contracts an international loan of 70\ducats. It must pay
  7\ducats interest at turns 2, 3 and 4 and refund the loan no later than turn
  4.

  Even if \POR refund 20\ducats of this loan at turn 2, the interests at turns
  3 and 4 are unchanged (7\ducats). However, if \POR fully refund the loan at
  turn 3 (in this case, by paying the 50\ducats left from turn 2), then it is
  no more ongoing and there are no interest to pay at turn 4.

  \smallskip

  At turn 1, \FRA contracts a national loan of 54\ducats. At turn 2, it has
  still 54\ducats of ongoing national loans and must pay 6\ducats interest
  (rounding in disfavour of the player, as always). Then, still at turn 2,
  \FRA decides to refund 30\ducats of this loan but contracts a new one of
  83\ducats. At turn 3, it has 54-30+83=107\ducats of ongoing national loans
  and must thus pay 11\ducats of interests. These loans do not need to be
  refunded and may well last for the whole game if the player wishes so (but
  interests must be payed each turn, actually this represent refunding old
  obligations and contracting new ones).
\end{exemple}

\aparag[Treasure collapse] At this point, if the sum of the \RT and the
\terme{Gross income} minus the loan interests and the mandatory refund is
negative, the country suffers a collapse. This usually happens when the \RT is
highly negative because of several turns of spending much more than the
income.
\bparag In case of collapse, the country \textbf{must} makes a \terme{Complete
  bankruptcy} (see~\ref{chExpenses:bankruptcy} below).



\subsection{Bankruptcy}\label{chExpenses:bankruptcy}

\aparag Players decide whether their country attempts a bankruptcy and which
kind (small, major or complete).
\bparag In case of Treasure collapse, the country \textbf{must} undergo a
Complete bankruptcy (see above).
\bparag Bankruptcies must be declared and resolved before planning
administrative actions as their result can prevent some of them.
\bparag Bankruptcies are declared and resolved immediately.
\bparag Notice that bankruptcies are declared and resolved after interests are
payed and after mandatory refund of international loans.

\aparag[Complete bankruptcy] In case of Complete Bankruptcy, do all the
following, in order:
\bparag Set the \RT to 0\ducats (change the value in \lignebudget{RT after
  Diplomacy}).
\bparag Erase all national loans: Write in \lignebudget{National loans
  bankruptcy} the amount which is currently in \lignebudget{National loans at
  start}.
\bparag Erase all international loans: Write in \lignebudget{International
  bankruptcy} the amount of ongoing international loans (depends on the amount
of previous partial refunds) and erase any value currently in
\lignebudget{International loans interests} and \lignebudget{International
  loans refunds} for the following turns.
\bparag Loss 30\VPs.
\bparag Apply the worst possible bankruptcy result: loss 2 \STAB; loss either
2 levels of \TradeFLEET or 1 level of \MNU (player's choice when a choice
exists); this turn, the country may not attempt domestic actions (\DTI or \FTI
improvement, \MNU placement, Exceptional taxes); the country has 2 \TFI
actions less than normal this turn.
\bparag Loss 1 level of \DTI (unless this would put it below the minimal value
of 1).
\bparag Counts as 2 bankruptcies: it will hamper further Exchequer tests for 5
turns.

\aparag[Major bankruptcy] In case of Major bankruptcy, do all the following,
in order:
\bparag Erase loans: choose between erasing all national loans (write in
\lignebudget{National loans bankruptcy} the amount which is currently in
\lignebudget{National loans at start}) or up to 200\ducats international loans
(write any number between 1 and 200 in \lignebudget{International bankruptcy}
and change the \lignebudget{International loans refunds} of the following
turns accordingly, if this puts the amount of ongoing international loans at
0\ducats, erase the value in \lignebudget{International loans interests} for
the following turns). Only one of the two possibilities can be made with each
Major bankruptcy.
\bparag Loss 15\VPs.
\bparag Apply the worst possible bankruptcy result: loss 2 \STAB; loss either
2 levels of \TradeFLEET or 1 level of \MNU (player's choice when a choice
exists); this turn, the country may not attempt domestic actions (\DTI or \FTI
improvement, \MNU placement, Exceptional taxes); the country has 2 \TFI
actions less than normal this turn.
\bparag Counts as 1 bankruptcy: it will hamper further Exchequer tests for 5
turns.

\aparag[Small bankruptcy] In case of Small bankruptcy, do all the following,
in order:
\bparag Determine amount: choose the amount of national loan erased, between 1
and 200\ducats. This amount may not be larger than the current amount of
national loans (\lignebudget{National loans at start}). Write this amount in
\lignebudget{National loans bankruptcy}.
\bparag Determine effects: roll 1d10, add the \ADM of the monarch and the
\STAB of the country (may be negative), plus any modifier listed
below~\ref{table:Bankruptcy Roll}. Find the result in the first column
of~\ref{table:Bankruptcy Roll} to determine the line in which effects are
read.
\bparag Loss \STAB: According to the effect, a certain amount of \STAB may be
lost.
\bparag Loss \TradeFLEET: According to the effect a certain number of
\TradeFLEET levels may be lost by the country. These may be lost in any \STZ
or \CTZ. If the result is 10 or less, the player may choose to loss 1 level of
\MNU instead of all the levels of \TradeFLEET if the choice exists (that is,
it is not possible to choose to ``lose'' 1 or 2 levels of nonexistent
\TradeFLEET in order to save a \MNU, if the country has less levels of
\TradeFLEET than what must be lost and the result is 10 or less, one level of
\MNU must be lost). However, if the country has no \TradeFLEET and no \MNU (or
no \TradeFLEET and the result is 11 or more), then nothing is lost.
\bparag Loss actions: According to the effect, a certain number of \TFI are
lost for this turn only (if this is more than the allowed number of actions in
a given turn, no \TFI are allowed this turn but there is no ``carry over'' of
lost action to the next turn). If the result is 14 or less, in addition, the
country may not do any domestic operation this turn (\DTI or \FTI improvement,
\MNU creation or Exceptional taxes).
\bparag Loss 5\VPs.
\bparag Counts as 1 bankruptcy: it will hamper further Exchequer tests for 5
turns.

\begin{playtip}
  Bankruptcies affect the Exchequer test for the next 5 turns. In order to
  remember this, one can put a small \textetoile in \lignebudget{Gross income
    A} of the next 5 turns.
\end{playtip}

\begin{exemple}
  At turn 10, with 150\ducats of national loan, more than its income, \RUS
  tries a bankruptcy. The monarch is \monarque{Ivan IV} who is not afraid to
  take money from its boyars without asking; he has an \ADM of 6. \RUS decides
  to do a small bankruptcy in order to ``erase'' all the debt
  (150\ducats). \RUS has a \STAB of 2.

  Thus, the die roll is modified by +6 (\ADM of \monarque{Ivan IV}) +2 (\STAB)
  -3 (larger than 100\ducats bankruptcy) = +5. \RUS rolls 7 for a net result
  of 12. Looking in the table in the line ``11-14'', \RUS lose 1 \STAB (going
  to 1), 1 \TradeFLEET level (but since it has none, nothing is lost), 1\TFI
  for this turn (again not a loss since \RUS has no \TFI in period
  \period{II}) an may not do any domestic action for this turn.

  If the die roll had been 4, and the result 9, then \RUS would had to choose
  between losing 1 \TradeFLEET level or 1 \MNU level, and since it has no
  \TradeFLEET at this point, it would had to lose 1 \MNU level.
\end{exemple}

\GTtable{bankruptcy}

\aparag Results of the bankruptcy are applied immediately, especially before
any administrative action is planned.
\bparag Since the modifiers for some actions depends on the \STAB, the levels
of \TradeFLEET or of \MNU, this may have an impact.
\bparag Moreover, knowing which \TradeFLEET are lost might give another
country trade opportunities (at monopolies or \terme{Trade centres}) and thus
affect the choice of actions and not only their resolution.

\begin{playtip}
  Complete and Major bankruptcies are very costly and should be avoided\ldots
  But sometimes it's better to hang the bankers rather than curse the debt.

  Small bankruptcies can be done quite frequently. Especially with a good
  monarch. \monarque{Felipe II} made a huge use of bankruptcies during his
  reign to avoid paying the bankers. With a good monarch and a good \STAB, a
  small bankruptcy can be almost harmless and a huge relief for the
  budget. However, a backfire is always possible, so don't try them during
  wars.

  The loss of \TradeFLEET and \TFI make small bankruptcies quite costly for
  the commercial powers (such as \ANG or \HOL), especially during the periods
  where the \terme{Trade centres} are disputed. On the other hand, powers such
  as \RUS or \POL will usually not loss much more than 1 \STAB. Beware that a
  poor result (below 10) will still cost them 1 \MNU which can be very
  expensive for these countries with less means to build new ones.

  Bankruptcies hamper the Exchequer test, but loans also do (and they cost
  money each turn in interests). So, using small bankruptcies is often a good
  way to manage loans\ldots Don't hesitate to borrow some money from your
  nobles when in need (at war, usually), even if you don't intend to repay
  them. But take into account the fact that you will need some time at peace
  in order to do your bankruptcies in a good situation (and to refund other
  loans if wanted). Moreover, the Economical system works better if you have a
  small amount of loans all the time, so take that into account when deciding
  whether to go bankrupt or not.
\end{playtip}



\section{Choices of actions}\label{chExpenses:Choice of actions}
\aparag All players simultaneously chose which administrative actions they
which to perform. The description of actions is done in the following Sections
(from \ref{chExpenses:Domestic} to \ref{chExpenses:Techno and Competition}).
\bparag Each planned action is written down. Even if the explanation of the
resolution of the action is done together with the explanation of the action
itself, resolution occurs only when all actions have been planned.

\aparag Maintenance and recruitment of troops and fortresses is also part of
the administrative actions. Even if they have their own description later and
their own Segments in the turn sequence, they must be planned as other
actions.

\section{Domestic operations}\label{chExpenses:Domestic}

\aparag Each country may attempt at most one \terme{Domestic operation} each
turn.
\bparag Bankruptcies may prevent countries from doing any Domestic operation
at a given turn.



\subsection{Manufacture creation}

\aparag The operation of \terme{\MNU creation} uses column
\ADM+\DTI-9+Investment.
\bparag The following modifiers to the die-roll are used:
\begin{modlist}
\item[+?]Stability of country
\item[-1]For \SPA if inflation level is 10\% or more.
\item[-1]For \RUS before construction of \ville{Saint-Petersbourg}, for \TUR
  and for \POL.
\item[+2]For \ENG, from period VI onward.
\item[\textplusminus?]By event
\end{modlist}
\bparag If the result is a ``S'', then one level of \MNU is gained. One can
either turn a counter on its second level side, or take a new \MNU counter.
\bparag The place where the \MNU is built must respect the location
restrictions (below).
\bparag For limits on the number of counters, see \ref{chThePowers:Period
  Limits} and \ref{chThePowers:Exceeding Limits}.
\bparag If the result is ``F'', nothing happens (and the money is lost).
\bparag If the result if ``\undemi'', use the normal procedure: roll 1d10 if
the result is less or equal than the \FTI, treat as ``S'' if larger than the
\FTI, treat as ``F''. Note that \FTI is used even if it played no other role
in this operation (ie it is not used to compute the column, only to resolve
\undemi).

\aparag[Manufacture placement]\label{chExpenses:Manufacture Placement} The \MNU
has to be placed on the European map, in a province which is both owned and
controlled. Some kind of \MNU have specific locations:
\bparag \RES{Cereals} \MNU must be put in a plain province.
\bparag \RES{Wood} \MNU must be put in a wooden province (either sparse or
dense forest). By exception, \ENG must put it preferentially in
\region{Irlande} (but may move it elsewhere if it loses the province), and
\TUR is allowed to put it in \province{Lubnan}.
\bparag \RES{Salt} \MNU must be put in a province with a Salt resource in it.
\bparag \RES{Fish} \MNU must be put in a coastal province.
\bparag \RES{Art} \MNU must be put in a province with an income of 5 or more.
\bparag Only one \MNU may be put in a single province, unless the country does
not own sufficiently many provinces.
\bparag Relocating a \MNU is possible only if the province is ceded or
conquered. In this case, simply take the counter and place it in another legal
province.
% Jym : ??  , but it loses one level doing this.

\begin{exemple}
  At turn 1, \POR wants to develop a \MNU (either to switch one of the two
  existing one from side \Facemoins to side \Faceplus or to create a new
  one). It has a \DTI of 3 and the monarch has an \ADM of 8. Thus, the base
  column is 8+3-9=2. Depending on the investment (30, 50 or 100 \ducats), the
  final column will thus be 2, 3 or 5.

  Spending 100\ducats would be quite a waste since column 5 does not
  exist. \POR may choose to either spend 50\ducats and roll on column 3 or to
  spend only 30\ducats and roll on column 2. The difference between the two
  columns is a \undemi\ changed into a S, that is roughly 10\% more
  success. However, with a DRM of +3 for its \STAB, \POR things that the odds
  are already pretty good and spending more money is useless. So he decides to
  only spends 30\ducats on this action.
\end{exemple}

\begin{designnote}
  Note that is it usually better in term of overall probabilities of success
  to attempt several similar actions with a Small investment rather than a
  single one with a higher investment. This is especially true for external
  actions (because there is no real limit to the number of levels of
  \TradeFLEET or \COL that one country may have, thus even in case of success
  you will do the action another time the next turn) but stay true for
  domestic ones. Making several actions, however, takes more time and if one
  needs the result immediately, a high investment can be a good idea.

  Typically, in this case, with a Small investment (column 2 at +3 with 2
  \FTI) \POR has 76\% of success, with a Medium one (column 3), this goes to
  84\% and to 92\% with a Strong one. Thus, for the same amount of money
  (100\ducats, that is 3 Small investments, 2 Medium or 1 Strong), the average
  number of \MNU created would be 0.92 with Strong investments, 1.68 with
  Medium investments and 2.28 with Small investments.
\end{designnote}

\begin{exemple}[continued]
  So, \POR decides not to spend too much money at once and do a Small
  investment, resulting in a column of 2, with a final DRM of +3 for \STAB. He
  rolls the die for a result of 3, modified to 6. In \ref{table:Administrative
    Actions}, cross-referencing column 2 and line 6 he reads the result of
  \undemi. So he must roll 1d10 under the \FTI. Since this is a domestic
  operation, special \FTI does not apply and the \FTI of \POR is only
  2. Rolling another die gives 1, less than the \FTI, thus the operation is a
  success. \POR may either flips one of the two existing \MNU from side
  \Facemoins to side \Faceplus or take another \MNU counter and put it on side
  \Facemoins.
\end{exemple}

\begin{designnote}
  The risk for exceeding limits of \MNU is only checked at the beginning of a
  turn, before the administrative phase. Thus, at the last turn of a period,
  if your limit in \MNU increase in the following period, you may attempt a
  \MNU creation at no risk.

  The \MNU period objectives require you to have more \MNU than the limit for
  the period. This can be done in two ways: (i) stay in the line all along and
  create an extra \MNU at the last turn of the period, but there is a risk of
  failure and unmet objective ; (ii) try to create the extra \MNU earlier to
  diminish the risk of last turn failure, but there is a risk in case of early
  success followed by a revolt\ldots
\end{designnote}



\subsection{Trade index improvement}

\aparag The improvement of either \FTI or \DTI uses column \ADM-9+Investment.
\bparag The following modifiers to the die-roll are used:
\begin{modlist}
\item[+?]Stability of country
\item[-1]For \SPA if inflation level is 10\% or more.
\item[-1]For \RUS before construction of \ville{Saint-Petersbourg}, for \TUR
  and for \POL.
\item[+2]For \ENG, from period VI onward.
\item[\textplusminus?]By event
\end{modlist}
\bparag A ``S'' is a success, and the \FTI or \DTI increases by 1.
\bparag A ``F'' is a failure and nothing happens.
\bparag A ``\undemi'' is resolved as normal: roll 1d10 and treat as ``S'' if
less or equal than \FTI, ``F'' otherwise.
\bparag For limits, see \ref{chThePowers:Period Limits}.

\aparag[Other Trade Indexes]\label{chExpenses:Special FTI} Some countries
(\POR, \RUS, \SPA, \HOL) have two \FTI: one reserved for some operations in
the \ROTW, and one for all other operations.
\bparag The specific \FTI can be used in all covered administrative operations
instead of the normal \FTI, to determine the column as well as for the case
where a \undemi\ is obtained.
\bparag Improvement of the \FTI does increase the two values; that is,
consider the special \FTI to be ``regular \FTI +n'' rather than a value by
itself.
\bparag However, it is possible to increase the special \FTI only (to avoid
going over the limit of regular \FTI). This is done by a regular \FTI
improvement action.
\bparag Conversely, if the special \FTI is already at its maximum, increasing
the regular \FTI does not change it.

\begin{exemple}
  At turn 1, \RUS has a \FTI of 1. Since this is also its limit for the first
  period, it may not increase it. However, \RUS has a special \FTI with a
  limit of 3 for period \period{I}. Thus it may attempt to increase its
  special \FTI. With a \ADM of 6, this gives a base column of -3 before
  investment. The DRM is +2 (+3 for \STAB but -1 for Russian under-development
  before \ville{Saint-Petersbourg}).

  Let's suppose that by the end of period \period{V}, \RUS managed to increase
  its special \FTI to 4 (the regular one is still at 1 since this is the
  limit). At the start of period \period{VI}, the Russian limit for \FTI
  becomes 2. Thus, \RUS may increase its regular \FTI to 2. In case of
  success, this also increases the special \FTI to 5.

  At the start of period \period{VII}, if \ville{Saint-Petersbourg} has been
  created, the limit of \FTI goes to 3. Thus, \RUS may increase if
  again. However, since the special \FTI is already at its limit of 5, it does
  not change.
\end{exemple}

% \aparag[Limits of Domestic operations]\label{chExpenses:Domestic Limits}
% There is a limit on the number of \MNU counters, \FTI and \DTI per country
% specific to the current period.
% \bparag The limit of \MNU (only) \emph{can} be overridden by a country, but
% there is a risk of catastrophe. See~\ref{chThePowers:Exceeding Limits}.
% % \bparag In case a revolt happens in a province of the player and the die
% % used to determine the strength of the revolt is even, then the player
% % immediately loses 1 level of \DTI, 1 level of \FTI, 1 \MNU, 2 \STAB
% % (usually
% % during the event phase).
% \bparag There is an absolute limit:
% % Jym: Now, no more exceeding FTI/DTI.
% % the \FTI and \DTI must not exceed 5 or the current limit+2, and
% the number of \MNU counters must not exceed the number of available counters
% nor the current limit+2.

\begin{playtip}
  Since the \DTI improves the column for \MNU creation, it is easier to first
  improve \DTI and then try to improve \MNU. But \MNU provide more advantages
  than simply money and you may want then asap.
\end{playtip}



\subsection{Exceptional taxes}\label{chExpenses:Exceptional Taxes}

\aparag[Condition] In order to raise \terme{Exceptional taxes}, a country must
both:
\bparag be at war (including civil or religious wars but excluding overseas
wars) and
\bparag be able to pay the eventual cost in \STAB (see below). That is, a
country at -3 \STAB may not raise exceptional taxes unless these don't cost a
loss of \STAB.

\aparag[Loss of \STAB] A country raising exceptional taxes immediately loses 1
\STAB unless both of the following conditions occur:
\bparag The country is involved in at least one regular war (ie neither civil
nor religious nor overseas) and
\bparag At least one national province is either controlled or besieged by an
enemy in this war.

\begin{designnote}
  In other words:
  \begin{itemize}
  \item Oversea wars don't allow one to raise exceptional taxes because the
    people won't see the need of raising funds to defend a few acres of snow.
  \item Religious or civil wars allow to raise taxes but this always causes
    turmoil (loss of \STAB) because there is always an opposite faction within
    the country to disagree with the need.
  \item ``Normal'' wars allow to raise for taxes and if a province is occupied
    the people even see that the nation is in danger and everybody gladly
    gives money for war effort without second thought.
  \item If you're not able to pay for the \STAB cost, that means that the
    country is so wary of the war that people can't and won't do more
    efforts\ldots except in case of great danger (where you don't need to loss
    \STAB).
  \end{itemize}
\end{designnote}

\aparag To proceed with the taxes, the player announces he will perceive the
taxes, and his country loses 1 \STAB level (if needed). The decrease in \STAB
occurs immediately at the beginning of the administrative segment (hence,
before any other administrative action is resolved).
\bparag The modifier is obtained by adding the \ADM of the monarch, the value
of Stability level multiplied by 3, and other possible modifiers (from
events). It is written in \lignebudget{Exceptional taxes modifier B} and
copied in \lignebudget{Exceptional taxes modifier A}.
\bparag Only at the end of turn (after expenses\ldots) will the real amount of
the taxes be known. It will be obtained by rolling 1d10, adding the previous
modifier, and multiplying this sum by 10. This number of \Ducats is added to
the \RT in \lignebudget{Exceptional taxes}.
\bparag The result could be negative, with a low \STAB.
\bparag \textbf{Remember:} do not roll for exceptional taxes during
income. Only write down the modifier. The exact roll will happen at end of
turn, once expenses are planned. Thus, one can only get a rough estimate of
this amount and must spend money according to this estimation.

\begin{exemple}[Good taxes]
  At the beginning of the Seven Years War, \monarque{Friedrich II} decides to
  gets extra income to prepare the invasion of Saxony and raises exceptional
  taxes. The \STAB of \PRU is +3 (as the war was declared with a free \CB), so
  \PRU is allowed to raise taxes. It immediately loses 1 \STAB.

  The \ADM of \monarque{Friedrich II} is 9, the \STAB of \PRU is now 2
  (3-1). Thus, the modifier is 9 (ADM) + 3 $\times$ 2 (\STAB) = 15, written in
  \lignebudget{Exceptional taxes modifier B} and copied in
  \lignebudget{Exceptional taxes modifier A}. At the end of the turn, \PRU
  checks the precise amount of the taxes by rolling 1d10 and gets 7. Thus, the
  final amount is 10 $\times$ (7+15) = 220\ducats. All in all, a good
  operation, but things could not go wrong with high \STAB and \ADM.

  \smallskip

  Later in this war, the Russian armies have invaded \provinceBrandenburg and
  are pillaging \villeBerlin! Moreover, the war in Bohemia did not went that
  well and war weariness took its toll, thus decreasing the Prussian \STAB to
  0. \monarque{Friedrich II} decides to raise exceptional taxes again. Since
  one national province (\provinceBrandenburg) is currently besieged by
  Russian troops, there is no need to loose \STAB. Hence the modifier for the
  taxes is 9+3$\times$0=9. This still guarantees a good income (at least
  100\ducats).
\end{exemple}

\begin{exemple}[Bad taxes]
  During the French wars of religion, the French monarchy is desperately
  looking for money to fund its campaign against the Huguenots and repeatedly
  summons the \emph{États généraux} in successive attempts to increase taxes.

  French king, \monarque{Henri III} has an \ADM of 6. and \FRA has a \STAB of
  -2 due to the already long turmoil. Since it can loss 1 \STAB, \FRA is
  allowed to raise exceptional taxes. Then compute the modifier of 6 (ADM) + 3
  $\times$ -3 (\STAB) = 6 - 9 = -3. Write this on the ERS.

  At the end of the turn, the exact amount of taxes is computed. Rolling 1d10
  gives only 2, for a final amount of 10$\times$(2-3) = -10\ducats\ldots \FRA
  actually has to pay some money as result of this operation (representing
  cost for gathering members of the \emph{États généraux} and to send tax
  collectors without a large success). Note that it is not possible to
  renounce this ``taxes'' once the result is known, so better check the
  modifier before deciding and estimate risks cautiously.

  \smallskip

  A couple of years later, the Holy League takes arms against the French king
  and immediately takes control of \villeParis. In a bold move,
  \monarque{Henri III} decides to assassinate the League leader,
  \leaderGuise. This creates such a turmoil that the \STAB of \FRA goes to -3
  and a new religious war erupts.

  \FRA would like to risk exceptional taxes again. However, its \STAB is
  already at -3 so it cannot pay for it. \villeParis is enemy-controlled but
  this does not provides ``cost-free'' taxes during religious wars. So,
  exceptional taxes are not possible until \FRA somehow manage to raise its
  \STAB.
\end{exemple}




\section{External Operations}



\subsection{Trade fleet Implantation}

\aparag[Commercial Fleets] The operation of \terme{Trade fleet implantation}
targets a \STZ/\CTZ. Then, use base column
\FTI-\#\emph{Fleets}+Investment.%, with the
% following modifiers:
\bparag See the access limitations in \ruleref{chExpenses:Limited Access} for
restriction on the seas that may be targeted.
\bparag \#\emph{Fleets} is the number of foreign commercial fleets in the
targeted \STZ/\CTZ, whatever their side.
\bparag When targeting its \CTZ, a country adds its \DTI to the initial
column.
\bparag When targeting the \CTZ of another country, the \DTI of the owner of
the \CTZ is subtracted from the initial column.
\bparag \POR and \HOL (after creation of the VOC) use their special \FTI if
targeting a \STZ in the \ROTW, both for column computation and to resolve
\undemi.
\bparag The following modifiers to the die-roll are used:
\begin{modlist}
\item[+1]If attempting country's \TradeFLEET is already \Faceplus.
\item[-1]If at least one pirate is present in the \STZ/\CTZ.%
\item[-1]If there was at least one battle, pirate or privateer during the
  previous turn in the targeted \STZ/\CTZ.
\item[\textplusminus?]By event.
\end{modlist}

\bparag A result of ``S'' increases the level of the \TradeFLEET by 1. If this
is the first level, put a counter in the \STZ/\CTZ (beware that the number of
counters is a restriction on the number of \TradeFLEET a country may have, and
some countries have other limits). If the \TradeFLEET reaches level 4, turn
the counter on its \Faceplus side. A \TradeFLEET may never have more than 6
levels.
\bparag See \ruleref{chExpenses:Trade Competition Mandatory} if two fleets are
\Faceplus in the same \STZ/\CTZ or when one \TradeFLEET reaches level 6.
\bparag A result of ``F'' is a failure: nothing happens but the money is lost.
\bparag A result of ``\undemi'' is treated as normal: roll 1d10, if less or
equal than \FTI (use special \FTI if allowed), treat as ``S'', otherwise,
treat as ``F''.

\begin{exemple}
  At turn 1, \ANG wants to increase its trade in the Baltic sea and attempts a
  \TFI in \stz{Baltique}. The base column is 2 (\FTI) - 4 (for the presence of
  4 others \TradeFLEET : \paysHollande, \paysDanemark, \paysHanse and
  \paysSuede) = -2, the DRM is 0. Maybe, trying to raise the \FTI first could
  be a good idea.

  At turn 1, \VEN wants to increase its trade in the Adriatic and attempts a
  \TFI in \ctz{Venise}. The base column is 3 (\FTI) + 3 (\DTI, since the
  target is its own \CTZ) = 6 which is thresholded to 4 (the maximum possible)
  and the DRM is +1 because the \TradeFLEET is already \Faceplus.

  At turn 1, \TUR wants to try and steal the Venetian trade in Adriatic and
  attempts a \TFI in \ctz{Venise}. The base column is 2 (\FTI) - 3 (\DTI of
  \VEN since the target is someone else \CTZ) -1 (for the presence of the
  Venetian \TradeFLEET) = -2 and the DRM is 0.
\end{exemple}



\subsection{Colonies}


\subsubsection{Normal procedure}
\aparag[Colonisation] The operation of \terme{Colonisation} targets a province
in the \ROTW that does not already contains a foreign \COL. Then use column
\FTI-Difficulty+Investment (where Difficulty is the Difficulty value for the
\Area).
\bparag See the access limitations in \ref{chExpenses:Pioneering},
\ref{chExpenses:Inland advance} and \ref{chExpenses:Native empires} for
restriction on the provinces that may be targeted.
\bparag Countries with a special \FTI use it, both for column computation and
to resolve \undemi.
\bparag The following modifiers to the die-roll are used:
\begin{modlist}
\item[+2]If the province has been pacified (all natives killed).
\item[-1]If at least one battle occured in the \Area during the preceding
  turn.
\item[+2]If it is the improvement of an already existing \COL.
  % Jym: not in the tables.
  % \item[-1]If target already enemy occupied (fort, forces).
\item[-3]For the first ever colonisation attempt by the country during the
  whole game.
\item[-2]For the second colonisation attempt (see above), if the first was a
  failure.
\item[-1]For the third colonisation attempt (see above), if the first two were
  failures.
\item[\textplusminus?]By event.
\end{modlist}
\bparag Use also one (and only one) of the following modifiers:
\begin{modlist}
\item[+M]Manoeuvre of a Conquistador or a Governor in the province.
\item[+B]Bonus of a Missionary in the province.
\item[+1]If a Governor is in the \Area.
\end{modlist}

\bparag A result of ``S'' increases the level of the \COL by 1. If this is the
first level, put a counter in the province (beware that the number of counters
usable during each period is restricted). If the \COL reaches level 4, turn
the counter on its \Faceplus side. A \COL may never have more than 6 levels.
\bparag A result of ``F'' is a failure: nothing happens but the money is lost.
\bparag A result of ``\undemi'' is treated as normal: roll 1d10, if less or
equal than \FTI (use special \FTI if allowed), treat as ``S'', otherwise,
treat as ``F''.

\aparag\label{chExpenses:Colony:Critical failure} An unmodified result of 1 or
2 (even if the action is a success) requires a second roll of 1d10: if it is
\emph{strictly less} than the Tolerance value for the \Area, the natives are
immediately activated and will attack during the redeployment phase (as per
\ruleref{chRedep:Native Attack}).
\bparag Note that \Area with no Tolerance (eg in \continent{America}) are not
subject to this critical failure.

\aparag[Special cases]
\bparag If per chance several players choose the same province for a first
\COL implantation, they will do an automatic competition between their
\COL. Resolve it as automatic competition between \TP with the loser(s) losing
1 level of \COL until only one country still has levels
here. See~\ref{chExpenses:Automatic Competition}.
\bparag If the province was occupied at the beginning of the administrative
phase (either by a fort, or military forces, or a \TP of another country), the
implantation gives an Overseas \CB to this country (even in case of
failure). By exception, this \CB is used at the end of the administrative
phase. Minors never use this \CB.
\bparag If a \COL is successfully created and survives automatic competition,
enemy Forces in the province are repatriated to the nearest \TP or \COL, an
enemy fort or \TP in the province is destroyed.
% \bparag To create a \COL in a province where there is an already-existing
% city, the country must have taken the military control of the city (in an
% Overseas war), and still hold it during the administrative phase (meaning
% that the war still is in effect). Note that since the war The \MAJ can
% attempt to put the \COL, but it does not say anything about keeping it
% during the military phase.
\bparag There is a rule to transform a \TP in a \COL if a city is present in
the province. See below.

\begin{exemple}
  At turn 1, \POR wants to raise its \COL of \construction{La Praya} in
  \granderegion{Cabo Verde}. The Difficulty is 3, the \FTI of \POR is 5
  (special \FTI for the \ROTW) and \POR chooses to make only a small
  investment. Thus the base column is 5 (\FTI) - 3 (Difficulty) = 2. There is
  a \bonus{+2} DRM because the \COL already exists.

  \POR rolls a 2 for a net result of 4, in column +2 this gives a \undemi. So
  \POR has to roll lower than its \FTI (use special \FTI again) and rolls a
  6. This is a failure; the \COL gains no level but the 30\ducats of the
  action are lost. It is an unmodified die roll of 1 or 2 and can thus
  activate natives! But since there is no Tolerance in \granderegion{Cabo
    Verde}, it has no impact.

  \smallskip

  On turn 2, \HIS has left \leaderColon in \granderegionCuba and wants to
  create a \COL in the rich new World. The \FTI of \HIS is 2, the Difficulty
  of the \Area is 3. This gives a base column of -1. Since \HIS wants a base
  in America has soon at possible, it does two \COL actions there and each of
  them with a Medium investment (50\ducats) to roll on column 0 (which has one
  less F than column -1).

  There is a \bonus{+3} DRM for the MAN of \leaderColon (halved on land for
  \LeaderE). Since \HIS has not succeed in any \COL attempt in the game, one
  of the attempts (player's choice, but in this case it is not important since
  they are otherwise the sames) will be the ``first one'' and suffer a DRM of
  \bonus{-3} while the other only has \bonus{-2}. Note that since all actions
  must be scheduled before any is resolved, the second malus will stay even if
  the first attempt is successful (the worst case for the player applies).

  So, the first attempt is in column 0 at 0. \HIS rolls 6 and gets \undemi, a
  second roll of 4 is larger than the \FTI, thus it is a failure. The second
  attempt is still in column 0, but at +1. \HIS rolls 7 for a net result of
  8. It's a success! A \COL of level 1 is put in the province and \HIS won't
  suffer the ``first attempts'' malus anymore (on following turns).
\end{exemple}

\begin{playtip}
  The MAN of a \LeaderC is very important for creating \COL (and \TP). Indeed,
  each point of MAN is basically 10\% more chances of success. So, at the end
  of a turn, you have to think in advance to where you'll want to colonise on
  the next turn and place your leaders there.

  For countries with many good \LeaderC (\HIS, \POR and in a smaller measure
  \HOL then \FRA), correct placement is the key to a very fast grow of the
  colonial empire at small cost. Since even in column -4 there is a success,
  with a \LeaderC with a MAN of 5, this means 60\% chance of S (plus some
  other for the \undemi)\ldots

  On the other hand, countries with less \LeaderC (\ANG) will sometimes need
  several actions to put a \COL. The bonus for an existing \COL will help them
  to concentrate on existing establishments, but spreading the empire is
  harder and must be done either with the few \LeaderC you'll get or with
  massive amounts of money (Large investments).

  The malus for first attempts is very painful, especially with no \LeaderC to
  overcome it. Basically, unless you have someone competent or vast amount of
  money at your disposition, consider that there is a 90\ducats ``fee'' to
  enter the colonial game and that the first three attempts are wasted in
  paying it. A good surprise may arise.
\end{playtip}


\subsubsection{Transforming a trading-post in a colony}\label{chExpenses:TP to
  Col}
\aparag A \TP in a province \emph{with a city} or \emph{with a mission} can be
turned into a \COL following this procedure:
\bparag The \MAJ announces this during the diplomatic phase.
\bparag The \MAJ must declare war (overseas or regular) to the \ROTW minor
country owning the \Area (if any). If already at war against it, there is no
need to declare a new war.
\bparag The natives of the province are automatically and immediately
activated and will attack at the end of the turn as per~\ref{chRedep:Native
  Attack}).
\bparag The country must spend one \COL action with strong investment
(100\ducats) during administrative phase. There is no die roll to resolve this
colonisation attempt.
\bparag The city must be controlled at the end of the military phase (either
taken this turn or a previous one). The fortress level used for the defence
against the native attacks is the better between the one of the \TP and the
one of the city.
\bparag If, after the native attack, the \TP still exists and the city is
controlled, the \TP is turned into a \COL of the same level as the \TP. The
fortress level used for the \COL is the maximum between the fortress of the
city and the one of the \TP (put a fortress counter for free is needed).
\bparag If a mission was used to convert the \TP to a \COL, there is no need
to capture the city (if any), but the mission can never be removed unless the
\COL is lost.

\aparag[Bengal] If \paysMogol own the \granderegion{Bengale}, and some country
has a \dipAT with them, the transformation of a \TP in a \COL in
\villeCalcutta will not generate a reaction by \pays{mogol} (only the natives
will attack) neither for the capture of the city, nor for the presence of
forces in the province to do it, and troops of \paysMogol do not participate
in the indigenous attack at the end of the turn (see also
\ruleref{chIncomes:TradeIndia}).
\bparag Note, however, that the presence of a \COL afterwards can still
trigger reaction of \paysMogol as per~\ref{chDiplo:Diplo-mogol}.

\begin{designnote}
  This is what is considered to have happened in \villeGoa (with Portuguese
  special rules), in \villeJakarta (by \leaderCoen) and in \villeCalcutta
  (established as a British centre of trade and power in India
  after~\ref{pVI:Last Great Mughals}).
\end{designnote}



\subsection{Trading posts}

\aparag[Trading posts] The operation of \terme{Trading-post establishment}
targets a province in the \ROTW that does not already contains a foreign
\COL. Then use column \FTI-Tolerance+Investment (where Tolerance is the
Tolerance value for the \Area, use Difficulty if the \Area has no Tolerance).
\bparag See the access limitations in \ref{chExpenses:Inland advance} and
\ref{chExpenses:Native empires} for restriction on the provinces that may be
targeted.
% \bparag See the access limitations in \ruleref{chExpenses:Limited Access}.
% for restriction on the provinces that may be targeted.
\bparag Countries with a special \FTI (except \HIS) use it, both for column
computation and to resolve \undemi.
\bparag The following modifiers to the die-roll are used:
\begin{modlist}
\item[-1]Per \TP of another country in the \Area.
\item[-1]If target already enemy occupied (fort, forces).
\item[+2]If the province has been pacified (all natives killed).
\item[-1]If at least one battle occured in the \Area during the preceding
  turn.
\end{modlist}
\bparag Use also one (and only one) of the following modifiers:
\begin{modlist}
\item[+M]Manoeuvre of a Conquistador or a Governor in the province.
\item[+B]Bonus of a Missionary in the province.
\item[+1]If a Governor is in the \Area.
\end{modlist}

\bparag A result of ``S'' increases the level of the \TP by 1. If this is the
first level, put a counter in the province (beware that the number of counters
usable during each period is restricted). If the \TP reaches level 4, turn the
counter on its \Faceplus side. A \TP may never have more than 6 levels.
\bparag A result of ``F'' is a failure: nothing happens but the money is lost.
\bparag A result of ``\undemi'' is treated as normal: roll 1d10, if less or
equal than \FTI (use special \FTI if allowed), treat as ``S'', otherwise,
treat as ``F''.

\aparag[Critical failure]\label{chExpenses:TP:Critical failure} A result of
``F\textetoile'' means that in addition to the failure, the natives are
immediately activated and will attack during the redeployment phase
(\ruleref{chRedep:Native Attack}). A result of ``F'' means a simple failure.

\aparag[Special cases]
\bparag If per chance several players choose the same province for a first
\COL and a first \TP implantation, the \TP is eliminated.
\bparag If the target province contains several \TP at the end of the round,
there will be an automatic competition between the \TP of the province.
\bparag If the province was occupied at the beginning of the administrative
phase (either by a fort, or military forces, or a \TP of another country), the
implantation gives an Overseas \CB to this country (even in case of failure of
if the automatic competition allows the former \TP to stay in place). By
exception, this \CB is used at the end of the administrative phase.
\bparag If a \TP is successfully created and survives automatic competition,
enemy Forces in the province are repatriated to the nearest \TP or \COL, an
enemy fort in the province is destroyed.

\aparag[Trading-posts, forts and cities] A European fort or \TP is considered
as a separate place as the city in the same province. Sieges are made against
one or the other (besieger's choice).
% \bparag They are the same for \ROTW countries, which means taking the
% military control of a \TP may requiring being at war against two different
% minor countries.
\bparag However, the \TP of a \ROTW country is considered to be in the
city. This means that in order to take military control of the \TP, a power
must take the city. This may require being at war against two minor countries
(typically in \continent{India} where \paysGujerat has \TP in cities owned by
\paysVijayanagar).

\begin{playtip}
  America is meant to be colonised, not to receive \TP. So, the missing number
  in American areas is the one used for \TP. If you don't remember whether the
  second or third number should be used for \COL or \TP placement, look in
  America: the missing number is the one to use for \TP.
\end{playtip}



\subsection{Limited access to the \ROTW}\label{chExpenses:Limited Access ROTW}

\aparag Trade and colonisation in the \ROTW must progress slowly through
unknown areas. Thus, it is not possible to colonise any province nor to
attempt trade (with \TradeFLEET) before having strong contacts with natives or
colons.

\aparag Some countries have other specific restrictions on where they may put
their establishments. See specific rules for details.


\subsubsection{Trade fleet}\label{chExpenses:Limited Access}
\aparag[Caspian sea] In order to target the \stz{Caspienne} for \TFI, one must
either own a province adjacent to the sea (even without port), the \CCs{Grand
  Orient} or the \CCs{Mediterrannee}.

\aparag[\ROTW \STZ] In order to target a \ROTW \STZ for \TFI, at least one sea
zone must have been discovered in the \STZ, and at least one condition among
the following must be fulfilled:
\bparag The country has a \COL/\TP bordering the \STZ
\bparag \rulelabel{chExpenses:Limited Access:Giving Rights} The country has
trade rights given by somebody that has a \COL/\TP bordering the \STZ (a minor
country that is at least in \MA will give the rights). Once given, the trade
rights are removed by \terme{trade refusal}, or as soon as the minor country
is no more in \MA.
\bparag The \STZ is either \stz{Canarias}, \stz{Guinee}, \stz{Oman},
\stz{Indien} or \stz{Formose} and the country has discovered all the sea zones
of the \STZ (i.e. the sea zone in which is the symbol, plus all neighboring
ones).
%% [TBD: Tempetes exclu car pas de civilisation construite et commercante ici]

\aparag If the condition allowing a new implementation disappears, the
\TradeFLEET can no more increase in level through administrative actions, but
remains where it is.

\aparag[Competition for trade fleets] \TradeFLEET competition can be done if
either a \TradeFLEET implementation would be allowed, or if a \TradeFLEET
already exists.

\begin{designnote}
  At least one sea must be known in order to known where to send
  traders. Then, the other conditions represent who will trade with you:
  either your own colons and merchants, or those of a country giving you trade
  rights or those of natives in organised areas once you know the sea
  sufficiently well.
\end{designnote}

\begin{exemple}
  Notice that even if \stz{Canarias} touches Europe and European provinces on
  the \ROTW (\provinceAcores and \province{Cabo Verde}), this is not
  sufficient to increase its trade there. A \COL/\TP (such as \construction{La
    Praya}) must be here. Specifically, at turn 1, \ANG, \FRA, \HIS and \POR
  all have a \TradeFLEET in \stz{Canarias} but only \POR is allowed to
  increase it. Other may do it after exploring all the seas of the \STZ.
\end{exemple}


\subsubsection{Pioneering [TBD]}\label{chExpenses:Pioneering}
% \begin{designnote}
%   Choose either old style or new style. New style is [TBD] and may be vetoed
%   by Pierre at any time.
% \end{designnote}
% \aparag[old style] Except for \COL producing gold and one extra \COL per
% country, a \COL in an \Area of \continent{America} that has only resources
% that don't exist yet cannot exceed level 2.
% % , except if it produces gold.
% \bparag For \HIS, the \COL of free level must be in \continent{Caraibes}.
% \bparag In addition, in each \Area producing \RES{Wood}, one \COL per
% country may be raised at any level, but it must exploit \RES{Wood} as soon
% as possible.
% \bparag Change of ownership of a \COL does not decrease its level, even if
% this causes a country to have more than one ``unproductive'' \COL of high
% level.

\aparag During periods \period{I}-\period{V}, a province with a \COL of level
2 or 3 may not be the target of colonisation attempts.
\bparag Exception: provinces with gold mines always ignore this restriction.
\bparag Exception: provinces with missions always ignore this restriction.
\bparag Exception: provinces with arsenals always ignore this restriction.
\bparag Exception: If a country benefits from an \ref{eco:Rush Colonists}, an
\ref{eco:Refugees}, or a \terme{Colonial Dynamism} political event (as well as
a few other events), it may ignore this restriction for the turn.
\bparag Exception: \TUR ignores this restriction in \Area belonging to Muslim
minors (\granderegionAden, \granderegionOman, \granderegionSoudan and, if they
still belong to \paysGujerat, \granderegionGujarat and \granderegionMalacca).
% Jym: otherwise, the Turkish colonial expansion is impossible...
\bparag Exception: \SUE ignores this restriction if it has a policy of
\terme{Overseas expansion}.
\bparag Exception: \HOL may ignore this restriction after~\ref{pIII:VOC} on
turns it choose to destroy another \COL.
% \begin{designnote}
%   This is both a bit more strict (no exception for \RES{Wood} or free \COL)
%   and a bit more loose (as two attempts on a level 1 \COL may raise it to
%   level 3). Political events allow for dynamic countries to override this
%   restriction and economical events allow for others to do so. But one must
%   be somewhat prepared and have a \COL of high level ready to be raised in
%   case such an event occurs.
% \end{designnote}


\subsubsection{Inland advance}\label{chExpenses:Inland advance}
\aparag[Settlements] A province may not be targeted by a \COL or \TP attempt
unless the province has been discovered by the country, is linked to a
province on the European map by a continuous path of known sea zones and
provinces (even if enemy-occupied or through closed straits fortifications),
and at least one of the following conditions is true:
\bparag The province is coastal.
% \bparag The \Area contains a \COL or \TP of the same country and the
% targeted province is within supply distance (12\MP) of it.
% \bparag The \Area is adjacent to an \Area containing a \COL or \TP of the
% same country and the targeted province is within supply distance (12\MP) of
% it.
% \bparag The \Area is adjacent to an European province owned by the country
% and the targeted province is within supply distance (12\MP) of it.
\bparag The province is within supply distance (12\MP) by land only of a \COL,
\TP or European province owned by the country.
\bparag No (other) exception.

\aparag[Inland advance [TBD]] A province with a \COL (any level) may not be
targeted by a colonisation attempt unless one of the following conditions is
true:
\bparag The province is coastal.
\bparag The \Area contains a \COL\faceplus or a mission of the same country.
\bparag The \Area is adjacent to an \Area containing a \COL\faceplus or a
mission of the same country.
\bparag The \Area is adjacent to an European province owned by the country.
\bparag No (other) exception.

\begin{designnote}
  Thus, one must first colonise coasts before going inland. Two attempts on an
  empty province may raise the \COL to level 2 without problem. Notice that to
  raise a \COL with an inland gold mine to a high level, you must still fulfil
  this condition (by, typically, building a mission in the \Area).
\end{designnote}


\subsubsection{Native empires}\label{chExpenses:Native empires}
\aparag[Siberia] A province of \continentSiberia, east of the \Area\
\granderegionSiberie may not be targeted for \TP or \COL implementation as
long as the minor country \paysSiberie exists. Provinces of \Area\
\granderegion{Siberie} can be targeted though.
\bparag This restriction is permanently removed when \pays{Siberie} is
destroyed (see~\ruleref{chSpecific:Siberia}).

\aparag[Cities] A province with a city may not be targeted by a colonisation
attempt unless one of the following conditions is true:
\bparag A \COL of the country already exists in the province.
\bparag This is an attempt to transform a \TP to a \COL as per
\ref{chExpenses:TP to Col}.
\bparag The country attempting the action has taken military control of the
city (in an Overseas war), and still holds it during the administrative phase
(meaning that the war still is in effect). Note that since the war is still
ongoing, the new \COL may well be destroyed by native attacks later this
turn\ldots
\bparag No (other) exception.

%%%%%%%%%%%%%%%%%%%%%%%%%%%%%%%%%%%%%%%%%%%%%%%%%%%%%%%%%%%%%%%%%%%%%%




\section{Other administrative operations}\label{chExpenses:Techno and Competition}



\subsection{Technology}\label{chExpenses:Technology}


\subsubsection{Procedure for technology progression}
\aparag The administrative action of raising technology is special because it
is done both by the major countries and the minors countries (or rather the
cultural groups). Moreover, some progression may occur due to events as well
as in the administrative phase. Lastly, the Administrative table is not read
in the usual way when resolving this action.
\bparag Once everybody has performed its increase technology action (both
majors and minors), an adjustment of counters occurs. Check the precise
procedure below and follow it closely. The order in which the adjustments
occur is important and must be precisely respected.

\aparag[Technology: general procedure] The improvement of technology is done
as follows, each step must be completed by all countries before moving to the
next. Some steps occur both during event and administrative phases while some
occur only during the administrative phase (during the Technology adjustment
segment).
\bparag Progression through events (majors and cultural groups, event phase);
\bparag Progression through administrative operations (majors, administrative
phase);
\bparag Minor countries progression (cultural groups, administrative phase);
\bparag Cultural groups adjustment (cultural groups, event and administrative
phases);
\bparag Goals adjustment (goals, event and administrative phases);
\bparag Goals time adjustment (goals, administrative phase).

\aparag[Progression] When a country or cultural group gains technological
levels, advance the corresponding marker (Land or Naval) in the corresponding
box.
\bparag A marker can never stack with a goal of the same kind (Land or
Naval). If a technology marker exactly reaches the box where a goal is, then
it gains one extra level for free and is put just after the goal.
% \bparag When \SPA reaches the Tercios marker, \SPA gains the Tercios
% technology.  When \VEN reaches the Galleass marker, \VEN gains this
% technology.  However, no extra box is gained for reaching one of these two
% markers.

\begin{exemple}
  At turn 11, the Land technology of \FRA is at level 20 and the goal \TARQ at
  level 21. \FRA manages to gain 1 level of Land technology, thus reaching
  level 21. Since the marker for \FRA may not stack with the marker for \TARQ,
  \FRA gains an extra level for free and is now at level 22.
\end{exemple}

\aparag[Events and majors] Follow the text of any event (economical or
political) that tells to move some technology markers.
\bparag Major country may progress in technology through an administrative
action. See~\ref{chExpenses:Technology Improvement} for details.

\aparag[Cultural groups progression] There are four symbols in the turn track:
\techlatin, \techislam, \techortho, \techrotw. When one of these symbols is in
the current turn box, the technology counters (both Land and Naval) for this
group advance of 1 box during the administrative phase.

\aparag[Cultural groups adjustment]
\label{chExpenses:Technology:Cultural Adjustment}
If the technology of a cultural group is 7 or more levels below the technology
(of the same kind: Land or Naval) of a major belonging to that group, increase
the level of the cultural group so that it is only 6 levels below the highest
major of that group.
\bparag Remind: \POR, \SPA, \FRA, \ANG, \VEN, \HOL, \SUE, \AUS, \PRU are in
the Latin group ; \POL is \textbf{both} in the Latin and Orthodox groups ;
\RUS is in the Orthodox group and also in the Latin group after its reform ;
\TUR is in the Muslim group.

\begin{exemple}
  At turn 11, the Land technology for \FRA is at 20 and for the Latin group at
  15. \FRA manages to raise its technology to 22. Latin do not increase
  normally at turn 11. However, at the end of the administrative phase, the
  Latin marker is more than 6 boxes below the French one. Since \FRA is part
  of the Latin group, increase the Latin Land level to 16 (=22-6).
\end{exemple}

\aparag[Goals adjustment]
\label{chExpenses:Technology:Goals Adjustment}
At the end of the event and administrative phases, each goal that was reached
during the current phase is moved down until it is in a box preceding a
country or group technology counter or two boxes above another goal of the
same kind.

\begin{exemple}[Blocked by a marker]
  At turn, 11, suppose that we have the following positions of markers and
  goals: Land \SPA at 18, Naval \VEN at 19, Land \FRA at 20 and \TARQ at
  21. \FRA tries and manages to raise its Land technology at level 22 thus
  reaching \TARQ. Suppose that none of the other markers moved (eg \SPA missed
  its action and other countries are too far away). After all technological
  improvements, the situation is: Land \SPA at 18, Naval \VEN at 19, \TARQ at
  21 and Land \FRA at 22.

  Since \TARQ was reached, it must be moved down at the end of the
  administrative phase. Since it is a Land technology, it ignores the Naval
  marker of \VEN (Land technologies always ignore Naval
  technologies). However, it is stopped by the Land technology marker of \SPA
  and is thus at level 19.

  After adjustment, the situation is: Land \SPA at 18, Naval \VEN and \TARQ at
  19, Land \FRA at 22. Now that the French learnt the \emph{trace italienne}
  and its art of fortification, the Spanish will be quick to learn on the
  field and should manage to get \TARQ on next turn\ldots but there will still
  be five years of technical domination by the French armies.
\end{exemple}

\begin{exemple}[Blocked by another goal]
  Suppose that \TREN is at level 3 (blocked by the \ROTW group at level 2),
  \TARQ at level 10 and only the \RUS marker, at level 9, is between
  them. \RUS increases its Land technology and gets \TARQ. Thus the marker
  must be adjusted down. There are no Land technology marker to block it, but
  it is blocked by the goal marker of \TREN and stays at level 5. The
  ``empty'' level 4 is here to force \ROTW to access new technologies one at a
  time.
\end{exemple}

\aparag[Time adjustment] If a goal is available (the current turn is larger or
equal to the turn written on the goal counter) and not blocked by another goal
or marker (as above) of the same kind (Land or Naval), the goal is adjusted
down by one level.
% as many boxes as the difference between the turn on the marker and the
% current turn, until it reaches a technology counter.

\begin{exemple}
  At turn 21, after technology improvement, suppose that the best Land
  technology of countries and groups is 27. Since \TMUS is available at turn
  21 and at level 30, it decreases to level 29. If nobody manages to raise its
  technology at turn 22, then \TMUS will still be available and ahead of
  countries, so it will decreases further to 28.

  Then \TMUS will be at level 28, blocked by the Land technology of someone at
  level 27, thus it will stop its time adjustment (but will do goal adjustment
  as soon as somebody acquires it).
\end{exemple}

\begin{exemple}
  Suppose that the \ROTW Land technology is at level 2, the \TREN at level 3
  and \TARQ was adjusted down at level 5 per Goal adjustment. Since \TARQ is
  available, it should move down due to Time adjustment. However, it is still
  blocked by \TREN (one free box must stay between two technologies) and stays
  in place.
\end{exemple}

\begin{playtip}
  Time adjustment only occurs for goals that nobody possess. Indeed, goals
  reached by someone undergo Goal adjustment which directly move them down
  until they are blocked. And since they are blocked, they do not move by time
  adjustment any more. So, only the newest technology may undergo Time
  adjustment.
\end{playtip}

\begin{designnote}
  The order in which the different steps occur is very important and should be
  respected carefully. Especially:
  \begin{itemize}
  \item Cultural group adjustment occurs after scheduled progression of
    groups.
  \item Goals adjustment occurs only once per phase (Event or Administrative)
    and not each time someone reaches a goal. So, a country 3 levels behind a
    goal may not hope for someone else to reach the goal and make it drop
    before raising its technology on the same turn (however, this can be done
    in 2 turns).
  \end{itemize}

  To do it properly: at the end of the administrative phase (technology
  progression is rare during events), one player should do the Cultural groups
  progression followed by all the adjustments in order.
\end{designnote}


\subsubsection{Technology improvement}\label{chExpenses:Technology
  Improvement}
\aparag[Administrative operation]
% CHANGED for 2-tech actions per turn
\bparag To increase its technology, a \MAJ must do an operation of
\terme{Technology improvement}. Both technological operations (Naval and Land)
can be done each turn but only one may have an investment higher than a
\terme{Basic investment} (either Naval or Land).
\bparag[Resolution] The base column for Technology improvement is \MIL-9
(minimum -4).
\bparag Add 1 (or 2) bonus column if the country has a \MNU of level 1 (or 2)
of the adequate type (\RES{Metal} for Land, \RES{Instruments} for Naval), even
if the province is not controlled, pillaged, in revolt, \ldots Only one \MNU
counts.
\bparag Then add 1 or 3 columns for Investment as usual.
\bparag The following modifiers to the die roll are used:
\begin{modlist}
  % \item[+?] According to the investment (30, 50 or 100\ducats), a bonus of
  %   +0/+1/+3.
  % \item[+M] The \MIL value of the monarch.
  % \item[+1] for an appropriate \MNU of level 1 (not cumulative).
  % \item[+2] for a \MNU of level 2 (not cumulative, mutually exclusive with
  %   above).
\item[+?] If the \MAJ is late behind its group, +1 per level beyond the fifth
  (see below).
\item[-1] for \TUR, depending of its Military Reforms
  (see~\ref{chSpecific:Turkey:Army Tech}).
\item[+?] By event.
\end{modlist}
\bparag When a \MAJ is late behind its own group, it receives a bonus of +1
per level beyond the fifth counting from the marker of its group (Latin for
countries belonging to two groups).

\aparag[Result of the Technology operation]
The result depends on whether the next Technology goal is available or not.
% (or already attained by any power), on any previous turn or not:
\bparag Result ``F'' is always a failure: the money is spent and no level of
technology is gained.
\bparag If the next Technology goal is available, Results ``S'' or
``S\textetoile'' add 2 levels, Result ``\undemi''\ adds 1 level (no test under
\FTI);
\bparag If the next Technology goal is not available, Result ``S\textetoile''
adds 2 levels, Results ``S'' adds 1 level, Result ``\undemi'' is treated as
normal: roll 1d10, if less or equal than \FTI treat as ``S'', otherwise, treat
as ``F''.

\bparag Count \TTER and \TVGA as a ``next Technology'' (that can be available)
for every country even if only \SPA and \VEN (respectively) gains the
advantage of these technology goals.

% PB: used to be: Result : 0 box for a ``F'' or ``F\textetoile'' result, 1 box
% if the result was \undemi and a second roll is strictly superior to \MIL, 2
% boxes otherwise.

\begin{exemple}[Next goal unavailable]
  At turn 10, \FRA has a land technology of 19, a \FTI of 2, a \MNU of
  \RES{Metal} with two levels in \provinceChampagne and its king is
  \monarque{Francois I} with a \MIL of 9. \FRA tries to raise its Land
  technology. The base column is 9-\MIL=0 and \FRA has 2 bonus columns for its
  \MNU (notice that another metal \MNU would be useless). So, the player
  decides to only makes a small investment and roll in column 2. There is no
  DRM.

  \FRA rolls 5 and gets \undemi. The next technology is \TARQ, available on
  turn 11, hence it is not available now and \undemi\ is treated as usual. So,
  \FRA rolls another die, gets 2 which is smaller than its \FTI, so the result
  is treated as ``S''. Since the next technology is not available, this only
  gives 1 level and \FRA is now level 20 in Land technology.
\end{exemple}

\begin{exemple}[Next goal available]
  On turn 11, \FRA still wants to increase its technology and still makes a
  basic investment, thus rolling again in column 2 at +0. \FRA rolls 3 and get
  another \undemi. However, now the next technology (\TARQ) is available, so
  this gives 1 level to \FRA. \FRA reaches level 21. Since this is also the
  level of the \TARQ goal, \FRA gets a bonus level and reaches 22. At the end
  of the phase (after technology improvement of other countries), \TARQ will
  need to be adjusted.
\end{exemple}

\begin{exemple}[Lagging behind]
  Suppose that \RUS is level 14 for Naval technology, and the Orthodox group
  is level 22. Therefore, \RUS will receive a +3 (= (22-14) - 5) bonus to his
  die-roll for naval improvement.
\end{exemple}

\begin{exemple}[Lagging behind goals]
  On turn 10, \RUS has still not reached the \TREN technology. So, for \RUS
  the ``next goal'' is \TREN, and it is available. \RUS will use the
  resolution for ``next goal available'' even is the absolute next goal of
  every countries is \TARQ, which is not available. The resolution is relative
  to the situation of the country attempting the action.
\end{exemple}


\subsubsection{New Technology}
\aparag[Reaching a new technology] After reaching a new technology, a country
must pay a cost of conversion to this new technology.
\bparag This cost of conversion has to be paid immediately for the totality of
armies, fleets or detachments of the concerned country (except vassals) that
are currently on the map.
% If not, the progression of the concerned technological marker is canceled
% and the player's marker returned to its original position on the track.
\bparag The cost is 10\ducats per \ARMY\faceplus or \FLEET\faceplus counter,
5\ducats per \ARMY\facemoins or \FLEET\facemoins, 1\ducats per detachment (any
kind).
% \bparag The cost is doubled if there are two or more technological
% breakthroughs are obtained at the same turn.
\bparag Naval forces composed only of \NGD do not pay for technological
conversion.
% \bparag Exceptionally, if the player does not have enough treasure to pay
% for the conversion, he may immediately get the right to obtain an
% exceptional loan (National or International) but at double the interest rate
% compared to the figure obtained on \tableref{table:Loans}.
\bparag Minor countries never have to pay conversion costs, whatever their
diplomatic status.
\bparag Write this amount in \lignebudget{Other expenses}, even if the new
technology was obtained during the Events phase (\emph{i.e.} it is a scheduled
expense that must be done this turn).

\aparag Remark that from now on, the price of the various forces is changed
(according to the new technology).
\bparag Since all administrative actions (including logistic) must be payed
before any is resolved, troop raised the turn a new technology is reached are
recruited at the old cost (the new technology is not reached when planning the
construction of troops) but conversion cost must be payed for them. Follow the
turn order, as well as the order of the lines on the ERS closely.

\aparag[Technology advantage] Beyond the fact that countries with different
technologies do not use the same tables for combat, technology has the
following effects:
\bparag There is a +1 DRM to the die-roll for interception in land combat if
the Land technology counter of the intercepting country is 6 boxes or more in
advance related to the intercepted force.
\bparag There is a +1 DRM to the die-roll for wind-gauge in naval combat if
the Naval technology counter of the country is 6 boxes or more in advance
related to the opposing force.

\begin{playtip}
  ``6 boxes behind'' is the limit where things occur. If a major is ``6 boxes
  behind'' its group, it starts getting a bonus to technology
  improvement. Conversely, groups may be ``6 boxes behind'' majors but no more
  before being adjusted. Between majors, being ``6 boxes behind'' gives a
  combat bonus to opponent.
\end{playtip}


\subsubsection{Special technologies}
\aparag Two technologies are available only for one country each:
\bparag \TTER (Land) is available only for \HIS.
\bparag \TVGA (Naval) is available only for \VEN.
\bparag Check the special rules of these countries for details on the effect
of these technologies.

\aparag The markers for the special goals never block the progression of other
technology markers.
\bparag Neither do these goal prevent stacking of markers on their box.
\bparag For example, any country (including \HIS) may have its Land technology
higher than the \TTER counter, or even on it, at any point.

\aparag When resolving a technology improvement action, if a special goal is
available but not the next regular goal, a country use the resolution for
``next goal available'' even if it cannot benefit from the special goal.
\bparag Special goals undergo Time adjustment.
\bparag Special goals never undergo Goal adjustment.

\aparag The special goal is reached by a country if both:
\bparag it's level is equal or larger than the level of the goal;
\bparag and the goal is accessible (the current turn is equal or greater than
the one written one the counter).

\aparag A special goal marker may be removed as soon as it stop having effect,
that is:
\bparag It has been reached by the allowed country;
\bparag and the next regular goal is available.


\subsubsection{Former majors}
\aparag When a major powers become minor during the game (\POR, \VEN, \POL),
do the following with its technology markers (both for Land and Naval):
\bparag If the marker is below the marker for the Latin group, immediately
remove the marker of the former major.
\bparag Otherwise, keep it.

\aparag Every time the technology of the Latin group increase, also increase
the technologies (both Land and Naval) of each former major by 1 level (if it
is still on the track).

\aparag As soon as the technology of the Latin group is at the same level (or
above) than the one of a former major, immediately remove the marker of the
former major. This occurs because only the group undergoes Cultural Group
adjustment.

\aparag When a former major (\paysPortugal, \paysVenise, \paysPologne) is
involved in battle:
\bparag use the technological level of the marker or this major if it is still
on the track;
\bparag otherwise, use the technological level of the Latin group.


\subsubsection{Military Revolutions}\label{chExpenses:Military Revolutions}
\aparag Some events/leaders give the possibility of Military Revolution. Only
one of the two effects below may happen each turn for each country:
\bparag[Catching up] If the country does not already have the newest
technological goal that can be obtained, it gains it and its marker goes to
the box immediately after it.
\bparag[Breakthrough] If the country has the highest technology goal
available, and the next one will become available during the current period,
the country obtains this goal, and its marker is placed \textbf{two} boxes
ahead of the goal marker which is not moved ; the goal will not undergo Goals
adjustment nor Time adjustment before the turn written on the counter. The
country may not increase its technology further before the goal is regularly
available for everybody.

\aparag[Spreading breakthrough] If a country participate in a battle including
at least one of its \ARMY and one \ARMY of a country that has a technology
which is not yet available, it gains the right to reach this technology on
following turns.
\bparag The countries that do not directly benefit from the breakthrough must
still increase their technology as usual in order to reach the goal.
\bparag The goal is still considered as not available when resolving the
action.
\bparag The goal does not undergo Goal or Time adjustment before being
regularly available.
\bparag Countries that did not directly benefit from the breakthrough must
stop at the level immediately above the level of the goal (ie, the level
immediately below the level of the country benefiting from the breakthrough).
\bparag Neither the country benefiting from the revolution nor the ones
getting the technology some other way may improve their technology further
until the technology is available for everybody.

\aparag Existing Military Revolutions:
\bparag during \eventref{pIV:English Civil War}, due to \monarque{Cromwell}
(\TBAR (representing the New Model Army), or \TARQ in period \period{III});
\bparag during \eventref{pIV:TYW}, due to \monarque{Gustav Adolf} (usually
\TBAR, representing \emph{L\"{a}derkanonen} and other innovations);
\bparag due to \monarque{Friedrich II} (\TL, representing the Oblique order).

\begin{exemple}
  At turn 27, \ref{pIV:Bohemian Revolt} occurs and at turn 28, it degenerates
  into \ref{pIV:TYW}. At turn 29, \SUE enters the war and, as per event
  description, it benefits form a Military Revolution on each turn of the war.

  Suppose that the current technology of \SUE is only \TARQ. Since \SUE does
  not have the best technology available (\TMUS), it only has a Catching up
  and immediately gets \TMUS for free. Nothing more happens. \SUE still need
  to pay for conversion costs on this turn.

  On turn 30, \SUE does have the best technology available. The next one is
  \TBAR, available at turn 33, which is during the current period
  (\period{IV}). So \SUE benefits from a breakthrough. It immediately gets
  \TBAR and is placed 2 levels above (hence level 42). It may not move before
  \TBAR is available (turn 33). \SUE still need to pay for conversion costs on
  this turn.

  During turn 30, a battle takes pace at Brettenfeld involving 1\ARMY of \SUE
  (plus Saxons allies) against 1\ARMY of \AUS (plus Bavarian allies). The
  Austrians are severely beaten, but since they experimented the new tactics
  the hard way, they can now reach \TBAR (of course, not before turn 31 since
  technology does not increase during military phase).

  At turn 31, \SUE cannot gain the next technology as it is not available
  during this period, so even if it still benefits from a military revolution
  (one per turn during the event), it has no effect. \AUS can get \TBAR and
  succeed. It must stop at level 41 (one level above the goal) and may not
  move further before turn 33 (when the goal will be available).

  During turn 31, a stack composed of 1\ARMY of \AUS and 1\ARMY of \HIS fight
  against 1\ARMY of \HOL. \HOL learns the new tactics the hard way, but \HIS
  learn them from watching their allies. Thus, at turn 32 both \HOL and \HIS
  will be allowed to reached \TBAR (and stop at level 41). At turn 33,
  everybody may reach \TBAR and countries that already have it may move
  further.
\end{exemple}



\subsection{Competitions}\label{chExpenses:Competition}

\aparag The \terme{competition mechanism} is the way to settle all matters of
conflicts of the administrative phase. This is used to reduce other people's
trading fleets, settle the cases where two \TP or \COL are installed at the
same time in the same province or where a single resource may be exploited by
several outposts, etc. There are two kinds of competition: normal competition
(one country pays for an action that will target another country), and
automatic competition (some conditions are not respected in a specific zone,
and there is competition until the conditions are respected).

\aparag[Sequence] Normal competition happens during the administrative phase,
at the same time as other administrative actions. Automatic competition
happens at the end of the administrative phase, after all other administrative
actions have been resolved, to solve conflicting situations. Automatic
competition for the exploitation of exotic resources may also happen after the
Peace phase (since peace may change owners of \TP or \COL or destroy \TP).

\subsubsection{Normal competitions}
\aparag[Target] The administrative action of normal competition targets an
item of another country (major or minor).
\bparag Competition may target: a commercial fleet, a \TP, or a \COL
exploiting resources.
% PB: remove competitions on MNU: never seen players using it...  , a \MNU.
% \bparag A \MNU can be targeted only if the player or one of its vassals has
% a common frontier with the target or a vassal of the target, or is at one or
% two sea zones of distance, and if the player has a \MNU of the same type of
% level 2.
\bparag A \TP or \COL can only be targeted if the country has a \TP or a \COL
in the same \Area.
% Jym: not in accordance with the "limited access to trade"
% rule. Reformulated.
% \bparag A commercial fleet can only be targeted if the player has a
% commercial fleet in the adequate \STZ/\CTZ.
\bparag A commercial fleet in the \ROTW may only be targeted if it is in a
legal \STZ according to~\ref{chExpenses:Limited Access}.

\aparag[Reaction] The target country, may react by paying a \terme{Basic
  investment}, \terme{Medium investment} or a \terme{Strong investment}.
\bparag This does not count towards its own limit of actions. There is no
limit on the number of reactions a country may do each turn.
\bparag The player must be informed of all the details of the action (target
and investment) before choosing whether to react. The player may wait to know
all the competitions that are done against him before deciding whether to
react for each of them. That is, Administrative actions (including
competitions) should be all planned, then announced publicly before deciding
to react; and only after reactions are decided can actions be resolved.
\bparag Minor countries automatically react with a medium investment.
\bparag Money expanded for reactions is recorded in
\lignebudget{Administrative reactions}.
% The bonus is reversed (+1 column becomes -1, etc.)
% \bparag If the player choose not to react at all, then its \FTI (and \DTI)
% will not be used for computing the column.
% \bparag Each country is entitled to a specific number of normal competitions
% per turn (depending on the period and some strategic choices of the
% player). Automatic competitions do not count in this limit.
\aparag[Column] The competition actions are resolved in
\tableref{table:Administrative Actions}. The column is: (\FTI country) +
(Investment country) - (\FTI target) - (Investment target).
\bparag Investment adds (or subtracts) 0, 1, 3 columns for Basic, Medium,
Strong (as usual).
\bparag If targeting a \TradeFLEET in a \CTZ, the owner of the \CTZ adds its
\DTI to his \FTI (as bonus if it is the acting country, malus if this is the
target).
\bparag If the target chooses not to react at all (no investment), then do not
subtracts its \FTI nor its \DTI for finding the column.

% \bparag[Commercial fleet] (\FTI country)+Investment-(\FTI target)
% \bparag In a \CTZ, the player adds its \DTI if it is its own \CTZ, the
% target subtracts its own \DTI if it is its own \CTZ.
% % \bparag[Manufacture] (\FTI player)+Investment-(\FTI target)-(\DTI target).
% \bparag[Exotic resource] (\FTI country)+Investment-(\FTI target)
% \bparag[Trading posts] (\FTI country)+Investment-(\FTI target)

\aparag[Modifiers] The die-roll is modified as follows:
% with the following parameters according to the target:
\bparag[Commercial fleet] -1 if there is at least one commercial fleet of a
third party in the target \STZ or \CTZ.
\bparag[\TP or \COL] -1 if at least one third party \TP is in the \Area.
% \bparag[Resource exploitation] -1 if at least one third party \TP or \COL is
% in the \Area (whether it exploits resources or not).
\bparag[Wars] -1 if there were battles (including fighting privateers or
piracy) in the \Area or the \STZ (or \CTZ) in the previous turn.
% \bparag[Events] Some events may grant a modifier to the die-roll for
% competition.

\aparag[Results]
\bparag A result of ``S'' decreases the level of the target by 1 (Exception:
\COL, see below). If this is the last level, remove the counter in the
province or sea zone. If the \TradeFLEET or \TP reaches level 3, turn the
counter on its \Facemoins side.
\bparag A result of ``F'' is a failure: nothing happens but the money is lost.
\bparag A result of ``\undemi'' is treated as normal: roll 1d10, if less or
equal than \FTI (use special \FTI if allowed), treat as ``S'', otherwise,
treat as ``F''.

\aparag[Competition on \COL] If a competition targets a \COL and succeed, the
\COL does not loss a level.
\bparag However, it loss, for this turn only, the possibility to exploit one
of its resources.
\bparag The resource is thus freed and can be exploited by other
establishments in the \Area.

\aparag[freeing resources] If a \COL is victim of competition, or a \TP is
victim of competition and does not have enough level to exploit all its
resources anymore, it must free one resource (or more) for other
establishments to exploit.
\bparag The freed resource is chosen by the owner of the establishment. Minor
countries always free the resources that currently cost less, in case of
equality the resource whose maximum price is the smallest (at random in case
of further equality).

\begin{exemple}
  At turn 2, \leader{Da Gama}, in a brief war against \paysGujerat, manages to
  seize the \TP in \province{Malabar S} and thus exploits the \RES{Spice} and
  the \RES{PO} that it exploits. During turn 3, \POR wants to seize the other
  \RES{Spice} in the \Area and does competition on the \TP of \paysGujerat in
  \provinceKolikot (since \POR has a \TP in the same \Area, it may do
  competition). \POR chooses to do two competitions (its limit for the period)
  on the \TP, each with Medium investment (60\ducats total).

  The minor country automatically reacts with Medium Investment and a \FTI of
  2 (as explained in ~\ref{chExpenses:Administration Minor}). Thus, the column
  is 5 (\FTI of \POR) + 1 (Investment of \POR) - 2 (\FTI of \paysGujerat) - 1
  (Investment of \paysGujerat) = 3. There is a \bonus{-1} DRM as battles (in
  this case sieges) occured in the \Area on turn 2 for the capture of the \TP.

  For the first action, \POR rolls 7 for a result of 6. It's a success and the
  \TP of \paysGujerat losses a level. For the second action, \POR rolls 3 for
  a result of 2, it's a \undemi. A second roll gives 5, less than the \FTI of
  \POR (5), so it's also a success and the \TP losses a second level.

  Since the \TP of \paysGujerat is now of level 1, it can not exploit 2
  \RES{Spice} anymore (it can only exploit 1). So it must free one of the two
  exploited resource (in this case, the choice has no importance). If \POR
  also managed to raise the level of its newly conquered \TP, it can
  immediately exploit this resource (otherwise, since there are no other
  establishment, the resource is not exploited immediately).
\end{exemple}


\subsubsection{Automatic competitions}
\label{chExpenses:Automatic Competition}
\aparag[Conditions] Automatic competition occurs when abnormal situations
arise after resolving administrative actions. Namely:
\bparag A \STZ or \CTZ contains several \TradeFLEET\faceplus.
\bparag A \STZ or \CTZ contains one level 6 \TradeFLEET and one or more other
\TradeFLEET.
\bparag Players disagree on the repartition of exploited resources in a given
\Area.
\bparag There are two \COL or two \TP in the same province (note that this may
only happen if both were created this turn).

\aparag[Mechanism] Automatic competitions use the following mechanism: every
country involved in an automatic competition rolls one die and checks the
result in \tableref{table:Administrative Actions}.
\bparag The column is usually (\FTI country) - (Highest \FTI of
opponent). There is no investment.
\bparag Treat \undemi as usual by a roll under \FTI.
\bparag A ``F'' implies the loss of one level for the field of competition.
\bparag The procedure is reiterated again until the conditions of automatic
competition do not apply any more.
\bparag The automatic competitions do not count in the limit of competitions
for the turn.
\bparag Resolve automatic competition in each \STZ, \CTZ or \Area
separately. That is, if a country competes against two different opponents in
two different \STZ, each one will use a different \FTI of enemy.

\aparag[Fleets]\label{chExpenses:Trade Competition Mandatory}
% When a commercial fleet reaches the level 6, there is a mandatory
% competition between this fleet and all other fleets in the \STZ or \CTZ
% until there remains only one level 6 fleet and no other fleets.
% \bparag There is also automatic competition between all the \Faceplus
% commercial fleets until only one remains \Faceplus.
All \TradeFLEET\faceplus in a given \STZ or \CTZ must compete between
them. Simultaneously, each level 6 \TradeFLEET competes again all other
\TradeFLEET in its \STZ or \CTZ.

% Jym, not sure about this one... But if the TF is considered as still here,
% that's the only solution... Otherwise, a TF of level 0 is removed but then
% the counter could be reused elsewhere.

% Since the TF was adjusted at the beginning of the phase, this is the case
% only if it already suffer normal competition this turn. So it should be OK.
\bparag Note that \TradeFLEET of \emph{current} level 0 do exist and thus
compete against \TradeFLEET of level 6 (and loose \emph{maximum} level in case
of failure).
\bparag Each country use the highest \FTI of opponents involved in competition
against it.
\bparag In its own \CTZ, a country adds its \DTI to its \FTI (both as a bonus
for itself and a malus for opponents), before finding the highest \FTI.
\bparag No modifiers apply.
\bparag Each country that does not obtain ``S'' (including after \undemi) loss
one level of \TradeFLEET.
\bparag Repeat the procedure until the conditions for competition do not exist
anymore.

\begin{exemple}
  Suppose, that during period \period{V}, the \ctz{France} contains a
  \TradeFLEET of level 6 of \HOL (\FTI 5), a \TradeFLEET of level 4 of \FRA
  (\DTI 2, \FTI 4) and a \TradeFLEET of level 2 of \ANG (\FTI 5). Since there
  is a \TradeFLEET of level 6, it must compete against all other.

  \FRA being in its \CTZ adds its \DTI to its \FTI for a total of 6. Each
  other only has 5. Since \ANG does not compete against \FRA (both their
  \TradeFLEET could co-exist), it use the \FTI of \HOL as opponent. So, \FRA
  rolls on columns 1 (6-5), \ANG in column 0 (5-5) and \HOL in column -1
  (5-6). \FRA rolls 7, it is a S; \ANG rolls 4, \undemi, a second rolls give
  7, thus a F and the level of the fleet decrease; and \HOL rolls 3, a F.

  So, after 1 round of competition, there is a \TradeFLEET of level 5 of \HOL,
  a \TradeFLEET of level 1 of \ANG and a \TradeFLEET of level 4 of \FRA. The
  English \TradeFLEET is no more in danger but there are still several
  \TradeFLEET\Faceplus, so competition goes on between \HOL (column -1) and
  \FRA (column 1) until one of them goes down to level 3.
\end{exemple}

\aparag[Establishments in a province] There is an automatic competition when
several countries happen to create a \COL or \TP at the same time in the same
province.
\bparag There is no automatic competition if a country creates a \COL and one
creates a \TP in the same province at the same time: the \COL remains.
\bparag Use the highest \FTI of opponents involved in the competition.
\bparag Any country that does not roll ``S'' loss one level to its
establishment.
\bparag The competition ends when only one establishment remains in the
province.

\aparag[Resource exploitation] When players disagree on the exploitation of
free resources in an \Area, automatic competition occurs.
\bparag Resources that were exploited on the previous turn are not subject to
this competition unless they are first freed somehow (usually, by regular
competition).
\bparag New resources appearing in a province are subject to automatic
competition if several countries have enough levels to exploit them and the
players disagree.
\bparag When several establishments gain level in a given \Area, there may be
more levels than necessarily to exploit all the remaining resources, in which
case automatic competition occurs if the players disagree.
\bparag A player may, as a diplomatic announcement, free some (or all)
resource exploited by its establishment in some \Area. Note that this happens
before the Income computation, thus the freed resources will not generate
income on this turn and can be exploited by someone else only at the end of
the Administrative phase.

\aparag[\Area with multiple resources] In \Area producing several kind of
resources, competition is done for each resource in an order chosen by the
involved country with the better initiative.

\aparag[Order of competitions] Each country involved in a competition, in
order of initiative, designate one \Area where competition is resolved.
\bparag Players may renounce to their rights of exploitation at any time
during the process. Typically, after a successful automatic competition in one
\Area, a player may magnanimously decide to leave the resources to someone
else in another \Area.
\bparag Agreements between players may be done globally for several \Area. Any
agreement announced publicly must be respected.
\bparag If a country chooses to stop competition before the end in a given
\Area (to avoid loosing levels), it may not exploit any of the remaining
non-attributed resources in this \Area during this turn (in case every body
else loses).

% \aparag[Resource exploitation] When there are free resources (either because
% of an increase of the exploited resources due to events or to the number of
% the turn, or because the number of level of establishments dropped and not
% everybody can maintain its former exploitation, or for diplomatic reasons),
% there might be automatic competition.
% \bparag Freeing resources can be done in the diplomatic phase, but cannot
% give it to another country (the resource may be grabbed by a third party).
% \bparag The automatic competition takes place in a specific \Area, for a
% specific resource. It goes on until all demands can be satisfied, area by
% area. The order is first case below, then second case. In case of
% disagreement on the order of the regions, the first player in order of
% initiative designates one \Area, the second another, etc.
% \bparag[First case] A new level is available to an establishment, either
% because its level increased, or because it just freed a resource. If there
% are not enough free resources to satisfy it, then there is a competition
% between this establishment and all the other establishments that exploit the
% same kind of resource that it wants, until there are enough levels to
% satisfy him.
% \bparag It is not possible to free one resource and ask the same resource.
% \bparag[Second case] A new resource appears (by event, because it was freed,
% or else), and several establishments with free levels want to exploit
% it. There is a competition between all the establishments that want the new
% resource, until all players are satisfied.
% \bparag An establishment that had to free a resource cannot be active.
% \bparag These two cases are the only ones of automatic competition for the
% exploitation of resources. Normal competition may take place, however.
% \bparag It is possible to leave the first case of automatic competition at
% any time by ceding enough resources (or all). All levels of the
% establishment count, not only the one that helped initiate the competition.
% \bparag It is possible to leave the second case of automatic competition at
% any time. If all remaining players agree, some resource can be assigned to
% the leaving player. However, the player will not be able to take more
% resources at the end of the phase.

\bparag \TP lose levels permanently, \COL do not lose levels but only rights
of exploitation for the turn. A \TP reaching level 0 is destroyed.
\bparag Resources are not freed in case of \terme{Speculation}
(\ruleref{chExpenses:Speculation}).% If not doing \terme{Speculation}, all
% countries do exploit their resources.
% \bparag Establishments that have levels that do not exploit a resource can
% only participate in the second case and do not fit the conditions of the
% first case.% can get resources for exploitation only in the second
% % case.
% It is never mandatory to try to exploit as many resources as possible.

\begin{exemple}\HOL, \ENG and \FRA are competing for the exploitation of
  two \RES{Spices} resources. \HOL and \FRA both have enough levels to exploit
  one and \ANG can exploit both. \ANG and \HOL have \TP while \FRA has a \COL.

  After one roll, nobody lost and the situation is unchanged. \ANG does not
  want to risk its levels and says it agree to leave the competition if the
  other leave 1 \RES{Spice} to \ANG. The other are reluctant but finally
  convinced after seeing the size of the English navy. So \ANG takes 1
  \RES{Spice} and loses the possibility to exploit the second one this
  turn. The competition goes on between \FRA and \HOL.

  At the next roll, both \ANG and \FRA are unsuccessful and thus lose the
  possibility to exploit the new \RES{Spice} this turn. Since \HOL had a \TP
  in this competition, it losses a level. The French \COL does not loss any
  level but is simply prohibited from exploiting one resource this turn.

  \ANG is not allowed to take the remaining free resource, even if it has the
  possibility to do so, because it has already renounced its rights this
  turn. This last \RES{Spice} will be exploitable next turn, both by \ANG
  without creating new Establishment (as one of its level still remains with
  no exploitation), by \FRA without creating new establishment (as the
  ``extra'' level of \COL will regain its rights for exploitation on the next
  turn) and by anybody successfully managing to create new levels.
\end{exemple}

\section{Resolution of actions}\label{chExpenses:Resolution of actions}
\aparag Once all players have chosen their actions, and written them down with
all relevant column, DRM, and such, they can be resolved.
\bparag In practise, conflicts are few and players may start resolving their
actions as soon as they have finish planning them.

\begin{playtip}
  Players should ``pair'' to solve their actions. One of them state his
  actions, one by one, with relevant column and DRM, and roll the die while
  the other check the table to see whether it's a success or a failure.
\end{playtip}

\aparag The procedure to resolve the action is described together with the
action itself, check the relevant Sections.

\section{Administration for minor countries}

\label{chExpenses:Administration Minor}



\subsection{Trade fleet}\label{chExpenses:Minor Commercial Fleets}

\begin{note}
  The minor's trading fleets are characterised by three different levels,
  their current level, their maximum level, and their reference level. If the
  first two resemble those of the major countries, the third one is the
  threshold that gives the minor countries \TFI actions.
\end{note}

\aparag[Commercial fleet levels] Some minor countries have commercial
fleets. Their \terme{reference level} is the one of 1492, sometimes changed by
events. Reference level is not reduced by competitions and piracy and may only
change by events explicitly stating the change.
\bparag[Exception] \paysPortugal and \paysVenise have their reference levels
fixed by \eventref{pIII:Portuguese Disaster} for \paysPortugal and at the time
it becomes minor (usually 1615) for \paysVenise.
% \bparag Before being set for \paysPortugal, consider there is no limitation
% (the \terme{reference level} is 6 in all \STZ and \CTZ).
% \bparag Once set, the \terme{reference level} never changes.

\aparag There also exists both a \terme{current level} and \terme{maximum
  level} for the fleet, as for the major countries, used to distinguish
between losses by piracy and losses by competition.
\bparag Competitions and \TFI affect both the \terme{maximum level} and the
\terme{current level}.
\bparag Piracy (either privateers or pirates) diminish the \terme{current
  level}, but not the \terme{maximum level}.
\bparag \terme{Current level} increase automatically at the beginning of the
administrative phase if it is lower than the \terme{maximum level}.
\bparag All in all, \terme{maximum} and \terme{current} levels work for minor
countries exactly in the same way as they work for majors.

% This is prone to errors, as well as strange cases with the FTI of some being
% "at least 3" but taking the FTI of patron if better. Seems to be of little
% influence, hence remove for clarification.

% \aparag[Competition] If the minor country is a vassal of a player, it is
% considered as belonging to the player for all actions of Competition and is
% included in both the player's costs and limits. It uses the player's \FTI.
% \bparag If the minor country is not a vassal and is at peace, it is
% considered to have a \FTI of 2 with a free medium investment when involved
% in a competition.
% \bparag[Exception] \pays{portugal}, \pays{venise}, \pays{genes},
% \pays{suede}, \paysDanemark and \pays{hollande} use a \FTI of at least 3,
% whatever their status.
% \bparag These two values are changed to respectively 3 and 4 after 1615
% (turn 26, period IV).

\aparag[Commercial indexes] \paysPortugal, \paysVenise, \paysGenes,
\paysSuede, \paysDanemark and \paysHollande have a \FTI of 3 in periods
\period{I}-\period{III} and 4 in periods \period{IV}-\period{VII}.
\bparag Other minors have a \FTI of 2 in periods \period{I}-\period{III} and 3
in periods \period{IV}-\period{VII}.
\bparag \paysHollande has a \DTI of 4.
% removing almost useless minor HOL in pIII+ before \ref{pIII:Dutch Revolt}
% and 5 afterwards.
Other minors have a \DTI equal to their \FTI.



\subsection{Colonisation by minor countries}

\aparag Former major countries (\paysPortugal and \paysVenise) keep their
colonial establishments when becoming minors. They have specific actions to
increase them, depending on the events that already occured.
\bparag Some other minor countries did explore and colonise during the game
period. They do not have actions and this is abstractly represented by
\ref{eco:Rush:Minor colony}.



\subsection{Administrative actions of minors}

\aparag Minor countries have administrative actions (\TFI, \COL, \TP and
concurrence).
\bparag Only \paysPortugal has \COL, \TP and concurrence actions. \paysVenise
also has a specific limit of \TFI.
\bparag These actions are free (payed by the minor) and always made at medium
investment.
\bparag These actions are never mandatory.
\bparag They are planned (and resolved) by the diplomatic patron of the
country. If neutral, use the first major not at war against this minor in the
preference order for controlling it.

\aparag[Portugal]
\bparag Before \ref{pIII:Portuguese Disaster}, \paysPortugal has each turn one
\COL, one \TP and one \TFI action.
\bparag Between \ref{pIII:Portuguese Disaster} and annexation by \HIS,
\paysPortugal has each turn one \COL or one \TP action (choice every turn) and
one \TFI.
\bparag While annexed by \HIS, \paysPortugal has no actions but \HIS may use
its own for the minor.
\bparag After \eventref{pVI:Treaty Methuen}, if it is no more annexed by \HIS,
\paysPortugal has each turn one \COL or one \TP or one \TFI action.
\bparag By exception, in period \period{III}, these actions are planned and
resolved by \HIS whatever the diplomatic status of \paysPortugal (and even if
\HIS is currently at war against \paysPortugal).

\begin{designnote}
  This latest exception prevents other players form poorly playing a country
  that will soon be part of \HIS for a long time.
\end{designnote}

% Let's be serious...
% \aparag[Holland] \pays{hollande} has 1 \COL and 1 \TP operation as a minor
% country before \eventref{pIII:Dutch Revolt}. Then, two \TFI, one \COL, one
% \TP and two concurrence actions between periods III and V and only one \TFI,
% one concurrence and one \COL or \TP action in periods VI and VII.
% \bparag If \pays{hollande} is neutral, \SUE has the control of those
% actions.
% \aparag No other major keeps diplomatic actions when becoming minor.

\aparag[Venice]
\bparag During periods \period{IV} and \period{V}, \paysVenise has 1 \TFI each
turn.
\bparag Afterwards, it is treated as other minor countries.

\aparag[Vassals]
\bparag Vassals have no actions. However, their diplomatic patron may use its
own \TFI for the benefits of a vassal.
\bparag In this case, the patron has to pay for the action.

\aparag[Other cases]
\bparag Non-vassal minors have 1 \TFI each turn if and only if there is at
least 1 \CTZ/\STZ where their \terme{maximum level} is strictly less than
their \terme{reference level}.
\bparag If performed, this action must target one \CTZ/\STZ where the
\terme{maximum level} is strictly less than the \terme{reference level}.

\subsection{Logistic of minors}
\aparag Like major countries, minors have to maintain and recruit
troops. See~\ref{chLogistic:Maintenance of minors}
and~\ref{chLogistic:Recruitment of minors} for details.

% Local Variables:
% fill-column: 78
% coding: utf-8-unix
% mode-require-final-newline: t
% mode: flyspell
% ispell-local-dictionary: "british"
% End:

% LocalWords: Carrack Tercios Galleass Boyars
