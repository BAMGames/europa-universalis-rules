% -*- mode: LaTeX; -*-

\section{Period II}\label{events:pII}



\subsection*{Event Table of Period II}

\begin{eventstable}[Period II events table]
  \tabcolsep=5pt\centering%
  \begin{tabular}{|l|*{6}{c}|l|}
    \hline
    1\up{st}\textarrow& 1-3 & 4-5 & 6 & 7 & 8 & 9 & 10 \\ \hline
    1 & 2  & 1  & 10  & 13  & 1  & 1   & \textbullet~1--2:   \\
    2 & 3  & R2 & 11  & R14 & 2  & R8  & +1 then\\
    3 & R4 & 3  & 12  & R15 & R3 & R2  & \nameref{events:pI} \\
    4 & 5  & 4  & 15  & 16  & R4 & 11  & \textbullet~3--10:  \\
    5 & 6  & 8  & 16  & 17  & R5 & 12  & \nameref{events:pI} \\
    6 & 7  & 9  & 17  & R18 & 7  & 13  & \\
    7 & 8  & 11 & R18 & R21 & R8 & R19 & \\
    8 & 10 & 12 & R8  & 1   & R9 & R21 & \\
    9 & R9 & 21 & R21 & 19  & 14 & R20 & \\ \hline
    10 & \multicolumn{7}{l|}{\nameref{events:pIII}} \\ \hline
  \end{tabular}
\end{eventstable}
\begin{eventstablespec}[Habsburg Hungary]
  The first \RD event beginning with turn 8 activates \ref{pI:Habsburg
    Hungary} instead of its normal effect if either \ref{pI:Hungarian
    Alliance} or \dynasticaction{C}{1} has been played, and \ref{pI:Hungarian
    Freedom} has not.
\end{eventstablespec}

\eventssummary{%
  pII:Act Supremacy|,%
  pII:War Scotland|,%
  pII:Emperor Election|O{pI:Emperor Election},%
  pII:Habsburg Dynastic Commitments|E/E/E/E,%
  pII:War Italy|,%
  pII:End Kalmar|,%
  pII:War Persia Turkey|E/E,%
  pII:Algeria Vassalisation|,%
  pII:Alignment of Barbaresques|,%
  pII:War Poland Turkey|,%
  pII:Reformation|O{pI:Reformation}/O{pI:Reformation2}/O{pI:Reformation3},%
  pII:Schmalkaldic League|,%
  pII:War Indian Ocean|,%
} \eventssummary{%
  pII:Portuguese Colonial Dynamism|E/E/E,%
  pII:Spanish Colonial Dynamism|E/E/E,%
  pII:Union Lublin|,%
  pII:Conquest Khanates|,%
  pII:Superiority over Khanates|,%
  pII:War Russia Poland|,%
  pII:War Russia Turkey|,%
  pII:Forward Baltic Sea|,%
  pII:American Resistance|E/E,%
  pII:Chinese Expansion|,%
  pII:Apparition Mughal Empire|E/E,%
  O|,%
  pII:Mughal Expansions|T{many times},%
  pII:Crusade|T{many times},%
}

\newpage\startevents



\event{pII:Act Supremacy}{II-1 (1)}{Act of Supremacy}{1}{Risto}

\history{1534, 1539}
% \condition{Takes only place if the event \ref{pI:Reformation2} has already
% occurred. Otherwise re-roll and do not mark off.}

\condition{Takes place when rolled for, or when \monarque{Henry VIII} dies.}

\phevnt
\aparag \ENG has to choose its Heir, in accordance with its current religion.
\bparag[Catholic/No Reform] Mary I Tudor, or Edward VI
\bparag[\CATHCR] Mary I Tudor, or Edward VI
\bparag[\CATHCO] Edward VI, Jane Grey
\bparag[Protestantism] Jane Grey

\aparag[Marie I Tudor] \ENG is forced to be \CATHCR.  If it has changed, both
general and particular effects of \ref{pI:Reformation2} are applied
immediately.
\bparag It has a mandatory Dynastic (Defensive and Offensive) Alliance with
\SPA for 3 turns.  If at war, \SPA and \ENG make an immediate white peace.
\bparag Roll for 2 \REVOLT in \ENG in the table, using 1d10-2 for
localisation.

\aparag[Lady Jane Grey] \ENG is forced to be Protestant.  If it has changed,
both general and particular effects of \ref{pI:Reformation2} are applied
immediately.
\bparag Alliance between \ENG and \SPA are forbidden for 3 turns.
\bparag All \CATHCR \MAJ and also \POR and \VEN receive a temporary \CB
against \ENG.
\bparag Roll for 2 \REVOLT in \ENG in the table, using 1d10+3 for
localisation.

\aparag[Edward VI] \ENG must choose freely its Religious Attitude.  If it has
changed, both general and particular effects of \ref{pI:Reformation2} are
applied immediately.  If it is now \CATHCO, Edward VI (and truly also Mary I)
will reign at most 2 turns. (Note: determine values at random, Edward VI may
also die, but its successor will last only the second turn).
\bparag At the beginning of the second turn, roll for 2 \REVOLT in \ENG in the
table, using 1d10-2 for localisation.
\aparag[After Edward VI: Elizabeth or Mary] At the beginning of the third
turn, \ENG may opt immediately to choose between two possibilities:
\bparag[Mary Stuart] \ENG chooses to remain \CATHCO, in which case none of the
effects described underneath are applied. Instead, \ENG loses {\bf 1} in \STAB
(for having to face humiliation from the Pope).
\bparag["Elizabethan Settlement"] \ENG becomes \PROTANG, that is Protestant as
defined in \ref{pI:Reformation2}.  Both general and particular effects of the
event are applied immediately.  The only difference between Anglicanism and
Protestantism is relative to the Religious and Civil Wars of \ENG.
\bparag The Monarch of \ENG is now \monarque{Elisabeth I}.
\bparag \ENG receives 250\ducats in its Treasury.
\bparag All \CATHCR \MAJ and also \POR and \VEN receive a temporary \CB
against \ENG.

% \aparag[Protestantism:]
%% \bparag All catholic/Counter-Reformation major powers and also \POR receive
%% a temporary \CB against \ENG.
% \bparag Event \ref{pIV:English Civil War} is immediately activated.
% \aparag[\CATHCR]
% \bparag \ENG may become Anglican, that is Protestant as defined in the event
% \ref{pI:Reformation2}. Both general and particular effects of the event are
% applied immediately. The only difference between Anglicanism and
% Protestantism is relative to the Religious and Civil Wars of \ENG.
% \bparag \ENG receives 250\ducats in its Treasury.
% \bparag All \CATHCR major powers and also \POR and \VEN receive a temporary
% \CB against \ENG.
% \bparag \ENG may choose to remain Conciliatory, in which case none of those
% effects are applied. Instead, \ENG loses {\bf 1} in \STAB (for having to
% face humiliation from the Pope).
% \aparag[\CATHCR]
%% \bparag All protestant major powers receive a temporary \CB against \ENG.
% \bparag Event \ref{pIV:English Civil War} is immediately activated.
%% \bparag If \SPA is \CATHCR and the king wins \ref{pIV:English Civil War},
%% \ENG will have the same turn/period limits as if \CATHCO.



\event{pII:War Scotland}{II-1 (2)}{War with Scotland}{1}{PBNew}

\history{1542}

\condition{}
\aparag Occurs only if \paysecosse is at present inactive. Otherwise re-roll.
\aparag \ENG can refuse this event (mark as played) by losing {\bf 2} \STAB
and 20 \VP. It also loses the control of \paysecosse and can then make no
diplomacy on it until the end of period.
\bparag If \ANG has chosen the "Mary Stuart" option in \ref{pII:Act
  Supremacy}, the refusal of the war costs only {\bf 1} \STAB (no \VP, no
diplomatic consequences).

\phevnt
\aparag \paysecosse declares war against \ENG, which loses the control of
\paysecosse.
\aparag Allies can be called for this war as per normal rules.
\aparag Control of \paysecosse is offered to the first country in the list:
\bparag Any current enemy of \ENG (follow the normal preferences to decide
which).
\bparag The current controller of \paysecosse or, failing that, another power,
according to the usual rules.

\phadm
\aparag For the duration of the event, \paysecosse receives reinforcements in
offensive attitude.



\event{pII:Emperor Election}{II-2 (1)}{Election of the \HRE
  Emperor}{1}{RistoMod}

\condition{Same event as \ref{pI:Emperor Election}.}
\aparag If \shortref{pI:Emperor Election} has not occurred, play this event.
\aparag If \shortref{pI:Emperor Election} has already occurred, play the
following event.



\event{pII:Habsburg Dynastic Commitments}{II-2 (2)}{Habsburg Dynastic
  Commitments}{4}{PB}

\phevnt
\aparag \SPA \textbf{must} immediately play one dynastic action of its choice,
without test nor cost. Annexation of a province of the North-East is a valid
choice. If there is no such actions possible, treat as no event and mark off.



\event{pII:War Italy}{II-3}{War in Italy}{1}{Ristomod}

\history{1521-1526 / 1526/1530 / 1536-1539 / 1542-1544 / 1552-1559}

\condition{This event continues \ref{pI:War Italy Napoli} and \ref{pI:War
    Italy Milano}.}
\aparag If either \ref{pI:War Italy Napoli} or \ref{pI:War Italy Milano} is in
effect, re-roll without marking.
\aparag If \ref{pI:War Italy Napoli} was not played, play it, mark off and do
not apply the remaining of the present event.
\aparag If \FRA owns \provinceLombardia, mark off the event which is
considered played with only one effect: \HAB after \ref{pI:Habsburg Milano} or
\SPA after \ref{pI:Spanish Milano} has a free \CB against \FRA at this turn.
\aparag The event may happen more than once. If a \ref{pII:War Italy} is
happening when another event is rolled for, the second one is marked off and
treated as a \RD.

\phevnt
\aparag \FRA has a Mandatory \CB against the owner of \provinceLombardia. This
\CB has to be used this turn or the next, at the phase of Declaration of
War. If \FRA is Counter-Reformation after \ref{pI:Reformation2}, the \CB is
free.
\aparag If \FRA is already at war against this country, the war has to become
the war linked to this event at this turn or the following (the choice is made
by \FRA during the Declarations of War) and that fulfils the Mandatory \CB.

\phdipl
\aparag[Refusing the event]
\bparag At the very beginning of the Declarations Phase, \FRA or the owner of
\provinceLombardia may refuse the event.
\bparag If \FRA refuses the event, it loses {\bf 2} \STAB (or none if the
current period is III or after) and the rest of the event is ignored.
\bparag If the owner of \provinceLombardia refuses the event, it loses {\bf 3}
\STAB and gives \provinceLombardia to \FRA. Then the rest of the event is
ignored. If this province is owned by the \HAB, \SPA may refuse the event (and
lose the \STAB).
\aparag[Milan as a Minor country]
If \provinceLombardia is owned by the Minor country \paysMilan, \HAB have a
free \CB in reaction to a Declaration of War of \FRA against this
country. \paysMilan is moved up to \EG on the diplomacy track of \HAB if it
was not already on a higher position.
\aparag[Diplomatic effects of the wars]
\FRA has a bonus of {\bf +2} for its diplomacy on \paysToscane and {\bf -1}
for \paysPapaute and \paysParme during the event.
\aparag[The Serenissima in the Wars in Italy]
This rule is applied only if \VEN has announced a \terme{Policy of Italian
  dominance}.
\bparag \VEN has a \CB against \FRA and/or the owner of \provinceLombardia, as
long as the war is not finished.
\bparag During this war, \VEN may make limited intervention at the side of any
involved alliance each turn. Such limited intervention can begin at any turn
(not only the first) and \VEN can change side between turns.  \VEN may force
any Italian \MIN in limited intervention for the enemy alliance, to be fully
involved in the war.
\bparag Conversely, \FRA and \HAB both have a (normal) \CB against \VEN, to be
used at any turn of the war.
\aparag[Swiss Mercenaries]
If \paysMilan is a vassal or a possession of \HAB (according to
\ref{pI:Habsburg Milano}), \HAB gain \paysSuisse in \CE.

\phmvt
\aparag \paysSavoie gives free access and supply in its province to \FRA
during the first turn of the war, if it stays neutral in this war. Supply from
or across a province is impossible if its city is under siege by an enemy of
this city.

\effetlong
\aparag Until the end of the current period, \FRA has a \CB against the owner
of \provinceLombardia.



\event{pII:End Kalmar}{II-4}{End of the Union of Kalmar}{1}{Risto}

\history{1523}

\phevnt
\aparag The effect of specific \ruleref{chSpecific:Sweden:Union Kalmar} is
terminated.
\aparag Because of troubles between \paysDanemark and \paysSuede, both
countries make mandatory white peaces, lowers the European market by 75\ducats
this turn for everyone.
\aparag If \SUE is a \MAJ, roll for 2 \REVOLT in \SUE and \SUE loses {\bf 1}
\STAB.



\event{pII:War Persia Turkey}{II-5}{War between Persia and Turkey}{2}{Risto}

\history{1526-1555}

\condition{Takes place only if \paysperse is inactive. Otherwise re-roll.}

\phevnt
\aparag \paysperse declares war against \TUR.
\aparag \paysperse and \TUR can immediately call allies as per normal rules.
\aparag If \paysperse is neutral, it does not call any ally and is played by
\SPA.

\phadm
\aparag \paysperse receives reinforcements on offensive status for the
duration of this war.



\event{pII:Algeria Vassalisation}{II-6 (1)}{Turkish Vassalisation of
  Algeria}{1}{Risto}

\history{1519}

\condition{Takes place only if leader \leaderBarbaros is alive. Otherwise mark
  as played, but do not re-roll.}

\phevnt
\aparag \paysAlgerie is immediately placed on \VASSAL of \TUR.
\aparag \leaderBarbaros is now also a Turkish leader, and as long as he is
alive, \paysAlgerie is permanent Vassal of \TUR not subject to diplomacy.
\aparag At the death of \leaderBarbaros, the {\bf -3} malus for \TUR to all
diplomacy attempts against all \terme{Barbaresque} countries is cancelled.



\event{pII:Alignment of Barbaresques}{II-6 (2)}{Alignment of
  Barbaresques}{1}{Risto}

\history{1540}

\phevnt
\aparag From now on, the {\bf -3} malus for \TUR to all diplomacy attempts
against all \terme{Barbaresque} countries is cancelled.
\aparag \paysTunisie is immediately placed on \VASSAL of \TUR if \leaderDragut
is alive.



\event{pII:War Poland Turkey}{II-7}{War between Poland and Turkey}{1}{PB}

\history{1526-1535 -- it was not a formal war}

\condition{Turkey may refuse the event, in which case it is not marked and no
  event is re-rolled for. If the event is not refused, apply the following}

\phdipl
\aparag \TUR has a bonus of {\bf +2} on diplomatic actions on minor countries
\paysMoldavie, \paysValachie and \paysTransylvanie.
\aparag \TUR has a free \CB to be used at this turn of the following one
against \POL if it has a province adjacent to this country, or a minor country
in \AM at least, that is adjacent to \POL.
\aparag If \TUR is at war with \POL, any minor country adjacent to \POL that
is in \AM or higher of \TUR will join full war against \POL without test, and
so is placed in \EG.

\phadm
\aparag If there is a Polish \paysUkraine, \POL gains a free \ARMY\facemoins
to fill the Ukrainian army at each turn of the war.



\event{pII:Reformation}{II-8}{Reformation}{3}{Risto}

\history{1522-1560}

\condition{This event is the same as in period I and continues the effects,
  provoking either \ref{pI:Reformation}, \ref{pI:Reformation2} or
  \ref{pI:Reformation3}.}



\event{pII:Schmalkaldic League}{II-9}{War of the Schmalkaldic
  League}{1}{RistoMod}

\history{1546-1547}

\condition{}
\aparag If \ref{pI:Reformation} has not yet occurred once, do not mark off and
re-roll.
\aparag This event cease with the breaking of the League as described in the
event or in \ref{pIV:TYW}.

\phevnt
\aparag The following countries form a defensive league: \paysHesse,
\paysSaxe, \paysThuringe and \paysWurtemberg. They are considered as one
country for declaration of wars, and one alliance for peace terms.
\aparag The Emperor loses diplomatic control of all countries of the League
and can no longer make diplomatic actions on them. Those countries leaves \GE
if there is one.

\phdipl
\aparag The Emperor has a permanent \CB against the League. This \CB is free
if the Emperor is The Sole Defender of Catholic Faith (free \CB also if the
\HAB are Emperors for the Austrian branch and Defender of the Catholic Faith
for the Spanish branch). A war against any country of the League is called a
war against the League; it is a \terme{war of Religion} (so external
intervention is constrained).
\bparag \SPA may ask for limited or full intervention of the \HAB in this war.
\aparag The Emperor may grant the \terme{Truce of Augsburg} regarding the
liberty of belief in the \HRE. Such a decision costs {\bf 1} \STAB and 20 \VP.
\aparag When a war against the League occurs, the minor countries are allied
for any purposes and are played by the first major player in the list that is
not at war against any country of the League: \HOL, \ENG if Protestant, \FRA
if Protestant, \POL if Protestant, \SUE (if Protestant and period III+), \SPA,
\ENG, \FRA, \POL if not. This power is called for as an ally of the League,
but may refuse at no cost. The League plays at the same round of the player
who plays it (whether involved in the war or not).
\aparag Any Major Country having one of the minor countries in the League on
its diplomatic chart can make a limited intervention against the Emperor, as
an ally of the League.

\phpaix
\aparag If the Emperor is Spanish or Habsburg, a test to begin the
\ref{pIV:TYW} is made at the end of each turn of any war between the League
and the Emperor. This test is modified by {\bf +4}. See \ref{pIV:TYW} for the
result of the test and the possibility of this Religious War. If no such war
occurs, peace can be made on the following conditions.
\aparag Each minor country obeys to the usual rules for peace. As they are
allied, a peace against only one country is a separate peace.
\aparag A minor country forced to sign an unconditional surrender breaks from
the League for ever. This replace all the peace conditions.
\aparag The League may be dissolved under the following conditions:
\bparag the last country in the League is forced out, or
\bparag \paysHesse or \paysSaxe has been forced out of the League and the
Emperor has granted, or grants immediately the Truce of Augsburg (at the cost
of {\bf 1} \STAB and 20 \PV).
\aparag If the League is dissolved without the Truce of Augsburg, \SPA keeps
the title of Emperor for one more monarch.
\bparag If the Emperor is from \SPA or \HAB, and has made a war against the
League and suffered a Major Defeat against land forces of the League, it can
decide at the phase of peace to become \CATHCO as in
\ref{pI:Reformation2}. The war ends immediately in a white peace and the
application of the Truce of Augsburg in the \HRE. Both general and specific
events of \shortref{pI:Reformation2} will be applied to \SPA at the following
event phase.

\effetlong
\aparag The countries of the Schmalkaldic League will join some wars caused by
events: \xref{pIV:TYW}, \xref{pIV:Augsburg Revocation}, and \xref{pIV:Unity
  HRE}. The League may reinforce the intervention of \paysPalatinat in
\ref{pIII:FWR}. The League exists no more when involved in the \ref{pIV:TYW}.



\event{pII:War Indian Ocean}{II-10}{War in the Indian Ocean}{1}{PB}

\history{1536-1538 / 1546}

\condition{}
\aparag If a Treaty is militarily enforced between \POR and \paysOman or/and
\paysAden, apply \xnameref{pII:WIO:Revolt Oman Aden} for this (or these)
countries.
\aparag If no Treaty is enforced, apply \xnameref{pII:WIO:War Oman Aden}
against this (or these) \MIN. Both a Revolt and a War can occur (against
different countries).


\subevent[pII:WIO:Revolt Oman Aden]{Revolt of Oman/Aden}

\phdipl
\aparag \TUR has an overseas \CB against \POR at this turn. \TUR gains the
discoveries of \seazone{Mascate} and \seazone{Indus}

\phmil
\aparag The Natives of the region \granderegionOman or \granderegionAden are
activated and will attack units of \POR at this turn. They will not attack
Turkish forces this turn.

\phinter
\aparag If the attack of the colony by the Natives at the end of turn result
in at least 1 level of \COL that should be lost, those levels are not applied
to the COL of \paysOman or/and \paysAden but break the Treaty status of the
country with \POR (they now have No Relation and Portuguese forces are
redeployed immediately).
\aparag If \paysOman or/and \paysAden breaks free from a Treaty with \POR and
\TUR is at war with \POR, \TUR gains a Treaty with this (these) \MIN.


\subevent[pII:WIO:War Oman Aden]{War with Oman/Aden}

\phevnt
\aparag \paysOman or/and \paysAden declare(s) an oversea war to \POR. If both
are at war, they are allied.
\aparag \TUR has an oversea \CB against \POR at this turn, to enter the war as
an ally of \paysOman or/and \paysAden and it gains the discoveries of
\seazone{Mascate} and \seazone{Indus}.  If the \CB is used, \TUR gains a
Treaty with \paysOman or/and \paysAden.

\phadm
\aparag \paysOman or/and \paysAden at war receive(s) Naval Reinforcement at
the first turn of the war.

\phinter
\aparag If \paysOman or/and \paysAden occupy a \TP of \POR at the end of the
turn, they do not burn it if they have a \TP counter available and this \TP is
transformed in a \TP of the minor country. If there is no counter available,
the \TP is burnt down. The choice of the \TP converted is random.



\event{pII:Portuguese Colonial Dynamism}{II-11}{Portuguese Colonial
  Dynamism}{3}{Risto}

\phdipl
\aparag \POR gains a bonus of {\bf +3} for any diplomatic action on
non-European minor countries at this turn.

\phadm
\aparag \POR receives one additional and free strong investment \TP placement
action.
\aparag \POR receives a shift of one column to its favour in the actions
results table for all its \COL/\TP placement attempts this turn.
\aparag \POR may ignore restriction of~\ref{chExpenses:Pioneering} for this
turn.



\event{pII:Spanish Colonial Dynamism}{II-12}{Spanish Colonial
  Dynamism}{3}{Risto}

\phdipl
\aparag \SPA gains a bonus of {\bf +3} for any diplomatic action on
non-European minor countries at this turn.

\phadm
\aparag \SPA receives one additional and free strong investment \COL placement
action.
\aparag \SPA receives a shift of one column in its favour in the actions
results table for all its \COL/\TP placement attempts this turn.
\aparag \SPA may ignore restriction of~\ref{chExpenses:Pioneering} for this
turn.



\event{pII:Union Lublin}{II-13}{Union of Lublin}{1}{PB}

\history{1568}

\condition{If \POL is Protestant or has chosen Support of Orthodoxes, the
  union is impossible. Mark off the case and play \RD instead, with the
  \REVOLT in \POL.}

\activation{}
\aparag The rest of the event is activated when \POL decides to sign the
Union. That is to be announced at any current or following phase of
declaration.

\phdipl
\aparag Both countries in \POL are linked by an Union. All effects described
in \ruleref{chSpecific:Poland:Before Lublin} are applied no more and the new
conditions are described in \ruleref{chSpecific:Poland:Union Lublin}.
\aparag If \POL is not at war against any Major Power at the time of the
Union, play two \REVOLT in \POL. If it is at war against a Major Power, do not
draw any \REVOLT .
\aparag \RUS and \SUE has a \CB against \POL at the turn of declaration of the
Union.

\effetlong
\aparag The Union of Lublin can be broken if someone imposes a peace of level
at least 3 on \POL, and this counts as the gain of 2 provinces (or their
equivalent in War Reparation) for the terms of peace.



\event{pII:Conquest Khanates}{II-14}{Russian conquest of the Khanates}{1}{PB}

\history{Kazan 1547-1552}

\activation{\RUS may refuse this event, in which case it is not marked but no
  other event is rolled for.}

\phevnt
\aparag If \ref{pI:Pskov Ryazan} has not been played, it is played as an
additional event this turn.
\aparag Else, or on a second occurrence of the event, apply the following
effect.
% (Jym) Additional or instead of ?

\phdipl
\aparag \RUS has a free \CB against a Khanate of its choice at this turn only.

\phpaix
\aparag This Khanate will surrender unconditionally and will be entirely
annexed by \RUS if \RUS controls its capital and half of the provinces of the
Khanate.



\event{pII:Superiority over Khanates}{II-15}{Russian Superiority over the
  Khanates}{1}{PB}

\history{Astrakhan 1554-1556}

\activation{\RUS may refuse this event, in which case it is not marked but no
  other event is rolled for.}

\phevnt
\aparag \RUS advances its \terme{Land Technology} marker of {\bf 3}
boxes. This event might place the Land Technology of \RUS higher than
\terme{Orthodox} \terme{Land Technology}. This is allowed and the marker stays
in place until the \terme{Orthodox} \terme{Land Technology} becomes higher
than the one of \RUS, in which case \RUS can resume its progression.

\phdipl
\aparag \RUS has a free \CB against a Khanate of its choice at this turn only.

\phpaix
\aparag This Khanate will surrender unconditionally and will be entirely
annexed by \RUS if \RUS controls its capital and half of the provinces of the
Khanate.



\event{pII:War Russia Poland}{II-16}{War between Russia and Poland}{1}{PB}

\history{1507-1522 / 1534-1537}

\condition{If \RUS and \POL are already at war against each other, mark off
  the case and play \RD instead.}

\phevnt
\aparag \RUS has a temporary \CB against \POL and \POL has a temporary \CB
against \RUS. Those \CB may be used this turn or the following turn. If a
power does not use, it loses 1 \STAB on the second turn.



\event{pII:War Russia Turkey}{II-17}{War between Russia and Turkey}{1}{PB}

\history{Crimea 1521-1523, 1559, 1572}

\activation{\RUS has the control of this event.}

\phdipl
\aparag \RUS has a free \CB against a Khanate of its choice at this turn only.
\aparag If this \CB is used, the attacked country is placed at least in \AM of
\TUR that has now the opportunity to enter war to support the minor country or
not.

\phpaix
\aparag This Khanate will surrender unconditionally and will be entirely
annexed by \RUS if \RUS controls its capital and half of the provinces of the
Khanate.
\aparag If \TUR did not enter the war to support the Khanate and it is
destroyed as a result of this war, \TUR has a free \CB against \RUS the turn
following the conquest.



\event{pII:Forward Baltic Sea}{II-18}{Forward to the Baltic Sea}{1}{PB}

\history{1558-1561}

\condition{}
\aparag If the \pays{Teutoniques1} do not exist any more (either by conquest
or by event \ref{pIII:Northern Secularisation}), mark off and play \RD
instead.
\aparag If \RUS has no province adjacent to the \pays{Teutoniques1}, do not
mark off and roll for another event.

\phevnt
\aparag \RUS has a free \CB against the \pays{Teutoniques1}.

\phadm
\aparag The \pays{Teutoniques1} take their reinforcements in offensive
attitude during the first turn of the conflict.

\phpaix
\aparag Before testing for any peace, 1d10 is rolled, modified by the peace
differential of \RUS against the \pays{Teutoniques1}. If the result is 6 or
more, the \pays{Teutoniques1} collapse and no peace occurs now. At the
following event phase, the first event considered rolled for is automatically
\ref{pIII:Northern Secularisation}.



\event{pII:American Resistance}{II-19}{Resistance of the American
  Empires}{2}{PB}

\history{not historic}

\condition{}
\aparag If there is no \COL in \continent{America} (excepted the islands) do
not mark off and re-roll.

\aparag If both empire have already collapsed, play \RD instead of this event
and mark off.

\aparag Else, \paysInca or \paysAzteque (decide randomly, or take the one that
did not collapse), is affected by the following event.

\phevnt

\aparag The permanent Treaty of this empire with European countries is
nullified. From now on, it is dealt with as a normal non-European country.

\aparag The technology of both \paysInca and \paysAzteque raise to the
technology of \paysChine and other countries of \ROTW.

\aparag Both empires can still be destroyed by capturing their capital city if
the invading forces survive an attack by Natives at the end of turn. The
normal rules are then applied: creation of a \COL of level 3, destruction of
the minor country, reduction to 2 \DT of the force of Natives in every
province of the region; if the conqueror is \SPA, a Mission is installed in
the new \COL and the highest rank Conquistador present in the region is
nominated as Vice-Roy.



\event{pII:Chinese Expansion}{II-20}{Chinese Oversea Expansion}{1}{PBNew}

\history{abandoned before 1492}

\condition{}
\aparag If \ref{pI:Chinese Expeditions} was not played, play this event and
mark off the present one.
\aparag If \ref{pI:Chinese Expeditions} has been played, play the remaining of
this event.

\phevnt
\aparag \paysChine installs one new \TP of level 1 in \granderegionFormose and
one in \granderegionPhilippines if there is any province still empty, with 1
\DT on each one. It takes the exploitation of one \RES{Products of Orient}
(without concurrence; a Major Power will have to make proper \CONC to take
them back).
\aparag If \paysChine has lost some \TP since \ref{pI:Chinese Expeditions}, it
declares an overseas war to any European country having a \TP or \COL in the
same region as any lost \TP. If it has lost none, it declares an Overseas War
to any European power having a \TP in \granderegionFormose or
\granderegionPhilippines.
% (Jym) What? What happens in a case of \MIN/\MIN?

\phmil
\aparag If \paysChine is at war due to this event, it adds one \ARMY\faceplus
to its basic forces, as an invasion force with a general from the minor
pool. Its reinforcements are increased in this war by \LD and \ND. It can of
course use its usual basic forces and reinforcements, and the Natives in
\paysChine.

\phpaix
\aparag If \paysChine controls a foreign \TP at the end of the military turn,
they do not burn it if they have a \TP counter available and this \TP is
transformed in a Chinese \TP. If there is no counter available, the \TP is
burnt down. The choice of the \TP that are converted is random if there is not
enough counters.
\aparag On the first turn of this war (only), \paysChine does not accept
automatically a white peace. A formal peace should be obtained.



\event{pII:Apparition Mughal Empire}{II-21}{Apparition of the Mughal
  Empire}{2}{PBNew}

\history{1526-1555}

\phevnt
\aparag On the first event, the non-European minor country \paysMogol is
created. It has 2 \ARMY\faceplus and the leader \leader{Great Mughal} (until
replaced by a further event).
\aparag The \paysMogol will try to invade \textbf{2} regions during the turn,
following the procedure \ref{pII:Mughal Expansions} described underneath.
\aparag Even if the country does gain no region, it still exists (and can gain
provinces with new events).



\event{pII:Mughal Expansions}{II-A}{\paysMogol Expansions}{*}{PBNew}

\activation{When a \nameref{pII:Mughal Expansions} is called for by an event.}

\phevnt
\aparag The \paysMogol will try to invade the regions in (or near) India in
the following order: \granderegionDelhi, \granderegionAfghanistan,
\granderegionAoudh, \granderegionBengale, \granderegionGujarat,
\granderegionPendjab,\granderegionIndus, \granderegionBalouchistan,
\granderegionOrissa, \granderegionGondwana, \granderegionMumbai,
\granderegionHyderabad, \granderegionMalabar, \granderegionKarnatika. A
circled number on the map shows this order.
\aparag Each event will call for a varying number of invasions (between 1 and
4). The province invaded are determined and the invasion resolved in
parallel. The provinces are aimed in the following order.
\bparag The regions with the lowest number and no \paysMogol
\countermark{Area} counter in it (so it is not ``conquered'' or ``lost'' due
to failed invasion or a rebellion) are the first aimed, by an invasion. Note
that a failed invasion during one event will force the \paysMogol to invade
again the same region during the next expansion.
\bparag Then if needed, the regions having a \paysMogol \countermark{Lost
  Area} counter and with the lowest number are second to be aimed at, for a
new invasion that will have a malus of \bonus{-1}.
\bparag If there is not enough uncontrolled regions to make all the attempts
called for by an event, a test of Rebellion is made in replacement for the
remaining actions called for. The regions aimed are those that are conquered
and have the highest number. A Rebellion is resolved as a invasion but with
\bonus{-1}.
\aparag The list of regions invaded is defined globally during the event, and
the resolution will wait the end of the turn. The \paysMogol is not
(technically speaking) at war with countries having \TP/\COL or regions in the
aimed regions. The invasion attempt will be resolved at the end of the
military phases. Thus, the expansion does not interfere with other kinds of
war that can take place and involve the \paysMogol.

\phinter
\aparag[European resistance to invasion]
\bparag Each country having a \TP/\COL in a province of an invaded region can
choose to oppose or not the Mughal invasion at the end of the military
rounds. The Major Powers decide simultaneously. This decision is taken
province by province (one can resist somewhere and do nothing somewhere else)
and one needs a land stack to resist in a given province. An opposition does
not affect the diplomatic status of any power with the \paysMogol
\bparag Non-European minor countries do not oppose invasion. European minor
countries may oppose if their diplomatic patron decides it. They can use their
non-European basic forces for this.
% (Jym) What about neutrals?
\bparag In each province where invasion is resisted, a land battle is fought
between the forces of the European country and the 2 \ARMY\faceplus of the
\paysMogol This complete force is used in each battle (assuming that they have
plenty of time to muster reserves).
\bparag If the region is not invaded but in Rebellion, the \paysMogol use only
one \ARMY\faceplus.
\bparag The current leader of the \paysMogol is used in each battle.
\bparag Depending on the winner of the battle, the invasion test will be
modified to improve or lower the chance of conquest by the \paysMogol. Note
that no resistance is not as bad as a failed resistance.
\aparag[Invasion tests] For each invaded region, a test is made on the
following table, by rolling 1d10 added to modifiers.

\newcommand{\mughalinvasions}{ \GT{mughalinvasions}{Mughal Invasions}%
  \GTcontent{%
    \graytabular\begin{tabular}{lll} 1d10+mod. & Result & \TP/\COL
      Loss\\\hline\ghline%
      \textlessequal1 & 1 adjacent province is lost& 0 \\\ghline%
      2--4 & failed conquest & 1 \\\ghline%
      5 & failed conquest & 2 \\\ghline%
      6--7 & conquest & 3 \\\ghline%
      8--9 & conquest & 4 \\\ghline%
      10--11 & conquest & 4 \\\ghline%
      \textgreatequal12 & conquest & 6 \\\ghline%
    \end{tabular}
  }%
  \GTlegend[caption=captiona,east north,text width=67mm]{%
    \begin{modlist}
    \item[+3] if \shortleader{Akbar} leads the invasion
    \item[+2] per battle gained in resistance in the region
    \item[-2] per battle lost in resistance in the region
    \item[-1] if the region belongs to a minor country or has a \TP of a
      non-European minor country in it.
    \item[-1] if the region was lost once, or is in Rebellion
    \item[\textpm?] modifier called by some events.
    \end{modlist}
  }%
  \GTdecorate%
}

\GTtable{mughalinvasions}

\aparag[Invasion results]
\bparag \textbf{Conquest} means a successful invasion. Put a counter in the
region showing that is now belongs to the \paysMogol The first time region
\granderegionBengale is conquered, its resources raise to 2 for each type.
\bparag \textbf{Failure} is just what it means ; the regions is left to its
current owner (even in case of a Rebellion).
\bparag On a \textbf{result of 1 or less}, the conquest is failed (or the
Revolt successful). One region is lost to the \paysMogol; put a \paysMogol
\countermark{Lost Area} counter in the region (or flip over the counter
already therein). The region affected is the first one in the list that is not
already lost by \paysMogol (we give here only the numbers): 2, 11, 14, 13, 12,
10, 9, 8, 7, 6, 5, 4, 3.
\bparag The \textbf{Losses} for \TP/\COL are the level lost by each colonial
settlement in the conquered province. Each level of fortification in the
\TP/\COL forfeited counts for one of those loses (including permanent
fortresses given by cities if there is a \COL; the level may be lost, and
comes back automatically for the next turn).
\bparag If a minor country (\paysGujerat, \paysVijayanagar, or \paysAfghans,
\paysMysore, \paysHyderabad) loses its last region due to an invasion, it is
destroyed immediately. It may reappear later due to new events.



\event{pII:Crusade}{II-B}{Call to Crusade}{*}{JymMod}

\history{Did not happen}
\dure{Until the end of the war.}

\condition{May be triggered by \TUR conquest of christian provinces.}

\phevnt
\aparag[Call to crusade] Each Catholic country has a mandatory free \CB
against \TUR to be used immediately.
\bparag As an exception, the \terme{Sole Defender of the Catholic Faith} must
decide first to use it or not. Then, these \CB are resolved in initiative
order.
\bparag All countries that use this \CB are call crusaders and are
automatically allied against \TUR.

\aparag[Mediation of the Pope] Any Catholic country can immediately propose a
white peace to any or all of its current Christian enemies.
\bparag If one or more of these peaces is refused, the free crusade \CB is
consider to be fulfilled (for the country that asked for the mediation). The
would-be crusader is not forced to declare war on \TUR or loss \STAB.
\bparag Catholic minors always accept this peace. Other minors never accept it
(and thus give an ``excuse'' for not participating).
\bparag If a country does not ask the mediation of the Pope, the fact that it
is at war is not an excuse for avoiding the Crusade.

\aparag[Refusing to participate] Any Catholic country that either refused to
participate or rejected the mediation of the Pope suffers from the following
effects:
\bparag Loss of 1\STAB (2\STAB for the \terme{Sole Defender of the Catholic
  Faith}).
\bparag Loss of the diplomatic control of \paysPapaute.
\bparag All other Catholic majors have a normal \CB against this country this
turn.

\aparag If no major country participates in a Crusade, no minor participates
either and the rest of the event is ignored.

\phdipl
\aparag[Minor Countries and Crusades] The following minor countries only: \HAB
(if Emperor or \paysHongrie has been inherited), \paysHongrie, \paysPapaute,
\paysGenes, \paysChevaliers, \paysToscane and \paysParme always participate in
a Crusade.
\bparag If they are on the diplomatic track of a crusader, they are
immediately raised in \EW (if not already higher).
\bparag Otherwise, they are temporarily put in \EW of the first crusader (the
first country that declared war on \TUR, either the \terme{Sole Defender of
  the Catholic Faith} or the one with higher initiative). They will return
back to the \Neutral box at the end of the crusade.

\aparag Other Catholic minors may participate if controlled by the crusader,
using the normal rules.

\aparag Protestant, Orthodox and Muslims minors may not participate in a
Crusade (even if controlled by a crusader).

\aparag[\paysHongrie, \paysHabsbourg, the \HRE.]
% Jym: Crusade stop in pIV when VEN is minor.
% \bparag \pays{Venise} (as a minor country), even if controlled by a player,
% participates also to the Crusade if one modified die-roll of 8 or more is
% obtained.  This die-roll is modified by +1 for each Venetian province of
% 1492 already conquered by the Turkish player.
\bparag If the \hab is the Emperor of the \HRE, it participates automatically
in the Crusade if at least one provinces of either \paysHabsbourg, \HRE or
\paysHongrie is owned by \TUR.
\bparag \paysHongrie automatically participates in the Crusade on a die roll
of 8 or more. This roll is modified by \bonus{+1} for each province of
\paysHongrie owned by \TUR.

\aparag[Endorsement of \paysPapaute]
Crusaders receive at the end of each Diplomatic phase a global diplomatic
income of 150\ducats, shared equitably between them in divisions of 25\ducats
(the surplus going on the first participant).
\bparag This money is coming from the \paysPapaute, so the usual 50\ducats
gift (see \ruleref{chSpecific:Papacy:Gold}) that \paysPapaute gives for a \AM
status is not perceived anymore.
\bparag This is valid during all the length of the current Crusade. At the
same time, the modifier value for \SUB on \paysPapaute becomes -150.

\phadm

\aparag[Crusader army] The crusaders, whether major or minor, may used the
Crusader \ARMY counters to hold troops of any crusader country.
\bparag Whatever the actual content of these counters, they are considered to
be of class \CAIII and have all the features of this class.
\bparag Track the nationalities of the \LD in these \ARMY in order to give
them back to their owner.
% Jym, adding:
\bparag Crusader \ARMY may be lead by \LeaderG of any crusader country, even
if it has no \LD inside.
\bparag Note that he may well ``pick up'' troops from other crusaders without
their agreement.

\aparag[Military Leader of the Crusade] A \LeaderG or \LeaderA of the first
participant player is chosen as leader of the Crusade. For the duration of the
Crusade, he is considered to possess the highest hierarchical rank (even above
monarchs).
\bparag He is allowed to lead any troops of crusaders countries. He may thus
lead a stack with no troops of his own nationality.

\phmil
\aparag[The way to Crusade] crusaders countries automatically give free access
to their territory and supply to other crusaders.
% Jym, adding:
\bparag In the rare case where two crusaders are still at war elsewhere, they
must choose upon entering enemy territory whether the stack is crusading (and
allied) or not. The status of a stack may not change before it exits enemy
territory. Crusader stacks still in enemy territory at the end of the Crusade
are immediately moved into friendly territory per the peace redeployment
procedure.

% Jym, precision on the definition of "waging crusade".
\begin{designnote}
  The following points are meant to force crusaders to really ``wage crusade''
  and not sit and watch. There are undoubtedly loopholes in them that tricky
  players will find and use to circumvent the Crusade rules. Remember here
  what the spirit of the rule is: if you're part of the Crusade, you must
  really participate in the Crusade. Use good sense and fair play. Do not let
  a player that really participated in the Crusade be punished by this. Do not
  let a player that found a loophole to abuse it. Make an homerule if you
  don't think this correct.
\end{designnote}
\aparag[Participating to the crusade] At the end of the first military round
of each turn of the Crusade, each major crusader country must design one of
its stack with at least 3\LD or \ND (or 6\NGD) belonging to it as a ``main
crusading stack''.

\aparag At the end of each following round, each crusader major country loss
1\STAB unless at least one of the following conditions is true.
\bparag All his troops initially in his main crusading stack (they may split)
have been destroyed (reinforcing the crusading stack does not prevent the
destruction of the initial troops).
\bparag All his troops initially in the crusading stack moved this round and
end up closer to the territory of \TUR or its allies.
\bparag Troops of this country (any troops) have participated this round in at
least one battle (land or sea) or siege (besieger or besieged) against \TUR or
its allies.
\bparag Troops of this country (any troops) are in a province owned by \TUR
(not its allies).

\phpaix
\aparag[Crusades and Separate Peace]
A crusader major country that makes separate peace with \TUR undergoes a loss
of 3 \STAB (instead of the usual 2 for breaking an alliance). This separate
peace also gives, in addition, a temporary \CB to all the other crusading
players against him, valid until the end of the Crusade (instead of the usual
next turn only).
% Jym: useless as TUR can always discuss "informally" with FRA/VEN/... and let
% *him* propose the separate peace. We're not trying to reimplement Grand
% Siecle's peace procedure...
% \bparag If the Turkish player asks for peace from a Crusader, he must ask it
% from all of them. He must choose the provinces part of peace conditions
% among Christian provinces the most recently conquered by him.
\bparag No \MIN participating in a Crusade may be tested by the Turkish player
for separate peace attempts, except if a \MAJ has signed a separate peace with
Turkey (including the same turn).

\aparag[Peace conditions] If \TUR cedes territory to the crusaders, it must be
chosen among the following provinces, in order:
\bparag Any province that were christian in 1492, in reverse order of conquest
by \TUR (the most recent conquest first) ; \provinceMoreas , \provinceHellas,
\province{Terra Sancta}, \provinceLubnan, \provinceAlep.
\bparag Only provinces controlled by crusaders may be chosen.
\bparag These provinces are given back to their 1492 owner (if Christian),
even if he did not participate in the Crusade and recreating it if it was
destroyed. Provinces initially belonging to an non-Christian country are given
to \paysChevaliers.
% Jym: let's avoid creating a "Latin states of Orient" minor...
\bparag Provinces of the \regionBalkans that are automatically annexed by
Christians during a Crusade are also given to \paysChevaliers.
\bparag Each province that \TUR loses during a Crusade give 10\VPs to each
crusader still at war against \TUR.

\stopevents

% Local Variables:
% fill-column: 78
% coding: utf-8-unix
% mode-require-final-newline: t
% mode: flyspell
% ispell-local-dictionary: "british"
% End:

% LocalWords: pI pIII pII Habsburg Kalmar Vassalisation Barbaresques Lublin
% LocalWords: Schmalkaldic Khanates Mughal Risto PBNew RistoMod Ristomod pIV
% LocalWords: Serenissima malus Barbaresque TYW Mascate Khanate Teutoniques
% LocalWords: lll captiona mughalinvasions HRE WIO Jym
