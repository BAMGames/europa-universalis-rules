\definechapterbackground{Redeployment}{redeployment}
\chapter{Redeployment}\label{chapter:Redep}

After the military phase, some ``cleaning'' is required before the next
turn. During the redeployment phase, lasting military affairs are
resolved. First, attacks by natives and privateers, then looting of occupied
provinces, extension of revolts and construction of \Presidios, and lastly
mandatory retreat of some troops and bringing gold back home.

\RedepPhase

\section{Attacks by Natives}\label{chRedep:Native Attack}

\aparag Natives activated during the turn, as well as forces of \ROTW minor
countries may attack colonial establishments.
\bparag Natives always attack in each and every province where they have been
activated during the turn (whatever the cause of activation).
\bparag Troops of \ROTW minor countries inside \Areas owned by the minor
always attack establishments of countries against which they are at war.
\bparag Troops of \ROTW minor countries outside \Areas owned by the minor may
attack establishments of countries against which they are at war. The
controlled of the minor decide whether they attack or not.

\aparag[Combined attacks]
\bparag If, in a given province, several forces attack, there are combined in
one and only one attack is resolved, totalling all the troops participating in
it.
\bparag This may includes natives of the province as well as one or more
(allied) \ROTW minors.
\bparag If there is only one leader in such a stack, he is considered as
commanding the attack. If there are two or more leaders, use normal rules to
determine who is leading the attack.
\bparag In case of a combined attack with minor troops and natives, the
controller of the minor may choose to attack with the minor troops only
(typically, in order to avoid malus if the natives were defeated this turn).

\aparag[Forces attacking]
\bparag In each province, sum up the number of \LD participating in the
attack.
\bparag Remember that each province of a given \Area holds the same number of
native \LD and that killing natives in one province does not change the number
of natives in other provinces of the \Area.
\bparag Example: There are 40\LD in \granderegionJapon. That means there are
40\LD in each of the four provinces of the \Area. Even if 30\LD have been
killed in \provinceEdo during a given turn, there are still 40\LD in
\provinceKyoto this turn.

\aparag[Resolving the attack]
\bparag Each attack is resolved by rolling one die on \ref{table:Pirates
  Natives Raids}. This die-roll is modified by:
\begin{modlist}
\item[+1] for each \LD in defence of the establishment (even besieged).
%\item[+2/+4] for each \ARMY\facemoins/\Faceplus in defence (even besieged).
\item[+N] level of the fortress.
\item[+M] Manoeuvre value of a \terme{land leader} in defence.
\item[-1] For each \LD of the attacking force (native or \ROTW minor).
\item[-1] For each foreign \COL or \TP in the same \Area.
\item[-M] Manoeuvre value of an attacking leader (native or \ROTW minor).
\item[+3] If the Natives were defeated at least once in the province this turn
  without being routed.
\item[+6]If the Natives were routed at least once in the province this turn.
\end{modlist}

\aparag[Reading the result]
\bparag The result is read by cross-referencing the (modified) die roll with
the last two columns of the table.
\bparag The ``Pillages \TP/\COL'' column gives a number of losses on the
settlement. Ignore the \textddag\xspace and \textdag.
\bparag The ``Perm. losses on land'' column gives a number of losses on land
forces.
\bparag Both results (losses on settlements \textbf{and} on land forces) are
applied.

\aparag[Applying losses]
\bparag Losses on settlements: The establishment of the province losses as
many levels as indicated. If it reaches level 0, it is immediately
destroyed. Exceeding losses are ignored.
\bparag Losses on land forces: as many \LD as indicated are lost. The losses
may also be applied to fortifications (loosing 1 level of fortress or 1 fort
instead of 1\LD). The controller chooses whether to loose troops or
fortifications, but as many loses as possible must be satisfied. Exceeding
loses are ignored.

\aparag[End of activation]
\bparag Once the attack is resolved in a province, natives of this province
cease to be active.
\bparag Remark: Natives of provinces owned by \ROTW minors at war will be
automatically reactivated next turn unless peace with the minor is signed.

% \begin{exemple}[Bantu raids]
%   In the late game, \ref{pVI:
% \end{exemple}

\GTtable{atendofturnattacks}

\section{Attacks by Pirates \& Privateers}\label{chRedep:CorsairAttack}

\aparag Pirates and privateers attack commercial fleets to attempt to decrease
their levels, and possibly to capture gold repatriated to Europe by these
fleets.
\bparag[Pirates] Pirates appear as explained in \ruleref{chEvents:Piracy} and
they remain until completely eliminated. They are active every turn.
\bparag[Privateers]
\label{chRedep:CorsairAttack Privateer}
Privateers are raised by Major Powers (see \ruleref{chExpenses:Recruiting
  Privateers}), or are in the basic forces of some minor powers (the
\pays{chevaliers} and the "Barbaresques" countries). They must go out at sea
on the first or second round or they will have no effect.
\bparag Beginning with the third round, they stay in the sea they were placed
in, and will be able to attack one \STZ or \CTZ in this sea or an adjacent
sea. The specific \STZ or \CTZ has to be annouced at that time.

\aparag[Raiding Fleets with Privateer Admirals]\label{chSpecific:Raiding
  Privateer admirals}
Privateer, or an Admiral with Privateer capacity, may lead one \corsaire he
starts the turn with.  He may lead it in the same stack as naval forces not
containing a \FLEET.  The \corsaire does not count for attrition, nor in
battle.  The stack acts both as regular naval force (and can attack, blockade,
and so on), and a Privateer stack (other players may attempt to suppress the
\corsaire counter).  The \corsaire does not count for attrition, nor in battle
(nor is affected by battles).  The stack may split at any time (for instance
if the naval force has to retreat in a port), and the leader chose which stack
he stays with.
\bparag As an exception to \ruleref{chRedep:CorsairAttack Privateer}, a
\corsaire led by Privateer or Privateer-Admiral may move after the second
round, and has to remain in place only on the last round (the player telling
at the beginning of this round which \CTZ\\STZ will be attacked if there are
several of them). However, it still has to be at sea at the end of every round
after the first, else (if at port), it cannot leave again for the rest of the
turn and will not attack commercial fleets (or loot) this turn.
\bparag Note that the leader may move as he prefers but can only lead the one
\corsaire he starts the turn with (even leaving it then coming back), or naval
forces.
%% PB 07/2008 : TBD ??  J'ai essaye de simplifier cette partie avec la version
%% ci-dessus, pour mieux
% représenter aussi ce qu'on veut.
%
% \corsaire be moved every round to another \corsaire or naval stack, excepted
% on the last round where it can not join a new \corsaire counter.
% \aparag[Raiding Fleets with Privateer Admirals]\label{chSpecific:Raiding
% Privateer admirals}
% \textbf{[To be deplaced in military]} Naval forces not containing a \FLEET
% that are led by a Privateer, or and Admiral with Privateer capacity may move
% with one \corsaire counter in their stack if it begins even after the 2nd
% round.  Naval forces not containing a \FLEET that are led by a Privateer, or
% and Admiral with Privateer capacity, may be announced to be conducting
% privateer attack of trading fleets (and potentially looting) at the latest
% at the end of the second round if at sea.
% \bparag In that case, replace one \ND for a \corsaire\facemoins, or two \ND
% for a \corsaire\faceplus.  AUTRE VERSION: sans transformation de pions
% \bparag In this case, they may not leave anymore the sea zones adjacent to
% the \CTZ or \CTZ they are attacking, for the remaining of the Military
% Phase.
% \bparag The force is hereafter for the rest of the Military Phase dealed
% with as being \corsaire\facemoins if they have between 2\de and 1\ND2\de,
% and \corsaire\faceplus if they have 2\ND (or \NGD) or more, with no
% capacity as a naval force. It tests for attrition only if decides to move to
% another sea zone; it can not attack, intercept of be attacked or intercepted
% (except to attack Convoys as per Privateer rules).
% \bparag If a roll to reduce Privateer succeed, the stack suffers 50\%
% attrition.  If it losed forces so as to have less than 2\de, the stack
% becomes a regular naval force at the beginning of its next movement segment.

\aparag[Looting by Pirates or Privateers] Pirates and Privateers may try to
loot Trading Posts or Colonies, and also enemy provinces for privateers, that
are a province bordered by the sea they are in.
\bparag Looted provinces, Colonies or Trading Posts may belong to minor
countries or to players. For privateers to be allowed to loot, it is necessary
that a state of war exists between the owner of the privateer unit and the
owner of the looted province. Overseas Wars are enough to loot \TP or \COL,
but not European provinces.
\bparag \textit{Exceptions:} Looting of European provinces by the
"Barbaresques" is permitted, as well as looting in their provinces. Sea Hounds
may loot European provinces also, see \ruleref{chSpecific:England:Sea Hounds}.
\bparag Pirate may loot following \ruleref{chEvents:PiracyTarget}. After a
turn of looting, non-eliminated pirates go back to the \STZ they belong to.
\bparag The privateer intending on looting is placed in the concerned
province, Trading Post or Colony. They have to disembark during any round
except the last from the sea zone they are operating in.
\bparag If privateer/pirate unit is still present at the Redeployment phase,
it loots. Looting privateer/pirate are unaffected by forces or battles (except
that those forces may attempt to destroy them during the military phases).
\bparag A maximum of 1 privateer/pirate unit (any side up) can loot the same
Colony/province in the same turn. Privateers/Pirates looting a province or
\COL/\TP can not attack at the same turn the \CTZ/\STZ.

\aparag[Actions of Pirates and Privateers] In \STZ/\CTZ where
pirates/privateers are active, one die is rolled on \tableref{table:Pirates
  Natives Raids}.
\bparag Pirates attack all fleets in the zone and are resolved first as a
separate atack in each zone.  countries.
\bparag Privateers of \terme{Barabaresque} countries make joint attacks after
Pirates.  Turkish privateers may be added to the same attack if at war against
all the aimed christian countries, or if taking advantage of rule
\ruleref{chSpecific:Balkans}.
\bparag Then one attack is resolved for all the Privateers of the same
alliance, in the order of Initiative.  Privateers may target only fleets of
countries against which their owner is at war (or overseas
war). \emph{Exception:} see the restricted Overseas War of the
\pays{chevaliers} or the "Barbaresques" countries.

\bparag The die is modified as follows:
\begin{modlist}
\item[+2] if the \corsaire is not exactly in the sea zone of the \STZ.
\item[+3] if only one \corsaire\facemoins
\item[+1] per side of target \TradeFLEET or \FLEET (a Convoy counts as 2
  sides)
\item[+1] If one or more \ND in defence and no \FLEET
\item[+2/+4] per \FLEET\facemoins/\Faceplus defending
\item[+M] \Man of a defending \LeaderA
\item[-1] per round at sea (max. -3) (NA on Convoy attacks)
\item[-M] \Man of a \corsaire admiral %(\textonehalf for land raids in Europe)
\item[+1] if a naval battle occurred in the sea %(not for land raids)
\item[-2] \pays{chevaliers} with Christian port on \seazone{Egee} or
  \seazone{Mediterranee E}
\end{modlist}
\bparag Levels losses are taken on the target players' fleets present in the
\STZ/\CTZ (including minors).

\aparag[Defense against Pirates and Privateers]
\bparag[Defending naval force] A defending naval force is a stack of
fleet/detachment markers present at the start of the Redeployment phase in a
sea zone that is part of the concerned \STZ/\CTZ, that decides to defend (and
can do so because of war status).
\bparag The player can choose to suffer his losses from his warships in the
\STZ/\CTZ (if any). In such a case, one eliminated \ND equals the loss of 1
level of commercial fleet.

\aparag[Return of lost levels] All levels eliminated by privateers and pirates
are losses to the \terme{current level}, not the \terme{maximum level}. See
\ruleref{chExpenses:Commercial Fleet Adjustment}.
\bparag Each \textetoile result destroys permanently one level of commercial
fleet (it reduces by one the \terme{maximum level}).  The largest fleet is
affected by this permanent loss. In case of equality, the owner of the
Privateer decide the fleet to be reduced; for Pirates, this is done at random.

\aparag[Income of Privateers] Each level eliminated by a Privateer brings an
income to the player that controls the privateer unit equal to the small
number printed in the \STZ. Privateers of minor countries give no income (even
if Vassals).
\bparag This ``privateers income'' is to be placed in \lignebudget{Pillages,
  privateers} of the privateering player.
\bparag In case of stacks with \corsaire from several powers, this income is
divided between the powers (including minor ones).

\aparag[Results of Pirates \& Privateers Looting] These lootings are resolved
exactly in the same manner: one die-roll on \tableref{table:Pirates Natives
  Raids}, modified as follows:
\begin{modlist}
\item[+3] if only one \corsaire\facemoins
\item[+2/+4] per \ARMY\facemoins/\Faceplus defending
\item[+1] Per \LD (including militia) %against land raids
\item[+M] \Man of a defending \LeaderG/\LeaderC/\LeaderGov
\item[-1] per round in province (max. -3) %(NA on Convoy attacks)
\item[-M] \Man of a \corsaire admiral (\textonehalf for land raids in Europe)
  % \item[+1] if a naval battle occurred in the sea (not for land raids)
\item[-2] \pays{chevaliers} with Christian port on \seazone{Egee} or
  \seazone{Mediterranee E}
\item[+N] Twice the level of the fortress, +1 for fort
\end{modlist}
\bparag If a result \textddag\ is obtained on the table, the province or
\TP/\COL is looted: place a marker \PILLAGE\faceplus. The privateer's owner
receives the total income of the province/settlement (gold stored here and
resources exploited included).
\bparag If a result \textdag\ is obtained on the table, the province or
\TP/\COL is weakly looted: place a marker \PILLAGE\facemoins. The privateer's
owner receives half the total income of the province/settlement (resources
exploited included), but Gold stored here is entirely captured.
\bparag Pirates receive no income for looting.
\bparag These incomes are placed in \lignebudget{Pillages, privateers}, except
stored gold in \lignebudget{Gold from ROTW & Convoys}.
\bparag There is neither loss of land forces due to the looting (opposite to
Natives attack) nor protection by sacrificing forces in the province.

\aparag[Minor countries against Piracy]
\bparag Minor countries at war can use their naval forces against Privateers
and Pirates in \STZ or \CTZ where they have a \TradeFLEET of their own.
\bparag Christian Minor countries whose \TradeFLEET are attacked by privateers
of \terme{Barbarques} may also use their naval forces to fight against those
privateers (Patron's choice to deplace their forces), even if at peace.
\bparag Against Pirates, minor countries at peace fight in an abstract way.
Roll 1d10 for each \STZ or \CTZ with Pirates and add 1 for each commercial
fleet of a minor country (+2 if the fleet is \Faceplus).  If the result is 8
or higher, one \corsaire\facemoins is eliminated.



\section{Extension of Revolts}

\aparag \REVOLT in inactive minor countries are automatically removed without
any roll.


\subsection{Loss of \STAB due to Revolts}
\label{chRedep:Revolts Stability}
\aparag [TBD] \REVOLT in minor countries cause loss of \STAB to their
diplomatic patron as if they were in it.

\aparag If one or more \REVOLT still exist in a country, the country loses
\STAB
\bparag for each \REVOLT\faceplus, it loses 1 \STAB level
\bparag for all \REVOLT\facemoins, it loses only one additional \STAB level
\aparag However, the maximum a player can lose from revolts of all types is 3
\STAB levels.


\subsection{Extension of \REVOLT}
\label{chRedep:Extension Revolts}
\aparag Then, after \STAB losses, \REVOLT extend. Adjust the \REVOLT markers
simultaneously:
\bparag A previously \REVOLT\facemoins becomes a \REVOLT\faceplus
\bparag A previously \REVOLT\faceplus generates a \REVOLT\facemoins in the
same or adjacent province (belonging to the victim country, or in the region
allowed if the \REVOLT was created by an event), of the choice of the player
controlling the \REVOLT.
% Jym, following JC.
\bparag \textbf{[TBD]} \REVOLT in \regionIrlande may extend this way across
the \seazoneMan into \ANG.
\bparag If the extension of a \REVOLT\faceplus is not possible, a \REVOLT \LD
is placed in the same province (or an \ARMY\facemoins if there was already 1
\REVOLT \LD present, the \LD being absorbed in the \ARMY).
\bparag Remember that there can be at most 2 \REVOLT markers stacked together.
\bparag Unbesieged cities in revolt and revolted troops generate a
\REVOLT\facemoins in their province if there is no \REVOLT counter in it.
% Jym, yes this was in EU6...  removed, 04/2011.
% \bparag When all provinces bordering all the sea zones attached to a
% particular \STZ or \CTZ are in revolt (unusual but possible), place one
% pirate \corsaire\facemoins in this \STZ/\CTZ at the beginning of the Peace
% phase (during the extension of revolts).

% Jym: si PB pas sur, je vire.
% \bparag \textbf{[TBD]} Revolt \Faceplus in provinces with at least an \ARMY
% counter of the owner is controlled: it does not extend.  (PB 07/2008: I
% favor that; 09/2009: I am not so sure now....)

% \begin{todo}
%   Check if this is useful. Jym believes the events allowing this explicitly
%   state the provinces allowed for proliferation. Anyway, this is too fuzzy
%   to be used as a rule without crosschecking every event that generates
%   revolts. Plus it may have a bad effect (eg Pugatchev revolt spreading in
%   still independant Khanates...)
% \end{todo}
% \aparag[Ethnic Revolts] Revolts appearing through the play of political
% events different from the R result on the event are considered as ethnic
% revolts.
% \bparag Ethnic revolts always proliferate in all provinces described in the
% political event, even if the owner of that province is not the country
% victim of the event.


\subsection{Independence of Revolted Principalities}\label{chRedep:Peace:Independence Revolt}
A \MAJ may give the independence to some groups of provinces (usually
separated from the mainland of the power) if all the provinces of the group he
owns (except at most one) have a revolt \Facemoins.  This announce is made
during the diplomatic phase.  See the rule
\ruleref{chSpecific:Peace:Independence Revolt}.


\subsection{Execution of the
  Monarch}\label{chRedep:ExecutionMonarchByRevolts}
\aparag If, at the moment of the Peace phase, half of all owned national
provinces (rounded up; they are shown on the map by a coat of arms of the
power) are in revolt, the regime of the country is overthrown and the monarch
is executed.
\aparag[Consequences]
All revolts present in the country are removed. A new monarch is
determined. The new monarch is considering as suffering a "Dynastic Crisis".
% Jym: useless, c'est deja l'effet DC...
% \bparag All the values of the new Monarch are obtained disregarding the
% values of the former ruler.
\bparag The \STAB is reduced by \textbf{2} levels%.
% Jym, memories of ruling during WoSS when I was FRA and HOL was overthrown.
and no \STAB improvement action is allowed this turn.
\bparag The \DTI is reduced by one (1 is the minimum).
\bparag 3 levels of Commercial fleets are reduced in the \CTZ box of the
country (chosen at random, even on other players' commercial fleets - NB: it
represents pillage and lost properties due to this really unstable situation!)
% \bparag The Stability is reduced by 4 levels.



\section{Land Military Looting}

Powers may receive supplementary income at this phase by looting provinces,
Colonies or Trading Posts of the other players (or minor countries) that their
units occupy.
\aparag A power can loot a province (or Colony/Trading Post) if he possesses
land units there (i.e. armies or detachments, or Turkish pashas) during this
phase (and so has been able to besiege the fortress).  Looting is allowed even
if the player does not in control the fortress of the province, Colony or
Trading Post.
\aparag[Adjustment of Already Existing Looting Markers]
Remove all \Facemoins looting markers, then adjust all \Faceplus looting
markers to their side \Facemoins, before to proceed to looting of the current
turn.
\aparag[Looting Income]
Unless there already was a looting marker in the province, the player that
loots receives immediately in his Royal Treasury a sum equal to the total
cumulated income of all looted provinces, not including exotic resources for
Colonies/Trading Posts.
\aparag[Placement of New Looting Markers]
When a looting takes place, one looting marker \Faceplus is placed on the map,
in the looted province, Colony or Trading Post.
\aparag[Looting in \ROTW]
If a land force has less than 2 \LD, it can only loot weakly a \COL/\TP.  Put
a \Facemoins looting marker in the province and the power gains only half of
the regular income.  If the power has at least 2 \LD, it may loot completely
the province.
\aparag[Consequences of Looting]
A side \Faceplus or side \Facemoins looting marker in a province (Colony,
Trading Post) cancels all land income of this province (or Colony, Trading
Post) and the exploitation of income from exotic resources.
\aparag[Elimination of Trading Posts]
A force occupying an enemy \TP at this phase may, instead of looting, destroy
it (player's choice).



\section{Mandatory Retreat of Sieges, \Presidios}

\aparag[Lack of Supply] Forces that have no LOS makes a mandatory redeployment
and have to test for attrition at {\bf +2} for this movement.
\aparag[Inadequate Siegeworks]
If a besieged fortress has no Siegework\faceplus placed over it, or has not
suffered a Breach or HW result during the course of the turn, the besieger has
to retreat to the closest friendly unbesieged city or port. This redeployment
causes an attrition roll for movement.
\aparag[Continued Sieges]
If a besieged fortress has a Siegework\faceplus, or a suffered from a Breach
or HW result during the course of the turn, the besieging player may either:
\bparag Decide to raise the siege and make a redeployement movement, with
attrition at {\bf -2};
\bparag Continue Siege, keeping only a Siegework\facemoins marker.

\aparag[Construction of \Presidios]\label{chRedep:BuildPresidios}
In some provinces, a Power may build or increase \Presidios if the province is
not owned but where its has military control, or is besieging the fortress
with adequate Siegeworks to continue the siege (even if the power decides to
raise the siege).
\bparag Provinces where \Presidios can be built are shown by a circled anchor
on the map.
\bparag A \Presidio is marked by a fortress counter. Its price is the same as
the fortress. The maximum level of a \Presidio is 3.  A power has to pay for
the Maintenance of \Presidio.  \Presidios may never be built or increased
during the Logistic Phase.
\bparag A power may only increase the fortress level of an existing \Presidios
at the exact same conditions for building it in the first place.
\bparag Ifever the power gains ownership of a province containing its own
\Presidio, the \Presidio is converted as a regular fortress for the province
(if higher than the current level), or dismantled.



\section{Return to Port}

\aparag Naval units being located in a sea zone have the choice to
\bparag retreat to a friendly, unblockaded port of their choice (not
necessarily the closest one). They roll for attrition in this movement.
\bparag stay at sea, by doing an attrition with a modifier of {\bf +2}, the
difficulty of the sea they are in, and the modifier due to being at sea of +3
or +6 depending of the distance to the nearest Sea Supply Source
(port/arsenal).
\aparag Then all privateers have to be repatriated to a port of their owner's
choice.
\aparag Pirates are never repatriated and remain permanently in their
\STZ/\CTZ of appearance until their destruction by a player.
\aparag No interception may ever occur during this phase, not even by
\Presidios or \StraitFort.



\section{Gold repatriation}\label{chRedep:GoldRepatriation}

\aparag[Before redeployment] For the income phase, see
\ruleref{chIncomes:GoldProduction}.
\bparag For the transportation for convoys, see
\ruleref{chIncomes:SpanishConvoys} and \ruleref{chMilitary:Convoys}.
\bparag Remark that gold that was recuperated during the military rounds
(through convoy reception or attack) is available as soon as it arrives in a
province on the European map of the power. Still, usually it is recorded
during this segment.
\aparag[Gold transportation]\label{chRedep:GoldTransportation} During the
redeployment phase, gold can be transported by land as long as it can find a
way of \TP, \COL or forts at most 12\MP apart crossing only friendly land. If
there is a land path to National Territory, it can not be stocked in \ROTW.
\bparag This is especially the case for \RUS exploiting mines in
\continent{Siberia}.
\bparag Gold stored in a \COL bordering the \seazone{Caspienne} (on the \ROTW
map) before the redeployment phase can be moved to a province on the other
side during this segment (and probably to their home country, either \POL or
\TUR).
\bparag Gold repatriated by land or stolen cannot be intercepted in any
way. If it reaches the main country, it is added to the royal treasury.


% Local Variables:
% fill-column: 78
% coding: utf-8-unix
% mode-require-final-newline: t
% mode: flyspell
% ispell-local-dictionary: "british"
% End:

