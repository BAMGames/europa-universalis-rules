\definechapterbackground{Redeployment}{redeployment}
\chapter{Redeployment}\label{chapter:Redep}

% After the military phase, some ``cleaning'' is required before the next
% turn. 

\section{Overview}

\aparag During the redeployment phase, lasting military affairs are
resolved. First, attacks by natives and privateers, then looting of
occupied provinces, extension of revolts and construction of \Presidios,
and lastly mandatory retreat of some troops and bringing \ROTW gold back
home.

\aparag[Sequence.]
\RedepDetails

\section{Attacks by Natives}\label{chRedep:Native Attack}
\begin{designnote}
  Ignore if using the experimental rules of Attacks during the military
  rounds.
\end{designnote}

\aparag Natives activated during the turn, as well as forces of \ROTW minor
countries may attack colonial establishments.
\bparag Natives always attack in each and every province where they have been
activated during the turn (whatever the cause of activation).
\bparag Troops of \ROTW minor countries inside \Areas owned by the minor
always attack establishments of countries against which they are at war.
\bparag Troops of \ROTW minor countries outside \Areas owned by the minor may
attack establishments of countries against which they are at war. The
controller of the minor decides whether they attack or not.

\aparag[Combined attacks]
\bparag If, in a given province, several forces attack, there are combined in
one and only one attack is resolved, totalling all the troops participating in
it.
\bparag This may includes natives of the province as well as one or more
(allied) \ROTW minors.
\bparag If there is only one leader in such a stack, he is considered as
commanding the attack. If there are two or more leaders, use normal rules to
determine who is leading the attack.
\bparag In case of a combined attack with minor troops and natives, the
controller of the minor may choose to attack with the minor troops only
(typically, in order to avoid malus if the natives were defeated this turn).

\aparag[Forces attacking]
\bparag In each province, sum up the number of \LD participating in the
attack (ignore any remaining \LDE).
\bparag Remember that each province of a given \Area holds the same number of
native \LD and that killing natives in one province does not change the number
of natives in other provinces of the \Area.
\bparag Example: There are 40\LD in \granderegionJapon. That means there are
40\LD in each of the four provinces of the \Area. Even if 30\LD have been
killed in \provinceEdo during a given turn, there are still 40\LD in
\provinceKyoto this turn.

\aparag[Resolving the attack]
\bparag Each attack is resolved by rolling one 
die on \ref{table:Pirates
  Natives Raids}. This die-roll is modified by:
\begin{modlist}
\item[+1] for each \LD in defence of the establishment (even besieged).
%\item[+2/+4] for each \ARMY\facemoins/\Faceplus in defence (even besieged).
\item[+N] level of the fortress.
\item[+M] Manoeuvre value of a \terme{land leader} in defence.
\item[-1] For each \LD of the attacking force (native or \ROTW minor).
\item[-1] For each foreign \COL or \TP in the same \Area (not belonging to the
  attacked country).
\item[-M] Manoeuvre value of an attacking leader (native or \ROTW minor).
\item[+3] If the Natives were defeated at least once in the province this turn
  without being routed, and there is at least 1\LD of natives in the attacking
  stack.
\item[+6]If the Natives were routed at least once in the province this turn,
  and there is at least 1\LD of natives in the attacking stack.
\end{modlist}
\bparag All rolls are simultaneous, that is an establishment destroyed by an
attack still provides a \bonus{-1} to attacks in the same \Area.

\aparag[Reading the result]
\bparag The result is read by cross-referencing the (modified) die roll with
the last two columns of the table.
\bparag The ``Pillages \TP/\COL'' column gives a number of losses on the
settlement. Ignore the \textddag\xspace and \textdag.
\bparag The ``Perm. losses on land'' column gives a number of losses on land
forces.
\bparag Both results (losses on settlements \textbf{and} on land forces) are
applied.

\aparag[Applying losses]
\bparag Losses on settlements: The establishment of the province losses as
many levels as indicated. If it reaches level 3, turn it \Facemoins. If it
reaches level 0, it is immediately destroyed. Exceeding losses are ignored.
\bparag Losses on land forces: as many \LD as indicated are lost. The losses
may also be applied to fortifications (loosing 1 level of fortress or 1 fort
instead of 1\LD). The controller chooses whether to loose troops or
fortifications, but as many loses as possible must be satisfied. Exceeding
loses are ignored.

\aparag[Exploited resources]
\bparag If an establishment loses levels and is still able to exploit all its
resources, nothing change.
\bparag If an establishment loses levels and is no more able to exploit all
its resources, it must free some of them (controller choice) until it has
sufficiently many levels to exploit the rest.
\bparag Resources freed this way will be attributed during the next
administrative phase to any establishment in the \Area with free levels to
exploit them, using the procedure for automatic competition in case of
disagreement between players.
\bparag Remember that it is possible to voluntarily free some resource as a
diplomatic announcement, typically in order to be allowed to exploit a more
valuable resource that has just been freed.

\aparag[End of activation]
\bparag Once the attack is resolved in a province, natives of this province
cease to be active.
\bparag Remark: Natives of provinces owned by \ROTW minors at war will be
automatically reactivated next turn unless peace with the minor is signed.

\GTtable{atendofturnattacks}

\begin{exemple}[Bantu raids]
  In the late game, \ref{pVI:Bantu Raids} occurs. As per event description,
  natives in 4 provinces are activated and attack with 6\LD and one
  leader. The 4 provinces are occupied as follows:
  \begin{itemize}
  \item \granderegionNyasa S.: \TP of level 2 of \HOL, no fortress.\\
    \HOL sends 2\LD and scores a victory against the Bantu (but no rout). 2\LD
    of natives are killed, the Dutch leader has a \Man of 3 and
    the Bantu has 4.
  \item \granderegionNatal N.: \TP of level 3 of \FRA, fortress of level 2.\\
    \FRA does not manage to send any troops to fight the natives ; the Bantu
    leader has 3 in \Man.
  \item \granderegionNatal S.: \TP of level 2 of \ANG, no fortress.\\
    \ANG sends 1\ARMY\Faceplus and routs the Bantu, killing 5\LD. The English
    leader has a \Man of 2 and the natives has 5.
  \item \granderegionCap E.: nothing.
  \end{itemize}
  The attacks are then resolved:
  \begin{itemize}
  \item \granderegionNyasa S.: the modifier is +2 (troops in defence) +3
    (\Man of the defending leader) -4 (4\LD attacking) -4
    (\Man of the native leader) +3 (native defeat) = 0. \HOL
    rolls 9 and loses 3 level of \TP (destroyed) and 2\LD (both killed).
  \item \granderegionNatal N.: the modifier is +2 (fortress in defence) -6 (6
    native \LD) -1 (presence of a foreign \TP in the \Area (the English one))
    -3 (native \Man) = -8. \FRA rolls 6-8 = -2 and loses 6 level
    of \TP (destroyed) and 8\LD or level of fortress (destroyed).
  \item \granderegionNatal S.: the modifier is +4 (\LD in defence) +2
    (\Man) -1 (native \LD) -1 (French \TP) -5 (\Man)
    +6 (rout) = +5. \ANG rolls 7+5 = 12 and loses 1 level of \TP (1 stays) and
    1\LD (3 remain).
  \end{itemize}
  Remark : when sending troops to fight off natives, don't do it
  half-heartily. Otherwise, you may lose your troops in addition to your
  establishment\ldots
\end{exemple}

\section{Attacks by Pirates \& Privateers}\label{chRedep:Corsair Attack}
\begin{designnote}
  Ignore if using the experimental rules of Attacks during the military
  rounds.
\end{designnote}

\begin{todo}[Should move in Military]
\aparag Pirates and privateers attack commercial fleets to attempt to decrease
their levels, and possibly to capture gold repatriated to Europe by these
fleets.
\bparag[Pirates] Pirates appear as explained in \ruleref{chEvents:Piracy} and
they remain until completely eliminated. They are active every turn.
\bparag[Privateers]
\label{chRedep:CorsairAttack Privateer}
Privateers are raised by Major Powers (see \ruleref{chLogistic:Recruiting
  Privateers}), or are in the basic forces of some minor powers (the
\pays{chevaliers} and the \Barbaresques). They must go out at sea on the first
or second round or they will have no effect.
\bparag Beginning with the third round, they stay in the sea they were placed
in, and will be able to attack one \STZ or \CTZ in this sea or an adjacent
sea. The specific \STZ or \CTZ has to be annouced at that time.  
\end{todo}

\begin{todo}[Should move in Military]
\aparag[Raiding Fleets with Privateer Admirals]\label{chSpecific:Raiding
  Privateer admirals}
Privateer, or an Admiral with Privateer capacity, may lead one \corsaire he
starts the turn with.  He may lead it in the same stack as naval forces not
containing a \FLEET.  The \corsaire does not count for attrition, nor in
battle.  The stack acts both as regular naval force (and can attack, blockade,
and so on), and a Privateer stack (other players may attempt to suppress the
\corsaire counter).  The \corsaire does not count for attrition, nor in battle
(nor is affected by battles).  The stack may split at any time (for instance
if the naval force has to retreat in a port), and the leader chose which stack
he stays with.
\bparag As an exception to \ruleref{chRedep:CorsairAttack Privateer}, a
\corsaire led by Privateer or Privateer-Admiral may move after the second
round, and has to remain in place only on the last round (the player telling
at the beginning of this round which \CTZ\\STZ will be attacked if there are
several of them). However, it still has to be at sea at the end of every round
after the first, else (if at port), it cannot leave again for the rest of the
turn and will not attack commercial fleets (or loot) this turn.
\bparag Note that the leader may move as he prefers but can only lead the one
\corsaire he starts the turn with (even leaving it then coming back), or naval
forces.
\end{todo}

%% PB 07/2008 : TBD ??  J'ai essaye de simplifier cette partie avec la version
%% ci-dessus, pour mieux
% représenter aussi ce qu'on veut.
%
% \corsaire be moved every round to another \corsaire or naval stack, excepted
% on the last round where it can not join a new \corsaire counter.
% \aparag[Raiding Fleets with Privateer Admirals]\label{chSpecific:Raiding
% Privateer admirals}
% \textbf{[To be deplaced in military]} Naval forces not containing a \FLEET
% that are led by a Privateer, or and Admiral with Privateer capacity may move
% with one \corsaire counter in their stack if it begins even after the 2nd
% round.  Naval forces not containing a \FLEET that are led by a Privateer, or
% and Admiral with Privateer capacity, may be announced to be conducting
% privateer attack of trading fleets (and potentially looting) at the latest
% at the end of the second round if at sea.
% \bparag In that case, replace one \ND for a \corsaire\facemoins, or two \ND
% for a \corsaire\faceplus.  AUTRE VERSION: sans transformation de pions
% \bparag In this case, they may not leave anymore the sea zones adjacent to
% the \CTZ or \CTZ they are attacking, for the remaining of the Military
% Phase.
% \bparag The force is hereafter for the rest of the Military Phase dealed
% with as being \corsaire\facemoins if they have between 2\de and 1\ND2\de,
% and \corsaire\faceplus if they have 2\ND (or \NGD) or more, with no
% capacity as a naval force. It tests for attrition only if decides to move to
% another sea zone; it can not attack, intercept of be attacked or intercepted
% (except to attack Convoys as per Privateer rules).
% \bparag If a roll to reduce Privateer succeed, the stack suffers 50\%
% attrition.  If it losed forces so as to have less than 2\de, the stack
% becomes a regular naval force at the beginning of its next movement segment.

\begin{todo}[Should move in Military]
\aparag[Looting by Pirates or Privateers] Pirates and Privateers may try to
loot Trading Posts or Colonies, and also enemy provinces for privateers, that
are a province bordered by the sea they are in.
\bparag Looted provinces, Colonies or Trading Posts may belong to minor
countries or to players. For privateers to be allowed to loot, it is necessary
that a state of war exists between the owner of the privateer unit and the
owner of the looted province. Overseas Wars are enough to loot \TP or \COL,
but not European provinces.
\bparag \textit{Exceptions:} Looting of European provinces by the
\Barbaresques is permitted, as well as looting in their provinces. Sea Hounds
may loot European provinces also, see \ruleref{chSpecific:England:Sea Hounds}.
\bparag Pirate may loot following \ruleref{chEvents:PiracyTarget}. After a
turn of looting, non-eliminated pirates go back to the \STZ they belong to.
\bparag The privateer intending on looting is placed in the concerned
province, Trading Post or Colony. They have to disembark during any round
except the last from the sea zone they are operating in.
\bparag If privateer/pirate unit is still present at the Redeployment phase,
it loots. Looting privateer/pirate are unaffected by forces or battles (except
that those forces may attempt to destroy them during the military phases).
\bparag A maximum of 1 privateer/pirate unit (any side up) can loot the same
Colony/province in the same turn. Privateers/Pirates looting a province or
\COL/\TP can not attack at the same turn the \CTZ/\STZ.
\end{todo}

\subsection{At sea}
\begin{designnote}
  Ignore if using the experimental rules of Attacks during the military
  rounds.
\end{designnote}

\aparag[Naval actions of Pirates and Privateers] In each \STZ/\CTZ where
\corsaire are active, an attack on \TradeFLEET occurs.
\bparag First, \pays{pirates} \corsaire attack all \TradeFLEET in the zone.
\bparag Then, each alliance resolve the attacks of its privateers, in order of
initiative.
\bparag \corsaire of different alliances that have the exact same targets
(same \TradeFLEET in a given \STZ/\CTZ) attack together as if they were
allied.
\bparag Especially, all \corsaire of \Barbaresques are considered as one
alliance, acting at the initiative of \TUR. They are automatically allied with
any \corsaire of \TUR if they have the exact same targets.
\bparag The \corsaire of \paysChevaliers acts at the initiative of its
diplomatic patron, or of the \SDCF if neutral. It is automatically allied with
any and all \corsaire of any and all countries having the same target (that
is, only \TUR).

\aparag[Targets of piracy]
\bparag \corsaire of \pays{pirates} target all \TradeFLEET in the \STZ/\CTZ
they are.
\bparag Note that some seas may belong to several \STZ or \CTZ (especially in
Europe). However, \corsaire of \pays{pirates} are specifically created in one
of them (\emph{e.g.} \ref{eco:Looting and insecurity} creates them in the \CTZ
of the country rolling it).
\bparag \corsaire of \Barbaresques attack all christian \TradeFLEET in the
\STZ/\CTZ they are. \TUR must declare when moving them which \STZ/\CTZ they'll
attack
\bparag \corsaire of other countries attack all \TradeFLEET of countries
against which they are at war (including overseas wars) in their
\STZ/\CTZ. Owner must declare which \STZ/\CTZ is targeted when moving a
\corsaire.
\bparag \TUR may attack Christian \TradeFLEET while
\xnameref{chSpecific:Balkans} is active. \TUR must declare upon moving its
\corsaire which \CTZ/\STZ and \TradeFLEET are targeted.

\begin{exemple}[Combined attack]
  Note that only \TradeFLEET in the current \STZ/\CTZ may be targeted. Thus,
  \emph{e.g.}, if \VEN is the only country with \TradeFLEET in one zone, a
  \corsaire of \Barbaresques will only target \VEN in that zone even if it
  would target other \TradeFLEET elsewhere. In this case, the \corsaire could
  combine with a \corsaire of \TUR being at war against \VEN only, even if in
  another zone the presence of a non-belligerent \TradeFLEET of \FRA (targeted
  by \Barbaresques but not by \TUR) prevents the combined attack.
\end{exemple}

\aparag[Resolving the attack]
\bparag Each attack is resolved by rolling one die on \ref{table:Pirates
  Natives Raids}. This die-roll is modified by:
\begin{modlist}
\item[+2] if the \corsaire is not exactly in the sea zone of the \STZ/\CTZ
  (the zone where the symbol is located).
\item[+3] if a lone \corsaire\facemoins is attacking.
\item[+1] per side of targeted \TradeFLEET.
% or \FLEET  (a Convoy counts as 2 sides)
% [convoys are attacked during the rounds, not here]
\item[+1] If one or more \ND defending (see below) and no \FLEET.
\item[+2/+4] per \FLEET\facemoins/\Faceplus defending (see below).
\item[+M] \Man of one defending \LeaderA (see below, count only one defending
  \LeaderA per attack).
\item[-1] per year in the sea zone (max. \bonus{-3}) (see below).
\item[-M] \Man of one \corsaire \LeaderA (count only one attacking \LeaderA
  per attack). %(\textonehalf for land raids in Europe)
\item[+1] if a naval battle occurred in the sea where the \corsaire is located
  during this turn. %(not for land raids)
\item[-2] if the \corsaire of \pays{chevaliers} is in the attack and there is
  a Christian port on \seazone{Egee} or \seazone{Mediterranee E}.
\end{modlist}

\aparag[Years at sea]
\bparag Each military round is named by letter ('S' or 'W') and a number (from
0 to 5).
\bparag Each round with a different number in its name, at the end of which
the \corsaire is in the zone, is considered as one ``year at sea'' and gives a
\bonus{-1} to the roll.
\bparag Two rounds with the same number in their name are considered as only 1
year and give only a bonus of \bonus{-1}.
\begin{exemple}[Years at sea]
  A \corsaire stays in the same zone during the rounds 'S4', 'W4' and
  'S5'. There are only two years at sea ('4' and '5'), hence a bonus of
  \bonus{-2}. If it is here during the rounds 'S2', 'W3', 'S4', 'W4' and 'S5',
  that is 4 years at sea, thus the maximum bonus of \bonus{-3}.
\end{exemple}

\aparag[Defending naval force] Any naval stack in any zone of a \STZ/\CTZ may
be declared as a ``defending force'' by its controller if it is allowed to
fight at least one of the \corsaire in the attack.


\aparag[Reading the result]
\bparag The result is read by cross-referencing the (modified) die roll with
the first three columns of the table.
\bparag The ``\TradeFLEET\faceplus'' column gives the number of levels
temporarily lost by an eventual \TradeFLEET\faceplus in the zone.
\bparag The ``\TradeFLEET\facemoins'' column gives the number of levels
temporarily lost by \emph{each} \TradeFLEET\facemoins in the zone.
\bparag The ``perm. loss'' column gives the number of levels
\textbf{permanently} lost (one per \textetoile) by some \TradeFLEET in the
zone.
\bparag \ND may be lost instead of levels of \TradeFLEET.
\bparag All results are applied.


\aparag[Applying losses]
\bparag First, \textbf{each} targeted \TradeFLEET\facemoins decreases its
\emph{Current} level (see~\ref{chAdministration:Commercial Fleet Adjustment})
by the number of loss obtained for \TradeFLEET\facemoins.
\bparag Then, if there is a \TradeFLEET\faceplus in the \STZ/\CTZ, decrease
its \emph{Current} level by the number of loss obtained for
\TradeFLEET\faceplus.
\bparag Instead of loosing \TradeFLEET levels, players may choose to loose \ND
(or \NGD) of defending naval stacks. The choice is made by the controller of
each defending stack on a 1 for 1 basis. It is possible to transform any
number of levels lost in \ND (or \NGD) lost (but no more than the total number
of \ND (or \NGD) in the defending stacks). Several defending stacks may thus
protect the same or different \TradeFLEET.
\bparag Lastly, each \textetoile obtained decrease by 1 the \textbf{maximal}
level of the largest targeted \TradeFLEET (the one with the largest maximal
level). Apply these \textetoile one by one. In case of equality, the
controller of the \corsaire chooses, at random if its neutral (\emph{e.g.}
\pays{pirates}) or in case of disagreement (allied \corsaire).
% Avoid gamey tactic with Pirates in Asia:
\bparag Exception: In the \ROTW, \textetoile are always applied to \TradeFLEET
of major countries or of countries at war (minors at peace and close to their
base manage to repulse local piracy).
\bparag This decrease of the maximal level cannot be transformed into lost
\ND. It does not decrease the current level unless the maximal level becomes
higher than the current (in other words, result ``5 \textetoile\textetoile''
on the first line means ``2 permanent losses and 3 temporarily losses'').

\begin{playtip}
  Remember that temporarily lost levels return one per turn per \TradeFLEET
  automatically, but the process can be sped up with \TFI actions. Permanent
  losses, on the other hand, are gone for good and \TFI actions must be use to
  regain these.

  However, monopolies (both for income and \VPs) as well as attribution of
  Trade centres are computed according to the \textbf{current} levels. Thus,
  when there is a strong trade competition, a few temporarily losses may
  change the owner of the centres for a couple of turns, with dramatic
  influence on incomes which may be crucial during wars.

  End of period \VPs are computed based on maximal levels in order to avoid a
  last moment backstab that would causes an important change in \VPs and
  create an ``end of period'' effect.
\end{playtip}

\aparag[Income of Privateers] Each level eliminated (temporarily or
permanently) by a \corsaire brings an income to its controller equal to the
small number printed in the \STZ (the presence income). \corsaire of minor
countries give no income (even if \VASSAL).
\bparag This ``privateers income'' is recorded in \lignebudget{Pillages,
  privateers}.
\bparag In case of stacks with \corsaire from several powers, this income is
equally divided between the powers (including minor ones), dropping any
fraction.

\begin{exemple}[Resolving the attack] During period \period{II},
  \leaderwithdata{Barbaros2} sails out of \villeAlger with a
  \corsaire\faceplus into \seazone{Lion} where he is joined by a
  \corsaire\facemoins of \paysTunisie. During the turn, \leaderwithdata{A
  Doria} leads a Genoese \FLEET\facemoins at sea and manage to hunt down the
  Tunisian \corsaire. At the end of the turn, the situation is as follows:
  \corsaire\faceplus of \paysAlgerie lead by \leader{Barbaros2},
  \FLEET\facemoins of \paysGenes lead by \leader{A Doria},
  \TradeFLEET\faceplus of level 4 of \FRA, \TradeFLEET\facemoins of level 3 of
  \paysHollande, \TradeFLEET\facemoins of level 3 of \paysGenes,
  \TradeFLEET\facemoins of level 2 of \SPA, \TradeFLEET\facemoins of level 2
  of \VEN and \TradeFLEET\facemoins of level 1 of \TUR.

  The \corsaire stayed at sea for 3 years. Now, it is time to check the result
  of \leader{Barbaros2} relentless attacks on the Christian trade. There are 6
  targeted sides of \TradeFLEET (2 of \FRA, 1 of each \paysHollande,
  \paysGenes, \HIS and \VEN; the Turkish \TradeFLEET is not targeted as
  \Barbaresques only target Christian \TradeFLEET). Thus, the total DRM is +6
  (targeted sides) +2 (\FLEET\facemoins in defence) +5 (\Man of \leader{A
    Doria}) -3 (years at sea) -5 (\Man of \leader{Barbaros2}) = +5.

  \TUR rolls one die and obtains 2 + 5 = 7. Thus, the \TradeFLEET\faceplus
  loses 2 level and \textbf{each} \TradeFLEET\facemoins loses 1 level. All
  these loses are temporarily, and there are no permanent loses. \HIS (the
  diplomatic Patron of \paysGenes) decides to lose 2 \NGD on the Genoese
  \FLEET in order to save the Spanish and Genoese \TradeFLEET (one level
  each).

  So, after the attack is resolved, the situation is as follows (with maximum
  level in parenthesis): \TradeFLEET\facemoins of level 2 (4) of \FRA,
  \TradeFLEET\facemoins of level 3 of \paysGenes, \TradeFLEET\facemoins of
  level 2 (3) of \paysHollande, \TradeFLEET\facemoins of level 2 of \HIS,
  \TradeFLEET\facemoins of level 1 (2) of \VEN and \TradeFLEET\facemoins of
  level 1 of \TUR.

  \FRA will be the most hurt by this attack as it will lose the monopoly
  (income and \VPs) of the \STZ for 2 turns.

  4 levels were actually lost in the attack. Thus, the \corsaire generate an
  income of 4 (levels lost) $\times$ 2 (presence income of the zone) =
  8\ducats. However, these are ``gained'' by a minor country (\paysAlgerie)
  and thus forgotten.
  
  \smallskip

  If the Genoese \FLEET was not here, then the DRM would have been -2,
  resulting in a 2-2=0 causing 5 levels lost on the French \TradeFLEET and 2
  on all other, and 2 permanent losses (the first must be on the French
  \TradeFLEET, the second on either \FRA, \paysGenes or \paysHollande, at the
  choice of \TUR (say, \paysGenes)). Thus, the situation at end would be:
  \TradeFLEET\facemoins of level 0 (3) of \FRA, \TradeFLEET\facemoins of level
  1 (2) of \paysGenes, \TradeFLEET\facemoins of level 1 (3) of \paysHollande,
  \TradeFLEET\facemoins of level 0 (2) of \HIS, \TradeFLEET\facemoins of level
  0 (2) of \VEN and \TradeFLEET\facemoins of level 1 of \TUR. A much more
  devastating result.
\end{exemple}

\aparag[Privateers and Trade Centres]
\bparag Privateer attacks may cause temporary loss of incomes of the Trade
Centre containing the \STZ/\CTZ they are located.
\bparag For each \textetoile obtained by a \corsaire allied with the country
owning the Trade Centre of the sea zone (at the moment of the attack),
decrease the income of the Centre by 10\ducats for the next turn (only).
\bparag Exception: The \corsaire of \pays{Chevaliers} does not decrease the
income of the Mediterranean Centre if it is owned by a Christian country.
\bparag Exception: The \corsaire of \Barbaresques do not decrease the income
of the Mediterranean Centre if it is owned by \TUR.

\begin{designnote}
  Trade and piracy don't go well together. Even your own merchants will become
  suspicious of the privateers respecting their target or turning to piracy,
  thus decreasing the overall trade in the seas and the income of the
  Centre. This is especially true when you are dominating the trade. Since
  most of the trade ships are yours, even your privateers may lack legitimate
  targets and start attacking anything at sight.

  In other words, to maintain a commercial domination, peace is required and
  frequent attacks on trade, even on the trade of someone else, will make it
  more dangerous and less profitable.
\end{designnote}

\begin{exemple}
  In period \period{V}, a \HOL-\ANG-\SPA alliance battles \FRA. The Atlantic
  Centre is in \HOL. \ANG chooses to send a \corsaire in \stz{Atlantique N} in
  a attempt to dominate the trade with the new World and causes 2 \textetoile
  on the French \TradeFLEET there. Meanwhile, \SPA sends a \corsaire in
  \ctz{France} and causes another \textetoile on the French \TradeFLEET
  here. All in all, 3 \textetoile have been obtained in zones of the Atlantic
  Centre by allies of the owner, thus decreasing its income by 30\ducats (to
  70\ducats) for the next turn.

  Note that eventual \textetoile in Mediterranean seas would have cause loss
  on the Mediterranean Trade Centre if it is also owned by a member of this
  alliance (but not if it is owned by \FRA or \TUR).
\end{exemple}

\aparag[Privateer and Peaces]
\bparag During Overseas wars (only), each \TradeFLEET\Faceplus reduced to
current level 0 or 1 counts toward peace.
\bparag See~\ref{chPeace:Privateer Effect} for details.

\begin{todo}[Should move in Military]
\aparag[Minor countries against Piracy]
\bparag Minor countries at war can use their naval forces against \corsaire in
\STZ or \CTZ where they have a \TradeFLEET of their own (only).
\bparag Christian Minor countries whose \TradeFLEET are attacked by \corsaire
of \Barbaresques may also use their naval forces to fight against those
privateers (Patron's choice to move their forces), even if at peace.
\bparag Remember that minors at peace have only Passive campaigns each turn,
thus the Patron must pay for moving (when entering the zone where \corsaire
are located).
\bparag Against \pays{pirates}, minor countries at peace fight in an abstract
way in the \STZ where there are no major country \TradeFLEET (usually in
\continent{Asia} in the early game): each round, roll 1d10 for each \STZ with
\pays{pirates} \corsaire and no major \TradeFLEET and add 1 for each side of
commercial fleet of a minor country.  If the result is 8 or higher, one
\corsaire\facemoins is eliminated.
\end{todo}

\subsection{On land}
\begin{designnote}
  Ignore if using the experimental rules of Attacks during the military
  rounds.
\end{designnote}

\aparag[Land actions of Pirates and Privateers] In each province where
\corsaire are active, an attack occurs.
\bparag First, resolve \pays{pirates} \corsaire attacks.
\bparag Then, each alliance resolve the attacks of its privateers, in order of
initiative.

\aparag[Resolving the attack]
\bparag Each attack is resolved by rolling one die on \ref{table:Pirates
  Natives Raids}. This die-roll is modified by:
\begin{modlist}
\item[+3] if a lone \corsaire\facemoins is attacking.
%\item[+2/+4] per \ARMY\facemoins/\Faceplus defending
\item[+1] Per full \LD in the province (including militia and \LD in \ARMY).
\item[+M] \Man of a defending \LeaderG/\LeaderC/\LeaderGov.
\item[-1] per year in province (max. -3) (computed as at
  sea). %(NA on Convoy attacks)
\item[-M] \Man of a \corsaire admiral (\textonehalf\ for land raids in
  Europe).
% \item[-2] \pays{chevaliers} with Christian port on \seazone{Egee} or
%   \seazone{Mediterranee E}
\item[+N] Twice the level of the fortress, +1 for fort.
\end{modlist}

\aparag[Reading and applying the result]
\bparag The result is read by cross-referencing the (modified) die roll with
the ``Pillages \TP/\COL'' column on the table, looking only the \textdag\ or
\textddag.
\bparag If a \textddag\ is obtained, the province is looted: place a
\PILLAGE\faceplus. The \corsaire owner receives the total income of the
province/settlement (including income of exploited resources at their current
price).
\bparag If a \textdag\ is obtained, the province is weakly looted: place a
\PILLAGE\facemoins. The \corsaire owner receives half the total income of
the province/settlement (including income of exploited resources at their
current price).
\bparag \pays{pirates} receives no income for looting.
\bparag This income is recorded in \lignebudget{Pillages, privateers}.
\bparag Note that existing \PILLAGE marker neither prevent new one nor prevent
the \corsaire from getting money.
\bparag There is neither loss of land forces due to the looting (opposite to
Natives attacks) nor protection by sacrificing forces in the province
(opposite to attacks at sea).

\aparag[Seizing gold]
\bparag If either a \textdag\ or \textddag\ was obtained in an attack against
a \COL/\TP with Gold stored in it, all the gold is stolen.
\bparag Remove all the stored gold from the establishment.
\bparag The owner of the \corsaire (nobody if \pays{pirates} or a minor)
records that amount in \lignebudget{Gold from ROTW and Convoys}.

\aparag[Reducing Pillages]
\bparag \PILLAGE placed by land raids are considered simultaneous with
military looting (\ref{chRedep:Looting}).
\bparag Especially, they are never reduced the turn they appear (contrary to
\PILLAGE caused by attrition during military rounds).

\section{\REVOLT and \REBELLION}\label{chRedep:Revolts}
\subsection{Revolts in minor countries}
\aparag \REVOLT/\REBELLION in inactive minor countries are automatically
removed without any roll.
\bparag \REVOLT/\REBELLION in active minor countries must be fought using the
normal rules.

\subsection{Loss of \STAB due to Revolts}
\label{chRedep:Revolts Stability}
\aparag If one or more \REVOLT/\REBELLION still exist in a country, this
country loses \STAB.
\bparag For each \REVOLT/\REBELLION\faceplus, it loses 1 \STAB level.
\bparag For all \REVOLT/\REBELLION\facemoins, it loses only one additional
\STAB level (only one, not one for \REVOLT and one for \REBELLION).
\aparag However, the maximum a country may lose from \REVOLT/\REBELLION of all
types is 3 \STAB levels. Ignore excess losses.

\aparag \REVOLT/\REBELLION in minor countries cause loss of \STAB as if they
were in their diplomatic patron.
\bparag Especially, if there is a \REVOLT\Facemoins in a country and one in
one of its minors, the country only losses 1 \STAB.
\bparag Similarly, the maximum loss for all \REVOLT in a country and all its
minor is 3, not 3 per country (the major and each minor).

\begin{designnote}[Revolts in minors]
  Since \REVOLT in inactive minor countries are removed before the loss of
  \STAB occur, the easiest way to get ride of \REVOLT in your minor allies is
  simply to keep them at peace. This prevent abusing minors by sending all
  their troops to a foreign war instead of fighting local troubles, and
  letting the situation of the minor deteriorate without end.
\end{designnote}

\subsection{Extension of \REVOLT}
\label{chRedep:Extension Revolts}
\aparag[Who extend?] After \STAB losses, \REVOLT and \REBELLION extend. Adjust
all the markers simultaneously:
\bparag each \REVOLT/\REBELLION\facemoins becomes a
\REVOLT/\REBELLION\faceplus;
\bparag each \REVOLT/\REBELLION\faceplus generates a
\REVOLT/\REBELLION\facemoins.
\bparag Unbesieged cities in revolt/rebellion and revolted/rebelled troops
generate a \REVOLT/\REBELLION\facemoins in their province if there is neither
\REVOLT nor \REBELLION counter in it.
\bparag Note that this apply only for cities controlled by or troops with
counter baring the name ``Rebellion''. When specific minor entities exists to
depict rebellion (typically, \pays{royalistes} or \pays{huguenots}), they do
not generate \REBELLION.

\aparag[Where to extend?]
\bparag When extending, \REVOLT only create \REVOLT and \REBELLION only create
\REBELLION.
\bparag When a \REVOLT/\REBELLION\Faceplus extends, the new counter is placed 
in the same or adjacent province. If there are two \REVOLT\Faceplus in the
same province, the two new \REVOLT\Facemoins may appear in separate
provinces.
\bparag This province must belong to the victim country in case of \REVOLT.
\bparag This province must belong to the region allowed by the event in case
of \REBELLION (the victim country if no region is specified).
\bparag The choice is made by the player controlling the
\REVOLT/\REBELLION. If none was specified, roll one country on the \REVOLT
table of the current period as controller for this turn (reroll until the
result is not currently allied with the victim).
% Jym, following JC.
\bparag \textbf{[TBD]} \REVOLT/\REBELLION in \regionIrlande may extend this
way across \seazoneMan into \ANG (and reciprocally): \provinceUlster is
considered adjacent to \provinceAlba, \provinceAyr, \provinceGalloway and
\provinceCumberland ; \provinceBrega is considered adjacent to
\provinceCumberland, \provinceLancashire and \provinceCymru ;
\provinceLeinster is considered adjacent to \provinceCymru and
\provinceCornwall.
\bparag If the extension of a \REVOLT/\REBELLION\faceplus is not possible (due
to overstacking of counters), a Revolt or Rebellion \LD is placed in the same
province (immediately merged with existing Revolted or Rebelled troops into an
\ARMY\Facemoins or \ARMY\Faceplus using usual rules for conversion).
\bparag If there are two \REVOLT\Facemoins (or \REBELLION\Facemoins) in the
same province, they are immediately merged in a \REVOLT/\REBELLION\Faceplus
(before checking stacking).
\bparag Remember, that there can be at most 2 \REVOLT/\REBELLION markers
stacked in each province.

\begin{exemple}
  Suppose that there is a \REVOLT\Faceplus in French \provinceBearn, a
  \REVOLT\Faceplus in \provinceSavoia (a French ally, active in an Italian
  war), a \REVOLT\Facemoins in \provinceBerry, another in \provincePfalz
  (another active French ally) and a last \REVOLT\Faceplus in
  \provinceLorraine (an inactive French minor). Note that this is a very
  unlikely situation that almost never happens out of examples.

  First, since \paysLorraine is inactive, the \REVOLT there is automatically
  removed without need for rolls or anything. Then \FRA loses 3 \STAB: 1 for
  each \REVOLT\Faceplus (in \provinceBearn and \provinceSavoia) and 1 for all
  the \REVOLT\Facemoins (in \provinceBerry and \provincePfalz).

  Next, \REVOLT extend. All the \REVOLT extend simultaneously, that is newly
  created \REVOLT do not extend in the phase they were created (otherwise,
  you're trapped in an infinite loop). Both \REVOLT\Facemoins become
  \REVOLT\Faceplus. The two \REVOLT\Faceplus create new
  \REVOLT\Facemoins. Since the \REVOLT\Faceplus in \provinceBearn is in \FRA,
  it may only extend in \FRA. Thus, the new \REVOLT\Facemoins may not be
  created in Spanish \provinceVizcaya, \provinceNavarra, \provincePirineos nor
  \provinceRosselo. It may, however, extend in French \provinceBearn (creating
  a second \REVOLT here), \provinceGuyenne or \provinceLanguedoc. The
  controller of the \REVOLT chooses to create it in \provinceBearn, hoping
  that the mountain will give some protection to it and that it will take
  longer to crush it (rather than risking an extension to \provinceLanguedoc
  that would cause more money loss but will likely be easily crushed next
  turn). The \REVOLT\Faceplus in \provinceSavoia may only extend in
  \paysSavoie, hence either \provinceSavoia, \provinceBresse or
  \provinceNice. It may not go in Spanish \provinceLombardia nor even in the
  French provinces even if \paysSavoie is a French ally.

  \smallskip

  Note that for \FRA, the best way to get ride of this dire situation is
  probably to sign peace, thus making its minors inactive (that will remove
  two \REVOLT\Faceplus and one \REVOLT\Facemoins) and freeing its armies from
  the front line to crush its peasants. However, French enemies are not likely
  to give an easy peace as they may want to take advantage of the
  troubles\ldots
\end{exemple}

\subsection{Revolts and fortresses}
% Jym (05/2013) : following Pierre's notes from 2007.
\aparag If a \REVOLT (not a \REBELLION) controls a fortress, reduce the
fortress to level 2 (turn 39 or earlier) or 3 (turn 40 or later) if it is
higher.
\bparag Each level lost that way gives one Revolt \LD which stays inside the
fortress and is immediately merged with existing troops. Exceptionally, the
troops in the fortress may exceed the usual fortress capacity (of 1\LD per
level).

\subsection{Independence of Revolted Principalities}\label{chRedep:Peace:Independence Revolt}
\aparag A \MAJ may give the independence to some groups of provinces if all
the provinces of the group he owns (except at most one) have a \REVOLT or a
\REBELLION. This announce is made during the diplomatic
phase. See~\ref{chSpecific:Peace:Independence Revolt} for the precise
conditions.


\subsection{Execution of the Monarch}\label{chRedep:Execution Monarch by
  Revolts}
\aparag If half of all owned national provinces (rounded up) are in revolt
(either a \REVOLT or \REBELLION counter or control of the city), the regime of
the country is overthrown. The tyrant is executed and a new benevolent monarch
accedes the throne.

\aparag[Consequences] Unless specified by an event, execution of the
Monarch has the following effects:
\bparag All revolts present in the country are removed: all \REVOLT and
\REBELLION counters and troops are removed, and fortresses they control are
given back to their legitimate owner.
\bparag A new monarch is immediately determined using normal rules. His first
turn of reign is considered to be the current one. The new
monarch is rolled as after a "Dynastic Crisis".
\bparag The \STAB is reduced by \textbf{2} levels %.
% Jym, memories of ruling during WoSS when I was FRA and HOL was overthrown.
and no \STAB improvement action is allowed this turn.
\bparag The \DTI is reduced by one (1 is the minimum).
\bparag 3 levels of \TradeFLEET are reduced in the \CTZ of the country (chosen
at random among all the levels present, even on other countries' \TradeFLEET)

\begin{designnote}
  This represents pillage and lost properties due to this really unstable
  situation!
\end{designnote}

\aparag[Execution and Civil wars]
\bparag In most events creating \REBELLION, execution of the monarch ends the
event in a loyalist defeat. Often, the precise effect is different from the
ones described above (\STAB loss, new monarch, \ldots) Check the precise
description of the event.

\begin{playtip}
  Beware that execution of the monarch happens after extension of
  \REVOLT. Thus, a seemingly controlled situation may get out of hands because
  of poor prevision of the extension. This is especially true for small
  countries with few provinces.

  Beware also that execution of the monarch happens after loss of \STAB. This
  usually leaves the country in a very bad state. Since the execution prevents
  \STAB improvement this turn, it is often a bad idea to use it as an ``easy''
  way to remove \REVOLT, especially during wars (when troops may seem more
  useful on the front line).
\end{playtip}


\section{Land Military Looting}\label{chRedep:Looting}
% Powers may receive supplementary income at this phase by looting provinces,
% Colonies or Trading Posts of the other players (or minor countries) that their
% units occupy.
% \aparag A power can loot a province (or Colony/Trading Post) if he possesses
% land units there (i.e. armies or detachments, or Turkish pashas) during this
% phase (and so has been able to besiege the fortress).  Looting is allowed even
% if the player does not in control the fortress of the province, Colony or
% Trading Post.

\aparag[Adjustment of Already Existing \PILLAGE Markers] In each
province where there is at least one \PILLAGE:
\bparag Remove one \PILLAGE\Facemoins if there is one.
\bparag Otherwise, flip one \PILLAGE\Faceplus to its \Facemoins side.
\bparag Exception: \PILLAGE markers put this turn due to land raids of
\corsaire are not touched. They will only be reduced next turn.

\begin{designnote}
  Each side of \PILLAGE represent one level, since there may be up to 2
  markers in any given province, there may be up to 4 levels of
  looting. During adjustment of the markers, the ``looting level'' of each
  province is reduced by 1. Simply be cautious not to flip a \PILLAGE\Faceplus
  after removing a \PILLAGE\Facemoins in the same province as this would
  reduce the level by 2.

  \smallskip

  \PILLAGE obtained because of attrition during the rounds may be immediately
  removed. Thus, a small ``tax'' on the local farms will have no
  impact. However, if the war stays in the same area for long,
  \PILLAGE\Faceplus may appear due to attrition and will cause a loss of
  income the next turn.

  \smallskip

  \PILLAGE obtained by \corsaire are not reduced the turn they appear. That
  is, they are considered to be placed at the same time as military looting,
  just after adjustment, but are resolved together with other actions of
  \corsaire.
\end{designnote}

\aparag[Looting]
\bparag Each stack in a enemy province may loot if it has sufficiently many
troop to besiege the province (either the province is already controlled, or
1\LD per level of fortress).
\bparag Looting is never mandatory.
\bparag The decision to loot or not is taken by the controller of the
stack. The choice may vary from one province to another (it is possible to
loot one province and decline the possibility in another province).
\bparag Looting are resolved in decreasing order of initiative: the alliance
with the higher initiative resolves all its looting, then the next and so on.

\aparag[Looting and \PILLAGE]
\bparag Place a \PILLAGE\Faceplus in each looted province.
% Jym, 09/2013:
% change "< 2LD" to "no A".
% reason: A has artillery and is more expensive in the ROTW.
\bparag Exception: in the \ROTW, if the looting stack contains no \ARMY
counter, only place a \PILLAGE\Facemoins (if another already exists here,
immediately merge both into a \PILLAGE\Faceplus).
\bparag If there are more than two \PILLAGE markers in any province, remove
the smallest one in each of these provinces (this can happen when looting a
province which has already a \PILLAGE\Faceplus and a \PILLAGE\Facemoins, or
two \PILLAGE\Faceplus).

\aparag[Looting Income]
\bparag If there was no \PILLAGE marker in the province before the looting
take place, money is gained from looting.
\bparag The controller of each looting stack gains income equal to the income
of the looted province.
\bparag In the \ROTW, only the regular income is taken, not income from exotic
resources.
\bparag In the \ROTW, if there was no \ARMY counter, gain only half the
income.
\bparag The sum of these incomes is recorded in \lignebudget{Pillages,
  privateers}.
\bparag Nobody gets money for provinces looted by minors country (including
\VASSAL). That is, the minor keep the money for itself.

\aparag[Burning \TP]
\bparag Instead of looting, troops controlling enemy \TP (not \COL) may choose
to burn it down.
\bparag Besieging is not sufficient to burn a \TP. The establishment must be
controlled, and a stack (at least 1\LDE) has to be here.
\bparag Simply remove any burned \TP from the map, it now has level 0 and
exploit no more resource.
\bparag No income is gained from burning a \TP. Only the destruction of the
establishment.

\begin{exemple}[Looting]
  In period \period{III}, \SPA took control of Dutch \provinceUtrecht and
  besieges \provinceZeeland, a stack is still present in \provinceUtrecht
  (defending against a potential counterattack). There is already a
  \PILLAGE\Facemoins in \provinceUtrecht (a \PILLAGE\Faceplus was here and was
  reduced).

  Since \SPA is still besieging \provinceZeeland and intends to continue the
  siege in the next turn, it chooses not to loot here. The \PILLAGE would
  hamper the siege by increasing attrition. However, \provinceUtrecht being
  already controlled is a good target for looting. Indeed, a \PILLAGE there
  would hamper a future attempt of Dutch reconquest\ldots The presence of an
  existing \PILLAGE here does not prevent looting: there are always more
  villages to burn and peasants to kill. A new \PILLAGE\Faceplus is put in
  \provinceUtrecht, but since another one was present, no income is gained for
  \SPA.

  Note that \SPA could have chosen to loot also \provinceZeeland. This would
  be especially useful if it did not intend to continue the siege as the
  \PILLAGE would prevent Dutch income for two turns. In that case, \SPA would
  gain 9\ducats (the income of \provinceZeeland).
\end{exemple}  

\begin{exemple}[Burning \TP]
  At the same time, \HOL is besieging a \TP\Facemoins of \paysPortugal and
  controls another \TP\Faceplus. It may choose to loot the \TP\Facemoins. It
  may not burn it as it does not control it. If looting it, \HOL will only
  gain 1\ducats (regular income of a \TP\Facemoins). For such a small amount,
  \HOL chooses to leave it, hoping to take it later. On the other hand, the
  \TP\Faceplus is controlled and \HOL chooses to burn it (rather than looting
  for 2\ducats). The province is now empty and open for attempts of Dutch \TP
  placement. An efficient albeit morally disputable way of freeing old markets
  for your merchants\ldots
\end{exemple}

\sectionJ{Building \Presidios}{\includegraphics[width=12bp]{anchor4.png}~\includegraphics[width=12bp]{anchor5.png}}\label{chRedep:Presidios}
\aparag[Where to build?]
\bparag \Presidios may only be built in provinces with a port or arsenal
depicted with a circled anchor on the map.
\bparag \Presidios may only be built in non-owned provinces.

\aparag[How to build?]
\bparag To build a \Presidio, a country must either control the province or
besiege it and not be forced to redeploy (that is, either \USURE\Faceplus,
Breach or HW obtained during the turn).
\bparag It is possible to build \Presidio even if deciding to voluntary
redeploy from the siege.
\bparag \Presidios are never build during administrative phase. Similarly,
raising a \Presidio can only happen at the same conditions than building it.

\aparag[Cost]
\bparag \Presidios are fortresses and cost the same price (construction and
maintenance). They have the same restrictions on levels and technology.
\bparag \Presidios may never be of level more than 3.
\bparag Contrary to fortresses, \Presidios may be built at any level
directly. The cost is then the sum of costs for each level.
\bparag This cost is recorded in \lignebudget{Presidios build}.

\aparag[\Presidios in play]
\bparag \Presidios are represented by fortresses of the owning power. Thus,
they do count toward the counters limit of that power.
\bparag At most one country may have a \Presidio in any given province.
\bparag Because of \Presidios, up to two fortresses of different countries may
exist in the same province. Put the regular fortress on top of the fortress
icon on the map, and the \Presidio on top of the anchor.
\bparag If a country ever gains ownership of a province where it has a
\Presidio, it may either dismantle the \Presidio or keep it and replace the
regular fortress of the province with it (remove any existing regular fortress
counter and move the \Presidio counter on top of the fortress icon to depict
the new fortress).

\section{Redeployment of land troops}\label{chRedep:Redeployment}
%\subsection{Mandatory and voluntary redeployments}
\aparag[Mandatory redeployment]
\bparag A stack without LOS must redeploy.
\bparag A besieging stack must redeploy if it is too small to maintain the
siege (less than 1\LD per level of the fortress). This typically occurs in
case of siege attrition during the last round.
\bparag A besieging stack must redeploy if there is no \USURE\Faceplus and
there was neither Breach nor HW obtained during this turn.

\aparag[Voluntary redeployments]
\bparag Other besieging forces may choose to redeploy. The controller of each
stack decides what to do with it.
\bparag Decisions to redeploy are taken in order of initiative. Redeployments
are resolved once all decisions have been taken.
\bparag Stacks may not partially redeploy. Either the whole stack redeploys or
the whole stack stays.
\bparag If the redeployment of a stack would cause another stack to be out of
supply, then the would-be OOS stack must also redeploy. This is, however,
considered as voluntary redeployment.

\aparag[Redeployment in the \ROTW]
\bparag Any stack in the \ROTW not in a controlled province may be redeployed
as per voluntary redeployment procedure.
\bparag This include stacks with unknown discoveries and this is a way to
bring back discoveries.

\aparag[Where to redeploy?]
\bparag Stacks must redeploy into the closest (in \MP) friendly controlled
territory. In case of equality, the controller of the stack chooses.
\bparag Exception: Redeployment by naval move is never forced if there is a
possibility to redeploy by land, even if it is further (in \MP).
\bparag Redeploying stacks may not enter provinces with non-redeploying
unbesieged enemy troops (troops that have finished their redeployment are
still ``redeploying'' until the end of the segment and thus do not hamper
other redeployments). Enemy fortresses, even unbesieged, do not hamper
redeployment.
\bparag Redeploying troops may use naval move if there is a large enough naval
stack adjacent to the redeploying stack at the beginning of the
redeployment. Redeployment is then done together with return to port of the
naval stack (see below). Note that combined land/sea movement is not possible
during redeployment.
\bparag Redeploying stacks may not split nor pick up more troops. All the
stack redeploys at the same place.
\bparag Redeploying stacks may enter or cross provinces with friendly troops
without any effect on these troops (no ``rout'').
\bparag Redeploying stacks may not be intercepted.
\bparag If redeployment is not possible in 12\MP or less, the stack is
destroyed. Any leader will reappear next turn as reinforcement.
\bparag After redeployment, if any province exceed its stacking limit (8\LD),
remove any exceeding troops.

\aparag[Redeployment and attrition]
\bparag Redeployment is a cause of attrition. Each redeploying stack must roll
for movement attrition with the usual modifiers.
\bparag In addition, troops redeploying because they have no LOS have a malus
of \bonus{+2} to this test.
\bparag Troops conducting a voluntary redeployment have a bonus of \bonus{-2}
to this test.

\aparag[Continuing siege]
\bparag Besieging troops which obtained either a \USURE\Faceplus, a Breach or
a HW during the turn may continue the siege for next turn.
\bparag In this case, remove all \USURE and keep only a \USURE\Facemoins.
\bparag Continuing siege does not cause attrition.

\begin{exemple}[Simple redeployment]
  A Turkish stack is besieging \provinceBanat and got a \USURE\Faceplus and a
  \USURE\Facemoins. It may choose to either stay (and replace both \USURE by a
  single \USURE\Facemoins) or redeploy to the first friendly province between
  \provinceValahia (1\MP), \provinceSerbia (2\MP) or \provinceBulgaristan
  (3\MP) (or some other provinces, depending on the military situation but
  these three are the most likely). If it redeploys, it must roll for
  attrition at +8 (entering one enemy province, namely \provinceBanat), -2
  (voluntary redeployment) -MAN (+2 if this is a large stack as this is an
  extra cause of attrition). If there was only a \USURE\Facemoins in
  \provinceBanat, then redeployment is mandatory and the bonus of \bonus{-2}
  to attrition not here.
\end{exemple}  

\begin{exemple}[Double redeployment]
  Suppose that there was another stack besieging \villeBuda in
  \provinceMagyarorszag, with a \USURE\Faceplus. If the stack in
  \provinceBanat redeploys, then the stack in \provinceMagyarorszag must also
  redeploy as it would otherwise by out of supply (its supply line goes
  through \provinceBanat). This is, however, voluntary redeployment, even if
  the redeployment in \provinceBanat was mandatory. That is, \TUR must choose
  to redeploy from \provinceMagyarorszag, but the presence of a
  \USURE\Faceplus allows for an orderly redeployment (and the \bonus{-2} to
  attrition). Note that it is likely that the closest province is the same
  from \provinceBanat and \provinceMagyarorszag, thus resulting in an
  overstacking and destruction of exceeding troops. So, voluntarily
  redeploying both stacks might be a bad idea\ldots This is not always the
  case due to rivers and mountain pass. For example, if there was no river
  between \provinceMagyarorszag and \provinceCroatie, then \provinceSerbia
  would be 4\MP from \provinceMagyarorszag, same as \provinceValahia, and \TUR
  could redeploy the \provinceBanat stack to \provinceValahia and the
  \provinceMagyarorszag stack to \provinceSerbia.
\end{exemple}  

\begin{exemple}[Mandatory redeployment]
  If a Polish counter-attack managed to relieve the siege of \provinceBanat
  (and stay there), then the Turkish stack in \provinceMagyarorszag is out of
  supply and must redeploy (with a malus of \bonus{+2}). The closest province
  is \provinceSerbia (provided there is no enemy troops in
  \provinceCroatie). The Polish stack in \provinceBanat may not intercept
  (neither in \provinceCroatie, nor in \provinceSerbia). If there was an
  \paysHongrie stack (even a lone \LD) in \provinceCroatie, then the stack
  must go to either \provinceMoldova (if friendly) of \provinceValahia (in
  that case, that's a 6\MP move, hence another cause of attrition resulting in
  another \bonus{+2} to the test).
\end{exemple}  

\begin{exemple}[Impossible redeployment]
  Suppose now that \provinceDalmacija belongs to \VEN and that there are enemy
  troops in both \provinceCroatie and \provinceErdely (due to a way too bold
  Turkish attack: destroy enemy troops before going deep in their
  territory). If \paysMoldavie is neutral, the Turkish stack is trapped and
  cannot redeploy: it is simply destroyed (it may not cross neutral
  territory). If \provinceMoldova is friendly, the troop may redeploy
  there. If both \provinceMoldova and \provinceBasarabia are enemy (eg Polish)
  and \provinceValahia is neutral, then the stack could possibly redeploy
  through \provinceKarpatok, \provinceBukovina, \provinceMoldova and
  \provinceBasarabia to \provinceRumeli. However, that 13\MP, more than the
  12\MP limit, thus this redeployment is not possible and the stack is
  destroyed.
\end{exemple}  

\begin{exemple}[Naval redeployment]
  A Turkish stack of two \Janissaire \ARMY\Faceplus landed in
  \provinceMalta. By the end of the turn, due to the heroic defence of
  \leader{La Valette}, \TUR only managed to get a \USURE\Facemoins and is thus
  forced to redeploy. Fortunately, the Turkish armada is still here to carry
  the troops back home. Since naval move is 3\MP whatever the distance and
  return to port is not constrained by distance, the stack can redeploy to any
  Turkish port (or arsenal). Contrary to regular naval move, the land stack
  may not move after landing in the port.

  Note that in addition to the mandatory redeployment, the stack is conducting
  a naval move embarking out of controlled port, and is a large stack. Two
  extra causes for attrition, each giving a \bonus{+2} to the roll.

  If there was no naval stack adjacent to \provinceMalta, then redeployment
  would have been impossible and the Turkish army is destroyed.

  \smallskip

  If \SUE is besieging \provinceDanzig and owns \provinceHinterpommern,
  then its closest redeployment possibilities are first by sea (3\MP)
  and then to \provinceHinterpommern (4\MP due to swamp and
  river). Redeployment by sea, however, does not takes precedence over
  redeployment by land and \SUE can freely choose any of the
  possibilities (but may not split its stack).
\end{exemple}

\section{Return to Port}\label{chRedep:Return port}
\aparag[Navies]
\bparag Naval stacks being located in a sea zone have the choice to either
\begin{modlist}
\item return to any friendly, unblockaded port (or arsenal) of the
  controller's choice (not necessarily the closest one);
\item OR stay at sea.
\end{modlist}
\bparag This is considered a move (even if staying at sea) and, as any naval
move, it causes attrition with the usual modifiers.
\bparag In addition, naval stacks staying at sea at the end of turn have a
malus of \bonus{+2} to this roll.
\bparag No interception, including by \Presidios or \StraitFort, may occur
during this move.

\aparag[Pirates and privateers]
\bparag \pays{pirates} \corsaire stay where they are. They will still be
active next turn.
\bparag Other \corsaire are repatriated to a port of their owner's choice.
\bparag No interception, including by \Presidios or \StraitFort, may occur
during this move.

\section{Gold repatriation}\label{chRedep:Gold Repatriation}

\aparag[Before redeployment]
\bparag[Income.] During Income phase, gold produced in a \COL may be stored in
any coastal \COL in the same or adjacent \Area. See~\ref{chIncomes:Gold
  Production}.
\bparag[Military.] During military rounds, gold may be moved either with
troops or ships (including convoys) and intercepted by
enemies. See~\ref{chIncomes:SpanishConvoys} for the apparition of the Spanish
convoys and~\ref{chMilitary:Convoys} for the attacks on convoys.

\aparag[Reaching Europe]
\bparag As soon as a land or naval stack carrying gold reaches a owned and
controlled province on the European map, the gold is emptied and the amount is
tallied in \lignebudget{Gold from ROTW and Convoys}.
\bparag Note that \COL of level 6, as well as \provinceAcores or
\province{Islas Canarias}, are European provinces but not on the European map
and are thus not sufficient to bring gold back home.

\aparag[Gold transportation]\label{chBudget:Gold Transportation}
\bparag During the Redeployment phase, gold gain a free land movement (only).
\bparag Exception: Gold in any establishment bordering the \seazoneCaspienne
at the beginning of this segment may cross it and reach any other province
bordering that sea.
\bparag It may thus moves any distance along a chain of friendly
establishments (\COL, \TP or fort).
\bparag Each establishment along the chain must be 12\MP or less from the
previous one (counting the cost as for \LD).
\bparag The path may not cross a province with a non-besieged enemy stack or
fortress.
\bparag Gold can thus be repatriated for any distance as long as each ``leg''
of the movement is 12\MP or less.
\bparag If Gold reaches an owned and controlled province on the European map,
it is immediately emptied and tallied in \lignebudget{Gold from ROTW and
  Convoys}.
\bparag This movement of Gold may not be intercepted in any way.

\begin{designnote}
  Siberian gold is usually repatriated this way and thus do not require
  any campaign or accompanying troops to move. 

  On the other hand, American gold must cross the Ocean and can only do
  so during the military rounds. This repatriation still allows to
  easily concentrate all the gold in one place and prepare for an
  immediate naval move during the next turn.
\end{designnote}

% Local Variables:
% fill-column: 78
% coding: utf-8-unix
% mode-require-final-newline: t
% mode: flyspell
% ispell-local-dictionary: "british"
% End:

