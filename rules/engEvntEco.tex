% -*- mode: LaTeX; -*-
\definechapterbackground{Economical events}{economicalevents}
% Suggestion: one of Turner's storm (bad weather).  See
% http://www.william-turner.org/

\chapter{Economical events}
\label{chapter:Events:Eco}

\section{Event Table of economical random events}

\begin{eventstable}[Random economical events]
  \centering{\graytabular\tabcolsep=4pt
    \begin{tabular}{|c|cccccccccc|}%
      \hline\ghline%
      {1\up{st}\rlap{>}}%
      &  1& 2& 3& 4& 5& 6& 7& 8& 9& 0\\\hline\ghline%
      1&29& 9&17&38&22& 7& 6&18& 4&45\\\ghline%
      2& 2&43&28&12&36&16&49&24& 3&15\\\ghline%
      3&42&33&18&45& 4&14&38& 7&46&10\\\ghline%
      4&22& 6&44&19&32&37&21& 7&40& 9\\\ghline%
      5&16&34& 8&24&13& 2&38&28&36&45\\\ghline%
      6&44&10&27&15&20&47&18& 6&14&30\\\ghline%
      7&23&38&17& 9& 5&43&11&41&26& 4\\\ghline%
      8& 8&35& 2&31&39&16&20&45&13&16\\\ghline%
      9&24& 7&19&14&12& 4& 5&25&35&48\\\ghline%
      0&38&17&37& 8&11& 9& 7&16&23& 1\\\hline%
    \end{tabular}}
\end{eventstable}
\eventssummary{%
  eco:Crisis of madness|,%
  eco:Excellent Minister|,%
  eco:Serious sickness|,%
  eco:Agricultural Crisis|,%
  eco:Naval losses|,%
  eco:Looting and insecurity|,%
  eco:Fiscal evasion|,%
  eco:Corruption|,%
  eco:Technological advance|,%
  eco:Mine Discovery|,%
  eco:Wave of obscurantism|,%
  eco:Piracy|,%
  eco:Development of warships|,%
}

\eventssummary{%
  eco:Military leader|,%
  eco:Drought|,%
  eco:Exceptional year|,%
  eco:Sales of honorary titles|,%
  eco:Epidemics|,%
  eco:Rush Colonists|,%
  eco:Refugees|,%
  eco:Gift to the State|,%
  eco:Scandal at the court|,%
  eco:Plots at the court|,%
  eco:Poor weather|,%
  eco:Death Heir|,%
  eco:Mine Depletion|,%
  eco:New ally|,%
  eco:Defection of an ally|,%
  eco:Desertions|,%
  eco:Death of a military leader|,%
  eco:Dynastic inheritance|,%
}\eventssummary{%
  eco:Inflation|,%
  eco:Offer of alliance|,%
  eco:Independence of a vassal|,%
  eco:Enthusiasm for the Army|,%
  eco:Renewal of popularity|,%
  eco:Enthusiasm for the Navy|,%
  eco:Agricultural technique development|,%
  eco:Reorganisation of the army or the fleet|,%
  eco:Conquistador|,%
  eco:Explorer|,%
  eco:Governor|,%
  eco:Diplomatic Preeminence|,%
  eco:Cultural expansion|,%
  eco:Deflation|,%
  eco:Economic Crisis|,%
  eco:Economic Boom|,%
  eco:Rectification|,%
  eco:Treachery|,%
} \newpage\startevents




\section{Description of Economical Events}



\event{eco:Crisis of madness}{E-1}{Crisis of madness}{1}{Orig}

Reduce all values of the monarch's characteristics by half for this turn
(rounded down). Modify next-turn survival die-roll by \bonus{+1}.



\event{eco:Excellent Minister}{E-2}{Excellent ministers}{3}{PBmod}

\phevnt
\aparag A Minister is appointed per \ref{chEvents:Excellent Ministers}. His
characteristics as ruler are rolled for the three values by 1d10 modified: a
die-roll of 1 becomes 7, of 2 becomes 8, 10 becomes 9.  Another die roll sets
the length of the Ministry:

\GTtable{excellentministres}

\aparag The office of the Minister include the current turn, and ends just
before the ``economical events'' segment of the events phase following the
last full turn of office.

\aparag A value of the Minister is used only if it is strictly superior to the
monarch's own characteristic.

\aparag If the Monarch dies when the Minister is still in office, a malus of
\bonus{-2} is applied to the characteristics determination die-rolls for the
monarch's successor, but only for a characteristic that was increased due to
the Minister by at least 2 above the Monarch value.



\event{eco:Serious sickness}{E-3}{Serious sickness}{1}{Orig}

% (JCD) What about monarchs that do no survival roll. Immune sudden death?
Reduce all characteristics of the monarch by 3 for this turn only, 1 being the
minimum value. In addition, roll a die. If the result is 10, the monarch
deceases immediately. Else, modify next-turn survival die-roll by \bonus{+1}.

If the current monarch did benefit from \ref{chEvents:Excellent Ministers},
the characteristics are only reduced by 1.

The monarch cannot lead armies or fleets during the turn
% (Jym) adding:
except if he must do so due to a political event.



\event{eco:Agricultural Crisis}{E-4}{Agricultural crisis}{4}{Orig}

The country has seen real trouble in crops and farming. The loss is of 50\%
(lowered by 10\% per unit of \RES{Cereals} \MNU already owned by the country)
of its income of provinces this turn (\lignebudget{Provinces income}). The
loss is registered in \lignebudget{Event}.

% (JCD) Should be added to the industrial income instead of RT?
Other countries that possess \RES{Cereals} \MNU gain immediately 10\ducats per
unit in their \RT, to be added from \lignebudget{RT at start of turn} to
\lignebudget{RT after Events}.



\event{eco:Naval losses}{E-5}{Naval losses}{2}{Orig}

Fires, storms and disasters spread at sea. Roll 2d10, and add \bonus{+2} if
the \MAJ has at least 3 \FLEET counters deployed at that time, or subtract
\bonus{-2} if it has only one (or none).  The number of \ND lost is given by
the result:

\centerline{\begin{tabular}{*{5}{c}} \textlessequal 4& 5--10& 11--15 & 16--19
    & \textgreatequal20\\\hline 0 & 1 & 2 & 3 & 4
  \end{tabular}}

The \ND can come from anywhere. \NGD count for half a loss only; \NTD can be
lost only if there are no warships or galleys left.



\event{eco:Looting and insecurity}{E-6}{Looting and insecurity}{3}{JCMod}

The country loses 10\% (rounded up) of its income of provinces this turn
(\lignebudget{Provinces income}). The loss is registered in
\lignebudget{Event}.

Place a \PIRATE\facemoins in the player's \CTZ (if any); in \stz{Baltique} if
the player has a port on this \STZ (and no \CTZ); in \stz{Adriatique} if the
player has a port on this \STZ (and no \CTZ). There may be no \PIRATE if there
are no such ports.



\event{eco:Fiscal evasion}{E-7}{Fiscal evasion}{5}{Orig}

\phevnt
Reduce the Royal treasury by 20\% of its absolute value (min. is 20\ducats)
this turn (from \lignebudget{RT at start of turn} to \lignebudget{RT after
  Events}). Furthermore, if \TUR receives this event, he has to check for
Pashas' corruption.


\digression[eco:Fiscal:Pashas Corruption]{Corruption of Pashas}

\phevnt
\aparag One \xnameref{chSpecific:Turkey:Pasha} becomes corrupted (turn the
counters on their corrupted side). This Pasha is chosen by the \SDCF (or
\MAJHAB is there is none). This pasha must be in owned Turkish provinces; if
none are available, displace the newly corrupted pashas in any province
(except the capital).



\event{eco:Corruption}{E-8}{Corruption}{3}{Orig}

\phadm
All costs of purchase double this turn (reinforcements and campaigns). Costs
of maintenance increase by 10\% (rounded up). In addition, \TUR suffers the
effects described in \ref{eco:Fiscal:Pashas Corruption}.



\event{eco:Technological advance}{E-9}{Technological advance}{4}{Orig}

The player can move one of his two technology marker (naval or land) a number
of boxes forward on the technology track determined by the roll of a die
(choice of the technology must be made before rolling the die):

\centerline{\begin{tabular}{*{3}{c}} \textlessequal 1--5 & 6--8 &
    9--10\\\hline 1 box & 2 boxes & 3 boxes
  \end{tabular}}



\event{eco:Mine Discovery}{E-10}{Discovery of mines}{2}{JCMod}

\phevnt
\aparag Place a \countermark{Gold Mine} counter in one national province of
the player (still controlled) in mountain terrain (or non-clear terrain if
none available, or clear terrain as a last resort), where there is not already
such a counter, and provided the country (not the player) did not benefit from
this event two times.
\bparag If the country had already benefited from this event two times, test
for \ref{eco:Depletion:Depletion America} instead.

If no controlled terrain is available, re-roll.



\event{eco:Wave of obscurantism}{E-11}{Wave of obscurantism}{2}{Orig}

Reduce the \STAB by {\bf 1} level if player is Protestant, and {\bf 2} levels
in all other cases.



\event{eco:Piracy}{E-12}{Pirates}{2}{JymMod}
\begin{todo}
  [TBD] Replace the two Pirates events by minor country colonisation.
\end{todo}
\phevnt
\aparag This event is only resolved during the economic situation segment
of the event phase.
% \aparag Count the number of piracy events that must occur this turn, including
% the one that may arise via the economic situation roll.
% \bparag If there are two or more, then the target is ``Everywhere''.
% \bparag If there is only one, and the target is not precised, roll one die:
% 1--5: America, 6--10: Asia.

% \aparag For each \STZ in the target, in the order listed below, roll one die
% and place a \PIRATE\facemoins if it is higher than the piracy level of the
% \STZ. See~\ruleref{chEvents:Piracy Level} and following for the details of
% piracy placements.

% \aparag List of targeted \STZ and order of test:
% \bparag Everywhere: \seazoneCaraibes, \seazone{Atlantique W}, \seazoneIndien,
% \seazoneOman, \seazoneGuinee, \seazoneRecife, \seazonePerou, \seazoneFormose,
% \seazonePatagonie, \seazoneTempetes, \stz{Canarias}.
% \bparag America: \seazoneCaraibes, \seazone{Atlantique W}, \seazoneGuinee,
% \seazoneRecife, \seazonePatagonie, \seazoneTempetes, \stz{Canarias}.
% \bparag Asia: \seazoneIndien, \seazoneOman, \seazonePerou, \seazoneFormose,
% \stz{Canarias}.



\event{eco:Development of warships}{E-13}{Development of warships}{2}{Orig}

The player advances his naval technology by 1 box.
% (JCD) Relation with \RES{Wood}? NO!



\event{eco:Military leader}{E-14}{Military leader}{3}{Orig}

Roll one die. If the result is even, draw a general, else draw an admiral. The
leader will be drawn from the anonymous pool of the player, and will not be
included in the minimum leaders limit for the period that the leader is
entitled to.

The leader is available for 1 turn if the result is between 1 and 5, 2 turns
(current and following) if it is between 6 and 10.



\event{eco:Drought}{E-15}{Drought}{2}{Orig}

The country loses 30\% (rounded up) of its income of provinces this turn
(\lignebudget{Provinces income}). The loss is registered in
\lignebudget{Event}.



\event{eco:Exceptional year}{E-16}{Exceptional year}{5}{Orig}

The country gains 10\% (rounded up) of its income this turn
(\lignebudget{Income}). The gain is registered in \lignebudget{Events}.



\event{eco:Sales of honorary titles}{E-17}{Sales of honorary titles}{3}{Orig}

The Major Power may opt to sell honorary titles. If it chooses so, roll 1d100.
The result gives the product of these sales in \xducats, added immediately to
\lignebudget{RT at start of turn} in \lignebudget{RT after Events}.  Then the
minimum number of generals of the power is lowered by one this turn (only).
If may opt to have none of these effects (before rolling the dies).



\event{eco:Epidemics}{E-18}{Epidemics}{3}{Orig}

The country loses 20\% (rounded up) of its income this turn
(\lignebudget{Incomes}). The loss is registered in \lignebudget{Events}.



\event{eco:Rush Colonists}{E-19}{Rush of colonists}{3}{JymMod}

If the country has no \COLaction or \TPaction, it may elect to ignore this
event and re-roll another one (to be decided immediately).

This event gives a bonus of \bonus{+3} to the die-roll of \COLaction, as well
as a supplementary and free \COLaction with small investment (30\ducats),
usable this turn or any other turn of the current period (lost if not used
before the end of the current period). Moreover, the country may ignore
restrictions of~\ref{chAdministration:Pioneering} for this turn.

If this is not period \period{I} also apply~\ref{eco:Rush:Minor colony}.


\digression[eco:Rush:Minor colony]{Minor country colonisation}

If this is not period \period{I}, roll on the following table; subtract {\bf
  3} in periods \period{II} and \period{III} and add {\bf 3} in periods
\period{VI} and \period{VII}.

\begin{modlist}[3em]
\item[-2] Destruction of a Minor establishment.
\item[-1] Creation of a Minor establishment in \continentBresil.
\item[0] Creation of a Minor establishment in
  \granderegionEcuador/\granderegionYucatan/\granderegionPanama.
\item[1--2] Creation of a Minor establishment in \continentCaraibes.
\item[3--4] Loss one side of a Minor establishment.
\item[5] Creation of a Pirate Haven in \continentCaraibes.
\item[6--7] Creation of a Minor establishment in a coastal province in the
  American zoom.
\item[8] Increase one Minor establishment.
\item[9] Creation of a Minor establishment in a coastal province in
  \continentIndia.
\item[10] Creation of a Pirate Haven in \granderegionMadagascar
\item[11] Creation of a Minor establishment in a coastal province in
  \continentIndia.
\item[12--13] Creation of a Minor establishment in \continentCaraibes.
\end{modlist}

\aparag[Creation of a Minor establishment.] Select one empty province at
random within the specified ones and put a Minor establishment \Facemoins in
it.
\bparag If there are no empty provinces in the specified ones or there are no
unused Minor establishment, turn this into a \terme{Increase of one Minor
  establishment} instead.

\aparag[Creation of a Pirate haven.] If one already exists in the specified
provinces, it is turned on level 2 (nothing happens if it is already level 2).
\bparag If there is no Pirate haven in the specified provinces, select an
empty one at random and put a Pirate haven of level 1 in it.
\bparag For \granderegionMadagascar, do not select the province at random. Use
\province{Madagascar N} if empty and \province{Madagascar S} otherwise.

\aparag[Destruction of a Minor establishment.] Select a Minor establishment at
random and remove it from the map.

\aparag[Loss of one side.] Select a Minor establishment at random.
\bparag If it is \Facemoins, remove it from the map.
\bparag If it is \Faceplus, turn it \Facemoins and select one of its exploited
resources at random which is no longer exploited.

\aparag[Increase of one Minor establishment.] Select one Minor establishment
\Facemoins at random and turn it \Faceplus.

\aparag[Creation/Increase of establishments.] Whenever a new side of Minor
establishment is created:
\bparag If there is at least one unexploited resource in the \Area, it
exploits one at random.
\bparag Otherwise, it exploits one of the existing resource at random,
stealing it from whoever exploits it.



\event{eco:Refugees}{E-20}{Refugees}{2}{JCMod}

If the country has no \COLaction or \TPaction, it may elect to ignore this
event and re-roll another one (to be decided immediately).

The player receives a free of charge strong investment that can be used for a
\TFI (but cannot be cumulative with another investment on the same \STZ/\CTZ).

This also gives in addition the same effect as \ref{eco:Rush Colonists}, but
with a bonus of \bonus{+2} only.



\event{eco:Gift to the State}{E-21}{Gift to the State}{1}{Orig}

The people make a gift of 1d100\ducats added immediately to \lignebudget{RT at
  start of turn} in \lignebudget{RT after Events}.



\event{eco:Scandal at the court}{E-22}{Scandal at the court}{2}{JCMod}

The player's monarch's Diplomatic value is reduced by 3 for this turn (to a
minimum of 1). The player also immediately loses 50\ducats, taken from
\lignebudget{RT at start of turn} into \lignebudget{RT after Events}.



\event{eco:Plots at the court}{E-23}{Plots at the court}{2}{Orig}

The player's monarch's Diplomatic value is reduced to 1 for this turn.  In
addition, he will add a modifier of \bonus{+2} to next turn's survival
die-roll for his monarch.



\event{eco:Poor weather}{E-24}{Poor weather}{3}{JymMod}

\phmil
\aparag This turn, add \bonus{+2} to each season continuation die roll. All
Winter round will be in bad weather.
\aparag[Frozen Sea] Moreover, if a Winter round happen after a die roll of 1
(before modifications), \seazoneOresund is frozen. No fleet can go through, in
or out of it (fleets in it at the beginning of the round stay there but suffer
no damage). Armies can cross it (it's an unfriendly rough terrain with no
effect on combat) but not stop in it. No battle or interception of any kind
may happen here. If retreat into \seazoneOresund is forced after a land
battle, the stack retreats one province further into solid ground but has a
malus of \bonus{+2} to it retreat die roll.



\event{eco:Death Heir}{E-25}{Death of the heir to the throne}{1}{Orig}

The player will receive a -1 malus to his die-roll for each one of the future
characteristics of his next monarch. This event may be drawn several times but
the malus will apply only once on the next monarch. This event has no effect
if the next monarch is a named monarch, including one whose characteristics
are not fixed but must be rolled.



\event{eco:Mine Depletion}{E-26}{Depletion of a mine}{1}{Orig}

Place a marker \countermark{Exhausted Mine} on a mine currently exploited by
the player (either in Europe or in the \ROTW), drawn at random, and check for
\ref{eco:Depletion:Depletion America}. If no mine qualifies, just do the
check.


\digression[eco:Depletion:Depletion America]{Depletion of mines in America}

\phevnt
\aparag Each time this is called for, all exploited mines in \continentAmerica
will be tested for depletion. This test is made at most once each turn.
\bparag The mines are tested in the following order: the Potosi mine (value
50), the Tenochtitlan mine (value 40), then the mines of the player exploiting
the largest number of mines in \continentAmerica (in an order chosen by the
player itself), then the next player, and so on.
\bparag A mine is depleted if a die-roll gives 1, or 1 or 2 in period V or
later.
% (JCD) Used to be turn 36, now 35
\aparag Only one mine per turn may be depleted this way. As soon as one as
been depleted this way, there is no further need to check the others.



\event{eco:New ally}{E-27}{New ally}{1}{Orig}

The player receives a modifier of \bonus{+3} in diplomacy on a minor of his
choice, valid for this turn. The choice of the minor has to be made
immediately and secretly. It will be revealed during the next Diplomacy phase.



\event{eco:Defection of an ally}{E-28}{Defection of an ally}{2}{PBMod}

One country in \VASSAL position that is not a special vassal (i.e. on which
diplomacy is possible) of the power, if any (chosen at random), is lowered by
3 boxes on the Diplomatic track.  If none qualifies, another country
determined at random among all the countries on Diplomatic track of the power
is lowered by 2 boxes.



\event{eco:Desertions}{E-29}{Desertions}{1}{Orig}

Desertions occur in the army. Roll 2d10, and add \bonus{+2} if the \MAJ has at
least 4 \ARMY counters deployed at that time, or subtract \bonus{-2} if it has
only one.  The number of \LD lost is given by the result:

\centerline{\begin{tabular}{*{5}{c}} \textlessequal 4& 5--10& 11--15 & 16--19
    & \textgreatequal20\\\hline 0 & 1 & 2 & 3 & 4
  \end{tabular}}



\event{eco:Death of a military leader}{E-30}{Death of a military
  leader}{1}{Orig}

Draw one leader at random in all military leaders of the player on the map.
The leader is removed from the game if it is a named one, returned to the pool
if it is an \anonyme one. The period limit is diminished by one for the turn.



\event{eco:Dynastic inheritance}{E-31}{Dynastic inheritance}{1}{Orig}

The player receives a \bonus{+5} bonus in his next diplomacy phase for a minor
country that may become a vassal. This minor must currently be located in the
\RM box or above on the player's diplomatic track. This minor has to be
nearest to the national territory of the player in term of number of provinces
(in case of tie, leave it to the player's choice).



\event{eco:Inflation}{E-32}{Inflation}{1}{JymMod}

Increase the level of inflation by 1, that is move the marker one box to the
right (without exceeding the maximum level). At most one event among
\xnameref{eco:Inflation} and \xnameref{eco:Deflation} can take place in a
single turn (treat as no event if a second one is rolled).



\event{eco:Offer of alliance}{E-33}{Offer of alliance}{1}{Orig}

The player receives a \bonus{+3} bonus in diplomacy to his die-roll for a
minor of his choice (to be decided immediately).



\event{eco:Independence of a vassal}{E-34}{Independence of a vassal}{1}{Orig}

A minor vassal that is not a special vassal (i.e. on which diplomacy is
possible) breaks its vassalisation and remains only an ally. The player has a
temporary \CB against this minor. Move the marker of the minor from the
\VASSAL box to the \RM box.



\event{eco:Enthusiasm for the Army}{E-35}{Enthusiasm for the Army}{2}{Orig}

The player may either receive 2 \LD free of charge, or increase his land
technology by 1 box.



\event{eco:Renewal of popularity}{E-36}{Renewal of popularity}{2}{Orig}

The player receives 20\ducats in his royal treasury (added immediately to
\lignebudget{RT at start of turn} in \lignebudget{RT after Events}). All the
following administrative operations: \TFI, \TPaction, \COLaction, \MNU
placement attempts, \DTI/\FTI improvement also receive an exceptional bonus of
\bonus{+2} to the die-roll for this turn.

On the other hand, a malus of \bonus{-10} to the die-roll is applied on the
\terme{Exceptional taxes raising} operation.



\event{eco:Enthusiasm for the Navy}{E-37}{Enthusiasm for the Navy}{2}{Orig}

The player may either receive 2 \NWD (or 4 \NGD) free of charge, or increase
his naval technology by 1 box.



\event{eco:Agricultural technique development}{E-38}{Agricultural technique
  development}{5}{Orig}

Increase the country's income by 2\ducats per controlled and owned province
(i.e. not including occupied, looted, controlled but still belonging to the
enemy, belonging to a vassal provinces) for this turn only. The gain is
registered in \lignebudget{Event}. In addition, for this turn only, the
country receives a bonus of \bonus{+3} to the die-roll for the
\terme{improvement of \DTI}, as well as all attempts to create a \RES{Cereals}
or \RES{Wine} manufacture.



\event{eco:Reorganisation of the army or the fleet}{E-39}{Reorganisation of
  the army or the fleet}{1}{Orig}

Gives a bonus of \bonus{+2} to the die-roll of either land or naval technology
improvement (the choice must be written down immediately). Also gives a 50\%
bonus discount to the unit reorganisation due to a new technology being
discovered.



\event{eco:Conquistador}{E-40}{Conquistador}{1}{Orig}

If the country has no \anonyme\LeaderC, it may elect to ignore this event and
re-roll another one (to be decided immediately).

The player receives a conquistador among the \anonyme \LeaderC markers still
available. It remains in play for this turn only.



\event{eco:Explorer}{E-41}{Explorer}{1}{Orig}

If the country has no \anonyme\LeaderE, it may elect to ignore this event and
re-roll another one (to be decided immediately).

The player receives an explorer among the \anonyme \LeaderE markers still
available. It remains in play for this turn only.



\event{eco:Governor}{E-42}{Governor}{1}{JCMod}

If the country has no \anonyme\LeaderGov, it may elect to ignore this event
and re-roll another one (to be decided immediately).

The player receives a governor among the \anonyme \LeaderG markers still
available, to be placed in a \TP or a \COL of the player.  It remains in play
for this turn only.



\event{eco:Diplomatic Preeminence}{E-43}{Diplomatic preeminence}{2}{Orig}

Gives the player a bonus of \bonus{+1} to the die-roll to all his diplomatics
actions on minors (either European or \ROTW), and a bonus of \bonus{+1} column
in his favour for all of his attempts of \TP and \COLaction for this turn
only.



\event{eco:Cultural expansion}{E-44}{Cultural expansion}{2}{Orig}

This gives a bonus of 20\ducats to any subsidies obtained by a minor vassal
reaching the \SUB diplomatic level. Any subsidies will yield at least
20\ducats, whatever the modifiers.  In addition, it has the same effect as
\ref{eco:Diplomatic Preeminence} above.



\event{eco:Deflation}{E-45}{Deflation}{4}{Orig}

Reduce the level of inflation by 1, that is move the marker one box to the
left (without exceeding the minimum level). At most one event among
\xnameref{eco:Inflation} and \xnameref{eco:Deflation} can take place in a
single turn (treat as no event if a second one is rolled).



\event{eco:Economic Crisis}{E-46}{Economic crisis}{1}{JCMod}

Demand for exotic resources decreases in Europe and prices fall. Adjust prices
as follows (without exceeding any normal limits, and only for already
available resources):
\begin{itemize}
\item \RES{Fish}, \RES{Salt}: no modification
\item \RES{Sugar}, \RES{Cotton}, \RES{Furs}: \bonus{-1} box
\item \RES{Slaves}, \RES{Spices}, \RES{Products of America}: \bonus{-2} boxes
\item \RES{Products of Orient}, \RES{Silk}: \bonus{-3} boxes
\end{itemize}
At most one event among \xnameref{eco:Economic Crisis} and
\xnameref{eco:Economic Boom} will take effect this turn. Re-roll if one was
already used.



\event{eco:Economic Boom}{E-47}{Economic boom}{1}{JCMod}

Demand for exotic resources increases in Europe and prices rise. Adjust prices
as follows (without exceeding any normal limits, and only for already
available resources):
\begin{itemize}
\item \RES{Fish}, \RES{Salt}: no modification
\item \RES{Sugar}, \RES{Cotton}, \RES{Furs}: \bonus{+1} box
\item \RES{Slaves}, \RES{Spices}, \RES{Products of America}: \bonus{+2} boxes
\item \RES{Products of Orient}, \RES{Silk}: \bonus{+3} boxes
\end{itemize}

At most one event among \xnameref{eco:Economic Crisis} and
\xnameref{eco:Economic Boom} will take effect this turn. Re-roll if one was
already used.



\event{eco:Rectification}{E-48}{Rectification}{1}{PBMod}

The monarch yields to pressures yielding to straighten the domestic and
foreign situation. The player can choose one option exactly among the three
following bonuses:
\begin{itemize}
\item Pay without overcosts all his land forces up to the triple of the normal
  limit.
\item Increase his construction limit for ships by 50\% (rounded up).
\item Obtain a bonus of \bonus{+5} to his die-roll concerning the action of
  \terme{improvement of Stability}.
\item Refund for free National Loans up to 200\ducats.
\end{itemize}
Choice must be written down immediately to be valid.



\event{eco:Treachery}{E-49}{Treachery}{1}{PBMod}

The player benefits from a treachery against one of his opponents with whom he
is \emph{already} at war (either a player or a minor country). The player can
choose one option immediately among the three following bonuses:
\begin{itemize}
\item Capture immediately an enemy fortress that he currently besieges, or
  obtain a one time bonus of \bonus{+4} to a \terme{siegeworks} action
  die-roll in the current turn (if he establishes a siege this turn).
\item Move himself one land stack of his opponent one time during his
  opponent's movement phase this turn, instead of his opponent. The player
  will pick the exact round. However, he cannot make this stack attack any
  units except a stack commanded by him, nor can he exceed 5 MP on land, or
  make a naval move with a modifier higher than +8 for attrition on sea.
\item Obtain a bonus of \bonus{+5} to one his diplomatic operations against a
  minor country whose marker is on his opponents diplomatic track, this turn
  only (the choice is announced along the diplomatic actions).
\end{itemize}

% (Pierre) Some choices have problems especially diplomacy during the war;
% convert the condition on naval move.

\stopevents

% Local Variables:
% fill-column: 78
% coding: utf-8-unix
% mode-require-final-newline: t
% mode: flyspell
% ispell-local-dictionary: "british"
% End:

% LocalWords: cccccccccc eco PBmod excellentministres malus JCMod Baltique
% LocalWords: Adriatique Atlantique Canarias JymMod turn's Potosi PBMod JCD
% LocalWords: vassalisation siegeworks overcosts Jym
