% -*- mode: LaTeX; -*-

\section{Diplomacy with European Minor Powers}

% RaW: [33]



\subsection{Presentation}
\subsubsection{Actions and control}
\aparag[Informal Overview]
After having negotiated between them, players may "negotiate" with minor
countries. Each player has 1 to 6 diplomatic actions per turn. This number is
given for each country and each period, as per the Limits table located on
Players' Aides.  Each diplomatic attempt against one minor country uses 1 such
action and an investment in ducats which can be basic, medium or strong.
Actions and diplomatic expenses have to be written on \lignebudget{Diplomatic
  actions}.
Results of those actions are assessed: each is solved with the help of three
dice. In case of success, the influence that the player exerts on the minor is
adjusted.  Each minor country that is influenceable by the diplomacy of
players has a diplomatic status marker displaying the relevant indications for
the diplomatic game.  Each such counter is placed on the diplomatic track
located on the Rest-of-the-world map. Such a counter must be found in
permanence placed in a square corresponding to its attitude towards a player
or in the square reserved to the neutrals.

\aparag[Levels of Diplomatic control]
The principle of the diplomacy with European minor countries is that there can
be only one influence of any one single player on a given minor, meaning that
this player has a preponderant influence, or diplomatic control of the minor
country; he is also names the "Patron" of the minor country. This influence is
divided into different levels of increasing importance, which are:
\begin{deflist}
  \listingabbrev{Neutral}{Neutral (not really a status, rather the fact of
    being independent)} \listingabbrev{RM}{Royal Marriage (dynastic ties unite
    the reigning families of the two countries)} \listingabbrev{SUB}{Subsidies
    (the countries share economic ties and have mutual debts)}
  \listingabbrev{MA}{Military Alliance (the two countries have concluded
    military alliances and may help each other during wars)}
  \listingabbrev{EC}{Expeditionary Corps (the minor country is susceptible of
    sending larger armed forces)} \listingabbrev{EW}{Entry in war (the minor
    country may be called for a full participation in a war)}
  \listingabbrev{VASSAL}{Vassal (the minor country is effectively dependant on
    the authority of the major country, and will participate in wars)}
  \listingabbrev{ANNEXION}{Annexation (the minor country has really become
    part of the major country in some form, and counts for many things as
    such)}
\end{deflist}

\aparag[Limit]
This influence may be limited sometimes to a maximum level for some specific
minor countries or for some particular players. It is even possible that a
player could not make diplomatic action against a particular minor (e.g. the
Turkish player against \paysPerse).

\aparag[The Diplomatic Track]
Each player has a line of his own on the diplomatic track situated on the
Rest-of-the-world map. Columns indicate the different diplomatic status that
the player can achieve on a minor, as described immediately above.

\aparag[Diplomatic Counters]
Each counter (front/back) regroups information concerning the minor country
mentioned on that counter.  All this information also figures in the Annexes
dealing with minor countries.

\subsubsection{Other}
\aparag It is possible to give a province to a minor country if either this is
a province formerly owned by it (at any point during the game) or it has a
blurred shield of the minor.
\bparag This is not an action, this does not count toward the limit of actions
per turn.

\aparag If the minor is not existing anymore, it is immediately recreated as a
\VASSAL of the major giving a province.
\bparag If the minor cannot be \VASSAL, put it on the highest possible
diplomatic level allowed for it instead.

\subsection{Diplomatic actions}\label{chDiplo:Diplomatic Actions}


\subsubsection{Principles of diplomatic actions}
\aparag A player has a number of diplomatic actions which is limited according
to the period in play (from 1 to 6 actions per turn). Even though the
Diplomatic Actions are resolved after the Declarations of Wars, the rules are
explained here (because Diplomatic control is helpful to understand the wars).
\bparag The action is aimed at increasing the level of control of the player
on that minor country, or decreasing the level of control of another power on
a minor country.
\bparag The player registers on his monarch sheet all his diplomatic actions
of the current turn, by specifying which minor countries are aimed at. He must
pay the cost of each action (written on \lignebudget{Diplomatic actions}) and
indicate on his monarch sheet the level of investment placed in that action
(either basic, medium or strong).
\bparag[Diplomatic Supports]\label{chDiplo:Diplomatic Support}
The player can also declare that he is supporting one action of another
player. This support is a diplomatic action of the player by itself (it has to
be paid as a basic investment diplomatic action), and must be written on the
supporting player monarch sheet.
\bparag Supports can be discussed and established as an informal agreement
between the player granting support and the one receiving it.
% \bparag In all cases, supports can never be sold to another player more than
% 30 \ducats each.
\bparag ``Selling'' supports is possible by contracting a loan treaty at the
same time, but remember the limits on loan treaties.

\aparag[Writing actions] When deciding which actions to make, a player should
write all of them in details on his monarch sheet: the turn at which the
action occurs, the country targeted, the amount of money spent (investment)
and the resulting bonus to die roll (as explained below). Writing all this
before actually resolving any of the actions will greatly speed up and
smoothen play.

\aparag A player can make only one action on a given minor country per turn.

\aparag No diplomatic action is allowed on a European minor country that is
fully involved in any war (even a Civil War) even by a major country that is
not part of the war. The only "diplomatic" action allowed on minors at war
with the player is separate peace. There is no such restriction for minors in
limited interventions.

\aparag{Cost of Diplomatic actions}
The costs are the following:
\bparag Basic investment: 20 \ducats
\bparag Medium Investment (+2 to the die-roll): 50 \ducats
\bparag Strong investment (+5 to the die-roll): l00 \ducats
\bparag Support (+1 to the die-roll): 20 \ducats

\aparag Actions must have been written down to be considered as valid.


\subsubsection{Resolution of an action}
\aparag[Order of Resolution] Intended actions are first written down by all
players, then they are announced and then solved, minor by minor, the order of
which being of no importance (choice of minor according to the initiative if
contentions between players), in the following order:
\bparag players decide of their reactions;
\bparag resolve opposed actions (on minor countries already controlled by a
power, or if two powers aim at the same Minor);
\bparag resolve remaining unopposed actions.
\bparag Note that all actions should be announced first, then all reaction
should be decided and only after should the action be resolved. If you start
resolving your actions earlier, don't complain that your opponent bases his
ones actions or reactions on the results of your actions.

\aparag[Reactions by Another Player on a Minor it controls]
\bparag When an action is made on a minor already on the track of a player,
this power may react depending on whether it was also making an action to
increase his own level of control, or not.
\bparag[If the Patron is doing an action] There is no "reaction" investment to
be paid by the controlling player excepts that the player may decide to
immediately raise his level of investment and pay the difference. This level
of investment is paid for his own action and the action will be considered at
the same time as a the "reaction".
\bparag[If the controlling player did not plan to make any action on that
minor] He is then allowed to take a "reaction" on that minor by paying the
investment required. This reaction is in addition to the actions he is
normally entitled for the current turn.
\bparag[If the controlling (i.e. defending) power refuses to make any
reaction] by not paying any investment in reaction, the minor country is
immediately placed in \Neutral position and defends itself according to his
new \Neutral stance.
\bparag Note that the defending player benefits from a bonus applicable to the
die roll according to the degree of control that he exerts on the minor.  This
bonus is reminded to the player's attention at the top of the Diplomatic track
on the map.
\bparag Money spent for reactions (if any) is recorded on
\lignebudget{Diplomatic reactions}.

\aparag[Resolution of Opposed Actions]
If several powers are doing actions (including reaction) on the same minor,
these actions are resolved together at the same time (each player rolls his
die-roll and modifies it). The player that obtained the best result (i.e. the
highest modified result) is selected to proceed further.
\bparag Solve ties by competitive unmodified die-rolls, but the original
result will be used for the resolution.
\bparag If a reaction (that was not originally an action) is the best result,
do not proceed further (no progression point can be gained, the reaction only
served to keep the minor).

\aparag[Resolution of the Action]
The power selected with the best result compares its result to the following
score:
\bparag the score in reaction (even if it was originally a normal action) of
the controlling power if it was opposing the attempt and did not achieve the
best result (only the controlling players can use his score here, not another
player attempting an action on the same minor);
\bparag otherwise, the sum of 2d10 in all other cases.
\bparag The player earns a number of progression points equal to the
difference between his (modified) die roll and this latter result.
\bparag If the difference is null or negative, it does get any points of
progression (there is no "negative" progression).

\aparag[Modifiers] Any player that rolls for this Minor Diplomacy has his
die-roll modified as follows: \diplomod
\aparag[How to read the Diplomatic Values of each Minor]
Each Status (i.e. box) on the diplomatic track has a variable cost of
progression, according to the level of control (status name is printed at the
top of the track) and the concerned minor country.
\bparag Political status \Neutral, \RM and \SUB cost always 1 point of
progression. Exception: to enter the \SUB box for \paysSuisse costs 3 points.
\bparag The cost is variable for the other status according to minor
countries. It is indicated on their diplomatic marker, as well as in the list
of minor countries located in the Appendix handbook
\bparag If a \textetoile figures on the diplomatic marker for a particular
status, it indicates that this political status is not achievable with this
minor country.
\bparag If initials appear instead of a figure (cost), they indicate that only
the country having these initials can reach this political status, under the
restriction that a specific event allowing it has occurred.

\aparag[Diplomatic Markers Adjustment]
Costs of progression indicate the minimum number of points of progression to
advance the counter of the concerned minor on the diplomatic track.
\bparag When all diplomatic actions have taken place, the minor country
diplomatic marker is moved according to the number of points of progression
obtained for that minor and the costs to enter the various status boxes, in
favour of the player having obtained the success on this minor.
\bparag Advancing a diplomatic counter is never mandatory. A player may always
stop the marker progression even if sufficient progression points remain.
\bparag Moving back a marker is mandatory. If the marker reaches the \Neutral
box while doing so and some remaining points of progression are still
available, the marker can then progress in favour of the player that has
succeeded in the action as explained below.
\bparag All points of progression balance that do not suffice to enter into
the box is lost and not applicable.
\bparag The diplomatic marker of a minor country is moved on the track until
it reaches a political status box, as allowed by the number of points of
progression and the various costs to enter those boxes.  If the marker has
progressed, intermediate boxes indications are ignored. Apply only the result
and benefits of the status corresponding to the box where the marker is
located.

\aparag[Handling reactions]
When an action is opposed by a reaction (or in case of a competitive action
lost by the controller), the score need to be compared both to the reaction
score and later to 2d10 (as per regular minor).
\bparag Comparing the action score with the reaction gives a number of
progressions points used to reach \Neutral.
\bparag Once the minor is \Neutral, roll 2d10 for it. Compare the (original)
score of the action with them to get a number of progression points, then
subtract the number of points previously used to reach \Neutral. The result
(if positive) is the number of progression points used to raise the

\begin{exemple}[A simple action]
  At turn 1, \FRA tries to do some diplomacy on \paysSavoie which is already
  in \MA, the French monarch is \monarque{Charles VIII} with a \DIP of 9 and
  he chooses to make a basic investment only. Both \FRA and \paysSavoie share
  the same religion (Catholicism). Thus, the total modifier for \FRA is
  \bonus{+12} (\bonus{+9} for \DIP, \bonus{+1} for religion and \bonus{+2} for
  control).

  \FRA rolls a 3, for a net result of 15. Someone else rolls 2d10 for
  \paysSavoie and gets 6 and 5 for a result of 11. \FRA thus scores 4
  progression points. \paysSavoie is already in \MA, the next box is
  \EC. According to the diplomatic value (in the Appendix), it costs 2 points
  to raise \paysSavoie to \EC. There are still 2 points left. However, raising
  \paysSavoie to \EW would cost 3 extra progression points which \FRA doesn't
  have. So, \paysSavoie stops in \EC and the 2 extra progression points are
  lost.
\end{exemple}

\begin{exemple}[A competitive action]
  At turn 1, both \ANG and \HIS want to make an action on \paysPalatinat
  (which is \Neutral). The three countries are Catholic (\paysPalatinat will
  become Protestant later but it begins Catholic). Both \ANG and \HIS choose
  to make a basic action, their respective \DIP is 7 and 6, thus giving
  modifiers of \bonus{+8} for \ANG and \bonus{+7} for \HIS.

  \ANG rolls 4 for a final result of 12 while \HIS rolls 7 for a final result
  of 14. Thus, only \HIS is allowed to do an action. Someone rolls two dice
  for \paysPalatinat and gets 4 and 9 for a total of 13, to the amusement of
  \ANG. \HIS thus only scores 1 progression point, enough to get
  \paysPalatinat in \RM but no further.
\end{exemple}

\begin{exemple}[A reaction]
  At turn 1, \FRA also wants to try and get \paysPapaute out of Spanish
  hands. Thus, he makes his second diplomatic action on it, still with a basic
  investment resulting in a \bonus{+10} modifier.

  \HIS did not plan any action on \paysPapaute and shocks when he learns about
  the French villainous move, claiming that he is the most Catholic king out
  there and should morally be the only one with ties to the Pope (after all,
  the soon to be elected Alexander VI is Spanish\ldots) \FRA smiles and calmly
  asks if \HIS wants to react to this action or forfeit his illegitimate
  claims on Rome (adding that bringing the Papacy back in Avignon looks like a
  promising idea).

  If \HIS chooses not to react, then \paysPapaute will immediately becomes
  \Neutral and the French action is then resolved normally. However, \HIS
  wants to keep his lead on \paysPapaute and thus chooses to react. It has to
  decide at which investment. Since its \DIP is only 6, a basic investment
  will yield in a \bonus{+8} modifier (\bonus{+1} for religion and \bonus{+1}
  for control), somewhat smaller than the French \bonus{+10}. So, \HIS decides
  to limit the risks and use a medium investment, thus spending an extra
  50\ducats but reaching a \bonus{+10} modifier.

  Both roll a die. \FRA rolls 7 for a total of 17 while \HIS only rolls 1 for
  a total of 11. Thus, \FRA gets 6 progression points. The first one is used
  to bring \paysPapaute back to \Neutral. Then, the rest of the action is
  resolved as against a normal \Neutral (and the extra progression points
  against \HIS are lost). \HIS swear to take his revenge and quickly grab two
  dice, rolling 6 and 8 for a total of 14. \FRA initial total was 17, so he
  has 3 progression points against \paysPapaute, however, one is considered to
  have been already used against \HIS, so there are 2 left, just enough to
  bring \paysPapaute in \SUB.

  \smallskip

  Even if \HIS had initially rolled 9, for a total of 19, higher than \FRA, he
  could not have raised his control on \paysPapaute because this was a
  reaction and not a planed action.
\end{exemple}

\begin{playtip}
  It is more efficient to have all the players simultaneously write down all
  the diplomatic actions they want to do this turn, including the computation
  of the bonus ; then have pair of players (as soon as they are finished) roll
  for their actions (with the other rolling for the minor) and write done the
  result (number of progression points) ; and lastly implement the results
  (going to the diplomatic track and moving the markers, maybe rolling for
  subsidies or dowries.

  This avoids numerous back and forth journeys to the diplomatic track to
  implement the results and speeds the rolling process by pre-computing
  everything (thus requiring less time overall).

  Note also that the influence of the diplomatic actions of other players on
  the immediate other phases (incomes and expenses) is almost null. So, as
  soon as one has resolved ones diplomatic actions, one can begin computing
  ones incomes and thinking about expenses. Only the military phase will
  require further synchronisation between players.
\end{playtip}

\aparag[Reading markers] The cost for entering the different boxes is
specified in the Appendix. Additionally, it is written on the diplomatic
counters for easy reference during game. The front of the counter shows values
for dowry, subsidies and \MA while the back (with the ``at war'' strip) shows
values for \EC, \EW, \VASSAL and \ANNEXION.



\subsection{Effects of the Diplomatic control}


\subsubsection{Royal Marriage}
\aparag The Royal Marriage (\MR) box gives the advantage of a bonus of +1
during any ulterior diplomatic phase as long as the player controlling the
minor country retains this status.

\aparag[The Dowry] When the minor country diplomatic marker reaches the \MR
box by advancing (not by moving back), the player rolls one die. If the result
is:
\bparag Even the player receives the sum of the dowry in ducats as indicated
on the diplomatic marker.
\bparag Odd: the player has to pay the dowry.
\aparag This sum (positive or negative) is written on \lignebudget{Subsidies
  and dowries}.

\aparag if the player refuses to pay the dowry, the marker is returned
immediately to the \Neutral box.


\subsubsection{Subsidies}
\aparag The position of a diplomatic marker on the Subsidy (\SUB) box gives a
bonus of +l during any ulterior diplomatic phase for the player controlling
the minor country.

\aparag[Payment of Subsidies.]
When the minor country diplomatic marker reaches the \SUB box by advancing
(not by moving back), the player rolls a 1d100 He modifies the obtained
die-roll result by the Subsidy modifier (always negative) indicated on the
minor country marker.  If the result is:
\bparag positive: it indicates the number of ducats that the player receives
from the minor;
\bparag negative: it indicates the number of ducats that the player has to pay
to the minor.
\aparag This sum (positive or negative) is written on \lignebudget{Subsidies
  and dowries}.
\aparag If the player refuses to pay, the marker is immediately and directly
returned to the \Neutral box.

\aparag The positive net amount obtained by Subsidies can never exceed 50
\ducats, except explicit precision of the contrary as explained in some
events.

\aparag When a players pays the subsidies, the ducats thus transferred to the
minor are deducted from the player treasury (and just marked-off i.e. there is
no such thing as "minor country treasury").

% \aparag Diplomatic "income" (i.e. positive subsidy) is credited to the
% player TR at the end of the Diplomatic phase.


\subsubsection{Military Alliance}
\aparag The position on the Military Alliance (\AM) box gives a bonus of +2
during any ulterior diplomatic phase for the player controlling the minor
country.

\aparag[Alliance.] As it is an alliance, if the \MIN is declared war upon or
if it declares war, it will call for its patron, that is also an ally.

\aparag[Limited Intervention in wars.]
Conversely, the \MIN is allowed to be involved in a limited way in the wars of
their patron. This declaration is a Reaction, and is shown by placing the
forces of the \MIN on the map. Additionally and as an exception to the rules
of reaction, a limited intervention can be declared at the instant a status of
\AM (or better) is obtained, so at the end of the phase of Diplomacy (and not
at the usual segment where reactions are allowed).
\bparag A limited intervention of a minor country is made only with its basic
forces. It can draw supply only from its own provinces (and so can not go
further than 12 MP from its country).
\bparag Units can not go out of the European map if the minor country has no
\TP/\COL on the \ROTW map. They can not participate in discoveries if it is
not specified for this minor power (mainly \pays{Portugal} and \pays{Hollande}
are allowed).
\bparag In \AM, the intervention is at most of one land stack and one naval
stack outside the minor country.
\bparag The \MIN receives reinforcements each turn in the administrative
phase. The base reinforcement is given in the Appendix. These reinforcements
are only used to recreate the basic force of the \MIN, should they be
diminished.
\bparag The \MIN has a free active campaign each turn, and free passive
campaign each other round. Its Patron may increase the level of the campaign
by paying for this.
\bparag The \MIN is in fact out of the war: its territories can not be
attacked or trespassed if it is only in limited intervention.  The \MIN is not
part of the Peace Treaty that will end the war. The \MIN may withdraw from the
war if its diplomatic status changes.
\bparag A \MIN that is announced in limited intervention in a war offers a
free \CB to the enemy alliance to involve fully the \MIN in the war.

\aparag[Full involvement in wars.] Some events, or declaration of wars may
involve fully the minor country in a war.
\bparag In this case, the status is shown by by putting the Diplomatic marker
of the \MIN on the side reading "At War" and the Diplomatic position is
increased to Entry in War (\EW).


\subsubsection{Expeditionary Corps}
\aparag The position on the Expeditionary Corps (\CE) box gives a bonus of +2
during any ulterior diplomatic phase for the player controlling the minor
country.
\aparag[Alliance.] As it is an alliance, if the \MIN is declared war upon or
if it declares war, it will call for its patron, that is also an ally.
\aparag[Limited Intervention in wars.]
Conversely, the \MIN is allowed to be involved in a limited way in the wars of
their patron. The conditions are the same as in \AM, except that the \MIN in
\CE add one \LD or \ND (controller's choice) to its reinforcements each turn.
\aparag[Full involvement in wars.] Some events or declaration of wars, may
involve fully the minor country in a war.  The conditions are the same as in
\AM.


\subsubsection{Entry in war}\label{chDiplo:EW Effects}
\aparag The position on the Entry in War (\EW) box gives a bonus of +3 during
any ulterior diplomatic phase for the player controlling the minor country.
\aparag[Alliance.] As it is an alliance, if the \MIN is declared war upon or
if it declares war, it will call for its patron, that is also an ally.
\bparag Additionally, the Patron may ask for a full entry in war on the minor
country, as an ally fully involved in the war. This is done during the
announces of Reactions to a declaration of war (as if calling for alliances of
\MAJ).  To participate, a minor must be rolled for and a modified result of 6
or more must be obtained on 1d10.
\bparag
Modifiers to this entry die-roll depend on the country the player wants his
minor to declare war upon. They are the following: \diplowar
\bparag Failure to this test lowers the diplomatic control to \CE immediately,
and forbids the Major power to declare a limited intervention of this Minor
country at the current turn in this war.

\aparag[Limited Intervention in wars.]
Conversely, the \MIN is allowed to be involved in a limited way in the wars of
their patron. The conditions are the same as in \AM, except that the \MIN in
\EG add one \LD or one \ND (controller's choice) to its reinforcements each
turn.

\aparag[Full involvement in wars.]\label{chDiplo:Entry War Minor}
Some events or declaration of wars may involve fully the minor country in a
war.
\bparag In this case, there is no restriction to the manner that the \MIN
conducts the war. The status is shown by putting the Diplomatic marker of the
\MIN on the side reading "At War".
\bparag It maintains up to its Basic Force at the begining of each turn.
Additional forces can be maintained by their Patron.
\bparag It receives reinforcements based on a roll on the Reinforcement Table.
It has, for free, an active campaign for each round, plus some major (or
multiple) campaigns given by the reinforcements table. The Patron may complete
the cost of those to a higher level of activity if need be.
\bparag It will have to sign a Peace Treaty to cease the war (a Separate Peace
or the common Peace Treaty).


\subsubsection{Vassalisation}
\aparag The position on the vassalisation (\VASSAL) box gives a bonus of +3
during any ulterior diplomatic phase for the player controlling the minor
country.

\aparag[Income] Vassal income from provinces, colonies, Trading Posts, exotic
resources and commercial fleets is included in the controlling player's
income, both during war or peacetime.
%\bparag Vassal provinces do not contribute to Manufacture percentage income.
\bparag Their income is added to blocked foreign trade for Foreign trade
income, and count for domestic income.

\aparag The territory of a \VASSAL country is always open to its controlling
power. The allies of this powers and its enemies can pass through the \VASSAL
if (and only if) the Patron has been in it before during the current turn.
\bparag Movements, supply passing through, staying in and battles are
permitted to those countries. The territory is friendly to the controlling
power and its allies, and enemies to others.
\bparag No siege or pillage are possible. The cities are supply sources only
to the Vassal minor country.
\bparag Fortresses may be maintained by the Patron.

\aparag[Alliance.] A \VASSAL is tightly associated to its Patron.
\bparag The controlling power may decide to fully use its \VASSAL in war, or
to declare only a limited intervention, or do nothing (except that the
territory of the \VASSAL is accessible as said above). All those declarations
can be made as reactions at any turn of the war. Once a \VASSAL is fully
involved in a war, it stays so until a Peace is signed.
\bparag The enemies of the Patron can declare during the diplomatic phase that
they fully include any \VASSAL in an existing war: the \VASSAL is now in full
war.  Also, a declaration of war against a \VASSAL is actually a joint
declaration of war against the Patron.
\bparag A \VASSAL can only be involved in war (full or limited way) if the
\VASSAL is at the distance of 12 MP or 4 sea zones from one enemy province of
a Power fully involved in the war.

\aparag[Limited Intervention in wars.] The \VASSAL can be involved in a
limited way in the wars of their patron. The conditions are the same as in
\AM, except that the \MIN in \VASSAL gain no free reinforcement each turn,
including its own basic reinforcements.
\bparag Instead, the Patron may pay for reinforcements, on his own treasury,
to raise troops up to the basic forces of the country.  The maximal
reinforcements so raised are the basic reinforcements indicated in the
Annexes, plus 2 detachments (\LD or \ND).
\bparag All the basic forces of the \MIN can be used.

\aparag[Full involvement in wars.] Some events or declaration of wars may
involve fully the minor country in a war.  Additionally, its Patron or the
enemies of this power may declare at any Diplomatic Phase that the \VASSAL is
now fully involved in the war.
\bparag The conditions are the same as in \EW.
\bparag[Vassals and Separate Peace] A vassal ally never accept to sign a
separate peace unless its capital is under enemy control (and unbesieged by
friendly forces), or it is forced to accept an unconditional peace (when
totally conquered), or its monarch is captured and ransomed for the right to
attempt a separate peace.


\subsubsection{Annexation}
\aparag The position on the Annexation (\ANNEXION) box gives a bonus of +3
during any ulterior diplomatic phase for the player controlling the minor
country.  When the minor country marker is in the AN box of a player, that
country is considered as annexed by the player.
\aparag[Units and Income of Annexed Minors]
All force of an annexed minor are removed, and provinces of that minor are
annexed, although they cannot be considered as national provinces of the
annexing player.
\bparag The player receives all income from annexed provinces as if they were
his own, including for Manufacture percentage income.
\bparag He may build units there as in other non-national provinces.
\bparag For military operations, the annexed country is part of the
controlling power.
\aparag[Condition of Annexation]
To be annexed diplomatically, a minor country has to be adjacent to a province
already controlled by the annexing player, otherwise the diplomatic counter of
this minor cannot be move up to the Annexation box.

\aparag[Dis-annexion]
An diplomatically annexed minor can be dis-annexed if another player succeeds
in moving the diplomatic marker of the \MIN on the diplomatic track, away from
the \ANNEXION box.
\bparag A minor can also be dis-annexed by a Diplomatic Agitation during the
event phase, by a change that could make the marker's present position be
moved one or more boxes.
\bparag Destroyed minor countries (possible by some events, or by rules on
Turkish, Russian or Polish Annexions) are not annexed for this rule: their
diplomatic marker is not put in Annexion and the Diplomatic Agitations do not
affect them.

% Local Variables:
% fill-column: 78
% coding: utf-8-unix
% mode-require-final-newline: t
% mode: flyspell
% ispell-local-dictionary: "british"
% End:

% LocalWords: influenceable minor's
