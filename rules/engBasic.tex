
% -*- mode: LaTeX; -*-

% (Jym), 06/2011: Simplification along two lines of thought: 1/ We address
% experienced wargamers. They know what a province is.  2/ This is a
% descriptive only chapter. Precise effects of the stuff should go in the
% corresponding rule Chapter rather than here. OK, the corresponding rule
% Chapter is usually the Military one and we can wait before getting the rule
% back...

% And also, getting all macros to new style when possible.

\definechapterbackground{Game components}{basics}
\chapter{Game components}\label{chapter:Basics}
\begin{designnote}
  This Chapter describes in details the components of the game, mostly the
  maps and counters.
\end{designnote}

\aparag \emph{Europa Universalis} is composed of:
\bparag Two maps. One depicting Europa and the other depicting the whole
World.
\bparag 3000 (?) counters.
\bparag This 550 (?) pages long book of rules.

\aparag In order to play, you will also need:
\bparag Separate printed version of the Players aids and the various record
sheet.
\bparag Pens and dices (ten-sided dices).
\bparag Some extra blank paper can be handy.

\aparag Players aids and record sheet work best if used the way they were
designed. See \ref{chapter:Playing} for details on this.




\section{Description of the world: the maps}

% RaW: [1]



\subsection{Europa and Rest of the World}

\aparag[Two maps] The world is divided in two distinct maps: the European map,
and the Rest of the World map (\ROTW). Although there is a lot in common to
the way these maps can be read, they do not work in the same way.
% \bparag Both maps feature \terme{provinces}, \terme{cities}, \terme{sea
% zones}, \terme{trade zones}.
% \aparag[Europe-specific features] The Europa map has the following specific
% features: \terme{wasteland area}, \terme{cold area}, \terme{shields},
% \terme{national provinces}.
% \aparag[\ROTW-specific features] The \ROTW map has the following specific
% features: \terme{areas}, \terme{regions}.

\aparag[Europa] The European map is where most of the military game is
played. Each player plays an European country.

\aparag[\ROTW] The \ROTW map is used for the great discoveries and
colonisation of the European powers.
\bparag It also holds the game turn, technology, diplomacy and exotic
resources tracks.



\subsection{Provinces and Sea Zones}\label{chBasics:Provinces}

\aparag[Provinces and Sea Zones] Each map is divided into provinces (on land)
and sea zones (at sea).
% (Jym) EU players are experienced wargammers. Removing: A \terme{province} is
% the smallest land piece on the maps. They are delimited with either black,
% blue or pale lines (representing either frontiers, rivers or mountain
% passes).
% \aparag[Sea zones] A \terme{sea zone} is a piece of sea, delimited by coasts
% (of provinces), white lines (frontiers with other sea zones) or red lines
% (fortified straits). They all have a name (usually a French name) and a
% difficulty (a large number ranging from 4 to 8, accompanied sometimes by a
% +1 or +2 indication).

\aparag[Names] Every European province has two names: the province name, and a
city name (beside a fortress icon).
\bparag Although provinces and cities do not play the same role, since these
names are unique, they can both be used to designate the province.
\begin{designnote}
  The local name is used, as far as it makes sense. Alternatives are sometimes
  written in parentheses. When the local name is too far from something
  readable or recognisable, a French equivalent is written in italic
  typeface. A transliteration is also provided (between square brackets) for
  non-Latin alphabets.
\end{designnote}

\bparag[Disconnected provinces]\label{chBasics:Disconnected Provinces} Some
provinces are in fact several pieces of land, e.g., the provinces
\provinceCyclades or \provinceBaleares (several islands), or
\provinceCanakkale (in Turkey). They are always treated as one province only.
\aparag[Multiple coasts]\label{chBasics:Multiple Coasts} Some provinces
have two coasts that are not connected together (by sea). This is the case of
\provinceSlesvig (in Denmark) and \provinceHellas (in Greece) in Europe, and
some others in the \ROTW. These provinces are subject to limitations when
entering and exiting provinces through different sea zones (see
\ruleref{chMilitary:Movement:Port Multiple Coasts}), and for blockading (see
\ruleref{chMilitary:Movement:Blockading Multiple Coasts}).
% (Jym) Only keeping descriptions here. Rules to be moved in the corresponding
% rule chapter. Removing:
% \bparag One of the coasts is the main coast. This is the coast used for
% blockading. In Europe, this is the one where the anchor lies; in \ROTW, this
% is the one near which is the city if there is one, and decided when placing
% a colonial establishment if there is no city.
% \bparag The other coast is considered to also have a port for disembarking
% only.
\bparag All other provinces have only one coast (that may span over different
sea zones).
% If a port (anchor) is present, sea units do not need movement points either
% to go from one part of the coast to another (while at port). However, there
% are some limitations also when entering and exiting the province for sea
% units (see \ruleref{chMilitary:Movement:Port Multiple Coasts}).

\aparag[Terrain] The colour of a provinces corresponds to its terrain
type. See the terrain chart on the map. Non-plain terrains affect movement and
battle.

\aparag Two regions are magnified on the \ROTW map and one on the European map
for practical purposes: North-Eastern America, India and Belgium-Holland.
% There is no over-cost involved there; the magnification is there for
% practical purposes.
\bparag Provinces and Sea zones are thus present twice: both in and out of the
magnified area. It is advised to use the magnified area for all military
counters and to keep economic counters out of the way on the un-magnified map.
\bparag Sea zones going around the magnified areas have all their contacts
shown. Especially, there is no contact between \seazoneMaldives and
\seazone{Indien S}, or \seazoneQuarantiemes and \seazoneIndien, but there is
between \seazoneQuarantiemes and \seazone{Indien S}.

\aparag[Frontiers] Adjacent provinces can be connected by river (blue),
mountain pass (pale), regular frontier (black) or straits (double-arrows).
% Two provinces are adjacent by frontier if they share a common black frontier
% in any place. They are adjacent by river or mountain pass if they share a
% common frontier (either blue or pale), and no black frontier.
\bparag[Straits, mountains pass and rivers] affect movement and battle. All
terrain effects (both from the province and the frontier) are cumulative.
% give a penalty to attack a province just behind the river or strait, and
% cost more \terme{points of movement} (PM).
\aparag[Lakes] Lakes are impassable and do not provide contact between
provinces.
\bparag Sea units cannot go through rivers or lakes.
% , except the Russian Fluvial Fleet per \ruleref{chSpecific:Russia:Fluvial
% Port}.
% \bparag Two provinces separated by a strait are adjacent, even if they do
% not touch. The straits are marked on the map by large arrows pointing at
% both provinces.
% \bparag Passing from land to land through a strait or a river has some
% effect on movement and combat. Passing from sea to sea through a strait has
% no special effect.
\aparag[Reaching the Bering Strait.]\label{chBasics:Secret Passage:Bering}
\granderegionKamchatka is considered adjacent by land with any of the four
provinces neighbouring the impassable area north of \seazoneOkhotsk.
% Supply is not provided through this passage.
It takes one full campaign round to make this move (12\MP).
% \bparag[Following rivers] On the \ROTW map, there is special interest to
% know if one path can be made along a river. The path must be done from
% provinces to provinces so that two consecutive provinces of the path always
% share a common blue frontier (either a river or a lake), and the water must
% be continuous all along. Crossing the water (if a river) is allowed (even
% several times).
\aparag[Small provinces] Some islands and similar places are too small and are
thus represented by a large square instead.
\bparag When the corners of the square are cut off, it means the island is on
rough terrain (usually forest). Similarly, the flag in the island is white for
plains and has a black cross for non-plain terrain.
\bparag The island of \province{Cap Breton} in \granderegionAcadie is
connected with its neighbouring province with a river.
\bparag\label{chBasics:Provinces:Ormus} \null\provinceOrmus is an
island in \seazonePersique. It is connected with a strait to \provinceBam and
\province{Oman E}. It can be seen on the European map, but all counters have
to be put on the \ROTW map. A fortress in \provinceOrmus is also a \Presidio
for \provinceBam.
\begin{exemple}[Frontiers and islands]
  \provinceNormandie and \provinceCaux are adjacent provinces by river,
  whereas \provinceNormandie and \provinceMaine are adjacent by frontier.

  % A path going from the southernmost province of \granderegion{Kansas} to
  % the south-east province of \granderegion{Texas} is not along a continuous
  % path of water if going through the northernmost province of
  % \granderegion{Texas} directly then in \granderegion{Kansas}.
%
  % A path going from the same province of \granderegion{Kansas} to the
  % south-west province of \granderegion{Mississippi} by way of the
  % northernmost province of \granderegion{Texas} is made by following a
  % continuous river (and the cost of entering the two provinces will be
  % reduced).
  The southernmost province of \granderegion{Grands Lacs} is not adjacent to
  the north-eastern province of \granderegionIllinois, even though they share
  a lake border.

  The island of \granderegion{Sainte-Helene} is a plain (regular square and
  white flag) while the island of \provinceGuadeloupe is a forest (cut off
  square and crossed flag).
\end{exemple}

% \aparag[Terrain type] The terrain type is shown by the colour of the
% background of the province. It is the same in all the province, and affects
% combat and movement.
% \bparag[Plains] Plains are in light yellow-grayish colour. Easy movement and
% cavalry bonuses are available. They are the only clear terrain. Some
% islands, too small to show terrain colour, have a triangular grey flag atop
% to indicate plains.
% \bparag[Mountains] Mountains are dark red in colour. They allow for easier
% defence.
% \bparag[Deserts] Deserts are yellow. They give penalties to both attack and
% defence, and provide only with a weak line of supply.
% \bparag[Swamps] Dark green-grayish provinces are swamps or marshes, very
% similar to mountains.
% \bparag[Forests] Green (either yellowish or pure) provinces are
% forests. They give penalties to both attack and defence. Yellowish green
% forests are normal forests. The others are sparse forests. The forests
% included in the \terme{Cold area} are called northern forests. The latter
% have some specific effect (see~\ruleref{chMilitary:Battle:Forests}). Some
% islands, too small to show terrain colour, have a rectangular green crossed
% flag atop to indicate forests.  Some swamp provinces can be flooded (see
% \ruleref{chSpecific:Holland:Flooding}).
% \aparag[Labels] On every European province can be found three labels: the
% province name, a city name (beside a fortress icon), an income value. There
% can also be several icons pertaining to the province: a coloured anchor, one
% or more shields, a gold mine, a salt resource. The province name is written
% in a large bold typeface.

\aparag[European provinces in \ROTW]\label{chBasics:Europe Provinces ROTW}
The \ROTW map bears some European provinces: \provinceHerat (as long as in
belongs to \paysPerse), \provinceCanarias and \provinceAcores. Those provinces
behave in every way as European provinces including the cost of movement,
income, etc.



\subsection{Symbols in provinces}

\begin{designnote}
  Of course, not everything could be shown through symbols on the map. For
  instance, annexation of Scotland to England has not been shown. In a word,
  symbols on the map are not the rules.
\end{designnote}
\aparag[Income value] This is the large number written in or beside the
province name (in Europe).
% , or a common value for a whole area (in \ROTW). It is used to evaluate the
% economic strength of the owner of the province.
% \bparag Only one province may change its value during the course of the game
% (the one where \ville{Saint-Petersbourg} will be built, see
% \ruleref{chSpecific:Russia:St-Petersburg}).
\aparag[Anchors] Anchors indicate that the province has ports (in Europe).
% This allows for easier movement between land and sea, easier defence of the
% cities, (not blockading the ports severely reduces chances of conquest),
% building of \emph{fisheries} in the province, fleet construction, etc.
% \bparag In \ROTW, all provinces containing a \COL, a \TP or a Fort is
% considered to have a port on all its coasts.
\bparag All coasts in a province with port are considered to have ports, even
if the anchor symbol does not touch all coasts. %This is especially
% the case for disconnected provinces, such as those quoted in
% \ruleref{chBasics:Disconnected Provinces}.
% \bparag See above (\ruleref{chBasics:Multiple Coasts}) for the provinces
% whose coasts span over several sea zones.
\bparag Anchors with a white circle are ports that can be blockaded with a
\Presidio% (a fortress specially designed to block the port, even if the
% province is not owned).
% \bparag A port cannot be built during the course of the game, except for
% \ville{Saint-Petersbourg}, neither can it be destroyed.
\bparag[Golden anchors]\label{chBasics:Arsenal} A golden anchor indicates
an \terme{arsenal}, a larger port that can hold and supply larger
fleet. %Two specific arsenals can be built in the course of the
% game in Europe (one by Russia, one by England), and some can be built on the
% \ROTW map. Arsenals can be dismantled as any fortress.
\bparag[Red anchor] \province{Cabo Verde} has a red anchor. This red anchor
means that sea units of the player that owns the province may pass from one of
the four connected sea zones to any other without paying the movement inside
the province.
\begin{figure}
  \def\imx#1{\includegraphics[height=0.5in]{#1}}
  \centerline{\begin{tabular}{|ccc|b{1.8in}|}\hline
      \rule{0mm}{0.6in}\imx{ville1}&\imx{villechine1}&\imx{villeinde1}&Cities, level 1\\
      \imx{ville2}&\imx{villechine2}&\imx{villeinde2}&Cities, level 2\\
      \imx{anchor}&\imx{anchor2}&\imx{anchor3}&Port, Arsenal, Waypoint\\
      \imx{anchor4}&\imx{anchor5}&\imx{blocus}&Port, Arsenal +possible \Presidio, Strait fortification\\
      \imx{shield_oresund}&\imx{shield_flandrebrabant}&\imx{ancrefluviale}&Some special shields, Fluvial Port\\
      \imx{bateau}&\imx{sel}&\imx{mine}&Sea/Country Trade Zone, Salt resource, Gold mine\\
      \hline
    \end{tabular}}
  \caption{A sample of possible symbols on the map}%
  \label{figure:map-symbols}
\end{figure}
\bparag[The Strait fortifications] Red see frontier represent permanent
\StraitFort guarding entrance to the sea. They are controlled from the
province with the guard tower symbol (see \ruleref{chMilitary:Strait
  Fortifications}).
% In some places near red sea frontiers some guard towers are pictured. They
% represent the permanent \StraitFort that guard the access to these seas
% \ruleref{chSpecific:Turkey:Dardanelles} and
% \ruleref{chSpecific:Venice:Corfu}).

\aparag[Shields] The shields bearing (somewhat simplified) arms of countries
are here as a reminder of several things, related to the relation between
provinces and countries.
\bparag[Major countries] Shields bearing the symbol of major countries define
the \terme{national provinces} of the major country. They may differ from the
initial setup.
\bparag[Minor countries existing in 1492] Shields bearing the arms of a minor
country recall the initial possessions of the minor country.
\bparag[Influence] Blurred shields represent influence of a country (either
major or minor) on a province. These are not national provinces.
% Some blurred shields simply indicate a group of provinces. This is the case
% for Duchy of Prussia, Pommerania, Ireland, Finland, Granada, Naples,
% Kurland, and some others. Some of these groups may revolt against their
% owner and be granted Independence (see \ruleref{chPeace:Peace:Independence
% Revolt}), others have special effects according to events.
\bparag[Commercial reminders] Half-parted shields in some provinces and sea
zones are reminders of
% \provinceBrabant, \province{Vlaanderen}, \province{Sjaelland},
% \province{Skane} and \province{Vastergotland}, as well as several see zones
% are used as reminder of
the commercial specificities of the Scheldt and the Sund
(see~\ruleref{chSpecific:Scheldt} and~\ruleref{chSpecific:Sund Levies}).
\bparag[Tordesillas reminders] On the \ROTW map, some shields on areas
indicate the Tordesillas belonging of the area
(see~\eventref{pI:Tordesillas}).
% Spanish belongings west of Europe and Portuguese belongings east of
% \province{Acores} are not recalled. Question marks reminders are for areas
% that could be colonised by either \POR or \SPA.
% \bparag[Major countries] For major countries (played throughout the game),
% the shields indicate the \terme{national provinces}, the core of the major
% country (see \ruleref{chThePowers:National Provinces}).
% \bparag[Blurred major countries] Some shields of major countries are
% blurred. The provinces showing those are not part of the national provinces,
% but may become part of them through events.

\begin{exemple}[Shields]
  The island of \provinceGotland (in Baltic sea) has a Swedish shield. It is
  thus a Swedish national province even if is does belong to Denmark at the
  beginning of the game. \provinceSkane has both a Swedish and a half-parted
  shield. It is a Swedish national province (Swedish shield) \textbf{and} it
  plays a role in the Baltic trade (half-parted shield also present in
  \seazoneBaltique).

  \provinceKreta has a blurred Venetian shield. It is \textbf{not} a Venetian
  national province (the shield is blurred) but Venice does have some
  influence here (in this case, namely, it owns the province in 1492).

  \provinceKuban (East of the Black Sea) has both a Georgian shield and a
  blurred Crimean shield. It does belong to \paysGeorgie in 1492 (shield of a
  minor country) and it may be annexed by \paysCrimee (blurred shield
  indicating influence).

  \provinceHinterpommern (Northern Germany) initially belongs to
  \pays{teutoniques2} (regular shield). It can be annexed by \paysHanse
  (blurred shield) and it can becomes part of \region{Duche de Prusse} (the
  other blurred shield). It can also be part of \paysPommeranie but this was
  not shown on the map.
\end{exemple}

\aparag[Gold mines] Some provinces hold gold mine. If there is a number in it
(in the \ROTW), this is the income of the mine.
% Some gold mines are placed on the map. They add gold to the national
% treasury of the country possessing the province. The may become depleted
% during the course of the game.
\aparag[Salt resources] Salt heap in Europe allow the construction salterns
(salt manufactures) in the province.% No other effect apart from that.


\subsubsection{Wasteland area}\label{chBasics:Wasteland}
\begin{histoire}
  This area represents the great size and the low density of population in
  those regions. This applies to movement and supply. This particularity will
  not extend to newly conquered areas, nor will the provinces lose their
  specificity if conquered by some country other than Khanates, Cossacks and
  Russia. The armies of the local countries had sufficient cavalry to
  compensate for those very large sullen areas.
\end{histoire}
\aparag[Geographical limits] The north-eastern provinces of the map are
bordered with a yellowish line. They represent the initial territories of
Khanates, Cossacks and Russian principalities, as well as some Lithuanian and
Ukrainian territories.


\subsubsection{Cold Area}\label{chBasics:Cold Area}
\aparag[in Europe] All provinces within a whitish line on the North of the map
form the European cold area.
\aparag[in the \ROTW] Areas with a snowflake ``exotic resource'' are all part
of the \ROTW cold area. This corresponds to northern America and Siberia.


\subsubsection{Germany, Italy, Persia}
\aparag The provinces of the Holy Roman Empire are bordered with a red
line.% Specifics rules apply for the countries within this limit.
\aparag The provinces of Italy are bordered by a blue line.% Specific rules
% apply for the countries and provinces within this limit.
\aparag The provinces of the \terme{Persian core}, the heart of Persia, are
bordered by a black line.



\subsection{Cities}

\aparag Cities represent the urban infrastructure of the provinces. There is
one city in each European province as well as in some \ROTW provinces.
\bparag Each city has a name.% It is written on the map using the same script
% conventions as province names, but with a normal typeface.
\bparag Each province actually contains many cities, however, only one of them
is used in the game (and represents all).
\aparag[Level] Cities are fortified with a fortress of level 1 (single tower)
or of level 2 (bunch of towers). Counters are used to mark higher levels.
% \bparag Fortresses can only lowered to a level less than what is on the map
% after a siege (with a minimum level of 1).  Unmaintained fortresses will
% lower down to the initial level of the city.
\begin{designnote}
  Sometimes, the cities simply did not exist in 1492, at the beginning of the
  game. Since it is in those cases rarely important, a more recent city was
  chosen. Cities built after the 17\up{th} century have been avoided. Also,
  keep in mind that a city represents the whole urban infrastructure of a
  province. Sometimes taking a city will really mean ``take a bunch of cities
  all hidden deep in the mountains''.

  In the \ROTW, cities are actually places where European control can be
  exercised, and conquest done; we do not mean to say that India or China were
  empty of cities (this would be most untrue), only that most places would not
  lend themselves to conquest.
\end{designnote}


\subsubsection{Control of a province}
\aparag Each province is owned by one country.
\bparag Change of ownership can only occurs by formal annexation (usually at
the end of a war, sometimes by wedding or other events).
\aparag During wars, provinces can also be \emph{controlled} by a country
different from their owner.
\bparag Control of a province occurs by taking and holding the city of the
province.
\aparag It is possible and common for a given province to be owned by a
country, but controlled another country.
% \aparag A province is controlled for the sake of movement if the main land
% (not the city, mission, colony, trading post) is occupied only by one's own
% units.
% \bparag However, for all other rules (victory points, income, events),
% except if specified, control of a province needs also control of the city
% and other establishments.
% \bparag Possession (or ownership) of a province is distinct from
% control. Possession can be lost only by formal annexation of the province
% (either by treaty or by event).



\subsection{Symbols in Sea Zones}

\aparag Each sea zone has a \emph{difficulty}. It is the number written in the
picture of a storm (or calm sea).

\aparag Some sea zones in the \ROTW have an additional \emph{malus}, either
\bonus{+1} or \bonus{+2} making travel through them even harder.

\aparag Some sea zones are also trade zones. The trade zones are depicted with
the silhouette of a ship in a coloured square (\emph{Sea Trade Zone}) or
circle (\emph{Country Trade Zone}).
\bparag The numbers in the silhouette are the incomes of the trade zone.



\subsection{\ROTW provinces}\label{chBasics:ROTW Areas}

\aparag On the \ROTW map, provinces are grouped in \terme{areas}. All the
provinces of an area share some characteristics: income value, colonisation
difficulty, trading-post implantation difficulty, initial number of natives
and exotic resources.
\bparag All provinces of the same area are grouped by a coloured line. The
characteristics of this area are written in a box of the same colour near the
area.
% \bparag They also share a common set or resources, shared throughout the
% area. They are available for the whole area, not for each province.
% \bparag Cities, gold mines and counters are not shared in the area, and
% belong to one province only.
\aparag The three numbers are (in order) the income, difficulty, and tolerance
of the area.
\bparag Each province of the area has these numbers.
\bparag If an area has no tolerance, use its difficulty whenever tolerance is
required.

\aparag The strength of natives in the area is written below the soldier
picture.
\bparag The choice of soldier (Indian, Zulu, Samurai, \ldots and a couple of
Easter eggs) is purely decorative and has no influence on the game.
\bparag Natives are present in each province of the area with the same
strength.
\aparag Exotic resources are depicted with symbol (for the type of resource)
and numbers (for the quantity of such resource).
\bparag Exotic resources are shared by all the provinces of the area. They are
not present in the same amount in each province. They can be exploited from
any province of the area and countries will need to agree (or fight\ldots) if
several of them want to exploit the same resource.
\bparag The snowflakes are not an exploitable resource but design the cold
areas in the \ROTW. The higher the number, the colder the area.
% \aparag[Movement] Movement cost (which also applies to supply) is increased
% in \ROTW (see \tableref{table:Movement Costs}).
% \aparag[Cities] Not all provinces have cities in the \ROTW. Level 6 colonies
% act like cities, esp. for movement, supply, construction.
% \bparag Movement is reestablished to its European values in provinces where
% there is a city.
\aparag[Round the world] The sea zones \seazone{Pacifique SE} and
\seazone{Pacifique NE} are adjacent to \seazonePacifique.



\subsection{Tracks}

\aparag The \ROTW map also holds several game tracks.
\bparag There are two diplomatic tracks, one for the European diplomacy with
one line per major country, and one for the \ROTW diplomacy with one box per
\ROTW minor country. Diplomacy works differently in Europe and in the \ROTW.
\bparag The exotic resources tracks are use to store both the total amount of
exotic resources exploited (by type) and the price of them. This information
is updated once per turn.
\bparag The technology track keeps both the technological level of countries
(both major and minor) and the technology goals to be reached (they will
move).
\bparag The turn track is coloured by periods. It can also be used to hold
those counters that only come into play at precise time (mostly historical
leaders).




\section{Counters}

\begin{todo}
  Add images of more or less all type of counters (not only leaders).
\end{todo}

% RaW: [2]

\aparag[Types of counters] There are several types of counters: Military (land
and sea units, fortresses) ; Leaders ; Economical (colonies and trading posts,
trade fleet, manufactures, gold mines, trade centers) ; Military markers
(control and ownership, siege-works, revolts, pillages) and Game markers
(diplomacy, technology and exotic resources).
\bparag The military markers are in unlimited quantity. If you need more of
them than provided, use whatever you think convenient to represent them.
\bparag All other counters are in quantity limited by the game. If you need
more than you have, too bad but you can't create them.

\aparag[Levels and side] Many two-sided counters act as ``containers'' for
smaller counters (military) or abstract economical stuff.
\bparag These counters can hold a certain number of ``levels'' (usually 2, 4
or 6).
\bparag They are used on the side marked \Facemoins if they hold half or less
than their maximum level and on the side \Faceplus if they hold more than
half.



\subsection{Military counters}

\aparag All these counters are intensionally in a limited amount. If you don't
have enough, you can't do what you intended.
\bparag \textbf{Exception:} \pays{rebelles} (both ``Rebellion'' and
``Revolt'') and \pays{pirates} counters are in unlimited amount.


\subsubsection{Land units}
\aparag[Detachments] The basic land unit is the land detachment (\LD). It
contains both infantry and cavalry.

\aparag[Armies] Army counters (\ARMY) contain both infantry, cavalry and some
artillery.
\bparag An \ARMY\facemoins is always exactly 2\LD (plus some artillery). An
\ARMY\faceplus is always exactly 4\LD (plus artillery).
\bparag \ARMY can be broken up at almost any time. The result is a number of
counters representing the same number of \LD without creating new \ARMY
counter.
\bparag Thus, an \ARMY\faceplus can be broken into an \ARMY\facemoins and
2\LD, or into 4\LD but never into 2\ARMY\facemoins.
\bparag An \ARMY\facemoins can be reinforced by 2\LD and turned into an
\ARMY\faceplus.
\bparag It is never possible to merge several \LD into a new \ARMY counter.

\aparag[Exploration detachments] In the \ROTW only, it is possible to break a
\LD into 3 land detachments of exploration (\LDE) as a result of battle or
attrition.
\bparag Conversely, 3\LDE can be merged into a \LD.
\bparag \LDE cease to exist the moment they enter a province or sea zone on
the European map.
% So, they can exists in Canaries...

\aparag[Army class] The roman number on each land unit it its \terme{Army
  Class}.
\bparag Army class is an abstract representation of the military doctrine of a
country. It plays a huge role during battle (it mostly represents relative
size of troops, as well as quantity and quality of artillery and cavalry).
\bparag The Arab number (or letter) is an identification number of the counter
and plays no in-game role.
\bparag The image has purely decorative function and plays no role whatsoever,
although armies of the same class tend to have similar images.

\aparag[Militia and natives]
\bparag The white counters represent colonial militi\ae{} and are used when
not at full strength (to keep track of the current strength).
\bparag Similarly, the \pays{natives} counters are used to keep track of
reduced native strength.


\subsubsection{Sea units}
\aparag[Detachments] The basic sea unit is the naval detachment (\ND). A naval
detachment is roughly 3 or 4 ships of the line plus accompanying smaller ships
(depending on the period).
\bparag Notice that \ND are on the back of \LD. Thus, creating a \ND \emph{de
  facto} reduce the number of available \LD. This is intended.

\aparag[Galleys] In the Mediterranean and Baltic seas, it is also possible to
use galleys detachments (\NGD).

\aparag[Transports] Naval transport detachments (\NTD) contain only transport
ships. They may not participate to battles but can be used to carry gold or
troops.

\aparag[Fleet] A fleet (\FLEET) counter is only a container of a certain
number of \ND (or \NGD) and \NTD.
\bparag The exact countenance of a \FLEET counter varies depending on the
period and the country.
\bparag It is always possible to break a \FLEET into its components (\ND and
\NTD).
\bparag It is always possible to group some \ND (and \NTD) into a \FLEET
counter, even if the counter is not full. It is even allowed (but usually
unwise) to create a \FLEET with a single \ND.

\aparag[Exploration detachments] As a result of battle or attrition, a \ND can
sometimes be broken into three naval detachments of exploration (\NDE).
\bparag Thus, a \NDE is 1 or 2 ships.
\bparag \NGD can never be broken into \NDE.
\bparag 3\NDE can always be merged into a \ND.
\bparag \NDE can exist both on the \ROTW and European maps.

\aparag[Privateers] (\corsaire) are smaller ships armed to harm enemy trade.
\bparag They are not military units per se but whenever needed, each side of a
\corsaire is considered to be equivalent to 1\ND (e.g. for hierarchy purpose).

\aparag The image and number on \FLEET and \ND counters are here for
identification purpose only.
\bparag The identification number can be used to keep track of the content of
each \FLEET on the corresponding record sheet.


\subsubsection{Fortresses}
\aparag[Levels] Fortresses can be of level 1 to 5.
\bparag Contrary to many counters, there is one different counter for each
level.
\bparag Fortresses of high level cannot be built at the beginning of the game
and have different conditions before becoming available.
\bparag Since the counters are double-sided, building a fortress of a given
level usually prevents another one from being built. Typically, building a
level 5 fortress prevents the country from building the level 4 fortress on
the back of the counter.

\aparag[Permanent fortress] Each city (in Europe or not) also holds a fortress
of level 1 or 2. These are permanent fortresses and no counter is needed to
represent them.

\aparag[Forts] Fort are considered as fortresses of level 0. They can only be
built in the \ROTW.

\aparag[Level 1] fortresses also exists as generic (white) counters. These are
in unlimited amount (make more if needed) and are used \textbf{only} to depict
temporally diminished fortresses during wars.
\bparag Each country has its own set of level 1 fortresses in case it needs a
permanent level 1 fortress (usually, in the \ROTW).



\subsection{Leader counters}


\subsubsection{Values of leaders}
\aparag[Categories and name] Leaders all have a symbol depicting their
\terme{category} (e.g. general, admiral, \dots) The category indicates which
actions a leader can do (e.g. an admiral leads fleet, not armies).
\bparag At the top of the counter, the name of the leader is written. It can
be either a real name (for historical leaders), a generic name (such as
``King''), or a \anonyme\ (for anonymous commanders).
\bparag The colour of the counter, as well as the shield on the right,
indicates which country the leader serves.

\aparag[Life and death] On the left of the counter of historical leaders are
two numbers. They indicate the turns at which the leader is available.
\bparag The leader is active during all these turns. Thus a leader with the
numbers ``3-7'' is available from the beginning of turn 3 to the end of turn
7.
\bparag Historical leaders may die during battle and thus become unavailable
earlier than what is indicated on the counter.
\bparag Some leaders have instead a first number in Roman number
(e.g. ``III-20''). These are leaders arriving into play by an event and the
number identifies this event (in this case, \ref{pIII:Mughal Akbar}).

\aparag[Values] The bottom of the counters holds the proper values of the
leader.
\bparag The letter is the \terme{rank}. The earlier in the alphabet, the
higher the rank (i.e. rank ``A'' is better than ``B'' and so on).
\bparag The three numbers following the rank are the values of (respectively)
\terme{manoeuvre}, \terme{fire} and \terme{shock}. They are use during
movement and battle and are often the most important piece of information
concerning a leader. They range between 1 (sometimes 0) and 6.
\bparag Some leader have a fourth value (between 1 and 4), the \terme{siege}
value.

\aparag[Modifiers] Some leaders have optional modifiers on the right of the
counter. See below for their meaning.
\bparag Special powers that only exists for leaders of one country are shown
using a different colour for the symbol depicting category.

\aparag[Pachas] are special Turkish leaders. See
\ref{chSpecific:Turkey:Pashas} for details.


\subsubsection{Symbols on leaders}
\aparag Information on the leader counters can be read as shown in
\figref{figure:leaders-counters}.
\aparag \hypertarget{link-meaning-leaders}{The meaning} of the various symbols
that define the leaders in this rulebook is as follows:
\bparag For the main category: \LeaderA (admiral), \LeaderC (conquistador),
\LeaderE (explorer), \LeaderG (general), \LeaderI (engineer), \LeaderK (king),
\Leaderd (admiral-king), \LeaderP (privateer), \LeaderGov (governor)
\bparag For the optional marks: R (allowed in the \ROTW), \textdollar{}
(allowed only in \continentAmerica), * (main side of the counter), @ (allowed
only in \continentAsia), P (is also a privateer), m (allowed only in the
\regionMediterranee), $\heartsuit$ (does not die at the first failed survival
test in battle).
\bparag The country-specific powers are: Portuguese Viceroy (red category),
English Sea hound (yellow category), French Licensed privateers (red category)
and Dutch Indonesian Conquistadors (red category).
\bparag Two leaders (\leader{Marlborough} and \leader{Friedrich II}) have
their name and values written in a different colour (white instead of black or
yellow). They have a bonus (actually, an absence of malus) to their survival
tests in battle.
% \bparag The numbers in brackets are for the turn of entrance and exit, or
% the event that triggers the entrance of the counter.
% \bparag The letter is the rank, then the statistics for Manoeuvre,Fire,Shock
% (and Siege, if applicable).

% \aparag[Double-sided leaders] The double-sided leader counters can be of two
% kinds: the two sides can either be two functions of the same leader
% (sometimes differing in statistics or category), or they can represent two
% different states of the leader (such as a change of nationality).
% \bparag In the first case, there is often a * that marks the main side of
% the counter; this is the category used for all limits considerations. The
% player using this leader may however choose to switch sides according to
% \ruleref{chMilitary:Double Sided Leaders}.
% \bparag Sometimes, the first case does not need a * because the category is
% the same on both sides. The side used is then chosen at random when the
% counter is first used until the leader either dies or becomes unnecessary
% (this is the case for minor kings that can have two different warfare
% statistics).
% \bparag In the second case, the side of the counter often depends on events,
% given in the description of the counter. The names on the two sides may be
% different, and the rules may refer to one side independently of the other.

\aparag[Double-sided leaders] Some leader counters have two sides (both
representing the same individual).
\bparag In most cases, one of the sides bears a *. This denotes the main side
of the leader.
\bparag Under no circumstances the two sides of a counter may be used at the
same time.
\bparag When the leader dies while on one side, and unless special rules
specifically counter this rule, the leader is definitely dead (for all sides).
\bparag See~\ref{chMilitary:Double Sided Leaders} for details.
\def\image(#1,#2)#3{\node (#3) at ($ (#1,#2) $)
  {\includegraphics[width=.5in]{counter_7/#3.png}};}
\def\legendedouble#1#2#3{\node[outer sep=0pt,inner sep=0pt,text=black,text
  centered,text width=1.1in] (lastleg) at ($
  .5*(#1_recto)+.5*(#1_verso)+(0,#2) $){#3};}
\def\legendesimple#1#2#3{\node[outer sep=0pt,inner sep=0pt,text=black,text
  centered,text width=.8in] (lastleg) at ($ (#1)+(0,#2) $){#3};}
\def\legendelongue#1#2#3{\node[outer sep=0pt,inner sep=0pt,text=black,text
  centered,text width=1.15in] (lastleg) at ($ (#1)+(0,#2) $){#3};}
\def\pointe(#1,#2)#3{\draw[draw opacity=.5,very thick] (lastleg)--($ (#3) +
  (#1,#2) $);}
\begin{figure}
  \begin{center}
    \begin{tikzpicture}
      \image(0,0){LeaderDouble_porviceroy_Albuquerque_recto}
      \image(2,0){LeaderDouble_porviceroy_Albuquerque_verso}
      \legendesimple{LeaderDouble_porviceroy_Albuquerque_recto}{1.3}{\textbf{Conquistador}}
      \legendesimple{LeaderDouble_porviceroy_Albuquerque_recto}{.9}{Viceroy}
      \pointe(.55,-.05){LeaderDouble_porviceroy_Albuquerque_recto}
      \legendesimple{LeaderDouble_porviceroy_Albuquerque_verso}{1.3}{\textbf{Explorer}}
      \legendesimple{LeaderDouble_porviceroy_Albuquerque_verso}{.9}{Privateer}
      \pointe(.55,-.05){LeaderDouble_porviceroy_Albuquerque_verso}
      \legendedouble{LeaderDouble_porviceroy_Albuquerque}{-1.1}{Double-sided
        counter\\Main side has a *} \image(4,0){Leader_angseahound_Drake}
      \legendesimple{Leader_angseahound_Drake}{1.3}{\textbf{Admiral}}
      \legendelongue{Leader_angseahound_Drake}{-1.3}{Manoeuvre/Fire/Shock}
      \pointe(0,-.5){Leader_angseahound_Drake}
      \legendelongue{Leader_angseahound_Drake}{.9}{Sea hound}
      \pointe(.55,-.05){Leader_angseahound_Drake}
      \image(6,0){Leader_turquie_Sirocco}
      \legendesimple{Leader_turquie_Sirocco}{1.3}{\textbf{Privateer}}
      \legendelongue{Leader_turquie_Sirocco}{-.9}{Mediterranean only}
      \pointe(.55,-.15){Leader_turquie_Sirocco}
      \legendesimple{Leader_turquie_Sirocco}{.9}{Name}
      \pointe(0,.5){Leader_turquie_Sirocco} \image(8,0){Leader_usa_Arnold}
      \legendesimple{Leader_usa_Arnold}{1.3}{\textbf{General}}
      \legendelongue{Leader_usa_Arnold}{.9}{In/Out turns}
      \pointe(-.5,.25){Leader_usa_Arnold} \pointe(-.5,-.15){Leader_usa_Arnold}
      \legendelongue{Leader_usa_Arnold}{-1.3}{America only}
      \pointe(.5,-.25){Leader_usa_Arnold}
      \legendesimple{Leader_usa_Arnold}{-.9}{Rank}
      \pointe(-.5,-.5){Leader_usa_Arnold} \image(10,0){Leader_france_Vauban}
      \legendesimple{Leader_france_Vauban}{1.1}{\textbf{Engineer\\(general)}}
      \legendelongue{Leader_france_Vauban}{-.9}{Siege value}
      \pointe(.55,-.5){Leader_france_Vauban} \image(0,-3.2){King_russie}
      \legendesimple{King_russie}{1.1}{\textbf{King\\(General)}}
      \legendesimple{King_russie}{-.9}{Title} \pointe(0,-.5){King_russie}
      \image(2,-3.2){King_venise_doge}
      \legendesimple{King_venise_doge}{1.1}{\textbf{King\\(Admiral)}}
      \legendesimple{King_venise_doge}{-.9}{Country}
      \pointe(0,.5){King_venise_doge} \pointe(.5,.35){King_venise_doge}
      \image(4,-3.2){Leader_mogol_Akbar}
      \legendesimple{Leader_mogol_Akbar}{1.3}{\textbf{Minor King}}
      \legendelongue{Leader_mogol_Akbar}{-.9}{Asia only}
      \pointe(.55,-.15){Leader_mogol_Akbar}
      \legendelongue{Leader_mogol_Akbar}{.9}{Entry event}
      \pointe(-.5,.25){Leader_mogol_Akbar}
      \pointe(-.5,-.15){Leader_mogol_Akbar}
      \image(6,-3.2){Leader_espagne_Toledo}
      \legendesimple{Leader_espagne_Toledo}{1.3}{\textbf{Governor}}
      \legendesimple{Leader_espagne_Toledo}{-.9}{ROTW}
      \pointe(.5,-.25){Leader_espagne_Toledo}
      \image(8,-3.2){Pacha_timar_Amar_recto}
      \legendesimple{Pacha_timar_Amar_recto}{.9}{Normal}
      \image(10,-3.2){Pacha_timar_Amar_verso}
      \legendesimple{Pacha_timar_Amar_verso}{.9}{Corrupted}
      \legendedouble{Pacha_timar_Amar}{1.3}{\Pasha counter}
      \legendesimple{Pacha_timar_Amar_verso}{-.9}{Pasha id}
      \pointe(-.45,.25){Pacha_timar_Amar_recto}
      \pointe(-.45,.25){Pacha_timar_Amar_verso}
      \legendesimple{Pacha_timar_Amar_recto}{-.9}{Force (\LD)}
      \pointe(.55,-.5){Pacha_timar_Amar_recto}
    \end{tikzpicture}
  \end{center}
  \caption{A sample of all leader counters possible}%
  \label{figure:leaders-counters}
\end{figure}



\subsection{Economical counters}

\aparag All these counters are intensionally in a limited amount. If you don't
have enough, you can't do what you intended.
\bparag \textbf{Exception:} Gold mines counters are in unlimited amount.
\bparag A country may freely destroy any of its \COL, \TP, \MNU or \TradeFLEET
at the beginning of the Administrative phase in order to reuse it elsewhere.


\subsubsection{Colonies and trading posts}
\aparag Colonies (\COL) and trading posts (\TP) represent the European
colonial effort to either populate the New World or trade with the natives.
\bparag Each counter can hold up to 6 levels.


\subsubsection{Manufactures}
\aparag Manufactures (\MNU) are pre-industrial centres of production of goods.
\bparag Each counter has only two levels: one per side.


\subsubsection{Gold mines}
\aparag These counters represent discovery (or depletion\ldots) of new mines
in Europe.
\bparag Although they are called ``gold'' mines, they can actually be silver,
gems, or other precious mineral.


\subsubsection{Trading fleet}
\aparag Trading fleet (\TradeFLEET) represent the relative commercial power of
each country in each of the sea trade zones.
\bparag Each \TradeFLEET can hold up to 6 levels.


\subsubsection{Trade centres}
\aparag Sea trade zones are grouped into commercial areas. The country with
the most levels of \TradeFLEET within a given area get the corresponding Trade
centre.
\bparag Trade centres provide a large amount of money to their owners.



\subsection{Military markers}

\aparag All these counters are in an unlimited amount. If you don't have
enough, print more or use whatever you think convenient to represent the
missing counter.

\aparag[Ownership] markers are used when the owner of a province changes:
simply put the marker of the new owner (with its shield) on top of the shield
printed on the map.
% \bparag The large ``area'' ownership markers are used in the \ROTW.

\aparag[Control] markers are used during wars as reminder of which fortresses
have fallen into the hands of another country.

\aparag[Siege-work, Revolt, Pillage, Flood] Use these markers whenever
required by the game.



\subsection{Game markers}

\aparag All these counters are intensionally in a limited amount. If you don't
have enough, you can't do what you intended.

\aparag[Technology] There is one technology marker for each of the
technological goals (both land and naval) that can be reach during the game.
\bparag Conversely, there are two such markers (land and naval) for each major
country as well as for some more or less culturally consistent groups of
minors.
\bparag Whenever the marker of a country is beyond the marker of a goal, that
means that the country has reached this technological goal. This usually
provides huge advantages in battles.

\aparag[Exotic resources] There are two markers for each kind of exotic
resource: one to keep track of the total amount exploited and the other to
keep track of the current price of the resource.
\bparag There is also one marker for the current percentage of inflation.

\aparag[European diplomacy] Each minor country in Europe has a diplomacy
marker with its diplomacy values written on it.
\bparag When a major country gains influence over a minor country, put the
corresponding diplomatic marker on the major track.
\bparag Thus, at most one major country can have influence over each European
minor country at a given time.

\aparag[\ROTW diplomacy] Most major country have Relation/Treaty
(\dipFR/\dipAT) markers. These are in limited amount.
\bparag When a major gains influence over a \ROTW minor, put one of his
\dipFR/\dipAT marker in the corresponding box.
\bparag Thus, several majors can have influence over the same \ROTW minor
country at the same time.

\aparag[Various markers] There are some other various markers (for turn and
round, for convoys, or for keeping track of various in-game information such
as variable incomes) to be used when needed.




\section{Tables and dice}

% RaW: [3]



\subsection{Tables}

\aparag Each player has its own \terme{Player's aids} that groups all tables
and information most relevant to play. It is recommended to keep at least one
clear copy of each.
% \aparag Tables are usually accompanied with a list of the modifiers that can
% come into play.

\aparag Most of the table are common for each country (8 pages). Each country
also has one page with its own specific tables and reminder of special rules.



\subsection{Dice}

\aparag The dice used in this game are ten-sided. A zero on the die always
represents ten.
\bparag Sometimes, the player will be asked to roll 1d100. This is done by
rolling two dice, using one as units and the other as tens. 00 always
represents 100.
\bparag When required to roll 2d10, roll two dice and add the results (thus
giving a result between 2 and 20).
\aparag Having between 10 to 20 dice appears to be a good number to play
comfortably without spending to much time looking for them\ldots



\subsection{Players aids and Record sheets}

\aparag[Players aids] There are two kinds of players aids:
\bparag The generic players aids (8 pages) contain all the tables common to
all players. They are organised roughly in turn order.
\bparag The specific players aids (1 page per country) contain the tables
specific to each country, as well as a quick reminder of the specific rules of
that country.

\aparag[Players record sheet] Each player has a set of record sheet to record
his actions, his military forces and strengths, his treasury, his income
calculations\ldots These are:
\bparag Two \terme{Economic Record Sheet} (\EcoRS), one for computation of
incomes and expenses and the other for keeping track of the Treasury and
Loans.
\bparag One \terme{Monarch sheet} on which characteristics of the country and
its ruler can be written. This is also used to write all diplomatic and
administrative actions before performing them.
\bparag One \terme{Colonial sheet} to keep track both of the colonial,
commercial and naval estates of the country.

\aparag[Global record sheets] are provided to keep track of global information
or as summary of some look-up rules, namely:
\bparag The \terme{Exotic resources sheet} keeps track of which country
exploit which exotic resource.
\bparag The \terme{Trade fleet sheet} keeps track of the levels of the various
\TradeFLEET in the various \STZ/\CTZ.
\bparag The summaries of minors countries and of objectives, as well as the
revolt tables are used as quick look-up.
\bparag The events tables are used to note which historical events already
happened.

\aparag See~\ref{chapter:Playing} for a detailed discussion on how these sheet
are meant to be used.




\section{Lexicon}

% RaW: [4]



\subsection{Major countries}

\label{chapter:Basics:Majors}
\def\xlistingabbrev#1{\item[\csname #1\endcsname]}
\aparag Full-time major countries
\begin{deflist}
  \xlistingabbrev{HIS}{Spain, named \paysmajeur{Espagne}}
  \xlistingabbrev{FRA}{France, named \paysmajeur{France}}
  \xlistingabbrev{ANG}{England, named \paysmajeur{Angleterre}}
  \xlistingabbrev{TUR}{Turkey, named \paysmajeur{Turquie}}
  \xlistingabbrev{RUS}{\paysmajeur{Russie}}
\end{deflist}
\aparag Part-time major countries
\begin{deflist}
  \xlistingabbrev{VEN}{Venice, named \paysmajeur{Venise}}
  \xlistingabbrev{POR}{Portugal, named \paysmajeur{Portugal}}
  \xlistingabbrev{POL}{Poland, named \paysmajeur{Pologne}}
  \xlistingabbrev{HOL}{Holland, named \paysmajeur{Hollande}}
  \xlistingabbrev{SUE}{Sweden, named \paysmajeur{Suede}}
  \xlistingabbrev{PRU}{\paysmajeur{Prusse} (see also \pays{brandebourg}). This
    name is only used for the major power}
  \xlistingabbrev{AUS}{\paysmajeur{Autriche} (see also \pays{habsbourg}). This
    name is used for the major power}
\end{deflist}
\aparag Part-time major countries: some notations
\begin{deflist}
  \xlistingabbrev{PORpor}{Either \paysmajeur{Portugal} or \PORmin}
  \xlistingabbrev{PRUpru}{Either \paysmajeur{Prusse} or \PRUmin}
  \xlistingabbrev{HOLhol}{Either \paysmajeur{Hollande} or \HOLmin or
    \pays{provincesne} or \pays{Vhollande}} \xlistingabbrev{POLpol}{Either
    \paysmajeur{Pologne} or \POLmin (and \paysmajeur{Lithuanie} before
    \xnameref{pII:Union Lublin})} \xlistingabbrev{AUSaus}{Either
    \paysmajeur{Autriche} or \AUSmin} \xlistingabbrev{MAJHAB}{When pointing to
    a player, either \paysmajeur{Autriche} if it is a major country; if not,
    \paysmajeur{Espagne} after \xnameref{pI:Habsburg Alliance}}
  \xlistingabbrev{GE}{German Empire (never existed, also
    \pays{german-empire})} \xlistingabbrev{HRE}{Holy Roman Empire, a political
    entity of central Germany (also \pays{saint-empire})}
\end{deflist}



\subsection{Various terms used throughout these rules}

\begin{deflist}
  \listingabbrev{ARMY}{Army, a large-size land force}
  \listingabbrev{ADM}{Administrative value of a Monarch}
  \listingabbrev{CB}{Casus Belli, a reason that makes declaring the war
    towards another country easier} \listingabbrev{CC}{Commercial centre (a
    regional platform of trade)} \listingabbrev{COL}{Colonies (overseas
    European settlement)} \listingabbrev{CTZ}{Country Trade Zone, something
    that represents the foreign trade of a country} \listingabbrev{GD}{A
    military detachment, either a \LD or \ND} \listingabbrev{Ducats}{Ducat,
    the monetary unit of the game} \listingabbrev{DC}{Dynastic Crisis, that
    may occur when some monarch dies (see~\ruleref{chEvents:Dynastic Crisis})}
  \listingabbrev{DIP}{Diplomatic value of a Monarch}
  \listingabbrev{DTI}{Domestic Trade Index, a value that measures the domestic
    commercial power of a major power} \listingabbrev{EcoRS}{Economic Record
    Sheet, displayed on page~\pageref{economic-record-sheet}}
  \listingabbrev{FLEET}{Fleet, a large-size naval force}
  \listingabbrev{FTI}{Foreign Trade Index, a value that measures the
    international commercial power of a major power}
  \listingabbrev{fortress}{One level of fortress, obtained through
    reinforcements}
  % \listingabbrev{HUG}{The major power controlling the \hug in the French
  % Wars of Religion}
  \listingabbrev{LD}{Land Detachment, a small land force}
  \listingabbrev{LDE}{Land Detachment of Exploration, a smaller land force
    meant for the \ROTW} \listingabbrev{LDND}{Land Detachment or Naval
    Detachment} \listingabbrev{LDENDE}{Detachment of Exploration (any kind)}
  % \listingabbrev{LIG}{The major power controlling the \lig in the French
  % Wars of Religion}
  \listingabbrev{LoS}{Line of Supply, a path along which supplies can be
    brought without crossing enemy territory.}  \listingabbrev{MAJ}{A major
    power, the main country of a player} \listingabbrev{MIL}{Military value of
    a Monarch} \listingabbrev{MIN}{A minor power}%, not played}
  \listingabbrev{MNU}{Manufacture, a centre of goods production in a country
    that is of special importance} \listingabbrev{MP}{Movement Points, to
    define distance on the maps} \listingabbrev{ND}{Naval Detachment (any
    kind)} \listingabbrev{NDE}{Naval Detachment of Exploration, 1 warship}
  \listingabbrev{NGD}{Naval Galley Detachment, about 10 galleys}
  \listingabbrev{NTD}{Naval Transport Detachment, about 10 transport ships}
  \listingabbrev{NWD}{Naval Warships Detachment, a small naval force (about 3
    warships)} \listingabbrev{PA}{Products of America, the goods that came
    from the New World: tobacco, dye and other various goods}
  \listingabbrev{PO}{Products of Orient: all kinds of goods coming from the
    Far East: tea, precious wood, porcelain, jade, etc.}
  \listingabbrev{corsaire}{Privateer or Pirate unit, a small naval force of
    privateers that aim for trade ships and pillage}
  \listingabbrev{RD}{Revolt/Disorder, a state of general disarray in Europe
    that makes certain alliances and internal conflicts go wrong}
  \listingabbrev{REB}{The major power controlling rebels in various events
    descriptions (not the rebel side itself)}
  \listingabbrev{ROTW}{Rest-of-the-World, everything on Earth outside Europe}
  \listingabbrev{RT}{Royal Treasury} \listingabbrev{STZ}{Sea Trade Zone,
    something that represents the sea trade throughout some area}
  \listingabbrev{TP}{Trading Post, a small commercial establishment used as a
    European foothold overseas} \listingabbrev{TradeFLEET}{Trade Fleet, a
    fleet of merchants represented by a level (from 1 to 6) on the Trade
    Fleets Sheet and a counter in their \STZ/\CTZ of activity}
  \listingabbrev{TFI}{Trade Fleet Implantation, an administrative operation
    that may increase the level of a \TradeFLEET}
  \listingabbrev{VGD}{Galeasses Detachment, large galleys used by Venice}
  \listingabbrev{VP}{Victory Points (accumulated through the game by each
    player). Also \anchorabbrev{VPs} (plural)}
\end{deflist}



\subsection{Continents in the \ROTW}
\begin{designnote}
  For game purposes, continents are composed of a given set of \Areas and
  provinces. In several cases, this significantly differs from the actual
  geographical continent baring the same name (\emph{e.g.} \continent{South
    America} does not include \continentBrazil, or \continentAsia does not
  include \continentSiberia). In game, continents are usually areas of
  influences of some power and thus include only the geographical zone where
  that power actually tried to impose an exclusive power (\emph{e.g.}
  \paysmajeurEspagne tried to impose a Spanish exclusive in \continent{South
    America} without bothering about the Portuguese in \continentBrazil (as a
  result of the Treaty of Tordesillas), thus in game we have set
  \continent{South America} to correspond to the Spanish World only and
  \continentBrazil is excluded from it).

  In short: beware that in game ``continents'' are not always exactly the same
  as geographical continents.
\end{designnote}

\begin{deflist}
\item[\anchorcontinent{Africa}] is the whole continent of Africa, inland from
  \granderegionMauritanie to \granderegionSoudan and including the islands of
  \granderegion{Sainte-Helene}, \granderegion{Cabo Verde},
  \granderegion{Madagascar}, \granderegion{Mascareignes} and
  \granderegion{Seychelles}.
\item[\anchorcontinent{America}] is all the New World, including
  \granderegion{Malouines}, \continent{Caraibes} and \continent{Brazil}.
\item[\anchorcontinent{Brazil}] Is the following \Areas: \granderegion{Belem},
  \granderegion{Recife}, \granderegion{Rio}.
\item[\anchorcontinent{South America}] contains all the inland areas of
  \continent{America} that lie to the south of \granderegion{Chichimeca}
  (included), excluding \continent{Brazil}.
\item[\anchorcontinent{North America}] contains all the inland areas of
  \continent{America} that lie to the north of \granderegion{Chichimeca}
  (excluded).
\item[\anchorcontinent{Caraibes}] is composed of \granderegion{Haiti},
  \granderegion{Cuba}, \granderegion{Antilles} and
  \granderegion{Florida}. Note that \granderegionFlorida is both part of
  \continent{North America} and \continentCaraibes.
\item[\anchorcontinent{Extreme Orient}] is the union of the following areas:
  \granderegion{Japan}, \granderegion{Formose}, \granderegionCoree,
  \granderegion{Mandchourie}, \granderegion{Pekin}, \granderegion{Nankin},
  \granderegion{Canton}.
% (jym) +Coree. +Philippines ?
\item[\anchorcontinent{India}] is the part of the Indian sub-continent
  magnified on the map. Namely: \granderegionDelhi, \granderegionAoudh,
  \granderegionBengale, \granderegionGujarat, \granderegionPendjab,
  \granderegionIndus, \granderegionOrissa, \granderegionGondwana,
  \granderegionMumbai, \granderegionHyderabad, \granderegionMalabar,
  \granderegionKarnatika and \granderegion{Ceylan}.
\item[\anchorcontinent{Indonesia}] is the archipelago South-East of
  \continent{Asia}, namely \granderegion{Sumatra}, \granderegion{Java},
  \granderegion{Borneo}, \granderegion{Celebes}, \granderegion{Sonde} and
  \granderegion{Epices}.
\item[\anchorcontinent{Indochina}] contains \granderegionBirmanie,
  \granderegionMalacca, \granderegionAyutthaya and \granderegion{Dai Viet}.
\item[\anchorcontinent{Middle East}] is the arabic peninsula
  (\granderegionNedj, \granderegionOman and \granderegionAden, including the
  island of \provinceSocotra), plus \provinceOrmus,
  \granderegion{Afghanistan}, \granderegion{Balouchistan} and
  \granderegion{Aral}.
% (jym) +Ormus
\item[\anchorcontinent{Siberia}] is the union of all northern territories from
  \granderegion{Siberie} to \granderegion{Amour} and \granderegion{Kamchatka}.
\item[\anchorcontinent{Asia}] is composed of \continent{Middle East},
  \continentIndia, \continentIndochina, \granderegionOceania, and
  \continent{Extreme Orient} (thus excluding \continentSiberia).  (jym)
  -Oceania? +Philippines? +Indonesia?
\end{deflist}



\subsection{Land and Sea regions in Europe}

\begin{deflist}
\item[\anchorregion{Autriche}] It contains all the national provinces of
  \paysmajeur{Autriche}, namely \province{Tirol}, \province{Salzburg},
  \province{Osterreich}, \province{Steiermark}, \province{Karnten},
  \province{Slovenija}.
\item[\anchorregion{Balkans}] It contains the following provinces:
  \province{Alabania}, \province{Hellas}, \province{Moreas},
  \province{Dalmacija}, \province{Montenegro}, \province{Corfu},
  \province{Bosna} and \province{Serbia}. These provinces are in a special
  state in 1492 and are subject to special rules (see
  \ruleref{chSpecific:Balkans}).
\item[\anchorregion{Baltique}] It contains the sea zones \seazone{Baltique},
  \seazone{Botnie}. Galleys may navigate it.
\item[\anchorregion{Noire}] It contains the two sea zones of the Black sea
  (\seazone{Noire W} and \seazone{Noire E}).
\item[\anchorregion{Belgique}] It contains the provinces of the Burgundian
  legacy that are not national provinces of \paysmajeur{Hollande}, namely
  \province{Vlaandern}, \province{Flandre}, \province{Hainaut},
  \province{Brabant}, \province{Limburg}, \province{Luxemburg},
  \province{Artois}. %
  % , \province{Zeeland}.  => National province of HOL.
  Note that \province{Franche-Comte} is part of the legacy but not part of
  \regionBelgique.
\item[\anchorregion{Denmark}] It contains all the initial provinces of
  \pays{Danemark} that are neither in \region{Norvege} nor in \region{Suede},
  namely \province{Slesvig}, \province{Sjaelland}, \province{Jylland}.
\item[\anchorregion{Duche de Kurland}] It is initially empty and may contain
  \province{Kurland} and \province{Livonija}.
\item[\anchorregion{Duche de Prusse}] It contains \province{Memel},
  \province{Preussen}, \province{Ost Pommern}.
\item[\anchorregion{Finlande}] It contains all the provinces of the would-be
  \pays{Vfinlande}, namely \province{Finland}, \province{Tavastland},
  \province{Nyland}, \province{Karelen} and \province{Savo}.
\item[\anchorregion{Irlande}] It contains all the provinces of the would-be
  \pays{VIrlande}, namely \province{Mumhan}, \province{Laighean},
  \province{Connacht}, \province{Brega}, \province{Uladh}.
\item[\anchorregion{Italie}] It contains all the provinces of the italian
  peninsula within the blue thick line, plus the Italian islands, namely
  \province{Savoia}, \province{Nice}, \province{Monferrato},
  \province{Liguria}, \province{Lombardia}, \province{Trentino},
  \province{Mantova}, \province{Veneto}, \province{Friuli}, \province{Parma},
  \province{Lucca}, \province{Modena}, \province{Romagna}, \province{Toscana},
  \province{Siena}, \province{Lazio}, \province{Umbria}, \province{Marche},
  \province{Abruzzo}, \province{Campania}, \province{Puglia},
  \province{Basilicata}, \province{Calabria}, \province{Sicilia},
  \province{Palermo}, \province{Saldigna}.
  % \region{Khanats}
\item[\anchorregion{Mediterranee}] It contains all the sea zones of the
  Mediterranean and Black seas (\seazone{Noire W}, \seazone{Noire E},
  \seazone{Marmara}, \seazone{Egee}, \seazone{Mediterranee E},
  \seazone{Adriatique}, \seazone{Ionienne}, \seazone{Thyrrenienne},
  \seazone{Syrte}, \seazone{Lion}, \seazone{Mediterranee W}). Galleys may
  navigate it.
\item[\anchorregion{Norvege}] It contains all the provinces of the would-be
  \pays{Vnorvege}, namely \province{Trondelag}, \province{Vestfold},
  \province{Ostlandet}.
\item[\anchorregion{Perse}] It contains the four easternmost European
  provinces of \pays{perse}, outlined in black. Namely, \province{Pars},
  \province{Isfahan}, \province{Bam}, \province{Meshhed}.
\item[\anchorregion{Suede}] It contains all the national provinces of
  \paysmajeur{Suede} that are not part of \region{Finlande}, namely
  \province{Smaland}, \province{Jamtland}, \province{Gastrikland},
  \province{Bergslagen}, \province{Svealand}, \province{Vastergotland},
  \province{Gotland}, \province{Skane}.
\item[\anchorregion{Ukraine}] It contains all the provinces that can possibly
  belongs to \pays{ukraine}, namely \province{Podolie}, \province{Ukraine},
  \province{Poltava}, \province{Zaporozhye}, \province{Donets} and
  \province{Don}.
\end{deflist}

% Local Variables:
% fill-column: 78
% coding: utf-8-unix
% mode-require-final-newline: t
% mode: flyspell
% ispell-local-dictionary: "british"
% End:

% LocalWords: pVI Tordesillas pI saltern salterns Khanates Gastrikland
