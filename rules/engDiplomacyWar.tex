% -*- mode: LaTeX; -*-

\section{On wars}

% RaW: [34]



\subsection{How Wars Begin}

Wars take place due to independent decisions of any player or players
(announced during the Diplomatic phase) or may be started by events.


\subsubsection{Wars caused by events}
\aparag Some wars may be caused by events, offering a \CB to some \MAJ, or
telling that some \MIN declares a war.
\bparag The description of political events may offer a \CB to some
countries. The \CB that are described under the "Event Phase" part are used
during the first step of the Diplomacy Phase, before formal Agreements are
made and before private discussions are allowed. By order of Initiatve, all
players announce which declaration(s) of war allowed by events they use, or
not.
\bparag The reaction on wars breaking down this way are resolved at that
time. Note that no new Formal Agreement could have been signed at this turn,
but Alliances of a past turn are usable (they finish in the next segment
only).
\bparag If an event gives a \CB under the ``Diplomacy phase'' part of the
description, then the \CB is used normally after discussion and other
agreement, including new alliances.
\bparag If an event gives several \CB, all countries using these \CB against
common enemies are automatically allied for this war (only), unless the event
specifically speaks of distinct wars being possible.

\aparag[Wars continuing other wars] If a war should begin between two
countries already at war against each other, the exact meaning of this depends
on the nature of the war about to begin for the country declaring the war:
mandatory, incompatible with other wars, or provoked by the country. Most
events are mandatory; the other ones are explicitly mentioned in the event.
\bparag[Mandatory war] The new conditions of war described in the event are
added to already existing conditions. A \MAJ can announce at the diplomatic
phase that an already running conflict becomes the new war. Calls for allies
are made at this point (according to the conditions of the new war) because
the war's motives change. The only thing that should be ignored is the initial
declaration of war, since the country is already at war (a \CB for this turn
is deemed to have been used).
\bparag[Incompatibility] The new war can be made incompatible with wars
between the two countries about to begin the new one. Usually, the event calls
for a replacement event (the event did not happen at all, and another one is
rolled for instead). However, a war with incompatibilities can be followed by
a mandatory war.
\bparag[Controlled war] The new war is indicated as being controlled by a
country. It may delay the event (which, as above, did not happen at all and is
replaced by another one), or accept the event and apply it as if it were
mandatory.
\bparag[Armistices] An Armistice may not be signed for an ongoing war that is
transformed by either a controlled or a mandatory war.


\subsubsection{Wars by voluntary declarations}
\aparag Wars are also declared during the Diplomatic phase by the attacking
player, in the fourth segment of the phase, after the segment of Announcements
of Formal Agreements. No private negotiation is permitted between the
Announcements and the Declarations of Wars.

\aparag War using \CB described under the section "Diplomacy Phase" of an
event, has to be declared at that time.

\aparag A whole segment of reactions following these declarations of wars is
then made.

\aparag[Restriction on Wars] A War is usually declared against an Alliance
that is either a power currently at peace, or an Alliance already formed in an
ongoing war.
\bparag The only way to declare against only one power of a warring Alliance
(instead of the whole Alliance) is if the attacker has a \CB (either permanent
or given by event, temporary \CB are not enough) and uses it against this
power.



\subsection{Casus Belli}

\aparag A Casus Belli (\CB) allows declaring war by losing only {\bf 1} \STAB
level, without any loss of victory points (\VP). \CB are of two different
types, permanent or temporary, and may be usual or free. Free \CB allows
declaring of wars without loss of \STAB.

\aparag[Temporary Casus Belli] The temporary \CB is provided by events, or by
the rules. Usually it may be used only once and is then cancelled; a temporary
\CB is valid for 6 turns, excepted if specified differently in the description
of the \CB. Some temporary \CB are linked to the existence of a condition: the
\CB is valid as long as the condition is met; if the \CB is used and the war
terminates, the \CB could still be valid if the condition is satisfied.

\aparag[Permanent Casus Belli] Here are the permanent \CB:
\bparag Following the event \eventref{pI:Reformation}, all Catholic countries
have a permanent \CB against all Protestant countries (and vice versa). This
is no longer valid after the end of \terme{Religious Enmities}.
\bparag \SPA has a permanent \CB against all Pagan or Muslim countries.  This
is no longer valid after 1700, included.
\bparag \TUR has a permanent \CB against all Christian countries, against
\pays{perse}, against \paysEgypte and against \paysDamas. This is no longer
valid after 1700, included.
\bparag A player has a permanent \CB against any country (player or minor)
that has has annexed a national province of the player.



\subsection{Cost of a War Declaration}

\aparag A declaration of war costs VP, as well as a loss of Stability,
according to whether the player has a \CB or not.

\aparag Cost in Victory Points
\bparag No \VP: with \CB
\bparag -10 \VP: without \CB, against a player or a minor country vassal of a
player.
\bparag -5 \VP: without \CB, against a minor (except vassal minor country -
see above).

\aparag Cost in Stability
\bparag none: with a Free \CB.
\bparag -1 level: with \CB.
\bparag -2 levels: without \CB.
\bparag[Note]
Cost in lost \STAB may be altered by existing treaties and alliances between
players, or also by event description. Especially, breaking and alliance
(either defensive or offensive) costs 2 extra levels of \STAB.

\aparag[Wars and reduction of Trade]
The war forces all belligerent players to refuse mutually the trade access to
their market. This influences the calculation of their foreign trade income as
follows:
\bparag The European market value of each power is decreased by the amount of
Income of the enemy player's provinces (including vassals).
\bparag Other commercial income sources (commercial fleets, exotic resources,
etc...) are not affected directly by the state of war.
\bparag Note that this reduction of Trade does not affect the commercial
fleet, as would do a Trade Refusal declaration (but a declaration for this
effect can be added to the war).



\subsection{Overseas Wars}\label{chDiplo:Overseas Wars}


\subsubsection{Commercial and Overseas \CB}
\aparag Some \CB are obtained to wage a restricted kind of war that is called
an \terme{Overseas War}. They are called Commercial \CB or Overseas \CB and
may be free, permanent or temporary as usual. Some events, or conditions in
the rules, give other Commercial or Overseas \CB, as indicated in their
description.  \overseascb

\aparag A Commercial/Overseas \CB may be used to initiate an Overseas
War. Declaring an Overseas War without a Commercial/Overseas \CB is not
allowed.

\aparag When an Overseas War is declared, reactions caused by the war may be
made as usual.


\subsubsection{Permanent State of Overseas War}
\aparag[Barbaresque countries.] \Barbaresques (countries of the Barbary coast)
are \paysCyrenaique, \paysTripoli, \paysTunisie, \paysAlgerie and
\paysMaroc. They are always in a state of restricted Overseas War against
every Christian countries.
\bparag It allow them to use Privateers and naval forces (no land forces) to
attack Christian countries. Christian countries can use their own naval forces
or \Presidios to fight against the Barbaresques.
\bparag As an exception, Privateers of the \Barbaresques may loot European
provinces adjacent to the \STZ they attack, even if they are European
provinces usually outside the scope of Overseas Wars.
\bparag \TUR plays the \terme{Barbaresques} that are neutral, and the
diplomatic patrons play those that are not. The specific rules tell the \STZ
that are attacked by the Privateer.
\bparag This state of war causes no loss of \STAB.
\bparag[Reinforcements] They receive some reinforcements each turn:
\pays{Algerie} gains a \corsaire\facemoins each turn; in periods I to III it
receives also a \ND or 2 \NGD (player's choice) and in periods IV and after,
only one \NGD or a \NDE. Other countries gain only a \corsaire\facemoins 2
turns after their Privateer has been destroyed.
\bparag \textit{Exception.} Whenever \leader{Dragut} is in play and if it used
in its Privateer leader role, a \corsaire\facemoins of \pays{Tunisie} is
raised (even if eliminated at previous turn).
\bparag[Mandatory Sea Sortie] The Privateers usually have to go out at sea
each turn, except if their Patron decides against it: a test is made at the
beginning of the 2nd round if the Privateer is not at sea, by rolling 1d10 for
each country the Patron wants to keep the Privateer at port.  This is
permitted if the result is lower or equal to the number of the current period
plus the Diplomatic status bonus and the geopolitical bonus.

\aparag[The Knights.] The \pays{chevaliers} is always in a state of restricted
Overseas War against \TUR.
\bparag It allow them to use Privateer and naval forces (no land forces) to
attack \TUR. \TUR can use their own naval forces to fight against them.
\bparag The diplomatic patron of the \pays{chevaliers} play this forces, or
\SPA if it is neutral.
\bparag The annexes specify the reinforcements gained by the \pays{chevaliers}
each turn: a \corsaire\facemoins (or \Faceplus if in \province{Rhodos}), and a
\NGD or a \NDE.
\bparag This state of war does not cause automatic \STAB loss at the end of
turn.  But, at each turn that the pirate of The Knights inflicts losses on
Turkish commercial fleets, \TUR loses at least 1 \STAB level (that is, the
Knights' privateer causes a loss of \STAB if and only if \TUR does not already
loose \STAB for another reason at the end of turn (war, revolts, \ldots))


\subsubsection{Restriction in Overseas Wars}
\aparag[Reaction of the victim.] A country that has an Overseas war declared
upon gains a temporary \CB against the attacker to declare a regular war.
\bparag If/When this \CB is used, the war changes and causes a whole new set
of reactions allowed by this new full-blown war. The state of Overseas war is
no more.
\bparag This \CB can be used in reaction as a free \CB on the first turn of
the war, or as a normal \CB to declare a full war on following turns (as long
as the Overseas war continues).
\aparag Reactions other than this case are restricted:
\bparag Calls of allies (Formal Alliance or Limited Alliance) are made as
usual excepted that they give only Overseas \CB;
\bparag No minor country may be involved completely in an Overseas war if it
was not the victim of the war, or if it is not a \VASSAL of an involved \MAJ;

\aparag[The course of the war.]
\bparag Overseas wars can cause no military action on the European mainland
(that is all land provinces on the European map), except provinces in
\Barbaresques, \pays{Egypte} and \pays{Irak}.
\bparag No trade refusal or reduction is applied (except if an added
declaration of Trade Refusal is made by one country).
\bparag An Overseas War is not exactly a state of War for the power.  If it is
its only war, a \MAJ would have to use the costs of Maintenance as if at
peace.
\bparag Minor countries in \EG cannot be called for a full intervention in the
war.
\bparag In any other aspect, except when specified, an Overseas War is
conducted as a regular war. For instance, any naval operation, attacks by
Privateers, fights in the \ROTW (\COL, \TP, in any provinces on the \ROTW map)
are allowed, as well as limited intervention of \MIN.

\aparag[Peace and Overseas wars.]
\bparag A minor country always accepts a proposed white peace to end an
Overseas War at the end of a turn.
\bparag A peace treaty ending an Overseas War may not involved change of
ownership of any province on the European map.
% PB - TBD: allow annexion on African Coast or not ? I favor not. Then
% Overseas wars against Barabresques may target Praesidios or Diplomatical
% Alignment.
\bparag Transfer of \TP (even \Facemoins) counts as a full province.
\bparag If an Overseas War is not finished at the end of a turn, the loss of
\STAB (due to this war) by involved countries is limited to 2 levels per turn
(instead of 4).



\subsection{Reactions to a Declaration of War} \label{chDiplo:diplo:Reactions}


\subsubsection{Generalities about Reactions}
\aparag On both segments allowing Declarations of wars, Reactions can be made
by any power, after all initial Declarations of War.  Going through in the
order of initiative, and then circling again until no-one has anything left to
declare, each power can make none, one, or several declarations in reaction.
\bparag Note that some reactions can only be made just following some initial
declaration (usually a new war, or mere new conditions due to events) -- at
the same turn and segment; whereas others can be made spontaneously at any
turn.  \diploreac


\subsubsection{Guidelines about successive declarations of wars.}
\aparag No new war can begin by reactions (excepted by reacting to a Trade
Refusal). Reactions are mere extensions of an existing war.  One can react
after a reaction, broadening further the scope of the war.
\aparag When a reaction puts a country in a war, this country has to join a
whole alliance and its thus at war against every enemies of this alliance. If
it is allied to countries in both sides of the war, it has to break one of the
alliances.
\aparag The sole possibilities to have multi-sided wars is then to have
different wars involving the same country(ies). All country that join the
alliance at war against several alliances at the same time will have to
declare war against all those alliances.
\bparag Conversely, entering the war at the side of an alliance B, when
alliance A is at war against B and C, is a war only against A and the
Neutrality is conserved regarding C, i.e. no co-operation, no supply, no
passing through provinces controlled or occupied by the other alliance.  Note
that this situation gives a \CB to alliances B and C against the other one, or
on the contrary, they could declare that they ally together in this war.
\bparag Three-sided wars (or more) where more than two alliances are at war
against each other are allowed.


\subsubsection{Signing an Alliance for Intervention}
\label{chDiplo:InterventionLimitee}
\aparag Alliances for Intervention are signed in reaction to a declaration of
war. Such an Alliance involves two Major powers, one at war and another
one. The second country enters then the war in a limited intervention at the
side of the alliance of the first power.
\bparag This is a kind of alliance and the intervening power uses a \CB given
by the alliance to enter the war in this limited way: it loses {\bf 1} \STAB.
\bparag Usually, only a country that is victim of a declaration of war (even
in reaction due to alliance, or by a minor country) can sign an Alliance for
Intervention.
\bparag Exception: \ENG and \PRU may always sign Alliances for Intervention
with attacking countries.
\bparag Signing an Alliance for Intervention is only possible on the first
turn of a war (or new developments), except if written otherwise in some
events.
\bparag Limited intervention is forbidden in Religious or Civil Wars, excepted
if the event explicitly says otherwise.

\aparag[Conditions of a limited intervention of a \MAJ.]
\bparag The power is not at war because of the intervention. It uses the costs
of Maintenance at peace (if not involved in another war).
\bparag The power can use up to one land stack and one naval stack to do
anything as part of the war. Once a land or naval stack has been committed, no
other land or naval (respectively) force of the power can be involved in this
war. These forces are the only one that can move in provinces at war, attack,
besiege, assault, do naval transport of forces at war, make a blocus, fight
against Privateers, and so on\ldots All conquests (including captured
monarchs) are made for the sake of the alliance at war (he chooses one
country, a \MAJ is possible). All pillages made by his stack go in his TR.
\bparag All other forces of the power doing a limited intervention are as if
at peace.  All provinces of the power are also not part of this war and only
its forces can enter them.
\bparag Minor countries controlled by the power are not part of the
intervention (this includes \VASSAL). Exception:
see~\ref{chSpecific:England:Minors at war}.
\bparag A power can do limited interventions at the same time in more than one
war. It cannot intervene on the side of enemy alliances.

\aparag[Continuation of a limited intervention.]
\bparag After the Truces, if the war is still going on, any power of the enemy
Alliance has first the possibility to declare Full war against the intervening
Power, having a \CB and paying 1 \STAB to do so.
\bparag Else, a limited intervention ends at the end of the turn, excepted if
the power doing the intervention spends 1 \STAB at the end of turn (after
\STAB improvement action), in addition to any other loss of \STAB.
\bparag If the intervention ends, the forces are redeployed as when signing a
white peace. There is no gain of \STAB.
\bparag If the intervention continues, the power will be able to send
reinforcements as long as those are stacked at the end of the first round with
the intervening stacks.
\bparag If the intervention continues, the enemy alliance has a free \CB at
the following Event Phase to declare a full war against the intervening power.

\begin{exemple}[Alliance going into flames]
  It is turn 10. \HIS, \VEN and \POL are allied in a holy Catholic league
  (defensive alliance) while \TUR and \FRA also have a defensive
  alliance. \TUR decides to send the \terme{Levant} convoy
  (see~\ref{chIncomes:Levant Convoy}) to \FRA, thus providing a commercial \CB
  to \VEN (who owns the Mediterranean centre of trade and thus believes he
  should get the convoy).

  \VEN decides to use this \CB (thus loosing 1 \STAB). \TUR reacts by turning
  the war into a full blown war, hoping to advance in the Balkans (no \STAB
  lost as this is a free \CB). Since \VEN has now been victim of a declaration
  of war, the Doge calls his Polish allies (to protect the Balkans) and \POL
  accepts and declares war on \TUR (cost 1 \STAB for \POL). \TUR then decides
  to call its minor \VASSAL, \paysCrimee, fully into the war to chop on the
  Polish flank.

  In the West, \HIS was not called into the war, however, \monarque{Charles V}
  decides that this is a good opportunity to try and seize
  \provinceTunis. Thus, \HIS uses the \CB provided by his alliance and declare
  war to \TUR and then to its \VASSAL, \paysTunisie (1 \STAB
  lost). \monarque{Francois I}, always eager to harm the Hapsburg, then uses
  its alliance to react to the Spanish aggression by also declaring war. He'd
  like to declare war only on \HIS but cannot as war must be declared against
  the full alliance, in this case \VEN, \HIS, \POL (and maybe some minors
  allies). This cost him 1 \STAB.

  \HIS would then like to call for a full war his ally, \paysPalatinat, in
  order to open a second front against \FRA. However, \paysPalatinat is only
  in \EW. Since \paysPalatinat is not adjacent to \FRA but nonetheless less
  than 6 MP away, and \HIS has no specific bonus on it, he must roll 6 or more
  on a die to successfully call it. \HIS rolls 7 and \paysPalatinat declares
  war on \FRA.

  Back in the East, \RUS believe that this could be an opportunity to weakens
  the Crimean. So, he react to the Turkish attack by signing an alliance for
  limited intervention with \HIS, \VEN and \POL (cost 1 \STAB).

  After Diplomatic actions on minors are made, both \paysKazan and
  \paysAstrakhan are on the Turkish diplomatic track, thus \TUR decides to
  call them for limited intervention in this full blown war (to defend
  \paysCrimee).

  Both \paysCrimee and \paysPalatinat are fully at war. They will thus receive
  reinforcements in the upcoming administrative phase. On the other hand,
  \paysKazan and \paysAstrakhan are only in limited intervention. They will
  only have their basic forces but are not part of the war (and thus cannot be
  entered by enemy troops). \RUS is also not fully at war. He will use the
  (more expensive) peace maintenance cost and cannot send more than one stack
  in the war ; moreover all his conquests will be made for the behalf of
  another major (for example \HIS), and count as his for peace purpose. But no
  enemy troops can enter Russia and besiege his fortresses.

  At the end of turn, \RUS can choose to stop its intervention. In this case,
  Russian troops go back in Russia but the fortresses he has conquered are not
  given back to \TUR (they are still controlled by \HIS). Alternatively, \RUS
  can choose to stay in intervention (loosing 1 \STAB). In this case, at turn
  11, \TUR can choose to generalise the war and fully imply \RUS in the war
  (with no \STAB lost, this is a free \CB to be used at the same time as \CB
  provided by events). If this is done, this new declaration of war can causes
  a full new set of reactions\ldots
\end{exemple}

\begin{exemple}[Three-sided wars]
  In 1700 (turn 42), \ref{pVI:Great Northern War} is rolled. As per event
  description, it provides both \RUS and \POL \CB against \SUE (plus some
  other conditions). Both \RUS and \POL separately decide to use them. So,
  there are two wars going on: \RUS (and eventual allies) against \SUE and
  \POL (and allies) against \SUE. However, Russian may not enter Poland or
  attack Polish troops and conversely as these countries are not in the same
  war. Swedish troops (and allies) can go both in Poland and Russia as \SUE is
  at war against both. Note that if a Swedish fortress is besieged and taken
  by \RUS, \POL cannot later go and besiege it as this would be an attack
  against a Russian fortress\ldots

  In turn 43, the war is going on. Since there are two alliances (namely \RUS
  and \POL) at war against the same third alliance (\SUE), they can do one of
  the following:
  \begin{itemize}
  \item Keep the wars separate and continue as the previous turn.
  \item Decide to join the wars. \RUS and \POL will then be allied for the
    duration of the war (only). They can now go in each other territory, stack
    troops together, \ldots but must sign a peace together.
  \item Declare war one to another. The alliance (\RUS or \POL) declaring the
    war loses 1 \STAB for this (normal \CB). Then, there will be a three-sided
    war between \SUE, \RUS and \POL. Each of them can go in each other
    territory, or attack each other troops. Polish troops can now besiege a
    Swedish fortress that was previously taken by \RUS and, in case of
    success, the fortress will be controlled by \POL (and count as such for
    peace). Three different peaces will need to be signed as there are 3 wars,
    each peace using specific differential for its own war\ldots
  \end{itemize}
\end{exemple}


\subsubsection{Armistice}
\aparag An armistice can be signed in any war that began in a previous turn
(but not if it begins this turn, or has new conditions due to an event or a
transformation from Overseas to full war).  All powers in both enemy alliances
has to agree the Armistice; if not, none is signed.
\bparag Usually, no Armistice is allowed in Religious or Civil Wars, excepted
if the event says otherwise.
\bparag Some events call for mandatory Armistices: no one has to agree\ldots
\aparag The countries stay at war for the turn but can make no offensive
action against the enemy alliance. All besieged provinces at the time of the
Armistice has to be freed on the first round. Provinces that are controlled by
the enemy stay so.
\bparag During the turn, it is forbidden to enter a province, \COL or \TP of
the enemy that was not controlled at the beginning of the turn.  Interception,
siege, attack by naval units or privateers are also forbidden.
\bparag Use of \Presidios or \StraitFort, however, is still allowed (as when
the countries are at peace).
\aparag At the end of the turn of the Armistice, if no peace is signed, the
enemy alliances lose 1 \STAB in addition to normal losses (after \STAB
improvement action), in remplacement of the \STAB losses normally caused by
this war. Moreover, this turn will not be counted as a turn of war to compute
the length of the war (and the \STAB loss associated).
\bparag The countries are still considered at war for attempts of \STAB
improvement
% Jym
and maintenance.


\subsubsection{Religious Wars, Civil Wars}\label{chDiplo:Religious Civil War}
\aparag Some wars caused by events are said Religious Wars, or Civil Wars. In
a Religious War, any Major Power that shares the religion of one of the two
sides may intervene in the war to help the side having the same religion. In a
Civil War, any Major Power can intervene for one side or the other.
\bparag Those interventions are ruled by the \xnameref{chDiplo:Foreign
  Intervention} limits.
\bparag Several kinds of more important interventions (limited war or full
war) may be allowed in the precise description of the event.  Except for those
allowed, interventions, any other kind of war or attempts to be involved in a
Religious or Civil War implies the effects described in "Excessive Foreign
Implication".
\bparag Exception: during \eventref{pIII:Dutch Revolt}, wars against \HIS or
\HOL do not qualify as Excessive Foreign Implication if fought out of Holland
and the Spanish Netherlands.
% Jym: shouldn't Artois/Flandre/Hainaut be also allowed for an hypothetical
% action of FRA in South Belgium ?  excessive intervention should be limited
% to HOL national ter. + Vlaanderen/Brabant/Liege/Limburg.
\bparag[List of Religious Wars.]
% A ADAPTER :
\begin{todo}
  Double- or triple-check the list of religious and civil wars\ldots
\end{todo}
\eventref{pII:Schmalkaldic League}, \eventref{pIII:FWR Detailed},
\eventref{pIII:Dutch Revolt}, \eventref{pIII:League Nassau},
\eventref{pIV:TYW},
\eventref{pIII:Religious War Sweden}, \eventref{pIII:Religious War Poland},
\eventref{pIII:Times of Troubles}, \eventref{pIV:Bohemian Revolt},
\eventref{pIV:Augsburg Revocation}, \eventref{pIV:English Civil War},
\eventref{pIV:La Rochelle}
\bparag[List of Civil Wars.]
\eventref{pIV:Fronde}, \eventref{pIV:Unity HRE}, \eventref{pIV:Swedish Nobles
  Unrest}, \eventref{pIV:English Restoration},
\eventref{pV:WoSS}, \eventref{pV:Glorious Revolution},
\eventref{pVI:WoPS}, \eventref{pVI:WoAS}, \eventref{pVI:Jacobite
  Rebellion}, \eventref{pVI:Kurland}, \eventref{pVII:Pugatchev Revolt},
\eventref{pVII:Independence War}, \eventref{pVII:French Revolution},
\eventref{pVII:Bavarian Succession}.
\bparag Added to these lists, any War of Succession following a Dynastic
Crisis becomes a Religious Civil War before the end of \terme{Religious
  Enmities}, and a Civil War afterwards.

\aparag[Foreign Intervention]\label{chDiplo:Foreign Intervention}
Other countries may, without declaring a war on the country suffering the
civil war, send units to fight in that country. In Religious Civil Wars, the
intervention is necessarily on the side of a faction that shares same religion
as that of the intervening player.
\bparag This Foreign Intervention is not a war (nor a declaration of war) and
costs {\bf 1 \STAB} for each intervention.
% announced by a given power in a given war.  PB: changed from no \STAB
It is announced as a reaction during the Diplomatic Phase.
\bparag
This intervention is limited to a maximum of one land stack of at most one
\ARMY\faceplus, and/or one \FLEET counter per allied player.  (i.e. per
country, not group of countries).  These forces are the only one that can move
in provinces involved in the Religious/Civil War (including provinces of
powers that are fully involved in the war); movements or campaigns in the
\ROTW is not allowed (excepted if the event says otherwise).  All conquests
are made for the sake of the side supported in the war.  All pillages made by
his stack go in his own TR.
\bparag Minor countries controlled by the power are not part of the
intervention (this includes \VASSAL).
\bparag A power can do Foreign interventions at the same time in more than one
war. It cannot intervene at the same time on the side of enemy alliances.
\aparag[Continuation of a Foreign Intervention.]
\bparag A Foreign intervention ends at the end of the turn if no force of the
Foreign power stays in a province at war.
\bparag If the Foreign Intervention continues, no reinforcement can be send in
the war; no \STAB is lost by the intervening power.  It is possible to end an
intervention and resumes is afterwards (see next point) so that new forces are
sent.
\bparag A Foreign intervention can be resumed at any turn after it has ended
but this costs {\bf 1} \STAB to the Foreign power intervening. In Civil Wars,
the Foreign intervention could resume as an ally of the other side.

\aparag[Excessive Foreign Implication.]
No player can send more than one \ARMY\faceplus on the side of any one faction
in such a war, if a limited or full intervention of his power is not allowed
in the event.
\bparag If ever a power declares war on the country where the civil war rages,
the civil war stops temporarily in a mandatory Armistice. The victim country
may use units of both factions in his civil war to fight against the
invader(s). In addition:
\begin{enumerate}
\item Revolts do not incur any Stability loss during excessive foreign
  interventions.
\item Rebel and loyal units may not collaborate (i.e. transport, stack and/or
  fight together).
\item If an Excessive Implication occurs, events concerning the same Civil War
  are still marked off but their application is suspended. On any following
  turn when the intervention is over, such already marked off events (during
  the above intervention turns) will occur in addition of regular events on a
  even roll of 1d10 (no more than 1 per turn).
\end{enumerate}
\bparag However, the units of both factions are kept under the control of the
victim country until the peace is signed with all foreign invaders.

\bparag Once the Excessive Implication is over, the civil war is resumed and
the rebels receive reinforcements if they have lost 25\% or more of their
initial strength (proceed as per first turn of the civil war).

\begin{designnote}
  Excessive foreign intervention is not really meant to happen. If you start
  to think that it is often a good thing to do to achieve your goals, you're
  probably abusing some loophole in the rules. Typical games should not see
  more than one or two excessive foreign intervention (and most of the time,
  none should occur).

  Typically, trying to use excessive foreign intervention to artificially
  lengthen a civil war, lower the \STAB or your enemy or destroy loyal troops
  while keeping rebels alive to give them the edge are abuses.

  Excessive foreign intervention should only arise when another event is
  rolled and call for a new war with a country already in civil war.
\end{designnote}

\begin{todo}
  Add a (high) VP cost for EFI unless using a \CB provided by event to
  dissuade players from abusing it ??? -30VP should be enough to prevent
  abuses.
\end{todo}



\subsection{Call for ally by Minor countries}


\subsubsection{Generalities}
\aparag A minor country can be involved in various ways in a war:
\bparag Limited intervention, as per the previous rules; this intervention is
possible in a war of its Patron if the diplomatic status is \AM, \CE, \EG or
\VASSAL;
\bparag Full intervention if it was declared war upon, or if it declares
war. When a European minor country is fully involved in a war, no-one is
allowed diplomacy action on it.
\bparag In Overseas wars, the intervention are of the same kinds, but
constrained by the limits of Overseas wars.

\aparag A minor country can declare a war in the following occasions:
\bparag Some events (including \RD);
\bparag A \VASSAL is fully involved by its Patron, as a reaction. This costs
no additional \STAB.
\bparag The country is in \EG and its Patron tests for declaration of war by
the minor country (as explained in \ruleref{chDiplo:EW Effects}) and
successes.
\bparag A country in the \ROTW may declare an Overseas war due to reaction
against European presence.

\aparag A minor country can be declared war upon in the following occasions:
\bparag As a usual declaration of war (with, or without \CB; sometimes caused
by events);
\bparag If it is a \VASSAL, only as part of a declaration of war jointly
against its controlling country ; or as a generalisation of the war against
the patron.
\bparag If it is in limited intervention in a war and the enemy alliance
decides to fully involve the minor country in the war (this is done in
reaction).
\aparag Note that some specific alliances are dealt with different rules. That
is for instance the case of the alliance between \SPA and \hab, or of some
alliances forced by events.


\subsubsection{When a minor country is attacked}
\aparag A minor country that is attacked will call for some help according to
the rules explained here. Those calls are the first reactions resolved, in a
random order, before other kinds of reactions announced by Major powers.

\aparag[If the minor country is Neutral.]
The first power listed in the Appendix in the preference list, and that is not
at war against the \MIN, is called as an ally in the war.
\bparag The \MAJ can refuse any help, in which case it plays the minor country
but is by no means involved in this war and the \MIN stays "Neutral";
\bparag If it accepts, he makes a limited intervention (as if signing a
Alliance for Intervention) in the war, and the minor country is put in \AM of
the intervening power.
\bparag If the limited intervention ends before the war against the minor
country, this is a break of alliance: it costs 2 \STAB to the power breaking
the alliance, and the \MIN is put as "Neutral".

\aparag[If the minor country is in \RM or \SUB.]
\bparag The controlling power can refuse its help, in which case the
diplomatic status is broken, the \MIN is now "Neutral" and the previous
situation is applied, ignoring the \MAJ that just refused to help.
\bparag The controlling \MAJ may accept to do a limited intervention (as if
signing an Alliance for Intervention) in the war, and the minor country is put
in \AM of the intervening power.
\bparag If the limited intervention ends before the war against the minor
country, this is a break of alliance: it costs 2 \STAB to the power breaking
the alliance, and the \MIN is put as "Neutral".
\bparag The controlling \MAJ may accept to do a full intervention in the war,
that is to declare a war with a \CB against the attacking alliance; the minor
country is then put in \EG of the \MAJ.

\aparag[If the minor country is in \AM, \CE, \EG or \dipAT (in \ROTW).]
\bparag The controlling power can refuse its help, in which case the
diplomatic status is broken, the \MIN is now "Neutral" and the previous
situation is applied, ignoring the \MAJ that just refused to help.  If the
status was \EG or \dipAT, the \MAJ loses {\bf 1} \STAB (for the breaking of
this alliance).
\bparag The controlling \MAJ may accept to do a limited intervention (as if
signing an Alliance for Intervention) in the war, and the minor country is put
in \AM of the intervening power (or stays in \dipAT in the \ROTW).
\bparag If the limited intervention ends before the war against the minor
country, this is a break of alliance: it costs {\bf 2} \STAB to the power
breaking the alliance, and the \MIN is put as "Neutral".
\bparag The controlling \MAJ may accept to do a full intervention in the war,
that is to declare a war with a \CB against the attacking alliance; the minor
country is then put in \EG of the \MAJ (or stays in \dipAT in the \ROTW).

\aparag[If the minor country is a \VASSAL or in \ANNEXION.]
The declaration of war is only possible jointly against the controlling power,
or if a war against this power is already active.

\aparag Note that in the frequent case where the \MAJ is already at war when
one minor country it controls is declared war upon, the existence of the
existing war is sufficient to respond the alliance (and the minor is raised in
\EW if it had a lower status).


\subsubsection{When a minor country is declaring war.}
\aparag[If the minor country is Neutral.]
Excepted if an event says otherwise, the first power listed in the Annexe in
the preference list that is not at war against the \MIN, is called as an ally
in the war.
\bparag The \MAJ can refuse any help, in this case he will play the minor
power, but he is by no means involved in this war and the \MIN stays
"Neutral";
\bparag If the \MAJ accepts, he makes a limited intervention (as if signing an
Alliance for Intervention) in the war, and the minor country is put in \AM of
the intervening power.
\bparag If the limited intervention ends before the war against the minor
country, this is a break of alliance: it costs 2 \STAB to the power breaking
the alliance, and the \MIN is put as "Neutral".

\aparag[If the minor country is in \RM or \SUB.]
\bparag The controlling power can refuse its help, in which case the
diplomatic status is broken, the \MIN is now "Neutral" and the previous
situation is applied (ignoring the \MAJ that just declined).
\bparag The controlling \MAJ may accept to do a limited intervention (as if
signing an Alliance for Intervention) in the war, and the minor country is put
in \AM of the intervening power.
\bparag If the limited intervention ends before the war against the minor
country, this is a break of alliance: it costs 2 \STAB to the power breaking
the alliance, and the \MIN is put as "Neutral".

\aparag[If the minor country is in \MA, \EC, \EW or \dipAT (in ROTW).]
\bparag The controlling power can refuse its help, in which case the
diplomatic status is broken, the \MIN is now "Neutral" and the previous
situation is applied (ignoring the \MAJ that just declined intervention). If
the status was \EG or \dipAT, the \MAJ loses {\bf 1} \STAB (for the breaking
of this alliance).
\bparag The controlling \MAJ may accept to do a limited intervention (as if
signing an Alliance for Intervention) in the war, and the minor country is put
in \AM of the intervening power (or stays in \dipAT in the \ROTW).
\bparag If the limited intervention ends before the war against the minor
country, this is a break of alliance: it costs 2 \STAB to the power breaking
the alliance, and the \MIN is put as "Neutral".
\bparag The controlling \MAJ may accept to do a full intervention in the war,
that is to declare a war with a \CB against the attacking alliance; the minor
country is then put in \EG of the \MAJ (or stays in \dipAT in the \ROTW).

\aparag[If the minor country is a \VASSAL.]
The declaration of war by a \VASSAL gives a free \CB to the controlling power,
to be used now (in reaction), or at any following turn as long as the war
continues.

% Local Variables:
% fill-column: 78
% coding: utf-8-unix
% mode-require-final-newline: t
% mode: flyspell
% ispell-local-dictionary: "british"
% End:
