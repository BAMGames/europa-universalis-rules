% -*- mode: LaTeX; -*-

\newcommand{\iamhere}{\begin{todo} The following is copy of older stuff and de
    facto cease to be relevant before I reread and rewrite it.
\end{todo}}

\definechapterbackground{Military Rules}{military}
\chapter{Military Rules}\label{chapter:MilitaryRules}

\begin{designnote}
  This Chapter focuses on the description of the Military phase in Segment
  order. The main concepts and common rules used during it are described in
  the next Chapter. Most of these concepts are common with many other wargames
  and do not need to be perfectly defined in order to understand the flow of
  the Military phase, which is why precise description is postponed. You
  should refer to the next Chapter whenever a point in these rules requires
  clarification.
\end{designnote}

\begin{todo}
  This Chapter is under heavy work. The random presence of detailed numbering
  of rules reflects this.
\end{todo}

\section{Overview}
\aparag[Sequence]
\MilitaryDetailsNew

\subsection{Military setup}
Setup for the military phase.

\subsection{Rounds}
The military phase is split in a certain number of \terme{rounds} (between 3
and 11). During each round, each alliance, in decreasing order of initiative,
has an \terme{impulse} where it can moves its troop and fight. Some matters
are resolved before any alliance has its impulse (continuation, wintering) and
some after each had its one (siege, piracy, revolt). After that, a new round
begins.
\subsubsection{Continuation roll}
Determine the new round to be played.

\subsubsection{Wintering}
Stacks may suffer from attrition, leaders may be redeployed.

\subsubsection{Impulses}
Each alliance, in decreasing order of initiative, plays an impulse. Each
impulse consists in five steps: supply, choice of campaign, movements (and
interceptions), exploration, and battles.

The alliance taking its impulse is called the \terme{phasing} alliance. All
other are non-phasing.

\subsubsection{Sieges}
Resolve all sieges, fights against \REVOLT/\REBELLION and \corsaire. Declare
new \terme{Exceptional Levies} and use old ones.

\subsection{New round}
A new round begin with the \terme{Continuation Roll} segment.

\subsection{Military cleanup}
Compute the final cost of all campaigns paid this turn.

\begin{playtip}[Simultaneity]
  Most of the time, there are separate wars that cannot affect each other
  (typically, with different antagonists in each), or even separate actions
  for a given alliance (typically action in the \ROTW and in Europe). In this
  case, the resolution of the impulses do not need to be as strictly
  sequential as explained in the rules. The military phase is long enough and,
  typically, a FRA-HIS war and a RUS-POL war can be played simultaneously in
  order to make things a little faster.

  It is normally recommended to synchronise all players for the continuation
  roll (because it is a point where some crucial new information is
  gained). Sometimes, it is however possible for two players to quickly
  completely play a small war (noting the results of the continuation roll in
  secret to communicate it later) while other players are still struggling in
  the first round of a big war. Sometimes even while other are still planning
  and resolving their administration\ldots This typically allows to ``free''
  those players to go and buy food for everybody\ldots

  \smallskip

  Similarly, the rules describe a strict order for the resolution of actions
  in the military phase but in practice it doesn't often need to be respected
  that strictly. Typically, two battles can be resolved simultaneously if they
  don't have the same participants, or sieges can be resolved in any order
  (rather than on a per alliance basis) if players agree that this has no
  importance. The rules, however, must provide a strict order to be able to
  solve any disagreement in the order of resolution of actions for the rare
  cases where it does matter.
\end{playtip}

\section{Military setup}
\subsection{Initiative}
Determine the alliances in effect this turn and the order of initiative
between them. Alliances are transitive (the ally of my ally is my ally) for
this purpose (but not for stacking purpose and such). Each alliance plays at
the lowest initiative of one of its member, resolve any tie at random.

Stacks in interventions (whether limited or foreign) act at the same
initiative than the alliance for which they intervene.

Minors at war alone act at the initiative of the country controlling them.

\subsection{Starting round}
Roll one die to determine the starting round (read the result in the boxes of
the rounds display). This die roll is never modified. The weather for this
round is determined as usual (see below).

It is possible for the Sund to be frozen during the first round if the die was
'1' and \ref{eco:Poor weather} happened this turn.

\section{Rounds}
During each round, perform the segments detailed below. The Impulses segment
is repeated for each alliance in decreasing order of initiative.

Each round is labelled by a letter indicating the \terme{season} ('S'ummer or
'W'inter) and a number (between '0' and '5') indicating the \terme{year}.

\begin{designnote}
  Do not take the seasons and years too seriously when interpreting what
  happens during a turn. This is more a modelisation artefact that gives good
  macro results than a real attempt to simulate military actions on a lower
  scale (especially, the 'S' and 'W' rounds happen simultaneously in the North
  and South hemispheres\ldots)
\end{designnote}

\subsection{Continuation roll}
Do not perform this segment at the first round of a turn.

Roll a die, modified by \bonus{+2} in case of \ref{eco:Poor weather} and
\bonus{-1} if this is period \period{VI} or \period{VII}. Follow the arrows on
the rounds display to determine the new round.

If the new round is the 'End' box, then the rounds stop immediately. Proceed
with \terme{Wintering} then \terme{Military cleanup}.

If the followed arrow is red (modified roll of 1-5 after a Summer round), then
the new round is played with an extended campaign. See choice of campaign for
the implications.

If there is an extended campaign after a (unmodified) continuation roll of
'1', '3' or '5', then the round is played with \terme{Bad weather}.

If \ref{eco:Poor weather} happened this turn, then each Winter round is played
with \terme{Bad weather}, no matter what was rolled (included after a
Winter/Winter transition). Additionally, in this case, if there is a Winter
after an unmodified roll of '1', the Sund is frozen (see \ref{eco:Poor
  weather}).

\begin{playtip}
  End of the Military phase may happen somewhat brutally. You should always
  check the probabilities before planning long term actions (sieges) in years
  4 or 5.

  Given the shape of the rounds track, at least one round in each column must
  happen. Thus, the minimum number of rounds left to play is the number of
  columns and the maximum is twice that number. Moreover, long Military phase
  implies lot of extended campaigns. Take that into account when planning both
  long term actions (sieges) and expenses for the rest of the phase.
\end{playtip}

\subsection{Wintering}
If the current year is different than the year of the previous round, then a
\terme{Wintering} segment occurs. Otherwise, skip to the \terme{Impulses}
segment.

There is a \terme{Wintering} segment when the 'End' box is reached, that is,
it is considered as being 'S6'.

\subsubsection{Cold area}
Any stack in a non-controlled, non-national province within the \terme{Cold
  area} rolls for attrition. Resolve this as a Supply attrition (see
below). It is, however, a different roll and a stack may have to roll for
attrition both during the Wintering and Supply segment in some cases.

\subsubsection{Pashas}
Any stack containing \Timar out of \TUR national territory rolls for
attrition. Check specific Turkish rules for details.

\subsubsection{Hierarchy}
Leaders may be redeployed. Leaders that were wounded during a previous round
come back now.

Each country may choose to redeploy its leaders on its stacks any way it wants
(no maximal distance or such, in other words it's a free teleportation). If it
chooses to do so (including return of wounded leaders), then the hierarchy
must be globally respected after this redeployment. It is always possible to
choose not to redeploy leaders at this point, but as long as at least one
leader is redeployed, the country must respect hierarchy globally.

Exception: besieged leaders as well as leaders on stacks with discoveries that
have not been brought back to Europe must stay in place and are not checked
toward global hierarchy.

\subsection{Impulses}
Each alliance, in decreasing order of initiative, plays an impulse. Each
impulse consists in five steps: supply, choice of campaign, movements (and
interceptions), exploration, and battles.

The alliance taking its impulse is called the \terme{phasing} alliance. All
others are non-phasing.


\section{Supply}
Each phasing land stack which has no supply for two consecutive rounds is
immediately destroyed. Getting back supply temporarily during the round is
enough to avoid this destruction.

Each phasing land stack which has no supply or is in weak supply must roll for
\terme{supply attrition}. Additionally, each phasing besieged land stack must
roll for \terme{siege attrition} (exception: if the fortress was supplied
during the previous round (see naval movement), the besieged stack does not
roll for attrition and is considered in full supply (even if it should
normally be only in weak supply)). Naval stacks never roll for attrition
during the supply segment. Beware that besieged stacks roll for \textbf{siege}
attrition during this segment, which has a similar procedure as supply
attrition but with slightly different modifiers.

If a besieged stack is also in weak supply, it does not roll for supply
attrition. The siege attrition replaces this roll.

\begin{todo}
  Supply markers for besieged fortresses?
\end{todo}

\subsection{\SoS, \LoS, weak supply}
\subsubsection{At land}
A Source of Supply (\SoS) is either a controlled city, \TP, \COL or fort, or a
large enough naval stack in an adjacent seazone. A Source of Supply may supply
as many stacks as wanted.

The Line of Supply (\LoS) goes from a Source of Supply to the stack. The \MP
cost is counted as if the stack itself was doing this movement (typically, \LD
in the \ROTW compute the length of their LoS using the special \MP costs for
\LD).

A stack is in weak supply if at least one of the following condition is true:
\begin{itemize}
\item its Line of Supply is as least 6\MP long (exception: in the \ROTW, \LD
  do not check this condition);
\item its Line of Supply goes through desert;
\item its Source of Supply is not owned by the same alliance;
\item it is supplied by a naval stack that is not adjacent to a port or
  arsenal able to supply it (exception: in the \ROTW, \LD do not check this
  condition).
\end{itemize}

It is possible to take its supply from a further \SoS to avoid weak supply
(typically, to have an owned \SoS, or to circumvent a desert).

A naval stack can act as a \SoS for an adjacent land stack if it is large
enough. Supplying is a naval action (see below) and thus must be declared
during the naval stack movement. It is valid until the next impulse of the
naval stack (which may be during the next turn).
\begin{itemize}
\item A stack containing only \NDE may supply stacks of at most 1\LD;
\item A stack containing only \ND may supply stacks of at most 3\LD and no
  \ARMY counter;
\item A stack with a \FLEET\facemoins (but no \FLEET\faceplus) counter may
  supply any small land stack (up to 5\LD and 1\Pasha);
\item A stack containing a \FLEET\faceplus counter may supply any land stack.
\end{itemize}
This information is summed up in the last two columns of
table~\ref{table:Naval Supply}: find the size of the naval stack (or smaller)
in the ``Naval Size'' column and read in the same line in the ``Land
supplied'' column the size of land stacks that can be supplied.

\GTtable{supplysize}

In the \ROTW, forts (including missions) can only supply \LD. \TP and \COL can
supply any stack.

Supply is never used up. Thus a \SoS can supply several stacks if it can
supply each of them individually.

\subsubsection{At sea}
A naval \SoS is a controlled port or arsenal large enough ot hold the naval
stack. Fort and mission can only supply \ND; \TP, \COL and regular ports can
supply stacks with at most 1\FLEET counter; arsenals can supply any naval
stack.

A naval \LoS goes from the \SoS to the stack. Its length is the number of sea
zones \emph{crossed} (both entered and exited). Thus, a naval stack adjacent
to a port has a \LoS of length 0.

\subsection{Supply Attrition}
Attrition (in Europe) is obtained by rolling one die, modified as follows, and
cross-reference the modified result in table~\ref{table:Discoveries and
  Attrition} in the ``Land, Europe'' column that corresponds to the size of
the stack. In the \ROTW, the result is read in the ``\ROTW and Sea''
column. Note that a result of 5 or less has no effect.

\begin{todo}
  The Attrition table does not want to be included here. I suspect too much
  dependence with adjacent table in the TikZ code\ldots
\end{todo}

%\GTtable{discoveriesattrition}

Modifiers for Supply attrition:
\begin{modlist}
\item[+2] per cause of attrition above the first
\item[+2] in case of \terme{massed forces} (the stack contains more than 5\LD
  and 1\Pasha)
\item[+2] if the stack has no \LoS
\item[+2] if the fortress of the province is controlled by the enemy
\item[-M] MAN of the commander of the stack
\item[+?] if the stack is supplied by a naval stack, and the \LoS of this
  naval stack goes through one or more \StraitFort, add the DRM of all the
  \StraitFort along this path (2 in Europe, level/2 in the \ROTW)
% \item[+6] if the fortress of the province is controlled by allies or if there
%   is no fortress in the province (in the \ROTW)
\item[+1] per \PILLAGE\facemoins, \REVOLT\facemoins or unfriendly
  \REBELLION\facemoins in the province
\item[+2] per \PILLAGE\faceplus, \REVOLT\faceplus or unfriendly
  \REBELLION\faceplus in the province
\item[+?] in an uncontrolled province of the \ROTW cold area, add the number
  of Snowflakes ``resource'' (+0 to +2 depending on the \Area)
\end{modlist}

\begin{designnote}[Massed force]
  Note that a \terme{massed force} is \textbf{not} a cause of supply attrition
  by itself (contrarily to movement attrition) but \textbf{if} the stack
  suffers attrition, it is still an aggravating factor. For the sake of space,
  this is not repeated in the tables and considered to be part of the ``per
  extra cause'' modifier, even if it's not an extra cause \emph{stricto
    sensu}.

  \smallskip

  Note also that for supply (or siege) attrition, the \terme{massed force}
  malus always only happen at 6 or more \LD, even in case of a campaign
  without logistic.
\end{designnote}

Modifiers for Siege attrition (besieged):
\begin{modlist}
%\item[+6] always (friendly fortress)
% \item[+?] in the \ROTW cold area, add the number of Snowflakes ``resource''
%   (+0 to +2 depending on the \Area)
\item[-S] siege value of one besieged leader
\item[+S] siege value of one besieger or blockading leader
\item[-3] if besieged in an unblocked port
\item[+1] per \USURE\facemoins
\item[+3] per \USURE\faceplus
\end{modlist}

\subsection{Result of attrition}
In Europe, the result of attrition is either nothing (---), a number
(between 1 and 3), a 'P' or both a number and a 'P'.

The number indicates how many \LD are lost immediately. The controller of the
stack chooses which.

A 'P' is interpreted differently according to the technology of the stack.
\begin{itemize}
\item If the technology is \TMED, \TREN or \TARQ, then 1 more \LD is lost and
  a \PILLAGE\facemoins is placed into the province (and immediately merged
  into a \PILLAGE\faceplus if there was already another one here).
\item If the technology is \TMUS or better, then the controller chooses to
  either loss one more \LD or place a \PILLAGE\facemoins (note that
  \terme{foraging} has no effect during siege or supply attritions).
\end{itemize}

Besieged troops cannot place \PILLAGE and thus must loss one \LD in case of
'P'. There is no extra effect for the lost \PILLAGE with low
technology. Similarly, if there are already two \PILLAGE\faceplus in the
province, it is not possible to add more and any 'P' must be resolved by
loosing one \LD.

Note that since \PILLAGE will make further attritions harder, it is sometimes
wiser to loss \LD rather than let the troops plunders. Note also that since
\PILLAGE\facemoins will be removed at the end of the turn, small troops with a
not too bad technology don't suffer a lot from attrition.

\smallskip

In the \ROTW, cross reference the percentage obtained with the size of the
stack in table~\ref{table:Attrition ROTW Remainders}. The resulting number
indicates the number of \LD still alive after attrition while the 'd'
indicates one or two \LDE still alive. If there is a \textetoile, then there
is 50\% chance to lose an extra \LDE.

Stacks in \COL of level 6 use the Attrition in Europe procedure.

\GTtable{attritionlosses}

\section{Choice of campaign}
Each country of the phasing alliance chooses its campaign for the current
round. Make a tick in the correct line of the ERS to count how many campaigns
of each type you've done so you can pay for them.

More expensive campaigns allow for more actions. In case of extended campaign,
there is a single campaign that spans over two rounds and that can be upgraded
at this point.

Each country must pay for campaign in order to move its troops, however
multinational stacks are moved as part of the campaign of the commander of the
stack.

\aparag[Campaigns for \MIN]
\bparag Minors in limited intervention have 1 simple campaign each turn and 1
passive campaign each round. The controlling \MAJ may upgrade to any kind of
campaign by paying the difference.
\bparag Minors fully at war (including oversea wars) have 1 simple campaign
each round. They may receive multiple campaigns through reinforcement. The
controlling major may upgrade to an kind of campaign by paying the
difference.

\aparag[Campaigns and interception.] Interception is allowed according to the
last campaign paid.
\bparag For player without initiative, this is the campaign of the previous
round.
\bparag During first round, players without initiative may intercept (before
their first move) as if they had done a passive campaign.

\subsection{List of campaigns}
\aparag[None] 0\ducats: No action, no movement, no exploration, no siege,
\ldots allowed (troops may retreat before battle and will fight back if
attacked). No interception allowed.

\aparag[Passive] 10\ducats:
\bparag Interception allowed only in friendly provinces.
\bparag On land: only passive moves.
% (Moving in friendly provinces ; maintaining sieges and fights against
% revolts ; exploration ; moving \LeaderG (and \LeaderC) alone).
\bparag At sea: Moving stacks of 1\FLEET maximum. No active action.
% \bparag Naval actions: friendly-to-friendly transport, maintain fight against
% \corsaire, exploration, maintain blocus.

\aparag[Active (aka Simple)] 20\ducats: All allowed by Passive plus
\bparag Any interception.
\bparag On land: one stack of $\leq$ 5 \LD + 1 \Pasha may do an active move.
\bparag At sea: any stack may move; one stack with at most 1\FLEET counter
may do an active action, other stacks may only do passive actions.

\aparag[Active/No Logistic] 10\ducats: Same as Active but
\bparag At sea: one stack \textbf{without} \FLEET counter may do an active
action.
\bparag On land: all stacks $\geq$ 3\LD roll for attrition (even if not
moving).

\aparag[Major] 50\ducats: All allowed by passive plus
\bparag On land: either one stack of any size may do an active move;
or all stacks $\leq$ 5 \LD + 1 \Pasha may do active moves.
\bparag At sea: either one stack without restriction (neither size nor acton)
or all stacks with at most 1\FLEET counter may do active actions.

\aparag[Multiple] 100\ducats: all stacks may act without restriction.

\aparag When moving both at sea and on land, the cost of both campaigns is
computed separately and only the maximum cost is paid.

\begin{exemple}
  A Major campaign allows to both:
  \begin{itemize}
  \item attack with one naval stack of 3\FLEET ;
  \item move without attacking (exploration possible) with as many naval
    stacks as wanted (passive actions are not restricted) ;
  \item maintain as many blockades and fight against \corsaire as wanted
    (maintaining blockades and fight against \corsaire are passive actions,
    only initiating them is active).
  \item attack with as many small ($\leq$ 5\LD) land stacks as wanted (the
    reason for which the campaign is Major needs not to be the same at sea and
    on land) ;
  \item move without attacking as many large land stacks as wanted
    (non-active movement is not restricted) ;
  \item maintain as many sieges and fights against revolts with large stacks as
    wanted (only movement is restricted).
  \end{itemize}
\end{exemple}

\section{Movements}
\subsection{Movement}
\subsubsection{Generalities}
Each country of the phasing alliance may move its stacks, according to the
campaign it paid for the round. Each movement must be completed before the
next one start. The phasing alliance choose in which order it moves its stacks
(at random in case of disagreement).

Each non-phasing stack may attempt interception. Resolve battles caused by
interception immediately.

Each phasing stack (moving or not) may have to roll for attrition.

All rules here apply to all movements and may be further restricted by choice
of campaign. That is, the ``no restriction'' in campaigns descriptions should
be understood as ``no \textbf{further} restriction''.

The moving stack may pick up and drop units along its movement, however all
these units are considered as having moved and may not move again. Since each
movement has to be completed before starting the next, this \emph{de facto}
prevents two stacks from rendezvousing, merging, and continue movement
together. Moreover, the distance moved is the total (even if no single unit
did its totality) and in case of attrition the size of the stack is considered
to be the total number of all troops that took part in the movement (and they
may all suffer from attrition).

Combined land and sea movement is possible and it is the only case where two
stacks (one naval and one land) move at the same time. The land stack must
start in a coastal province and embark immediately but may continue moving
after disembarking. The naval stack, however, can move before picking up the
land stack but is constrained in what it does after. If a naval stack
transporting troops suffer loses, troops inside may also suffer some loses.

\subsubsection{Land}
When moving, land stacks expend \MP depending both on the terrain of each
province they enter and the frontier by which they entered it. The cost of
each terrain differ in Europe and in the \ROTW and are indicated in
table~\ref{table:Movement points costs}.

\GTtable{movecost}

Each unit may move a maximum of 12\MP. Stacking limits may be ignored during
movement, but at the end of movement of each stack, units over the stacking
limit must be destroyed (chosen by the controller of the stack).

A land stack may be carried over water by a sea stack. The land stack must
start its movement in the province where it embark but may move after
disembarking. The land stack may embark or disembark from a non-controlled
port (evacuation or landing), but at least one of the endpoint must be a
controlled port/arsenal large enough to contain the naval stack.

A land stack must roll for attrition if at least one of the following cases is
true:
\begin{itemize}
\item the stack contains at least 6\LD (not counting those in \Pasha) or at
  least 2 \Pashas, such large stacks roll for attrition even if they don't
  move;
\item the stack moved at least 6\MP;
\item the stack moved at least 3\MP and this is a campaign without logistic;
\item the stack used ships to make a naval move and at least one of the
  endpoint is not a controlled port/arsenal large enough to hold the
  transporting navy.
\end{itemize}

Note that large stacks must roll for attrition even if they don't move. Thus,
it is usually better to have half of the stack doing a back and forth move so
that none of the part has to test attrition. While this is harmless at home,
doing this on the frontline can increase the cost of campaign (because coming
back on a siege is an aggressive move) or offer opportunities for interception
to your opponents (and thus opportunities to crush your stack half by half),
so use this trick wisely.

In case of interceptions, the conditions for attrition are checked for each
part of the movement (between the beginning and the first interception;
between each interception; between the last interception and the end). If one
part of the movement triggers attrition, it is resolved immediately before the
interception battle occurs and its results (especially \terme{foraging}) are
applied and effective for the next battle only. If none of the parts trigger
interception but the whole movement does, then one and only one final
attrition is resolved at the end of the movement (thus, possibly after a lost
interception battle).

\begin{exemple}
  A small stack moves 6\MP and is intercepted. Since 6\MP is a cause of
  attrition, it resolves it immediately. If the stack wins the battle, it may
  move on. It moves 6 more \MP (for a total of 12) and thus rolls for
  attrition again. If it has a regular battle at the end of the movement, then
  any \terme{foraging} result obtained during the first interception is
  ``used'' by the first battle and no more effective.

  \smallskip

  A small stack moves 4\MP, get intercepted and wins, move 4\MP, get
  intercepted and wins and moves a final 4\MP. None of the ``legs'' of
  movement were enough to trigger attrition, but the total movement of 12\MP
  is. Hence the stack makes a final attrition test at the end of its
  movement. Note that even if the first two ``legs'' represent 8\MP and thus
  would trigger attrition, it is not done before the battle as only the latest
  ``leg'' is checked at this point. However, if the stack loses its second
  battle, then it must stop movement and has moved 8\MP, thus it must resolve
  an attrition test now (after the battle).

  \smallskip

  A small stack moves 6\MP and get intercepted. It resolves its attrition and
  then wins the battle. It moves 5 more MP before ending its moves. The last
  leg is not enough to trigger attrition. The whole movement is enough to
  trigger attrition but since an attrition test already happened (before the
  battle), it is considered done and no extra test occur.
\end{exemple}

Any movement that enters a province with unfriendly forces (land stack,
\REVOLT/\REBELLION, fortress (even if besieged), \ldots) is \terme{active} and
cannot occur as part as a passive move. Passive campaign allow only passive
moves while other campaigns allow one or more active moves. Note that not
moving from, or exiting an unfriendly province can be done as a passive move.

Land stacks may not enter provinces that contains neutral unbesieged forces.

When exiting a province whose fortress is not controlled, a stack must either
empty the province (removing the siege) to go back to a friendly province (no
enemy (even besieged), no \REVOLT/\REBELLION, \ldots) or leave enough troops
to maintain siege (1\LD per level of the fortress).

Movement of leaders alone is allowed. A leader moving alone has a move
capacity of 20\MP. It never rolls for attrition. It can use naval transport
even if it does not start adjacent to sea, and it does not need a naval stack
to be carried (the leader sails on some non represented ships). On sea, a
leader alone cannot enter sea zones with malus and has (for the land leaders)
a \Man of 0. A leader alone dies immediately if he enters a province with
unbesieged enemy forces (and no friendly force) or if he is
intercepted. Moving leaders alone can help enforce hierarchy or quickly change
your frontline. It is considered a passive moves even if the leader enters a
province with enemy force (typically to take command of a stack). Note that
hierarchy rules strictly constraint moving leaders alone and such change of
command is normally done during the wintering segment.

A stack entering a province with enemies may declare an \terme{overrun} if it
contains at least four times as many \LD. Overruns are resolved after all
interception attempts in the province are resolved. If the phasing stack
contains at least 8 times as many \LD, then the non-phasing stack is
immediately destroyed without battle and its leader must roll for wounds as
after a regular battle where all the troops are destroyed; the phasing stack
does not roll for attrition before the overrun and may then continue moving as
if nothing happened. If the phasing stack contains at least 4 times as many
\LD, then the battle occurs immediately, as an interception battle (roll for
attrition if needed, the phasing stack may continue movement if it wins), the
phasing stack is still the attacker.

Phasing stacks may not enter provinces with presence (even besieged) of
neutral countries (that are neither at war with (allied) nor against (enemy)
the phasing alliance). In the \ROTW, provinces that do not explicitly belong to
minors are not concerned by this restriction.

In the \ROTW, each time a land stack enters a province whose natives are not
yet activated, it must test for activation. Exception: leaders alone never
test for activation, natives in some \ROTW countries are not tested if the
moving stack has an \dipAT with the country.

To test for activation, roll 1d10. If it is equal or less than the
\terme{Tolerance} of the \Area (third value), the natives of the province
(only) are activated and attack the stack. Treat this as an interception
battle where the natives are the interceptor.

Before the test for activation, the stack may decide to attack the natives. In
this case, do not roll for activation. The phasing stack must stop and the
battle will be resolved together with regular battles. Note that even if this
is a regular battle, there was no enemy in the province before the stack
entered it (and activated them), hence this is still a passive move.

In the \ROTW, a land stack may try to convert natives to its side in each
province it moves into by rolling on Table~\ref{table:Conquistadors
  Effects}. It may only attempt to do so if the natives in the province are
not already activated when it enters it. This roll is made before any
activation test and will automatically activate the natives.

\GTtable{conquistadors}

\aparag[Conquistador table]
The Conquistador table may be used only:
\bparag in \continent{America} and \continent{Africa}, by any \LeaderC and
\LeaderE (half values) ;
\bparag in \continent{Indonesia} by \leader{Coen}, \leader{van Diemen} and
\leader{Maetsuycker} only ;
\bparag in \continent{India} by all \LeaderC restricted to \continent{Asia}
(@). Namely, \leader{Clive}, \leader{Dupleix}, \leader{Bussy} and the minimum
\LeaderC@ of \FRA and \ANG in period VII.

Roll 1d10, modified as follows and cross-reference the result with the sum of
the \LeaderC value. The result may contain: a \textdag, a ---, a R followed by
a percentage or a D followed by a percentage. Apply all the results obtained,
starting with the D.

List of modifiers (cumulative):
\begin{modlist}
\item[+1] for each previous time in the game that the table was used in any
  province of the \Area.
\item[-1] if there is a \LeaderMis in the stack.
\item[+1] if the stack contains at least 4\LD.
\item[-1] if the stack does not contain any \ARMY counter (only take into
  account counters of actual countries, not the generic \pays{natives} ones).
\item[+1] if the leader using the table has a sum of stats of 6 or less.
\end{modlist}

Explanation of the results:
\begin{itemize}
\item ---: all the natives in the province resist (same as 'R00').
\item \textdag: no native resist (same as 'R100').
\item Dxx: apply Table~\ref{table:Attrition ROTW Remainders} with xx\% on the
  number of natives in the province. The result is the number of native \LD
  that desert and join the stack. Use \pays{natives} counters to denote them
  (using \ARMY counters to denote 2 or 4 \LD as needed).
\item Ryy: apply Table~\ref{table:Attrition ROTW Remainders} with yy\% on the
  number of natives in the province. The result is the number of native \LD
  that resist and fight.
\item All natives that neither resist nor desert are eliminated from the
  province (they will come back at the end of the turn following normal
  rules).
\end{itemize}

Next, a battle occurs between the resisting native and the stack (including
the new recruits). Treat this as an interception made by the resisting
natives.

Natives in a stack never count toward stacking limit and do not hamper
technology or moral. They must stay stacked with the \LeaderC that used the
table and cannot use naval transport. They are automatically disbanded if the
\LeaderC is wounded or killed, leave land, or at the end of the turn.

\begin{exemple}
  If there are 6\LD of natives in the province and the result is 'R30/D80' the
  by cross-referencing 80\% with 6\LD we see that there is 1\LD of natives
  joining the \LeaderC and by cross-referencing 30\% with 6\LD, we see that
  there are 4\LD that resist and fight. The last \LD of natives is
  killed. That is the province now only contain 4\LD (those that resist) which
  are activated and fight following usual rules.

  Note that due to the way the table is read, 'R30/D80' actually means that
  there is 100-80=20\% of the natives that desert and 100-30=70\% that resist.
\end{exemple}

\subsubsection{Sea}
Naval stacks have an unlimited movement capacity. A naval stack that does not
stay in a port always rolls for attrition (even if it does not move and simply
stays at sea). The attrition roll is modified by the distance moved, hence the
farther a naval stack moves, the more attrition it suffers (especially for
large stacks).

Each new sea zone or port entered during the movement is counted as a ``zone''
for the distance. Entering ports is usually done at the end of movement, to
avoid certain dangerous sea zones, or to pick up troops for combined move.

At the end of its movement, a naval stack is allowed to do one naval action
(some actions are actually composed of several others). This is never
mandatory. Especially, battle is not mandatory and naval stacks of different
alliances may coexist in the same sea zone without problem. Moreover, in order
to engage an enemy naval stack one must first find it.

Most actions are active and can only be done as part of an active
campaign. Some are passive and can be done as part as any campaign. Note that
many passive actions are simply ``continuing the active action from previous
round without moving.''

Actions are announced when the naval stack enters the sea zone where the
action will occur. Announcing a naval action ends the movement of the stack
and attrition (for the whole movement) is rolled immediately before any
interception (in the last zone) is declared and before actually resolving the
action.

Naval stacks may only enter port/arsenal that are \emph{large enough}, that is
that are \SoS for it. Especially, it is never possible to enter a port/arsenal
controlled by another alliance.

Naval stacks may be intercepted. In case of interception (during movement), do
not roll for attrition before the interception battle. Attrition is only
resolved once at the end of movement. The attrition takes into account the
whole movement of the stack, whatever the number of interception battles that
may have occurred during it. If a naval stack is intercepted and looses
battle, roll for attrition only once (for both movement and retreat), taking
into account the whole movement (\emph{e.g.} the distance moved is the total
of what was moved before the battle and during the retreat).

Each time it tries to enter or exit a blockaded port, a stack must roll to
escape blockade. Roll 1d10 modified as follows:
\begin{modlist}
\item[+M] \Man differential between the moving leader and the blockading
  one. Only count it if it is positive (\emph{i.e.} if the moving leader has
  more \Man).
\item[+1] if the blockading stack is smaller (in number of \ND).
\item[+1] if the blockading stack is composed of \NWD and does not have
  technology \TSF.
\end{modlist}

If the result of the modified roll is 8 or more, the stack managed to escape
the blockade and may pursue its movement as wanted.

If the result is 6 or 7, the stack did not manage to escape blockade. If it
was trying to exit a port, it must stays in (and its movement ends). If it was
trying to enter it, it may either stop moving or return to the closest
friendly port/arsenal (at choice, if any), in both cases it may not do any
action.

If the result is 5 or less, the stack did not manage to escape blockade (as
above). Additionally, the blockading stack may choose to immediately engage
the moving stack. This is treated as an interception battle (\emph{i.e.} the
blockading stack automatically succeeds any interception attempt against the
moving stack).

\GTtable{navyblockade}

\aparag[List of active naval actions.]
\bparag[Battle.] The moving stack declares that it attempts battle. It must
successfully intercept its target (exception: if the target stack is
blockading, this interception automatically succeeds). The battle will be
resolved during the Battle Segment of the impulse. A non-phasing stack that is
engaged in battle may not intercept any more. Each non-phasing stack may only
be attacked by one phasing stack (failed interception do not count as an
attack even if they do count as an action)
\bparag[Blockading an enemy port/arsenal.] The naval stack must have
sufficient size according to \ref{table:Naval Size for Blockade}, depending on
the size of the fortress. Note that to blockade a level 2 or 3 fortress, one
must have have at least a \FLEET\Facemoins counter \textbf{and} 2\ND in the
stack similarly blockading a level 4/5 fortress requires a \FLEET\Faceplus
counter \textbf{and} 3\ND in the stack. Put the stack close to the blocked
port/arsenal. Blockade makes siege easier. Only one stack may blockade each
port (if several stacks want to blockade the same port, they are automatically
merged). Blockade last until the next movement of the naval stack, including
possibly in the next turn.
\bparag[Supplying troops.] The naval stack must have sufficient size for
supplying the troops as indicated in \ref{table:Naval Supply}. Note that a
naval stack is allowed to supply several adjacent land stacks since it acts as
a \SoS and supply is never ``used up''. Supply last until the next movement of
the naval stack, including possibly in the next turn.
\bparag[Blockade + Supply.] A stack may both blockade a port/arsenal and
supply land troops. However, in this case it can only supply the land stack in
the province of the port/arsenal it is blockading.
\bparag[Supplying a besieged port.] The stack must have exited a non-blocked
friendly port (either starting from it or moving in and out) during its
movement and entered the besieged port. Moreover, it must be a stack large
enough to blockade the supplied port. This will remove the ``blockade''
situation of the port for the siege segment of this round (only) and remove
any attrition roll for troops inside for the next supply step (only). Note
that this is mostly useful if the port is blockaded and thus require escaping
it\ldots Moreover, the stack will end its movement in the supplied port as
actions end movement. It is possible to decline the possibility to supply in
order to continue moving (for example to embark besieged troops and retreat
them before the fortress falls).
\bparag[Fight \corsaire.] The stack can initiate fight against \corsaire from
the sea zone it is in. The \corsaire counter (or counters) targeted by the
stack must be in the same sea zone. Only one stack per alliance may fight
against \corsaire in a given sea zone.
\bparag[Naval Transport.] See details below. Can be combined with blockade
and/or supply of the invaded province (or reinforced fortress) only.
% \bparag[Naval transport in the \ROTW.] In the \ROTW, when either the naval
% stack contains a \LeaderE or the land stack contains a \LeaderC or a
% \LeaderGov, then the disembarking itself is not an action. The naval stack may
% continue moving and may do an action, it may also do the disembarking after
% doing its action. Doing this is still considered as active for campaign
% purpose (\emph{i.e.} a naval stack may continue moving after a \ROTW invasion
% of an enemy province, and this stack may do another active action later; but
% even if it does not do any active action later, the invasion is still
% considered ``active'' for campaign cost purposes).

% (Jym): No. F acting after disembarking means 2 stacks allowed to move at the
% same time. It's bad.

\aparag[List of passive naval actions.]
\bparag[Exploration.] Resolved during next step.
\bparag[Continuing blockade and/or supply.] The naval stack may not move and
must have already been blockading the same port (or supplying from the same
sea) on the previous round (including on previous turn if this is the first
round). Conditions on stack size and on combined blockade+supply are still
enforced.
\bparag[Continuing fight against \corsaire.] The naval stack may not move and
must have already fighting \corsaire from the same sea zone on previous
round. Only one stack per alliance may fight against \corsaire in a given sea
zone.
\bparag[Friendly naval transport.] If both endpoints of the naval transport
are friendly, the transport is a passive action. Additionally, in the \ROTW,
if the combined stack contains a \LeaderE, \LeaderC or \LeaderGov and no \ARMY
counter then embarking from or disembarking in a province with no enemy
presence (activated natives, city, establishment, troops, \ldots) is
considered as a friendly port (not arsenal).
% \bparag[Friendly naval transport in the \ROTW.] As for active transport, a
% passive transport is not an action in the \ROTW if the combined stack contains
% a \LeaderE, \LeaderC or \LeaderGov. Note that if the land stack tries to
% disembark in an unknown province and fails its exploration, it must reembark
% and thus will be destroyed if the naval stack is not here anymore\ldots

\subsubsection{Combined move}
A combined move, or naval transport, happens when a naval stack carries a land
stack and they move together. The land stack may not move before embarking but
may move after. On the other hand, the naval stack may move before embarking
troops but disembarking troops is a naval action and ends its movement. Hence,
there is always only one stack moving, first it is solely naval, then it is a
combined land and naval stack and finally it is solely land. A naval stack may
not embark troops if it has already been engaged in a battle this impulse
(\emph{i.e.} after winning an interception battle).

The naval stack must be large enough to hold the land stack, as indicated in
Table~\ref{table:Sea Transport for Armies}. Each \ARMY\Faceplus, depending on
its \terme{army class} and the period, needs a certain number of
\terme{transport points} as indicated in the table. An \ARMY\Facemoins
requires half that number, a \LD requires 2 points and a \LDE half a
point. Conversely, each \NWD or \NGD provides 1 point, each \NTD 3 points and
each \NDE half a point. The sum of the transport capacity of the naval stack
must be at least equal to the sum of the transport points required for the
land stack.

\GTtable{fleettransport}

Either the start (evacuation) or end (invasion) of the naval transport may be
uncontrolled but not both. That is, at least one of the endpoint must be a
controlled port or arsenal large enough to contain the naval stack. Since the
transport is an action of the naval stack, when disembarking in a controlled
port, the naval stack ends its movement in that port, not in the adjacent sea
zone. It is possible to choose to disembark out of the port to keep the naval
stack at sea but this obviously removes any advantage of disembarking in the
port (it becomes an active action, cannot be done after an evacuation, costs
more \MP for the land stack and is a cause of attrition for it).

Even if the transport is an action for the naval stack it may be followed by a
blockade+supply of the landing province or city (only). That is, the action is
transport + blockade + supply (invasion) or transport + supply (reinforcing a
besieged fortress).

As any action, the landing must be announced when the naval stack enters the
zone, before any interception are declared or resolved. Announcing a landing
gives bonus to interception. As for any action, it is resolved after any
interception and resulting battle (and thus only if victorious).

If both endpoints contains a large enough controlled port or arsenal, then the
transport is a passive action. Otherwise, it is an active action. If the
combined stack contains a \LeaderE, \LeaderC or \LeaderGov, then any province
with no enemy presence (activated natives, enemy troops, enemy city or
establishment (even besieged), \ldots) is considered to contain a controlled
port (not arsenal) (both for embarking and disembarking).

Note that disembarking troops in a besieged fortress is actually a passive
action since the port is controlled. However, if the fortress is resupplied as
part of the same move, then it becomes an active action as supplying a
fortress is an active action. It is possible to decline the possibility to
re-supply the fortress in order to keep the transport as a passive action.

A land stack may not stay in the ships. That is, both embarking and
disembarking must occur as part of a single move. Especially, if the naval
stack is intercepted and looses the battle, it retreats to port (as per normal
rules) and the land stack is automatically disembarked. As usual, the land
stack may continue its move after such an automatic landing (it did not loose
an interception battle, only the naval stack did).

As an exception, in the \ROTW, a land stack that do not contain any \ARMY
counter may stay inside ships at the end of movement and even from one round
to the next. If it starts the round in the naval stack, it is considered to
have embarked in a controlled port for all purposes.

If the naval stack suffers attrition or losses during exploration, then the
land stack suffers the exact same percentage of loses (one cannot choose which
ships are sunk in a storm\ldots) This does include the attrition done for the
movement of the naval stack as it is resolved before the landing itself
(attrition of naval stacks is always resolved before their action). However,
if the naval stack suffers loses from battle, the land stack does not suffer
any damage (it is assumed that the loaded ships were better protected and the
exposed empty ships took the damage).

If the transport capacity of the naval stack falls below the requirement to
carry the land stack (due to loss during attrition, exploration or battle),
then the land stack immediately suffers enough loss to be small enough to be
carried by the resulting naval stack.

For the land stack, embarking or disembarking out of a controlled port is a
cause for attrition. It also costs more \MP than if both endpoints are
controlled ports.

If the land stack fights in the province where it lands, whatever the cause
(previous enemy presence or interception), it will suffer from the
disembarking malus on the 1st day of battle. This includes both the case where
the land stack disembarks in a besieged controlled fortress and immediately
attempts a sortie against the besieger and the case where a landing is
intercepted in a previously completely friendly province.

Gold can be carried as land forces. It requires 1 transport point for every
5\ducats. Gold can only be embarked in a controlled port and is immediately
disembarked when it reaches a friendly port on the European map (and added to
\lignebudgetlong{Gold from ROTW and Convoys}). Gold may stay inside ships from
one round to another. It does suffer attritions and exploration loses in the
same proportion than the carrying stack. During battle (especially during the
retreat), gold can be specifically targeted by the enemy. The \terme{Flota del
  Oro} and \terme{Flota del Per\'u} convoys may also be used to carry gold (it
is tere only purpose). On the other hand, the others convoys are already
created full of gold and cannot be used to carry more.

% Autre limite, un pion naval ne peut en général transporter qu'un seul pion
% terrestre. Pour être plus précis, un pion F de navires peut transporter
% jusqu'à une A+ (soit l'équivalent de 4 DT, éventuellement en 2 pions, mais pas
% 3) ; si le transport est assuré par des DGa et des DTr dans F, la limite passe
% à 2A+ en 3 pions. Un détachement naval (de toute nature : DTr, DN, DE) ne peut
% contenir plus d'un pion (donc DC, DT ou A- pour les armées les plus petites).

% (Jym) : bof, ça me semble bien trop compliqué à gérer pour ce que ça
% apporte.

\subsection{Attrition}
\subsubsection{Generalities}
Moving stacks may have to roll for attrition. The procedure is similar on land
and at sea but with different modifiers and a different way to read the
table. Attrition is always resolved by cross-referencing a modified die roll
and the correct column in Table~\ref{table:Discoveries and Attrition}. On sea
or in the \ROTW, a further read in Table~\ref{table:Attrition ROTW Remainders}
is required.

Beware that even if the procedure is very similar to supply or siege
attrition, the modifiers and the way to implement the results are different.

\subsubsection{Land}
Check the movement rules to determine when a moving land stack has to roll for
attrition. The modifiers are:
\begin{modlist}
\item[+2] per cause of attrition above the first
\item[-M] MAN of the commanding leader
\item[+2] if the stack had no \LoS at the beginning of its move
\item[+?] if, at the beginning of its movement, the stack is supplied by a
  naval stack, and the \LoS of this naval stack goes through one or more
  \StraitFort, add the DRM of all the closed \StraitFort along this \LoS (2 in
  Europe, level/2 in the \ROTW)
\item[+2] if the stack \emph{enters} at least one enemy province during its
  movement (\emph{i.e.}  not when exiting enemy territory)
\item[+1/+2] per \PILLAGE\Facemoins/\Faceplus, \REVOLT\Facemoins/\Faceplus and
  unfriendly \REBELLION\Facemoins/\Faceplus in each province exited or entered
  during the move (\emph{i.e.} count all the \PILLAGE, \REVOLT and unfriendly
  \REBELLION along the path, including in the start and end provinces)
\item[+?] for each uncontrolled province of the \ROTW cold area exited or
  entered during the move, add the number of Snowflakes ``resource'' (+0 to +2
  depending on the \Area)
\end{modlist}

In Europe, cross-reference the result with the ``Land, Europe'' column of the
Table that corresponds to the size of the stack. The result of attrition is
either nothing (---), a number (between 1 and 3), a 'P' or both a number and a
'P'.

The number indicates how many \LD are lost immediately. The controller of the
stack chooses which.

A 'P' is interpreted differently according to the technology of the stack.
\begin{itemize}
\item If the technology is \TMED, \TREN or \TARQ, then 1 more \LD is lost and
  a \PILLAGE\facemoins is placed into one of the provinces of the movement
  (possibly the start or end of the move).
\item If the technology is \TMUS, \TBAR or \TMAN, then the controller chooses
  to either loss one more \LD; or to both place a \PILLAGE\facemoins in one
  province of the movement and to \terme{forage} during the next battle of
  this impulse (only).
\item If the technology is \TL then the controller chooses
  to either loss one more \LD; or to place a \PILLAGE\facemoins in one
  province of the movement.
\end{itemize}
As usual, two \PILLAGE\Facemoins are immediately merged into a
\PILLAGE\Faceplus. It is not possible to add a \PILLAGE in a province that
already contains two \PILLAGE\Faceplus.

Note that \terme{foraging} only affects one battle (interception or regular,
whichever occurs first) and necessarily during this impulse. Hence, a stack
that does not fight immediately will not suffer from \terme{foraging},
typically, if the opposing alliance attacks it during the next impulse.

\smallskip

In the \ROTW, the result is read in the ``\ROTW or Sea'' column. Cross
reference the percentage obtained with the size of the stack in
Table~\ref{table:Attrition ROTW Remainders}. The resulting number indicates
the number of \LD still alive after attrition while the 'd' indicates one or
two \LDE still alive. If there is a \textetoile, then there is 50\% chance to
lose an extra \LDE. Repeat this procedure for every set of 10\LD in the stack,
and for the remainder.

If all the provinces of the movement (including start and end) are European
(including \COL of level 6), use the Attrition in Europe procedure. If at
least one is a \ROTW province, use the Attrition in the \ROTW province. Note
that only the land provinces are considered for this. Sea zones crossed during
a naval transport play no role in deciding whether to use attrition in EUrope
or in the \ROTW.

\subsubsection{Sea}
Naval stacks at sea always roll for movement attrition, even if they are not
actually moving. Naval stacks that stay in a port do not roll for attrition.

When computing Attrition modifiers for naval stacks, it is important to know
the \emph{greatest sea difficulty}. First, consider all the sea zones of the
movement (including start and end) and look at their difficulties; next reduce
the difficulty of any sea zone with a \SoS for the stack (\emph{i.e.} friendly
large enough port) by 2; lastly take the greatest of these modified values.

\begin{exemple}
  If a naval stack moves from a sea zone of difficulty 5 with a port into a
  sea zone of difficulty 4 without port, then its greatest sea difficulty is
  4.
\end{exemple}

When computing both distance moved and distance to port, remember that only
the number of zones (either ports or sea zones) \emph{entered} counts. Thus a
naval stack adjacent to its port has a \LoS length of 0 and a naval stack that
simply moves from a port to the adjacent sea zone has a move length of 1.

List of naval attrition modifiers:
\begin{modlist}
\item[-6] Always
\item[-M] MAN of the commanding leader
\item[+?] if, at the beginning of its movement, the \LoS of the stack goes
  through one or more \StraitFort, add the DRM of all the closed \StraitFort
  along this \LoS (2 in Europe, level/2 in the \ROTW)
\item[+X] add the malus of each sea zone with malus \emph{entered} during the
  move
\item[+1] for \NWD in \TCAR technology
\item[-1] for any stack in \TBAT technology
\item[-2] for any stack in \TVE or \TTD technology
\item[-3] for any stack in \TSF technology
\item[+2] if there is \terme{Bad weather} this round
\item[+?] Greatest sea difficulty (as explained above)
\item[+1] per 4 zones entered (round down), if the stack contains 0 or 1
  \FLEET counter [BLP]
\item[+2] per 2 zones entered (round down), if the stack contains 2 \FLEET
  counters [BLP]
\item[+4] per 2 zones entered (round down), if the stack contains 3 \FLEET
  counters [BLP]
\item[-1] if the stack contains several \FLEET counters and moves from an
  arsenal to another arsenal
\item[+3] if the stack starts its movement 1 sea zone away from its port
  (\emph{ie} not adjacent to port)
\item[+6] if the stack starts its movement 2, 3 or 4 sea zones away from its
  port
\item[+9] if the stack starts its movement 5 or more sea zones away from its
  port
\item[+1/+2] per unfriendly \corsaire\Facemoins/\Faceplus in each sea zone
  exited or entered during the move (\emph{i.e.} count all the \corsaire along
  the path, including in the start and end zones) [BLP]
\item[-S] siege of one \LeaderA or \LeaderE in the stack if it is blockading a
  port a the start of its move [BLP]
\item[+?] half the level (round up) of fortress blockaded at the start of the
  move [BLP]
\end{modlist}
Note that the distance moved modifier (for large stacks) is ``+2 (or 4)/2
zones'' and \textbf{not} ``+1 (or 2)/1 zone''. Thus, a large stack that only
moves one zone does not have this malus.

The result is read in the ``\ROTW or Sea'' column. Cross reference the
percentage obtained with the size of the stack in table~\ref{table:Attrition
  ROTW Remainders}. The resulting number indicates the number of \ND still
alive after attrition while the 'd' indicates one or two \NDE still alive. If
there is a \textetoile, then there is 50\% chance to lose an extra \NDE. If
the stack contains more than 10\ND, apply this procedure for each set of 10\ND
and for the remaining \ND separately. The controller of the stack chooses
which \ND are sunk.

Any \NTD that must suffer at least 1\NDE of loss is entirely destroyed.

\subsection{Interception}
\subsubsection{Generalities}
Non-phasing stacks may attempt to intercept a moving enemy stack under certain
conditions. Interceptions are declared whenever a moving stack enters a new
zone (sea zone or province) and resolved immediately. Interceptions can be
attempted by stacks in the same or adjacent zone (sea zone, port or province)
All interceptions are announced before any is resolved.

A non-phasing stack may decide to split before intercepting, that is intercept
with only part of the stack (typically to maintain a siege or blockade with
the other part). Hierarchy and other usual rules for movement and splitting
stacks must be respected in such a case.

If an interception is successful, the intercepting stack is moved into the
zone of the interception (note that this does allow free moves that are not
counted as part as any campaign and may allow, typically, to lay new
sieges). Once all interceptions are resolved, a battle occurs immediately
between the moving stack and all the successful interceptors.

If stacks of different alliances attempt to intercept the same stack in the
same zone, then interceptions are announced and resolved in decreasing order
of initiative. Once one alliance successfully intercepted one stack in a zone,
other alliances may not attempt to intercept the same stack in the same zone
anymore (to avoid three-sided battles). It is however possible to intercept in
another zone if the moving stack wins and goes on.

After an interception battle, if the moving stack won it may continue moving,
if it didn't it must stop movement. For campaign costs purposes, check only
the part of the movement that was effectively done, not the intention
(\emph{i.e.}  if a naval stack declares a landing (an active action) but is
defeated in an interception battle before it actually occurs, then it has done
no action and does not count toward an active campaign, similarly, a land
stack that is defeated before it could enter enemy territory only did a
passive move).

A stack that already lost a battle this impulse (\emph{i.e.} a previous
interception) may not attempt to intercept. A stack that is already engaged in
battle (to be resolved after all moves) may not attempt interception.

Each non-phasing stack may attempt to intercept each moving stack only once
during the whole move (not once per zone). If a moving stack drops and picks
up units, it is considered as a single moving stack even if all the units
actually composing it changed.

Interceptions are resolved by a modified die roll. These rolls are similar on
land and at sea but with different modifiers. \Man plays a huge role in these.

\subsubsection{Land}
Non-phasing stack of countries who made no campaign at all during their
previous impulse (except during the first round if they had no impulse yet)
may not intercept at all. Non-phasing stacks commanded by countries who paid
only a passive campaign during their previous impulse (including, during the
first round of each turn, stacks that had no impulse yet) can only intercept
in a friendly province (no enemy presence, even besieged).

A non-phasing stack which is in the same province as a non-besieged,
non-moving phasing stack is engaged into battle and cannot intercept. Thus,
whenever a moving stack enters a province, any non-phasing stack here may (i)
attempt to intercept and fight immediately, in case of victory it will be able
to intercept again this impulse but in case of defeat the moving stack may
continue its move; or (ii) do nothing, this \emph{de facto} locks both the
moving and non-phasing stack in a battle (to be resolve later) but prevent
further interception from that stack, moreover, the phasing alliance may now
move more troops in the province before the battle occurs.

After one or more successful interceptions, before the interception battle is
resolved, phasing stacks may attempt to counter-intercept. Stacks already
``locked'' in battle may not counter-intercept. Counter-interception is
resolved in the same way as interception. It is not possible to
counter-counter-intercept. That is, first non-phasing players declare and
resolve all interceptions, next phasing players declare and resolve all
counter-interceptions and finally the battle is resolved. As any interception,
counter-interceptions are free (they do not count toward campaign cost).

Before resolving the interception battle, check if attrition of the moving
stack is triggered and resolve it if any. Any \terme{foraging} result apply
for the interception battle. Intercepting and counter-intercepting stacks do
not roll for attrition.

The interception battle normally occurs between all the successful
interceptors versus the moving stack and all the successful
counter-interception. If one side contains troops of an European country and
more than 8\LD+2 \Pashas, exceeding troops do not fight (but stays in
place). After the battle, if there is still too many troops in the province
(more than the stacking limit), any intercepting or counter-intercepting troop
may choose (controller's choice) to retreat in the province where it was
before intercepting. This retreat does not trigger attrition and does not
counted toward any campaign cost. Any exceeding troops after that are
destroyed. Such an overstacking typically occurs when many unlikely
interceptions are declared and they all succeed\ldots

Note that if a stack moves into a province where there is another friendly
stack (typically in order to merge stacks) and get intercepted, the
interception battle occurs immediately after entering the province, hence
\textbf{before} the two phasing stacks actually have a chance to
meet. Especially, the immobile phasing stack does not participate in battle
(it may attempt to counter-intercept in order to join the battle). In such a
case, the intercepting stack must retreat after battle (in the province where
it came from) even if victorious (this retreat does not trigger attrition and
does not count toward campaign cost).

\begin{designnote}
  Fighting separate enemies before they could merge was one of Napoleon's
  favourite move\ldots
\end{designnote}

To resolve an interception, roll a die modified as follows (all modifiers are
cumulative):
\begin{modlist}
\item[\textplusminus?] \Man differential between the commanding leaders
  (intercepting stack-intercepted stack); in case of counter interception,
  consider that all the interceptors are merged in a single stack before
  finding the commanding leader.
\item[+1] If the intercepting stack has a technology which is at least 6
  levels above the technology of the intercepted stack.
\item[+1] If the target province contains a friendly force (other than the
  intercepting one) or city (even besieged).
\item[+1] If intercepting in the same province.
\item[-1] If the target province contains swamps.
\item[-1] If the intercepting stack is in a swamp province or a flooded
  province.
\item[-2] If there is \terme{Bad weather}.
\item[-2] If the interception occurs across a river or mountain pass.
\item[-2] If the target province contains an unbesieged enemy force other than
  the intercepted one.
\item[-1] If the interceptor is besieging.
\end{modlist}

If the unmodified die is 10, or if the modified roll is 8 or more, then the
interception is successful.

\subsubsection{Sea}
\subsubsection{\Presidios}
\subsubsection{Convoys}
\iamhere

\aparag Actions passives
\bparag Interception : une flotte d'un joueur inactif (pas en phase)
peut tenter d'intercepter chaque pile qui bouge dans sa mer ou une mer
adjacente (autant de tentatives que d'opportunités, mais un pays/une
alliance ne peut faire qu'une seule tentative par mer).
Si au port : dans une des mer adjacente.

\aparag Interception is allowed according to the last campaign paid.
\bparag For player without initiative, this is the campaign of the previous
round.
\bparag During first round, players without initiative may intercept (before
their first move) as if they had done a passive campaign.

\aparag[Rappel des mod. d'interception] VOIR TABLE
\interceptiona
\bparag Pour les flottes faisant Invasion/Naval Transport : le bonus +2 remplace
le malus de -3 au port ou le bonus de +1(même mer) si intention de
débarquée a été indiquée. Pour une flotte en mer: +2 si c'est dans la
même zone; flotte dans arsenal : +2 dans les mers qui bordent l'arsenal;
flotte dans port: +2 si c'est dans la province du port.

\aparag[Pour l'interception]
\bparag Noter que c'est l'intercepteur qui attaque toujours.
et que la bataille est résolue tout de suite entre la pile qui intercepte et la force interceptée.
Il peut choisir de regrouper dans sa pile les forces immobiles amies dans la province.
\bparag un intercepteur peut être intercepté à son tour [les forces qui
interceptent viennent en renfort de la pile qui a été la 1e interceptée] ;
c'est tout de même le joueur qui n'est pas en phase qui est l'attaquant.
\bparag Si une force intercepte dans une zone où il y a déjà une force ennemie,
cette for peut intercepter comme par la paragraphe précédent pour se joindre à
la défense [remarque : il peut alors y avoir plus de 8 DT, les forces supplémentaire
n'ont simplement aucun effet sur le combat].
\bparag  Si un combat d'interception a lieu dans une province avec des forces non impliquée
dans le combat, ces forces suivent la retraite éventuelle de la pile allée engagée en
bataille (mais sans perte). [TBD]
\bparag une pile qui a combattu pendant une phase de mouvement et n'a
pas gagné (perdu ou ex-aequo) ne peut plus bouger ni intercepter.
\bparag malus aux interceptions : \\
	intercepter à travers passe de montagne -2 \\
	intercepter depuis ou vers un marais -1 \\
	intercepter à travers détroit : impossible \\

\paragraph{e. Interception}
Une pile ne peut tenter d'intercepter qu'une seule fois pendant
le mouvement d'une pile ennemie. Une même pile peut
cependant être victime de plusieurs tentatives d'interception
par différentes piles ennemies durant son mouvement,

Lors d'une interception, la pile interceptante peut laisser des forces
en arrière et se réorganiser après le mouvement avec une force déjà
sur place (pour respecter la limite de pions dans la région). Cette
pile du camp en réaction est l'attaquant lors du combat qui est
immédiatement résolu (avant tout autre mouvement) ; si l'interception
est dans une province contenant des forces terrestres amies de
l'intercepteur, c'est la force en mouvement qui est l'attaquant.

La force interceptante doit faire un jet d'attrition si elle entre dans
une province ennemie non occupée par des troupes amies (qui
feraient le siège). Celle interceptée teste l'attrition si elle a déjà
bougé d'au moins 6 PM.

Après le combat, si la force interceptée a gagné le combat, elle peut
poursuivre son mouvement (mais n'aura plus à tester l'attrition si
elle a déjà dû le faire).

\paragraph{Interception.}
Durant les mouvements, les flottes ennemies peuvent tenter d'intercepter dans
leur mer ou la mer adjacente la pile maritime qui bouge, sans limite du
nombre de tentative. La force en mouvement peut tenter d'intercepter
une fois (et une seule) toute pile maritime en mer en étant dans sa zone, afin de
forcer le combat tout de suite. On résout d'abord les interceptions
de forces inactives avant de tenter celles de la pile active.
En cas de réussite, on fait immédiatement les tests
d'attrition des flottes (celle active, et celle inactive si c'est elle
qui intercepte). La flotte active pourra continuer son mouvement
après interception si elle gagne la bataille. Une flotte défaite ne
peut plus intercepter pendant jusqu'à début de sa prochaine phase de
mouvement.

De toute façon, et sans test d'interception,
le combat devient possible et automatique contre une force maritime
dans la même mer après les déplacements (mais ceci compte alors comme
l'action de la force navale pour le round).

\subsection{Convoys}\label{chMilitary:Convoys}
Do not use if using the experimental system for Revolts, \corsaire and Natives.

\aparag[Convoy movements and Pirates/Privateers]
A convoy (or a naval stack carrying Gold)
entering a sea zone of \STZ or \CTZ is attacked on
\tableref{table:Pirates Natives Raids} by the pirates, and each
privateer allowed to attack the owner of the convoy present in the \STZ
(even if not in the right sea zone) if it has the right to attack the
power (see \ref{chRedep:Corsair Attack}).
\bparag Only \corsaire\faceplus may attack (be they pirates or
privateers).
\bparag The pirates attack first (one only, with leading
named Pirate if any), then the privateers in order of initiative. The
attack is resolved before regular naval interceptions.
\bparag Only one attack for all the pirates in a given \STZ or \CTZ; and
one attack per \corsaire is allowed per move.

\aparag[Attack Procedure]
\bparag Roll for naval interception. Pirates with no leader use 2 as
Manoeuvre.  If successful, reduce the \corsaire to \Facemoins and proceed with
the attack, else test for the next interception.
\bparag Before the attack, an accompanying fleet may try to disperse and
reduce the pirates or privateers by making a roll on the corresponding
table.  If successful, the \corsaire is not reduced but the attack is aborted.
\bparag If not aborted, resolve the attack on the  Pirate/Privater raid table.
\bparag Each level in the column \TradeFLEET\faceplus corresponds to one \NTD
captured (with 15\ducats).
\bparag Afterwards, \corsaire goes at port and are finished for the
turn. However Pirates stay in the \CTZ or \STZ and will attack normally Trade
Fleets .
\bparag The \terme{Barbaresque} corsairs cannot attack a Convoy if it is
not in \region{Mediterranee}.

\aparag[Flota de Oro] \label{chMilitary:FlotaDeOroMovement} As soon as the
\terme{Flota de Oro} (and only this convoy) is sunk or reaches Europe, it
reappears in a Spanish port on the Atlantic coast.

\section{Exploration}\label{chMilitary:Discoveries}
Each phasing naval stack not engaged in battle (including interception) may
attempt to explore an adjacent unknown seazone. In case of success, the stack
automatically moves into the explored zone.

Next, each phasing land stack not engaged in battle (including interception)
may attempt to explore an adjacent unknown province. In case of success, the
stack automatically moves into the explored province.

Note that a naval stack embarking troops may explore a new seazone from which
the troops may then disembark to explore a new province.

\subsection{On discoveries [58]}
\aparag As long as the forces doingthe discoveries have not came back to an
establishment existing at the beginning of the military phase, the discovery
are not yet usable by other forces.  Neither is 'rendez-vous' authorized
between stacks having made independent discoveries (no stacking).

\aparag[Diffusion of discoveries]
\bparag[On sea] At the beginning of period IV, all discovered sea zones of
Atlantique Ocean are known to everyone else at the beginning of the military
phase of the turn following the discovery. Other discovered sea zones have a
bonus of {\bf -2} for discoveries by other players.
\bparag [On land] At the beginning of period IV, all provinces containing a
\COL or \TP are known to everyone else at the beginning of the military phase
of the turn following the discovery.


\section{Battles}
Resolve all non-interception battles caused by the movement, in order of
choice of the phasing alliance (at random in case of disagreement). Each
battle must be fully resolve before the next one starts.

\subsection{Les batailles}
\aparag[Victoire majeure]
Elle est obtenue aux conditions suivantes
\bparag Sur terre en Europe : déroute du perdant et différence des
pertes égale à 3\LD ou plus
\bparag Sur terre en \ROTW : déroute du perdant, perdant avait au moins
un pion \ARMY européen et différence des pertes égale à 3\LD ou plus
\bparag Sur mer : déroute du perdant et différence des pertes
d'au moins 5\ND ou 8 DGa.

\subsubsection{Les batailles terrestres}
\aparag[Organisation des armées et cavaleries] Valable si au moins un pion armée
de la classe en question.
\bparag[\terme{Sipahi}]  TUR (avant réforme M-2) a +1 en choc et poursuite plaine/désert
\bparag[iim]  bonus +1 au choc en I-IV en plaine/forêt orientale
\bparag[tercios] toutes les autres armées ont un malus -1 en choc contre eux sauf
       i, im, ii et iim en I-V
\bparag[iiim] bonus +1 au choc en IV-V en plaine et forêts occidentales,
\bparag[SUE] bonus de +1 au choc en II-VI en forêts nordiques
\bparag[iv] bonus +1 en III-V en plaines et forêts occidentales

\aparag[Test de survie des généraux]
En Europe, sur terre, on ne teste pas le général d'un camp si son adversaire
n'a pas au moins 3 \LD.

\subsubsection{Les batailles navales}
\aparag Deux jours au maximum en cumulant les pertes jusqu'à la fuite (volontaire
ou obligatoire) d'un camp.
\bparag Fuite obligatoire
       si le moral arrive à 0, ou
       si le nombre de pertes reçues est > nbre de D de la flotte
               (tenir compte ici des modificateurs finaux aux pertes).
     Dans les deux cas c'est une déroute (avec poursuite, etc.).
\bparag Fin 2e jour si egalité en moral ;
peuvent choisir de retourner à un port (au choix) ou rester en mer (attaquant
d'abord).
\bparag Les flottes continue leur action si elles ont gagné la bataille, sinon
elles ont fini pour le tour (devant soit retourner auport, soit choisir de rester en mer
en cas d'égalité mais sans rien faire de plus).
Exception : on ne peut débuter un transport maritime après une bataille, même gagnée.

\aparag[Effet de la différence de taille des forces] Modificateurs au dé de bataille
\bparag Si la flotte est de taille >= à (taille+1) adverse, +1 au choc
\bparag Si la flotte est de taille >= à (taille+3) adverse, +1 aux feu, choc
\bparag Si la flotte est de taille >= à (taille+5) adverse, +1 aux feu, choc et poursuite
\bparag Si la flotte est de taille >= à (taille+7) adverse, +1 aux feu,  +2 choc et +1 poursuite
\bparag Si le moral perdu est > moral adverse perdu, -1 aux feu et choc

\aparag[Effet de la taille des forces]
Appliquer variation des pertes:
\bparag Si moins de 6 \ND : réduction des pertes \\
Si plus de 6 \ND (ne pas compter les \NDE):
line +1 if 7 to 12\DN; +2 if 13 to 18\DN; +3 if 19+\DN
de la table 'Size Comparison'.

\bparag Si l'adversaire a dérouté, les pertes sont minimales sont 1.
\bparag Le max de pertes que peut faire une flotte est le double de sa taille
       (1 si 'de' seul).


\aparag[Répartition des pertes]
\bparag  Integer {\bf losses split} evenly in \terme{Damaged}, \terme{Destroyed} and \terme{At port}, in units of \ND.
\bparag {Winner:} 1st \ND\ lost  \terme{Damaged}, 2nd \terme{Destroyed} and 3rd refitted (then loop over).
\bparag {Loser (or equality.):} 1st \ND\ lost  \terme{Damaged}, 2nd \terme{Destroyed} and 3rd \terme{Damaged}.
\bparag Fractions (\tu or \td) vs. \NGD or \NTD are  rounded up. Fractions vs. \ND are applied as \NDE or 2\NDE  in the next category of loss.
\bparag Examples 2: 3\td losses against losing \ND: 2\ND+2\NDE {Damaged}, 1 \terme{Destroyed}\bparag Examples 2: 4\tu losses against winning \NGD = 5 losses: 2  \terme{Damaged}, 2 \terme{Destroyed} (one immediately refitted fo no effect).


\aparag[Pertes en poursuite] En plus des pertes normales, elles permettent
de capturer ou attaquer les transports. Le niveau de capture est égal
au nombre d'étoiles.
\bparag Capture de navires de guerre = le gagnant peut capturer un DN ou
2 DGa par * en poursuite        (pris d'abord sur les Imm, puis les End, puis les autres)
\bparag Couler Transports = 2 DTr coulé par * de poursuite dédiée à ceci
       une force terrestre transportée au minimum égale à ce que
       ces DTr perdus transportent doivent être détruits.
\bparag Capturer Or = 2 DTr par * de poursuite dédiée à ceci avec 5 ors perdus, 10 \ducats capturés.
Les transports sont gardés avec la flotte (et peuvent être repris, attaqué par pirates etc)
jusqu'à un port de la métropole où ils disparaissent (et or dans le RT).

\aparag[Damaged ships] \terme{Damaged} \ND are written down globally by naval zones:
Mediterranean Sea, Atlantic in Europe, Atlantic in \ROTW, Indian, Asian and East Pacific.
They are refitted for usage:
\bparag cost = 0.5*coût achat DN à un round suivant pour les remettre en état.
               Effet = remet tout de suite en jeu les DN voulus.
\bparag gratuit au début du tour suivant si on entretient la flotte;
\bparag on peut la garder \terme{Damaged} pour un coût d'entretien divisé par 2 ;
\bparag On les remet en priorité dans un Arsenal de la zone, sinon dans un
port capable de les contenir.

\aparag[Convoy in battles] If a battle occurs between two naval forces,
one of them containing a convoy or Transports, the convoy does not take
part in the battle, nor incurs losses during it.
\bparag However, at the end of the battle, the pursuit \textetoilex\
result may apply to the convoy or the Transports if the winner decides
so.
\bparag Each \textetoilex\ captures 2 \NTD with corresponding transports
points sunk if loaded with troops, or 10 \ducats captured and 5 \ducats
sunk for \NTD loaded of Gold.
\bparag The rest of the convoy is kept by the loser.

\subsubsection{Séquence de la bataille}
-- NE PAS PRENDRE EN COMPTE --

\noindent
La bataille commence après les tentatives éventuelles d'interception, de
retraite avant combat, et les jets d'attrition des forces qui ont
bougé. \\
{\bf A.} Les rounds de combat, simultanés.
Si à la fin d'un des 4 rounds, une armée a craqué au moral (aussi appelé déroute,
c'est-à-dire est arrivé à 0 ou moins au moral), \textit{passer directement en C.1} \\
1. Premier round de feu~: chaque camp lance un dé modifié sous la colonne de feu. On retient les
pertes faites par les deux camps. \\
2. Premier round de choc~: chaque camp lance un dé modifié sous la colonne de choc. On ajoute les
résultat aux pertes faites par chaque camp. \\
{\bf B.} Possibilité de rompre le combat, défenseur puis attaquant. Si les
pertes sont à ce moment suffisantes pour que, une fois modifiées par les pertes variables,
un camp soit éliminé, le combat cesse et on passe au C. \\
3. Second round de feu~: chaque camp lance un dé modifié sous la colonne de feu, avec -1 au dé.
On ajoute le résultat aux pertes faites par chaque camp. \\
4. Second round de choc~: chaque camp lance un dé modifié sous la colonne de choc, avec -1 au dé.
On ajoute le résultat aux pertes faites par chaque camp. \\
{\bf C.1} Si une armée a craqué au moral et pas l'autre, effectuer un jet de poursuite (colonne E). \\
{\bf C.2} Si une armée a moins de moral restant que l'autre mais n'est pas en déroute, elle perd le combat.
\textit{Le vainqueur fait une poursuite qui peut causer une déroute}. \\
{\bf C.3} Si les deux armées ont le même moral final ou que les deux ont craqué au moral,
chaque camp retourne d'où il vient, on ne fait pas de poursuite et personne ne gagne \\
{\bf D.} On totalise les pertes de chaque camp (des 4 rounds et la poursuite) qui sont modifiées
en fonction de la taille de l'armée causant les pertes,
\textit{ensuite de sa classe comparée à celle de
l'armée prenant les pertes}. \\
{\bf E.} Le perdant du combat (qui doit retraiter) fait un test d'attrition au cours de la retraite
qui peut accroître ses pertes de $1/2$ ou 1. La man{\oe}uvre du général est utilisée
si la force n'est pas en déroute. \\
{\bf F.} Les pertes sont arrondies à l'entier inférieur (sauf 1/2 qui devient 1).\\
{\bf G.} Les tests de perte des généraux sont faits (règle usuelle). On regarde si
il y a eu bataille majeure. \\

\subsubsection{Pendant les rounds de combat}
-- NE PAS PRENDRE EN COMPTE --


\paragraph{1. Technologie et feu.}
- Une armée en Médiéval ne lance pas de jet de feu. \\
%- \textbf{version de Bertrand :} Une force en Renaissance utilise toujours la table de feu, même
%si il n'y a que des DT (selon la clarification des Q\&A). Elle ne fait que les pertes au moral \\
%- \textbf{version de Pierre :}  Une armée en Renaissance ne fait
%que les pertes de moral indiqués par le feu ; si il n'y a que des D, pas de test de feu
%(selon la règle de combat rapide de P. Thibaut).  \\
- En Renaissance, une force utilise le table de feu si elle contient des pions armées, ou bien
si elle n'a que des DT, quand elle combat des indigènes ou pays non européens en Médiéval.
Dans les deux cas elle ne fait que les pertes au moral. \\
- En Arquebuse, les pertes obtenues sur la table doivent être divisées par deux
(arrondies à l'inférieur). \\
- Pour toutes les technologies après Arquebuse,  les pertes sont celles indiquées.

\paragraph{2. Modificateurs.}
Ils sont indiqués à côté de la table de combat (terrain; -1 au second round;
effet des généraux).


\paragraph{3. Avantage de cavalerie.}
Chaque pile contient un nombre de cavalerie qui dépend de sa taille et
de la quantité de cavalerie par équivalent détachement. Une valeur \textit{qui dépend
de la nationalité de l'armée en question (voir ci-dessous)} est multipliée par
le nombre d'équivalent de détachement et donne ainsi la quantité de cavalerie
dans la force. Les DT contiennent autant de cavalerie  par détachement que les A.

Si un camp a au moins deux fois plus de cavalerie que son adversaire,
il a +1 au dé pour le choc et la poursuite si la bataille est en plaine
(qu'il soit défenseur ou attaquant), désert, ou dans les forêts orientales (voir
\ref{foretOrientale}, pour certaines technologies seulement).


\paragraph{4. Poursuite.}
Les jets de poursuite sont affectés par le différentiel de choc,
le terrain, l'avantage de cavalerie et la condition à la fin du combat : \\
+1 si l'adversaire a craqué au feu, \\
+2 si l'adversaire a craqué à un des 2 premiers rounds (cumulable avec le précédent).

\paragraph{5. Rompre le combat.}\label{RetraiteEnBataille}
Lors du segment B. de la bataille, entre les 2 premiers rounds de feu puis choc et les
deux derniers, une armée peut décider de rompre le combat.
Le défenseur a la possibilité de le faire et, si il décline ou échoue, l'attaquant
peut le tenter.

Un jet de dé inférieur à la man{\oe}uvre du général plus le moral restant à l'armée
permet de finir la bataille tout de suite ;
celui qui rompt le combat est le perdant. On finit le combat par le segment  C.2
et les suivants. Si le test est échoué, le combat continuera et l'adversaire a un
bonus de +2 à son jet de feu subséquent.


\subsubsection{Variation des pertes}
-- NE PAS PRENDRE EN COMPTE --

\label{PertesVariables}
Le résultat des pertes est le total de ce qui est fait aux différents tests de feu,
choc et éventuellement poursuite (mais sans la retraite) donnant un nombre d'équivalent
détachement encaissé par
l'armée adversaire. Cependant la table est prévue pour donner le nombre de
pertes faites par une pile de 2 A+ à une armée de même taille. Notez
qu'avoir plus de 8 DT dans une pile (pour la Turquie avec les pachas) ne
donne aucun avantage : pas de pertes supplémentaires (à la différence
du traitement des indigènes, voir \textit{infra}).

{\bf 1.} Pour tenir compte de la taille réelle de l'armée, on consulte la table des
pertes variables (voir tables de combat) qui indique combien de perte enlever pour obtenir le nombre
de pertes final. \textbf{On applique une limitation importante à ce stade :
le total des
perte ne peut être supérieur à la taille de l'armée causant les pertes, comptée
en équivalent détachement}.

{\bf 2.}
\textit{On compare ensuite le type de chaque armée.} Il y a 5 groupe d'armée qui sont
les suivants, leur taille étant indiquée pour les sept périodes (la répartition précise est
indiquée dans les annexes pour les mineurs et sur les tableaux des majeurs) :


Le tableau suivant (à droite) permet alors de déterminer le différentiel selon la taille de
chaque armée. On a mis en caractère gras les lignes et colonnes qui servent
usuellement, en caractères normaux celles où des armées
de tailles différentes sont mélangées (colonne 1, 5 et 6).
%Le différentiel est variable selon les périodes ; après ce chiffre
%sont indiquées les périodes de validité du modificateur.
L'armée qui subit les
dommages est prise en ordonnée sur une colonne, celle qui les inflige sur une
ligne ; le tableau est
symétrique avec un changement de signe par rapport à la diagonale.


\textit{Algorithme : diviser la différence de taille entre l'armée la plus grande et la plus
petite par 3 et arrondir au modificateur le plus proche pour obtenir le +?
accordé à l'armée de taille plus grande.}


{\bf 3.}
Les pertes véritablement infligées sont alors celles données par le tableau
ci-dessus, la ligne 0 correspondant au nombre de perte calculé à l'étape A,
avant le modificateur
dû à la comparaison des classes d'armée.


{\bf 4.}
On ajoute à la valeur obtenue le nombre de pertes données par la table de
retraite (qui n'est pas modifié donc par le point {\bf 3}).
Les pertes obtenue sont arrondies à l'unité inférieure (sauf \f\ qui donne 1)
et donnent la valeur en équivalent détachement du nombre de pertes effectuées.

{\bf 5.}
Si une force indigène compte au final plus de 8 équivalent DT, elle inflige une fois
des pertes par fraction de 8 DT complète et une de plus pour la fraction restante
en lançant plusieurs dés sur la table de combat (un par 8 DT ou fraction).
Cette dernière fraction fera des dégats diminués par la table des pertes variables.



\subsubsection{Qui gagne le combat}
-- NE PAS PRENDRE EN COMPTE --


Les différentes issues du combat sont données dans la séquence des batailles et
détaillées ici.
%\newpage\null%\newpage

\textbf{Le vainqueur de la bataille.}

{\bf C.1} Si une armée seule armée craque au moral (arrive à 0 ou moins) et pas l'autre à la fin
d'un round, l'adversaire gagne le combat. Il effectue une poursuite, on ajuste les
pertes. Le perdant recule dans une zone amie adjacente et fait un test d'attrition
sans soustraire la man{\oe}uvre du général. Les PVs normaux sont accordés.

{\bf C.2} Si aucune armée n'a craqué au moral après les 4 rounds, l'armée qui a le moins de moral restant perd
le combat. \textit{Le vainqueur fait une poursuite.} Les pertes dues à la poursuite
peuvent entraîner une déroute du perdant, en quel cas la fin de la procédure est la même
que C.1. Autrement, le perdant recule dans une zone amie adjacente et fait un test d'attrition
modifié par la man{\oe}uvre du général. Des PVs réduits de moitié sont accordés.

{\bf C.3} Si les deux armées ont le même moral final, ou si les deux ont dérouté,
chaque camp retourne d'où il vient. C'est-à-dire qu'un siège continue à être maintenu, qu'une
armée qui vient de se déplacer ou d'intercepter retourne dans la zone où elle était juste
avant le combat. Il n'y a pas de poursuite; IL Y A ATTRITION de retraite et personne ne gagne
ni ne marque de PV.\\

Une \textbf{victoire majeure} est accordée si le perdant a effectivement perdu
\textbf{3} détachements
de plus que le vainqueur (après modifications de la classe, retraite et arrondis),
ou \textbf{4} DT si le perdant avait un modificateur de comparaison de taille
égal à -2.

Les pertes sont réparties par celui qui les subit comme il le veut parmi ses forces.
Il peut détruire des pions armées (une A+ = 4D) par exemple ou faire tout son
possible pour en garder (par exemple 2 A+ subissant 4 pertes peuvent rester sous
forme de 1 A+ ou 2 A- ou encore 4D -- ce qui poserait des problèmes d'empilement...)

\subsubsection{Bataille navales rapides}
-- NE PAS PRENDRE EN COMPTE --

\paragraph{a. Séquence de bataille.}
Les batailles navales sont résolues sur un système de combat accéléré
qui utilise la table des résultats des batailles rapides. Le combat peut
se poursuivre sur plusieurs jours selon la séquence suivante.
\begin{enumerate}
\item Décider du type de navires (Nav, Ga ou Tr) mis en avant (et en déduire le moral,
les colonnes utilisées, et les modificateurs pour l'avantage du vent) ; ils
doivent constituer au moins 1/4 des détachements de la pile.
\item Déterminer l'avantage du vent (sauf dans un combat de galères
contre galères).
\item Segment de feu ; noter les pertes. Elles sont réduites de moitié
si des Ga sont en 1e ligne. Si un camp craque au moral,
aller directement en 7.
\item Retraite optionnel du camp sous le vent, sans poursuite (mais avec
suivi et attrition).
\item Segment d'abordage ; noter les pertes.
\item Ajouter les pertes de 3 et 5 et les ajuster en fonction du nombre de DN
présents. Les retirer des forces navales.
\item Si un camp est à 0 en moral, il rompt le combat et se réfugie au port
(un des ports amis les plus proches), avec poursuite si l'autre camp n'a
pas craqué au moral, et suivi éventuel pour établir un blocus.
\item Si personne n'a craqué au moral, les deux joueurs choisissent en secret
de rester ou de se replier au port. Si seule une force se replie, l'autre peut suivre
au port pour mettre le blocus. Si les deux forces restent, reprendre une journée
de combat au segment 2 en utilisant les valeurs de moral restant
(contrairement aux batailles terrestres, le combat n'est pas limité à deux
journées, le modificateur -1 s'appliquant à partir du 2\up{e} jour).
Si jamais les DN du type de navires mis en avant sont tous détruits, il
faut choisir un nouveau type et le moral est le minimum entre celui après la journée
de combat et celui du nouveau type de navire en 1e ligne.
\end{enumerate}

\paragraph{b. Effet des pertes.}
Les pertes obtenues en 3 et 5 sont ajoutées et modifiées alors
par un pourcentage dépendant du nombre de DN dans la flotte
qui fait les pertes (les DE comptent comme une moitié de DN mais
l'arrondi est fait vers le bas). Voir le \textbf{tableau des modificateurs des
taille de flottes}. On arrondi les fractions du résultat au demi supérieur,
sauf si le résultat est $\le 0,3$ qui est réduit à 0.

Ensuite ces pertes sont réparties entre des détachements coulés,
immobilisés (reviennent au tour suivant dans un port ami à
décider immédiatement) et endommagés (reviennent au round suivant dans cette
flotte) en consultant la \textbf{table de répartition des pertes navales}.
Si les pertes sont plus que 3 D, utiliser plusieurs fois la table pour chaque
tranche de 3 D et le reste.
Les DE se combinent ici selon l'équivalence 1DN=3DE.

Les pertes affectent d'abord les détachements des navires mis en avant,
ensuite sur les autres navires.


\paragraph{c. Poursuite ou suivi au port.} La poursuite est un jet de perte
sur la colonne E. Les pertes sont modifiées par la taille de la force navale
qui poursuit (après les pertes du combat) et se répartissent pour moitié (arrondie
à l'inférieur) comme des vaisseaux capturés et endommagés (reviennent rou
round suivant dans un port ami au choix), et le restant comme des vaisseaux
coulés.

Ensuite une force navale est parfois autorisée à suivre son adversaire qui
retraite pour mettre le blocus devant le port atteint. Elle peut ne suivre qu'avec
une partie de la flotte (qui cesse son mouvement en blocus, après attrition
pour le déplacement) tandis que le restant de la flotte continue son
mouvement si le combat résultait d'une interception, ou fera une autre
action navale dans la mer ensuite (blocus, lutte contre des pirates, etc).

\paragraph{d. Galères et galéasses.} Les DGa et DGal comptent en empilement
dans les pions F comme un demi-DN. Ainsi les flottes de galères sont en général
plus importantes. Cependant elle font des pertes réduites de moitié à l'étape 3
de feu (arrondir les fractions 1/4 vers le bas), ceci avant de les modifier par la taille.

Venise peut avoir jusqu'à deux détachements de galéasses. La présence
d'un détachement de galéasses en combat contre des galères (uniquement) permet d'utiliser
les pertes complètes au feu ; la présence de 2 DGal fait que le feu se fait
en plus avec +1 au dé de feu.
Contre des vaisseaux, les galéasses combattent comme des galères.

\paragraph{e. Transports en combat.}
Les DTr ne comptent pas du tout en combat maritime et prennent des pertes si les
navires de guerre qui les accompagnent sont déjà perdus.
Des DTr mis en avant au début d'un jour de combat rompent toujours au moral
après le feu quel que soit le résultat adverse et ne font pas de pertes (même si des
navires de guerre sont en retrait).

\section{Sieges}
Resolve all sieges, fights against \REVOLT/\REBELLION and \corsaire.

Each alliance, in decreasing order of initiative, resolves all of its actions
in an order of its choice (at random in case of disagreement).

\aparag[Blocus]
Il faut avoir au moins la force navale voulue selon le niveau de forteresse
(voir table ou supra).
\bparag Coupe le bonus de -3 au test de ravitaillement des assiégés ;
\bparag Flotte qui veut sortir ou entrer : doit faire un test pour échapper au blocus
(ou attaquer la flotte en blocus)

\subsection{Les sièges}
\aparag Pour la sape, effet du terrain (non cumulatif)
\bparag           -2      Port sans blocus, terrain clair
\bparag           -3      Port sans blocus, terrain autre que clair
\bparag -2 Terrain accidenté (montagne, marais, forêt, désert) sans port
ou blocus

\aparag[TBD] si un assaut a causé au moins 1 perte (sans modif de
taille ni bonus ``grosse armée'' dans le tour : +1 à la sape et à
l'assaut (max +1, non cumulatif avec le +2 de brèche).


\aparag Les tables sont à jour !

\aparag[Expérimental]
Un assaut qui a obtenu au moins 1 pertes (sans compter les bonus
de Janissaires, \RUS, \POL) sans prendre la forteresse donnera un
bonus de {\bf +1} aux jets de sappe et aux assauts suivants du tour.

\aparag[Port Siegeworks]
Ports that are besieged with at least one level of Siegework are
submitted to a fire from the siegework that works the same way as the
Presidios, with a {\bf +1} per counter Siegework\faceplus.  {\bf But the
 port is not blockaded.}

\aparag[Impossibilité de tenir un siège]
Ceci est regardé au début de la phase de siège (nbre de DT >= niveau) ;
si impossible, mvt de rédéploiement forcé vers chez soi \\
- en fin de tour: si pas Usure\faceplus, redéploiement forcé.


\subsection{Les sièges}
-- NE PAS PRENDRE EN COMPTE --

%Les règles ne sont quasiment pas changées par rapport au combat rapide de la 2nde extension.

\subsubsection{L'assaut}
-- NE PAS PRENDRE EN COMPTE --


\paragraph{Les rounds d'assaut.}
L'assaut se fait en deux jets, un de feu puis un de choc sauf que
le choc n'est pas fait par un camp qui a craqué au moral .
Les tables de combat montrent une colonne spécifique à l'assiégé et une pour l'assiégeant.
Noter que l'assiégé fait une perte en moins au feu et au choc si le combat est
suite à une \textbf{brèche}.


\paragraph{Modificateurs.}
L'assiégeant ajoute 1 si le défenseur est médiéval, soustrait 1 si le défenseur est en arquebuse
ou mieux, à son feu et son choc.
L'assiégeant soustrait aussi le niveau de la forteresse aux deux si il n'y a pas eu de
brèche. Enfin, l'artillerie ajoute  +1 en assaut si l'assiégeant a au moins
4 fois le niveau de la forteresse en artillerie (sauf contre un fort).

\paragraph{Les ajustements aux pertes.}
\begin{itemize}
\item 1- si l'assiégeant n'a pas 2 A+, le tableau des pertes variables réduit
ce qu'il inflige ;
\item 2- la Turquie et la Russie jusqu'en 1614, et la Pologne jusqu'en 1559 augmentent
les pertes faites en assaut de 1/2 par A+ présente ;
\item 3- l'assiégeant prend une demie-perte en plus si il a craqué au moral.
\item 4- les pertes de l'assiégeant sont limitées au nombre de DT dans la
fortification plus 2 fois la résistance de
la forteresse (ajustée par la brèche).
\end{itemize}

\paragraph{Résistance de la forteresse.}
Les pertes faites à l'assiégé sont d'abord prises sur les unités enfermées dans
la forteresse, puis sur la résistance de celle-ci. Cette résistance est égale à
son niveau, mais est réduite en cas de brèche. Elle revient à son niveau
maximum après chaque assaut.

\paragraph{La victoire.} Elle revient au camp selon l'ordre de priorité suivant~:
\begin{enumerate}
\item Assiégé, si l'assiégeant est éliminé ;
\item Assiégeant, si les troupes à l'intérieur sont éliminées et la résistance atteint 0,
ou si l'assiégé craque au moral
(même si l'assiégeant déroute) ;
\item Assiégé, si l'assiégeant seul ou si personne ne craque au moral.
\end{enumerate}


\subsubsection{La sape}
-- NE PAS PRENDRE EN COMPTE --

\paragraph{Mettre le siège.}
%\begin{minipage}[b]{0.5\linewidth}
Le siège par usure n'est presque pas modifié. Un pion armée  contient toujours
un nombre d'artillerie égal à celui de la contenance maximum de sa nation à la période en cours.
Une armée sur la face - contient l'artillerie de l'armée + divisée par 2 et arrondie
à l'inférieur.

Il faut pour maintenir le siège devant une forteresse disposer d'au moins autant
d'équivalent détachement que le niveau de la forteresse.
Si l'assiégeant ne peut maintenir le siège en fin de round (après un assaut ou une
bataille), il doit immédiatement retraiter dans une province amie (avant de pouvoir
piller) et jouer l'attrition.
Si il choisit de maintenir le siège, il doit soit lancer un assaut, soit faire
un test dur la table de sape (qui peut être suivi d'un assaut en cas de brèche).

Le propriétaire de la forteresse peut laisser des troupes dans celle-ci.
L'empilement dans une forteresse est d'au plus 2DT par niveau de la
forteresse, ou d'un DT dans les forts. Ces forces subissent une attrition
à chaque fin de phase de mouvement si le siège est déjà
établi. Une fois enfermés dans une forteresse, une force ne peut
en sortir qu'en fin de siège (victorieux ou non) et n'a pas le
droit s'attaquer les assiégeants.

\paragraph{Résolution de la sape.} On utilise la table des annexes,
avec les modificateurs indiqués.

Les pertes assiégeantes obtenues sur la table des sièges se résolvent
en lançant 1d10, diminué des valeurs en siège des généraux
et augmenté de 1 par DT (ou équivalent) en défense dans la forteresse.
Si le résultat est inférieur (strictement) au nombre de round
de siège écoulé, l'assiégeant doit faire un test d'attrition sur la table
adéquante (Europe ou non) avec les modificateurs indiqués.

\subsubsection{Prise des forteresse}
Une forteresse qui tombe par assaut ou sape perd 2 niveaux de
fortification (avec la valeur mise sur la carte en tant que minimum),
sauf si le nouvel occupant décide immédiatement de mettre un
garnison. Il doit pour cela utiliser un DT qui est perdu (le DT peut
provenir de la séparation d'un pion armée).



\section{New round}
A new round begin with the \terme{Continuation Roll} segment.

\section{Military cleanup}
???

Normally nothing to do here.

% Local Variables:
% fill-column: 78
% coding: utf-8-unix
% mode-require-final-newline: t
% mode: flyspell
% ispell-local-dictionary: "british"
% End:
