% -*- mode: LaTeX; -*-

\definechapterbackground{Military Rules}{military}
\chapter{Military Rules}\label{chapter:MilitaryRules}

\begin{designnote}
  This Chapter focuses on the description of the Military phase in Segment
  order. The main concepts and common rules used during it are described in
  the next Chapter. Most of these concepts are common with many other wargames
  and do not need to be perfectly defined in order to understand he flow of
  the Military phase, which is why precise description is postponed. You
  should refer to the next Chapter whenever a point in these rules requires
  clarification.
\end{designnote}

\begin{todo}
  This Chapter is under heavy work. The random presence of detailed numbering
  of rules reflects this.
\end{todo}

\section{Overview}
\aparag[Sequence]
\MilitaryDetailsNew

\subsection{Military setup}
Setup for the military phase.

\subsection{Rounds}
The military phase is split in a certain number of \terme{rounds} (between 3
and 11). During each round, each alliance, in decreasing order of initiative,
has an \terme{impulse} where it can moves its troop and fight. Some matters
are resolved before any alliance has its impulse (continuation, wintering) and
some after each had its one (siege, piracy, revolt). After that, a new round
begins.
\subsubsection{Continuation roll}
Determine the new round to be played.

\subsubsection{Wintering}
Stacks may suffer from attrition, leaders may be redeployed.

\subsubsection{Impulses}
Each alliance, in decreasing order of initiative, plays an impulse. Each
impulse consists in five steps: supply, choice of campaign, movements (and
interceptions), exploration, and battles.

The alliance taking its impulse is called the \terme{phasing} alliance. All
other are non-phasing.

\subsubsection{Sieges}
Resolve all sieges, fights against \REVOLT/\REBELLION and \corsaire.

\subsection{New round}
A new round begin with the \terme{Continuation Roll} segment.

\subsection{Military cleanup}
Compute the final cost of all campaigns paid this turn.

\begin{playtip}[Simultaneity]
  Most of the time, there are separate wars that cannot affect each other
  (typically, with different antagonists in each), or even separate actions
  for a given alliance (typically action in the \ROTW and in Europe). In this
  case, the resolution of the impulses do not need to be as strictly
  sequential as explained in the rules. The military phase is long enough and,
  typically, a FRA-HIS war and a RUS-POL war can be played simultaneously in
  order to make things a little faster.

  It is normally recommended to synchronise all players for the continuation
  roll (because it is a point where some crucial new information is
  gained). Sometimes, it is however possible for two players to quickly
  completely play a small war (noting the results of the continuation roll in
  secret to communicate it later) while other players are still struggling in
  the first round of a big war. Sometimes even while other are still planning
  and resolving their administration\ldots This typically allows to ``free''
  those players to go and buy food for everybody\ldots

  \smallskip

  Similarly, the rules describe a strict order for the resolution of actions
  in the military phase but in practice it doesn't often need to be respected
  that strictly. Typically, two battles can be resolved simultaneously if they
  don't have the same participants, or sieges can be resolved in any order
  (rather than on a per alliance basis) if players agree that this has no
  importance. The rules, however, must provide a strict order to be able to
  solve any disagreement in the order of resolution of actions.
\end{playtip}

\section{Military setup}
\subsection{Initiative}
Determine the alliances in effect this turn and the order of initiative
between them. Alliances are transitive (the ally of my ally is my ally) for
this purpose (but not for stacking purpose and such). Each alliance plays at
the lowest initiative of one of its member, resolve any tie at random.

Stacks in interventions (whether limited or foreign) act at the same
initiative than the alliance for which they intervene.

Minors at war alone act at the initiative of the country controlling them.

\subsection{Starting round}
Roll one die to determine the starting round (read the result in the boxes of
the rounds display). This die roll is never modified. The weather for this
round is determined as usual (see below).

It is possible for the Sund to be frozen during the first round if the die was
'1' and \ref{eco:Poor weather} happened this turn.

\section{Rounds}
During each round, perform the segments detailed below. The Impulses segment
is repeated for each alliance in decreasing order of initiative.

Each round is labelled by a letter indicating the \terme{season} ('S'ummer or
'W'inter) and a number (between '0' and '5') indicating the \terme{year}.

\begin{designnote}
  Do not take the seasons and years too seriously when interpreting what
  happens during a turn. This is more a modelisation artefact that gives good
  macro results than a real attempt to simulate military actions on a lower
  scale (especially, the 'S' and 'W' rounds happen simultaneously in the North
  and South hemispheres\ldots)
\end{designnote}

\subsection{Continuation roll}
Do not perform this segment at the first round of a turn.

Roll a die, modified by \bonus{+2} in case of \ref{eco:Poor weather} and
\bonus{-1} if this is period \period{VI} or \period{VII}. Follow the arrows on
the rounds display to determine the new round.

If the new round is the 'End' box, then the rounds stop immediately. Proceed
with \terme{Wintering} then \terme{Military cleanup}.

If the followed arrow is red (modified roll of 1-5 after a Summer round), then
the new round is played with an extended campaign. See choice of campaign for
the implications.

If there is an extended campaign after a (unmodified) continuation roll of
'1', '3' or '5', then the round is played with \terme{Bad weather}.

If \ref{eco:Poor weather} happened this turn, then each Winter round is played
with \terme{Bad weather}, no matter what was rolled.

\begin{playtip}
  End of the Military phase may happen somewhat brutally. You should always
  check the probabilities before planning long term actions (sieges) in years
  4 or 5.

  Given the shape of the rounds track, at least one round in each column must
  happen. Thus, the minimum number of rounds left to play is the number of
  columns and the maximum is twice that number. Moreover, long Military phase
  implies lot of extended campaigns. Take that into account when planning both
  long term actions (sieges) and expenses for the rest of the phase.
\end{playtip}

\subsection{Wintering}
If the current year is different than the year of the previous round, then a
\terme{Wintering} segment occurs. Otherwise, skip to the \terme{Impulses}
segment.

There is a \terme{Wintering} segment when the 'End' box is reached, that is,
it is considered as being 'S6'.

\subsubsection{Cold area}
Any stack in a non-controlled, non-national province within the \terme{Cold
  area} rolls for attrition. Resolve this as a Supply attrition (see
below). It is, however, a different roll and a stack may have to roll for
attrition both during the Wintering and Supply segment in some cases.

\subsubsection{Pashas}
Any stack containing \Timar out of \TUR national territory rolls for
attrition. Check specific Turkish rules for details.

\subsubsection{Hierarchy}
Leaders may be redeployed.

Each country may choose to redeploy its leaders on its stacks any way it wants
(no maximal distance or such, in other words it's a free teleportation). If it
chooses to do so, then the hierarchy must be globally respected after this
redeployment. It is always possible to choose not to redeploy leaders at this
point, but as long as at least one leader is redeployed, the country must
respect hierarchy globally.

Exception: besieged leaders as well as leaders on stacks with discoveries that
have not been brought back to Europe must stay in place and are not checked
toward global hierarchy.

\subsection{Impulses}
Each alliance, in decreasing order of initiative, plays an impulse. Each
impulse consists in five steps: supply, choice of campaign, movements (and
interceptions), exploration, and battles.

The alliance taking its impulse is called the \terme{phasing} alliance. All
others are non-phasing.


\section{Supply}
Each phasing land stack which has no supply for two consecutive rounds is
immediately destroyed. Getting back supply temporarily during the round is
enough to avoid this destruction.

Each phasing land stack which has no supply or is in weak supply must roll for
\terme{supply attrition}. Additionally, each phasing besieged land stack must
roll for \terme{siege attrition}. Naval stacks never roll for attrition during
the supply segment. Beware that besieged stacks roll for \textbf{siege}
attrition during this segment, which has a similar procedure as supply
attrition but with slightly different modifiers.

If a besieged stack is also in weak supply, it does not roll for supply
attrition. The siege attrition replaces this roll.

\subsection{\SoS, \LoS, weak supply}
A Source of Supply (\SoS) is either a controlled city, \TP, \COL or fort, or a
large enough naval stack in an adjacent seazone. A Source of Supply may supply
as many stacks as wanted.

The Line of Supply (\LoS) goes from a Source of Supply to the stack. The \MP
cost is counted as if the stack itself was doing this movement (typically, \LD
in the \ROTW compute the length of their LoS using the special \MP costs for
\LD).

A stack is in weak supply if at least one of the following condition is true:
\begin{itemize}
\item its Line of Supply is as least 6\MP long (exception: in the \ROTW, \LD
  do not check this condition);
\item its Line off Supply goes through desert;
\item its Source of Supply is not owned by the same alliance;
\item it is supplied by a naval stack that is not adjacent to a port or
  arsenal able to supply it.
\end{itemize}

It is possible to take its supply from a further \SoS to avoid weak supply
(typically, to have an owned \SoS, or to circumvent a desert).

\subsection{Attrition roll}
Attrition (in Europe) is obtained by rolling one die, modified as follows, and
cross-reference the modified result in table~\ref{table:Discoveries and
  Attrition} in the ``Loss in Europe'' column that corresponds to the size of
the stack. In the \ROTW, the result is read in the ``\ROTW Losses''
column. Note that a result of 11 or less has no effect.

\begin{todo}
  The Attrition table does not want to be included here. I suspect too much
  dependence with adjacent table in the TikZ code\ldots
\end{todo}

%\GTtable{discoveriesattrition}

Modifiers for Supply attrition:
\begin{modlist}
\item[+2] per cause of attrition above the first
\item[+2] in case of \terme{massed forces} (the stack contains more than 5\LD
  and 1\Pasha)
\item[+2] if the stack has no \LoS
\item[-M] MAN of the commander of the stack
\item[+?] if the stack is supplied by a naval stack, and the \LoS of this
  naval stack goes through one or more \StraitFort, add the DRM of all the
  \StraitFort along this path (2 in Europe, level/2 in the \ROTW)
\item[+8] if the fortress of the province is controlled by the enemy
\item[+6] if the fortress of the province is controlled by allies or if there
  is no fortress in the province (in the \ROTW)
\item[+1] per \PILLAGE\facemoins, \REVOLT\facemoins or unfriendly
  \REBELLION\facemoins in the province
\item[+2] per \PILLAGE\faceplus, \REVOLT\faceplus or unfriendly
  \REBELLION\faceplus in the province
\item[+?] in the \ROTW cold area, add the number of Snowflakes ``resource''
  (+0 to +2 depending on the \Area)
\end{modlist}

Modifiers for Siege attrition (besieged):
\begin{modlist}
\item[+6] always (friendly fortress)
\item[+?] in the \ROTW cold area, add the number of Snowflakes ``resource''
  (+0 to +2 depending on the \Area)
\item[-S] siege value of one besieged leader
\item[+S] siege value of one besieger or blockading leader
\item[-3] if besieged in an unblocked port
\item[+1] per \USURE\facemoins
\item[+3] per \USURE\faceplus
\end{modlist}

\subsection{Result of attrition}
In Europe, the result of attrition is either nothing (---), a number
(between 1 and 3), a 'P' or both a number and a 'P'.

The number indicates how many \LD are lost immediately. The controller of the
stack chooses which.

A 'P' is interpreted differently according to the technology of the stack.
\begin{itemize}
\item If the technology is \TMED or \TARQ, then 1 more \LD is lost and a
  \PILLAGE\facemoins is placed into the province (and immediately merged into
  a \PILLAGE\faceplus if there was already another one here).
\item If the technology is \TMUS or better, then the controller chooses to
  either loss one more \LD or place a \PILLAGE\facemoins.
\end{itemize}

Besieged troops cannot place \PILLAGE and thus must loss one \LD in case of
'P'. There is no extra effect for the lost \PILLAGE with low
technology. Similarly, if there are already two \PILLAGE\faceplus in the
province, it is not possible to add more and any 'P' must be resolved by
loosing one \LD.

Note that since \PILLAGE will make further attritions harder, it is sometimes
wiser to loss \LD rather than let the troops plunders. Note also that since
\PILLAGE\facemoins will be removed at the end of the turn, small troops with a
not too bad technology don't suffer a lot from attrition.

\smallskip

In the \ROTW, cross reference the percentage obtained with the size of the
stack in table~\ref{table:Attrition ROTW Remainders}. The resulting number
indicates the number of \LD still alive after attrition while the 'd'
indicates one or two \LDE still alive. If there is a \textetoile, then there
is 50\% chance to lose an extra \LDE.

\GTtable{attritionlosses}

\section{Choice of campaign}
\begin{todo}
  The following is copy of older stuff and de facto cease to be relevant
  before I reread and rewrite it.
\end{todo}

Each country of the phasing alliance chooses its campaign for the current
round. Make a tick in the correct line of the ERS to count how many campaigns
of each type you've done so you can pay for them.

More expensive campaigns allow for more actions. In case of extended campaign,
there is a single campaign that spans over two rounds and that can be upgraded
at this point.

Each country must pay for campaign in order to move its troops, however
multinational stacks are moved as part of the campaign of the commander of the
stack.

\aparag[Campaigns for \MIN]
\bparag limited intervention: 1 simple campaign each turn. 1 passive campaign
each round. More can be paid by the \MAJ (pay only the difference in cost of
campaign, not the full campaign).
\bparag full intervention: 1 simple campaign each round. Multiple campaigns
may be obtaind by reinforcement. More can be paid by the \MAJ (as above).


\subsection{Military campaigns}
\aparag Interception is allowed according to the last campaign paid.
\bparag For player without initiative, this is the campaign of the previous
round.
\bparag During first round, players without initiative may intercept (before
their first move) as if they had done a passive campaign.

\aparag When moving both at sea and on land, the cost of both campaigns is
computed separately and only the maximum cost is paid.
\begin{exemple}
  A Major campaign allows to both:
  \begin{itemize}
  \item attack with one naval stack of 3\FLEET ;
  \item move without attacking (exploration possible) with as many naval
    stacks as wanted (non-aggressive movement is not restricted) ;
  \item maintain as many blocus and fight against \corsaire as wanted (only
    movement is restricted) ;
  \item attack with as many small ($\leq$ 5\LD) land stacks as wanted (the
    reason for which the campaign is Major needs not to be the same at sea and
    on land) ;
  \item move without attacking as many large land stacks as wanted
    (non-aggressive movement is not restricted) ;
  \item maintain as many sieges and fights against revolts with large stacks as
    wanted (only movement is restricted).
  \end{itemize}
\end{exemple}

\aparag[None] 0\ducats: No action, no movement, no exploration, no siege,
\ldots allowed (troops may retreat before battle and will fight back if
attacked). No interception allowed.

\aparag[Passive] 10\ducats:
\bparag Interception allowed only in friendly provinces.
\bparag On land: Moving in friendly provinces ; maintaining sieges and fights
against revolts ; moving \LeaderG (and \LeaderC) to retablish hierarchy.
% Jym :
% "reorganiser stacking", j'imagine que c'est que les chefs, sinon je sens la
% faille format San Andreas...
%       \bparag terre: mvt dans ami seulement; continuer sièges; réorganiser
%      stacking
\bparag At sea: Moving stacks of 1\FLEET maximum. No attack.
\bparag Naval actions: friendly-to-friendly transport, maintain fight against
\corsaire, exploration, maintain blocus.

\aparag[Active (aka Simple)] 20\ducats: All allowed by Passive plus
\bparag Any interception.
\bparag On land: one stack of $\leq$ 5 \LD + 1 \Pasha without restriction
[TBD: or +2 pashas ?]
% Jym :
% 2 Pashas garantient presque 3LD, donc +1 au choc. Ca me semble trop
% bourrin. Avec 1 pasha, il n'a en a qu'un a 3LD.
\bparag At sea: one stack with at most 1\FLEET counter without restriction.

% Jym :
% Meme si c'ast plus en cout croissant, ca me semble plus logique de le mettre
% apres pour pouvoir dire "/me too".
\aparag[Active/No Logistic] 10\ducats: Same as Active but
% Jym :
% Diffenrece suivante subtile et prompte a l'erreur (je l'ignorais avant de
% traduire)
% Est-elle vraiment necessaire ? Je ne vois pas trop le genre d'abus faisable
% sans elle...
\bparag At sea: one stack \textbf{without} \FLEET without restriction.
\bparag On land: all stacks $\geq$ 3\LD roll for attrition (even if not moving).

\aparag[Major] 50\ducats: All allowed by passive plus
\bparag On land: either one stack without restriction (neither size nor acton)
or all stacks $\leq$ 5 \LD + 1 \Pasha without restriction  [TBD: or +2 pashas]
% Jym :
% cf simple.
\bparag At sea: either one stack without restriction (neither size nor acton)
or all stacks with at most 1\FLEET counter without restriction.
% Jym :
% Si pas en faveur, je commente...
%            [TBD: ne pas autoriser 2\FLEET sauf \NGD ??? Pas en faveur de ça actuellement]

\aparag[Multiple] 100\ducats: all stacks may act without restriction.


\subsection{Supply, Attrition, Sieges}

\section{Movements}
Each country of the phasing alliance may move its stacks, according to the
campaign it paid for the round. Each movement must be completed before the
next one start. The phasing alliance choose in which order it moves its stacks
(at random in case of disagreement).

Each non-phasing stack may attempt interception. Resolve battles caused by
interception immediately.

Each phasing stack (moving or not) may have to roll for attrition.

\aparag[Conquistador table]
The Conquistador table may be used only:
\bparag in \continent{America} and \continent{Africa}, by any \LeaderC and
\LeaderE (half values) ;
\bparag in \continent{Indonesia} by \leader{Coen}, \leader{van Diemen} and
\leader{Maetsuycker} only ;
\bparag in \continent{India} by all \LeaderC restricted to \continent{Asia}
(@). Namely, \leader{Clive}, \leader{Dupleix}, \leader{Bussy} and the minimum
\LeaderC@ of \FRA and \ANG in period VII.

\aparag[Transport maritime]
\bparag Considéré comme l'action de la flotte.
\bparag Les forces terrestres doivent partir soit d'une côté quelconque,
soit d'un port/arsenal pouvant contenir la pile navale. A destination
nécessairement d'un port/arsenal ami (contrôlé, allié,...) pouvant
contenir la pile navale. La force navale doit stopper son mouvement là.
\bparag Les forces terrestres ont dépensé 2 MP si de port ami à port ami
(3 en \ROTW), 3 MP sinon (6 MP en \ROTW), et peuvent rester dans la
forteresse  ou continuer le mouvement comme province si la forteresse n'est pas assiégée.
Elles peuvent être intercepté dans la province si ils sortent et ont alors le
malus débarquement pour la bataille [TBD ?].
\bparag Jet d'attrition de mouvement pour force terrestre si a embarqué
ailleurs qua dans un port/arsenal.
\bparag Exception avec C ou Gouv ou Expl dans la pile (combinée),
l'embarquement et débarquement en \ROTW est tjs considéré dans un port
ami si il n'est pas opposé (pas de ville, \COL, \TP ou forces d'un
ennemi en guerre contre soi).

\aparag[Invasion navale] Force terrestre : part d'un port ou arsenal
contrôlé.  La flotte doit pouvoir entrer dans le port.
\bparag Il coûte 3 MP à la force terrestre pour être laissé sur une côte
sans port contrôlé (6 MP en \ROTW).
\bparag Jet d'attrition de mouvement pour force terrestre si a débarqué
ailleurs que dans un port/arsenal.

\aparag[combiner mvt terre/mer]
Une force terrestre doit commencer dans la province côtière.

\aparag[Mouvement le long d'un rivière en \ROTW]
se qualifie si un même fleuve ou lac est adjacent
aux deux provinces. Ajouter le coût de traversée du
fleuve le cas échéant.
\bparag Ne sert pas au mvt de pions A; sert pour mvt de LD, LDE
et au ravitaillement.

\aparag[Combat d'ecrasement (Overrun)]
En cas de disproportion des forces en présence, un combat d'écrasement est
possible pendant le mouvement qui ne l'arrête pas (et l'attrition n'est testée
que plus tard); inversement, si une force attaque un adversaire en surnombre,
le défenseur peut déclarer un combat d'écrasement immédiat.
\bparag Si 4 \LD vs. 1 \LD ou moins :  résoudre le combat et si la force la
plus nombreuse gagne, elle peut continuer son mouvement (ou continuer des
interceptions si ce n'était pas sa phase).
\bparag Si 8 \LD vs. 1 \LD ou moins : la force la plus faible est éliminée
automatiquement sans combat (et son chef ne fait pas de test de perte).
\bparag Dans les 2 cas, le défenseur peut dire qu'il se retire dans la forteresse
de la province avant le combat d'écrasement.

\aparag[Rappel de tout ce qui vaut pour une action de la flotte]

\bparag[Exploration] -- résolu pendant le mouvement de la pile
\bparag[Transport naval ou invasion navale]
(c'est-à-dire embarquement, débarquement, y compris ravitaillement et
éventuel blocus de la province d'arrivée) : achève le mouvement naval. \\
Le joueur doit anoncer en entrant dans la mer son intention de débarquer
des forces et dans quelle province. \\
Exception: une pile qui débarque avec un C - ou un E - ne compte pas comme une action. \\
Noter : après bataille navale, interdiction de débuter une transport naval (oui invasion) au
même round (on peut seulement finir celui-ci, si victorieux)
\bparag[Blocus d'un port et/ou ravitaillement maritime] d'un force sur une côte
\bparag[Ravitaillement martitime]
\bparag [Ravitaillement d'un port sous blocus]  il faut entrer dans le port en étant passé
dans un autre port (peut s'accompagner d'un débarquement de
troupes dans le port) et avoir au moisn autant de forces navales que ce qui serait
nécessaire pour le blocus.
\bparag attaque -- à la fin du mouvement de la pile
\bparag lutte contre pirates et corsaires -- résolu à la fin du round

\aparag Actions passives
\bparag Mouvements
\bparag Interception : une flotte d'un joueur inactif (pas en phase)
peut tenter d'intercepter chaque pile qui bouge dans sa mer ou une mer
adjacente (autant de tentatives que d'opportunités, mais un pays/une
alliance ne peut faire qu'une seule tentative par mer).
Si au port : dans une des mer adjacente.


\aparag[Rappel des mod. d'interception] VOIR TABLE
\interceptiona
\bparag Pour les flottes faisant Invasion/Naval Transport : le bonus +2 remplace
le malus de -3 au port ou le bonus de +1(même mer) si intention de
débarquée a été indiquée. Pour une flotte en mer: +2 si c'est dans la
même zone; flotte dans arsenal : +2 dans les mers qui bordent l'arsenal;
flotte dans port: +2 si c'est dans la province du port.

\subsection{Interception}
\aparag Interception is allowed according to the last campaign paid.
\bparag For player without initiative, this is the campaign of the previous
round.
\bparag During first round, players without initiative may intercept (before
their first move) as if they had done a passive campaign.

\aparag[Pour l'interception]
\bparag Noter que c'est l'intercepteur qui attaque toujours.
et que la bataille est résolue tout de suite entre la pile qui intercepte et la force interceptée.
Il peut choisir de regrouper dans sa pile les forces immobiles amies dans la province.
\bparag un intercepteur peut être intercepté à son tour [les forces qui
interceptent viennent en renfort de la pile qui a été la 1e interceptée] ;
c'est tout de même le joueur qui n'est pas en phase qui est l'attaquant.
\bparag Si une force intercepte dans une zone où il y a déjà une force ennemie,
cette for peut intercepter comme par la paragraphe précédent pour se joindre à
la défense [remarque : il peut alors y avoir plus de 8 DT, les forces supplémentaire
n'ont simplement aucun effet sur le combat].
\bparag  Si un combat d'interception a lieu dans une province avec des forces non impliquée
dans le combat, ces forces suivent la retraite éventuelle de la pile allée engagée en
bataille (mais sans perte). [TBD]
\bparag une pile qui a combattu pendant une phase de mouvement et n'a
pas gagné (perdu ou ex-aequo) ne peut plus bouger ni intercepter.
\bparag malus aux interceptions : \\
	intercepter à travers passe de montagne -2 \\
	intercepter depuis ou vers un marais -1 \\
	intercepter à travers détroit : impossible \\

\subsection{Convoys}\label{chMilitary:Convoys}
Do not use if using the experimental system for Revolts, \corsaire and Natives.

\aparag[Convoy movements and Pirates/Privateers]
A convoy (or a naval stack carrying Gold)
entering a sea zone of \STZ or \CTZ is attacked on
\tableref{table:Pirates Natives Raids} by the pirates, and each
privateer allowed to attack the owner of the convoy present in the \STZ
(even if not in the right sea zone) if it has the right to attack the
power (see \ref{chRedep:Corsair Attack}).
\bparag Only \corsaire\faceplus may attack (be they pirates or
privateers).
\bparag The pirates attack first (one only, with leading
named Pirate if any), then the privateers in order of initiative. The
attack is resolved before regular naval interceptions.
\bparag Only one attack for all the pirates in a given \STZ or \CTZ; and
one attack per \corsaire is allowed per move.

\aparag[Attack Procedure]
\bparag Roll for naval interception. Pirates with no leader use 2
as Manoeuvre.  If successful, reduce the \corsaire to \Facemoins and proceed with the attack,
else test for the next interception.
\bparag Before the attack, an accompanying fleet may try to disperse and
reduce the pirates or privateers by making a roll on the corresponding
table.  If successful, the \corsaire is not reduced but the attack is aborted.
\bparag If not aborted, resolve the attack on the  Pirate/Privater raid table.
\bparag Each level in the column \TradeFLEET\faceplus corresponds to one \NTD
captured (with 15\ducats).
\bparag Afterwards, \corsaire goes at port and are finished for the
turn. However Pirates stay in the \CTZ or \STZ and will attack normally Trade
Fleets .
\bparag The \terme{Barbaresque} corsairs cannot attack a Convoy if it is
not in \region{Mediterranee}.

\aparag[Flota de Oro] \label{chMilitary:FlotaDeOroMovement} As soon as the
\terme{Flota de Oro} (and only this convoy) is sunk or reaches Europe, it
reappears in a Spanish port on the Atlantic coast.

\section{Exploration}
Each phasing naval stack not engaged in battle (including interception) may
attempt to explore an adjacent unknown seazone. In case of success, the stack
automatically moves into the explored zone.

Next, each phasing land stack not engaged in battle (including interception)
may attempt to explore an adjacent unknown province. In case of success, the
stack automatically moves into the explored province.

Note that a naval stack embarking troops may explore a new seazone from which
the troops may then disembark to explore a new province.

\section{On discoveries [58]}\label{chMilitary:Discoveries}
\aparag As long as the forces doingthe discoveries have not came back to an
establishment existing at the beginning of the military phase, the discovery
are not yet usable by other forces.  Neither is 'rendez-vous' authorized
between stacks having made independent discoveries (no stacking).

\aparag[Diffusion of discoveries]
\bparag[On sea] At the beginning of period IV, all discovered sea zones of
Atlantique Ocean are known to everyone else at the beginning of the military
phase of the turn following the discovery. Other discovered sea zones have a
bonus of {\bf -2} for discoveries by other players.
\bparag [On land] At the beginning of period IV, all provinces containing a
\COL or \TP are known to everyone else at the beginning of the military phase
of the turn following the discovery.


\section{Battles}
Resolve all non-interception battles caused by the movement, in order of
choice of the phasing alliance (at random in case of disagreement). Each
battle must be fully resolve before the next one starts.

\subsection{Les batailles}
\aparag[Victoire majeure]
Elle est obtenue aux conditions suivantes
\bparag Sur terre en Europe : déroute du perdant et différence des
pertes égale à 3\LD ou plus
\bparag Sur terre en \ROTW : déroute du perdant, perdant avait au moins
un pion \ARMY européen et différence des pertes égale à 3\LD ou plus
\bparag Sur mer : déroute du perdant et différence des pertes
d'au moins 5\ND ou 8 DGa.

\subsubsection{Les batailles terrestres}
\aparag[Organisation des armées et cavaleries] Valable si au moins un pion armée
de la classe en question.
\bparag[\terme{Sipahi}]  TUR (avant réforme M-2) a +1 en choc et poursuite plaine/désert
\bparag[iim]  bonus +1 au choc en I-IV en plaine/forêt orientale
\bparag[tercios] toutes les autres armées ont un malus -1 en choc contre eux sauf
       i, im, ii et iim en I-V
\bparag[iiim] bonus +1 au choc en IV-V en plaine et forêts occidentales,
\bparag[SUE] bonus de +1 au choc en II-VI en forêts nordiques
\bparag[iv] bonus +1 en III-V en plaines et forêts occidentales

\aparag[Test de survie des généraux]
En Europe, sur terre, on ne teste pas le général d'un camp si son adversaire
n'a pas au moins 3 \LD.

\subsubsection{Les batailles navales}
\aparag Deux jours au maximum en cumulant les pertes jusqu'à la fuite (volontaire
ou obligatoire) d'un camp.
\bparag Fuite obligatoire
       si le moral arrive à 0, ou
       si le nombre de pertes reçues est > nbre de D de la flotte
               (tenir compte ici des modificateurs finaux aux pertes).
     Dans les deux cas c'est une déroute (avec poursuite, etc.).
\bparag Fin 2e jour si egalité en moral ;
peuvent choisir de retourner à un port (au choix) ou rester en mer (attaquant
d'abord).
\bparag Les flottes continue leur action si elles ont gagné la bataille, sinon
elles ont fini pour le tour (devant soit retourner auport, soit choisir de rester en mer
en cas d'égalité mais sans rien faire de plus).
Exception : on ne peut débuter un transport maritime après une bataille, même gagnée.

\aparag[Effet de la différence de taille des forces] Modificateurs au dé de bataille
\bparag Si la flotte est de taille >= à (taille+1) adverse, +1 au choc
\bparag Si la flotte est de taille >= à (taille+3) adverse, +1 aux feu, choc
\bparag Si la flotte est de taille >= à (taille+5) adverse, +1 aux feu, choc et poursuite
\bparag Si la flotte est de taille >= à (taille+7) adverse, +1 aux feu,  +2 choc et +1 poursuite
\bparag Si le moral perdu est > moral adverse perdu, -1 aux feu et choc

\aparag[Effet de la taille des forces]
Appliquer variation des pertes:
\bparag Si moins de 6 \ND : réduction des pertes \\
Si plus de 6 \ND (ne pas compter les \NDE):
line +1 if 7 to 12\DN; +2 if 13 to 18\DN; +3 if 19+\DN
de la table 'Size Comparison'.

\bparag Si l'adversaire a dérouté, les pertes sont minimales sont 1.
\bparag Le max de pertes que peut faire une flotte est le double de sa taille
       (1 si 'de' seul).


\aparag[Répartition des pertes]
\bparag  Integer {\bf losses split} evenly in \terme{Damaged}, \terme{Destroyed} and \terme{At port}, in units of \ND.
\bparag {Winner:} 1st \ND\ lost  \terme{Damaged}, 2nd \terme{Destroyed} and 3rd refitted (then loop over).
\bparag {Loser (or equality.):} 1st \ND\ lost  \terme{Damaged}, 2nd \terme{Destroyed} and 3rd \terme{Damaged}.
\bparag Fractions (\tu or \td) vs. \NGD or \NTD are  rounded up. Fractions vs. \ND are applied as \NDE or 2\NDE  in the next category of loss.
\bparag Examples 2: 3\td losses against losing \ND: 2\ND+2\NDE {Damaged}, 1 \terme{Destroyed}\bparag Examples 2: 4\tu losses against winning \NGD = 5 losses: 2  \terme{Damaged}, 2 \terme{Destroyed} (one immediately refitted fo no effect).


\aparag[Pertes en poursuite] En plus des pertes normales, elles permettent
de capturer ou attaquer les transports. Le niveau de capture est égal
au nombre d'étoiles.
\bparag Capture de navires de guerre = le gagnant peut capturer un DN ou
2 DGa par * en poursuite        (pris d'abord sur les Imm, puis les End, puis les autres)
\bparag Couler Transports = 2 DTr coulé par * de poursuite dédiée à ceci
       une force terrestre transportée au minimum égale à ce que
       ces DTr perdus transportent doivent être détruits.
\bparag Capturer Or = 2 DTr par * de poursuite dédiée à ceci avec 5 ors perdus, 10 \ducats capturés.
Les transports sont gardés avec la flotte (et peuvent être repris, attaqué par pirates etc)
jusqu'à un port de la métropole où ils disparaissent (et or dans le RT).

\aparag[Damaged ships] \terme{Damaged} \ND are written down globally by naval zones:
Mediterranean Sea, Atlantic in Europe, Atlantic in \ROTW, Indian, Asian and East Pacific.
They are refitted for usage:
\bparag cost = 0.5*coût achat DN à un round suivant pour les remettre en état.
               Effet = remet tout de suite en jeu les DN voulus.
\bparag gratuit au début du tour suivant si on entretient la flotte;
\bparag on peut la garder \terme{Damaged} pour un coût d'entretien divisé par 2 ;
\bparag On les remet en priorité dans un Arsenal de la zone, sinon dans un
port capable de les contenir.

\aparag[Convoy in battles] If a battle occurs between two naval forces,
one of them containing a convoy or Transports, the convoy does not take
part in the battle, nor incurs losses during it.
\bparag However, at the end of the battle, the pursuit \textetoilex\
result may apply to the convoy or the Transports if the winner decides
so.
\bparag Each \textetoilex\ captures 2 \NTD with corresponding transports
points sunk if loaded with troops, or 10 \ducats captured and 5 \ducats
sunk for \NTD loaded of Gold.
\bparag The rest of the convoy is kept by the loser.

\section{Sieges}
Resolve all sieges, fights against \REVOLT/\REBELLION and \corsaire.

Each alliance, in decreasing order of initiative, resolves all of its actions
in an order of its choice (at random in case of disagreement).

\aparag[Blocus]
Il faut avoir au moins la force navale voulue selon le niveau de forteresse
(voir table ou supra).
\bparag Coupe le bonus de -3 au test de ravitaillement des assiégés ;
\bparag Flotte qui veut sortir ou entrer : doit faire un test pour échapper au blocus
(ou attaquer la flotte en blocus)

\subsection{Les sièges}
\aparag Pour la sape, effet du terrain (non cumulatif)
\bparag           -2      Port sans blocus, terrain clair
\bparag           -3      Port sans blocus, terrain autre que clair
\bparag -2 Terrain accidenté (montagne, marais, forêt, désert) sans port
ou blocus

\aparag[TBD] si un assaut a causé au moins 1 perte (sans modif de
taille ni bonus ``grosse armée'' dans le tour : +1 à la sape et à
l'assaut (max +1, non cumulatif avec le +2 de brèche).


\aparag Les tables sont à jour !

\aparag[Expérimental]
Un assaut qui a obtenu au moins 1 pertes (sans compter les bonus
de Janissaires, \RUS, \POL) sans prendre la forteresse donnera un
bonus de {\bf +1} aux jets de sappe et aux assauts suivants du tour.

\aparag[Port Siegeworks]
Ports that are besieged with at least one level of Siegework are
submitted to a fire from the siegework that works the same way as the
Presidios, with a {\bf +1} per counter Siegework\faceplus.  {\bf But the
 port is not blockaded.}

\aparag[Impossibilité de tenir un siège]
Ceci est regardé au début de la phase de siège (nbre de DT >= niveau) ;
si impossible, mvt de rédéploiement forcé vers chez soi \\
- en fin de tour: si pas Usure\faceplus, redéploiement forcé.


\section{New round}
A new round begin with the \terme{Continuation Roll} segment.

\section{Military cleanup}
???

Normally nothing to do here.

% Local Variables:
% fill-column: 78
% coding: utf-8-unix
% mode-require-final-newline: t
% mode: flyspell
% ispell-local-dictionary: "british"
% End:
