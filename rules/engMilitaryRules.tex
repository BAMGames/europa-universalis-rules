% -*- mode: LaTeX; -*-

\newcommand{\iamhere}{\begin{todo} The following is copy of older stuff and de
    facto cease to be relevant before I reread and rewrite it.
\end{todo}}

\definechapterbackground{Military Rules}{military}
\chapter{Military Rules}\label{chapter:MilitaryRules}

\begin{designnote}
  This Chapter focuses on the description of the Military phase in Segment
  order. The main concepts and common rules used during it are described in
  the next Chapter. Most of these concepts are common with many other wargames
  and do not need to be perfectly defined in order to understand the flow of
  the Military phase, which is why precise description is postponed. You
  should refer to the next Chapter whenever a point in these rules requires
  clarification.
\end{designnote}

\begin{todo}
  This Chapter is under heavy work. The random presence of detailed numbering
  of rules reflects this.
\end{todo}

\section{Overview}
\aparag[Sequence]
\MilitaryDetails

\subsection{Military setup}
Setup for the military phase.

\subsection{Rounds}
The military phase is split in a certain number of \terme{rounds} (between 3
and 11). During each round, each alliance, in decreasing order of initiative,
has an \terme{impulse} where it can moves its troop and fight. Some matters
are resolved before any alliance has its impulse (continuation, wintering) and
some after each had its one (siege, piracy, revolt). After that, a new round
begins.
\subsubsection{Continuation roll}
Determine the new round to be played.

\subsubsection{Wintering}
Stacks may suffer from attrition, leaders may be redeployed.

\subsubsection{Impulses}
Each alliance, in decreasing order of initiative, plays an impulse. Each
impulse consists in five steps: supply, choice of campaign, movements (and
interceptions), exploration, and battles.

The alliance taking its impulse is called the \terme{phasing} alliance. All
other are non-phasing.

\subsubsection{Sieges}
Resolve all sieges, fights against \REVOLT/\REBELLION and \corsaire.

\subsubsection{End of round}
Declare new \terme{Exceptional Levies} and use old ones. Refit \terme{Damaged}
ships.

\subsection{New round}
A new round begin with the \terme{Continuation Roll} segment.

\subsection{Military cleanup}
Compute the final cost of all campaigns paid this turn.

\begin{playtip}[Simultaneity]
  Most of the time, there are separate wars that cannot affect each other
  (typically, with different antagonists in each), or even separate actions
  for a given alliance (typically action in the \ROTW and in Europe). In this
  case, the resolution of the impulses do not need to be as strictly
  sequential as explained in the rules. The military phase is long enough and,
  typically, a FRA-HIS war and a RUS-POL war can be played simultaneously in
  order to make things a little faster.

  It is normally recommended to synchronise all players for the continuation
  roll (because it is a point where some crucial new information is
  gained). Sometimes, it is however possible for two players to quickly
  completely play a small war (noting the results of the continuation roll in
  secret to communicate it later) while other players are still struggling in
  the first round of a big war. Sometimes even while other are still planning
  and resolving their administration\ldots This typically allows to ``free''
  those players to go and buy food for everybody\ldots

  \smallskip

  Similarly, the rules describe a strict order for the resolution of actions
  in the military phase but in practice it doesn't often need to be respected
  that strictly. Typically, two battles can be resolved simultaneously if they
  don't have the same participants, or sieges can be resolved in any order
  (rather than on a per alliance basis) if players agree that this has no
  importance. The rules, however, must provide a strict order to be able to
  solve any disagreement in the order of resolution of actions for the rare
  cases where it does matter.
\end{playtip}

\section{Military setup}
\label{chMilitary:Military setup}
\subsection{Initiative}
Determine the alliances in effect this turn and the order of initiative
between them. Alliances are transitive (the ally of my ally is my ally) for
this purpose (but not for stacking purpose and such). Each alliance plays at
the lowest initiative of one of its member, resolve any tie at random.

Stacks in interventions (whether limited or foreign) act at the same
initiative than the alliance for which they intervene.

Minors at war alone act at the initiative of the country controlling them.

\subsection{Starting round}
Roll one die to determine the starting round (read the result in the boxes of
the rounds display). This die roll is never modified. The weather for this
round is determined as usual (see below).

It is possible for the Sund to be frozen during the first round if the die was
'1' and \ref{eco:Poor weather} happened this turn.

\section{Rounds}
\label{chMilitary:Rounds}
During each round, perform the segments detailed below. The Impulses segment
is repeated for each alliance in decreasing order of initiative.

Each round is labelled by a letter indicating the \terme{season} ('S'ummer or
'W'inter) and a number (between '0' and '5') indicating the \terme{year}.

\begin{designnote}
  Do not take the seasons and years too seriously when interpreting what
  happens during a turn. This is more a modelisation artefact that gives good
  macro results than a real attempt to simulate military actions on a lower
  scale (especially, the 'S' and 'W' rounds happen simultaneously in the North
  and South hemispheres\ldots)
\end{designnote}

\subsection{Continuation roll}
\label{chMilitary:Rounds:Continuation roll}
Do not perform this segment at the first round of a turn.

Roll a die, modified by \bonus{+2} in case of \ref{eco:Poor weather} and
\bonus{-1} if this is period \period{VI} or \period{VII}. Follow the arrows on
the rounds display to determine the new round.

If the new round is the 'End' box, then the rounds stop immediately. Proceed
with \terme{Wintering} then \terme{Military cleanup}.

If the followed arrow is red (modified roll of 1-5 after a Summer round), then
the new round is played with an extended campaign. See choice of campaign for
the implications.

If there is an extended campaign after a (unmodified) continuation roll of
'1', '3' or '5', then the round is played with \terme{Bad weather}.

If \ref{eco:Poor weather} happened this turn, then each Winter round is played
with \terme{Bad weather}, no matter what was rolled (included after a
Winter/Winter transition). Additionally, in this case, if there is a Winter
after an unmodified roll of '1', the Sund is frozen (see \ref{eco:Poor
  weather}).

\begin{playtip}
  End of the Military phase may happen somewhat brutally. You should always
  check the probabilities before planning long term actions (sieges) in years
  4 or 5.

  Given the shape of the rounds track, at least one round in each column must
  happen. Thus, the minimum number of rounds left to play is the number of
  columns and the maximum is twice that number. Moreover, long Military phase
  implies lot of extended campaigns. Take that into account when planning both
  long term actions (sieges) and expenses for the rest of the phase.
\end{playtip}

\subsection{Wintering}
\label{chMilitary:Rounds:Wintering}
If the current year is different than the year of the previous round, then a
\terme{Wintering} segment occurs. Otherwise, skip to the \terme{Impulses}
segment.

There is a \terme{Wintering} segment when the 'End' box is reached, that is,
it is considered as being 'S6'.

\subsubsection{Cold area}
Any stack in a non-controlled, non-national province within the \terme{Cold
  area} rolls for attrition. Resolve this as a Supply attrition (see
below). It is, however, a different roll and a stack may have to roll for
attrition both during the Wintering and Supply segment in some cases.

\subsubsection{Pashas}
Any stack containing \Timar out of \TUR national territory rolls for
attrition. Check specific Turkish rules for details.

\subsubsection{Hierarchy}
Leaders may be redeployed. Leaders that were wounded during a previous round
come back now.

Each country may choose to redeploy its leaders on its stacks any way it wants
(no maximal distance or such, in other words it's a free teleportation). If it
chooses to do so (including return of wounded leaders), then the hierarchy
must be globally respected after this redeployment. It is always possible to
choose not to redeploy leaders at this point, but as long as at least one
leader is redeployed, the country must respect hierarchy globally.

Exception: besieged leaders as well as leaders on stacks with discoveries that
have not been brought back to Europe must stay in place and are not checked
toward global hierarchy.

\subsection{Impulses}
Each alliance, in decreasing order of initiative, plays an impulse. Each
impulse consists in five steps: supply, choice of campaign, movements (and
interceptions), exploration, and battles. Check details of these steps below.

The alliance taking its impulse is called the \terme{phasing} alliance. All
others are non-phasing.

\subsection{Sieges}
Each alliance, in decreasing order of initiative, resolve its sieges, fights
against \REVOLT/\REBELLION and \corsaire. Check details below.

\subsection{End of round}
Each alliance, in order of initiative, declare new \terme{Exceptional Levies}
and use old ones, refit \terme{Damaged} ships. Check details below.

\section{Supply}
\label{chMilitary:Supply}
Each phasing land stack which has no supply for two consecutive rounds is
immediately destroyed. Getting back supply temporarily during the round is
enough to avoid this destruction.

Each phasing land stack which has no supply or is in weak supply must roll for
\terme{supply attrition}. Additionally, each phasing besieged land stack must
roll for \terme{Siege Attrition} (exception: if the fortress was supplied
during the previous round (see naval movement), the besieged stack does not
roll for attrition and is considered in full supply (even if it should
normally be only in weak supply)). Naval stacks never roll for attrition
during the supply segment. Beware that besieged stacks roll for \textbf{siege}
attrition during this segment, which has a similar procedure as supply
attrition but with slightly different modifiers.

If a besieged stack is also in weak supply, it does not roll for supply
attrition. The \terme{Siege Attrition} replaces this roll.

\begin{todo}
  Supply markers for besieged fortresses?
\end{todo}

\subsection{\SoS, \LoS, weak supply}
\subsubsection{At land}
A Source of Supply (\SoS) is either a controlled city, \TP, \COL or fort, or a
large enough naval stack in an adjacent seazone. A Source of Supply may supply
as many stacks as wanted.

The Line of Supply (\LoS) goes from a Source of Supply to the stack. The \MP
cost is counted as if the stack itself was doing this movement (typically, \LD
in the \ROTW compute the length of their LoS using the special \MP costs for
\LD).

A stack is in weak supply if at least one of the following condition is true:
\begin{itemize}
\item its Line of Supply is as least 6\MP long (exception: in the \ROTW, \LD
  do not check this condition);
\item its Line of Supply goes through desert;
\item its Source of Supply is not owned by the same alliance;
\item it is supplied by a naval stack that is not adjacent to a port or
  arsenal able to supply it (exception: in the \ROTW, \LD do not check this
  condition).
\end{itemize}

It is possible to take its supply from a further \SoS to avoid weak supply
(typically, to have an owned \SoS, or to circumvent a desert).

A naval stack can act as a \SoS for an adjacent land stack if it is large
enough. Supplying is a naval action (see below) and thus must be declared
during the naval stack movement. It is valid until the next impulse of the
naval stack (which may be during the next turn).
\begin{itemize}
\item A stack containing only \NDE may supply stacks of at most 1\LD;
\item A stack containing only \ND may supply stacks of at most 3\LD and no
  \ARMY counter;
\item A stack with a \FLEET\facemoins (but no \FLEET\faceplus) counter and at
  least 2\ND in the stack may supply any small land stack (up to 5\LD and
  1\Pasha);
\item A stack containing a \FLEET\faceplus counter and at least 3\ND may
  supply any land stack.
\end{itemize}
\NTD in Convoys are not counted when checking Supply capacity of a naval
stack. This information is summed up in the last two columns
of~\ref{table:Naval Supply}: find the size of the naval stack (or smaller) in
the ``Naval Size'' column and read in the same line in the ``Land supplied''
column the size of land stacks that can be supplied.

\GTtable{supplysize}

In the \ROTW, forts (including missions) can only supply \LD. \TP and \COL can
supply any stack.

Supply is never used up. Thus a \SoS can supply several stacks if it can
supply each of them individually.

\subsubsection{At sea}
A naval \SoS is a controlled port or arsenal large enough ot hold the naval
stack. Fort and mission can only supply \ND; \TP, \COL and regular ports can
supply stacks with at most 1\FLEET counter; arsenals can supply any naval
stack.

A naval \LoS goes from the \SoS to the stack. Its length is the number of sea
zones \emph{crossed} (both entered and exited). Thus, a naval stack adjacent
to a port has a \LoS of length 0.

\subsection{Supply Attrition}
Attrition (in Europe) is obtained by rolling one die, modified as follows, and
cross-reference the modified result in~\ref{table:Discoveries and
  Attrition} in the ``Land, Europe'' column that corresponds to the size of
the stack. In the \ROTW, the result is read in the ``\ROTW and Sea''
column. Note that a result of 5 or less has no effect.

\GTtable{discoveriesattritiononly}

Modifiers for Supply attrition:
\begin{modlist}
\item[+2] per cause of attrition above the first
\item[+2] in case of \terme{massed forces} (the stack contains more than 5\LD
  and 1\Pasha)
\item[+2] if the stack has no \LoS
\item[+2] if the fortress of the province is controlled by the enemy
\item[-M] MAN of the commander of the stack
\item[+?] if the stack is supplied by a naval stack, and the \LoS of this
  naval stack goes through one or more \StraitFort, add the DRM of all the
  \StraitFort along this path (2 in Europe, level/2 in the \ROTW)
% \item[+6] if the fortress of the province is controlled by allies or if there
%   is no fortress in the province (in the \ROTW)
\item[+1] per \PILLAGE\facemoins, \REVOLT\facemoins or unfriendly
  \REBELLION\facemoins in the province
\item[+2] per \PILLAGE\faceplus, \REVOLT\faceplus or unfriendly
  \REBELLION\faceplus in the province
\item[+?] in an uncontrolled province of the \ROTW cold area, add the number
  of Snowflakes ``resource'' (+0 to +2 depending on the \Area)
\end{modlist}

\begin{designnote}[Massed force]
  Note that a \terme{massed force} is \textbf{not} a cause of supply attrition
  by itself (contrarily to movement attrition) but \textbf{if} the stack
  suffers attrition, it is still an aggravating factor. For the sake of space,
  this is not repeated in the tables and considered to be part of the ``per
  extra cause'' modifier, even if it's not an extra cause \emph{stricto
    sensu}.

  \smallskip

  Note also that for supply (or siege) attrition, the \terme{massed force}
  malus always only happen at 6 or more \LD, even in case of a campaign
  without logistic.
\end{designnote}

Modifiers for Siege attrition (besieged):
\begin{modlist}
%\item[+6] always (friendly fortress)
% \item[+?] in the \ROTW cold area, add the number of Snowflakes ``resource''
%   (+0 to +2 depending on the \Area)
\item[-S] siege value of one besieged leader
\item[+S] siege value of one besieger or blockading leader
\item[-3] if besieged in an unblocked port
\item[+1] per \USURE\facemoins
\item[+3] per \USURE\faceplus
\end{modlist}

\subsection{Result of attrition}
In Europe, the result of attrition is either nothing (---), a number
(between 1 and 3), a 'P' or both a number and a 'P'.

The number indicates how many \LD are lost immediately. The controller of the
stack chooses which.

A 'P' is interpreted differently according to the technology of the stack.
\begin{itemize}
\item If the technology is \TMED, \TREN or \TARQ, then 1 more \LD is lost and
  a \PILLAGE\facemoins is placed into the province (and immediately merged
  into a \PILLAGE\faceplus if there was already another one here).
\item If the technology is \TMUS or better, then the controller chooses to
  either loss one more \LD or place a \PILLAGE\facemoins (note that
  \terme{foraging} has no effect during siege or supply attritions).
\end{itemize}

Besieged troops cannot place \PILLAGE and thus must loss one \LD in case of
'P'. There is no extra effect for the lost \PILLAGE with low
technology. Similarly, if there are already two \PILLAGE\faceplus in the
province, it is not possible to add more and any 'P' must be resolved by
loosing one \LD.

Note that since \PILLAGE will make further attritions harder, it is sometimes
wiser to loss \LD rather than let the troops plunders. Note also that since
\PILLAGE\facemoins will be removed at the end of the turn, small troops with a
not too bad technology don't suffer a lot from attrition.

\smallskip

In the \ROTW, cross reference the percentage obtained with the size of the
stack in~\ref{table:Attrition ROTW Remainders}. The resulting number indicates
the number of \LD still alive after attrition while the 'd' indicates one or
two \LDE still alive. If there is a \textetoile, then there is 50\% chance to
lose an extra \LDE.

Stacks in \COL of level 6 use the Attrition in Europe procedure.

\GTtable{attritionlossesonly}

\section{Choice of campaign}
\label{chMilitary:Choice of campaign}
Each country of the phasing alliance chooses its campaign for the current
round. Make a tick in the correct line of the ERS to count how many campaigns
of each type you've done so you can pay for them.

More expensive campaigns allow for more actions. In case of extended campaign,
there is a single campaign that spans over two rounds and that can be upgraded
at this point.

Each country must pay for campaign in order to move its troops, however
multinational stacks are moved as part of the campaign of the commander of the
stack.

\aparag[Campaigns for \MIN]
\bparag Minors in limited intervention have 1 simple campaign each turn and 1
passive campaign each round. The controlling \MAJ may upgrade to any kind of
campaign by paying the difference.
\bparag Minors fully at war (including oversea wars) have 1 simple campaign
each round. They may receive multiple campaigns through reinforcement. The
controlling major may upgrade to an kind of campaign by paying the
difference.

\aparag[Campaigns and interception.] Interception is allowed according to the
last campaign paid.
\bparag For player without initiative, this is the campaign of the previous
round.
\bparag During first round, players without initiative may intercept (before
their first move) as if they had done a passive campaign.

\subsection{List of campaigns}
\aparag[None] 0\ducats: No action, no movement, no exploration, no siege,
\ldots allowed (troops may retreat before battle and will fight back if
attacked). No interception allowed.

\aparag[Passive] 10\ducats:
\bparag Interception allowed only in friendly provinces.
\bparag On land: only passive moves.
% (Moving in friendly provinces ; maintaining sieges and fights against
% revolts ; exploration ; moving \LeaderG (and \LeaderC) alone).
\bparag At sea: Moving stacks of 1\FLEET maximum. No active action.
% \bparag Naval actions: friendly-to-friendly transport, maintain fight against
% \corsaire, exploration, maintain blocus.

\aparag[Active (aka Simple)] 20\ducats: All allowed by Passive plus
\bparag Any interception.
\bparag On land: one stack of $\leq$ 5 \LD + 1 \Pasha may do an active move.
\bparag At sea: any stack may move; one stack with at most 1\FLEET counter
may do an active action, other stacks may only do passive actions.

\aparag[Active/No Logistic] 10\ducats: Same as Active but
\bparag At sea: one stack \textbf{without} \FLEET counter may do an active
action.
\bparag On land: all stacks $\geq$ 3\LD roll for attrition (even if not
moving).

\aparag[Major] 50\ducats: All allowed by passive plus
\bparag On land: either one stack of any size may do an active move;
or all stacks $\leq$ 5 \LD + 1 \Pasha may do active moves.
\bparag At sea: either one stack without restriction (neither size nor acton)
or all stacks with at most 1\FLEET counter may do active actions.

\aparag[Multiple] 100\ducats: all stacks may act without restriction.

\aparag When moving both at sea and on land, the cost of both campaigns is
computed separately and only the maximum cost is paid.

\begin{exemple}
  A Major campaign allows to both:
  \begin{itemize}
  \item attack with one naval stack of 3\FLEET ;
  \item move without attacking (exploration possible) with as many naval
    stacks as wanted (passive actions are not restricted) ;
  \item maintain as many blockades and fight against \corsaire as wanted
    (maintaining blockades and fight against \corsaire are passive actions,
    only initiating them is active).
  \item attack with as many small ($\leq$ 5\LD) land stacks as wanted (the
    reason for which the campaign is Major needs not to be the same at sea and
    on land) ;
  \item move without attacking as many large land stacks as wanted
    (non-active movement is not restricted) ;
  \item maintain as many sieges and fights against revolts with large stacks as
    wanted (only movement is restricted).
  \end{itemize}
\end{exemple}

\section{Movements}
\label{chMilitary:Movements}
\subsection{Generalities}
Each country of the phasing alliance may move its stacks, according to the
campaign it paid for the round. Each movement must be completed before the
next one start. The phasing alliance choose in which order it moves its stacks
(at random in case of disagreement).

Each non-phasing stack may attempt interception. Resolve battles caused by
interception immediately.

Each phasing stack (moving or not) may have to roll for attrition.

All rules here apply to all movements and may be further restricted by choice
of campaign. That is, the ``no restriction'' in campaigns descriptions should
be understood as ``no \textbf{further} restriction''.

The moving stack may pick up and drop units along its movement, however all
these units are considered as having moved and may not move again. Since each
movement has to be completed before starting the next, this \emph{de facto}
prevents two stacks from rendezvousing, merging, and continue movement
together. Moreover, the distance moved is the total (even if no single unit
did its totality) and in case of attrition the size of the stack is considered
to be the total number of all troops that took part in the movement (and they
may all suffer from attrition).

Combined land and sea movement is possible and it is the only case where two
stacks (one naval and one land) move at the same time. The land stack must
start in a coastal province and embark immediately but may continue moving
after disembarking. The naval stack, however, can move before picking up the
land stack but is constrained in what it does after. If a naval stack
transporting troops suffer loses, troops inside may also suffer some loses.

\subsection{Land}
When moving, land stacks expend \MP depending both on the terrain of each
province they enter and the frontier by which they entered it. The cost of
each terrain differ in Europe and in the \ROTW and are indicated
in~\ref{table:Movement points costs}.

\GTtable{movecost}

Each unit may move a maximum of 12\MP. Stacking limits may be ignored during
movement, but at the end of movement of each stack, units over the stacking
limit must be destroyed (chosen by the controller of the stack).

A land stack may be carried over water by a sea stack. The land stack must
start its movement in the province where it embark but may move after
disembarking. The land stack may embark or disembark from a non-controlled
port (evacuation or landing), but at least one of the endpoint must be a
controlled port/arsenal large enough to contain the naval stack.

A land stack must roll for attrition if at least one of the following cases is
true:
\begin{itemize}
\item the stack contains at least 6\LD (not counting those in \Pasha) or at
  least 2 \Pashas, such large stacks roll for attrition even if they don't
  move, ignore this cause between interceptions (see below);
\item[OR] the stack contains at least 3\LD and this is a campaign without
  logistic, such large stacks roll for attrition even if they don't move,
  ignore this cause between interceptions (see below);
\item the stack moved at least 6\MP;
\item the stack moved at least 3\MP and there is \terme{Bad weather};
\item the stack used ships to make a naval move and at least one of the
  endpoint is not a controlled port/arsenal large enough to hold the
  transporting navy.
\end{itemize}

Note that large stacks must roll for attrition even if they don't move. Thus,
it is usually better to have half of the stack doing a back and forth move so
that none of the part has to test attrition. While this is harmless at home,
doing this on the frontline can increase the cost of campaign (because coming
back on a siege is an aggressive move) or offer opportunities for interception
to your opponents (and thus opportunities to crush your stack half by half),
so use this trick wisely.

Note also that a campaign without logistic does decrease the ``massed forces''
threshold but only one ``massed forces'' causes may apply (\emph{i.e.} a 6\LD
stack moving without logistic has only 1 cause of attrition). On the other
hand, \terme{Bad weather} creates a new attrition cause, and the normal
``distance'' cause may also apply (\emph{i.e.} a stack moving 6\MP or more
during \terme{Bad weather} has two causes of attrition: ``6\MP'' and ``3\MP
during \terme{Bad weather}'' hence it must roll for attrition and will suffer
from a ``extra cause'' malus).

In case of interceptions, the conditions for attrition are checked for each
part of the movement (between the beginning and the first interception;
between each interception; between the last interception and the end). If one
part of the movement triggers attrition, it is resolved immediately before the
interception battle occurs and its results (especially \terme{foraging}) are
applied and effective for the next battle only. Exception: being a large stack
is not enough to trigger attrition between interceptions, only at the end of
movement. It is, however, still considered an extra cause of attrition with
the corresponding malus.

If none of the parts trigger interception but the whole movement does, then
one and only one final attrition is resolved at the end of the movement (thus,
possibly after a lost interception battle). For this final check, being a
large stack is a cause for attrition as usual.

\begin{exemple}
  A small stack moves 6\MP and is intercepted. Since 6\MP is a cause of
  attrition, it resolves it immediately. If the stack wins the battle, it may
  move on. It moves 6 more \MP (for a total of 12) and thus rolls for
  attrition again. If it has a regular battle at the end of the movement, then
  any \terme{foraging} result obtained during the first interception is
  ``used'' by the first battle and no more effective.

  \smallskip

  A small stack moves 4\MP, get intercepted and wins, move 4\MP, get
  intercepted and wins and moves a final 4\MP. None of the ``legs'' of
  movement were enough to trigger attrition, but the total movement of 12\MP
  is. Hence the stack makes a final attrition test at the end of its
  movement. Note that even if the first two ``legs'' represent 8\MP and thus
  would trigger attrition, it is not done before the battle as only the latest
  ``leg'' is checked at this point. However, if the stack loses its second
  battle, then it must stop movement and has moved 8\MP, thus it must resolve
  an attrition test now (after the battle).

  \smallskip

  A small stack moves 6\MP and get intercepted. It resolves its attrition and
  then wins the battle. It moves 5 more MP before ending its moves. The last
  leg is not enough to trigger attrition. The whole movement is enough to
  trigger attrition but since an attrition test already happened (before the
  battle), it is considered done and no extra test occur.

  \smallskip

  A large stack moves 5\MP and is intercepted. The movement is not enough to
  cause attrition and stack size is ignored at this point. If the stack loses,
  at the end of movement it has been a large stack moving, a cause for
  attrition, and thus must resolve it.

  A large stack lands in enemy territory and is intercepted. The movement is
  enough to trigger attrition and the ``large stack'' will count as an extra
  causes and gives \bonus{+2} to the roll. If it loses the battle, since it
  already rolled for attrition during its move, it does need to do so again.
\end{exemple}

Any movement that enters a province with unfriendly forces (land stack,
\REVOLT/\REBELLION, fortress (even if besieged), \ldots) is \terme{active} and
cannot occur as part as a passive move. Passive campaign allow only passive
moves while other campaigns allow one or more active moves. Note that not
moving from, or exiting an unfriendly province can be done as a passive move.

Land stacks may not enter provinces that contains neutral unbesieged forces.

When exiting a province whose fortress is not controlled, a stack must either
empty the province (removing the siege) to go back to a friendly province (no
enemy (even besieged), no \REVOLT/\REBELLION, \ldots) or leave enough troops
to maintain siege (1\LD per level of the fortress).

Movement of leaders alone is allowed. A leader moving alone has a move
capacity of 20\MP. It never rolls for attrition. It can use naval transport
even if it does not start adjacent to sea, and it does not need a naval stack
to be carried (the leader sails on some non represented ships). On sea, a
leader alone cannot enter sea zones with malus and has (for the land leaders)
a \Man of 0. A leader alone dies immediately if he enters a province with
unbesieged enemy forces (and no friendly force) or if he is
intercepted. Moving leaders alone can help enforce hierarchy or quickly change
your frontline. It is considered a passive moves even if the leader enters a
province with enemy force (typically to take command of a stack). Note that
hierarchy rules strictly constraint moving leaders alone and such change of
command is normally done during the wintering segment.

A stack entering a province with enemies may declare an \terme{overrun} if it
contains at least four times as many \LD. Overruns are resolved after all
interception attempts in the province are resolved. If the phasing stack
contains at least 8 times as many \LD, then the non-phasing stack is
immediately destroyed without battle and its leader must roll for wounds as
after a regular battle where all the troops are destroyed; the phasing stack
does not roll for attrition before the overrun and may then continue moving as
if nothing happened. If the phasing stack contains at least 4 times as many
\LD, then the battle occurs immediately, as an interception battle (roll for
attrition if needed, the phasing stack may continue movement if it wins). It
is possible to overrun forces that just intercepted the moving stack
(typically when several interceptions are attempted but only one succeed).

Similarly, if a moving stack enters a province with 4 or 8 times as many enemy
\LD, the non-phasing stack may declare an overrun. In that case, there is no
need for interception (it is automatic). If the province already contains
forces friendly to the moving one (including besieged forces), the overrun is
only possible if the non-phasing forces contains at least four time as many
\LD as the moving one.

Phasing stacks may not enter provinces with presence (even besieged) of
neutral countries (that are neither at war with (allied) nor against (enemy)
the phasing alliance). In the \ROTW, provinces that do not explicitly belong to
minors are not concerned by this restriction.

In the \ROTW, each time a land stack enters a province whose natives are not
yet activated, it must test for activation. Exception: leaders alone never
test for activation, natives in some \ROTW countries are not tested if the
moving stack has an \dipAT with the country.

To test for activation, roll 1d10. If it is equal or less than the
\terme{Tolerance} of the \Area (third value), the natives of the province
(only) are activated and attack the stack. Treat this as an interception
battle where the natives are the interceptor.

Before the test for activation, the stack may decide to attack the natives. In
this case, do not roll for activation. The phasing stack must stop and the
battle will be resolved together with regular battles. Note that even if this
is a regular battle, there was no enemy in the province before the stack
entered it (and activated them), hence this is still a passive move.

In the \ROTW, a land stack may try to convert natives to its side in each
province it moves into by rolling on~\ref{table:Conquistadors Effects}. It may
only attempt to do so if the natives in the province are not already activated
when it enters it. This roll is made before any activation test and will
automatically activate the natives.

\GTtable{conquistadorsonly}

\aparag[Conquistador table]
The Conquistador table may be used only:
\bparag in \continent{America} and \continent{Africa}, by any \LeaderC and
\LeaderE (half sum of values, round down) ;
\bparag in \continent{Indonesia} by \leader{Coen}, \leader{van Diemen} and
\leader{Maetsuycker} only ;
\bparag in \continent{India} by all \LeaderC restricted to \continent{Asia}
(@). Namely, \leader{Clive}, \leader{Dupleix}, \leader{Bussy} and the minimum
\LeaderC@ of \FRA and \ANG in period VII.

Roll 1d10, modified as follows and cross-reference the result with the sum of
the \LeaderC value. The result may contain: a \textdag, a ---, a R followed by
a percentage or a D followed by a percentage. Apply all the results obtained,
starting with the D.

List of modifiers (cumulative):
\begin{modlist}
\item[+1] for each previous time in the game that the table was used in any
  province of the \Area.
\item[-1] if there is a \LeaderMis in the stack.
\item[+1] if the stack contains at least 4\LD.
\item[-1] if the stack does not contain any \ARMY counter (only take into
  account counters of actual countries, not the generic \pays{natives} ones).
\item[+1] if the leader using the table has a sum of stats of 6 or less.
\end{modlist}

Explanation of the results:
\begin{itemize}
\item ---: all the natives in the province resist (same as 'R00').
\item \textdag: no native resist (same as 'R100').
\item Dxx: apply~\ref{table:Attrition ROTW Remainders} with xx\% on the
  number of natives in the province. The result is the number of native \LD
  that desert and join the stack. Use \pays{natives} counters to denote them
  (using \ARMY counters to denote 2 or 4 \LD as needed).
\item Ryy: apply~\ref{table:Attrition ROTW Remainders} with yy\% on the
  number of natives in the province. The result is the number of native \LD
  that resist and fight.
\item All natives that neither resist nor desert are eliminated from the
  province (they will come back at the end of the turn following normal
  rules).
\end{itemize}

Next, a battle occurs between the resisting native and the stack (including
the new recruits). Treat this as an interception made by the resisting
natives.

Natives in a stack never count toward stacking limit and do not hamper
technology or moral. They must stay stacked with the \LeaderC that used the
table and cannot use naval transport. They are automatically disbanded if the
\LeaderC is wounded or killed, leave land, or at the end of the turn.

\begin{exemple}
  If there are 6\LD of natives in the province and the result is 'R30/D80' the
  by cross-referencing 80\% with 6\LD we see that there is 1\LD of natives
  joining the \LeaderC and by cross-referencing 30\% with 6\LD, we see that
  there are 4\LD that resist and fight. The last \LD of natives is
  killed. That is the province now only contain 4\LD (those that resist) which
  are activated and fight following usual rules.

  Note that due to the way the table is read, 'R30/D80' actually means that
  there is 100-80=20\% of the natives that desert and 100-30=70\% that resist.
\end{exemple}

\subsection{Sea}
Naval stacks have an unlimited movement capacity. A naval stack that does not
stay in a port always rolls for attrition (even if it does not move and simply
stays at sea). The attrition roll is modified by the distance moved, hence the
farther a naval stack moves, the more attrition it suffers (especially for
large stacks).

Each new sea zone or port entered during the movement is counted as a ``zone''
for the distance. Entering ports is usually done at the end of movement, to
avoid certain dangerous sea zones, or to pick up troops for combined move.

At the end of its movement, a naval stack is allowed to do one naval action
(some actions are actually composed of several others). This is never
mandatory. Especially, battle is not mandatory and naval stacks of different
alliances may coexist in the same sea zone without problem.

Most actions are active and can only be done as part of an active
campaign. Some are passive and can be done as part as any campaign. Note that
many passive actions are simply ``continuing the active action from previous
round without moving.''

Actions are announced when the naval stack enters the sea zone where the
action will occur. Announcing a naval action ends the movement of the stack
and attrition (for the whole movement) is rolled immediately before any
interception (in the last zone) is declared and before actually resolving the
action.

Naval stacks may only enter port/arsenal that are \emph{large enough}, that is
that are \SoS for it. Especially, it is never possible to enter a port/arsenal
controlled by another alliance.

Naval stacks may be intercepted. In case of interception (during movement), do
not roll for attrition before the interception battle. Attrition is only
resolved once at the end of movement. The attrition takes into account the
whole movement of the stack, whatever the number of interception battles that
may have occurred during it. If a naval stack is intercepted and looses
battle, roll for attrition only once (for both movement and retreat), taking
into account the whole movement (\emph{e.g.} the distance moved is the total
of what was moved before the battle and during the retreat).

Each time it tries to enter or exit a blockaded port, a stack must roll to
escape blockade. Roll 1d10 modified as follows:
\begin{modlist}
\item[+M] \Man differential between the moving leader and the blockading
  one. Only count it if it is positive (\emph{i.e.} if the moving leader has
  more \Man).
\item[+1] if the blockading stack is smaller (in number of \ND).
\item[+1] if the blockading stack is composed of \NWD and does not have
  technology \TSF.
\end{modlist}

If the result of the modified roll is 8 or more, the stack managed to escape
the blockade and may pursue its movement as wanted.

If the result is 6 or 7, the stack did not manage to escape blockade. If it
was trying to exit a port, it must stays in (and its movement ends). If it was
trying to enter it, it may either stop moving or return to the closest
friendly port/arsenal (at choice, if any), in both cases it may not do any
action.

If the result is 5 or less, the stack did not manage to escape blockade (as
above). Additionally, the blockading stack may choose to immediately engage
the moving stack. This is treated as an interception battle (\emph{i.e.} the
blockading stack automatically succeeds any interception attempt against the
moving stack).

\GTtable{navyblockade}

\aparag[List of active naval actions.]
\bparag[Battle.] The moving stack declares a battle. The battle will be
resolved during the Battle Segment of the impulse. A non-phasing stack that is
engaged in battle may not intercept any more. Each non-phasing stack may only
be attacked by one phasing stack (\emph{i.e.} you need to merge before
battle).
\bparag[Blockading an enemy port/arsenal.] The naval stack must have
sufficient size according to~\ref{table:Naval Size for Blockade}, depending on
the size of the fortress. Note that to blockade a level 2 or 3 fortress, one
must have have at least a \FLEET\Facemoins counter \textbf{and} 2\ND in the
stack similarly blockading a level 4/5 fortress requires a \FLEET\Faceplus
counter \textbf{and} 3\ND in the stack. Put the stack close to the blocked
port/arsenal. Blockade makes siege easier. Only one stack may blockade each
port (if several stacks want to blockade the same port, they are automatically
merged). Blockade last until the next movement of the naval stack, including
possibly in the next turn.
\bparag[Supplying troops.] The naval stack must have sufficient size for
supplying the troops as indicated in~\ref{table:Naval Supply}. Note that a
naval stack is allowed to supply several adjacent land stacks since it acts as
a \SoS and supply is never ``used up''. Supply last until the next movement of
the naval stack, including possibly in the next turn.
\bparag[Blockade + Supply.] A stack may both blockade a port/arsenal and
supply land troops. However, in this case it can only supply the land stack in
the province of the port/arsenal it is blockading.
\bparag[Supplying a besieged port.] The stack must have exited a non-blocked
friendly port (either starting from it or moving in and out) during its
movement and entered the besieged port. Moreover, it must be a stack large
enough to blockade the supplied port. This will remove the ``blockade''
situation of the port for the siege segment of this round (only) and remove
any attrition roll for troops inside for the next supply step (only). Note
that this is mostly useful if the port is blockaded and thus require escaping
it\ldots Moreover, the stack will end its movement in the supplied port as
actions end movement. It is possible to decline the possibility to supply in
order to continue moving (for example to embark besieged troops and retreat
them before the fortress falls).
\bparag[Fight \corsaire.] The stack can initiate fight against \corsaire from
the sea zone it is in. The \corsaire counter (or counters) targeted by the
stack must be in the same sea zone. Only one stack per alliance may fight
against \corsaire in a given sea zone.
\bparag[Naval Transport.] See details below. Can be combined with blockade
and/or supply of the invaded province (or reinforced fortress) only.
% \bparag[Naval transport in the \ROTW.] In the \ROTW, when either the naval
% stack contains a \LeaderE or the land stack contains a \LeaderC or a
% \LeaderGov, then the disembarking itself is not an action. The naval stack may
% continue moving and may do an action, it may also do the disembarking after
% doing its action. Doing this is still considered as active for campaign
% purpose (\emph{i.e.} a naval stack may continue moving after a \ROTW invasion
% of an enemy province, and this stack may do another active action later; but
% even if it does not do any active action later, the invasion is still
% considered ``active'' for campaign cost purposes).

% (Jym): No. F acting after disembarking means 2 stacks allowed to move at the
% same time. It's bad.

\aparag[List of passive naval actions.]
\bparag[Exploration.] Resolved during next step.
\bparag[Continuing blockade and/or supply.] The naval stack may not move and
must have already been blockading the same port (or supplying from the same
sea) on the previous round (including on previous turn if this is the first
round). Conditions on stack size and on combined blockade+supply are still
enforced.
\bparag[Continuing fight against \corsaire.] The naval stack may not move and
must have already fighting \corsaire from the same sea zone on previous
round. Only one stack per alliance may fight against \corsaire in a given sea
zone.
\bparag[Friendly naval transport.] If both endpoints of the naval transport
are friendly, the transport is a passive action. Additionally, in the \ROTW,
if the combined stack contains a \LeaderE, \LeaderC or \LeaderGov and no \ARMY
counter then embarking from or disembarking in a province with no enemy
presence (activated natives, city, establishment, troops, \ldots) is
considered as a friendly port (not arsenal).
% \bparag[Friendly naval transport in the \ROTW.] As for active transport, a
% passive transport is not an action in the \ROTW if the combined stack contains
% a \LeaderE, \LeaderC or \LeaderGov. Note that if the land stack tries to
% disembark in an unknown province and fails its exploration, it must reembark
% and thus will be destroyed if the naval stack is not here anymore\ldots

\aparag[Flota de Oro] \label{chMilitary:FlotaDeOroMovement} As soon as the
\terme{Flota de Oro} (and only this convoy) is sunk or reaches Europe, it
reappears in a Spanish port on the Atlantic coast.

\subsection{Combined move}
A combined move, or naval transport, happens when a naval stack carries a land
stack and they move together. The land stack may not move before embarking but
may move after. On the other hand, the naval stack may move before embarking
troops but disembarking troops is a naval action and ends its movement. Hence,
there is always only one stack moving, first it is solely naval, then it is a
combined land and naval stack and finally it is solely land. A naval stack may
not embark troops if it has already been engaged in a battle this impulse
(\emph{i.e.} after winning an interception battle).

The naval stack must be large enough to hold the land stack, as indicated
in~\ref{table:Sea Transport for Armies}. Each \ARMY\Faceplus, depending on its
\terme{army class} and the period, needs a certain number of \terme{transport
  points} as indicated in the table. An \ARMY\Facemoins requires half that
number, a \LD requires 2 points and a \LDE half a point. Conversely, each \NWD
or \NGD provides 1 point, each \NTD 3 points and each \NDE half a point. The
sum of the transport capacity of the naval stack must be at least equal to the
sum of the transport points required for the land stack.

\GTtable{fleettransport}

Either the start (evacuation) or end (invasion) of the naval transport may be
uncontrolled but not both. That is, at least one of the endpoint must be a
controlled port or arsenal large enough to contain the naval stack. Since the
transport is an action of the naval stack, when disembarking in a controlled
port, the naval stack ends its movement in that port, not in the adjacent sea
zone. It is possible to choose to disembark out of the port to keep the naval
stack at sea but this obviously removes any advantage of disembarking in the
port (it becomes an active action, cannot be done after an evacuation, costs
more \MP for the land stack and is a cause of attrition for it).

Even if the transport is an action for the naval stack it may be followed by a
blockade+supply of the landing province or city (only). That is, the action is
transport + blockade + supply (invasion) or transport + supply (reinforcing a
besieged fortress).

As any action, the landing must be announced when the naval stack enters the
zone, before any interception are declared or resolved. Announcing a landing
gives bonus to interception. As for any action, it is resolved after any
interception and resulting battle (and thus only if victorious).

If both endpoints contains a large enough controlled port or arsenal, then the
transport is a passive action. Otherwise, it is an active action. If the
combined stack contains a \LeaderE, \LeaderC or \LeaderGov, then any province
with no enemy presence (activated natives, enemy troops, enemy city or
establishment (even besieged), \ldots) is considered to contain a controlled
port (not arsenal) (both for embarking and disembarking).

Note that disembarking troops in a besieged fortress is actually a passive
action since the port is controlled. However, if the fortress is resupplied as
part of the same move, then it becomes an active action as supplying a
fortress is an active action. It is possible to decline the possibility to
re-supply the fortress in order to keep the transport as a passive action.

A land stack may not stay in the ships. That is, both embarking and
disembarking must occur as part of a single move. Especially, if the naval
stack is intercepted and looses the battle, it retreats to port (as per normal
rules) and the land stack is automatically disembarked. As usual, the land
stack may continue its move after such an automatic landing (it did not loose
an interception battle, only the naval stack did).

As an exception, in the \ROTW, a land stack that do not contain any \ARMY
counter may stay inside ships at the end of movement and even from one round
to the next. If it starts the round in the naval stack, it is considered to
have embarked in a controlled port for all purposes.

If the naval stack suffers attrition or losses during exploration, then the
land stack suffers the exact same percentage of loses (one cannot choose which
ships are sunk in a storm\ldots) This does include the attrition done for the
movement of the naval stack as it is resolved before the landing itself
(attrition of naval stacks is always resolved before their action). However,
if the naval stack suffers loses from battle, the land stack does not suffer
any damage (it is assumed that the loaded ships were better protected and the
exposed empty ships took the damage).

If the transport capacity of the naval stack falls below the requirement to
carry the land stack (due to loss during attrition, exploration or battle),
then the land stack immediately suffers enough loss to be small enough to be
carried by the resulting naval stack.

For the land stack, embarking or disembarking out of a controlled port is a
cause for attrition. It also costs more \MP than if both endpoints are
controlled ports.

If the land stack fights in the province where it lands, whatever the cause
(previous enemy presence or interception), it will suffer from the
disembarking malus on the 1st day of battle. This includes both the case where
the land stack disembarks in a besieged controlled fortress and immediately
attempts a sortie against the besieger and the case where a landing is
intercepted in a previously completely friendly province.

Gold can be carried as land forces. It requires 1 transport point for every
5\ducats. Gold can only be embarked in a controlled port and is immediately
disembarked when it reaches a friendly port on the European map (and added to
\lignebudgetlong{Gold from ROTW and Convoys}). Gold may stay inside ships from
one round to another. It does suffer attritions and exploration loses in the
same proportion than the carrying stack. During battle (especially during the
retreat), gold can be specifically targeted by the enemy. The \terme{Flota del
  Oro} and \terme{Flota del Per\'u} convoys may also be used to carry gold (it
is tere only purpose). On the other hand, the others convoys are already
created full of gold and cannot be used to carry more.

% Autre limite, un pion naval ne peut en général transporter qu'un seul pion
% terrestre. Pour être plus précis, un pion F de navires peut transporter
% jusqu'à une A+ (soit l'équivalent de 4 DT, éventuellement en 2 pions, mais pas
% 3) ; si le transport est assuré par des DGa et des DTr dans F, la limite passe
% à 2A+ en 3 pions. Un détachement naval (de toute nature : DTr, DN, DE) ne peut
% contenir plus d'un pion (donc DC, DT ou A- pour les armées les plus petites).

% (Jym) : bof, ça me semble bien trop compliqué à gérer pour ce que ça
% apporte.

\section{Attrition}
\label{chMilitary:Attrition}
\subsection{Generalities}
Moving stacks may have to roll for attrition. The procedure is similar on land
and at sea but with different modifiers and a different way to read the
table. Attrition is always resolved by cross-referencing a modified die roll
and the correct column in~\ref{table:Discoveries and Attrition}. On sea or in
the \ROTW, a further read in~\ref{table:Attrition ROTW Remainders} is
required.

Beware that even if the procedure is very similar to supply or siege
attrition, the modifiers and the way to implement the results are different.

\subsection{Land}
Check the movement rules to determine when a moving land stack has to roll for
attrition. The modifiers are:
\begin{modlist}
\item[+2] per cause of attrition above the first
\item[-M] MAN of the commanding leader
\item[+2] if the stack had no \LoS at the beginning of its move
\item[+?] if, at the beginning of its movement, the stack is supplied by a
  naval stack, and the \LoS of this naval stack goes through one or more
  \StraitFort, add the DRM of all the closed \StraitFort along this \LoS (2 in
  Europe, level/2 in the \ROTW)
\item[+2] if the stack \emph{enters} at least one enemy province during its
  movement (\emph{i.e.}  not when exiting enemy territory)
\item[+1/+2] per \PILLAGE\Facemoins/\Faceplus, \REVOLT\Facemoins/\Faceplus and
  unfriendly \REBELLION\Facemoins/\Faceplus in each province exited or entered
  during the move (\emph{i.e.} count all the \PILLAGE, \REVOLT and unfriendly
  \REBELLION along the path, including in the start and end provinces)
\item[+?] for each uncontrolled province of the \ROTW cold area exited or
  entered during the move, add the number of Snowflakes ``resource'' (+0 to +2
  depending on the \Area)
\end{modlist}

In Europe, cross-reference the result with the ``Land, Europe'' column of the
Table that corresponds to the size of the stack. The result of attrition is
either nothing (---), a number (between 1 and 3), a 'P' or both a number and a
'P'.

The number indicates how many \LD are lost immediately. The controller of the
stack chooses which.

A 'P' is interpreted differently according to the technology of the stack.
\begin{itemize}
\item If the technology is \TMED, \TREN or \TARQ, then 1 more \LD is lost and
  a \PILLAGE\facemoins is placed into one of the provinces of the movement
  (possibly the start or end of the move).
\item If the technology is \TMUS, \TBAR or \TMAN, then the controller chooses
  to either loss one more \LD; or to both place a \PILLAGE\facemoins in one
  province of the movement and to \terme{forage} during the next battle of
  this impulse (only).
\item If the technology is \TL then the controller chooses
  to either loss one more \LD; or to place a \PILLAGE\facemoins in one
  province of the movement.
\end{itemize}
As usual, two \PILLAGE\Facemoins are immediately merged into a
\PILLAGE\Faceplus. It is not possible to add a \PILLAGE in a province that
already contains two \PILLAGE\Faceplus.

Note that \terme{foraging} only affects one battle (interception or regular,
whichever occurs first) and necessarily during this impulse. Hence, a stack
that does not fight immediately will not suffer from \terme{foraging},
typically, if the opposing alliance attacks it during the next impulse.

\smallskip

In the \ROTW, the result is read in the ``\ROTW or Sea'' column. Cross
reference the percentage obtained with the size of the stack
in~\ref{table:Attrition ROTW Remainders}. The resulting number indicates the
number of \LD still alive after attrition while the 'd' indicates one or two
\LDE still alive. If there is a \textetoile, then there is 50\% chance to lose
an extra \LDE. Repeat this procedure for every set of 10\LD in the stack, and
for the remainder.

If all the provinces of the movement (including start and end) are European
(including \COL of level 6), use the Attrition in Europe procedure. If at
least one is a \ROTW province, use the Attrition in the \ROTW province. Note
that only the land provinces are considered for this. Sea zones crossed during
a naval transport play no role in deciding whether to use attrition in EUrope
or in the \ROTW.

\subsection{Sea}
Naval stacks at sea always roll for movement attrition, even if they are not
actually moving. Naval stacks that stay in a port do not roll for attrition.

When computing Attrition modifiers for naval stacks, it is important to know
the \emph{greatest sea difficulty}. First, consider all the sea zones of the
movement (including start and end) and look at their difficulties; next reduce
the difficulty of any sea zone with a \SoS for the stack (\emph{i.e.} friendly
large enough port) by 2; lastly take the greatest of these modified values.

\begin{exemple}
  If a naval stack moves from a sea zone of difficulty 5 with a port into a
  sea zone of difficulty 4 without port, then its greatest sea difficulty is
  4.
\end{exemple}

When computing both distance moved and distance to port, remember that only
the number of zones (either ports or sea zones) \emph{entered} counts. Thus a
naval stack adjacent to its port has a \LoS length of 0 and a naval stack that
simply moves from a port to the adjacent sea zone has a move length of 1.

List of naval attrition modifiers:
\begin{modlist}
\item[-6] Always
\item[-M] MAN of the commanding leader
\item[+?] if, at the beginning of its movement, the \LoS of the stack goes
  through one or more \StraitFort, add the DRM of all the closed \StraitFort
  along this \LoS (2 in Europe, level/2 in the \ROTW)
\item[+X] add the malus of each sea zone with malus \emph{entered} during the
  move
\item[+1] for \NWD in \TCAR technology
\item[-1] for any stack in \TBAT technology
\item[-2] for any stack in \TVE or \TTD technology
\item[-3] for any stack in \TSF technology
\item[+2] if there is \terme{Bad weather} this round
\item[+?] Greatest sea difficulty (as explained above)
\item[+1] per 4 zones entered (round down), if the stack contains 0 or 1
  \FLEET counter [BLP]
\item[+2] per 2 zones entered (round down), if the stack contains 2 \FLEET
  counters [BLP]
\item[+4] per 2 zones entered (round down), if the stack contains 3 \FLEET
  counters [BLP]
\item[-1] if the stack contains several \FLEET counters and moves from an
  arsenal to another arsenal
\item[+3] if the stack starts its movement 1 sea zone away from its port
  (\emph{ie} not adjacent to port)
\item[+6] if the stack starts its movement 2, 3 or 4 sea zones away from its
  port
\item[+9] if the stack starts its movement 5 or more sea zones away from its
  port
\item[+1/+2] per unfriendly \corsaire\Facemoins/\Faceplus in each sea zone
  exited or entered during the move (\emph{i.e.} count all the \corsaire along
  the path, including in the start and end zones) [BLP]
\item[-S] siege of one \LeaderA or \LeaderE in the stack if it is blockading a
  port a the start of its move [BLP]
\item[+?] half the level (round up) of fortress blockaded at the start of the
  move [BLP]
\end{modlist}
Note that the distance moved modifier (for large stacks) is ``+2 (or 4)/2
zones'' and \textbf{not} ``+1 (or 2)/1 zone''. Thus, a large stack that only
moves one zone does not have this malus.

The result is read in the ``\ROTW or Sea'' column. Cross reference the
percentage obtained with the size of the stack in~\ref{table:Attrition ROTW
  Remainders}. The resulting number indicates the number of \ND still alive
after attrition while the 'd' indicates one or two \NDE still alive. If there
is a \textetoile, then there is 50\% chance to lose an extra \NDE. If the
stack contains more than 10\ND, apply this procedure for each set of 10\ND and
for the remaining \ND separately. The controller of the stack chooses which
\ND are sunk.

Any \NTD that must suffer at least 1\NDE of loss is entirely destroyed.

\section{Interceptions}
\label{chMilitary:interceptions}
\subsection{Generalities}
Non-phasing stacks may attempt to intercept a moving enemy stack under certain
conditions. Interceptions are declared whenever a moving stack enters a new
zone (sea zone or province) and resolved immediately. Interceptions can be
attempted by stacks in the same or adjacent zone (sea zone, port or province)
All interceptions are announced before any is resolved.

A non-phasing stack may decide to split before intercepting, that is intercept
with only part of the stack (typically to maintain a siege or blockade with
the other part). Hierarchy and other usual rules for movement and splitting
stacks must be respected in such a case.

If an interception is successful, the intercepting stack is moved into the
zone of the interception (note that this does allow free moves that are not
counted as part as any campaign and may allow, typically, to lay new
sieges). Once all interceptions are resolved, a battle occurs immediately
between the moving stack and all the successful interceptors.

If stacks of different alliances attempt to intercept the same stack in the
same zone, then interceptions are announced and resolved in decreasing order
of initiative. Once one alliance successfully intercepted one stack in a zone,
other alliances may not attempt to intercept the same stack in the same zone
anymore (to avoid three-sided battles). It is however possible to intercept in
another zone if the moving stack wins and goes on.

After an interception battle, if the moving stack won it may continue moving,
if it didn't it must stop movement. For campaign costs purposes, check only
the part of the movement that was effectively done, not the intention
(\emph{i.e.}  if a naval stack declares a landing (an active action) but is
defeated in an interception battle before it actually occurs, then it has done
no action and does not count toward an active campaign, similarly, a land
stack that is defeated before it could enter enemy territory only did a
passive move).

A stack that already fought a battle without winning this impulse (\emph{i.e.}
a previous interception, lost or tied) may not attempt to intercept. A stack
that is already engaged in battle (to be resolved after all moves) may not
attempt interception.

Each non-phasing stack may attempt to intercept each moving stack only once
during the whole move (not once per zone). If a moving stack drops and picks
up units, it is considered as a single moving stack even if all the units
actually composing it changed.

Interceptions are resolved by a modified die roll. These rolls are similar on
land and at sea but with different modifiers. \Man plays a huge role in these.

\subsection{Land}
Non-phasing stacks of countries who made no campaign at all during their
previous impulse (except during the first round if they had no impulse yet)
may not intercept at all. Non-phasing stacks commanded by countries who paid
only a passive campaign during their previous impulse (including, during the
first round of each turn, stacks that had no impulse yet) can only intercept
in a friendly province (no enemy presence, even besieged).

A non-phasing stack which is in the same province as a non-besieged,
non-moving phasing stack is engaged into battle and cannot intercept. Thus,
whenever a moving stack enters a province, any non-phasing stack here may (i)
attempt to intercept and fight immediately, in case of victory it will be able
to intercept again this impulse but in case of defeat the moving stack may
continue its move; or (ii) do nothing, this \emph{de facto} locks both the
moving and non-phasing stack in a battle (to be resolve later) but prevent
further interception from that stack, moreover, the phasing alliance may now
move more troops in the province before the battle occurs.

After one or more successful interceptions, before the interception battle is
resolved, phasing stacks may attempt to counter-intercept. Stacks already
``locked'' in battle may not counter-intercept. Counter-interception is
resolved in the same way as interception. It is not possible to
counter-counter-intercept. That is, first non-phasing players declare and
resolve all interceptions, next phasing players declare and resolve all
counter-interceptions and finally the battle is resolved. As any interception,
counter-interceptions are free (they do not count toward campaign cost).

Before resolving the interception battle, check if attrition of the moving
stack is triggered and resolve it if any. Any \terme{foraging} result apply
for the interception battle. Intercepting and counter-intercepting stacks do
not roll for attrition.

The interception battle normally occurs between all the successful
interceptors versus the moving stack and all the successful
counter-interception. If one side contains troops of an European country and
more than 8\LD+2 \Pashas, exceeding troops do not fight (but stays in
place). After the battle, if there is still too many troops in the province
(more than the stacking limit), any intercepting or counter-intercepting troop
may choose (controller's choice) to retreat in the province where it was
before intercepting. This retreat does not trigger attrition and does not
counted toward any campaign cost. Any exceeding troops after that are
destroyed. Such an overstacking typically occurs when many unlikely
interceptions are declared and they all succeed\ldots

If a stack moves into a province where there are already forces (whether
friendly or enemy) and get intercepted, then the interceptor can choose to
either resolve this immediately as an interception battle, hence only between
the moving and intercepting troops; or to resolve this later as a regular
battle, hence merging all the troops in the province prior to battle and
``locking'' everybody into battle (but the phasing side may decide to bring
more troops).

To resolve an interception, roll a die modified as follows (all modifiers are
cumulative):
\begin{modlist}
\item[\textplusminus?] \Man differential between the commanding leaders
  (intercepting stack-intercepted stack); in case of counter interception,
  consider that all the interceptors are merged in a single stack before
  finding the commanding leader.
\item[+1] If the intercepting stack has a technology which is at least 6
  levels above the technology of the intercepted stack.
\item[+1] If the target province contains a friendly force (other than the
  intercepting one) or city (even besieged).
\item[+1] If intercepting in the same province.
\item[-1] If the target province contains swamps.
\item[-1] If the intercepting stack is in a swamp province or a flooded
  province.
\item[-2] If there is \terme{Bad weather}.
\item[-2] If the interception occurs across a river or mountain pass.
\item[-2] If the target province contains an unbesieged enemy force other than
  the intercepted one.
\item[-1] If the interceptor is besieging.
\end{modlist}

If the unmodified die is 10, or if the modified roll is 8 or more, then the
interception is successful.

\subsection{Sea}
Non-phasing stacks of countries who made no campaign at all during their
previous impulse (except during the first round if they had no impulse yet)
may not intercept at all. Countries who paid only a passive campaign during
their previous impulse (including, during the first round of each turn, stacks
that had no impulse yet) may only intercept with stacks containing at most 1
\FLEET counter.

Non-phasing stacks engaged into battle may not intercept. It is not possible
to intercept in a port/arsenal (only in a sea zone). It is not possible to
counter-intercept. Non-phasing stacks that already lost a battle this impulse
(\emph{i.e.} losing an interception battle) may not intercept anymore.

The correct timing for moving and intercepting is:
\begin{enumerate}
\item moving stack enters a zone;
\item moving stack either declares an action in the zone or decline to do one;
\item if the moving stack declared an action, it rolls immediately for
  attrition;
\item non-phasing stacks declare interceptions;
\item interceptions are resolved;
\item interception battle is resolved, if any;
\item
  \begin{itemize}
  \item if the moving stack is victorious, it may either do its action (if it
    declared any) or continue moving (if it declared no action);
  \item if the moving stack is defeated, it must retreat; additionally, if it
    did not roll for attrition (because it declared no action), the distance
    of the whole movement is added to the distance of the whole retreat for
    the attrition of the retreat.
  \end{itemize}
\end{enumerate}

Note that it is possible to try and intercept a moving stack that just
declared it will also battle. If a non-phasing stack allied to the attacked
stack succeed in interception, it may decide to either resolve this
immediately as an interception battle or to merge with the attacked stack and
resolve this later as a regular battle. If a non-phasing stack not allied with
the attacked stack succeed in interception, this must be resolved as an
interception battle.

Several non-phasing stacks may attempt interception. As on land, only one
alliance may actually succeed. As on land, the stacking limit is only enforced
after the battle and an optional retreat of exceeding forces, but exceeding
forces do not take part in the battle.

As on land, interception occurs as soon as the stack enters the sea zone,
hence before it can merge with other forces here.

If the moving stack attempts to avoid a blockade and roll 5 or less (``Forced
battle''), then the blockading stack may declare an interception which
automatically succeeds. It must still be allowed to intercept (\emph{i.e.} not
already engaged into battle), this simply remove the need for rolling.

Unless in case of automatic success, interception is resolved by a die roll
modified as follows. Cumulative modifiers:
\begin{modlist}
\item[\textplusminus?] \Man differential between the commanding leaders
  (intercepting stack-intercepted stack).
\item[+1] If the intercepting stack has a technology which is at least 6
  levels above the technology of the intercepted stack.
\item[-2] If there is \terme{Bad weather}.
\item[-1] If the interceptor is blockading.
\end{modlist}
Non-cumulative modifiers:
\begin{modlist}
\item[+1] If the intercepting stack has technology \TTD or better.
\item[+1] If intercepting in the same sea zone.
\item[+2] If the moving stack announced a landing either (i) from the same sea
  zone as the one where the intercepting stack is; or (ii) in the province
  where the intercepting stack is stationed at port; or (iii) from a sea zone
  adjacent to the arsenal where the intercepting stack is stationed.
\item[-3] If the interceptor is in a port (not arsenal).
\end{modlist}

Note that the \bonus{+2} in the non-cumulative list is the very reason why
actions must be declared upon entering zone (and before
interceptions). Landing close to an intercepting stack can provide huge bonus
to interception (it is assumed that the battle actually occurs while the
moving stack is closing the cost and beginning the preparation for landing,
hence it is much easier to find).

If the modified roll is 8 or more, or if the unmodified roll is 10, the
interception succeeds.

\begin{exemple}
  A naval stack at port in \provinceCornwall may intercept in either
  \seazoneIrlande, \seazoneCeltique or \seazoneManche, all adjacent to the
  province. It will, however suffer a malus of \bonus{-3} to the roll
  representing the time it take to actually get out of port (or ports) and
  regroup for battle. As long as it is at port, it is also protected from
  enemy attacks (they may not enter the port to fight) but is likely to be
  blockaded.

  If an enemy stack declares a landing on \provinceCornwall, where the stack
  is located, interception becomes much easier since the enemy stack is
  basically staying very close to the actual costs of the province and cannot
  really try to avoid contact while still attempting to disembark troops\ldots
  Not only the \bonus{-3} disappears, but it is replaced by a \bonus{+2}.

  \smallskip

  The naval stack may decide to cruise in \seazoneCeltique rather than stay at
  port in \provinceCornwall. In this case it will have to roll for attrition
  every round (for staying at sea) and is likely to be attacked by an enemy
  naval stack. However, it will not only be allowed to intercept in
  \seazoneIrlande, \seazoneCeltique and \seazoneManche but also in
  \seazoneGascogne and \seazoneRockall. Moreover, its interceptions no longer
  have the \bonus{-3} malus for being at port, and may even get a \bonus{+1}
  for the same sea zone if intercepting in \seazoneCeltique.

  If an enemy stack attempts a landing on \provinceCornwall (or another) from
  \seazoneCeltique, this \bonus{+1} is replaced by \bonus{+2}, not only the
  stacks are closed but the enemy is landing. However, if the enemy attempts a
  landing on \provinceCornwall from \seazoneManche the \bonus{+1} is lost (not
  the same sea zone) for good (it is assumed that the stack more or less
  patrols and the enemy has more possibilities to snick in unnoticed).

  \smallskip

  If a stack is stationed in the \villePortsmouth arsenal in
  \province{Wessex}, it can only intercept in \seazoneManche. However, it has
  no malus for interception as arsenals are assumed to be designed as military
  ports built for quick reaction with small unrepresented flotillas running
  around. Even better, if an enemy attempt to disembark \emph{anywhere} from
  \seazoneManche (\emph{e.g.} \provinceWessex, \provinceNormandie,
  \provincePicardie or \provinceKent from a naval stack located in
  \seazoneManche), then the stack in \villePortsmouth gets the \bonus{+2}
  bonus for interception. The arsenal is designed to allow any fleet in it
  extend its control into the neighbouring waters.

  \smallskip

  Now, you should be able to build a ``\emph{Fleet in being}'' strategy.
\end{exemple}

\subsection{\Presidios and \StraitFort}
A \Presidio or \StraitFort can intercept a naval stack. \StraitFort can
intercept stacks moving through it. \Presidios can intercept stack entering or
exiting the port. The interception is decided by the controller of the
fortress.

Although it it an interception, it is resolved differently than usual ones and
does not result in battle. It may, however, stop the moving stack and cause
some automatic loses on it.

The interception is resolved by rolling 1d10 modified by
\begin{modlist}
\item[+?] level of the \Presidio;
\item[+2] \StraitFort on the European map;
\item[+?] half the level (round down) of the fortress controlling the
  \StraitFort on the \ROTW map;
\item[+1] if the moving stack contains at least one \FLEET counter.
\end{modlist}

If the result is 9 or more, then the moving stack must stop moving immediately
(before crossing the Strait or \Presidio) and roll for attrition as usual. It
may not attempt any action this round.

Additionally, if the result is 11 or more, the stack loses 1\ND (or 1 side of
\corsaire) and if the result is 13 or more, the stack loses 1 more \ND (or
side of \corsaire) (for a total of 2).

If a naval stack carrying troops is stopped by a \Presidio or \StraitFort, it
must go back to the closest (in number of zones) friendly port large enough to
hold it, ignoring ports that are behind a \Presidio or \StraitFort. This is a
move, hence a cause of attrition, and may be intercepted.

\subsection{Convoys}
\begin{designnote}
  Do not use if using the experimental system for Revolts, \corsaire and
  Natives.
\end{designnote}

Naval stacks carrying gold, most notably convoys, may be intercepted and
attacked by \corsaire when they enter a sea zone of the \STZ or \CTZ in which
the \corsaire is acting. Note that this does allows interception by a
\corsaire which is \textbf{not} adjacent to the entered sea zone (\emph{e.g.}
a \corsaire in \seazone{Petites Antilles}, acting in \stz{Caraibes}, may
attack a convoy entering \seazoneMexique; it is assumed that the \corsaire
acts in the whole \STZ and its location represents its main ports rather than
its actual zone of activity). These attacks never stop the movement of the
stack.

In order to attack the \corsaire must be allowed to attack the owner of the
convoy and must be \Faceplus. \corsaire of \Barbaresques may only attack this
way stacks that are in the \region{Mediterranee}. Several \corsaire may attack
the same convoy in the same zone, but only one per alliance. These attacks are
resolved after the moving stack declared an action (or declined to do one) but
before regular interceptions are declared. \pays{pirates} \corsaire always
attack (if \Faceplus) and their attacks are always resolved first, then any
other in decreasing order of initiative; each attack being resolved before the
next is declared. If several \corsaire of \pays{pirates} may attack, only the
one commanded by the highest ranking \LeaderA does (at random in case of
ties). If several \corsaire of an alliance may attack but their controller do
not reach an agreement on who should do the attack, none does. \corsaire of
Neutral minor countries are considered as controlled by the first country in
their list of controller.

\aparag[Attack Procedure]
\bparag Roll for naval interception by the \corsaire. \corsaire of
\pays{pirates} with no \LeaderA use 2 as \Man.
\bparag If the interception fails, the attack stops immediately; if it is
successful, reduce the \corsaire to \Facemoins and proceed further.
\bparag The attacked stack may roll on~\ref{table:Reduce Revolt or Piracy}
(see \ref{chMilitary:Revolt} for details); if successful, do not reduce
further the \corsaire but stop the attack immediately; if it fails, proceed
further.
\bparag Resolve the attack by rolling on~\ref{table:Pirates Natives Raids}
(see \ref{chRedep:Corsair Attack} for details); read the result in column
``\TradeFLEET\faceplus'', the number corresponds to the number of \NTD
captured (each with 15\ducats).
\bparag Gold captured by \corsaire of minor countries is lost; gold captured by
\corsaire of major countries is immediately added to
\lignebudgetlong{Pillages, privateers} of the controller (\emph{i.e.} the gold
is ``teleported'' in the \RT and may not be intercepted again).
\bparag After the attack, \corsaire of \pays{pirates} stay in the sea zone and
may act again. Any other \corsaire must return to port immediately (as if they
lost an interception battle) and may not act again this turn.

\section{Explorations}\label{chMilitary:Discoveries}
\label{chMilitary:Exploration}
Each phasing naval stack not engaged in battle (including interception) may
attempt to explore an adjacent unknown seazone. In case of success, the stack
automatically moves into the explored zone. Exploration is the action of the
naval stack.

Next, each phasing land stack not engaged in battle (including interception)
and not carried by a naval stack engaged in battle (including interception)
may attempt to explore an adjacent unknown province. In case of success, the
stack automatically moves into the explored province.

Note that a naval stack embarking troops may explore a new seazone from which
the troops may then disembark to explore a new province.

Any stack may attempt an exploration, but the presence of a \LeaderE, \LeaderC
or \LeaderGov will greatly reduce the risks encountered. A leader alone may
also explore but the presence of troops greatly remove the risks of death
during the exploration.

Exploration is not a movement and thus may not be intercepted and is never
subject to attrition (but severe loses are usually suffered in the
process). Notably, a landing during an exploration is never a cause for
attrition.

Exploration is resolved by rolling on~\ref{table:Discoveries and Attrition},
the result is read \textbf{both} in the ``Discovery'' column corresponding to
the type of exploration (land or sea) \textbf{and} the ``\ROTW or sea''
column. The roll is modified by:
\begin{modlist}
\item[+4] for all explorations;
\item[-M] \Man of the commanding \LeaderE, \LeaderC or \LeaderGov
  \textbf{only}; or \Man of a \LeaderMis in the stack (see below).
\end{modlist}

For exploration at sea, the roll is further modified by:
\begin{modlist}
\item[+X] when exploring a sea zone with malus, add the malus;
\item[+1] if the exploring stack has technology \TCAR;
\item[-1] if the exploring stack has technology \TBAT;
\item[-2] if the exploring stack has technology \TVE or \TTD;
\item[-3] if the exploring stack has technology \TSF;
\item[+2] if there is \terme{Bad weather};
\item[-2] if this is period \period{IV} or later and the sea zone is already
  known by at least one other country (including minor countries).
\end{modlist}

% dirty trick
\newcommand{\myxxa}{\phantom{a}\xxa\xspace}
\newcommand{\myxxc}{\phantom{a}\xxc\xspace}

Note that \LeaderG or \LeaderA do not use their \Man during exploration (nor
during a possible further test in case of\myxxa or\myxxc). On the
other hand, if a \LeaderMis is present in the stack (even if not commanding)
its \Man may be used instead of the one of the commanding leader.

\LeaderE may explore coastal provinces (with or without troops) and stay in
the province afterwards; they may never go further inland but may explore a
coastal province from an adjacent coastal province (they don't need to
reembark, this represents moves along the coast with smaller ships). On land,
all their values (especially \Man) are halved (round down).

The result of the exploration is read \textbf{both} in the ``Discovery''
column \textbf{and} the ``\ROTW or Sea'' column. The later gives a percentage
of loses that is applied to the stack as if it suffered it from
attrition.

The ``Discovery'' column contains either a 'S', a 'F' or a '\undemi'; plus
sometimes a\myxxa or\myxxc.
\begin{itemize}
\item 'S' means Success. The exploration succeed. The stack is moved in the
  newly explored province or sea zone. Exception: if the explored province
  contains an establishment of a neutral country, the stack stays in
  place. The discovery is noted on the discovery sheet of the player but
  cannot be used until being ``brought back home'' (see below).
\item 'F' means Failure. The stack does not move and the zone is still
  unknown.
\item '\undemi' are resolved by rolling an unmodified die on the small table
  on the right. Treat results of '1--3' as F\xxa; '4--5' as F\xxc; '6--8'
  as S\xxa; and '9--10' as S\xxc.
\item If there is a '\xxa', or if there is a '\xxc' and the exploration is
  done by a leader alone; then the leader (whose \Man was used (either the
  commanding one or a \LeaderMis in the stack)) must test his death. Roll
  1d10, if this is strictly more than the \Man (halved for \LeaderE on land as
  usual), the leader is dead (remove him from the game); otherwise, he
  survives. Note that \LeaderA or \LeaderG having a \Man of 0 for exploration,
  they automatically fail this test (and die) if they explore\ldots
\end{itemize}

If a leader die during exploration and the country falls below its limit of
\LeaderC or \LeaderE, the new one may be put in the same place as the old one
only if some counters are still present (a replacement \LeaderA, \LeaderG or
\LeaderGov must be placed following usual rules). Otherwise, it is assumed
that the whole expedition has been whipped out and the discoveries it may had
are lost\ldots

\smallskip

When a discovery is successful, only the stack that made it knows about it (in
case of a multi-national stack, only the commanding country (and its units)
carry this knowledge). This means that other stacks of the same country still
consider that the zone is unknown for all purposes. Typically, they must
explore again if they want to enter that zone. Even if two stacks know the
zone (but the country doesn't), they may not merge in it.

As soon as a stack carrying discoveries reaches an establishment (fort, \TP,
\COL or European province) that was existing at the beginning of the military
phase, the countries immediately and definitely gains knowledge of the
discoveries. Any stack of that country also gets that knowledge as soon as it
is not itself carrying unknown discoveries (\emph{i.e.} either it is wandering
in known land, possibly enemy or empty, or it has finished its own
explorations and reached home again).

A stack carrying discoveries may ``meet'' another stack if this happen in a
zone whose discovery has been brought back \textbf{and} both stacks have
knowledge of this zone. In that case, if both stacks are of the same country
(not alliance), they share their discoveries: they may both use the zones
discovered by the other and any of them is able to bring all these back
home. Discoveries are never shared that way with another country.

If a stack carrying discoveries is entirely destroyed, its discoveries are
lost (unless they've already been shared with another stack).

Discoveries carried by stack must be noted and remembered. Once a discovery is
brought back home, shade out the corresponding zone on the mini-map of the
player to remember this.

Note that bringing the discoveries home is not a naval action and happens
automatically as soon as the stack enters an establishment. Thus, typically,
when exploring seas around one of your \COL, it is possible to move in the
port of the \COL (bringing discovery of previous round), move out and a bit
further and then as an action explore an adjacent sea. This makes the move a
bit longer (thus increasing attrition risk) but greatly reduce the risk of the
stack being sunk with many discoveries (something you definitely do not want
to happen).

If two stacks (allied, enemy or neutral) are in the same zone and both of them
are carrying the discovery without it being brought back home yet, they may
not interact (fight, merge, \ldots) If one of the stack is from a country who
already knows of this zone, then this stack may initiate any interaction
(battle, interception, \ldots) following regular rules.

When a land stack enters a province by exploration, test for activation of
natives as if this was a movement (including the possibility to use the
Conquistador table and to voluntarily attack the natives).

\aparag[Diffusion of discoveries]
\bparag From period \period{IV} onward, discoveries spread among all players.
\bparag[Atlantic] At the beginning off turn 25 (period \period{IV}), all sea
zones without malus in the \region{Atlantic} that are known by at least one
country (including minor) are immediately known by everybody; from now on, any
newly explored sea zone without malus from the \region{Atlantic} is known by
everybody at the beginning of the turn following its discovery by anybody.
\bparag[On sea] If this is period \period{IV} or later, any sea zone that is
known by at least one country (including minor) has a bonus of {\bf -2} for
discoveries by other countries
\bparag [On land] At the beginning of period \period{IV}, all provinces
containing a \COL or \TP become known by everybody.
\bparag [On land] If this is period \period{IV} or later, any province
containing a \COL or \TP become known by everybody at the beginning of the
turn following the one were the establishment was created.
\bparag[Level 6] \COL of level 6 are European provinces and thus known by
everybody. As soon as a \COL reach level 6 (even during period \period{I}),
its province becomes immediately known by everybody.

\section{Battles}
\label{chMilitary:Battles}
\subsection{Generalities}
Resolve all non-interception battles caused by the movement, in order of
choice of the phasing alliance (at random in case of disagreement). Each
battle must be fully resolve before the next one starts.

Battles caused by interceptions are resolved immediately (during movement) but
follow the same procedure as this one.

Land and sea battles follow the same sequence with slight modifications.

Battles are always only fought between two stacks of belligerent
alliances. There cannot be a three-sided battle. If, for some reason, several
stacks of the same alliance participate in a battle, they are automatically
considered merged before the battle starts (notably to find the commander of
the stack) and for its entire duration.

% \textbf{Attacker and defender.} The tables use the terms \terme{attacker} and
% \terme{defender} to design the sides. They are determined as follows: if this
% is an interception battle, then the attacker is the non-phasing alliance and
% the defender is the phasing alliance. If this is not an interception battle,
% then the attacker is the phasing alliance and the defender is the non-phasing
% alliance. Attacker and defender mostly play a role for terrain modifiers. We
% try to avoid these terms in the rules.
% \begin{todo}
%   And to remove them from the table\ldots
% \end{todo}

For the sack of clarity, commanding leaders are always referred as \LeaderG or
\LeaderA even if (in the \ROTW), they may also be \LeaderC, \LeaderGov or
\LeaderE. If a \LeaderE happens to fight on land (typically activating the
natives after an exploration), each of its values is halved (rounded
down). Similarly, \terme{Shock} is often used instead of \terme{Boarding} for
sea battles.

\subsubsection{Battle sequence}
\aparag Each battle follows this sequence of steps:
\bparag \xnameref{chMilitary:Battle:Evasion}: non-phasing, non-intercepting
stacks may attempt to avoid the battle and retreat before it starts.
\bparag \xnameref{chMilitary:Battle:Parameters}: determine battle parameters,
notably the columns of the CRT and DRM to use.
\bparag \xnameref{chMilitary:Battle:Fight}: each battle can last up to 2
\terme{days}. Each day is further split in a \terme{Fire} and a \terme{Shock}
(or \terme{Boarding} at sea) roll.
\bparag \xnameref{chMilitary:Battle:Pursuit}: the winner (if any) pursues the
loser.
\bparag \xnameref{chMilitary:Battle:Loss modifications}: actual losses are
computed from the battle result.
\bparag \xnameref{chMilitary:Battle:Retreat}: any non-winning stack retreats.
\bparag \xnameref{chMilitary:Battle:Cleanup}: determine major victory, test
death of leaders, \ldots

This Battle sequence, and its details explained below, must be followed in
strict order. Changing the order of some steps will result in incorrect
results that can have drastic influence. Especially, because of the threshold
on some effects (\emph{e.g.} 'Major Battle'), a small change can have a big
consequence.

\subsection{Evasion test}
\label{chMilitary:Battle:Evasion}
A non-phasing, non-intercepting stack engaged in battle may attempt to evade
the battle and retreat before it starts. Evasion is decided by the controller
of the stack at the time of the battle. Failing an evasion test does not
entail any penalty.

The evasion test is made by rolling 1d10 and adding the \Man differential
(non-phasing - phasing) of the leaders, \textbf{only if positive}. If the
result is 8 or more, the evasion is successful.

If the evasion is successful, the 'battle' is over. Resolve the evasion and
then proceed with~\xnameref{chMilitary:Battle:Cleanup:Aftermath}.

\textbf{At sea}, the evading stack must go to the closest (in number of zones)
friendly port large enough to hold it (controller's choice if any). This is a
move, hence it must roll for attrition. If it's large enough, the phasing
stack may follow and blockade said port (only). If it choose to follow and
blockade, this is also a move, hence the phasing stack must also roll for
attrition.

\textbf{Automatic success.} If a land stack is in a province with a friendly
fortress large enough to hold it (1 level/\LD, including those in \Pashas),
then it may retreat in the fortress with no test. It may choose to roll to try
to retreat in another province but doing so forfeit the possibility to
automatically retreat in the fortress of the province with no test.

\textbf{On land}, the stack may roll for evasion. If successful, it must move
into an adjacent friendly unbesieged province (\emph{i.e.} a province where it
could go with a passive campaign). It is not possible to evade battle via
naval transport. If the province where the battle occurs is friendly, it is
possible to retreat part of the stack in its fortress and another part in an
adjacent province (but this does require a successful evasion roll). It is not
possible to split the stack in any other way. It is possible to split even if
the whole stack could fit in the fortress.

Land stacks evading do not roll for attrition.

\subsection{Battle parameters}
\label{chMilitary:Battle:Parameters}
\subsubsection{Sortie}
If this is a land battle in a province with besieged troops friendly to one
side, the troops may decide to attempt a sortie and join the battle. In this
case, they are considered merged into a single stack for the battle (notably
to determine commanders). It must, however, be remembered which troops come
from the sortie as they may have to retreat back in the fortress.

\subsubsection{Replacements leaders}
If a stack engaged in battle has no leader (the commanding country has no
leader in the stack), a replacement leader is rolled.

\GTtable{replacement}

Cross-reference 1d10 (never modified) with the commanding country
in~\ref{table:Replacement leaders}. The three values correspond to the \Man,
\Fire and \Shock of the leader. These values will be used for the whole
duration of the battle.

Additionally, when rolling for a \POL \LeaderA, subtract 1 from \Man and when
rolling for a \PRU \LeaderA, subtract 1 from \Fire.

Note that Replacement leaders are rolled after evasion and hence cannot be
used to evade battle.

If the stack is commanded by the \TUR Vizier, and if the Vizier was already
rolled for an evasion test, do not roll him again but use the same values.

\subsubsection{First line ships and Wind Advantage}
This only happens for sea battles.

If a side is composed of both \NGD and \NWD, it must choose which kind of
ships will go in battle. The other kind does not take part in the battle. It
will, however, retreat or pursue with the stack if needed. It may also suffer
loses if the loses are more than the number of first line ships. Any \VGD are
automatically used if (and only if) the first line ships are \NGD.

If a side is composed solely of \NTD (notably, convoys alone), it must use
them as first line ships. This is the only case where \NTD may be used us
first line ships. In this case, the opponent automatically gets \terme{Wind
  Advantage}, \NTD do not roll during the battle and they automatically break
and rout after the First Fire (they have 0 Morale).

\begin{designnote}
  While mixed fleet did exists, the battle mechanisms (and parameters) are
  handled differently in game for galleys and warships. Hence, mixing them in
  battle would complicated things a lot\ldots Moreover, most of the mixed
  fleet were using mainly one kind of ships (the other kind being in much
  smaller quantities and often mostly used as support ship). This is also the
  reason why in game \NGD are only allowed in the \region{Mediterranee} and
  the \region{Baltique}: even if Galleys were also used elsewhere, it was
  mostly as second line ships and North Sea fleets (for example) were composed
  mostly of warships plus some galleys (that are thus represented as part of
  the counters).

  \smallskip

  Note that if you have 1\NWD and 10\NGD, you may be tempted to put the \NWD
  on first line to try and save the \NGD\ldots However, if you suffer more
  than 1 loss, the extra will carry onto the \NGD.
\end{designnote}

Next, in a sea battle where at least one stack is not using \NGD as first line
ships, roll for \terme{Wind Advantage}.

\GTtable{windadvantageonly}

First, each side cross-reference its technology (line) with the technology of
its opponent (column) on~\ref{table:Wind Advantage Determination} to find a
DRM ('-' means no DRM). Only one side has a DRM for \terme{Wind Advantage}.
(no, \TBAT has no DRM against \TGF, it already has +1 morale and that's a
large enough bonus\ldots)

Next, each side rolls 1d10 and adds the \Man of its commanding \LeaderA. The
side with a technological DRM adds it to its result.

The side with the highest modified result has \terme{Wind Advantage} for the
whole duration of the battle. In case of tie (or in a battle where both side
use \NGD), no side has \terme{Wind Advantage}.

\subsubsection{Morale and CRT column}
\label{chMilitary:Battle:Parameters:CRT}
Use either~\ref{table:Naval Fire Shock} or~\ref{table:Land Fire Shock} to
determine morale of both sides and the columns of the CRT it will use during
the whole battle.

\GTtable{navaltech}

\GTtable{landtech}

First, each side finds the line corresponding to its technology and reads in
the 'Morale' column its morale (between 1 and 4). \terme{Veteran} stacks
(see~\ref{chMilitary:Veteran Conscripts}) add 1 to this value. Stacks with
\TTER technology add 1 until \TARQ included (cumulative).

A stack with \TREN technology has morale 1 if the enemy has \TMED technology
and 2 otherwise (+1 if \terme{Veteran}, as usual).

\begin{designnote}
  \TREN already provides the possibility to fire. Together with an extra
  morale point, this would give them a devastating advantage against \TMED
  stacks. This way, the fire step already gives \TREN vs \TMED a big advantage
  to the \TREN stack but the possibility of surprise still exists\ldots
\end{designnote}

Next, determine the CRT columns to use for the battle. Each side will use
possibly different columns. Column \textbf{A} is better than Column \textbf{B}
and so forth.

Each side cross-references its technology (line) with the technology of the
opponent (column) and find two letters. The first is the \terme{Fire CRT
  column}, the second is the \terme{Shock} (or \terme{Boarding}) \terme{CRT
  column}. The bold letters in the table correspond to the symmetrical
situation where both sides have the same technology (which happens most of the
time).

Stacks with \TMED technology have only one letter in this table. They never
roll for fire and thus only have a \terme{Shock CRT column}.

\subsubsection{Die rolls modifiers}
\label{chMilitary:Battle:Parameters:DRM}
Compute DRMs for the battle. Note these DRMs for easy use later on. Unless
explicitly stated otherwise, all these DRMs are cumulative.

All the modifiers here are computed at the beginning of the battle and are
valid for its entire duration, even if the conditions to use them are no more
met (because the stack suffered from loses, typically). The two modifiers that
appear mid-battle are listed in~\ref{chMilitary:Battle:Fight} (they are:
morale modifier on sea and failed retreat modifier on land).

\begin{playtip}
  It is advised to prepare a sheet of blank paper to note the battle results
  as it unfolds. Make two columns (one for each side). On the top of each
  column, write the DRMs this side will have for the 4 rolls of the
  battle. Use the rest of the column to write the battle results themselves.

  Modifiers are usually very similar for both days of battle. Hence, it is
  usually easier to only compute modifiers for the first day and update them
  at the end of the first day. The four modifiers that are different from one
  day to the other are the 'second day' modifier, the 'foraging' modifier as
  well as the two 'crossing' terrain modifiers.
\end{playtip}

Modifiers that do not change between days are listed here in a 'F/S' format
where F is applied to both Fire rolls and S to both Shock rolls. Modifiers
that do change between days are listed in a 'F1/S1//F2/S2' format.

\aparag[General modifiers.] Apply these modifiers during any battle.
\begin{modlist}
\item[+F/0] \Fire differential. The side with the greatest \Fire value (for its
  leader) adds the differential between \Fire of the leaders to all its Fire
  rolls. [TBD: max +3]
\item[0/+S] \Shock differential. The side with the greatest \Shock value (for
  its leader) adds the differential between \Shock of the leaders to all its
  Shock rolls. [TBD: max +3]
\item[0/0//-1/-1] always (second day malus).
\item[0/-2] if 'Facing the Ottomans' (see~\ref{chTurkey:Facing Ottomans}).
\end{modlist}

\aparag[Sea technology modifiers.] Apply these during any sea battle,
depending on the technologies of the participants and the kind of \terme{first
  line ships} used.
\begin{modlist}
\item[+1/0] for a side that has \VGD in its stack if the other side uses \NGD
  as \terme{first line ships}. Before \TBAT, a side needs two \VGD to gain
  this modifier; starting with \TBAT, all \NGD stacks are considered to
  contain \VGD. If both side have \VGD in their stack, they may both have
  this bonus.
\item[+1/+1] for a side that uses \NGD, if both (i) the battle occurs in the
  \region{Mediterranee}; (ii) the other side uses \NWD or \NTD and (iii) this
  is turn 25 or earlier.
\item[-1/-1] for a side that uses \NGD if both (i) the other side is using
  \NWD and (ii) this is turn 35 or later.
\end{modlist}

\aparag[Other sea modifiers.] Apply these during any sea battle.
\begin{modlist}
\item[+1/+1] for the side with \terme{Wind Advantage}.
\item[0/+1] for a side that has at least 1 more \ND than the other side.
\item[0/+1] for a side that has at least 7 more \ND than the other side
  (for a total of 0/+2 when combined with previous modifier).
\item[+1/0] for a side that has at least 3 more \ND than the other side.
\end{modlist}

Remark: The ``5 more \ND'' tier will provide a bonus to \terme{Pursuit}.

Remark: count all the \ND used as \terme{first line ship}. Notably, each \NGD
counts as 1\ND. Because there are usually much more \NGD than \NWD in a stack
(the stacking limits being twice as big), \NGD vs \NWD often get these
modifiers. Take that into account when trying to figure out the best time to
switch from an \NGD navy to a \NWD one\ldots

\aparag[Other land modifiers.] Apply these during any land battle.
\begin{modlist}
\item[-1/-1//0/0] for a side that is \terme{foraging}.
\item[0/-1] for a side that is \textbf{not} \CAI, \CAIM, \CAII or
  \CAIIM, if the enemy has \TTER.
\end{modlist}

\aparag[Terrain modifiers.] Apply these during any land battle.
\begin{modlist}
\item[0/0] if the battle occurs in a Plain province.
\item[-1/-1] if the battle occurs in a Forest (any kind), Swamp or
  Desert province.
\item[-1/-1] for the phasing side if the battle occurs in a Mountain
  province \textbf{and} this is not an interception battle.
\item[0/0] if the battle occurs in a Mountain province, for everybody but
  a phasing, non-intercepted side.
\item[-1/-1//0/0] for the phasing side, it if has crossed a river or a
  mountain pass to enter the province (\textbf{including} when being
  intercepted immediately after said crossing).
\item[-2/-3//0/0] for the phasing side, if it has crossed a strait or used
  naval transport to enter the province, \textbf{including} when being
  intercepted, including in a previously friendly province (hence including
  when being intercepted at the end of a friendly naval transport).
\end{modlist}

\aparag[Artillery modifiers.] Apply these in any land battle.
\begin{modlist}
\item[-1/0] for a stack with no \ARMY counter. Exception: during periods
  \period{I} to \period{IV}, a stack commanded by a \LeaderC (not a \LeaderGov
  or a \LeaderE) does not suffer from this malus. This malus does apply for
  \ROTW countries; it also applies for natives with less than 2\LD (natives
  with 2\LD are automatically merged into an \ARMY\Facemoins).
\item[+1/0] for a stack that contains at least 6 Artillery
  (check~\ref{chMilitary:Stacks:Artillery} for how to compute Artillery in a
  stack).
\end{modlist}

\aparag[Size Cavalry modifier.] Apply this in any land battle.
\begin{modlist}
\item[0/+1] for a stack with at least 3\LD more than its opponent.
\end{modlist}

\aparag[Structural Cavalry modifiers.] During any land battle, any stack gain
a 0/+1 DRM if both (i) it contains an \ARMY counter of the indicated country
or army class; and (ii) this is the indicated periods; and (iii) the battle
occurs in the indicated terrain.
\begin{modlist}
\item[\CAIIM] during periods \period{I} to \period{IV}, in a Plain or Sparse
  Forest province.
\item[\CAIIIM] during periods \period{IV} or \period{V}, in a Plain or Dense
  Forest province.
\item[\CAIV] during periods \period{III} to \period{V}, in a Plain province.
\item[\SUE] during periods \period{III} to \period{VI}, in a Plain or Northern
  Forest province.
\item[\TUR] before reform M-2, in a Plain or Desert province (\terme{Sipahi}
  cavalry).
\end{modlist}
Whatever the conditions, this bonus may only be claimed once by each side of
the battle.

Note that since Northern Forest provinces are necessarily Dense Forest and
\SUE is of class \CAIIIM, the specific \SUE bonus is actually only effective
during periods \period{III} and \period{VI}.

\begin{exemple}[Battle parameters]
  This is late period \period{III} and \HIS tries to helps \pays{sainte-ligue}
  put the Duke of Guise on the French throne. In the Northern French plains
  (\provinceArtois), 1\ARMY\faceplus and 1\LD (total of 5\LD) of
  \terme{Veteran} Spanish Tercios (technologies \TMUS and \TTER) encounter
  2\ARMY\faceplus (total of 8\LD) of \terme{Veteran} French troops (technology
  \TMUS) coming from \province{Ile-de-France}.

  First, Morale and CRT columns are computed. With \TMUS, each side has a
  Morale of 3; being \terme{Veteran} each side has a bonus of \bonus{+1} to
  Morale; Hence the Morale is 4 (2+1+1) for both.

  Remark: \TTER gives no Morale bonus in \TMUS. However, with \TARQ\ \FRA
  would have a Morale of 3 (2+1) and \HIS of 4 (2+1+1).

  Cross-referencing the \TMUS line with the \TMUS column, both sides find a
  \textbf{C/B} for the columns to use during this battle. If each side has a
  different technology, the columns will be different, here with the same
  technology the same result will obviously be read twice.

  Next, the DRMs are computed. First a comparison is made between the \LeaderG
  of each stack. Let's suppose that the French troops are lead by a 242
  \LeaderG while the Spanish are lead by a 423. This result if +2 to Fire for
  \FRA and +1 to Shock for \HIS. The terrain is Plain and no river or
  mountain pass was crossed, hence no terrain modifier.

  \HIS has only 1\ARMY\Faceplus, hence little Artillery. On the other hand,
  \FRA may have enough\ldots In period \period{III}, \FRA has 3 Artillery per
  \ARMY\Faceplus. However, when computing Artillery for a stack, the second
  (and subsequent) \ARMY may not add more than 2, hence the French stack has
  only 5 Artillery which is not enough to get a bonus. Each side has an \ARMY
  counter, so no malus either.

  \FRA is of class \CAIV. In period \period{III}, with a least one \ARMY
  counter in the stack and in Plain, this gives a \bonus{+1} cavalry
  bonus. \FRA has 3 more \LD than \HIS and thus gets another \bonus{+1}
  cavalry bonus.

  \HIS is \TTER, this gives a \bonus{-1} malus to the French Shock.

  Once summed, the total modifiers are:
  \begin{itemize}
  \item French Fire: +2 (leader); \quad French Shock: +2 (cavalry) -1 (\TTER)
    = +1;
  \item Spanish Fire: +0 (nothing); \quad Spanish Shock: +1 (leader).
  \end{itemize}

  Thus, after factoring the second day modifiers, we obtain for \FRA
  +2/+1//+1/0 and for \HIS 0/+1//-1/0. \FRA has an overall advantage with the
  DRMs.

  Note that if \FRA only had 7\LD in the Battle, that would cancel its second
  cavalry bonus and result in modifiers of +2/0//+1/-1 making its overall
  advantage slighter on par with the Spanish ones. Don't forget to grab all
  those \bonus{+1}, in the end they will save the day\ldots

  The combat result sheet should now look like:
  \begin{tabular}{c|c}
    \FRA & \HIS \\
    \hline
    +2/+1//+1/0 & 0/+1//-1/0\\
  \end{tabular}
\end{exemple}

\subsection{Two days of battle}
\label{chMilitary:Battle:Fight}

\GTtable{combatresultsonly}

The battle itself consist in two \terme{days}, each one split in a Fire and
Shock rolls. Each roll is made on~\ref{table:Combat Results}, by
cross-referencing the correct column
(see~\ref{chMilitary:Battle:Parameters:CRT}) with the result of 1d10 modified
by the correct DRM (see~\ref{chMilitary:Battle:Parameters:DRM}).

The results of each roll can be --- (nothing); a number of \emph{losses} (in
increment of \texttu); or some \textetoile (\emph{Morale losses}). Both the
losses and Morale losses are applied (if any).

Between the rolls, various checks are made to see if the battle ends now
(\emph{e.g.} if one side is routed or wants to retreat). Again, apply the
battle sequence described here in strict order.

Losses and morale losses are tallied during the battle. It is advised to note
them down when they happen to ease the process.

\begin{playtip}
  On the sheet where you've noted the DRM, you should have some space to note
  the result. Write in each column the results inflicted by the corresponding
  side (alternatively, you may write the results suffered by said side, but in
  any case make a choice and stick to it).

  To handle the full losses and \texttu or \texttd ones, it is convenient to
  note the full losses as | and the \texttu{s} as dots: one dot for a \texttu
  (.), a second one for a \texttd (:) and connect them in a | when it becomes
  a full loss (see example below).

  Morale losses can be written as \textetoile. Be sure to leave some space for
  writing more things as the battle unfolds (it may be convenient to tally the
  losses on the left-hand part, from right to left (\emph{i.e.} from centre to
  outside) while the \textetoile are tallied on the right from left to right
  (\emph{i.e.} from centre to outside), that way you won't run out of space or
  risk one tally to ``crash'' into the other).

  \smallskip

  The first few battles you play will probably be a bit slow to unfold
  cautiously. After a handful of battles (that is, usually, one turn of war),
  you'll lean the process and thinks will move smoothly. Beware that land
  battles are more frequent, so be sure to check the small differences when
  handling the less frequent sea battles.
\end{playtip}

% Be sure to use the exact same phrasing each time.
\newcommand{\chMilBattleRoll}[4][]{\bparag[#4.] Both sides roll for #4 of
  the #2 day. \ifx#1\relax\else At sea, the side that suffered the most
  \textetoile previously in this battle has a \bonus{-1} DRM (note that this
  is \textbf{not} necessarily the side with less Morale or less remaining
  Morale); in case of tie, nobody gets the malus.\fi #3 results.}
\newcommand{\chMilFireRoll}[3][]{\chMilBattleRoll[#1]{#2}{#3}{Fire}}
\newcommand{\chMilFireTech}{
  \bparag[Technology.] On the European map, stack with \TREN technology and no
  \ARMY counter does not roll for Fire. A stack with \TREN technology only
  apply the \textetoile. A stack with \TARQ technology only does half (round
  down to lesser \texttu) the indicated losses. A stack with \NGD and no \VGD
  only does half (round up to larger \texttu) the indicated losses.}
\newcommand{\chMilMoraleCheck}{
  \bparag[Rout.] If one side has suffered at least as many \textetoile than
  its \terme{Morale}, it is immediately routed. It loses the battle and its
  opponent wins it. Go to~\ref{chMilitary:Battle:Pursuit}.}
\newcommand{\chMilWindAdvRetreat}{
  \bparag[Retreat.] At sea (only), the side with \terme{Wind advantage} may
  decide to retreat. If it does, it loses the battle and its opponent wins it
  the but there is no pursuit. Go directly to~\ref{chMilitary:Battle:Loss
    modifications}.}
\newcommand{\chMilShockRoll}[1]{\chMilBattleRoll[t]{#1}{Tally}{Shock}}

\newcommand{\chMilFire}[4][]{
  \aparag[#4 fire]
  \chMilFireRoll[#1]{#2}{#3}
  \chMilFireTech
  \chMilMoraleCheck
  \chMilWindAdvRetreat
}

\newcommand{\chMilShock}[2]{
  \aparag[#2 shock]
  \chMilShockRoll{#1}
  \chMilMoraleCheck
}

\chMilFire{first}{Note}{First}

\chMilShock{first}{First}

\aparag[End of first day: Destruction.] Only do this for land battle. Using
the computation described at~\ref{chMilitary:Battle:Loss modifications}, check
if one side has suffered more losses than its size. If so, go
to~\ref{chMilitary:Battle:Pursuit} (the Pursuit may cause extra Morale loss
and turn this into a rout and, possibly, a \terme{Major Battle}). The
destroyed stack loses the battle and its opponent wins it.

\aparag[End of first day: Retreat.] Each side has the possibility to attempt to
break battle and retreat (thus loosing the battle). If this an interception
battle, the phasing side decides (and resolves) first, otherwise the
non-phasing side decides first.
\bparag To attempt a retreat, one side must roll 1d10 and compare it to the
sum of the \Man of its commanding leader and its remaining Morale (initial
Morale minus number of \textetoile suffered).
\bparag If the result is \leq Morale+\Man, the retreat succeeds. The
retreating side loses the battle and its opponent wins it. Go
to~\ref{chMilitary:Battle:Pursuit}.
\bparag Otherwise, the retreat fail. the opposing side will get a +1/+1
modifier for 2nd day.
\bparag It is possible to attempt retreat after a failed retreat by the
opponent, thus it is possible for both sides to get the +1/+1 modifier for
failed enemy retreat\ldots

\chMilFire[t]{second}{Tally}{Second}

\chMilShock{second}{Second}

\begin{exemple}
  During a sea battle, suppose that one side started with a Morale of 4 and
  suffered from 2\textetoile while the opposing side started with a Morale of
  only 2 but suffered only 1\textetoile. In that case, the first side will get
  the \bonus{-1} malus for ``more Morale losses'' even through its remaining
  Morale (2) is higher than its opponent (1).
\end{exemple}

On land, it may happen that a \ROTW minor or \pays{natives} stack fight with
more than 8\LD. In that case, the stack rolls, for each roll, 1d10 for each
group of 8\LD. One of these dice must be designed (before roll) as 'main die'
and another one as 'last die'. Only count the \textetoile caused by the 'main
die'. Count the losses of \textbf{all} dice but tally the losses of the 'last
die' separately (they will be reduced as the 'last die' may correspond to less
than 8\LD).

Any other stack fights normally, whatever the number of \LD in it. Notably,
\TUR stacks may have more than 8\LD due to \Pashas but nonetheless use the
normal rules. Similarly, this rule only apply for \ROTW countries (including
\pays{natives}) and thus stacks with \LeaderC that ``converted'' natives to
their side may contain a large number of \LD but still fight normally.

\begin{exemple}
  The Togukawa shogun decides to get ride of the Portuguese traders and the
  population in \ville{Nagasaki} attacks them. There are 40\LD of Natives in
  the province. Hence they will use 40/8=5 dice in battle. Only one of them
  will cause \textetoile but they will all cause losses\ldots

  If, after a first battle, there are only 35\LD of Natives in the province,
  then they still use 5 dice in battle but now the last one only represent
  3\LD and not 8 (35 = 4 \textmultiply 8 + 3). Losses caused by this die are
  tallied separately as they will need to be reduced as any losses caused by
  'only' 3\LD.
\end{exemple}

\begin{exemple}
  Continuing the previous \FRA-\HIS battle. \FRA has 8\LD and modifiers of
  +2/+1//+1/0 with a Morale of 4. \HIS has 5\LD and modifiers of 0/+1//-1/0
  with a Morale of 4. Both roll on columns \textbf{C/B}.

  \begin{minipage}{0.7\linewidth}
    \emph{First day, Fire:}
    \begin{itemize}
    \item \FRA rolls 8+2 = 10, resulting in
      1\texttd\textetoile\textetoile.
    \item \HIS rolls 6+0 = 6 resulting in \texttu.
    \end{itemize}
  \end{minipage} %
  \hfill %
  \begin{minipage}{0.25\linewidth}
    \begin{tabular}{c|c}
      \FRA & \HIS \\
      \hline
      +2/+1//+1/0 & 0/+1//-1/0\\
      \textcolor{red}{{\normalfont :}|\textetoile\textetoile} & \textcolor{red}{.}
    \end{tabular}
  \end{minipage}

  \FRA has not lost any Morale. \HIS has lost 2 Morale, less than its initial
  4 and thus still has Morale left. Nobody routs, hence the battle goes on.

  \begin{minipage}{0.7\linewidth}
    \emph{First day, Shock:}
    \begin{itemize}
    \item \FRA rolls 4+1 = 5 resulting in \texttu. Its total is now
     2\textetoile\textetoile.
    \item \HIS rolls 7+1 = 8 resulting in 1\texttu\textetoile. Its total is now
      1\texttd\textetoile.
    \end{itemize}
  \end{minipage} %
  \hfill %
  \begin{minipage}{0.25\linewidth}
    \begin{tabular}{c|c}
      \FRA & \HIS \\
      \hline
      +2/+1//+1/0 & 0/+1//-1/0\\
      {\normalfont :}\hspace{-2.15pt}\textcolor{red}{|}|\textetoile\textetoile
           & \textcolor{red}{\normalfont :}.\hspace{-2.3pt}\textcolor{red}{|\textetoile}
    \end{tabular}
  \end{minipage}

  \FRA has lost 1 Morale, less than its initial 4. \HIS has lost 2 Morale,
  less than its initial 4. Nobody routs, the battle goes on.

  Since this is the end of first day, one check whether one side is completely
  eliminated. See details of the computation in~\ref{chMilitary:Battle:Loss
    modifications}. \FRA has actually suffered only 1 losses (1\texttd-\texttd
  because \HIS only has 5\LD), less than its initial 8\LD hence he's still
  alive. \HIS has actually suffered 2\texttd losses (2 becoming 2\texttd
  because \FRA has +1 size), less than its initial 5\LD hence he's still
  alive. Since there are still men in both troops the battle goes on.

  Remark: complete elimination of a troop after first day only happens with
  small troops (3\LD or less) or extreme results (rolling 10 for both Fire and
  Shock with good modifiers).

  Since this is the end of first day, each side has the possibility to try and
  retreat. \FRA suffered almost no losses and decide to stay. \HIS has already
  lost 2\texttd\textetoile\textetoile and feels like a major defeat is
  coming\ldots Since he has 4 in \Man, he decide to try to flee. 4 \Man + 2
  remaining Morale, the retreat threshold is 6. \HIS rolls\ldots 7 and fails!

  During the second day, \FRA thus gets a +1/+1 modifier (it basically cancels
  the -1/-1 for second day) and thus actually stays at +2/+1. \HIS, as
  computed, is now at -1/0.

  Notice that the artillery and cavalry bonus (or any other) are not
  recomputed between each day and stay the same than in the beginning of the
  battle, even if some of the troops were lost. The actual losses are only
  implemented at the end of the battle and at this stage we still use the
  initial forces.

  \begin{minipage}{0.7\linewidth}
    \emph{Second day, Fire:}
    \begin{itemize}
    \item \FRA rolls 2+2 = 4 resulting in nothing. Its total is now
      2\textetoile\textetoile.
    \item \HIS rolls 9-1 = 8 resulting in 1\textetoile. Its total is
      now 2\texttd\textetoile\textetoile.
    \end{itemize}
  \end{minipage} %
  \hfill %
  \begin{minipage}{0.25\linewidth}
    \begin{tabular}{c|c}
      \FRA & \HIS \\
      \hline
      +2/+1//\textcolor{red}{+2/+1} & 0/+1//-1/0\\
      ||\textetoile\textetoile
           & \textcolor{red}{\normalfont :}{\normalfont :}\hspace{-2.15pt}\textcolor{red}{|}|\textetoile\textcolor{red}{\textetoile}
    \end{tabular}
  \end{minipage}

  Since the beginning of the battle, both side suffered from 2 Morale losses,
  less than their initial Morale. Nobody routs, hence the battle goes on.

  \begin{minipage}{0.7\linewidth}
    \emph{Second day, Shock:}
    \begin{itemize}
    \item \FRA rolls 6+1 = 7 resulting in 1\textetoile. Its total is now
      3\textetoile\textetoile\textetoile.
    \item \HIS rolls 5+0 = 5 resulting in \texttu. Its total is now
      3\textetoile\textetoile.
    \end{itemize}
  \end{minipage} %
  \hfill %
  \begin{minipage}{0.25\linewidth}
    \begin{tabular}{c|c}
      \FRA & \HIS \\
      \hline
      +2/0//+2/+1 & 0/+1//-1/0\\
      \textcolor{red}{|}||\textetoile\textetoile\textcolor{red}{\textetoile}
           & {\normalfont:}\hspace{-2.15pt}\textcolor{red}{|}||\textetoile\textetoile
    \end{tabular}
  \end{minipage}

  Nobody routs. With 3 losses on each side, the battle even looks quite
  balanced\ldots so far.
\end{exemple}

\begin{exemple}[Variations]
  Keeping the previous example, if \HIS was \terme{Conscript} it would have
  lost its last Morale point and rout during the last Shock. Similarly, if
  \FRA was \terme{Conscript} (and \HIS \terme{Veteran}), they would both end
  up with only 1 Morale left, resulting in a tie (see below).

  Don't neglect that extra Morale point for \terme{Veteran} troops, it may
  cost some upkeep but not paying it will cost much more once the fighting
  actually start. A good drill makes a good soldier.

  \smallskip

  If both side had \TARQ technology, then the Fire losses are halved (round
  down) and the final result would be 2\textetoile\textetoile\textetoile for
  \FRA and 2\textetoile\textetoile for \HIS. A much less deadly result.
\end{exemple}

\subsection{Pursuit}
\label{chMilitary:Battle:Pursuit}

\aparag[Determine winner.]
\bparag First, using the computation described at~\ref{chMilitary:Battle:Loss
  modifications}, check if one side has suffered more losses than its number
of \LD. If so, that side is declared loser and the other winner, even if a
winner and loser were already declared during the battle.
\bparag Next, if one side has already been declared winner, it wins and the
other side loses.
\bparag Otherwise, the side with the most remaining Morale (starting Morale
minus number of \textetoile suffered) loses and the other side wins.
\bparag In case of tie, no side wins and no Pursuit happens. Go
to~\ref{chMilitary:Battle:Loss modifications}.

\begin{designnote}[Destruction]
  It may happen that one side breaks and routs due to no more Morale during
  the battle. But, at the end of the day and after gathering remaining troops
  and such, they discover that the other side is actually no more a
  battle-effective force (total destruction). Thus they can stay and keep the
  ground. This may indeed change the winner of the battle.

  Entirely destroyed forces are pursued nonetheless because Pursuit can cause
  a rout (and a \terme{Major Battle}), and because on sea Pursuit can capture
  ships.
\end{designnote}

\begin{exemple}[Destruction]
  4\LD of \HIS fight against 8\LD of \TUR in \TARQ. \TUR has a \terme{size
    differential} of +2. \TUR is \terme{Conscript} (because of its \Timar) and
  thus only has 2 Morale while \HIS has 4 Morale (\terme{Veteran} and \TTER).

  At the end of the Second Fire, \TUR has suffered
  \texttd\textetoile\textetoile (modified \HIS rolls were 7, 3 and 7).  At the
  same time, \HIS has suffered 3\textetoile\textetoile\textetoile (modified
  \TUR rolls were 9, 7 and 7).

  With \textetoile\textetoile, \TUR breaks and routs. This happens
  immediately. However, before rolling for Pursuit, a quick check is
  made. \TUR has actually only suffered 0 losses (\texttd-\texttd because of
  only 4\LD causing them). \HIS has actually suffered 4\texttu losses (because
  of the +2 \terme{size differential}). Hence, the Spanish troop is actually
  entirely destroyed. \TUR wins the battle. The poor state of its troops
  (rout) does not allow it to pursue the looser (which could turn this in a
  \terme{Major Battle}), but \TUR nonetheless stays in place and is considered
  winner for all purposes.

  Basically, the Turks broke during the battle but in the days following it
  they gathered their troops and discovered that they still had an army. At
  the same time, between losses in the battle and desertions in the following
  days, the Spanish could not manage to gather a combat-effective force and
  simply had to leave.
\end{exemple}

The winner pursues the loser. Exception: if the winner has been routed during
the Battle, it does no pursue. Pursuit is made by rolling 1d10, modified as
follows, on column \textbf{E} of the CRT. Exception: use column \textbf{C}
when pursuing \NTD or \NGD (used as \terme{first line ships} by the loser).

General modifiers (used in any battle):
\begin{modlist}
\item[+2] if the battle ended during the First day (including total
  destruction at the end of the first day).
\item[+1] if the battle ended after a Fire roll.
\end{modlist}
Note that these modifiers are cumulative, hence a Pursuit after First Fire
will get a \bonus{+3} modifier.

Land modifiers (used in land battles only):
\begin{modlist}
\item[+S] \Shock differential. Differential between the \Shock of the pursuing
  \LeaderG and the pursued \LeaderG, only if positive.
\item[-1] if the battle did not occurred in Plain terrain.
\item[+1] for \TUR \terme{Sipahi} cavalry, if (i) there is at least one \TUR
  \ARMY counter in the pursuing stack; (ii) reform M-2 has not been made; and
  (iii) the battle occurred in Plain or Desert terrain. This is exactly the
  same conditions as for getting the Cavalry bonus to Shock and because losses
  haven't been applied yet, \TUR gets either both or none.
\end{modlist}

Sea modifiers (used in sea battles only):
\begin{modlist}
\item[+M] \Man differential. Differential between the \Man of the pursuing
  \LeaderA and the pursued \LeaderA, only if positive.
\item[+1] if the pursuer has at least 5\ND more than its opponent. Note that
  losses haven't been applied yet, hence this uses the number of \ND each side
  had at the beginning of the battle.
\item[+1] if the pursuer has \terme{Wind Advantage}.
\end{modlist}

The result of the Pursuit is read by cross-referencing the modified roll with
column \textbf{E} of the CRT. Exception: use column \textbf{C} when pursuing
\NTD or \NGD. \textetoile and losses obtained are tallied together with the
ones obtained during the battle. Large \pays{natives} stacks roll one die per
8\LD as for any roll of the battle.

Additionally, at sea, for each \textetoile obtained the pursuer may choose to
capture either 1\NWD, 2\NGD, or 2\NTD. These captures may occur on any ships
in the stack (not only the \terme{first line ships}). Remove them from the
pursued stack and add the same number to the pursuing one (they may be of any
nationality already present in that stack, controller's choice). Next:
\begin{itemize}
\item If the pursued stack was transporting troops, it must immediately loses
  a number of troops whose transport points are equal or greater than the
  transport capacity of the captured ships (only). It is assumed that the
  pursuit specifically targets those troops-heavy transports that were
  purposely left out of the battle\ldots
\item If the \NTD captured were carrying gold, for each \NTD captured 5\ducats
  are sunk (lost) and 10\ducats are seized by the pursuer. These ducats are
  immediately tallied in \lignebudgetlong{Gold from ROTW and Convoys}
  (\emph{i.e.} for the sake of simplicity they are 'teleported' in Europe and
  we do not handle chain of capture and re-capture of the Gold fleet).
\end{itemize}

Finally, check if the losing side has any Morale left. If no, then there is a
\terme{Rout}. Morale left plays no role from now on.

\begin{exemple}
  In the previous battle, no side is entirely destroyed at the end of the two
  days. \FRA has 2 Morale left while \HIS has only 1. Hence, \FRA wins the
  battle and pursues the Spanish. The DRM for Pursuit is here 0 (\Shock
  differential is negative, battle occurred in Plain, battle ended after
  Second Shock). \FRA rolls 9 which, in column \textbf{E} gives
  1\textetoile. Now, that fourth \textetoile brings the Spanish Morale down to
  0 thus transforming an organised Retreat into a Rout!

  The \textetoile can now be forgotten. \FRA suffers from 3 losses. \HIS
  suffers from 4 loses and a rout.

  \smallskip

  If \FRA had \terme{Conscript}, then both sides would end up with 1 Morale
  left hence there is no winner. Note that the extra Morale point of
  \terme{Veteran} gives the victory to \FRA!

  \smallskip

  Also, note that \FRA had globally worse rolls than \HIS (8, 4, 2, 6 vs 6, 7,
  9, 5) but the bonus DRM make them sufficient to fetch victory. Note also
  that what's important in a battle is to grab those \textetoile (\emph{i.e.}
  in the usual \textbf{B} or \textbf{C} column to roll 7 or higher, 10 or
  higher being better). Even if they inflicted the same number of losses
  during the battle (3 each), \FRA managed to inflict one more \textetoile
  (thanks to its \bonus{+2} to Fire and its \bonus{+1} to Shock). That's all
  it need to win the battle! If they had roll the same number of \textetoile,
  then the battle would end in a draw rather than a French victory\ldots

  Also, note that the failed Spanish retreat is here very costly. The
  \bonus{+1} to the Second French Shock turns a 6 into a 7, causing the third
  \textetoile. Globally, the +1/+1 gives 10\% more chances to get
  \textetoile\textetoile on each roll (at +0, there are three \textetoile and
  one \textetoile\textetoile; at +1 there are still three \textetoile but two
  \textetoile\textetoile), hence around 0.4\textetoile. Don't try to retreat
  without good Morale or good \Man, this is a sure recipe for disaster.

  Lastly, note that in \TMED vs \TMED, with only 1 or 2 Morale (for
  \terme{Veteran}) and rolling in column \textbf{A} (with
  \textetoile\textetoile on 9 or more) the battle can end very brutally and
  any side may rout unexpectedly. On the other hand, once both sides start
  having a lot of Morale and using columns \textbf{B} or \textbf{C}, the
  battle cannot anymore end on a single roll and several good results are
  needed, thus greatly diminishing the luck factor (and giving more weight to
  the DRM).
\end{exemple}

\begin{playtip}
  Don't send \terme{Conscript} stacks into battle. Ever. Don't be cheap with
  your upkeep and make sure that all your stacks are \terme{Veteran}. Always.

  Don't raise new troops (that will be \terme{Conscript}) close to your
  opponents (especially if you don't have initiative) as you want to
  reorganise your stacks into \terme{Veteran} ones before sending them to
  battle.

  Paying for new troops just to see the opponent march on them and win the
  battle ``by default'' because they started with 1 more Morale is a painful
  experience. Seeing the troops you've just payed for being annihilated
  immediately because they don't know how to fight is very painful. Having
  your opponent claim a \terme{Major Battle} in the process is an error you
  normally make only once.

  Before upkeeping and sending \terme{Conscript} into battle, think twice. And
  then take a break and think again. If you still believe it's a good idea,
  you're probably wrong. Just pay that extra upkeep and drill your soldiers.

  You may think that a very good general will get sufficient bonus to win
  despite having untrained soldiers. Alternatively, you may want to give your
  best troops to your best general that is going in the front lines to
  destroyed the enemy. Once the enemy has no more troops, it will be time to
  lay siege with these \terme{Conscript} that survived so far.
\end{playtip}

\subsection{Loss modifications}
\label{chMilitary:Battle:Loss modifications}
The losses obtained during the battle are correct for similar stacks (in terms
of actual composition of typical armies of the period) of 8\LD. They must be
corrected taking two factors into account. Firstly, the losses causes by small
stacks (less than 8\LD) are reduced; next losses are adjusted to the relative
structural size of both armies. A similar procedure is used at sea.

In any cases, the modified losses may not go below 0. If this should happen
for any reason, the modified losses becomes 0. (for example, if 1\LD only
inflicts \texttu losses, this results in 0 losses, not \texttu-2=-1\texttd).

\subsubsection{Small stacks}
\GTtable{lossestablesonly}

Reduced the losses inflicted by any stack with less than 8\LD (or
6\ND). Firstly, find in~\ref{table:Losses Tables} the column corresponding to
the number of \ND (first line) or \LD (second line) in the stack. Next, read
in the last line (``Mod.'') a modifier to apply to the losses inflicted by
this stack (with 7\LD, there is only 50\% chances to have -\texttu).

Finally:
\begin{itemize}
\item on land, losses inflicted are capped by the size of the stack, remove
  any exceeding loses;
\item at sea, losses inflicted are capped by twice the size of the stack,
  remove any exceeding loses; if a side was routed, the losses it suffers are
  always at least 1\ND, increase them if needed.
\end{itemize}

\begin{exemple}
  Suppose that at the end of a battle, 2\LD inflicted 4 losses. First, find
  the correction for 2\LD which is '-1\texttu', thus the losses are reduced to
  4-1\texttu = 2\texttd. This is still more than the number of \LD in the
  stack, hence the losses are capped to 2 before proceeding
  further. Basically, it is not possible to kill more enemies that you have
  soldiers.
\end{exemple}

A native force that rolled more than one die has to apply this procedure for
the losses caused by its 'last die' (only), using the number of \LD
represented by that die. After that, all the losses caused by that force can
be added together. Other forces with more than 8\LD (\emph{e.g.} \TUR with
\Pashas) can simply keep this step.

\subsubsection{Size Comparison}
Note that the procedure described here to find the \terme{size differential}
and use it looks more complicated than it actually is\ldots

\GTtable{armyclasses}

\GTtable{armysizesonly}

First, find the \terme{Army size} of each stack involved into battle. For
mono-national stacks, cross-reference the nationality or army class with the
current period in~\ref{table:Army Classes} to find a number between 0 and
7. For multi-national stacks, check~\ref{chMilitary:Stacking}.

Next, compare the size of both sides. This can be done in two ways. The
easiest way is to cross-reference both sizes in~\ref{table:Size Comparison} to
read a \terme{size differential} between -2 and +2. The Table is symmetrical
hence if one side has, say, a \terme{size differential} of -1 then the other
automatically has +1. The other way is to directly compute the \terme{size
  differential} by taking the difference between sizes, dividing by 3 and
rounding to the closest integer.

Both methods give the exact same result and using one or the other is mostly a
matter of ease with the computation or the cross-referencing. Note that a
difference between sizes of 0 or 1 gives a \terme{size differential} of 0, a
difference of 2, 3 or 4 gives a \terme{size differential} of +1/-1 and a
difference of 5 or more gives +2/-2.

Note that \terme{size differentials} are \textbf{not} transitive. For example,
in period \period{I}, \POL has +1 vs \FRA and \FRA has +1 vs \HIS but \POL
still only has +1 against \HIS.

At sea, a stack with 7 to 12\ND uses a \terme{size differential} of +1, a
stack of 13 to 18\ND uses +2 and a stack of 19\ND or more uses +3.

\GTtable{sizeonly}

\aparag Finally, once the \terme{size differential} is found, adapt losses
inflicted according to~\ref{table:size}:
\bparag On the line ``-1/0'' (in bold), find the number of losses inflicted
and use that column. If this is not an integer (\emph{e.g.} 2\td), find
both the number of full losses and the number of thirds. If more than 6 losses
were inflicted, use the last column.
\bparag Cross-reference the column (or columns) obtained above with the line
corresponding to the \terme{size differential}. The new number (or sum of
numbers) is the actual losses inflicted. If more than 6 losses were inflicted,
you'll find instead a number to add or subtract to the losses.

Note that this last correction is \textbf{not} capped as the previous one. It
basically represents the fact that, what is represented by a \TUR \LD
contained actually almost twice as many men as what is represented by a \HIS
\LD. Thus, during a \TUR-\HIS battle, if \TUR inflicts 1 losses, this roughly
correspond to 1 of \emph{its} \LD and is a much larger \HIS casualty.

\begin{exemple}
  Continuing the previous \FRA-\HIS battle. \HIS had 5\LD and inflicted 3
  losses. \FRA had 8\LD and inflicted 4 losses.

  First, because \HIS has less than 8\LD, its losses are reduced according to
  <L1>. In the '4/5\LD' column, the last line reads '-\texttd'. Hence, \HIS
  has only inflicted 3-\texttd=2\texttu losses. This is less than its number
  of \LD (5), hence it is not capped further. \FRA had 8\LD and its losses do
  not need to be reduced (they may still need to be capped in extremely bloody
  battles).

  Next, compute the \terme{size differential}. In period \period{III}, \FRA
  has a size of 2 and \HIS has a size of 0. Hence the \terme{size
    differential} is +1/-1 favouring \FRA.

  Lastly, adapt losses according to it. \HIS first looks in the ``0'' line to
  find the columns corresponding to both '2' (fourth column) and '\texttu'
  (first column). Next, \HIS has to look at the numbers in this columns in the
  ``-1'' line. Because the ``-1'' and ``0'' lines are the same (``-1/0''), the
  same numbers are obviously found. Hence the losses inflicted by \HIS are
  actually 2\texttu.

  \FRA finds in the ``-1/0'' line the column corresponding to '4' (sixth
  column). Next, it looks in the ``+1'' line what's written in this column and
  finds a 4\texttd. These are the losses really inflicted by \FRA.

  Note that the differences in size, both structural (the +1 of \FRA) and
  conjectural (\HIS only has 5\LD) end up playing a huge role on the final
  losses. The losses differential was 1 before these computation and has now
  become 2\texttu!
\end{exemple}

\begin{exemple}[Size differential]~

  Inflicting 4 losses at -2 results in 3 actual losses.

  Inflicting 3\texttd losses at +1 results in 3\texttd+\texttd=4\texttu.

  Inflicting 2\texttd losses at -2 results in 1\texttu+\texttu=1\texttd.

  Inflicting 7 losses at +2 gives a '\#+2' modifier and results in 7+2=9
  actual losses.

  Inflicting 8\texttu losses at +3 gives a '\#+4' modifier (the thirds are
  ignored if more than 6 losses are inflicted, the last column already takes
  that into account). This results in 12\texttu actual losses.
\end{exemple}

\begin{designnote}
  \terme{Size differential} usually does not vary a lot, especially since your
  favourite enemy won't change. You'll quickly learn against who you have a +1
  or a -1. Then, the use of table <L2/S2> is much quicker to do than to
  explain\ldots Find a number (in the correct line) and go up or down one or
  two lines to find the result. After half a dozen battles, this will become
  an extremely quick procedure, even if its description in the rules seems
  very complicated.
\end{designnote}

\subsubsection{Losses at sea}
At sea, losses must be split in three categories:
\begin{itemize}
\item \terme{Refitted} ships come back into play immediately. They are not
  actually lost.
\item \terme{Damaged} ships are no more battle effective. They may however be
  refitted later this turn (during the End of round segment) or during a later
  turn (during the Administrative phase). Tally \terme{Damaged} ships on the
  Colonial Record Sheet and remove them from play. They are grouped per region
  where the battle occurred (owner's choice in case of a battle in
  \seazoneHorn).
\item \terme{Destroyed} ships are sunk and removed from play immediately.
\end{itemize}

\begin{todo}
  Refitting seem to be missing from the Admin phase.
\end{todo}

\begin{designnote}
  The \terme{Damaged} ships represent a structural size of the various
  navies. Even if numerous ships were actually sunk, powers of that time were
  usually quick to rebuild a navy of comparable size. Much quicker than they
  would be to build the same number of ships ``in normal time''. Hence,
  \terme{Damaged} ships represent ships that are actually sunk but that the
  power somehow 'think' that they should be rebuilt fast.

  This is typically what happened after Lepanto where the Turkish navy was
  practically annihilated (loosing more than 80\% of the forces engaged in the
  battle) but nonetheless quickly rebuilt in the couple of years following the
  battle.
\end{designnote}

Losses are split in these categories by group of 3. One loss is affected in
the first group, one in the second, and one in the third. Then, looping around
to the first group, remaining losses are affected in the same way. Any thirds
are always affected as if they were the last loss. When \NWD suffer thirds of
losses, use \NDE (1\NDE=\texttu\NWD). When \NGD or \NTD suffer thirds of
losses, round them up to the nearest largest integer.

\begin{itemize}
\item For the winner of the battle (if any), the first loss is
  \terme{Damaged}, the second is \terme{Destroyed}, and the third is
  \terme{Refitted}.
\item For any non-winner, the first loss is \terme{Damaged}, the second is
  \terme{Destroyed}, and the third is \terme{Damaged}.
\end{itemize}

If some ships were captured during Pursuit, reduce the number of
\terme{Damaged} ships by the same amount (this does not reduce the amount of
losses suffered).

Losses (whatever the category) must be taken first from the \terme{first line
  ships}. If there are more losses than \terme{first line ships}, other ships
suffers from losses. In any cases, the controller of the stack chooses which
\ND actually suffer the losses (when there is a choice).

If the losses are more than the total number of \ND in the stack, the
splitting in losses category must still be done entirely. Then, start by
applying the \terme{Destroyed} losses, next the \terme{Damaged} ones and
finally the \terme{Refitted} ones.

Additionally, if a fleet containing \VGD and using \NGD/\VGD as \terme{first
  line ships} is routed, then at least one of the losses (either
\terme{Damaged} or \terme{Destroyed}, controller's choice) must be a \VGD.

\begin{exemple}[Applying naval losses]
  A winning fleet suffers 4 losses. The first is \terme{Damaged}, the second
  is \terme{Destroyed}, the third is \terme{Refitted}, then, looping back to
  the start, the fourth is \terme{Damaged}. This results in 2 \terme{Damaged}
  \ND, 1 \terme{Destroyed}, and 1 \terme{Refitted}. Decrease the number of \ND
  in the stack by 3 (\terme{Destroyed} and \terme{Damaged}) and tally 2
  \terme{Damaged} ships in the correct geographical area on the colonial
  sheet. The number of losses suffered by the stack is still 4 for all
  purposes (notably for determining if this was a \terme{Major Battle}).

  \smallskip

  A loosing \NWD fleet suffers 5\texttd losses. The first is \terme{Damaged},
  the second is \terme{Destroyed}, the third is \terme{Damaged}; looping back
  the fourth is \terme{Damaged}, the fifth is \terme{Destroyed}; there is no
  full sixth loss, hence the thirds act as a partial sixth loss and are
  \terme{Damaged}. Reduce the number of \ND in the stack by 5\texttd
  (\emph{i.e.} 5\ND and 2 \NDE), and note that 3\texttd are \terme{Damaged}.

  \smallskip

  If a \NGD stacks suffers 5\texttu losses, then because there are no
  thirds of \NGD, the losses are rounded up to 6 and split in the usual way.

  \smallskip

  A winning stack of 5\ND suffers 8 losses. Splitting 8 losses for a winning
  stack gives 3 \terme{Damaged}, 3 \terme{Destroyed}, and 2
  \terme{Refitted}. Because the losses exceed the actual size of the stack,
  the \terme{Destroyed} are applied first (2\ND remain) and next the
  \terme{Damaged}. This results effectively in 3 \terme{Damaged} and 2
  \terme{Destroyed}.

  \smallskip

  A stack of 1\NWD and 10\NGD chooses to use its \NWD as \terme{first line
    ships}. It loses and suffers 8\texttd losses (using \NGD as \terme{first
    line ships}, its opponent used these +1/+2 bonus for larger fleet with
  deadly efficiency). Splitting the losses gives 3 \terme{Destroyed} and
  5\texttd \terme{Damaged}. The \terme{Destroyed} are applied first, on the
  \terme{first line ships}. Once there are no more \terme{first line ships},
  the remaining losses (2 \terme{Destroyed} and 5\texttd \terme{Damaged}) are
  applied to the rest of the stack (here on the \NGD, hence the \texttd is
  rounded up). If 2\NGD were captured during the Pursuit (\textetoile), then
  ``only'' 4 are \terme{Damaged} now. The losses suffered stays at 9 anyway.
\end{exemple}

Note that losses on land are \textbf{not} applied yet. They will only be
applied after the Retreat.

Even if losses are applied now for naval stacks, the commander of the stack
does not change before the Retreat.

\subsection{Retreat}
\label{chMilitary:Battle:Retreat}
Any non winning force must retreat. Any losses suffered during the retreat are
counted as losses suffered during the battle (notably, to determine if this is
a \terme{Major Battle}).

A winning stack may also decide to retreat (typically, because it is now too
small to besiege a fortress), but this is never mandatory. If it retreats, it
has too follow the whole procedure here (including the roll and losses); this
does not change the winner of the battle (the stack is still winning the
battle and may still claim a \terme{Major Victory}).

\subsubsection{At sea}
Retreat is done toward the closest (in number of zones), non-blockaded,
friendly port large enough to hold the remaining fleet (before applying the
retreat losses).

Retreat is a move and thus any retreating fleet must roll for attrition as for
any naval move. If a \terme{Replacement} \LeaderA was rolled for the battle,
he is still available during the retreat (providing its \Man for the attrition
roll). However, if the fleet was routed during the battle, then the \Man of
its \LeaderA is considered to be 0 for the retreat.

Any \ND sunk during the retreat counts as 1 loss for the Battle. Any \NDE
counts as \texttu.

If it is large enough, the winning fleet (if any) may decide to follow the
retreating fleet and blockade the port where it goes (only). This is also a
move, hence a following fleet must roll for attrition as usual (using the
\terme{Replacement} \LeaderA if applicable). Any \ND or \NDE sunk during this
follow-up are counted as battle losses. A winning stack may decide to split
and only part of it follows (typically, after an interception to allow the
rest to carry on with its move and action). The part that follows must be
large enough to blockade the port and usual hierarchy rules are enforced when
splitting the stack.

The retreat may not be intercepted and is not subject to \Presidios or
\StraitFort. A possible follow-up may not be intercepted but can be subject of
\StraitFort.

\subsubsection{On land}
\GTtable{retreatonly}

Any retreating stack must roll on~\ref{table:retreat}. Roll 1d10 and, if the
stack was not routed, subtract the \Man of the commanding \LeaderG (a
\terme{Replacement} leader used during the battle still provides its
\Man). The resulting losses are added to the current total.

Retreating forces must retreat in one of the following possibilities. These
possibilities have no priority (controller's choice) but are further
restricted by the conditions of the battle.
\begin{itemize}
\item the province of the battle itself;
\item the fortress of the province, if friendly to the retreating stack;
\item an adjacent province with no unbesieged enemy presence (notably, with no
  unresolved battle).
\end{itemize}

In some cases, explicitly specified, it is possible (or mandatory) to retreat
through one or more adjacent neutral provinces and into a further friendly
province. This is notably the case for a phasing stack that entered the
province from one where it is not allowed to stay (\emph{e.g.} \HRE, frozen
Sund).

It is possible to split a retreating stack between the fortress of the
province and an adjacent province, only. Any other splitting is not allowed.

It may happen that a troop cannot retreat in its entirety or cannot retreat at
all, usually because the fortress of the province is too small to accommodate
the troop (1\LD/level) and there is no legal adjacent province. In this case,
exceeding troops are destroyed and the losses are part of the battle. On the
other hand, stacking in provinces is only checked after the battle ends and
any loss due to overstacking is \emph{not} part of the battle.

If this was an interception battle, then the non-phasing side has to retreat
first (if needed) followed by the phasing side (if needed). If this was a
regular battle, then the phasing side has to retreat first (if needed)
followed by the non-phasing side (if needed). Exception: if the winning side
decides to retreat, it always retreat second.

It is only possible to retreat in the province itself if retreating second
(especially, it is not possible if the other side won and does not retreat).

If the fortress of the province was besieged prior to battle, and only one
side has to retreat, then the troops that were in the fortress must retreat
back in it and any other troops must retreat in an adjacent province. In case
of tie, any friendly troop can retreat in or out of the fortress.

The only adjacent province in which a phasing stack may retreat is the one
were it come from. If this was a landing, the phasing stack must retreat back
in the ships and the carrying fleet must move back to a friendly port and roll
for attrition accordingly; it may not be intercepted during this move and it
is not subject to \Presidios and \StraitFort.

The only adjacent provinces in whicĥ a non-phasing stack may retreat after an
interception battle are the ones from which a least one intercepting \LDE came
from. If interceptors came from several provinces, they may not retreat in
several province; they must retreat in a single adjacent province (or the
fortress) as a single stack but can choose in which (among those from where
interceptors came).

Retreat is not a move. It may not be intercepted and never causes attrition.

\begin{designnote}
  Most of the time, only one side has to retreat in which case the other
  (winning) side stays in place and occupy terrain.

  Retreat must be done toward ``where you came from'' which can have various
  meaning\ldots

  When both sides retreat (tie), then the one who moved into the province
  first retreats second and thus may decide to stay in place. Basically, the
  battle is assumed to happen on the frontier of the province and either the
  invasion force does not manage to enter (regular battle) or interceptors are
  repulsed. Even if there is no winner, hence this is not really a decisive
  battle, one side happens to keep control of the land.

  \smallskip

  As said in the Interception, when a phasing stack enters a province where
  there is a non-phasing stack and gets intercepted by a force of the same
  alliance, then the interceptor can choose to either resolve this as a
  separate interception battle or to merge and resolve this as a regular
  battle. In case of tie, if this is resolved as a regular battle, then the
  non-phasing (non-intercepting) side was here first and the phasing must
  retreat first.

  On the other hand, if this is resolved as an interception battle (between
  phasing and interceptor only) then the phasing stack was here first and the
  interceptor must win in order to stay in place. If the phasing side stays
  after such an interception (either winning, or tied and decide to stay),
  then this is in a province with unbesieged enemy presence, hence the phasing
  stack must stop movement and will resolve a regular battle against the
  troops that are still here (\emph{i.e.} the ones that were here before the
  interception).

  \smallskip

  When a battle happens around a besieged fortress (usually to attempt to lift
  the siege), then a tie allow to mix and match troops in and out of the
  fortress during the retreat. The siege is not necessarily lifted (depending
  on the precise outcome) but in the chaos of the battle troops were able to
  sneak in or out. On the other hand, if the besieger side wins, then the
  besieged side must retreat the ``in'' troops in the fortress and the ``out''
  troops out of the fortress. And of course, if the besieged side wins, the
  siege is lifted and the question does not arise anymore.
\end{designnote}

Lastly, on the European map, round total losses suffered by each side to the
closest integer. This rounding is part of the losses suffered during the
battle.

Each full loss corresponds to 1\LD and each \texttu to 1\LDE. These losses
must be removed now. As usual, it is possible to break \ARMY (and \LD)
counters to get the ``change''. The controller of the stack choose which
troops are killed. If the stack is smaller than the losses, then it is
destroyed and the losses are considered to be only the size of the stack
(\emph{i.e.} if a 2\LD stack suffer 6 losses, it is entirely destroyed but it
is considered that it only suffered 2 losses as only 2\LD are killed).

In some case, there is not enough counters to satisfy the losses (\emph{e.g.}
if an \ARMY\faceplus suffers from 1 loss but the country has no more \LD
available). In this case, losses are increased until they can be satisfied
(\emph{e.g.} rounded up from 1 to 2 and the \ARMY\faceplus is flipped
\Facemoins). This increase is counted as losses suffered during the battle.

Similarly, if \TUR decides to kill one or more \Pashas that represent more \LD
than the number of losses, then the total number of \LD killed is counted
toward the total losses.

\begin{exemple}
  Continuing the \FRA-\HIS battle. \HIS was routed and suffered 4\texttd
  losses while \FRA won and suffered 2\texttu losses. Being the only
  non-winning stack, \HIS must retreat.

  Being routed, \HIS simply roll 1d10 on Sub-table <L3> (\Man is not used
  after a rout). Luckily, it rolls 2 and this does not cause any extra loss.

  Losses are now rounded because the battle occurred in Europe. \HIS rounds
  the 4\texttd losses it suffered to 5 (nearest integer) and is actually
  entirely destroyed!  \FRA rounds the 2\texttu losses it suffered to 2
  (nearest integer). The final result is that \HIS is routed and suffers 5
  losses while \FRA only suffers 2 losses.

  The Spanish stack is removed from play. The French stack losses 2\LD
  (typically, flipping an \ARMY from \Faceplus to \Facemoins).
\end{exemple}

\begin{exemple}[From an actual gaming session]
  At the beginning of~\ref{pIV:English Civil War}, \pays{royalistes} controls
  \provinceCornwall but neither \provinceGloucester nor \provinceWessex. It
  nonetheless decides to put its newly recruited army (2\ARMY\faceplus,
  \terme{Conscript}) in \provinceCornwall, commanded by
  \leaderwithdata{Rupertroy}.

  Due to an early and already finished~\ref{pIV:TYW}, \monarque{Gustav Adolf}
  has nothing to do and decides to go and save Protestantism in England (maybe
  hoping to turn that Anglicanism into true Protestantism). The Swedes do a
  foreign intervention with 1\ARMY\faceplus (\terme{Veteran}) and
  \leaderwithdata{Gustav-Adolf}.

  The Swedes land in \provinceWessex where \leader{Rupertroy} fails its
  interception, and march onto \provinceCornwall. Being \TBAR vs \TMUS, \SUE
  rolls in columns \textbf{B/B} and \pays{royalistes} in columns
  \textbf{D/C}. Moreover, \SUE has 4 Morale and \pays{royalistes} only 3. \SUE
  fights at +3/+3 (structural cavalry bonus) while \pays{royalistes} is only
  at +1/+1 (artillery and size cavalry bonus).

  For the first Fire, \leader{Rupertroy} rolls only 5+1=6, causing \texttu. At
  the same time, \leader{Gustav-Adolf} rolls a lucky 10+3=13 and causes
  3\textetoile\textetoile\textetoile! Trembling at the mere name of the Lion
  of the North, the royalists troops flee in panic at the first shots of the
  \emph{L\"{a}derkanonen}!

  \leader{Gustav-Adolf} now pursues at +5 (+2 \Shock, +2 First Day, +1 after
  Fire) and rolls 4+5=9 causing 1 more. These 4 losses are
  reduced to 3\texttu due to the small \SUE stack (4\LD) and the +1
  \terme{size differential} brings that back to 4. On the other hand, the
  \texttu losses suffered by \SUE are not modified (8\LD causing them, -1
  \terme{size differential} does not change anything) but are rounded to 0.

  \leader{Rupertroy} retreats and roll 3, causing \texttu more. The 4\texttu
  losses are rounded to 4. However, the only retreat possibility for
  \pays{royalistes} is the fortress of \ville{Plymouth} which can only
  accommodate 1\LD. The rest of the troops are killed trying to get in, simply
  disband and desert without any possibility to gather them into a fighting
  force, and many of them surrendered to the Swedes. Thus, the actual losses
  suffered by \pays{royalistes} are 7\LD (8\LD initial - 1\LD who managed to
  retreat). Note that in this particular case, there was actually no need to
  roll for the retreat\ldots
\end{exemple}

\begin{designnote}
  ``Lost'' troops can be either killed, captured, disbanded, deserting, \ldots
  In any way, they are no more combat-effective, but a huge casualty does not
  necessarily means a bloodbath. The game use the term ``kill'' to denote all
  these casualties for the sake of simplicity.
\end{designnote}

\subsection{Battle cleanup}
\label{chMilitary:Battle:Cleanup}

\subsubsection{Major Battle}
A Battle is called a \terme{Major Battle} if all of the following conditions
are true:
\begin{modlist}
\item one side won;
\item[AND] the loosing side was routed;
\item[AND] (on land) the loss differential is at least 3\LD, or 4\LD if the
  winning side has a \terme{size differential} of +2;
\item[AND] (at sea) the loss differential is at least 5 (if winner used \NWD)
  or 8 (if winner used \NGD);
\item[AND] (on land, in the \ROTW) the loosing side had at least one \ARMY
  counter from an European country at the start of the battle.
\end{modlist}

\begin{designnote}
  The battle has to be decisive (rout and loss differential). If far away
  (\ROTW), it also need to include sufficiently many (\ARMY counter) ``real''
  (European) troops. Thus, typically, victories of \HIS over \paysAzteque are
  never \terme{Major Battle} because nobody in Europe cares about killing a
  bunch of natives. On the other hand, battles like Saratoga or the Plains of
  Abraham involved European \ARMY counter on the loosing side. They had huge
  impact on the European politics and population and thus do qualify as
  \terme{Major Battle}. In other words, there may be \terme{Major Battles} in
  the \ROTW but they need sufficiently many European troops to have an impact
  on European population.

  Even if the \emph{looser} needs an European \ARMY, it is not the case for
  the winner. If a large Spanish invasion force happen to be crushed in the
  New World, this will have a huge impact on the Spanish (and European)
  population, maybe slow down the conquest a bit (``It's too dangerous to go
  there, it's not worth it''), \ldots Similarly, a few lone \LD may sometimes
  score a \terme{Major Victory}.
\end{designnote}

\aparag In case of \terme{Major Battle}, apply all the following effects:
\bparag the loosing side loose 1 \STAB (no penalty if already at -3);
\bparag the winning side wins 1 \STAB (no bonus if already at +3);
\bparag the winning side wins 5\VPs.

Minor countries never gain or loss \STAB or \VPs. It is, however, possible to
gain or loss \STAB or \VPs when fighting against a minor country. In case of
multi-national stacks, the gain or losses are for the country which commanded
the troops during the battle.

\begin{designnote}
  Note that because a side may not suffer more losses than it has troops
  (exceeding losses are not counted), it is never possible to have a
  \terme{Major Defeat} if the loosing side has only 2\LD involved. Also, keep
  in mind that a rout is necessarily for a \terme{Major Battle}.
\end{designnote}

\subsubsection{Death of leaders}
Any \terme{Replacement} leader used during the battle is now forgotten. It's
not needed any more and a new \terme{Replacement} leader will be rolled the
next time it's needed.

Every commanding leader has to test a possible casualty in battle. roll 1d10,
modified as follows:
\begin{modlist}
\item[-1] for any non-winning side;
\item[-5] for a side entirely destroyed during the battle;
\item[-1] if the leader has a value of 6 in either \Fire or \Shock (exception:
  \leader{Marlborough} and \leader{Friedrich II} do not have this malus, this
  is hinted on the tokens with their names and stats in white instead of black
  or yellow).
\end{modlist}

Exception: if the battle occurs on the European map and the opposing side had
strictly less than 3\LD at the beginning of the battle, do not roll for leader
casualty.

\begin{designnote}
  In Europe, if you don't risk \terme{Major Defeat}, you can't kill the enemy
  leader either (this prevents a gamey technique).
\end{designnote}

If the result is 1 or less, the leader suffered from the battle. Roll 1d10
(never modified) to check the result:
\begin{itemize}
\item if this result is odd, the leader is killed, remove it from play;
\item if this result is even, the leader is merely wounded. The wounds will
  last for a number of rounds equal to half the result of the die. Count
  rounds on the Round track using the longest possible path (S1, W1, S2, W2,
  \ldots) and place the leader in the box corresponding to the round when it
  will come back (or the 'End' box if this is after W5).
\end{itemize}

Countries that have less leaders than their minimal will get new ones at the
beginning of next round.
\begin{todo}
  When? Link with example already written. Complete correct part.
\end{todo}

\begin{designnote}
  Here also, a 'killed' or 'wounded' leader is a game term not to take too
  seriously. It may mean a capture, a disgrace removing the leader from
  commanding position (permanently or temporarily), or a severe wound causing
  the leader to stop the military carrier and maybe become minister or
  ambassador, or retiring in his family castle, \ldots
\end{designnote}

\subsubsection{Aftermath}
\label{chMilitary:Battle:Cleanup:Aftermath}
If any side is now in a province with an enemy fortress, the fortress is
considered besieged for all purpose (even if the winning side lacks sufficient
troops, this will only be check during the Siege segment). If the fortress was
already besieged, the siege merely continue from the same situation (yes, this
does allow to ``steal'' a siege in three-sided wars).

On the other side, blockade is not automatic and can only be started if the
fleet is large enough.

If the fortress of the province was besieged and the province now contains
only friendly troops, the siege is now lifted. Remove any \USURE counter there
might be here, move besieged troops out of the fortress and merge them with
the stack here.

Similarly, if a blockading naval stack is now too small to blockade, the
blockade is immediately lifted and the stack stays in the sea.

If one of the stack involved in the battle is now overstacked (normally due to
retreat in an already crowded province or lifting a siege without taking any
loss), exceeding troops are removed. This is not part of battle casualties and
thus does not count toward \terme{Major Battle}.

Change commanding leaders of stacks according to the new situation.

If this is an interception battle won by the phasing side, the stack may
continue its movement. If this is an interception battle, any non-winning side
may not intercept or counter-intercept again during this impulse.

\begin{exemple}[Ending the \FRA-\HIS battle]
  \HIS was routed and suffered 5 losses while \FRA won with only 2
  losses. Thus, the losses differential is 5-2=3 and all the conditions for a
  \terme{Major Battle} are met. \FRA immediately gains 1 \STAB and 5\VPs while
  \HIS loses 1 \STAB.

  Once again, note how the seemingly tie after the two days of battle
  (3\textetoile\textetoile\textetoile vs 3\textetoile\textetoile) turned into a
  Spanish disaster due to that single additional \textetoile (mostly) aided by
  the various external factors correcting the losses (\terme{size
    differential} and small stack).

  Next, leader casualties are checked. \FRA rolls 3, not modified, and nothing
  happens. \HIS rolls 6 - 1 (loser) - 5 (entirely destroyed) = 0, hence its
  \LeaderG is part of the casualties. \HIS rolls again to see if he's dead or
  wounded and obtains 4. It's an even number, hence the general is wounded for
  4/2=2 rounds. Suppose this was W4, he will come back in two rounds, that is
  during W5. Put the token on the W5 box of the round tracks.

  \FRA now besiege \villeArras.

  \smallskip

  If the second roll of \HIS was 7, an odd number, then the leader would have
  been killed and removed from the game. If it was 10, he would be wounded for
  5 rounds. That is the non-existent 'S7' hence the \LeaderG would be put on
  the 'End' box and returned at the beginning of next turn
  (see~\ref{chInter:No lasting wounds}).
\end{exemple}

\begin{playtip}[Winning battles]
  In order to win battles, you should, in decreasing order of importance:
  \begin{enumerate}
  \item roll 10s;
  \item convince your opponent to roll 1s;
  \item have a better general;
  \item have more troops, more artillery, more cavalry, \ldots and catch your
    opponent crossing a river;
  \item have a better technology.
  \end{enumerate}
  Joke aside, this does not mean that battles are a mere luck-fest, especially
  in the late game. But the surprise can always arise and if one side roll
  four 10 while the other roll four 1, there is little doubt on who the victor
  will be, whatever the other conditions.

  Especially, in the late game, with a lot of Morale, many high rolls are
  needed to win the battle. A battle in \TMED, even with \terme{Veteran}
  troops, ends as soon as one side rolls 9 or more and that can indeed feel
  extremely luck driven (well, Medieval battles tend to be quite
  unpredictable). On the other hand, after \TMUS, with 4 Morale, and,
  typically, if rolling at -1 because of Forest, then routing the enemy
  require 4 rolls of 8 or higher. This is not often achieved by mere luck and
  usually require a bit more planning and preparation. Typically, in the
  \SUE-\pays{royalistes} example, even if \SUE got lucky by rolling 10,
  fighting at \bonus{+3} gives a huge advantage: you only need to roll 4 or
  more to cause Morale loss.

  As illustrated in the long \FRA-\HIS example, what's important in a battle
  is not to cause losses but to cause \textetoile. They are linked, but
  typically, the difference between \texttd and 1\textetoile is huge while the
  difference between 1\textetoile and 1\texttu\textetoile is almost
  negligible. When you fight, you need to roll 7 or more to cause those
  \textetoile. Baring luck and rolling 10s while your opponent rolls 1s, you
  best possibility is to come with as much positive DRM as possible and give
  as much negative DRM as possible to your opponent. Crossing a river to
  attack a troop stationed in Mountain means fighting at -2/-2, now you need 9
  or more to cause \textetoile while your opponent need 7 or more. A pretty
  bad situation.
\end{playtip}

\begin{playtip}[Losses]
  The other part of the battle is the losses. That lead to different tactics
  depending on the situation. Typically, if you have a global advantage on the
  strategical level, you may want to seek a decisive battle, destroying as
  many enemy troops as possible to prevent any counter-attack or similar funny
  move and to start besieging in peace. In that case, you want the battle to
  happen with as many positive DRM as possible, even if it costs you some more
  troops. Try to fight in Plains with large stacks.

  On the other hand, if you're on the defensive, then you probably want to
  keep you troops alive as long as possible. 5\LD can be a good ``army in
  being'' able to launch a raid to lift a siege close to its end, to cut
  supply or similar actions, while zero troops are obviously useless. In that
  case, you want to get as many negative modifiers as possible. Try to fight
  in Mountain or Forest, to intercept enemy as soon as they cross rivers,
  \ldots These often lead to inconclusive battles (draw) but that is often
  enough to delay your opponent for 1 or 2 rounds, which is usually all that
  you need.

  \smallskip

  An example of successful delay of the ``guerrilla'' side happened during
  another gaming session. During~\ref{pVII:Independence War}, \ANG tried to
  bring troops to America and land them in one of the pro-British
  \COL. However, they got intercepted by \leaderwithdata{Washington} and some
  patriots. Due to the -2/-3 for disembarking and the -1/-1 of the Forest, the
  Royal troops couldn't do a lot of damage during the Battle. At the end of
  the First Day, \leader{Washington} decided to retreat which, with 6 \Man and
  4 Morale was automatic\ldots The British won the battle and could move on
  but the American raid costed them maybe \texttu or \texttd, enough to flip
  one \ARMY counter and remove artillery bonus for the next battle. A
  perfectly well done raid disrupting the \ANG move and forcing the player to
  think a bit more about what to do.
\end{playtip}

\section{Sieges}
\label{chMilitary:Sieges}
Resolve all sieges, fights against \REVOLT/\REBELLION and \corsaire.

If a land stack is in a province with an enemy fortress and has less than 1\LD
per level of the fortress, it must roll for \terme{Siege Attrition} (see
below). If a land stack is in a province with an enemy fortress and has no
\LoS, it must also roll for \terme{Siege Attrition}. If, for some reason, a
blockading naval stack is too small to blockade the fortress (what did you do
wrong?), the blockade is lifted and the stack stays in the sea.

Next, each alliance, in decreasing order of initiative, resolves all of its
actions (sieges, fight against \REVOLT/\REBELLION an \corsaire) in an order of
its choice (at random in case of disagreement).

Naval stacks whose action is to fight \corsaire resolve it now. Land stacks in
a province with either an enemy fortress or an enemy \REVOLT/\REBELLION must
besiege or fight the \REVOLT/\REBELLION. If a stack is in a province with both
an enemy fortress and an enemy \REVOLT/\REBELLION, it cannot both besiege and
fight. It must do one of the two but the controller of the stack chooses
which; if it has no \LoS or is too small to undermine, it may not fight the
\REVOLT/\REBELLION either (\emph{i.e.} it must either assault or redeploy).

\begin{designnote}
  Order of resolution is rarely relevant. Sieges, \REVOLT/\REBELLION and
  \corsaire may usually be resolved in any order. Use the precise order here
  only if there are disagreements. Especially, when there are separate wars,
  it is normally possible to resolve sieges simultaneously in each war
  (notably if they are waged by different players) rather than waiting
  pointlessly.
\end{designnote}

\subsection{Sieges}
A besieging stack may either \terme{Undermine} the fortress, \terme{Assault}
it, or \terme{Redeploy}. In order to undermine, a stack must have a \LoS and
at least as many \LD as the level of the fortress. Besieging stacks must do
one (and only one) of the three actions listed above. It is not possible to do
nothing.

If a stack chooses to \terme{Redeploy}, it abandon the siege (remove any
\USURE on the fortress) and goes back to a friendly unbesieged province using
the \terme{Redeployment} procedure (see~\ref{chRedep:Redeployment}). These
\terme{Redeployments} are not simultaneous (they are resolved immediately when
the decision is taken); they may happen by sea at the usual conditions for
\terme{Redeployment}; they are cause of attrition as normal
\terme{Redeployments}. A stack may choose to redeploy even if it can
undermine. A too small stack (or a stack with no \LoS) that redeploys rolls
twice for \terme{Attrition} (one \terme{Siege Attrition} because it is too
small and one \terme{Movement Attrition} because of the redeployment).

\begin{designnote}
  A very small stack managing to assault a very large fortress represents a
  surprise attack with troops managing to storm the fortress before its gates
  are closed or similar heroic action.
\end{designnote}

\GTtable{artillerybonus}

Both undermining and assault use an \terme{Artillery bonus} against
fortresses. To obtain it, find in~\ref{table:Artillery bonus against
  Fortresses} the column corresponding to the level of the fortress (column
'0' corresponds to forts). In that column, find the largest number that is
still less or equal than the number of Artillery in the besieger stack
(check~\ref{chMilitary:Stacks:Artillery} for how to compute Artillery in a
stack). In the line corresponding to that number, find in the last column the
\terme{Artillery bonus}.

\begin{exemple}
  Again, this is a computation which is more complicated to explain than to
  perform\ldots

  A stack with 4 Artillery besiege a fortress of level 1. In the '1' column,
  the numbers are '1', '3' and '6'. The stack has less than 6 Artillery, hence
  it cannot claim that bonus. It has more than 3, hence it can claim that
  bonus. This is the '+2' line, thus its \terme{Artillery bonus} is
  \bonus{+2}.

  Against a fortress of level 2, this stack still has \bonus{+2} but against a
  fortress of level 3 it only has \bonus{+1} and no bonus against a fortress
  of level 4 or 5.
\end{exemple}

\begin{designnote}
  As usual, game terms used here must not be interpreted too strictly when
  figuring out what is happening. Even if some very long sieges did happen
  (such as the famous 20 years long siege of \ville{Candia}), most sieges of
  the time were rather short (\leader{Vauban}, in his \emph{Traité sur
    l'attaque des places} and \emph{Traité sur la défense des places},
  consider that a correctly defended fortress can hold 48 days against a
  correct siege). Hence, besieging a city for more than one round is not
  ``realistic''.

  However, actual provinces contain much more than a single city. The ``city''
  in a province actually represents the whole defence system of the province
  consisting not only in many fortified cities, but also various other
  fortresses spread over the land and along roads. Thus, a \USURE token, even
  if it uses the standard one-city siege vocabulary may actually represent the
  fall of one or more of these fortifications and the whole ``siege'' of a
  province corresponds to many actual sieges in the same area.

  As usual, game terms and mechanisms should not be strictly interpreted. They
  provide a good macro effect suitable for our level of abstraction.
\end{designnote}

\subsubsection{Undermining}
To resolve the undermining, roll 1d10. The roll is modified in two different
ways, thus giving two different results. The first one is used to check the
effect of the undermining on the fortress (usually adding some \USURE); the
second one is used to check if the besieger suffer from \terme{Siege
  Attrition}. It is possible to both suffer from attrition and force the
fortress to surrender (especially toward the end of long sieges).

\GTtable{underminingonly}

Effect on fortress modifiers:
\begin{modlist}
\item[-N] level of the fortress;
\item[+?] \terme{Artillery bonus};
\item[-?] terrain malus (see~\ref{chMilitary:Concepts:Siege:Terrain} for
  details), non cumulative:
  \begin{modlisti}
  \item[-2] Europe, non-Plain, no port;
  \item[-2] Europe, non-Plain, blockaded port;
  \item[-2] Europe, Plain, non-blockaded/supplied port;
  \item[-3] Europe, non-Plain, non-blockaded/supplied port;
  \item[-2] \ROTW, non-blockaded/supplied port;
  \item[-2] \ROTW, non-Plain, no port;
  \item[-1] fort, non-blockaded/supplied port;
  \item[-1] fort, non-Plain, no port;
  \item[-0] all other cases;
  \end{modlisti}
\item[+1] [TBD] if there has been no \terme{Breach} this turn, but an assault
  that managed to cause at least 1 loss;
\item[+2] if a \terme{Breach} was obtained previously this turn;
\item[+1/+3] for each \USURE\facemoins/\faceplus on the fortress;
\item[-S] \terme{Siege} value of one besieged leader;
\item[+S] \terme{Siege} value of one besieger or [BLP] blockading leader;
\item[+1] if there are troops inside the fortress, but no \ARMY counter;
\item[+3] if there is an \ARMY counter inside the fortress.
\end{modlist}

When using \terme{Siege} value of leaders, only one leader in each side may
provides its value but he does not need to be the commanding leader. In any
case, always use the largest \terme{Siege} value available in the stack.

\begin{todo}
  'Assault +1'/'Breach +2' counters?
\end{todo}

The effect of undermining is read in~\ref{table:Undermining}. The various
possible effects are:
\begin{modlist}
\item[\textbf{S}\facemoins] add a \USURE\facemoins counter. If there are now
  two \USURE\facemoins counters, immediately merge them in a
  \USURE\faceplus. If, after that, there are three or more \USURE, only keep
  the largest two.
\item[\textbf{S}\faceplus] add a \USURE\faceplus counter. If there are three
  or more \USURE, only keep the largest two.
\item[\textbf{B}] The walls have been breached. The besieged may immediately
  attempt an \terme{Assault} using the (more favourable) \terme{Breach}
  condition. This \terme{Assault} is not mandatory. This is the only case
  where both undermining and assault can be done on the same fortress in the
  same round. If the besieger does not assault, the presence of at least one
  \terme{Breach} still makes further undermining easier.
\item[\textbf{B} or \textbf{WH}] The besieger must decide to either apply the
  \textbf{B} result or to give \terme{War Honour} to the besieged. If giving
  \terme{War Honour}, the fortress falls and any troops inside the fortress,
  plus 1\LD (representing the garrison) are ``given back'' to the besieged who
  must immediately place them in any unbesieged friendly province (they may be
  split in different provinces, they may 'teleport' at the other end of the
  World for the sake of rules simplicity).
\item[\textbf{R}] The fortress surrenders. It falls and troops inside it are
  destroyed (captured, likely, as executing defenders after the fall was not a
  common practise), any leader inside is ``given back'' to the besieged who
  must immediately place them in any unbesieged friendly province. Exception:
  if a \terme{Monarch} happen to be trapped in a surrendering fortress, he is
  captured.
\end{modlist}

When a fortress falls (\textbf{WH} or \textbf{R}), first put a control marker
of the besieged (of the country of the commander) on the fortress. This
fortress is now controlled by that country. Next, the new controller must
choose to either man the fortress again or not. If the fortress is manned
again, the besieger must loose 1\LD (from the besieging stack) and the
fortress loses 1 level; otherwise, the fortress loses 2 levels. In the \ROTW,
fortresses may be entirely destroyed that way. In Europe (including European
provinces in the \ROTW and \COL of level 6), the minimal level a fortress may
have after this is 1. Fortress may ``fall down'' to level 1 even if the
minimal level on the map is 2, in which case a white level 1 fortress is used
to denote this.

\begin{playtip}
  Free tip: when taking a fortress of level 1 or 2, whether you re-man it or
  not, it will end up level 1. Don't waste a \LD for nothing\ldots When taking
  a fortress of higher level, the question arises.
\end{playtip}

\begin{playtip}
  Non-Plain terrain provides a \bonus{-2} to the roll, which is more than an
  extra level for the fortress (only \bonus{-1}). However, with lower level,
  it is easier to get \terme{Artillery bonus} and the fortress is more likely
  to fall in case of \terme{Assault}. Thus, as a rule of thumb, you can
  consider that non-Plain terrain roughly has the same defensive power as one
  more level of fortification (a bit less for low-level fortress that are
  likely to be assaulted immediately).

  Thus, when planning your defence, if you want an evenly spread defence line
  you should put lower level fortresses in non-Plain terrain along your
  ``wall''; but if you want an impregnable stronghold holding at all costs
  even if the rest of the land is lost, you should put high level fortresses
  in non-Plain terrain.

  The same reasoning roughly goes for ports if your enemy has no naval power
  to blockade it (typically, when \SUE wants to defend against \RUS or \POL).
\end{playtip}

Attrition check modifiers:
\begin{modlist}
\item[+4] if this is the first round of siege this turn for this fortress;
\item[-2] if the siege is continuing from a previous turn;
\item[-S] siege value of one besieged leader;
\item[-?] number of \LD in the fortress (\ARMY\facemoins = 2\LD,
  \ARMY\faceplus = 4\LD).
\end{modlist}

If the result is strictly smaller than the distance (in \MP) from a \SoS, then
the stack must roll for \terme{Siege Attrition}.

There is no adverse effect of using a weak supply source for this test. Thus,
it is possible to choose, during the Supply Segment, a full \SoS that is, say
5\MP away (and thus avoid \terme{Supply Attrition}) and to choose for this
check a weak \SoS that is only 2\MP away (and thus possibly also avoid
\terme{Siege Attrition}). Basically, there is no memory of which \SoS was used
previously and as long as the stack is not too far away from a \SoS,
everything is fine.

Note also that the \SoS is only checked when needed. Because sieges are
actually resolved in a chosen order, it may happen that a first siege succeed
and thus create a now \SoS (the fortress is now controlled) to use for the
next siege.

\subsubsection{Assault}
Assault looks like a simplified battle. It has many concepts in common, but a
few specificities. Notably: there are less columns in the CRT; there is only
one day of fight; there are no terrain modifiers; losses modifications are
simpler (no \terme{size differential} to compute); \ldots As for Battles, the
sequence described here must nonetheless be strictly followed.

\GTtable{assaultonly}

\textbf{Assault parameter.} If one side has no leader, draw a
\terme{Replacement} general. Next, depending on its technology, find the
\terme{Morale} of each side (besieged are always \terme{Veteran}, even if
there is a lot of \terme{Conscript} troops in the fortress).

\textbf{CRT column.} The besieger always use the ``\textsc{Besieger}'' columns
of~\ref{table:Assault Results}. The columns used by the besieged depends
whether there has been a \terme{Breach} or not. Note that the Morale losses
column (\textetoile) is always the same. If this assault \emph{immediately}
follows a \textbf{B} result during an undermining, then the besieged uses the
``\emph{Breach}'' columns of the Table, otherwise, it use the ``Fire/Shock''
columns (even if a previous \terme{Breach} was obtained, hasty reparation have
cancelled most of its effect). Note that the ``\emph{Breach}'' columns do
always exactly 1 less loss than their counterpart, that is, assaulting after a
\terme{Breach} is much less dangerous.

\aparag[DRMs for the besieged:]~\\
\begin{modlist}
\item[+F/0] \Fire differential, if positive;
\item[0/+S] \Shock differential, if positive.
\end{modlist}

\aparag[DRMs for the besieger:]~\\
\begin{modlist}
\item[+F/0] \Fire differential, if positive;
\item[0/+S] \Shock differential, if positive;
\item[+1/+1] if the besieged has \TMED technology;
\item[-1/-1] if the besieged has \TARQ technology or better;
\item[-N/-N] level of the fortress (0 for forts);
\item[+?/+?] \terme{Artillery bonus} against the fortress;
\item[+1/+1] [TBD] if a previous assault caused at least 1 loss.
\end{modlist}

\textbf{Fire.} Both sides roll for Fire and tally both the losses and
\textetoile they inflict. The besieged always Fire at full power (heavy siege
artillery is stored in the fortress). Besieger stacks with \TMED technology do
not Fire; besieger stacks with \TREN technology only counts the \textetoile
and besieger stacks with \TARQ technology halve the losses they cause (round
to the nearest lesser \texttu).

\textbf{Shock.} If a side has been routed during Fire, the other side rolls
for Shock anyway. If no side has been routed, both sides roll for Shock. Tally
losses and \textetoile.

\textbf{Losses modifications for besieger.} If the besieger has less than
8\LD, the losses it causes are modified as follows (cumulative):
\begin{modlist}
\item[-\texttu] if 6\LD or less;
\item[-\texttu] if 4\LD or less (hence -\texttd when combined with the
  previous one);
\item[-\texttd] if there is no \ARMY counter.
\end{modlist}
Next, add \texttd if the besieger stack contains at least one \ARMY\faceplus
(not 2\ARMY\facemoins !) of \POL (during periods \period{I} or \period{II}),
\RUS (during periods \period{I} to \period{III}), or \TUR \Janissaire (before
M-1a, during periods \period{I} to \period{III}).

\GTtable{resistanceonly}

\textbf{Losses modifications for besieged.} Cap the losses caused by the
besieged to twice the resistance of the fortress (read
in~\ref{table:Fortresses Resistance}) plus the number of \LD inside
(\ARMY\facemoins = 2\LD, \ARMY\faceplus=4\LD). Next, add \texttd if the
besieger has been routed.

\textbf{Result of the assault.} If the besieged has suffered as least as many
\textetoile as its Morale, the fortress fall. Otherwise, losses caused by the
besieger are first used to kill troops inside the fortress. If the remaining
losses are at least equal to the resistance of the fortress, it falls. Losses
suffered by the besieger are rounded to the nearest integer in Europe and then
simply remove the corresponding number of \LD from the stack.

\textbf{Resistance} of the fortress depends on its level and whether the
assault immediately follows a \textbf{B} result or not. As for the use of the
``\emph{Breach}'' column, use the lesser resistance only if the assault
immediately follows a \textbf{B}. Use this resistance both for capping
besieged losses and to determine if the fortress falls.

\textbf{Aftermath.} If the fortress didn't fall, nothing happens. It regain
all its \terme{Resistance} for any future \terme{Assault}. If the fortress has
fallen due to lack of Morale, troops inside are removed (captured or killed)
and any leader inside are given back to the besieged who must immediately
place them in an unbesieged friendly province (exception: monarchs are
captured).

When a fortress falls (no more Morale or Resistance), first put a control
marker of the besieged (of the country of the commander) on the fortress. This
fortress is now controlled by that country. Next, the new controller must
choose to either man the fortress again or not. If the fortress is manned
again, the besieger must loose 1\LD (from the besieging stack) and the
fortress loses 1 level; otherwise, the fortress loses 2 levels. In the \ROTW,
fortresses may be entirely destroyed that way. In Europe (including European
provinces in the \ROTW and \COL of level 6), the minimal level a fortress may
have after this is 1. Fortress may ``fall down'' to level 1 even if the
minimal level on the map is 2, in which case a white level 1 fortress is used
to denote this.

\subsection{Siege Attrition}
\terme{Siege Attrition} is resolved as any attrition by cross-referencing a
modified 1d10 on the correct column of~\ref{table:Discoveries and
  Attrition}. Use the ``Land, Europe'' column corresponding to the size of the
stack if the siege is in Europe, and the \ROTW column +~\ref{table:Attrition
  ROTW Remainders} if the siege is in the \ROTW.

Note that besieged roll for \terme{Siege Attrition} during the Supply Segment
while besieger roll for it (sometimes) during the Siege Segment.

Modifiers for \terme{Siege Attrition} (besieger):
\begin{modlist}
\item[+2] if the stack has no \LoS;
\item[+2] if the stack contains 6\LD or more;
\item[+2] if there is \terme{Bad weather};
\item[+?] if the stack is supplied by a naval stack, and the \LoS of this
  naval stack goes through one or more \StraitFort, add the DRM of all the
  \StraitFort along this path (2 in Europe, level/2 in the \ROTW);
\item[+?] in the \ROTW cold area, add the number of Snowflakes ``resource''
  (+0 to +2 depending on the \Area);
\item[+1] per \PILLAGE\facemoins, \REVOLT\facemoins or unfriendly
  \REBELLION\facemoins in the province
\item[+2] per \PILLAGE\faceplus, \REVOLT\faceplus or unfriendly
  \REBELLION\faceplus in the province
\item[-S] siege value of one besieger or blockading leader; even land stacks
  may use the siege value of one blockading \LeaderA;
\item[+S] siege value of one besieged leader;
\end{modlist}

Note that there is no \bonus{+2} for being in an enemy province as the
besieger more or less already has control of the countryside, the fields,
and the roads. Thus, it can easily take part of its supply on the spot. In the
Table, the \bonus{+2} is cancelled by a specific \bonus{-2} for the
besieger\ldots

Result for \terme{Siege Attrition} is read as usual. As for \terme{Supply
  Attrition}, \terme{foraging} has no effect here (so we don't have to
remember which stack was or wasn't foraging a long time ago) and thus can be
ignored.

\subsection{Fight against \REVOLT/\REBELLION}

\subsection{Fight against \corsaire}

\iamhere

\aparag[Blocus]
Il faut avoir au moins la force navale voulue selon le niveau de forteresse
(voir table ou supra).
\bparag Coupe le bonus de -3 au test de ravitaillement des assiégés ;
\bparag Flotte qui veut sortir ou entrer : doit faire un test pour échapper au blocus
(ou attaquer la flotte en blocus)

\subsection{Les sièges}
\aparag Pour la sape, effet du terrain (non cumulatif)
\bparag           -2      Port sans blocus, terrain clair
\bparag           -3      Port sans blocus, terrain autre que clair
\bparag -2 Terrain accidenté (montagne, marais, forêt, désert) sans port
ou blocus

\aparag[TBD] si un assaut a causé au moins 1 perte (sans modif de
taille ni bonus ``grosse armée'' dans le tour : +1 à la sape et à
l'assaut (max +1, non cumulatif avec le +2 de brèche).


\aparag Les tables sont à jour !

\aparag[Expérimental]
Un assaut qui a obtenu au moins 1 pertes (sans compter les bonus
de Janissaires, \RUS, \POL) sans prendre la forteresse donnera un
bonus de {\bf +1} aux jets de sappe et aux assauts suivants du tour.

\aparag[Port Siegeworks]
Ports that are besieged with at least one level of Siegework are
submitted to a fire from the siegework that works the same way as the
Presidios, with a {\bf +1} per counter Siegework\faceplus.  {\bf But the
 port is not blockaded.}

\aparag[Impossibilité de tenir un siège]
Ceci est regardé au début de la phase de siège (nbre de DT >= niveau) ;
si impossible, mvt de rédéploiement forcé vers chez soi \\
- en fin de tour: si pas Usure\faceplus, redéploiement forcé.


\subsection{Les sièges}
-- NE PAS PRENDRE EN COMPTE --

%Les règles ne sont quasiment pas changées par rapport au combat rapide de la 2nde extension.

\subsubsection{L'assaut}
-- NE PAS PRENDRE EN COMPTE --


\paragraph{Les rounds d'assaut.}
L'assaut se fait en deux jets, un de feu puis un de choc sauf que
le choc n'est pas fait par un camp qui a craqué au moral .
Les tables de combat montrent une colonne spécifique à l'assiégé et une pour l'assiégeant.
Noter que l'assiégé fait une perte en moins au feu et au choc si le combat est
suite à une \textbf{brèche}.


\paragraph{Modificateurs.}
L'assiégeant ajoute 1 si le défenseur est médiéval, soustrait 1 si le défenseur est en arquebuse
ou mieux, à son feu et son choc.
L'assiégeant soustrait aussi le niveau de la forteresse aux deux si il n'y a pas eu de
brèche. Enfin, l'artillerie ajoute  +1 en assaut si l'assiégeant a au moins
4 fois le niveau de la forteresse en artillerie (sauf contre un fort).

\paragraph{Les ajustements aux pertes.}
\begin{itemize}
\item 1- si l'assiégeant n'a pas 2 A+, le tableau des pertes variables réduit
ce qu'il inflige ;
\item 2- la Turquie et la Russie jusqu'en 1614, et la Pologne jusqu'en 1559 augmentent
les pertes faites en assaut de 1/2 par A+ présente ;
\item 3- l'assiégeant prend une demie-perte en plus si il a craqué au moral.
\item 4- les pertes de l'assiégeant sont limitées au nombre de DT dans la
fortification plus 2 fois la résistance de
la forteresse (ajustée par la brèche).
\end{itemize}

\paragraph{Résistance de la forteresse.}
Les pertes faites à l'assiégé sont d'abord prises sur les unités enfermées dans
la forteresse, puis sur la résistance de celle-ci. Cette résistance est égale à
son niveau, mais est réduite en cas de brèche. Elle revient à son niveau
maximum après chaque assaut.

\paragraph{La victoire.} Elle revient au camp selon l'ordre de priorité suivant~:
\begin{enumerate}
\item Assiégé, si l'assiégeant est éliminé ;
\item Assiégeant, si les troupes à l'intérieur sont éliminées et la résistance atteint 0,
ou si l'assiégé craque au moral
(même si l'assiégeant déroute) ;
\item Assiégé, si l'assiégeant seul ou si personne ne craque au moral.
\end{enumerate}


\subsubsection{La sape}
-- NE PAS PRENDRE EN COMPTE --

\paragraph{Mettre le siège.}
%\begin{minipage}[b]{0.5\linewidth}
Le siège par usure n'est presque pas modifié. Un pion armée  contient toujours
un nombre d'artillerie égal à celui de la contenance maximum de sa nation à la période en cours.
Une armée sur la face - contient l'artillerie de l'armée + divisée par 2 et arrondie
à l'inférieur.

Il faut pour maintenir le siège devant une forteresse disposer d'au moins autant
d'équivalent détachement que le niveau de la forteresse.
Si l'assiégeant ne peut maintenir le siège en fin de round (après un assaut ou une
bataille), il doit immédiatement retraiter dans une province amie (avant de pouvoir
piller) et jouer l'attrition.
Si il choisit de maintenir le siège, il doit soit lancer un assaut, soit faire
un test dur la table de sape (qui peut être suivi d'un assaut en cas de brèche).

Le propriétaire de la forteresse peut laisser des troupes dans celle-ci.
L'empilement dans une forteresse est d'au plus 2DT par niveau de la
forteresse, ou d'un DT dans les forts. Ces forces subissent une attrition
à chaque fin de phase de mouvement si le siège est déjà
établi. Une fois enfermés dans une forteresse, une force ne peut
en sortir qu'en fin de siège (victorieux ou non) et n'a pas le
droit s'attaquer les assiégeants.

\paragraph{Résolution de la sape.} On utilise la table des annexes,
avec les modificateurs indiqués.

Les pertes assiégeantes obtenues sur la table des sièges se résolvent
en lançant 1d10, diminué des valeurs en siège des généraux
et augmenté de 1 par DT (ou équivalent) en défense dans la forteresse.
Si le résultat est inférieur (strictement) au nombre de round
de siège écoulé, l'assiégeant doit faire un test d'attrition sur la table
adéquante (Europe ou non) avec les modificateurs indiqués.

\subsubsection{Prise des forteresse}
Une forteresse qui tombe par assaut ou sape perd 2 niveaux de
fortification (avec la valeur mise sur la carte en tant que minimum),
sauf si le nouvel occupant décide immédiatement de mettre un
garnison. Il doit pour cela utiliser un DT qui est perdu (le DT peut
provenir de la séparation d'un pion armée).

\section{End of round}
\label{chMilitary:End of round}
\subsection{Exceptional levies}

\subsection{Refit}

\section{New round}
A new round begin with the \terme{Continuation Roll} segment.

\section{Military cleanup}
\label{chMilitary:cleanup}
???

Normally nothing to do here.

% Local Variables:
% fill-column: 78
% coding: utf-8-unix
% mode-require-final-newline: t
% mode: flyspell
% ispell-local-dictionary: "british"
% End:
