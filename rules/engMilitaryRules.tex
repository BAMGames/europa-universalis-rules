% -*- mode: LaTeX; -*-

\definechapterbackground{Military Rules}{military}
\chapter{Military Rules}\label{chapter:MilitaryRules}

\begin{designnote}
  This Chapter focuses on the description of the Military phase in
  Segment order. The main concepts and common rules used during it are
  described in the previous Chapter.
\end{designnote}

\section{Overview}
\aparag[Sequence]
\MilitaryDetailsNew

\section{Setup}
\subsection{Initiative}
Determine the alliances in effect this turn and the order of initiative
between them. Alliances are transitive (the ally of my ally is my ally) for
this purpose (but not for stacking purpose and such). Each alliance plays at
the lower initiative of one of its member, resolve any tie at random.

\subsubsection{Starting round}
Roll one die to determine the starting round (read the result in the boxes of
the rounds display). This die roll is never modified. The weather for this
round is determined as usual. It is possible for the Sund to be frozen during
the first round.

\section{Rounds (general stuff)}
\subsection{Continuation roll}
Do not perform this segment at the first round of a turn.

Roll a die, modified by poor weather and pVI+. Follow the arrows on the rounds
display to determine the new round. Determine weather as indicated.

% Local Variables:
% fill-column: 78
% coding: utf-8-unix
% mode-require-final-newline: t
% mode: flyspell
% ispell-local-dictionary: "british"
% End:
