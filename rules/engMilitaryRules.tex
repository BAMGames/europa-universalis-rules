% -*- mode: LaTeX; -*-

\definechapterbackground{Military Rules}{military}
\chapter{Military Rules}\label{chapter:MilitaryRules}

\begin{designnote}
  This Chapter focuses on the description of the Military phase in
  Segment order. The main concepts and common rules used during it are
  described in the next Chapter. Most of these concepts do not need to be
  perfectly defined in order to understand he flow of the Military phase,
  which is why precise description is postponed. You should refer to the next
  Chapter whenever a point in these rules requires clarification.
\end{designnote}

\begin{todo}
  This Chapter is under heavy work. The absence of detailed numbering of rules
  reflects this.
\end{todo}

\section{Overview}
\aparag[Sequence]
\MilitaryDetailsNew

\subsection{Military setup}
Setup for the military phase.

\subsection{Rounds}
The military phase is split in a certain number of \terme{rounds} (between 3
and 11). During each round, each alliance, in decreasing order of initiative,
has an \terme{impulse} where it can moves its troop and fight. Some matters
are resolved before any alliance has its impulse (continuation, wintering) and
some after each had its one (siege, piracy, revolt). After that, a new round
begins.
\subsubsection{Continuation roll}
Determine the new round to be played.

\subsubsection{Wintering}
Stacks may suffer from attrition, leaders may be redeployed.

\subsubsection{Impulses}
Each alliance, in decreasing order of initiative, plays an impulse. Each
impulse consists in five steps: supply, choice of campaign, movements (and
interceptions), exploration, and battles.

The alliance taking its impulse is called the \terme{phasing} alliance. All
other are non-phasing.

\subsubsection{Sieges}
Resolve all sieges, fights against \REVOLT/\REBELLION and \corsaire.

\subsection{New round}
A new round begin with the \terme{Continuation Roll} segment.

\subsection{Military cleanup}
???

Normally nothing to do here.

The real cleanup is done during Redeployment.

\begin{playtip}[Simultaneity]
  Most of the time, there are separate wars that cannot affect each other
  (typically, with different antagonists in each), or even separate actions
  for a given alliance (typically action in the \ROTW and in Europe). In this
  case, the resolution of the impulses do not need to be as strictly
  sequential as explained in the rules. The military phase is long enough and,
  typically, a FRA-HIS war and a RUS-POL war can be played simultaneously in
  order to make things a little faster.

  It is normally recommended to synchronise all players for the continuation
  roll (because it is a point where a lot of crucial new information is
  gained). Sometimes, it is however possible for two players to quickly
  completely play a small war (noting the results of the continuation roll in
  secret to communicate it later) while other players are still struggling in
  the first round of a big war. Sometimes even while other are still planning
  and resolving their administration\ldots This typically allows to ``free''
  those players to go and buy food for everybody\ldots

  \smallskip

  Similarly, the rules describe a strict order for the resolution of actions
  in the military phase but in practice it doesn't often need to be respected
  that strictly. Typically, two battles can be resolved simultaneously if they
  don't have the same participants, or sieges can be resolved in any order
  (rather than on a per alliance basis) if players agree that this has no
  importance.
\end{playtip}

\section{Military setup}
\subsection{Initiative}
Determine the alliances in effect this turn and the order of initiative
between them. Alliances are transitive (the ally of my ally is my ally) for
this purpose (but not for stacking purpose and such). Each alliance plays at
the lowest initiative of one of its member, resolve any tie at random.

Stacks in interventions (whether limited or foreign) act at the same
initiative than the alliance for which they intervene.

Minors at war alone act at the initiative of the country controlling them.

\subsection{Starting round}
Roll one die to determine the starting round (read the result in the boxes of
the rounds display). This die roll is never modified. The weather for this
round is determined as usual (see below).

It is possible for the Sund to be frozen during the first round if the die was
'1' and \ref{eco:Poor weather} happened this turn.

\section{Rounds}
During each round, perform the segments detailed below. The Impulses segment
is repeated for each alliance in decreasing order of initiative.

Each round is labelled by a letter indicating the \terme{season} ('S'ummer or
'W'inter) and a number (between '0' and '5') indicating the \terme{year}.

\begin{designnote}
  Do not take the seasons and years too seriously when interpreting what
  happens during a turn. This is more a modelisation artefact that gives good
  macro results than a real attempt to simulate military actions on a lower
  scale (especially, the 'S' and 'W' rounds happen simultaneously in the North
  and South hemispheres\ldots)
\end{designnote}

\subsection{Continuation roll}
Do not perform this segment at the first round of a turn.

Roll a die, modified by \bonus{+2} in case of \ref{eco:Poor weather} and
\bonus{-1} if this is period \period{VI} or \period{VII}. Follow the arrows on
the rounds display to determine the new round.

If the new round is the 'End' box, then the rounds stop immediately. Proceed
with \terme{Wintering} then \terme{Military cleanup}.

If the followed arrow is red (modified roll of 1-5 after a Summer round), then
the new round is played with an extended campaign. See choice of campaign for
the implications.

If there is an extended campaign after a (unmodified) continuation roll of
'1', '3' or '5', then the round is played with \terme{Bad weather}.

If \ref{eco:Poor weather} happened this turn, then each Winter round is played
with \terme{Bad weather}, no matter what was rolled.

\begin{playtip}
  End of the Military phase may happen somewhat brutally. You should always
  check the probabilities before planning long term actions (sieges) in years
  4 or 5.

  Given the shape of the rounds track, at least one round in each column must
  happen. Thus, the minimum number of rounds left to play is the number of
  columns and the maximum is twice that number. Moreover, long Military phase
  implies lot of extended campaigns. Take that into account when planning both
  long term actions (sieges) and expenses for the rest of the phase.
\end{playtip}

\subsection{Wintering}
If the current year is different than the year of the previous round, then a
\terme{Wintering} segment occurs. Otherwise, skip to the \terme{Impulses}
segment.

There is a \terme{Wintering} segment when the 'End' box is reached, that is,
it is considered as being 'S6'.

\subsubsection{Cold area}
Any stack in a non-controlled, non-national province within the \terme{Cold
  area} rolls for attrition. Resolve this as a Supply attrition (see
below). It is, however, a different roll and a stack may have to roll for
attrition both during the Wintering and Supply segment in some cases.

\subsubsection{Pashas}
Any stack containing \Timar out of \TUR national territory rolls for
attrition. Check specific Turkish rules for details.

\subsubsection{Hierarchy}
Leaders may be redeployed.

Each country may choose to redeploy its leaders on its stack any way it wants
(no maximal distance or such, in other words it's a free teleportation). If
it chooses to do so, then the hierarchy must be globally respected after this
redeployment. It is always possible to choose not to redeploy leaders at this
point, but as long as at least one leader is redeployed, the country must
respect hierarchy globally.

Exception: besieged leaders as well as leaders on stacks with discoveries that
have not been brought back to Europe must stay in place and are not checked
toward global hierarchy.

\subsection{Impulses}
Each alliance, in decreasing order of initiative, plays an impulse. Each
impulse consists in five steps: supply, choice of campaign, movements (and
interceptions), exploration, and battles.

The alliance taking its impulse is called the \terme{phasing} alliance. All
other are non-phasing.


\section{Supply}
Each phasing land stack which has no supply for two consecutive rounds is
immediately destroyed. Getting back supply temporarily during the round is
enough to avoid this destruction.

Each phasing land stack which has no supply or is in weak supply must roll for
\terme{supply attrition}. Additionally, each phasing besieged land stack must
roll for \terme{siege attrition}. Naval stacks never roll for attrition during
the supply segment. Beware that besieged stacks roll for \textbf{siege}
attrition during this segment, which has a similar procedure as supply
attrition but slightly different modifiers.

If a besieged stack is also in weak supply, it does not roll for supply
attrition. The siege attrition replace this roll.

\subsection{\SoS, \LoS, weak supply}
A Source of Supply (\SoS) is either a controlled city, \TP, \COL or fort, or a
large enough naval stack in an adjacent seazone. A Source of Supply may supply
as many stacks as wanted.

The Line of Supply (\LoS) goes from a Source of Supply to the stack. The \MP
cost is counted as if the stack itself was doing this movement (typically, \LD
in the \ROTW compute the length of their LoS using the special \MP costs for
\LD).

A stack is in weak supply if at least one of the following condition is true:
\begin{itemize}
\item its Line of Supply is as least 6\MP long (exception: in the \ROTW, \LD
  do not check this condition);
\item its Line off Supply goes through desert;
\item its Source of Supply is not owned by the same alliance;
\item it is supplied by a naval stack that is not adjacent to a port or
  arsenal able to supply it.
\end{itemize}

It is possible to take its supply from a further \SoS to avoid weak supply
(typically, to have an owned \SoS, or to circumvent a desert).

\subsection{Attrition roll}
Attrition (in Europe) is obtained by rolling one die, modified as follows, and
cross-reference the modified result in table~\ref{table:Discoveries and
  Attrition} in the ``Loss in Europe'' column that corresponds to the size of
the stack. In the \ROTW, the result is read in the ``\ROTW Losses''
column. Note that a result of 11 or less has no effect.

\begin{todo}
  The Attrition table does not want to be included here. I suspect too much
  dependence with adjacent table in the TikZ code\ldots
\end{todo}

%\GTtable{discoveriesattrition}

Modifiers for Supply attrition:
\begin{modlist}
\item[+2] per cause of attrition above the first
\item[+2] if the stack contains more than 5\LD and 1\Pasha
\item[-M] MAN of the commander of the stack
\item[+2] if the stack has no \LoS
\item[+?] if the stack is supplied by a naval stack, and path from this naval
  stack to its port/arsenal goes through one or more \StraitFort, add the DRM
  of all the \StraitFort along this path (2 in Europe, level/2 in the \ROTW)
\item[+8] if the fortress of the province is controlled by the enemy
\item[+6] if the fortress of the province is controlled by allies or if there
  is no fortress in the province (in the \ROTW)
\item[+1] per \PILLAGE, \REVOLT or unfriendly \REBELLION \facemoins in the
  province
\item[+2] per \PILLAGE, \REVOLT or unfriendly \REBELLION \faceplus in the
  province
\item[+?] in the \ROTW cold area, add the number of Snowflakes ``resource''
  (+0 to +2 depending on the \Area)
\end{modlist}

Modifiers for Siege attrition (besieged):
\begin{modlist}
\item[+6] always (friendly fortress)
\item[+?] in the \ROTW cold area, add the number of Snowflakes ``resource''
  (+0 to +2 depending on the \Area)
\item[-S] siege value of one besieged leader
\item[+S] siege value of one besieger or blockading leader
\item[-3] if besieged in an unblocked port
\item[+1] per \USURE\facemoins
\item[+3] per \USURE\faceplus
\end{modlist}

\subsection{Result of attrition}
In Europe, the result of attrition is either nothing (---), a number
(between 1 and 3), a 'P' or both a number and a 'P'.

The number indicates how many \LD are lost immediately. The controller of the
stack chooses which.

A 'P' is interpreted differently according to the technology of the stack.
\begin{itemize}
\item If the technology is \TMED or \TARQ, then 1 more \LD is lost and a
  \PILLAGE\facemoins is placed into the province (and immediately merged int a
  \PILLAGE\faceplus if there was already another one here).
\item If the technology is \TMUS or better, then the controller chooses to
  either loss one more \LD or place a \PILLAGE\facemoins.
\end{itemize}

Besieged troops cannot place \PILLAGE and thus must loss one \LD in case of
'P'. There is no extra effect for the lost \PILLAGE with low technology.

Note that since \PILLAGE will make further attritions harder, it is sometimes
wise to loss \LD rather than let the troops plunders. Note also that since
\PILLAGE\facemoins will be removed at the end of the turn, small troops with a
not too bad technology don't suffer a lot from attrition.

\smallskip

In the \ROTW, cross reference the percentage obtained with the size of the
stack in table~\ref{table:attritionlosses}. The resulting number indicates the
number of \LD still alive after attrition while the 'd' indicates one or two
\LDE still alive. If there is a \textetoile, then there is 50\% chance to
lose an extra \LDE.

\GTtable{attritionlosses}

\section{Choice of campaign}
Each country of the phasing alliance chooses its campaign for the current
round. Make a tick in the correct line of the ERS to count how many campaigns
of each type you've done so you can pay for them.

More expensive campaigns allow for more actions. In case of extended campaign,
there is a single campaign that spans over two rounds and that can be upgraded
at this point.

\section{Movements}
Each country of the phasing alliance may move its stacks, according to the
campaign it paid for the round. Each movement must be completed before the
next one start. The phasing alliance choose in which order it moves its stacks
(at random in case of disagreement).

Each non-phasing stack may attempt interception. Resolve battles caused by
interception immediately.

Each phasing stack (moving or not) may have to roll for attrition.

\section{Exploration}
Each phasing naval stack not engaged in battle (including interception) may
attempt to explore an adjacent unknown seazone. In case of success, the stack
automatically moves into the explored zone.

Next, each phasing land stack not engaged in battle (including interception)
may attempt to explore an adjacent unknown province. In case of success, the
stack automatically moves into the explored province.

Note that a naval stack embarking troops may explore a new seazone from which
the troops may then disembark to explore a new province.

\section{Battles}
Resolve all non-interception battles caused by the movement, in order of
choice of the phasing alliance (at random in case of disagreement). Each
battle must be fully resolve before the next one starts.

\section{Sieges}
Resolve all sieges, fights against \REVOLT/\REBELLION and \corsaire.

Each alliance, in decreasing order of initiative, resolves all of its actions
in an order of its choice (at random in case of disagreement).

\section{New round}
A new round begin with the \terme{Continuation Roll} segment.

\section{Military cleanup}
???

Normally nothing to do here.

% Local Variables:
% fill-column: 78
% coding: utf-8-unix
% mode-require-final-newline: t
% mode: flyspell
% ispell-local-dictionary: "british"
% End:
