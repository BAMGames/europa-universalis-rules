% -*- mode: LaTeX; -*-

\definechapterbackground{Military Rules}{military}
\chapter{Military Rules}\label{chapter:MilitaryRules}

\begin{designnote}
  This Chapter focuses on the description of the Military phase in
  Segment order. The main concepts and common rules used during it are
  described in the next Chapter. Most of these concepts do not need to be
  perfectly defined in order to understand he flow of the Military phase,
  which is why precise description is postponed. You should refer to the next
  Chapter whenever a point in these rules requires clarification.
\end{designnote}

\begin{todo}
  This Chapter is under heavy work. The absence of detailed numbering of rules
  reflects this.
\end{todo}

\section{Overview}
\aparag[Sequence]
\MilitaryDetailsNew

\subsection{Military setup}
Setup for the military phase.

\subsection{Rounds}
The military phase is split in a certain number of \terme{rounds} (between 3
and 11). During each round, each alliance, in decreasing order of initiative,
as an \terme{impulse} where it can moves its troop and fight. Some matters are
resolved before any alliance has its impulse (continuation, wintering) and
some after each had its one (siege, piracy, revolt). After that, a new round
begins.
\subsubsection{Continuation roll}
Determine the new round to be played.

\subsubsection{Wintering}
Stacks may suffer from attrition, leaders may be redeployed.

\subsubsection{Impulses}
Each alliance, in decreasing order of initiative, plays an impulse. Each
impulse consists in four steps: supply, choice of campaign, movements and
battles.

\subsubsection{Sieges}
Resolve all sieges, fights against \REVOLT/\REBELLION and \corsaire.

\subsection{New round}
A new round begin with the \terme{Continuation Roll} segment.

\subsection{Military cleanup}
???

Normally nothing to do here.

The real cleanup is done during Redeployment.

\section{Military setup}
\subsection{Initiative}
Determine the alliances in effect this turn and the order of initiative
between them. Alliances are transitive (the ally of my ally is my ally) for
this purpose (but not for stacking purpose and such). Each alliance plays at
the lowest initiative of one of its member, resolve any tie at random.

Stacks in interventions (whether limited or foreign) act at the same
initiative than the alliance for which they intervene.

Minors at war alone act at the initiative of the country controlling them.

\subsection{Starting round}
Roll one die to determine the starting round (read the result in the boxes of
the rounds display). This die roll is never modified. The weather for this
round is determined as usual.

It is possible for the Sund to be frozen during the first round if the die was
'1' and \ref{eco:Poor weather} happened this turn.

\subsection{Rounds}
During each round, perform the segments detailed below. The Impulses segment
is repeated for each alliance in decreasing order of initiative.

Each round is labelled by a letter indicating the \terme{season} ('S'ummer or
'W'inter) and a number (between '0' and '5') indicating the \terme{year}.

\begin{designnote}
  Do not take the seasons and years too seriously when interpreting what
  happens during a turn. This is more a modelisation artefact that gives good
  macro results than a real attempt to simulate military actions on a lower
  scale.
\end{designnote}

\section{Continuation roll}
Do not perform this segment at the first round of a turn.

Roll a die, modified by \bonus{+2} in case of \ref{eco:Poor weather} and
\bonus{-1} if this is period \period{VI} or \period{VII}. Follow the arrows on
the rounds display to determine the new round.

If the new round is the 'End' box, then the rounds stop immediately. Proceed
with \terme{Wintering} then \terme{Military cleanup}.

If the followed arrow is red (modified roll of 1-5 after a Summer round), then
the new round is played with an extended campaign. See choice of campaign for
the implications.

If there is an extended campaign after a (unmodified) continuation roll of
'1', '3' or '5', then the round is played with \terme{Bad weather}.

If \ref{eco:Poor weather} happened this turn, then each Winter round is played
with \terme{Bad weather}, no matter what was rolled.

\begin{playtip}
  End of the Military phase may happen somewhat brutally. You should always
  check the probabilities before planning long term actions (sieges) in years
  4 or 5.

  Given the shape of the rounds track, at least one round in each column must
  happen. Thus, the minimum number of rounds left to play is the number of
  columns and the maximum is twice that number. Moreover, long Military phase
  implies lot of extended campaigns. Take that into account when planning both
  long term actions (sieges) and expenses for the rest of the phase.
\end{playtip}

\section{Wintering}
If the current year is different than the year of the previous round, then a
\terme{Wintering} segment occurs. Otherwise, skip to the \terme{Actions}
segment.

There is a \terme{Wintering} segment when the 'End' box is reached, that is,
it is considered as being 'S6'.

\subsection{Cold area}
Any stack in a non-controlled, non-national province rolls for
attrition. Resolve this as a Supply attrition (see below). It is, however, a
different roll.

\subsection{Pashas}
Any stack containing \Timar out of \TUR national territory rolls for
attrition. Check specific Turkish rules for details.

\subsection{Hierarchy}
Leaders may be redeployed.

\section{Impulses}
Each alliance, in decreasing order of initiative, plays an impulse. Each
impulse consists in five steps: supply, choice of campaign, movements,
exploration and battles.

The alliance taking its impulse is called the \terme{phasing} alliance. All
other are non-phasing.

\begin{playtip}[Simultaneity]
  Most of the time, there are separate wars that cannot affect each other
  (typically, with different antagonists in each), or even separate actions
  for a given alliance (typically action in the \ROTW and in Europe). In this
  case, the resolution of the impulses do not need to be as strictly
  sequential as explained in the rules. The military phase is long enough and,
  typically, a FRA-HIS war and a RUS-POL war can be played simultaneously in
  order to make things a little faster.

  It is normally recommended to synchronise all players for the continuation
  roll (because it is a point where a lot of crucial new information is
  gained). Sometimes, it is however possible for two players to quickly
  completely play a small war (noting the results of the continuation roll in
  secret to communicate it later) while other players are still struggling in
  the first round of a big war. Sometimes even while other are still planning
  and resolving their administration\ldots This typically allows to ``free''
  those players to go and buy food for everybody\ldots
\end{playtip}

\subsection{Supply}
Each phasing stack which has no supply for two consecutive rounds is
immediately destroyed.

Each phasing stack which has no supply or is in weak supply must roll for
attrition.


\subsection{Choice of campaign}
Each country of the phasing alliance choose its campaign for the current
round.

\section{Movements}
Each country of the phasing alliance may move its stacks, according to the
campaign it paid for the round. Each movement must be completed before the
next one start. The phasing alliance choose in which order it moves its stacks
(at random in case of disagreement).

Each non-phasing stack may attempt interception. Resolve battle caused by
interception immediately.

Each phasing stack (moving or not) may have to roll for attrition.

\section{Exploration}
Each phasing naval stack not engaged in battle may attempt to explore an
adjacent unknown seazone. In case of success, the stack automatically moves
into the explored zone.

Next, each phasing land stack not engaged in battle may attempt to explore an
adjacent unknown province. In case of success, the stack automatically moves
into the explored province.

Note that a naval stack embarking troops may explore a new seazone from which
the troop may then disembark to explore a new province.

\section{Battles}
Resolve all non-interception battles caused by the movement, in order of
choice of the phasing alliance (at random in case of disagreement). Each
battle must be fully resolve before the next one starts.

\section{Sieges}
Resolve all sieges, fights against \REVOLT/\REBELLION and \corsaire.

\section{New round}
A new round begin with the \terme{Continuation Roll} segment.

\section{Military cleanup}
???

Normally nothing to do here.

% Local Variables:
% fill-column: 78
% coding: utf-8-unix
% mode-require-final-newline: t
% mode: flyspell
% ispell-local-dictionary: "british"
% End:
