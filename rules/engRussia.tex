\sectionJ{\anchorpaysmajeur{Russie}}{\blason{russie}}



\subsection{Russian under-development}

\aparag[Economic weakness.]
\RUS has a malus of {\bf -1} to die-rolls when attempting to raise its
\DTI, \FTI or to place \MNU. This malus ends when
\ville{Saint-Petersbourg} is finished building.

\aparag National provinces of \RUS are wasteland provinces, see
\ruleref{chMilitary:Movement:Wasteland} and
\ruleref{chLogistic:Fortresses Wasteland}.  Both effects end when
\ville{Saint-Petersbourg} is finished building, or at the end of
\eventref{pVI:Great Northern War}, whichever is
first.\label{chSpecific:End Wasteland}

\aparag[Construction of \anchorville{Saint-Petersbourg}]%
\label{chSpecific:Russia:St-Petersburg}
\bparag A new major Russian city can be built on the Baltic sea,
beginning with period V. % in \province{Neva}.
% [Jym : j'ai enleve les autres car Neva est nationnale (donc toujours
% prenable) et wasteland (donc moins fortifiable et de toutes façons
% plus facile à prendre). TBD] PB: Certes mais historiquement tr�s
% plausibles donc je les remets.
The Russian player decides of a province whose city he controls among
\province{Karelen}, \province{Neva}, \province{Estland},
\province{Livonija} and \province{Kurland}.
\bparag \RUS has to spend 100\ducats per turn during 3 turns
(consecutive or not). Such an expense can not be made if the city is
besieged or the province is pillaged. If ever the \RUS loses the
military control of the province, the process will have to be renewed
from the start.
\bparag On the first spending, put the
\anchorconstruction{Saint-Petersbourg} fortress counter on level
0. Increase it by one level for each turn the spending is done. If
besieged, the city has a fortress level that is the maximum of the
instrinsic or regular fortress of the province and the current
\construction{Saint-Petersbourg} fortress.
\bparag If \RUS controls the city at the end of a turn following the
third expense, and the city is not besieged, then he annexes the
province immediately (with no need of peace treaty) and puts here a
fortress of level 3, using the \construction{Saint-Petersbourg} counter.
The intrinsic minimal fortification of the city is now the level 3 if
\RUS controls it, and 2 if conquered by another power.  Note that the
maximum level is the one authorised by the land technology of \RUS.  The
level of the fortress can then be raised using usual rules.
\bparag The former fortresses and cities in the province do not exist
any more, for the remainder of the game.

\bparag The income of the province for the \RUS equals the normal income
multiplied by the level of the fortress in the province, with a maximum
of 20\ducats.
\bparag If another player pillages the province, its uses the Russian
income.  If the province is ceded to another player, the previous
(unmodified) income is used by this player; the fortification is of
level 3 and can not be raised. The city remains
\ville{Saint-Petersbourg}.
\bparag[Naval Shipbuilding.] The construction limit of \ND per turn of
\RUS is raised by 2 when \RUS controls \ville{Saint-Petersbourg}.  It
also gains one action of Concurrence, its limits of \DTI is increased by
one, and the limit of \FTI is increased by one in period VII.
\bparag[A new capital.] If \monarque{Peter the Great} is the Russian
Monarch, or if its reign has ended, \ville{Saint-Petersbourg} becomes a
new capital of Russia (who has from now on 2 capitals).

% Jym, 05/2011
% Removed. The NT of RUS is now small enough (in the South).
%\aparag[Recruitment Area.]\label{chSpecific:Russia:Recruitment Area} The
%\terme{Recruitment Area} of \RUS is limited to its capital province
%\province{Moskva} and the one used to build \ville{Saint-Petersbourg}.

\aparag[Arkhangelsk and the Russia \CTZ.]  Arkhangelsk is a Russian port
on the White Sea, located in Europe, but effectively out of the European
map. It cannot be attacked, blockaded or conquered.
\bparag[Creation of the Port of Arkhangelsk.]  Arkhangelsk is created by
the event \eventref{pIII:Creation Arkhangelsk}, or automatically in 1615
(turn 26) if the event did not occur. As long as the port is not
created, the Russian player cannot build any commercial fleets.
\bparag If Arkhangelsk is created only in 1615 (and not by event) the
advantages of the Muscovy Trade companies to England (see event
description) are not applied.
\bparag[Russia \CTZ.]
The Russia \CTZ does not exist before the creation of Arkhangelsk. Once
that port is created, the \CTZ brings a monopoly income of 5
\ducats. This income is increased to 10\ducats once
\ville{Saint-Petersbourg} is created.

\aparag[Colonial expansion] The \COL and \TP built by \RUS must be at
supply distance by land from either European provinces of \RUS or from
another \COL or \TP.
\bparag For this rule only, the \granderegion{Kamchatka} is considered
``at supply distance by land'' to the provinces touching the impassable
area north of the \seazone{Okhotsk}, to the coastal provinces of
\granderegion{Amour} and to the provinces of \granderegion{Alaska} (see
also \ruleref{chBasics:Secret Passage:Bering}).
\bparag Similarly, all the provinces of \granderegion{Alaska} are ``at
supply distance by land'' to the provinces of \granderegion{Oregon} (but
a \COL or \TP has to be put in \granderegion{Oregon} before reaching
\granderegion{California}).

\aparag[\leaderYermak] [BLP] \leaderYermak may use the table of
conquistadors in \continentSiberia.

\aparag[Foreign trade index] \RUS has a specific \FTI for \COL and \TP
operations, that is different from its \FTI (see
\ruleref{chAdministration:Special FTI}).

\aparag[\anchorconstruction{Sebastopol}]

\subsection{The Russian military system}

\aparag[Russian conquests.]  When the Russian player wins a war and
receives provinces, he can annex the province containing the minor
country capital. In this case, the minor has a new capital in another
province (chosen by its controlling player). If the \MIN has no province
left, it is destroyed.
\bparag[Validity.]  This is valid only if the conquered province is
adjacent to a Russian province, connex by land to a Russian National
province, and occupied by a Russian military unit (not by a Russian
minor ally or vassal).

\aparag[Russian Boyars Army.] \label{chSpecific:Russia:Boyars Army} Markers of
both the land and naval technology of Russia can never be higher than
the boxes where the ``Orthodox'' minor entities technology markers
are. This is enforced at the end of the administrative phase (after
possible moves of minors and mobile markers).
\bparag The Russian player can use no more than 5 army counters and 1
fleet counter before a reform. The number and types of detachments are
not limited.
\bparag[Pugnacity.] \RUS will be forced to sue for peace only if it
stands for 3 consecutive turns at -3 \STAB level (instead of the regular
2 turns).

\aparag[Russian army reform.]\label{chSpecific:Russia:Army Reform} \RUS can
reform its army using one of the two following possibilities:
\bparag It is the reign of \monarque{Peter the Great}; roll for 1 revolt
in \RUS.
\bparag It is in period VI or VII; roll for 3 revolts in \RUS and \RUS
loses 1 \STAB.

\aparag[The New Russian Army.]
\bparag The number of counters increased to 6 \ARMY and 3 \FLEET.
\bparag The number of artillery in each \ARMY is increased.
\bparag \RUS is not limited in Technology levels.  \RUS is now both a
``Latin'' and ``Orthodox'' \MAJ: the ``Orthodox'' Land Technology will
still be dragged by the progression of \RUS but \RUS may use the
``Latin'' markers for technological lateness bonuses.
\bparag Diminish by 1 \LD the limit per turn of land force building.
\bparag The rules of \localruleref{chSpecific:Russia:Boyars Army} are not
applied anymore.

\aparag[The Cossacks.]\label{chSpecific:Russia:Cossacks} If the Russian power
controls one or more of the provinces of \pays{Ukraine} (as Vassals or
by annexation): \province{Ukrainya}, \province{Poltava},
\province{Podolie}, \province{Zaporozhye}, \province{Don},
\province{Donets}; or if \RUS owns all provinces in \pays{Kazan} or
\pays{Astrakhan}, it receives the following advantages.
\bparag It may use the 4 Cossack \LD as its own forces. They may be
incorporated in armies. By exception, these \LD are always
\terme{Conscripts}.

\bparag He raises one free Cossack \LD each turn in one of these
provinces, and has 2 \LD added to its basic forces (by exception, these
\LD are maintained as \terme{Conscripts}).

\bparag In period III and IV, it has each turn a free simple campaign to
move a force in \granderegion{Siberie}, and attack. This campaign may be
added to a regular campaign during any round.

\aparag[Fluvial Port.]\label{chSpecific:Russia:Fluvial Port} Beginning with
\monarque{Peter the Great}, Russia can use the rivers in Ukraynia as a
fluvial ports. One \FLEET can be built then stored on one of the
following rivers, in specified provinces:

\bparag on the Don river (in \province{Don}) or the Donets river (in
\province{Donets}) acting as a port on \seazone{Noire W} only, and
having the possibility to blockade \ville{Azov} only;

\bparag on the Dniepr river (in \province{Zaporozhye}) acting as a port
on \seazone{Noire E} only, to put blockade on \ville{Odessa};

\bparag on the Volga river (in \province{Samara}) acting as a port on
\seazone{Caspienne} only, to put blockade on \ville{Astragan}.

\bparag There can be at most one such \FLEET at the same time (but it
can be destroyed and built anew on the same or another river).  Until it
gains a proper port, it can only operates for a blockade on the
specified fortress, or for naval interceptions and battles in the
specified sea zone.  It has to go back at part at the end of each turn.
\bparag The \FLEET can be blockaded (at the mouth of the river) as if it
was in a port bordering the sea.  If its port province is not available
at the end of a turn, the \FLEET is destroyed.



\subsection{Religious Attitude of Russia}\label{chSpecific:Russia:Orthodoxy}

\aparag In 1492, \RUS is seen as the Champion of the Orthodox religion.
When \eventref{pI:Reformation} occurs, it may change this attitude to
\terme{Religious Tolerance}, or remain with attitude \terme{Championship
  of Orthodoxy}.

\aparag[Religious Tolerance.]  Add one diplomatic action to \RUS in
periods I to IV. \RUS has no malus to diplomacy because of religious
troubles between Christians. \RUS loses the free maintenance of one
\ARMY\facemoins for the rest of the game, and can not benefit of the
rules about Cossacks (see \ruleref{chSpecific:Russia:Cossacks}).

\aparag[Championship of Orthodoxy.] This is the historical option. No
change has to be made.



\subsection{\sectionpaysmajeur{Russie} in play}


\subsubsection{Great Russian Monarchs}
\aparag[\anchormonarque{Ivan III}] is the monarch in 1492, with values
6/7/8, that dies at the beginning of turn 4.
\aparag[\anchormonarque{Ivan IV}] is the first monarch to begin its
reign after period I (turn 7 or later). Ivan the terrible begins his
reign as a child and will last 11 turns. His values are 6/9/8 (remember
to lower them by 2, then 1, during the first 2 turns). He does not test
for survival during the first seven turns of his reign.
\bparag He is a general \leaderwithdata{Ivan IV} from the third turn of
his reign on.
\bparag \RUS gains one \ARMY\faceplus of \terme{basic forces} during his
reign, beginning with the third turn.
\bparag Event \eventref{pIII:Oprichnina} depends on \monarque{Ivan IV}.

\aparag[The Time of Troubles] Due to event \eventref{pIV:Times of
  Troubles}, \anchormonarque{Godunov} of values 5/8/4 (and general
\leaderwithdata{Godunov}) may rule in \RUS, and be followed by either
\anchormonarque{Romanov} (values 6/5/6) or \anchormonarque{Dmitry}
(values 4/7/5 and general \leaderwithdata{Dmitry}). See the conditions
in the event.

\aparag[\anchormonarque{Peter the Great}]\label{chSpecific:Russia:Peter the Great}
is a special Russian monarch who arrives by one of the 2 following
conditions:
\bparag event \eventref{pV:Peter the Great} happens;
\bparag it is period V or after and the Russian monarch is adult, has at
least 8 in ADM and 18 in the sum of his characteristics.
\bparag This monarch is then \monarque{Peter the Great}. It may enter
only once per game. \monarque{Peter the Great} has the values 9/9/9 as a
monarch (regardless of what could have been obtained), is also a general
\leader{Piotr I} and an admiral. He reigns 7 turns, with no survival
test during the first 5 turns.
\bparag The \terme{basic forces} of \RUS is raised by one \ARMY\faceplus
during his whole reign.
\bparag At the moment when the Monarch is known as \monarque{Peter the
  Great}, the \STAB of \RUS increases of 2.

%% Pierre II par Jym, d'apres discussion a Bordeaux.
% Modif PB a faire : va vraisemblablement �tre un �v�nement.
\aparag[\anchormonarque{Pierre II}] is the first sovereign whose reign
begins in period VII. He has values 3/3/3 and his reign last 1 turn. At
the time he takes power, \RUS makes a mandatory white peace with all its
enemies. His successor is \monarque{Catherine II}.

\aparag[\anchormonarque{Catherine II}] has values 7/9/8, and her reign
last 5 turns. She does no test for survival during the first 3 turns.
She cannot be used as general. The \terme{basic forces} of \RUS is
raised by one \ARMY\faceplus during her whole reign.

\aparag[\anchorministre{Potemkine}] may be named minister through
\eventref{pVII:Potemkin}. He has values 9/8/8 and remains a random
number of turns; its values can be used for the next monarch's values
determination if a succession takes place while he is still alive.


\subsubsection{Available counters}
\aparag[Military] 6 \ARMY (5 usable at start), 3 \FLEET (1 usable at
start), 1\corsaire, 10\LDND, 10\LD, 2\NTD, 8\LDENDE, 4 fortresses 1/2, 4
fortresses 2/3, 2 fortresses 3/4, 1 fortresses 4/5, special
\ville{Saint-Petersbourg} counters, Arsenal 2/3 \ville{Sebastopol}, 10
forts.
\aparag[Economical] 11\COL, 5\TP, 8\MNU, 7\TradeFLEET, 2 \ROTW treaty
counters.

% LocalWords: malus Petersbourg Boyars pIII Oprichnina Kurland Khanates
% pVI pV LocalWords: Lithuanie Estland Livonija Karelen Podolie Siberie
% Noire pVII pI LocalWords: Moskva Kaluga Vyatka Crimee Ukrainya pIV
% Russie
