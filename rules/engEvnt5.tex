% -*- mode: LaTeX; -*-
\chapter{Political Events of Period V}
%\section{Period V}
\label{events:pV}



\subsection*{Event Table of Period V}

\begin{eventstable}[Period V events table]
  \tabcolsep=5pt\centering%
  \begin{tabular}{|l|*{5}{c}|l|}
    \hline
    1\up{st}\textarrow& 1-4 & 5-6 & 7 & 8 & 9 & 10 \\ \hline
    1 & 1  & 7  & 1  & 21  & R3   & \textbullet~1--2:\\
    2 & 2  & 8  & R2 & R22 & R4   & +1 then\\
    3 & 3  & 9  & R3 & 2   & 5    & \nameref{events:pIV}\\
    4 & 4  & 10 & 4  & 3   & 6    & \textbullet~3--10:\\
    5 & 5  & 11 & 6  & 9   & R16  & \nameref{events:pIV}\\
    6 & 6  & 12 & 7  & R10 & 17   & \\
    7 & 14 & 13 & 15 & 12  & 18   & \\
    8 & 17 & 15 & 23 & 13  & R19  & \\
    9 & 18 & 16 & R4 & 14  & R20  & \\ \hline
    10 & \multicolumn{6}{l|}{1--6 \nameref{events:pVI}, 7--10 \nameref{events:pIV}} \\ \hline
  \end{tabular}
\end{eventstable}

\eventssummary{%
  pV:Devolution War|,%
  pV:Chamber of Reunion|,%
  pV:League Augsburg|,%
  pV:Glorious Revolution|,%
  pV:WoSS|,%
  pV:Colbertian Mercantilism|,%
  pV:Expulsion French Protestants|,%
  pV:Grand Siecle|,%
  pV:English Dynamism|,%
  pV:Montecuccoli to Eugen|,%
  pV:de Witt|,%
  pV:Peter the Great|,%
  pV:Saxon King Poland|,%
} \eventssummary{%
  pV:Kingdom Prussia|,%
  pV:War Sweden Denmark|,%
  pV:Koprulu|,%
  pV:Fights Iroquois|,%
  pV:Slave Revolts WI|E/E,%
  pV:Wars India|E/E,%
  pV:Treaty Nerchinsk|,%
  pV:Invasion Formosa|,%
  pV:Japan Trade|,%
  pV:Revolt Cossacks|O{pIV:Revolt Cossacks},%
  pV:Revolt Catalunya|,%
  pV:Hungary|,%
  pV:Transylvania|,%
  pV:Cretan war|,%
  pV:Morean war|,%
  pV:Revolt Pueblos|,%
  pV:Tangiers|,%
  pV:Khoikhoi|E/E,%
  pV:Bill Test|,%
  pV:Kuruc|,%
}

\newpage\startevents



\event{pV:Devolution War}{V-1}{War of Devolution}{1}{Risto}

\history{1667-1668}

\condition{Can occur only if \FRA is not currently in a war (including Civil
  Wars).  Otherwise, re-roll.}

\phdipl
\aparag \FRA receives a free \CB for this turn against one owner of either
\provincePicardie, \provinceArtois, \provinceFlandre or \provinceHainaut.
This event is triggered off by \FRA using this \CB to declare a war, and if it
declines to do this the rest of the event does not occur.
\aparag \HOL and \ENG may each sign a Defensive Alliance with the victim of
\FRA declaration of war per above, provided both sides agree, immediately at
this turn or on the following turn.  The alliance provides an immediate \CB as
reaction against the declaration of war of \FRA.

\phadm
\aparag \FRA can collect incomes in the above mentioned provinces whenever
they are militarily conquered by \FRA.

\phpaix
%\aparag If victorious, \FRA receives 30 \PV at the end of this war.
\aparag If \FRA is not victorious, \HOL receives, if it was at war, 30 \PV at
the end of a war against \FRA triggered by this event.



\event{pV:Chamber of Reunion}{V-2 (1)}{Chamber of Reunion}{1}{Risto}

\history{1681-1684}

\condition{}
\aparag Cannot occur if there is a German Empire. In that case mark off, but
do not consider as played for the first time.
\aparag Cannot occur if \provinceAlsace is not part of \paysAlsace.  In that
case mark off and considered as played for the first time.

\phevnt
\aparag \FRA annexes \provinceAlsace. This provides \SPA, \HOL, \ENG and \AUS
a temporary \CB against \FRA for this turn.
\aparag If \FRA currently militarily occupies \provincePicardie,
\provinceRosselo, \province{Franche-Comte} and/or \provinceArtois, it can
immediately annex any such province without any peace treaty.

\phdipl
\aparag The current Emperor (or \SPA if \AUSmin is Emperor) receives a bonus
of \bonus{+3} for its diplomacy on all \HRE minors this turn.



\event{pV:League Augsburg}{V-2 (2)}{War of the League of Augsburg}{1}{Risto}

\history{1688-1697}

\condition{}
\aparag Can occur only if \FRA is not involved in a war (including civil
war). Otherwise re-roll.
\aparag Cannot occur if \ref{pV:Devolution War} has not already been
finished. Otherwise re-roll.

\phevnt
\aparag \FRA may immediately annex one of the following provinces:
\provincePicardie, \provinceRosselo, \province{Franche-Comte},
\provinceLuxemburg, \provinceAlsace or \provinceLorraine.  Such annexation is
regarded as a free declaration of war against the owner of the province chosen
(unless eliminated in the process).
\aparag If \FRA uses this opportunity to annex a province, \HOL and \ENG
receive a temporary free \CB against \FRA for this turn. They do not
necessarily have to be in alliance with the victim of French aggression or
with each other (but they may decide so if both sides agree).



\event{pV:Glorious Revolution}{V-3}{The Glorious Revolution in
  England}{1}{PBMod}

\history{1688-1690}

\condition{}
\begin{todo}
  If \ANG is \CATHCR?
\end{todo}
\aparag If \ENG is \PROTRIG:
\bparag Put a \REVOLT \facemoins in each Irish province except \provinceUladh,
one \LD and one general in one of the revolted provinces. \ENG is not in Civil
war, the \REVOLT are controlled by \HOL.
\bparag \paysEcosse declares war on \ANG (breaking any alliance it may have
with \ANG) and call for allies as per normal rules.
\aparag Otherwise (\CATHCR, \CATHCO or \PROTANG), use the rest of the event.
% (Jym) What if \ANG is \CATHCR? (JCD) Rest of the event seems to support that
% the event is used anyway.

\phevnt
\aparag \ENG is considered to have overthrown its current monarch. \ENG is now
in Civil war between two sides: the Rebels, called ``Royalists'' (followers of
the old king) are \CATHCR , and the Loyalists, called ``Orange'' are \PROTANG
(see \ref{chDiplo:Religious Civil War}).
\bparag The Royalists are controlled by a Catholic \FRA, or \SPA
otherwise. They use the counters of \paysroyalistes.
% \bparag Roll for the statistics of the new English monarch from the House of
% Orange. If the same House is ruling in \HOL, both countries share an offensive
% alliance and a mandatory defensive alliance. They make an immediate white
% peace.
\begin{todo}
  \ANG choose be able to choose the order in which he propose the crown to
  other protestant countries.

  Clarify the rules for the union in case ``Orange'' is not \HOL.
\end{todo}
\bparag The loyalists are controlled by the English player and use the
counters of \ANG. They are automatically allied with the first country in the
following list who accept: \HOL, Protestant \FRA, \SUE. These countries are
allied as per (REF NEEDED, See Special Rule for \ANG) and immediately makes a
white peace.
\aparag In support of the overthrown monarch, two \REVOLT are rolled for in
England. Furthermore a \REVOLT \faceplus is placed both in \provinceConnacht
and \provinceMumhan and the rebels control both fortresses. A \LD and a
general are placed in one Irish province.
\bparag If this event is caused by \ref{pIV:English Restoration}, a royalist
\ARMY \faceplus is raised in \provinceCymru (or in any province of Scotland
if~\ref{pIV:Union Scotland} is effective). The Royalists control the fortress
in this province and one other (or two other provinces if in
Scotland). Otherwise, Royalists get an \ARMY\facemoins and control of the
fortress in this province and one other.
\aparag If~\ref{pIV:Union Scotland} is effective, \paysEcosse allies itself to
\paysroyalistes and is at war with the Loyalists (with no declaration of
war).

\phdipl
\aparag The controller of the rebels has a \CB against \ENG to make a limited
intervention against \ENG this turn, that can become a full intervention on
the second turn. If \ENG was \CATHCR or the event was caused by
\ref{pIV:English Restoration}, the controller may make a full intervention
from the first turn on.

\phadm
\aparag The Royalists roll for reinforcements in offensive or naval status
(but with \bonus{-2} for naval).
\aparag All reinforcements must be placed in a province with existing rebel or
allied units, not just \REVOLT or cities. If none, no reinforcements are
received.

\phpaix
\aparag Peace is determined with usual rules except that:
\bparag The Royalists surrender unconditionally if they have no forces nor
\REVOLT left (fortresses do not count).
\aparag If the the new English king is overthrown by \REVOLT , it also
surrenders unconditionally to the Royalists and their controller.
\aparag[Victory of Royalists] If the Royalists win (alone or with their
controller), the king is restored (with his values as a monarch) and the House
of Orange is expelled.
\bparag \ENG becomes \CATHCR (except if it was \CATHCO, in which case it
remains so). It loses 50 \PV.
\bparag \xnameref{pVI:Act Union} is broken. If it did not happen yet, it may
occur later.
\aparag[Total Victory of Royalists] If the Royalists and their controller
(making a full intervention) impose an unconditional surrender to \ENG,
additional consequences are:
\bparag \ENG makes a mandatory Dynastic Alliance with the controller of the
Rebels and must give a \COL or \TP as dowry.
\bparag \xnameref{pVI:Act Union} is broken. If it did not happen yet, it may
not occur later (with some modifications).%  TODO This paragraph is subject to
% debate
% (Jym) Total defeat allows the Act of Union but not partial defeat?  (JCD)
% Yes, WTF ? The Act of Union is more straightforward about that
%
% (Jym, 2013) Pierre's notes from 2008 seem to be in the opposite
% direction...
\bparag \ENG makes a mandatory offensive alliance with the controller of the
rebels for 2 turns. It cannot declare war against it (except with \CB from
events; in this case the alliance has to be broken with the usual cost in
\STAB).



\event{pV:WoSS}{V-4}{The War of Spanish Succession}{1}{PBMod}

\history{1700-1713}

\activation{}
\aparag This event cannot occur before period V. Re-roll and do not mark off
if this is not the case.
% ELSE: before the end of the Religious struggles.
\aparag When the event occurs, its effects are not actually applied. They will
be triggered at the death of the current Spanish Monarch.
\aparag If there is a \GE, see the specific modifications in
\ref{pIV:TYW:German Empire}.

\tour{Death of the Monarch}

\phevnt
\aparag \SPA may concede immediately white or losing peace to all its current
enemies. Unaligned \MIN always accept a white peace.

\phdipl
\aparag \SPA designates an heir to the Spanish throne. The choice must be made
among the following countries:
\bparag One of the following \MAJ that is Catholic: \FRA, \AUS, \ENG;
\bparag \AUSMin;
\bparag Another Catholic minor country.

\aparag A \MAJ may decline the offer, but cannot then take part in any war
ensuing from this event, nor can it be positively affected by the event (for
objectives or any possible gain in the event).
\bparag In that case, \SPA proposes a different Heir, and so on, until one
accepts (minor powers always accept).
\bparag The power that accepts will be designated as the \emph{Heir} in the
rest of the event.
\bparag If the Heir is a minor power, all its decisions are made on its behalf
by \SPA.

\aparag If \AUSaus is not the chosen Heir, the dynastic alliance between the
Habsburg powers is now cancelled.
\bparag \AUSMin becomes also the major \AUS.

\aparag The Heir has to propose a settlement for the Spanish
possessions. Three attitudes are possible:
\begin{itemize}
\item \xnameref{pV:WoSS:Integrity Inheritance}
\item \xnameref{pV:WoSS:Seizing Inheritance}
\item \xnameref{pV:WoSS:Dividing Inheritance}
\end{itemize}

\aparag Several parts of the Inheritance are desired by some Major Powers.
Here is the list of the different parts at stake, especially the regional
groups for all province owned by \SPA that are not in its National territory
and the \MAJ that can be nominated for receiving these parts:
\bparag[Spanish Low Countries] In national territory of \paysmajeurHollande or
in former country \paysBourgogne except for \province{Franche-Comte}.
Interested: \FRA, \ENG, \AUS, \HOL, \SPA.
\bparag[South Italy] Provinces of Kingdom of Naples and Sicily (\paysNaples).
Interested: \FRA (if Catholic), \ENG (if Catholic), \AUS, \SPA.
\bparag[North Italy] All the remaining provinces in \regionItalie and
\payssuisse (except \provinceNice) plus \provinceMalta.  Interested: \FRA (if
Catholic), \AUS, \SPA.
\bparag[French Borders] All the provinces adjacent to or in French National
Territory that are not in one of the previous groups (that includes
\provinceNice, \province{Franche-Comte} and \provinceRoussillon).  Interested:
\FRA, \AUS, \SPA.
\bparag[North Africa] All provinces and \Presidios in North Africa.
Interested: \FRA (if Catholic), \ENG, \SPA.
\bparag[The Remaining] All other European provinces owned by \SPA that are not
its National Territory.
\bparag[Mediterranean Concessions] \provinceGibraltar, \provinceBaleares and 1
\COL (of \HIS or a major heir).  Interested: \ENG, \HOL, \SPA.
\bparag[Dynastic link and alliance with Portugal] This can only be chosen if
\paysportugal is either annexed by \HIS as per \ref{pIII:POR Ann.:Portugal
  Annexed} or if \ref{pVI:Methuen:Normal} did not happen yet and \paysportugal
is on the diplomatic track of \HIS. Apply immediately \ref{pVI:Methuen:WoSS}
with the \MAJ taking this spoils has the beneficiary of the Treaty and
consider that event played. Interested: \FRA, \ENG, \HOL, \SPA.
% (Jym) réécriture pour tenter la compatibilité avec le traité de Methuen...
% (Jym) Formulation alambiquée...  Si Methuen a eu lieu, a provoqué une 2ème
% révolte de POR que SPA a re-gagné, cette formulation laisse la possibilité
% de le voler maintenant...  (Pierre) : Applicable only if \ref{pVI:Treaty
% Methuen} is not active yet, and \SPA has still dynastic ties with \POR. For
% \SPA, this is the reaffirmation of Dynastic Ties and Pretense over \POR. For
% other \MAJ, this condition voids the effect of the Spanish Dynastic Ties
% with \POR and it activates an Alliance analog to the Treaty of Methuen for
% the \MAJ that takes it. Interested: \FRA, \ENG, \HOL, \SPA.
\bparag[Asiento] See \ref{chSpecific:Spain:Asiento}. Interested: \FRA, \ENG,
\HOL, \SPA.
% (Jym, 2013) If I understand Pierre's notes correctly:
\bparag[Colonial Empire] Two \COL of \HIS or the heir (if \MAJ). Interested:
\ANG, \FRA, \HOL, \HIS.

\aparag The attitude chosen gives the Heir some constraints on the Inheritance
project (which groups are attributed to which power).
\bparag Note that for \AUS, some groups count only as half: \emph{North
  Italy}, \emph{South Italy}, \emph{French Borders}.

\begin{digressions}[The Inheritance Project]
  \bgroup\def\EUEVtypeofdigression{choice}


  \digression[pV:WoSS:Integrity Inheritance]{Integrity of the Inheritance.}
  \phase[(Diplomatic before the war, Peace after the war)]{Diplomatic or Peace
    phase}{}
  \aparag The Heir decides to keep all provinces Spanish.
  \aparag The Heir obtains a compulsory offensive alliance lasting 5 turns
  with \SPA. \SPA must always honour this alliance, if called to do so. It
  cannot make a separate peace from the Heir, unless compelled to do so by
  enforced surrender. It is also considered as a Dynastic Alliance.
  \aparag The Heir may take one of the following advantage: \terme{Dynastic
    link and alliance with Portugal}, \terme{Asiento}, \terme{Mediterranean
    Concessions} or \terme{North Italy} (\AUS only for this last one) if
  interested.
  \aparag Then \SPA cedes two provinces of its choice to the Heir.


  \digression[pV:WoSS:Seizing Inheritance]{Seizing the Inheritance.}
  \aparag The Heir takes any or all the groups at stake defined above as
  interesting him.
  \aparag The Heir obtains a compulsory offensive alliance lasting 3 turns
  with \SPA. \SPA must always honour this alliance, if called to do so. It
  cannot make a separate peace from the Heir, unless compelled to do so by
  enforced surrender. It is also considered as a Dynastic Alliance.


  \digression[pV:WoSS:Dividing Inheritance]{Dividing the Inheritance}
  \aparag The Heir decides to share the spoils of the Spanish possessions with
  other Powers. It may propose any/all of the groups above to Powers that have
  interest in the share, and can take some of them for its own sake.
  \bparag Choosing this option costs 1 \STAB to the Heir plus 1 \STAB per part
  of the inheritance given to someone else than \HIS or the heir, as well as
  15\VPs per part given to someone else than \HIS or the Heir (due to its
  bargaining of the Heirdom) (or to \SPA of the Heir is a \MIN). The \STAB has
  to be paid, if the heir (or \HIS) has not enough \STAB, it may not give more
  parts.
  \bparag Each power may obtain at most two groups.
  \aparag The Heir obtains a compulsory defensive alliance lasting 3 turns
  with \SPA.  \SPA must always honour this alliance, if called to do so, yet
  it can make separate peace if it wants. It is also considered as a Dynastic
  Alliance.
  %
  \egroup
\end{digressions}

\begin{digressions}[Conditions of the War of Spanish Succession]


  \digression[pV:WoSS:War Spanish Succession]{War of Spanish Succession}

  \phdipl
  \aparag Some powers (if not chosen as Heir) may want to contest the
  Inheritance and declares a War to both \SPA and its Heir, jointly: \HOL,
  \FRA, \AUS, \ENG.
  \bparag They have a free \CB to do so.
  \bparag All the powers contesting the Inheritance are automatically in the
  same Alliance, called the Opposing Alliance.
  \bparag As par usual rules, other \MAJ may be called to participate in one
  or the other Alliance.
  \bparag If the Heir is a minor power, \SPA leads the Heir alliance and a
  Separate Peace against this minor does not affect the war.
  \bparag If the Heir is a major power, it decides for the Alliance (excepted
  if out of the war before \SPA).

  \aparag If none contest the Inheritance, this ends the event and the Heir
  and \SPA are deemed to have won the War, and all the other powers to have
  lost it.

  \aparag If \ref{pV:WoSS:Dividing Inheritance} has been taken, a power to
  whom at least one group has been proposed has the choice, in case there is a
  war, to contest the Inheritance (as per above), or to support the Division
  and join the Heir Alliance. In that case, it has to declare war and has a
  \CB to do so.

  \aparag If there is a war, any country that is not in one of the Alliances
  forfeits all possible benefits due to the war.

  \aparag The Heir, \SPA and the \MAJ in their Alliance take all the groups
  they are entitled by the chosen Inheritance attitude immediately. Those
  gains are temporary in the sense that they may revert to other powers
  depending on the result of the war. The Opposing Alliance powers will
  receive nothing before the end of the war.

  \aparag[Maximilian's change of side] [BLP] If the Heir is not the emperor
  and there is a war, the Heir may choose one electorate. For the duration of
  the war, he has a bonus of \bonus{+5} for diplomacy on this
  minor. Exceptionally, diplomacy may be made on this minor even if it is at
  war.

  \phadm
  \aparag For the duration of the event, \ENG receives the use of the leader
  \leader{Royal Marines}. This is in addition to the normal limits.


  \digression[pV:WoSS:Peace Spanish Succession]{Peace following Spanish
    Succession}

  \phpaix
  \aparag The result of the war depends of the level of the peace signed
  between the Alliances. The War ends when \SPA or its Heir is making Peace
  and the other is doing the same or is already out of the war.
  \aparag In this Peace, the victory condition is first the application (or
  not) of the proposed Inheritance project, second the giving of some of the
  groups presented before as compensations. To them, one adds the following
  groups (that are spoils for war only):
  % (Jym) Adding:
  \bparag[Dynastic link and alliance with Portugal] At the peace, this can
  also be chosen if \paysportugal was given to a country in the opposing
  alliance at the beginning of the war. It is not possible to choose this
  compensation at peace if \ref{pVI:Methuen:Normal} was triggered as a regular
  event and gave the Portuguese alliance to a country other than \HIS.
  % (Jym) Remet complètement l'alliance portuguaise en jeu à la fin de la
  % guerre, comme les autres compensations. Précautions à prendre pour que ça
  % ne soit remis en cause que si ça avait été donné pour "acheter" un MAJ au
  % début de la guerre. Si l'alliance est antérieure à la guerre, elle ne peut
  % plus bouger.
  \bparag[Territorial Concessions] Give any two provinces to any power (only
  province not in a group given to anyone, except \SPA). In priority:
  provinces adjacent to provinces already owned by the \MAJ.  Interested:
  \FRA, \ENG, \AUS.
  \bparag[Independence of Catalunya] Only if a \REVOLT or the Opposing
  Alliance controls \provinceCatalogne: it becomes an independent minor
  country.  Counts as half an objective only.  Interested: \FRA, \ENG.
  \bparag[Olivares politics cancelled] This nullifies the effects of
  \ref{pIV:Olivares}.  Counts as half a group objective only.  Interested:
  \FRA, \ENG, \HOL.
  \aparag If the Heir Alliance is victorious, with a PD of 3 or more: the
  proposed Inheritance project is applied completely.
  \aparag If the Heir Alliance is victorious, with a PD of 1 or 2: the
  proposed Inheritance project is applied but the Heir has to give a group as
  a compensation to one of the \MAJ in the enemy alliance (chosen by the
  Heir).
  \aparag If a white Peace is signed: the proposed Inheritance project is
  applied but the Heir has to give two groups as a compensation to \MAJ in the
  enemy alliance (proposed by the Heir).
  \aparag If the Opposing Alliance is victorious, with a PD of 1: the proposed
  Inheritance project is applied but the Heir has to give two groups as a
  compensation to \MAJ in the enemy alliance (chosen by the Opposing
  Alliance).
  \aparag If the Opposing Alliance is victorious, with a PD of 2: the proposed
  Inheritance project is not applied. The Opposing Alliance decides of a new
  Inheritance project based on the rules of \xnameref{pV:WoSS:Dividing
    Inheritance} that is applied and cannot be contested.
  \aparag If the Opposing Alliance is victorious, with a PD of 3 or more: the
  proposed Inheritance project is not applied. The Opposing Alliance decides
  of a new Inheritance project based on the rules of
  \xnameref{pV:WoSS:Dividing Inheritance} that is applied and cannot be
  contested. The restriction that at most 2 groups may be given to a power is
  lifted.
  \aparag If \HAB was the Heir and the Inheritance project is overruled, the
  Dynastic Alliance between the Habsburg ends and \AUSmin becomes \AUS.
  \aparag If \SPA is victim of an Unconditional Peace, the new dynasty is
  overthrown.
  \bparag The Heir loses 30 \PV and the Dynastic Alliance is cancelled.
  \bparag \SPA lose all the groups at stake in the Inheritance.
  \bparag If the war still goes on, they are temporarily given to the Heir
  until the end of the war.  If the Heir wins the war anyway, any group that
  should have been attributed to \SPA is considered to be his before applying
  the Peace conditions.  If there are groups he is not interested into that
  are still his afterwards, he has to freely give them to any power (including
  \SPA, as an exception to this rule and the following).
  \aparag If a power makes a Separate Peace, it forfeits all the possible
  benefits to be gained in the war (all the groups mentioned before).
  \bparag If it already had any (thanks to a Division of Inheritance), the
  objective are given back to \SPA (or the Heir if \SPA is out of the war).
  \aparag If, at the end of the war, \provinceNaples is owned by someone else
  than \HIS (or an autonomous \VASSAL of \HIS), then \paysSavoie annex
  \provinceSaldigna
  \begin{todo}
    It should be both provinces of Sicily, exchanged for Sardinia
    after~\ref{pVI:WoQA}.
  \end{todo}

  \effetlong
  \aparag \provinceGibraltar becomes an \terme{arsenal} if attributed and
  owned by this event to a player that is not \SPA.
\end{digressions}



\event{pV:Colbertian Mercantilism}{V-5}{Colbertian Mercantilism in
  France}{1}{RistoMod}

\history{1661-1683}

\condition{}
\aparag \FRA may decline the event if he wants so. Mark off the event as
played and ignore the rest.

\phevnt
\aparag \FRA receives an Excellent Minister \strongministre{Colbert} with
values 8/9/8. He will last a random length for Minister, see
\ref{eco:Excellent Minister}.
\aparag All major powers with commercial fleets in \ctz{France} must pay 10
\ducats per level they want to keep. The money goes to French treasury. All
minor commercial fleets in \ctz{France} are permanently removed (their
reference level is 0).
\aparag Moreover, if either \CATHCR or \CATHCO, \FRA receives 5 levels of
\TradeFLEET in \ctz{France}.  Mandatory competition is solved immediately if
need be.
\aparag All major powers who lost at least two levels of \TradeFLEET in the
process have an \OCB against \FRA until the end of the next period.
\aparag \FRA receives an additional \RES{Art} \MNU level, if available.

\phdipl
\aparag For the rest of the game, \FRA has an \OCB against everyone with
\TradeFLEET in \ctz{France}, and a \CB against a power having a
\TradeFLEET\faceplus in this \CTZ.

\phadm
\aparag\label{pV:Colbert:limits} Some French turn and period limits and basic
forces are raised during some periods.
\aparag As long as \ministreColbert is Minister, \FRA increases by half its
basic naval construction limit.
\aparag From now on, all new non-French \TradeFLEET levels placed in
\ctz{France} cost 10\ducats tax to be payed directly to French Treasury at the
moment such fleet levels are placed on the map.
\aparag\label{pV:Colbert:CB} \FRA receives a permanent additional bonus of
\bonus{+5} for all competition attempts it makes in \CTZ France. However there
is no malus for making competition attempts against \FRA.
% \begin{oldcompta}
%   \aparag From now on, \FRA may begin the construction of Versailles.
% \end{oldcompta}
\aparag \FRA may ignore restriction of~\ref{chAdministration:Pioneering} for
this turn (only).

\phpaix
\aparag The permanent tax implied in the event and the \CB can be later
annulled and \ministreColbert dismissed by scoring an unconditional victory
against \FRA and claiming their annulment in place of the taking of one
province. \FRA retains the other benefits (\numberref{pV:Colbert:limits},
\numberref{pV:Colbert:CB}).



\event{pV:Expulsion French Protestants}{V-6}{Expulsion of the French
  Protestants}{1}{PBNew}

\history{1685}

\condition{}
\aparag If \FRA is Protestant, roll for one (Catholic) \REVOLT in France and
consider the event as played (mark off, do not reroll).
\aparag If \FRA is \CATHCO, it can refuse the event and loses 3 \STAB and 10
\PV.
\aparag If \FRA is \CATHCR, it can refuse the event and loses 4 \STAB and 30
\PV.
\aparag If \FRA refuses the event, it can no more use \CB given by events
\shortref{pV:Glorious Revolution} and \shortref{pVI:Jacobite Rebellion}.

\phevnt
\aparag \FRA loses 1 level from both its current \FTI and \DTI.
\aparag The first protestant in the following list of precedence:
\HOL/\ENG/\SUE, gains one \TradeFLEET level of its choice taken from a
\TradeFLEET fleet in a \STZ where both countries are present (does not apply
if none available) and two free \COL attempt with strong investments.
\bparag The country receiving these actions may ignore restriction
of~\ref{chAdministration:Pioneering} for this turn.
\bparag If there is no Protestant power, \FRA loses one \TradeFLEET of his
choice.



\event{pV:Grand Siecle}{V-7}{``Le Grand Si\`ecle''}{1}{PBNew}

\history{1661-1702}

\phevnt
\aparag \FRA chooses, when all events of this turn have been drawn, to apply
one of the following events (that did not happen yet): \ref{pV:Devolution
  War}, \ref{pV:Chamber of Reunion} (or \ref{pV:League Augsburg} if it already
happened), \ref{pV:Colbertian Mercantilism} or \ref{pV:Expulsion French
  Protestants}.

\phadm
\aparag \FRA may ignore restriction of~\ref{chAdministration:Pioneering} for
this turn (only).



\event{pV:English Dynamism}{V-8}{English Dynamism}{1}{PBNew}

\history{}

\phevnt
\aparag \ENG chooses, when all events of this turn have been drawn, to apply
any one of the following events (if it did not happen yet): \ref{pIII:East
  Indian Company}, \ref{pIV:London Stock Exchange}, \ref{pIV:Act Navigation},
\ref{pVI:Treaty Methuen}.

\bparag The chosen event must be playable (no more than 1 period before or
after the current one).

\aparag In addition, \ENG has one free \OCB against \HOL, to be used before
the end of the period.

\phadm
\aparag \ENG may ignore restriction of~\ref{chAdministration:Pioneering} for
this turn.



\event{pV:Montecuccoli to Eugen}{V-9}{From \sectionleader{Montecuccoli}
  to \sectionleader{Prinz Eugen}}{1}{PBNew}

\history{1645-1700}

\phevnt
\aparag Depending on the current turn, check if the following general is still
in play ; if he is not, recall him immediately (even if he is dead: he was
only severely wounded and retired but the military situation require his
presence):
\bparag \leaderPappenheim between turns 28 and 32 (inclusive) ;
\bparag \leaderMontecuccoli between 33 and 38 (inclusive) ;
\bparag \leader{ER Starhemberg} between 39 and 41 (inclusive) ;
\bparag \leader{Prinz Eugen} between 42 and 49 (inclusive).
\aparag Armies of \AUSaus are now of class \CAIV.
\aparag \AUSMin now has a Land Technological marker that increases of two
levels each turn, beginning on the Latin level.
% (Jym) 05/2013: commented out in Pierre's notes from 2007...
% \bparag On turn 41, \HAB automatically take the \TMAN technology.



\event{pV:de Witt}{V-10}{\ministre{de Witt}}{1}{Risto}

\history{1653-1672}

\condition{\HOL can refuse this event if it wishes so. In that case mark off
  as played.}
\aparag \HOL can freely dismiss \strongministre{de Witt} (if Minister) at the
end of any following monarch survival phase and the event terminates.

\phevnt
\aparag \HOL receives a personality \ministre{de Witt} who may be used as
Monarch of a \terme{Parliament} government, or an excellent minister of a
\terme{Stadhouder} government, with values 9/7/9.  He will last for a random
length for Minister, see \ref{eco:Excellent Minister}.
\aparag During the last two turns of \ministre{de Witt}'s term in office (be
it Monarch or Minister), add \bonus{+1} to the monarch survival test.  If the
monarch dies during these two turns, \ministre{de Witt} is also removed and
this terminates the event before the new monarch is chosen.

\phadm
\aparag \HOL basic forces are increased by \FLEET\facemoins and \ARMY\faceplus
during every turn if is engaged in a war (Overseas, limited or full-fledged)
as long as \ministre{de Witt} is minister or monarch.



\event{pV:Peter the Great}{V-11}{Peter the Great}{1}{Risto}

\history{1689-1725}

\condition{}
\aparag If this is period \period{IV} and~\ref{pIV:Times of Troubles} is not
finished, do not mark off and reroll.
\aparag If \monarque{Peter the Great} was already received, nothing happens
  with this event (do not apply \RD instead).

\phevnt
\aparag The heir of the current monarch of \RUS is automatically
\monarque{Peter the Great} with values 9/9/9. See \ref{chSpecific:Russia:Peter
  the Great}.

\phadm
\aparag \RUS may ignore restriction of~\ref{chAdministration:Pioneering} for
this turn.



\event{pV:Saxon King Poland}{V-12}{Augustus II, a Saxon king in
  Poland}{1}{Risto}

\history{1697-1733}
\dure{Until there is a change of dynasty in \POL.}

\condition{}
\aparag If \POL is Orthodox or \CATHCR, the event is ignored. Mark off and
play \RD with the \REVOLT in \POL instead.
\aparag If \POL is at war against \paysSaxe, the event is ignored. Do not mark
off and re-roll.

\phevnt
\aparag The king of \POL is replaced, if it is a named general, he stay to
serve \POL as a general, otherwise, he is removed from the game. The new king
is \monarque{August II}, elector of Saxony.
\bparag He is scheduled to last for 7 turns.
\bparag His value are randomly chosen like after a \terme{Dynastic Crisis}.
\bparag \monarque{August II} may not be used as a general.
\bparag This is a change of dynasty in \POL.
\aparag \paysSaxe becomes a permanent \VASSAL of \POL as long as the event
lasts.
\bparag No diplomacy is allowed on \paysSaxe while the dynasty rules in \POL.
\bparag \paysSaxe is considered to be part of \POL for declaring wars of
signing peace (no separate peace is allowed, \ldots)
\aparag Any war against either \paysSaxe or \POL when the event occurs
immediately becomes a war against both (without formal declaration of war).

\phadm
\aparag \paysSaxe still get reinforcements as a minor country when at war. Its
troops can freely cross the \HRE and \POL. \POL can raise extra troops from
\paysSaxe (German mercenaries).
\aparag Troops of \POL do not get extra rights to enter countries of the \HRE
(however, \paysSaxe is always allied).
% (Jym) Not enter Saxony ? or \HRE ? After TYW ?  (Pierre) Polish forces are
% not allowed to enter Saxony or a country of the HRE that is not allied or at
% war with Poland. They are managed normally according to the status of Poland
% (minor or major). (JCD) I rewrote from the French version I leave former
% English version in comments
%
% \aparag Troops of \POL may not enter \paysSaxe or any other country of the
% \HRE which is not allied or at war against \POL.

\phpaix
\aparag Only an unconditional surrender can force either \POL or \paysSaxe to
a separate peace.
\bparag In this case, the losing country cannot enter the same war again but
the alliance between \POL and \paysSaxe is still in effect.

\effetlong
\aparag As long as the dynasty of \paysSaxe rules in \POL, the king can try to
impose Absolutism at the conditions of \ref{pIV:Polish Civil War}.
\bparag This can be done at the beginning of the second turn of reign of
\monarque{August II} and then whenever a new king (of the dynasty of
\paysSaxe) rules \POL.
\bparag This must be announced at the beginning of the event phase,
\numberref{pIV:Polish Civil War} is considered to be the first event rolled
for this turn.



\event{pV:Kingdom Prussia}{V-13}{Creation of the Kingdom of
  Prussia}{1}{RistoMod}

\history{1701}

\condition{If \ref{pIV:Great Elector} as not been played yet, mark off and
  play \numberref{pIV:Great Elector} instead.}
\begin{todo}
  Should be \PRUpru instead of \paysBrandebourg.
\end{todo}
\phevnt
\aparag If \POLpol still owns provinces of \region{Duche de Prusse}, they are
immediately annexed by \paysBrandebourg. \POL gets an immediate free \CB
against \paysBrandebourg.
\aparag \paysBerg is annexed by \paysBrandebourg.
\bparag Another country owning \paysBerg, either renounces it (and gives it to
\paysBrandebourg), or is declared war upon by \paysBrandebourg.

\effetlong
\aparag Basic forces of \paysBrandebourg are now 2 \ARMY\faceplus, one general
and 3 levels of fortification.
\bparag Its counters limit becomes 3 \ARMY and 5 \LD and its basic
reinforcement becomes 2 \LD.
\aparag Troops of \paysBrandebourg can freely cross the \HRE even if not at
war, in the same way the Emperor can.
\aparag The Elector of \paysBrandebourg wants to become king. This happens as
soon as one of the following condition is true:
\bparag The emperor grants the royal crown. \paysBrandebourg is put in \EC of
the Emperor (usually \HAB).
\bparag The country of the Emperor gives a unfavourable peace to
\paysBrandebourg. Instead of one peace conditions, \paysBrandebourg gets the
royal crown.
\bparag The Emperor signs an unfavourable peace of level 3 or more against
anyone. \paysBrandebourg takes the royal crown and the emperor has a free \CB
against it at the following turn.
% (Jym) free CB unique (for next \phdiplo) or permanent but usable only once?
\aparag Whatever the condition, the emperor loses {\bf 1} \STAB when
\paysBrandebourg becomes the kingdom or Prussia (the minor country is still
called \paysBrandebourg).
% (Jym) Maybe possibility for Emperor to take back the crown, as with an
% unconditional or 3+ peace?



\event{pV:War Sweden Denmark}{V-14}{War between \paysmajeurSuede and
  \paysDanemark}{1}{PB}

\history{1675-1679}

\phevnt
\aparag \DANMin and \PRUmin, if inactive, declare war to \SUE.
\aparag \PRU as a major country has a \CB against \SUE. If it doesn't use this
\CB, it loses 1 \STAB and the control of \paysDanemark. If it uses this \CB,
it gains \paysDanemark in \EG.
\aparag \DAN as a major country has a \CB against \SUE. If it doesn't use this
\CB, it loses 1 \STAB and the control of \PRUmin. If it uses this \CB, it
gains \PRUmin in \EG.
\aparag Normal call for allies occur. Especially, a major country with
diplomatic control (\AM or better) of either \DANmin or \PRUmin is called by
the minor.
\aparag \SUE does lose diplomatic control of both \paysDanemark and
\paysBrandebourg.



\event{pV:Koprulu}{V-15}{\ministreKoprulu}{1}{RistoMod}

\history{1656-1683}

\condition{\TUR can refuse this event if it wishes so. In that case mark off
  as played.}
\aparag If \TUR has performed any reform of level 2, mark off and play \RD
instead, with the \REVOLT in \TUR.
\aparag \TUR can freely dismiss \strongministre{Koprulu} at the end of any
following monarch survival phase and the event terminates.

\phevnt
\aparag \TUR receives an Excellent Minister \ministreKoprulu with values
8/9/7. He will last for 8 turns.
% 2 turns plus a random length for Minister, see \ref{eco:Excellent Minister}.
The Minister is not dismissed if the \TUR monarch dies ; \TUR rolls for the
values of the new monarch using the values of the Monarch only with no malus
nor bonus.
\aparag \TUR receives an additional level of \MNU of \RES{Metal}.
\aparag Four corrupted pashas may be removed immediately with no penalty.
\aparag \leader{Sadri Azam} is replaced by \leaderKoprulu while the event is
in effect. If this general is killed, captured or defeated in a Major Victory,
\TUR loses two additional \STAB or may choose to end immediately the event. If
the event is not ended, the general comes back in play (another one in the
same dynasty) on the following turn.

\phadm
\aparag Turkish Reforms cannot be attempted while the event is in effect.



\event{pV:Fights Iroquois}{V-16}{Fights against the Iroquois}{1}{Risto}

\phevnt
\aparag Roll 1d10:
\bparag If the result is even, \paysIroquois declares an Overseas war to one
power that has a \COL/\TP adjacent to them (this \COL/\TP is chosen randomly
to decide which power is the target). It will first try to invade this
settlement, and will go against the other ones of the same country only if
this one is captured/destroyed.
\bparag If the result is odd, the natives of a randomly chosen \COL of a major
power (including annexed Portugal) in an unsubdued area in \continent{North
  America} are activated and will attack this \COL at the end of the turn.



\event{pV:Slave Revolts WI}{V-17}{Slave Revolts in the West Indies}{2}{Risto}

\phevnt
\aparag Roll 1d10 for each power having \COL in areas \granderegionCuba,
\granderegionHaiti and/or \granderegionAntilles. On a result of 7 or more, a
\REVOLT \facemoins is placed in one randomly chosen \COL of the power.



\event{pV:Wars India}{V-18}{Wars in India}{2}{PBNew}

\history{Aurangzeb (1658-1707) / Revolts of the Marathi}

\phevnt
\aparag If the non-European minor country \paysMogol does not exist, it is
created now. Its ruler is now \leader{Grand Moghol} (replacing \leaderAkbar if
he was in play).
\aparag If it was still existing, minor country \paysVijayanagar is destroyed
(by internal fights).
\aparag \granderegionBengale and \granderegionKarnatika becomes rich region,
with 2 resources of each kind shown on the map (instead of 1).
\aparag If the \paysMogol exist, they invade one province with a modifier of
\bonus{-2}, the next in the list according to the event \ref{pII:Mughal
  Expansions}.
\aparag From now on, \paysMysore and \paysHyderabad are created as soon as no
other country owns their region.
\aparag Every \TP/\COL in \continentIndia that is in a region owned by a minor
country will face an attack by the natives of the area (disregarding the
existence or not of a Treaty). Attacks caused by this event will be resolved
at the end of turn with a modifier of \bonus{+4}.



\event{pV:Treaty Nerchinsk}{V-19}{The Treaty of Nerchinsk}{1}{RistoMod}

\history{1689}

\phevnt
\aparag \paysChina annexes all provinces in \granderegionAmour, and all
provinces adjacent to Mongolia (the white zone) in \granderegionBaikal. Its
Activation level is 6 in these provinces.
\aparag \RUS and any power having \COL/\TP in any of these provinces may now
make diplomacy on \paysChina in order to obtain \dipAT with it. This Treaty
allows the power to have at most 2 \COL/\TP that will draw no reaction from
\paysChina.
\aparag It is not possible for one power to have a \dipAT status for this
effect, and another one for a \TP in \paysChina. It is one or the other.

\phadm
\aparag \RUS may ignore restriction of~\ref{chAdministration:Pioneering} for
this turn.



\event{pV:Invasion Formosa}{V-20}{Invasion of Formosa by China}{1}{RistoMod}

\history{1683}

\begin{todo}
  Add test depending on situation and possibility of failure?
\end{todo}

\phevnt
\aparag \paysChina invades \granderegionFormose.  This province is now owned
by \paysChina and subjected to all the relevant rules. Activation level for
this province is 6.
\aparag Any foreign \TP/\COL in the region will be attacked by the Natives of
the province this turn.
% \aparag If \ref{pIII:CCA:Closure China} is in effect, a power still having a
% \TP in the province may sign a \dipAT with \paysChina. If it succeeds (before
% its \TP is destroyed), it may keep the \TP as if it were there at the moment
% of the closure.
\aparag If a \TP has survived, \pays{Chine} concedes a new \dipAT to the owner
of the \TP, if it didn't have any. The owner still has to pay as for usual
\dipAT with \paysChine.



\event{pV:Japan Trade}{V-21}{Trade Regulations in Japan}{1}{PB}

\history{1638 and afterwards}

\phevnt{}
\aparag If \ref{pIV:JCA:Closure Japan} happened, reduce any \TP in Japan by 2
levels.
\aparag If \ref{pIV:JCA:Japan Commercial Dynamism} happened or none of
\ref{pIV:Japan Colonial Attitude}, apply now \ref{pIV:JCA:Closure Japan}.



\event{pV:Revolt Cossacks}{V-22}{Revolt of the Cossacks}{1}{PB}

\history{1654-1667}

\condition{This event is the same as \ref{pIV:Revolt Cossacks} which happens
  now if it did not occur yet. Else, treat as \RD, with \REVOLT in \POL.}



\event{pV:Revolt Catalunya}{V-23}{Revolt in \provinceCatalunya}{1}{PBNew}

\history{1640-1652 / 1705-1707}

\phevnt
\aparag Place a \REVOLT \facemoins in \provinceCatalunya ; the \REVOLT
controls also the fortress. Any military force in the province must retreat.
\bparag If this event happens during \ref{pV:WoSS}, the \REVOLT is \faceplus
instead.
\aparag If \SPA is at war against \FRA, \ENG or \AUT, the \REVOLT is friendly
to the first of those countries that is an enemy of \SPA.

% Placeholder

\event{pV:Hungary}{V-s}{Revolt in \paysHongrie}{1}{PBNotEvenWritten}
\begin{todo}
  Probably a duplicate of \ref{pV:Kuruc}.

  Remove the army class change from \ref{pV:Montecuccoli to Eugen}.
\end{todo}

\phevnt
\aparag 4 (or 5 ?) random \HAB provinces in former territory of \paysHongrie
revolt: roll for strength at random.
\bparag The rebels are controlled by \TUR and friendly to \TUR.
\aparag \TUR has a free \CB against \HAB this turn.

\phadm
\aparag Armies of \AUSaus are now of class \CAIV.

\event{pV:Transylvania}{V-t}{Christian prince in
  \paysTransylvanie}{1}{PBNotEvenWritten}
\history{1648 (George II R\'{a}k\'{o}czi + Turkish Invasion)? / 1687
  (Transylvania recognise sovereignty of \HAB)? / 1699 (Treaty of Karlowitz)?}

\begin{todo}
  Maybe in early pVI.

  Maybe handled differently (Transylvania goes to owner of Buda).

  Maybe part of \ref{pV:Kuruc}.
\end{todo}

\phdipl If \paysTransylvanie is on the Diplomatic track of \TUR, it becomes
\Neutral.

\event{pV:Cretan war}{V-u (1)}{Cretan war}{1}{PBNotEvenWritten}
\history{1645-1669}

\begin{todo}
  hist. : 3 expeditions to the Dardanelles, \TUR annexes Creta, \VEN make
  small gains in Dalmatia.
\end{todo}

\event{pV:Morean war}{V-u (2)}{Morean war}{1}{PBNotEvenWritten}
\history{1684-1699}

\begin{todo}
  Morosini + conquest of Morea.

  Could be in early pVI also.
\end{todo}

\phpaix If \paysVenise sign a white or favourable peace, it annex and
additional province in \regionBalkans or Mediterranean island.

\event{pV:Revolt Pueblos}{V-v}{Revolt of the Pueblos}{1}{PBNotEvenWritten}
\history{1680}


\event{pV:Tangiers}{V-w}{Reconquest of Tangiers}{1}{PBNotEvenWritten}
\history{16??}

\begin{todo}
  Probably to remove. Should be handled by the diplo event where minor retake
  a presidio. Can be added to~\ref{pVI:Barbaresques} if needed.
\end{todo}

\event{pV:Khoikhoi}{V-x}{Khoikhoi-Dutch wars}{*}{RistoMoved}
\history{1659/1673/1674-1677}
\begin{todo}
  May replace \ref{pV:Slave revolt WI} in the table since this one has
  been moved in the \REVOLT table.
\end{todo}

\phevnt
\aparag Natives in \granderegion{Cap} W. province are activated with 2\LD and
a leader, whatever the printed value.

\event{pV:Bill Test}{V-y}{Bill of Test}{1}{RistoMoved}
\history{1673}

Same as~\ref{pVI:Bill Test}. Should be moved in pV.

\event{pV:Kuruc}{V-z}{The Great Kuruc Uprising}{1}{PBNotEvenWritten}

\history{1678-1684}[Part of the Great Turkish War (series of wars fought from
1662 to 1699) that lead to the famous (second) siege and battle of Vienna of
1684]
\dure{Until the end of the war caused by the event.}

\phevnt
\aparag[] [BLP] If \paysHongrie still exists, it is immediately destroyed and
all its provinces are annexed by \AUS. No \VPs are gained for these
annexations.

\aparag[] [BLP] \ref{chSpecific:Little war} is no more active.

\stopevents

% Local Variables:
% fill-column: 78
% coding: utf-8-unix
% mode-require-final-newline: t
% mode: flyspell
% ispell-local-dictionary: "british"
% End:

% LocalWords: Heirdom Minister malus reroll PBNew pIV pVI pV WoSS Colbertian
% LocalWords: Siecle Montecuccoli de Koprulu Nerchinsk Catalunya Kuruc Risto
% LocalWords: Franche PBMod TODO Habsburg Asiento Olivares RistoMod ecle Azam
% LocalWords: Starhemberg Stadhouder Duche Prusse Sadri Moghol LocalWords:
% PBNotEvenWritten
%  LocalWords:  Morean
