% -*- mode: LaTeX; -*-

\chapter*{Introduction}

\section*{Don't panic!}
If you just discover this game, you're probably panicking right now. Don't
panic, you'll manage\ldots

\emph{Europa Universalis} is without a doubt a monster game. The core rules
are several hundreds pages long. Each player has a 9 pages player's aid. The
game components include two huge maps and almost 3500 counters. Lastly, a turn
usually require 2 to 6 hours to play. Hence a \terme{great campaign}, the way
it is meant to be played, represents between 250 and 350 hours of
playing. That is, if you play one week-end per month, your game will last for
a couple of years (the game has a rather low density of counters and writing
down positions between play sessions in not too hard).

However, \emph{Europa Universalis} is not that hard to play\ldots Great
efforts have been made to streamline the rules (apart from the specific
rules). The motto being ``the game is complex, not complicated''. Most actions
in the game are quite similar from one turn to another. Quickly, you will learn
to do them without hesitating. Quickly, you'll will be able to concentrate on
the time-consuming but interesting aspects of the game: Diplomacy and
Strategy.

\emph{Europa Universalis} is about Diplomacy. This is a multiplayers game. You
cannot win alone. You will need to discuss a lot with other players. You will
see that the Diplomacy phase is very important and has a lot of
depth. Countries are not balanced in term of strength. If you play a weak
country, you'll need to find alliances to get money, troops, peace, \ldots If
you play a strong country, you'll need to divide your opponents or they will
still be able to crush you. We do not advice to put a time limit on the
Diplomacy phase. It is not uncommon for a single Diplomacy phase to last for 1
hour, sometimes 2. These are usually intense hours worth playing.

\emph{Europa Universalis} is about Strategy. The Military phase is the other
important phase. It can lasts for 2 or 3 hours during big wars. You will see
that the Military rules are quite detailed, maybe the most complicated rules
around. That makes a good strategy really worthwhile. You will learn the
geographic strengths and weakness of your country. You will soon cherish those
few \bonus{+1} die roll modifiers you can grab. You will know the thrill of a
good strategy winning you the war. You will curse the stroke of bad luck that
can turn an easy battle into a disaster.

\emph{Europa Universalis} is a very deep game. You will feel the real position
of a Monarch taking decisions that greatly affect your country. \emph{Europa
  Universalis} is a game with a great emotional implication of the
players. You will probably want to recall some of your best (or worse) moves
even years after, or to tell them to other EU players.

\begin{playtip}
  Due to the highly interactive Military phase (with many interceptions
  possible), \emph{Europa Universalis} is not well suited to play-by-mail and
  we advice a face-to-face game.

  When playing a game with beginners, the best is to have at least two
  experienced players. One can play the monster (Spain), hard to begin with,
  while the other can play a less important country (such a Portugal or
  Poland) and act as a rule layer and arbiter. If you have only one
  experienced player, it is probably better to have him play a small country
  and be a rule layer (with time to answer questions). Give Spain to a player
  who is not afraid of monster games\ldots

  Before playing with beginners, we advice you to play an initiation
  session. After a quick overview of the rules, you can ``jump'' into a new
  game. Everybody will make a lot of errors (both ``cheating'' (rule errors)
  and strategic or tactical errors). But after 3 to 6 turns (1 or 2 days,
  usually), everybody should know the rules well enough. Then, you can start
  the real game. ``Loosing'' 2 days of play may seem long, but compared to the
  duration of the game, this is actually OK.

  Before playing for real, make sure that everybody knows the most common
  rules. Each player should also read the specific rules of his country. The
  rest can be interesting but is not necessarily. Decide also on an arbiter
  (usually the most experienced player) as rules conflicts will probably
  arise.
\end{playtip}


\section*{Organisation of the game}
The game is composed of: the rule book, itself split in six parts (rules,
events, appendix, tables, scenarios, indexes), the counters (more than 3000),
the maps (two A0 maps), the record sheets (most individual record sheets,
except two global record sheets), ten-sided dices and pens (not included).

The first rulebook part contains the game description and mechanics. After a
short overview of the game in chapters I and II, the following chapters follow
roughly the turn order. The last chapter is dedicated to specific rules.
% This is a mandatory reading for each player.

The second rulebook part is the set of historical events (or almost
historical) that make the game tick. It is divided in economic events and
political events, ordered by period of apparition (period I starting in 1492
and period VII ending with the French Revolution).
%These can be read when they occur, however reading the big events helps
%planning.

The third rulebook part contains various listings such as minor countries
characteristics. Most of this information is already available on
counters.
% This should be read only when needed (except for basic forces and
% reinforcements information).

The fourth rulebook part holds the game tables and players aids. There are
eight pages of general tables plus one page per country played (thirteen
different countries are available during the course of the game).
% A copy of these eight pages should be given to each player along with the
% page of the country that will be played.

The fifth rulebook contains the scenarios as well as some advices on playing
the game. The game is meant to be played as a \emph{Great campaign} spanning
over 300 years of history, from Columbus journey to America to the French
Revolution. Other scenarios, for a shorter game as well as ways to learn the
rules progressively, might be written someday.

The sixth rulebook part contains the table of contents, the index and various
lists. It is not necessary for the game (but may be useful while browsing the
rulebook).

Apart from this introduction, each point of this rulebook is fully numbered
(such as ``paragraph C.3 of section III.3.3.3'').

The counters are not pre-cutted. Thus, you will need to print them (23 pages,
in full colours), glue them and cut them. Beware that some counters are
double-sided while some are simple-sided. Beware that there are two size of
counters (plus the triangle shaped manufactures). Gluing and cutting is a
tiresome process. We advice to do it all before playing (rather than waiting
for the counter to be needed). You will probably need counters trays (6 is
good) to hold them.

The maps are intended to be printed on a A0 sheet (each). You can try printing
them smaller (A1) but they will then be very crowded with the counters. That
means that you will need one (or two) large table to hold the maps, enough
room for nine players around, and some private space for secret
diplomacy\ldots

\section*{Organisation of the rules}
%As stated, the whole rulebook is organised into several booklets.
\subsection*{Rules}

Following this informal Introduction, the rules are organised in chapters,
Sections, and numbered paragraphs. Each point being fully numbered for easy
reference.

The first two chapters, \ref{chapter:Basics} and \ref{chapter:ThePowers}, are
introductory chapters. They describe the components and the main concepts of
the game. The following chapters, from \ref{chapter:Events} to
\ref{chapter:Inter} describe the core rules. Each chapter corresponds to one
phase of the game turn (or part of one), in order. These should be read by
each player before playing.

Then, \ref{chapter:Victories} deals about fame, glory and all that. Victory
Points (\VPs) are earned slowly during the game, a bit at each turn and
slightly more at the end of each period. Each player will need to read the
corresponding part of this chapter in time, but reading all of it is not
necessarily.

The rules close with \ref{chapter:Specific}. This chapter explain the specific
rules of each country. Both the majors and minors countries do have specific
rules (``ways to cheat''). These specific rules are the salt of the game, the
thing that makes each country different from the other. Each player must at
the very least read the specific rules of his country. Reading the specific
rules of your opponents is not required but can greatly help build a strategy
against them. Reading the specific rules of all minors countries is usually
not needed but you should probably read the rules for countries with which you
will interact (that is, countries in your geographical area).

\subsection*{Events}
\subsubsection*{Political events}
If the specific rules of each country are the salt of the game, the political
events are the real meat. Each turn, four historical events are rolled for in
a more or less organised way. These events create special conditions to
apply. Often, they give opportunities (or obligations!) to create new
wars. Sometimes, they provide with drastic changes of alliances or new
political situations.

Most of the events happened historically. Some of them did not happened but
seemed plausible enough to be added to the game. Political events are grouped
by periods so that, say, the American Revolution may not occur before England
has a chance to colonise North America.

Even if grouped by historical periods, the events occurs in a randomly
fashion. This provides a unique, yet hopefully plausible, historical
background for each game. Moreover, the way the players react to the events
can be quite different from the historical reactions of the monarchs of this
time. Thus, the history in game can be quite different from the History as it
did happen.

In order to win the game, each player will need to react properly to the
events. Trying to be in a good position to exploit opportunities and to avoid
major drawbacks requires a careful playing. Between the (political) choices
proposed for each event and the actual military conduct of the wars, players
will have a lot to do.

Reading the events is not necessarily before playing. In some way, it is even
better if nobody knows them as each player will then enjoy the surprise of
things as the historical monarchs did. However, knowing the events (especially
the big ones) helps planning. Moreover, after playing once, you will know the
big events and this will give you an edge over beginners. Each group should
choose which policy to apply toward events (read them in advance or not) and
stick to it (and, especially, use fair play if you decide to keep the events
secret\ldots)

\subsubsection*{Revolts, diplomacy and economic events}
In addition to the political events, there are also some economical
events. These occur once per turn per country, always among the same set of
events. They give an additional random flavour to the game.

Sometimes, political events will simply result in a peasant's revolt as well
as some diplomatic instability. This is resolved using the revolt and
diplomacy events.
% Notice that the brutality of the diplomatic turmoil is actually an
% approximation for the sake of simulation: a minor country suddenly breaking
% its alliance after 30 years of loyal services might in fact represent a
% gradual worsening of the relations.

\subsection*{Tables, Appendices, Scenarios, Index}
\subsubsection*{Appendices, Tables and Index}
The appendices mostly contain the complete description of minors
countries. You don't need to read them before you actually need them. You may
want to browse through it in order to know which are the strong minors. It
contains mostly a lot of quite arid information that makes little sense before
you actually start playing.

The Player's aids contain the summary of all the tables required to play. They
are also in the rules, so there is nothing new in the aids worth reading
before playing. Each player will need a full set of player's aid during the
game, so make sure to print enough.

The index is used it for reference purpose only.

\subsubsection*{Scenarios}
The scenarios booklet should contains several scenarios for larger and larger
games. It currently only contains the largest one: the Great Campaign.

\emph{Europa Universalis} is designed to be played as a Great Campaign. You'll
find there the setup for this scenario.

Someday, we might add additional setups for shorter games, both shorter
campaigns and ``battle'' or ``war'' scenarios. But frankly, this is not our
most important task today.

\section*{A couple of meta-tips about the rules}
These rules use many visual tools to help the reader. Coloured boxes are
sometime used to highlight some features of the game such as the following
ones:
\begin{designnote}
  Sometimes, we feel the need to explain stuff about the meaning of the rules.
\end{designnote}
\begin{histoire}[History of the game]
  In the 1990's, Philippe Thibault wrote the original \emph{Europa
    Universalis} game, that we refer as ``EU6'' as is was designed for 6
  players only. In the early 2000's, Pierre Borgnat and Bertrand Asseray wrote
  an addendum to these rules, adding two players and modifying many aspects of
  the game. Quickly, Jean-Yves Moyen and Jean-Christophe Dubacq joined the
  project. We decided to rewrite the whole rules from scratch and to add a
  ninth player. The result is thus called ``EU9''.
\end{histoire}

Most terms in these rules that do refer to something precise, such as a rule,
an event, a leader, \ldots are usually hyperlinks. That means that if you're
reading the rules on an electronic device (computer, tablet, \ldots) you can
click on almost anything and jump to the page in the rules where it is
described.
\begin{exemple}
  Try clicking on the following and see where it leads (your device probably
  has a ``back'' feature to come back here afterwards):\\
  \ANG, \ref{chapter:Incomes}, \ref{chAdministration:bankruptcy},
  \continentBrazil, \ref{pI:Tordesillas}, \ref{pIV:TYW:Creation of the
    Germanic Alliances}, \ref{pVII:Revolution:TerrorGov}, \monarqueSuleyman,
  \ministreRichelieu, \leader{Prinz Eugen}.
\end{exemple}

Lastly, the choice of language, fonts and colours in which terms are written
also carries information on what kind of entity it actually depicts. If two
terms are written in the same way, that usually means they depict similar
entities (leaders, country, troops, cities, \ldots)
\begin{exemple}
  Consider the differences between:\\
  \paysVnorvege (minor country, in Latin), \regionNorvege (a region on the
  map, in English), \provinceNorvege (a single province, in local language)
  and \seazoneNorvege (a sea zone, in French).
\end{exemple}

\section*{Getting help}
If you have questions about the game, or if you need help on the rules, please
feel free to contact us.

You can ask us on the EU mailing list at Yahoo groups, either in English
(\url{EU-list@yahoogroups.com}) or in French
(\url{EuropaUniversalis@yahoogroupes.fr}). Since both of these mailing lists
were created for the original Azure Wish edition of the game (EU6) and not for
this BAMGames rewrite (EU9), please make sure that you state clearly that your
question is about the BAMGames rewrite. Otherwise, you might annoy people and
you'll probably get answers based on the EU6 version of the rules which, in
some points, is quite different from this one.

Or you can ask at the forum \url{http://europa-universalis.frbb.net/forum.htm}
(in French, but we'll answer in English). This forum is specifically about the
BAMGames version of the game. There is also a dedicated thread on the
BoardGameGeek forums:
\url{https://www.boardgamegeek.com/thread/1278052/eu9-9-players-rewrite}

Some of us are also present on other gaming forums such as ConsimWorld
(\url{http://talk.consimworld.com/}, in English) or Strategikon
(\url{http://www.strategikon.info/phpBB3/}, in French). This is however not
the most reliable way to reach us.

% Local Variables:
% fill-column: 78
% coding: utf-8-unix
% mode-require-final-newline: t
% mode: flyspell
% ispell-local-dictionary: "british"
% End:
