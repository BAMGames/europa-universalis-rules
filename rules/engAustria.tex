\sectionJ{\anchorpaysmajeur{Autriche} and Habsburg Empire}{\blason{habsbourg}}\label{chSpecific:Austria}

\subsection{The Habsburg Empire}
\begin{todo}
  check denominations in all the rules !!!!  \MAJHAB = \AUS sinon \SPA;
  \AUS = le majeur ; \hab ou \pays{Habsbourg} = le mineur ; \HAB =
  regroupe les deux (majeur ou mineur)...
\end{todo}

\aparag There are several designations for the Hasburg Empire in these
rules: \HAB is the generic denomination. \pays{Habsbourg} is the minor
country, often written \hab if an independent minor country (after
dissociation from Spain). \paysmajeur{Autriche} and \AUS are used for
\HAB if the rule only applies to an independent major country. When
player actions are done for \MAJHAB, \MAJHAB indicates \AUS if possible and
\SPA otherwise.
% \bparag Even if not \MAJ, \HAB has its own diplomatic track.
\aparag \HAB is Emperor of the \HRE, unless specific events say the
contrary (especially \eventref{pI:Emperor Election} and
\eventref{pII:Emperor Election}.

\subsection{Austria as a minor country}

\subsubsection{The Habsburg Dynastic Alliance}\label{chSpecific:Habsburg Dynastic Alliance}

\aparag Until \eventref{pV:WoSS} takes place, \SPA and \pays{Habsbourg}
share a special relationship. As such, no country may till then make
diplomacy on \pays{Habsbourg}.
\bparag Before \eventref{pI:Habsburg Alliance}, \HAB is considered to be
in \EW of \SPA. It reacts like a normal minor country.
\bparag Between those two events, \pays{Habsbourg} is a special ally of
\SPA: the Spanish Habsburg Dynastic Alliance.
\bparag At the begining of \eventref{pV:WoSS}, the Habsburg Dissociation
happens.  There isn't anymore a Dynastic lliance and\pays{Habsbourg} (if
it happens to be played as a minor power) is now a regular power .

\aparag[Effects of the Habsburg Dynastic Alliance]
The general principle is that \SPA and \pays{Habsbourg} are
involved in a permanent Defensive and Offensive Alliance. Moreover, it
gives a free \CB to \SPA when \pays{Habsbourg} is calling for its
Defensive ally (but only in this case).
\bparag When a declaration of war is made against \pays{Habsbourg}, \SPA may
announce that this is also a declaration of war against \SPA (at no
extra cost), and thus be fully involved in the war. This is not
mandatory, in which case \SPA may not be fully involved in the war.
\bparag The reverse is not true in principle (some events may overrule
this), and \SPA must pay in \STAB to have \pays{Habsbourg} enter a war
declared against or by \SPA.
\bparag \pays{Habsbourg} has the right to declare war, at no cost in \STAB to
\SPA, on any country declaring war on a minor member of the \HRE.
This may be changed after \eventref{pIV:TYW}.
\bparag \pays{Habsbourg} may make limited or foreign interventions in
other wars, decided by \SPA.
\bparag When \SPA asks for full involvement of \pays{Habsbourg},
however, the roll automatically succeeds (as if \pays{Habsbourg} was
\VASSAL of \SPA).

\aparag[Separate peace] \pays{Habsbourg} may however accept to negotiate
separate peace, as any other minor, unless events say so.
%\bparag The above rules do not apply to \AUS.

\aparag[Access to \HAB] When \SPA is the Emperor:
\bparag It gains free access to all \pays{Habsbourg} territories even in
peace.
\bparag Its monarch may leads troop of \AUS.

\subsubsection{Conduct of the \hab country}
\aparag As a minor country, \pays{Habsbourg} always uses the \CB offered by
political events. When declaring war, it always calls upon any country
that would have developed diplomatic relations at least in \EW, like if
it was a \MAJ offensive alliance. If attacked, it calls upon any country
like if it was a defensive alliance between \MAJ. The penalty for not
honouring the alliance is the loss of the diplomatic position.
%\bparag Remark that the army class of \pays{habsbourg} becomes \CAIV at
%the dissociation if not before.

\aparag \pays{Habsbourg} always takes back the control of the autonomous
Habsburg states, except for \pays{HNaples}.

\aparag A few events may place countries on the diplomatic track of \HAB
even when it is only a \MIN. They are managed (until the dissociation)
--- giving their incomes or entering in war --- as if they were on \SPA
diplomatic track.
\bparag Before the dissociation, if \SPA does not defend their position
on the track, they are automatically defended by \pays{Habsbourg} (with
the \DIP of \SPA and a small investment).
\bparag After the dissociation, they are defended by \HAB as if \HAB was
an abandoned \MAJ (see \ruleref{chSpecific:Campaign:Minor Diplomatic
  Track}).

\aparag After Dissociation due \eventref{pV:WoSS}, the \terme{basic
  forces} of \pays{Habsbourg} are increased by \ARMY\faceplus and 2
levels of fortification.

\subsection{Specific affairs of \HAB}

This section applies to both major and minor \HAB.

\subsubsection{Crusades against the Ottomans}

\aparag[Catholicism and Crusades]
The religion of \pays{Habsbourg} is always Catholic, and
Catholic/Counter-Reformation as soon as possible (from
\eventref{pI:Reformation2}).
\bparag \HAB has a free \CB against \TUR if a \terme{Crusade} is called
for. If Emperor, \pays{Habsbourg} will always use it. Else, a test must be made
(\ruleref{chSpecific:Crusades}).

\aparag[Reconquest of Hungary]
After \eventrefshort{pI:Habsburg Hungary} or \eventrefshort{pI:Fall
  Hungary}, apply the following.
\bparag \HAB has a permanent \CB against \TUR as long as \TUR own a
province initially in \pays{Hongrie}.
\bparag If both \pays{Habsbourg} (under this provision) and \SPA
declares a war against \TUR at the same turn, it costs the penalty (in
\STAB and \PV) according only to the \CB that \SPA has.
\bparag The \terme{basic forces} of \HAB are increased by
\ARMY\facemoins if \pays{Hongrie} was split due to \eventref{pI:Fall
  Hungary} and \ARMY\faceplus if due to \eventref{pI:Habsburg Hungary},
or \ARMY\faceplus if \HAB owns at least 10 provinces of \pays{Hongrie}
(the best applies).
\bparag \HAB uses the leaders of \pays{Hongrie} as if its own.  \HAB may
use the counters of \pays{Hongrie} if it owns at least 7 provinces of
\pays{Hongrie}, and only one \ARMY and half of the \LD if it owns
between 4 and 6 provinces.

\aparag[Technology]
 \pays{Habsbourg} begins the game as \CAIII, Latin technology,
until its army class is changed from \CAIII\ to \CAIV\ in period V or by either
\eventref{pV:Montecuccoli to Eugen} or the Habsbourg Dissociation.
\bparag \pays{Habsbourg} troops never use \TTER technology.

\subsection{Austria as a major country}

\subsubsection{Diplomacy of Austria}

\aparag See the \ruleref{chSpecific:Campaign:Transfer Holland} (or
\ruleref{chSpecific:Campaign:Transfer Sweden}) for the conditions of the
transfer of a player to \AUS.

\aparag[The Habsburg Dynastic Alliance]
\label{chSpecific:Habsburg Dynastic Alliance:Major}
\bparag At the beginning,  \SPA and \AUS are always linked by a
mandatory alliance, even if they fail to answer it or even at war against
one another (so that they still may answer the alliance aganist other powers).
They can do full or limited intervention, both in offensive or defensive stance.
\bparag During that time, \SPA does not lose \STAB\ to use the defensive
alliance to help \AUS.
\bparag However, they are not mandatorily allied if they are not using
\CB given by this Alliance, except if they announce it (and could so
make separate peace at no cost, and so on).
\bparag They may be at war against one another, but only if using a
legitimate \CB to do so.
\bparag They are no limit to money transfer between them.
\bparag At the end of \eventref{pIV:TYW}, if both \SPA and \AUS has
achieved Neutral or Losing positions, the mandatory alliance becomes
defensive only and is weakened in the sense that a limited intervention
is sufficient to fulfill it. The mandatory alliance is not offensive
anymore.
\bparag At the begining of  \eventref{pV:WoSS}, there isn't anymore a Dynastic
Alliance. Note however that, depending on the choice of the Heir, there might
be different kinds of Dynastic Ties as decribed in this event.

\aparag[\HRE] If \AUS is Emperor of the \HRE,
it gains a free \CB in reaction on any country declaring war on a minor
member of the \HRE.
This may be changed after \eventref{pIV:TYW}.

\subsubsection{Baltic Fleet}\label{chSpecific:Austria:Baltic Fleet}
\aparag The ownership of at least one province that did belong to the
\pays{Hanse} minor country increases the construction limits and
\terme{basic forces} of \AUS.
\bparag The \AUS\ \FLEET counter can then also be used (anywhere).
\bparag In this case, \AUS may use \TradeFLEET, but only in periods VI
and VII
\aparag \AUS has no \CTZ

\subsubsection{Autonomous Habsburg States}
\aparag See \ruleref{chSpecific:Spain:Autonomous States} for the rules about
the autonomy that can be given to cadet branches.
\aparag At the time of the dissociation, \HAB may decide to remove the
autonomy given to the autonomous kingdoms of \pays{HHongrie} and
\pays{HBoheme}. The same applies to \pays{HMilan} if \HAB gets the
province of \province{Lombardia} in the resolution of
\eventref{pV:WoSS}.
\aparag[Sicilia] \pays{HNaples} must be given its autonomy if owned by \HAB, and
\province{Sicilia}, \province{Palermo}, \province{Saldigna} are in this
case part of it.
\aparag[Hungary] Increase of Basic force and usage of Hungarian counters
are cancelled if \pays{HHongrie} is granted autonomy.

\subsubsection{Inheritance of the Hasburg Empire}
\aparag If \eventref{pI:Habsburg Alliance} had been contracted, \HAB
always consider all provinces of \pays{provincesne},
\eventref{pI:Burgundy Inheritance}, \eventref{pI:Spanish Naples} and
\eventref{pI:Habsburg Milano} as former provinces, for the sake of
\ref{chPeace:Transfer Provinces Peace}.
\aparag \HAB  always consider all provinces of  \eventref{pI:Habsburg Hungary}
and \eventref{pI:Habsburg Bohemia} as  former
provinces, for the sake of the same rule.
\aparag \HAB may, under the same conditions, annexe the capital province of
those minor powers, even if it has been recreated.

\subsubsection{Grouped annexions in Italy}
\aparag \HAB may consider \province{Palermo} and \province{Sicilia}
as one province when signing a winning peace, so as to take them as one
Peace condition.
\aparag \HAB may consider 2 provinces among \province{Campania},
\province{Basilicata}, \province{Abruzzo}, \province{Puglia},
\province{Calabria} as one province when signing a winning peace, so as
to take them as one Peace condition.
\aparag \HAB may consider all the provinces \province{Campania},
\province{Basilicata}, \province{Abruzzo}, \province{Puglia},
\province{Calabria} as being two provinces when signing a winning peace,
so as to take them as two Peace conditions.



\subsection{\sectionpaysmajeur{Autriche} in play}
\subsubsection{The Austrian monarchs}
\aparag[\anchormonarque{Ferdinand II}] is the monarch at the beginning
of pIV or of \eventref{pIV:TYW} if it happens in pIII in the
nine-players version. His values and length are obtained at random
(Dynastic Crisis are not possible).
\aparag[\anchormonarque{Ferdinand III}] is the monarch at the time of
the dissociation of the Habsburgs (caused by \eventref{pV:WoSS}), for
the eight-players version.  He has values 6/8/7, whose reign length
should be rolled for (further Dynastic Crisis are not possible !).
\aparag[\anchormonarque{Maria Theresia}] becomes the Archduchess of
Austria at the beginning of \eventref{pVI:WoAS}. She has values 8/8/7
and lasts 8 turns. She does not roll for survival for the first 5 turns
of her reign. She cannot be used as general. \AUS gains \ARMY\faceplus
of \terme{basic forces} during her reign.
\aparag[\anchorministre{Kaunitz}] may be named minister through
\eventref{pVII:Kaunitz}. He has values 9/8/7 and remains a random
number of turns; its values can be used for the next monarch's values
determination if a succession takes place while he is still alive.

\subsubsection{Available counters}
\aparag[Military] 4\ARMY, 1\FLEET (see \ruleref{chSpecific:Austria:Baltic
Fleet}), 2\LDND, 6\LD, 2\NTD, 3\LDENDE, 2 fortresses 1/2, 4 fortresses 2/3,
4 fortresses 3/4, 3 fortresses 4/5.
\aparag[Economical] 7\MNU, 2\TradeFLEET (see \ruleref{chSpecific:Austria:Baltic
  Fleet}).



% LocalWords:  Habsburg habsbourg Theresia pVI WoAS Hollande Autriche HNaples
% LocalWords:  WoSS Sicilia Saldigna HHongrie Montecuccoli pV Hongrie pI
