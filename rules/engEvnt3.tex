% -*- mode: LaTeX; -*-

\section{Period III}\label{events:pIII}



\subsection*{Event Table of Period III}

\begin{eventstable}[Period III events table]
  \centering\tabcolsep=4.5pt%
  \begin{tabular}{|l|*{6}{c}|l|}
    \hline
    1\up{st}\textarrow& 1-3 & 4-5 & 6 & 7 & 8 & 9 & 10 \\ \hline
    1 & 1  & 1  & 22  & 5   & 22  & R15 & \textbullet~1--2:\\
    2 & 6  & 12 & 11  & 1   & R11 & R11 & +1 then\\
    3 & 8  & 11 & 18  & 11  & R6  & 12  & \nameref{events:pII}\\
    4 & 1  & 2  & 19  & R6  & R7  & R13 & \textbullet~3--10:\\
    5 & 11 & 3  & R20 & 4   & 8   & 14  & \nameref{events:pII}\\
    6 & 14 & 4  & R21 & 10  & 9   & 20  & \\
    7 & 15 & 5  & 11  & R13 & 10  & 21  & \\
    8 & 17 & 9  & 7   & R15 & 17  & R23 & \\
    9 & 20 & 13 & 3   & 16  & 18  & R2  & \\ \hline
    10& \multicolumn{7}{l|}{\nameref{events:pIV}} \\ \hline
  \end{tabular}
\end{eventstable}
\begin{eventstablespec}[General modifiers for the period]
  For each 4 (complete) turns during which \SPA has taxed \paysHollande since
  the beginning of the game (as per \ref{chSpecific:Spain:Dutch Tax}), the
  second die-roll is modified by \textbf{-1} until \ref{pIII:Dutch Revolt}
  occurs.
\end{eventstablespec}

\eventssummary{%
  pIII:Dutch Revolt|,%
  pIII:VOC|,%
  pIII:League Nassau|,%
  pIII:Amsterdam Stock Exchange|,%
  pIII:East Indian Company|,%
  pIII:End Auld Alliance|,%
  pIII:War Sweden Denmark|,%
  pIII:Oxenstierna|,%
  pIII:War England Scotland|,%
  pIII:Portuguese Disaster|,%
  pIII:Portuguese Annexation|,%
  pIII:Northern Secularisation|,%
  pIII:War Persia Turkey|S{pIII:WPT:Persian Attack}/S{pIII:WPT:Annexation
    Iraq},%
  pIII:Revolt Grenade|,%
  pIII:FWR|L{pIII:FWR Beginning}/L{pIII:FWR Barthelemy}/L{pIII:FWR League}/%
  L{pIII:FWR Succession}/L{pIII:FWR Last Stand},%
  pIII:Revolt Corsica|,%
  pIII:Union Poland Sweden|,%
  pIII:Union Lublin|,%
  pIII:Oprichnina|,%
  pIII:Times of Troubles|O{pIV:Times of Troubles},%
  pIII:War Siberia|,%
} \eventssummary{%
  pIII:Creation Arkhangelsk|,%
  pIII:Persian Safavids|E/O{pIII:Sultanate of Aceh},%
  pIII:Revolt Ceylon|,%
  pIII:Mughal Akbar|E/E/O{pIII:Sultanate of Aceh},%
  pIII:Fall Vijayanagar|E/E,%
  pIII:China Colonial Attitude|E/O{pIII:Sultanate of Aceh},%
  pIII:Sultanate of Aceh|,%
  pIII:Japanese Expedition|,%
  O|,%
  pIII:Union Russia Poland|T{alt. hist.},%
  pIII:Religious War Sweden|T{alt. hist.},%
  pIII:Religious War Poland|T{alt. hist.},%
  pIII:FWR Detailed|,%
  pIII:FWR Beginning|,%
  pIII:FWR Barthelemy|,%
  pIII:FWR League|,%
  pIII:FWR Succession|,%
  pIII:FWR Last Stand|,%
  pIII:FWR Final|,%
}

\newpage\startevents



\event{pIII:Dutch Revolt}{III-1 (1)}{Revolt of the United
  Provinces}{1}{RistoMod}

\history{1568-1609}

\condition{For each occurrence of \ref{pIII:Dutch Revolt}, check the effect
  here.}
\aparag Can only occur after the beginning of period III, unless
\paysVhollande exists. Otherwise re-roll and do not mark off.
\bparag This event triggers the \xnameref{pI:Reformation2} if it has not yet
occurred or the \xnameref{pI:Reformation3} if the second Reformation event had
occurred and not the third.
\bparag If \paysVhollande exists, the Revolt is triggered immediately (either
\xnameref{pIII:DR:First Holland Revolt} or \xnameref{pIII:DR:Subsequent
  Revolts}).
\bparag If \HOL is a major country, apply \ref{pIII:VOC} the second time, and
\ref{pIII:League Nassau} the third time.
% (JCD) If \Holmin exists not owned by \SPA, no further revolt?
\bparag If \HOLmin exists and is not on Spanish diplomatic track, apply
\ref{pIII:League Nassau} instead, then \ref{pIII:VOC} the third time.
\bparag In all other cases, the Revolt of \HOL occurs (possibly again). Keep
reading.
\aparag[Revolt and Spanish religious choice]
\bparag If \SPA is \CATHCR, the Revolt is triggered immediately (either
\xnameref{pIII:DR:First Holland Revolt} or \xnameref{pIII:DR:Subsequent
  Revolts}).
\bparag If \SPA is \CATHCO, \SPA must refuse or grant \emph{Commercial
  Liberties} to \paysHollande. A refusal triggers the Revolt as above.
\bparag If \SPA gives \emph{Commercial Liberties} to \paysHollande, \SPA gains
{\bf 1} \STAB then 1d10 is rolled, added to the following modifiers:
\begin{modlist}
\item[\bonus{+1}] for each turn of taxes on \paysHollande
\item[\bonus{-2}] if the Truce of Augsburg is in effect
\item[\bonus{-1}] if \ENG is Catholic
\item[\bonus{-1}] if \xnameref{pIII:FWR} has occurred at least once and the
  \paysHuguenots never had a favourable truce.
\end{modlist}
The result is:
\begin{modlist}[2em]
\item[\textlessequal0] \paysHollande becomes a Special \VASSAL of
  \SPA\NGTnb{a}
\item[1--2] \paysHollande becomes a normal minor, initially vassal of
  \SPA\NGTnb{b}
\item[3--5] \paysHollande becomes a neutral minor\NGTnb{c}%
  \begin{tikzpicture}[remember picture,overlay]
    \NGTnotabene{}{See~\xref{pIII:DR:Independence without Revolt}}
  \end{tikzpicture}
\item[\textgreatequal6] Revolt (either \xnameref{pIII:DR:First Holland Revolt}
  or \xnameref{pIII:DR:Subsequent Revolts}). The \xnameref{pIII:DR:War Holland
    Portugal} may also be activated.
\end{modlist}


\subevent[pIII:DR:First Holland Revolt]{First Revolt against the Spanish
  Crown}

\phevnt
\aparag The Major country \paysmajeurHollande (or \HOL) is created and
\MAJHOLx changes to this new power according to the rules for the Grand
Campaign.
\aparag \HOL owns its national territory: \theminorprovincesshort{hollande},
regardless of their last owner. \SPA loses 10 \VP for each of the provinces
now of \HOL that were not his just before the event. \paysprovincesne is
dissolved and does not exist anymore.
\bparag Former (non-Spanish) owners of those provinces can declare a war
against \HOL but have no \CB.
\aparag\label{onlyfirstrevolt1} \provinceBrabant and \provinceLimburg are
militarily controlled by \HOL, regardless of their owner at the time of event.
\bparag If this owner is not \SPA, he has the choice to give them to \HOL or
has to declare a limited intervention immediately, as an ally of \SPA in the
Religious War; \HOL may then freely involve fully this power in the war. Both
provinces are valid ground for the war even if the intervention is limited.
\bparag If the owner was \SPA and \SPA conceded \emph{Commercial Liberties} at
the beginning of the event, both provinces are now owned by \HOL; else they
remain Spanish.
\aparag \SPA owns a \Presidio of level 3 in \provinceZeeland.
\aparag \HOL has a \STAB of {\bf +2}, a \DTI and a \FTI of 3; the
technological markers of \HOL are placed 1d6 boxes in front of the Latin
technology, and 7-1d6 for Naval technology (same roll!) Its initial Royal
Treasury is 400\ducats.
\aparag \HOL deploys the following counters: \MNU of \RES{Instruments} in
\provinceZeeland, of \RES{Cloth} in \provinceUtrecht, of \RES{Metal} in
\provinceGelderland (all of level 1, level 2 if \emph{Commercial Liberties}
were granted); 1 \ARMY\facemoins, 1 \FLEET\faceplus, 2 \DT, and 4 levels of
fortification anywhere in owned provinces.
\aparag\label{onlyfirstrevolt2} The current \HOL monarch is
\monarque{Willem I} with values 7/9/9. He lasts seven turns and does not check
for survival during the first three. He is also a general
\leaderwithdata{Willem I}. The government is \terme{Stadhouder}.
\aparag\label{onlyfirstrevolt3} \HOL knows 
% Pierre notes, 2008
\seazoneAcores, \seazoneCanarias and 8 other zones of its choice. Sea zones
with malus count as 1+malus zones in this count.
% 1d10+4 sea zones,
% chosen between those of \SPA and \POR, forming a continuous path from Europa
% (add 4 if \emph{Commercial Liberties} were granted).
\aparag All non-Dutch units inside territories held by \HOL are removed as per
normal peace procedure.
\aparag \HOL is at war with \SPA and \SPA is considered to be victim of a
declaration of war at this turn. No calls for allies are made.  This is a
Religious Civil War between \HOL and \SPA (see \ref{chDiplo:Religious Civil
  War}).
\aparag\label{onlyfirstrevolt4} Place a Dutch controlled \REVOLT \faceplus in
\provinceVlaanderen and \REVOLT \facemoins in \provinceFlandre and
\provinceHainaut.

\phdipl
\aparag An Armistice will be possible, after the first turn of revolt (this is
an exception to the rules on Religious Wars).
\aparag Usual foreign interventions are permitted.  if \FRA is involved in
\ref{pIII:FWR}, its intervention is restricted as follows.
\bparag If \FRA is \CATHCR or Protestant, \FRA may only use its own forces
(and not those of the heretic minor) to help the side sharing its
religion. The French heretic minor country may make a foreign intervention by
its own to help the side sharing its religions; this is decided by the \MAJ
that controls this country when it rebels.
\bparag If \FRA is \CATHCO, it can make a foreign intervention with any side
(not both at the same time).

\phadm
\aparag During the first turn of war, \HOL can exceed the purchase limits for
naval units and buy land forces without any double or triple price multipliers
for exceeding the basic allowance.
\aparag\label{onlyfirstrevolt5} All units bought during the first turn of the
war and placed under \monarque{Willem I} become automatically \Veteran.

\phpaix
\aparag This event can terminate in two ways:
\bparag \SPA conquers all \HOL national provinces. In this case \SPA has won
the war and \HOL is no more. \MAJHOL player has to wait for another
opportunity to play a Major country (according to the rules of the Grand
Campaign).  All the rules for \paysHollande possessed by \SPA are applied
again. The \COL or \TP of \paysHollande remain and are part of \SPA for
military aspects, but they can not be improved. The commercial fleets are
managed as before the war. The Taxation of Holland is possible anew.
\bparag A peace of any kind is made between \SPA and \HOL. Exceptionally, a
peace of level 5 allows the transfer of any number of provinces (3 if the
powers do not agree). As an additional condition to normal peace conditions
\SPA must recognise the independence of \HOL after which all normal rules
apply and \HOL has become an ordinary player country.
\aparag A peace treaty between \SPA and \HOL cannot be made during the same
turn the revolt event occurred.  White peace is not allowed to end this war.
\aparag Any peace treaty between \SPA and \HOL entails an enforced peace of 3
consecutive turns between those two countries, that can only be broken by
using a \CB given by an event.  During this period, neither of them can
declare war to the other, nor to their respective vassals.
\aparag After peace has been made between \HOL and \SPA, \HOL can continue
harassing Spanish annexed Portugal (see \ref{pIII:DR:War Holland Portugal})
until the end of period IV.
\aparag During the war between \HOL and \SPA neither side loses \STAB due to
the number of turns engaged in war as per normal rules. Instead, for being at
war with each other, or with the allies of each other, they lose the following
fixed amounts:
\begin{modlist}
\item[Period III] \SPA {\bf 1} \STAB, \HOL {\bf 1} \STAB.
\item[Period IV] \SPA {\bf 2} \STAB, \HOL {\bf 1} \STAB.
\item[Period V+] \SPA {\bf 3} \STAB, \HOL {\bf 2} \STAB.
\end{modlist}
\aparag This applies only to the war between \HOL and \SPA due to this event
and only to \SPA and \HOL.  Other allies involved in this war lose \STAB in
the usual manner as well as \HOL and \SPA for non-connected wars.

\phinter
\aparag \SPA receives 5\VP each turn that the Independence of \HOL is not
recognised (the war of Revolt goes on or the Revolt has failed) in period III.
This bonus is reduced to 2\VP during period IV and terminates in period V. The
bonus is given even if the turn was spent in Armistice.


\subevent[pIII:DR:War Holland Portugal]{War between Holland and Portugal}

\condition{If \HOL is in Revolt against \SPA and \paysPortugal has been
  annexed by \SPA according to \ref{pIII:Portuguese Annexation}, add the
  following event to a Revolt (first and subsequent ones).}

\phevnt
\aparag \paysPortugal and \HOL are involved in an Overseas War, as long as the
War of Revolt continues between \SPA and \HOL.
\aparag \paysPortugal uses its forces as defined in \ref{pIII:POR
  Ann.:Portugal Annexed} and \SPA can help it as they are allied.

\phdipl
\aparag An Armistice in the war between \SPA and \HOL does not imply an
Armistice between \PORmin and \HOL.

\phadm
\aparag All \COL and \TP of \POR occupied by \HOL give all their revenue to
\HOL (and none to \SPA) as if owned.

\phinter
\aparag All \TP\facemoins of \POR occupied by \HOL can be replaced by \HOL \TP
with 1 level less.
\aparag Portuguese \TP may not be annexed in this way or burnt by \HOL at the
turn where \SPA recognises the Independence of \HOL (but see afterwards).

\phpaix
\aparag This war terminates at the end of period IV, or if \HOL is conquered
or recognised by \SPA or if \PORmin breaks free from \SPA. In the latter case,
\HOL has a free \OCB against \paysPortugal to be used immediately. Else, \HOL
has to leave Portuguese territory at the end of the turn.
\aparag When the Independence of \HOL is recognised, \HOL can immediately
annexe two \COL or \TP of \paysPortugal, or only one \COL or \TP if the peace
is unfavourable.  In both cases, the level of the \COL/\TP remains the
same. \HOL must have military control of these settlements, but the agreement
of \SPA about which \TP/\COL are gained is not needed.
\bparag Instead of one \TP/\COL, \HOL may obtain the right of implantation of
fleets in \STZ bordering Portuguese \COL/\TP.
\aparag Until the end of Period IV, \HOL having won the Revolt gains an \OCB
against \paysPortugal as long as this country is annexed by \SPA.


\subevent[pIII:DR:Subsequent Revolts]{Subsequent Revolts}

\phevnt
\aparag If the Revolt occurs again after a failed Revolt, the rules are the
same as in \xnameref{pIII:DR:First Holland Revolt} except for the following
points.
\aparag Points \XNRofsectionfalse\ref{onlyfirstrevolt1},
\ref{onlyfirstrevolt2}, \ref{onlyfirstrevolt3},
\ref{onlyfirstrevolt4}\XNRofsectiontrue \ and \ref{onlyfirstrevolt5} above are
ignored.
\aparag Technological markers are where they were left at the end of the
previous Revolt, or at the box of Latin technology (the better). The Treasury
of \HOL is 200\ducats.  The monarch is determined at random; \monarque{Willem
  I} is not available, neither as a Monarch nor as a General.


\subevent[pIII:DR:Independence without Revolt]{Independence without Revolt}

\phevnt
\aparag \future{This option is experimental...}%
\paysHollande becomes a minor country composed of all its national territory:
\theminorprovincesshort{hollande}, regardless of their last owner.  \SPA loses
5 \VP for each of the provinces now in \HOL that were not his own just before
the event. \paysprovincesne is dissolved and does not exist anymore.
\bparag Former (non-Spanish) owners of those provinces can declare a war
against \HOL but have no \CB.
\aparag The characteristics of \paysHollande are as defined in the Annexes.
It has one action of \TP, one action of \COL, one action of \CONC all with
medium investment. It places its \TradeFLEET as in period I or II until the
end of period V; afterwards it has one action for commercial fleet.
\aparag If \paysHollande is not a special \VASSAL of \SPA:
\bparag Any war engaged in period III between \SPA and this country becomes a
Revolt, as per \xnameref{pIII:DR:First Holland Revolt} (keeping existing \COL
or \TP and all discoveries of sea zones made by \SPA (and \POR if annexed by
\SPA) and all its own land discoveries);
\bparag The player formerly in charge of the \TradeFLEET of \paysHollande has
the mandatory task of resolving administrative actions of \HOL and will
resolve its discoveries;
\bparag This player earns \VP for any monopolies of \paysHollande.
\bparag \paysHollande is subject to normal diplomacy;
\aparag If \paysHollande is a special \VASSAL of \SPA, this country has the
task of resolving the administrative actions (which are mandatory).
\paysHollande is not subject to diplomacy.
\aparag \paysHollande may be involved in Overseas Wars, and may declare one
(controller's choice).

\phadm
\aparag If \paysHollande is a special \VASSAL of \SPA, \SPA gains 50\ducats
per turn plus 2\ducats for each face of \COL/\TP of \paysHollande (funds
raised from \paysHollande), instead of the usual income of the provinces for a
vassal.
\aparag Until the end of period V, if at peace or doing limited intervention
only, \paysHollande raises one \FLEET\faceplus and one \ARMY\faceplus to be
used overseas each turn, in discoveries and battles against Natives; it also
has one simple campaign at each round. The named \LeaderE and \LeaderC of \HOL
are used, with a minimum of one \LeaderE and one \LeaderC to be taken in
unnamed counters. The discoveries or wars are resolved by the player in charge
of the administrative actions.
\aparag If at war, it uses its full forces and reinforcements.

\effetlong
\aparag \paysHollande may Revolt against \SPA because of some war between
these two countries in period III.
\aparag Or \paysHollande may break free or/and become a Major Power because of
\ref{pIV:TYW}.
\aparag Finally, a peace of level 5 against \SPA breaks the special status of
\VASSAL and \paysHollande becomes a neutral minor country; the player waiting
to play \HOL according to the rules of the Grand Campaign has the choice to
immediately become \HOL.
\aparag In all those cases, the event and the rules described here terminate.



\event{pIII:VOC}{III-1 (2)}{Vereenigde Oostindische Compagnie}{1}{RistoMod}

\history{Vereenigde Oostindische Compagnie was created in 1602}

\condition{}
\aparag If this event already happened because of \ref{pIV:Dutch Colonial
  Dynamism}, reapply \numberref{pIV:Dutch Colonial Dynamism} instead.
\aparag If \HOL does not satisfy 2 conditions over 3 re-roll and do not mark
off: having at least 3 \TP in \continent{Asia}; this is turn 20 or after;
Dutch government is \terme{Parliament}.

\phevnt
\aparag \HOL may create the VOC at any event phase, as soon as it wants. It
costs 100\ducats and causes the rest of the event.
\aparag At the moment the VOC is created:
\bparag \HOL receives 3 levels of commercial fleets to be placed in any
eligible \STZ bordering \continent{Asia}.
\bparag \FTI for \HOL is immediately raised by one level.


\phadm
\aparag The turn the VOC is created, \HOL may ignore restriction
of~\ref{chExpenses:Pioneering}.

\effetlong
% Maximum \FTI in the \ROTW is now 5 in periods III and IV.
% \aparag \HOL gains one action of \TP and a minimum of one \LeaderC in period
% III.
\aparag \HOL gains an \OCB against any Catholic country having \TP or \COL in
\continent{Asia}, valid during periods III and IV.
\aparag Periods limits of \HOL change once the VOC is created. 


\event{pIII:League Nassau}{III-1 (3)}{League of Nassau}{1}{PBNew}

\history{Alternative history}

\condition{}
\aparag If \HOL is a Major country and \SPA did not recognise it, apply \RD
with a \REVOLT in the following table instead of this event and mark off.
\bparag 1.~\provinceZeeland, 2.~\provinceHolland, 3.~\provinceUtrecht,
4.~\provinceLimburg, 5.~\provinceLiege, 6.~\provinceLuxemburg,
7.~\provinceHainaut, 8.~\provinceFlandre, 9.~\provinceVlaanderen,
10.~\provinceBrabant.
\aparag If the Independence of \HOL was recognised or if \paysHollande is
minor country, apply the rest of the event.

\phevnt
\aparag \paysHollande breaks any diplomatic status with \SPA, whether special
\VASSAL or regular diplomatic status and becomes neutral.
\aparag The countries \paysOldenburg, \paysHanovre, \paysHanse and \paysBerg
form an offensive alliance, called the League of Nassau. They leave an
existing \GE.  They are considered as one country for declarations of war and
for peace terms (excepted for separate peaces).
\aparag The League of Nassau declares a war to \paysTreves, \paysCologne and
\paysMayence. The Emperor of the \HRE has a free \CB to declare war to the
League of Nassau and be allied to the three Archdioceses; in this case it
controls these countries. If the Emperor does not involve himself in the war,
the Sole Defender of the Catholic Faith will have control of those
Archdioceses during the war, or \SPA is nobody has this responsibility.
\aparag Any country having diplomatic status with one of these minor countries
can do a limited intervention to support this side (and then has to break
diplomatic relations with minor countries of the enemy side), except the
Emperor who can only enter war with the Archdioceses (and can do this in a
limited way or full war).
\bparag Note that if the Emperor is \AUSmin, \SPA can make a limited
intervention on the side of \AUSmin as well.
\aparag If \HOL exists, it can do a limited intervention as an ally of the
League of Nassau.
\aparag The League of Nassau is controlled by the following Major existing
power: \HOL, the player responsible for the administrative actions of
\payshollande (if not \SPA), \ENG if Protestant, \FRA if Protestant, \SUE
(regardless of religion).

\phadm
\aparag The three Archdioceses can use the counter of the \HRE for their
troops even if the Emperor is not at war along them. They take their
reinforcements in defensive mode during the first turn of the war.
\aparag The countries in the League of Nassau take their first turn
reinforcements in offensive mode, except \paysHanse which has Offensive or
Naval reinforcements (controller's choice).

\phmil
\aparag The minor countries of the \HRE that are at war can pass through and
stop in every province of the \HRE. They can not siege or pillage provinces
belonging to minor countries not involved in this war.
\aparag The troops of the Emperor have the same right of passing through and
stopping in the \HRE, as well as the forces in limited intervention of other
Major countries.

\phpaix
\aparag A test to begin a Religious War in \HRE is made at the end of the
first turn of this war started by the League of Nassau.  This test is modified
by \bonus{+2} if \SPA if \CATHCR and \bonus{0} if it is \CATHCO. See
\ref{pIV:TYW} for the result of the test and the possibility of this Religious
War, and the renewal or not of the test on following turns.  If no such war
occurs, peace can be made on the following conditions.
\aparag Each minor country obeys to the usual rules for peace. Those in the
League are allied so a peace against only one country is a separate peace.
\aparag A minor country forced to sign an unconditional surrender breaks from
the League for ever. The League ceases to exist when only one country remains
in it, or at the time of \shortref{pIV:TYW}.
\aparag If the three Archdioceses are not supported by the Emperor, the League
tries to obtain peace using the system for minor countries as if it was one
major country (the controller of the League of Nassau decides of the terms of
peace).
\aparag The controlling player of both sides gain 5 \PV for each level of
favourable peace signed at the end of the war, and 10 \PV for each enemy minor
country that had to sign an unconditional surrender; they lose 10 \PV for each
minor country of their side that had to sign an unconditional surrender.
Those \PV are not awarded if the war triggers \shortref{pIV:TYW}.

\effetlong
\aparag If the League of Nassau exists when \shortref{pIV:TYW} occurs, it will
join the Protestant side. The League ceases existence when there is only one
minor country left in the League at the end of a war.



\event{pIII:Amsterdam Stock Exchange}{III-2}{Amsterdam Stock
  Exchange}{1}{Risto}

\history{1608}

\effetlong
\aparag \HOL can from now on lend 150\ducats in the Diplomacy phase, plus
100\ducats during the turn (instead of 100\ducats plus 50\ducats).
\aparag \HOL has more money available for international loans.
\begin{oldcompta}
  \aparag From now on \HOL receives a bonus equal to its \DTI to all die-rolls
  on international loan amount and interest (not length) in the loans table
  % (Jym) Table just said "international" and usually more up to date
  \aparag \HOL is also more resistant to Bankrupt and more tolerant to
  trespassing of commercial limits.
  % (Jym) What? According to chapter 3 it should be less resilient
\end{oldcompta}



\event{pIII:East Indian Company}{III-3 (1)}{East Indian Company}{1}{Risto}

\history{1600}

\condition{}
\aparag If both following conditions are not satisfied: this is turn 20+ and
\ENG has at least 2 \TP in \continent{Asia}, apply first \ref{pIII:End Auld
  Alliance}, or re-roll if already played.

\phevnt
\aparag \ENG may create the EIC at any event phase, as soon as it wants. It
costs 100\ducats and causes the rest of the event.
\aparag \ENG receives 2 levels of commercial fleets to be placed in any
eligible \STZ bordering \continent{Asia}.

\effetlong
\aparag \FTI for \ENG is immediately raised by one level and its maximum level
is permanently raised as written in the tables.
\aparag Turn limits for \ANG change.



\event{pIII:End Auld Alliance}{III-3 (2)}{End of the Auld Alliance}{1}{PBNew}

\history{1560 - Treaty of Edinburgh}
% \dure{until the end of the Anglo-Scottish war}

\condition{}
\aparag Occurs only if \paysecosse is at present inactive. Otherwise re-roll.
\aparag If \ANG has chosen the ``Mary Stuart'' option in \ref{pII:Act
  Supremacy}, this event is void of any effect.
\aparag \ENG can refuse this event (mark as played) by losing {\bf 2} \STAB
and 20 \VP. It also loses the control of \paysecosse and can then make no
diplomacy on it until the end of period.

\phevnt
\aparag Apply \ref{pI:Reformation3} (John Knox in Scotland!).
\aparag \paysecosse wants to declare itself liege of \FRA.  \ENG has the
choice to contest this declaration, by using a free \CB against \paysecosse.
In this case, \paysecosse stays Neutral and Allies can be called for this war
as per normal rules. Else, \paysecosse becomes \VASSAL of \FRA.

\phadm
\aparag For the duration of the event, \paysecosse receives reinforcements in
defensive attitude.



\event{pIII:War Sweden Denmark}{III-4 (1)}{War between Sweden and
  Denmark}{1}{PB}

\history{1563-1570}

\condition{This event can not occur if \SUE is not a Major Power; do not mark
  off and re-roll if it is not the case.}

\phevnt
\aparag \paysDanemark declares a war to \SUE. If \SUE was at peace,
\paysDanemark is controlled according to the normal rules. If it was not, the
controller is chosen in priority among the countries at war against \SUE.

\phadm
\aparag During the first turn \paysDanemark will take its reinforcements in
offensive status with an added bonus of \bonus{+2}. For the following turns,
the attitude is free but \paysDanemark keeps the \bonus{+2} to reinforcements
during all this war.



\event{pIII:Oxenstierna}{III-4 (2)}{Oxenstierna}{1}{PBNew}

\history{1612-1654}
\dure{as long as \strongministre{Oxenstierna} remains the excellent minister}

\phevnt
\aparag \SUE receives an excellent Minister, \ministre{Oxenstierna}, which has
values 6/8/8. He will last for 3 turns plus a random length for Minister, see
\ref{eco:Excellent Minister}.
\aparag \SUE gains immediately {\bf 1} in \STAB.

\phadm
\aparag \SUE may ignore restriction of~\ref{chExpenses:Pioneering} for this
turn.



\event{pIII:War England Scotland}{III-5}{War between England and
  Scotland}{1}{Risto}

\history{1542-1548}

\condition{}
\aparag Occurs only if \paysecosse is at present inactive. Otherwise re-roll.
\aparag Cannot take place if \ref{pIV:Union Scotland} has already occurred.
In that case mark-off and re-roll. May cancel \shortref{pIV:Union Scotland} if
the latter occurs while the present event is still active.
\aparag \ENG can refuse this event (mark as played) by losing {\bf 3} \STAB
and 20 \VP.  It also loses the control of \paysecosse and can then make no
diplomacy on it until the end of period.

\phevnt
\aparag \paysecosse declares war against \ENG, which loses the control of
Scotland.
\aparag \ENG can immediately call allies as per normal rules.
\aparag If this leads to declarations of war against \paysecosse, the
controller of \paysecosse may come to its help as per normal rules, and so on.
\aparag If \paysecosse is neutral, its control is decided randomly between
\SPA and \FRA unless one of them is already at war with \ENG (and the other
not), in which case that country takes precedence and receives \paysecosse in
\EG. Control cannot be refused.

\phadm
\aparag For the duration of the event \paysecosse receives reinforcements in
offensive attitude.



\event{pIII:Portuguese Disaster}{III-6}{Portuguese Disaster in
  Africa}{1}{Risto}

\history{1578}

\condition{}
\aparag Can occur only if \paysportugal exists as a minor country, otherwise
re-roll.
\aparag If \ref{pIII:Portuguese Annexation} is in effect, apply \RD with a
\REVOLT in \SPA.
\aparag Else if \dynasticaction{C}{3} was played, activate
\shortref{pIII:Portuguese Annexation} just after the effects of this event.

\phevnt
\aparag If \paysPortugal is currently activated in a war, it immediately
offers a mandatory white peace to all its enemies.
\aparag \paysPortugal loses all its non-national provinces (excepted
\provinceTanger and \provinceAcores); they are given back to their owner of
1492.
\aparag Whatever the current status of \paysPortugal, the reference level of
each Portuguese \TradeFLEET in the \ROTW map is reduced by one (even if being
thus eliminated).
\aparag All Portuguese fortifications in the \ROTW map outside
\continent{Asia} and \continent{Brazil} lose {\bf 1} level.  Remaining
fortifications are added to the basic forces maintained by \paysPortugal (but
will not be rebuilt once destroyed).
\aparag From now on, \paysPortugal has only one action of \TP/\COL each turn,
and no fleet action.



\event{pIII:Portuguese Annexation}{III-7}{Annexation of Portugal by
  Spain}{1}{RistoMod}

\history{1580-1640}

\condition{Can occur only if \paysPortugal is a minor power.}

\phevnt
\aparag \SPA receives a free \CB against \paysPortugal until the end of
current period.  If \SPA is \CATHCR, then during the first turn of a war
caused by this event, \paysPortugal receives no reinforcements.
\aparag In addition to the usual involvement of a \MAJ to help an attacked
minor country, \ENG and \FRA can make a limited intervention to help
\paysPortugal.
\aparag[Annexation]
\bparag If \xnameref{pIII:Portuguese Disaster} has not happened yet and \SPA
achieves an unconditional victory over \paysPortugal, this minor is considered
to have been annexed to \SPA in a special way and \xnameref{pIII:POR
  Ann.:Portugal Annexed} is applied. The political marker of \pays {Portugal}
is placed in \ANNEXION of \SPA.
\bparag If \nameref{pIII:Portuguese Disaster} has happened, \paysPortugal is
at war by its own (neither full nor limited intervention), \SPA can annex
\paysPortugal by winning a peace of level 2 against it.
\bparag If \nameref{pIII:Portuguese Disaster} has happened and \paysPortugal
has help from a \MAJ, \SPA will annex \paysPortugal by winning a peace of
level 4 against it.


\digression[pIII:POR Ann.:Portugal Annexed]{Portugal in Annexation}

\phdipl
\aparag \paysPortugal is permanently annexed to \SPA, and its political marker
is placed accordingly. The counters of \paysPortugal are not removed from
play.
% \aparag \provinceTanger is given by \paysPortugal to \SPA.
\aparag In game terms \paysPortugal is treated as a part of \SPA mainly for
\VP purposes.  In most other respects it becomes a special, permanent \VASSAL
of \SPA. This applies to separate wars and peace treaties, placement of units
and markers, etc. and covers all aspects not specially modified in this event
description.  If \paysPortugal was currently engaged in a separate war against
someone else than \SPA, its enemies must immediately sign a white peace with
it, or declare war to \SPA with a free \CB (unless they are already at war
with \SPA).
\aparag \SPA annexes all non national provinces of \paysPortugal except
\provinceAcores.
\aparag \SPA cannot voluntarily cede any part of \paysPortugal, including
\COL/\TP to other players. Neither can it sell Portuguese sea charts or grant
authorisation of trade in a sea bordering a Portuguese \COL/\TP.
\aparag A War declared against annexed \paysPortugal gives a free \CB (\OCB if
this is an Overseas war) to \SPA to intervene in the war.  A war against \SPA
does not imply necessarily \paysPortugal in the war.

\phadm
\aparag \SPA receives a part of the incomes of \paysPortugal: it receives all
income from \TP/\COL, Exotic Resources, \TradeFLEET (but no income from
European provinces, foreign or domestic commerce, manufactures -- these are
removed). This income can not be higher than 400\ducats, plus the East Indies
convoy.
\bparag \SPA gains the \PV for the monopolies detained by \paysPortugal. It
does not combine resources or fleets of \paysPortugal with it to determine
monopolies or the ownership of a Commercial Centre.
\aparag \SPA must pay for the maintenance and recruitment of Portuguese units
and fortresses as if they were Spanish units (except that their content
remains that of Portuguese units and they can only be placed within Portuguese
territory, including \COL/\TP).  \SPA has 3\GD of basic forces and an
additional limit of recruitment of 1\LD and 1\ND to maintain or raise
Portuguese units. One unnamed Portuguese \LeaderE leads the naval forces.
\aparag \SPA can make administrative actions for Portuguese \TradeFLEET and
\COL/\TP, but using Portuguese \FTI/\DTI (without the former Portuguese bonus
for \ROTW actions).  \SPA has 2 (in periods III and IV) or 1 (period V)
actions for Portuguese \COL and can use also its own actions for Portuguese
establishments. One of these actions can be used on a Portuguese \TP each
turn. \SPA has one action of \TradeFLEET in periods III and IV for Portuguese
fleets.

\phmil
\aparag \SPA must pay for all campaign activations of Portuguese units jointly
with Spanish units.

\phpaix
\aparag \SPA can renounce annexation at the end of any peace phase (except on
the same turn when \ref{pIV:Portuguese Revolt} occurs) losing control of
\paysPortugal and {\bf 3} \STAB.
\bparag If \SPA renounces the inheritance before \shortref{pIV:Portuguese
  Revolt} occurs, \paysPortugal is placed in forced \EG of \SPA until the
death of current Spanish monarch. After that, it is treated as normal minor
and subject to diplomacy.
\bparag If \SPA renounces the inheritance after \shortref{pIV:Portuguese
  Revolt} has occurred, \paysPortugal becomes neutral and it makes a white
peace with \SPA.  The rebels are considered to have won.



\event{pIII:Northern Secularisation}{III-8}{Secularisation of
  \pays{Teutoniques1}}{1}{PB}

\history{1561}

\condition{}
\aparag If \ref{pI:Fall Teutonic} was not played, it is played this turn as a
supplementary event.

\phevnt
\aparag Minor country \pays{Teutoniques1} is destroyed.  Its provinces are
shared as follows:
\bparag \provinceEstland is given to \SUEsue.
\bparag \provinceMemel joins the \region{Duche de Prusse} and is given to
whoever controls this Duchy (\POLpol or \paysBrandebourg).
\bparag \provinceLivonija and \provinceKurland are associated as the
\region{Duche de Kurland}. This Duchy is claimed by \SUEsue and \POLpol.
\bparag If one of these provinces was conquered by another country than the
one that should take it, this wronged country has a \CB against the country
possessing the province. A minor country will always use this \CB.
\bparag All other provinces are given to their legitimate owner in 1492 (as
indicated on the map).
\aparag[War for Kurland]
\bparag \POLpol has a \CB against \SUEsue; refusal to use it costs {\bf 1
  \STAB} and gives all the \region{Duche de Kurland} to \SUEsue. \POLmin
always uses the \CB.
\bparag \SUEsue has a \CB against \POL; refusal to use it costs {\bf 1 \STAB}
and gives all the \region{Duche de Kurland} to \POL. \SUEMin always uses the
\CB.
\bparag If both countries use their \CB against the other one, \POLpol owns
both provinces, but \SUE has initially the military control of
\provinceLivonija.  They can make no Armistice on the first turn of this war.
\bparag If neither \SUE nor \POL use this \CB, \payscourlande is created as a
normal minor country with the two provinces.

\phdipl
\aparag Any country which was at war against \pays{Teutoniques1} has an
immediate free \CB to be used jointly against \POLpol and \SUEsue (and
\payscourlande if it exists).  This might provoke a three-sided war (excepted
if one of \POL or \SUE at least has abandoned the \region{Duche de Kurland})
in which the invading country keeps its eventual initial military control of
any province in \pays{Teutoniques1}.
\aparag If such a country does not declare war, its forces are withdrawn from
\pays{Teutoniques1} and it gives up any conquered province that was owned by
\pays{Teutoniques1} in 1492 to their new owner (as defined above).
\aparag Any other country adjacent to \pays{Teutoniques1} when they disappear
has a \CB to be used jointly against \POLpol and \SUEsue (and \payscourlande
if it exists).



\event{pIII:War Persia Turkey}{III-9}{War between Persia and Turkey}{1}{Risto}

\history{1606-1639}

\condition{}
\aparag If main provinces of \paysperse are conquered, activate a
\xnameref{chSpecific:Persia:uprising}.
\aparag First time : if \paysperse is inactive, use \xnameref{pIII:WPT:Persian
  Attack}.
\aparag Second time, or first time and \paysperse is currently at war against
\TUR, use \xnameref{pIII:WPT:Annexation Iraq}.
\aparag Otherwise, re-roll and do not mark off.


\subevent[pIII:WPT:Persian Attack]{Persian Attack of Turkey}

\activation{}
\aparag If \TUR does not own provinces that were Persian at the beginning of
the game, it may refuse the event in two ways:
\bparag By losing {\bf 3} \STAB and 150\ducats.
\bparag Or, by surrendering immediately to \paysPerse conceding a peace of
level 2 and ceding a province bordering Persian territory (in priority a
province adjacent to \paysPerse).
\aparag In this case the box is marked off, but the event can happen later if
rolled for anew.

\phevnt
\aparag \paysPerse declares war against \TUR.
\aparag \TUR can immediately call for allies as per normal rules.
\aparag If this leads to declarations of war against \paysperse, the
controller of \paysperse may come to its help as per normal rules, and so on.
\aparag If \paysperse is neutral, it is played by \SPA (which cannot then come
to its aid).

\phadm
\aparag \paysperse receives reinforcements in offensive status for the
duration of the event.


\subevent[pIII:WPT:Annexation Iraq]{Annexation of Iraq}

\phevnt
\aparag \paysIrak is annexed to \paysperse and removed from game.
\aparag If \TUR owns any province initially in \paysIrak, place there a
\REVOLT \faceplus and one or 2 \REVOLT \facemoins controlled by \paysperse;
one \REVOLT in each province, the \REVOLT \faceplus is placed at random.

\phadm
\aparag If either of the conditions above are met with, Iraqi basic force is
added to the forces of \paysperse until the end of the war.



\event{pIII:Revolt Grenade}{III-10}{Revolt in Sierra Nevada}{1}{Risto}

\history{1568-1570}

\phevnt
\aparag Place a \REVOLT \facemoins in non-Muslim \provinceGranada,
\provinceCordoba and \province{La Mancha}. The \REVOLT are controlled by \TUR.

\phdipl
\aparag \TUR has a \CB against all the owners of revolted provinces.
\bparag Exceptionally, \TUR may make a limited intervention at the side of the
\REVOLT as if this was a civil war.
\aparag If \TUR declares war to the controller of \provinceGranada or is in
limited intervention against it, it receives 5 \VP at the moment its (or its
minor allies) troops arrive to any of the revolted provinces. This does not
have to be done during the current turn, but the bonus \VP are gained only
once.

\phmil
\aparag During the rebellion there exists an additional malus of \bonus{-3} to
all attempts to suppress \REVOLT in \provinceGranada if \SPA is \CATHCR. An
additional malus of \bonus{-1} is received for each Turkish or minor allied
\LD inside any province in \REVOLT (even if besieged).

\phinter
\aparag \REVOLT caused by this event can never extend beyond \provinceGranada,
\provinceCordoba, \provinceMurcia and \province{La Mancha} (with a maximum of
two \REVOLT counters per province).
\aparag If the \REVOLT survives the first turn, place a minor general on it.
\aparag For each interphase this event continues \TUR receives 2 \VP.  This
bonus is increased to 10 \VP per interphase whenever \TUR or its minor ally
units are within \provinceGranada (a war must have been declared to the
controller to do this).
\aparag If a \REVOLT \faceplus exists for a whole turn in \provinceGranada
without being suppressed at any point during this turn, a new minor
\paysGrenade is created and becomes a permanent \VASSAL of \TUR (but the war
is not necessarily ended). It owns any of the 4 mentioned provinces having a
\REVOLT in them, but has no capital (so can be destroyed by any country).
\aparag If \provinceGranada is later annexed by any other player than \TUR,
place a \REVOLT \facemoins in the province during the peace phase and consider
this event as having been activated again, but without the malus of \bonus{-3}
for suppress of \REVOLT . If \paysGrenade still exists (owning other provinces
than \provinceGranada), consider this \REVOLT as being controlled by it.

\effetlong
\aparag[Final expulsion of the Moriscos] Certain effects of the politics of
expulsion are removed.

\event{pIII:FWR}{III-11}{Wars of Religion in France}{5}{PBNew}

\history{1560-1598}

\condition{See at the end of this section the \ref{pIII:FWR Detailed} which is
  the detailed description of those wars.}



\event{pIII:Revolt Corsica}{III-12}{Revolt in Corsica}{1}{Risto}

\history{1564-1567}

\phevnt
\aparag A \REVOLT \faceplus is placed in \provinceCorsica. The preference list
for the control of this \REVOLT is the one for the (would-be) \paysCorse.
However, the \REVOLT cannot be controlled by the controller of \paysGenes, who
is omitted from this list.
\aparag \paysGenes immediately offers white peace to any enemy currently
engaged in war with it.  From now on, it cannot declare war on anyone as long
as the event lasts.
\aparag If no-one controls \paysGenes at present, the controller is chosen as
per normal rules when minor neutral is activated.
\aparag This event is treated as a civil war in \paysGenes (see
\ref{chDiplo:Religious Civil War}). Only the controllers of \paysGenes and of
the \REVOLT are allowed to do a \terme{Foreign Intervention} with their own
forces.

\phadm
\aparag This event must be played even if no player country is involved in
it. \paysGenes receives reinforcements and can use its troops as if activated
in a war.

\phinter
\aparag If the \REVOLT survives the first turn, place \leaderSampiero who is
now available for 5 turns.
\aparag If the \REVOLT survives four turns, a new minor country \paysCorse is
created and the rebellion is over. The controller of the \REVOLT gains 10
\VPs.
\aparag If the rebellion is crushed, controller of \paysGenes gains 10\VPs.


\event{pIII:Union Poland Sweden}{III-13}{Union between \paysmajeurPologne and
  \paysmajeurSuede}{1}{PB}

\history{1595-1599}

\condition{}
\aparag If there is no Major power \POL, re-roll and do not mark off.
\aparag If there is no Major power \SUE and \POL is not Supporter of
Orthodoxy, re-roll and do not mark off.
\aparag If the Polish Monarch is \monarque{Zygmunt I} during its first 5 turns
of reign, re-roll and do not mark off.
\aparag Apply one of the following events, according to the religious
attitudes:
\bparag If \SUE is Catholic, apply \ref{pIII:Religious War Sweden};
\bparag If \POL is Supporter of Orthodoxy, apply \ref{pIII:Union Russia
  Poland};
\bparag If \POL and \SUE are Protestant, apply \ref{pIII:Religious War
  Poland}.
\bparag If \SUE is Protestant and \POL is Catholic, use this present event.
\bparag If none of the preceding situations happened, mark off the box and
apply \RD.

\phevnt
\aparag The Polish Monarch dies and the Heir of the Swedish Crown is elected
in Poland. \POL has now the Monarch \monarque{Zygmunt III}, with values 5/5/6
and is also general \leaderwithdata{Zygmunt III}. Its reign will last 9 turns.
\aparag The Vasa Dynasty remains on the Polish throne until a Dynastic crisis
occurs in Poland or an event (or some elected specific general) changes the
Dynasty; \POL has to lose 2 \STAB to keep its Dynastic Claims or this
terminates the event.  From now on, \POL has Dynastic Claims over \SUE.

\effetlong
\aparag \POL can renounce its Claims at any diplomatic phase (that is a
declaration) and that terminates the event. \POL loses 1 \STAB.
\aparag Each time there is a new monarch in \SUE, \POL has a \CB against \SUE
at this turn to claim for its Inheritance. In case of Dynastic Crisis in \SUE,
\POL is a valid pretender as long as it has Dynastic Claims over Sweden.
\bparag The first time after the beginning of the event that this situation
happens, \POL must either use the \CB or lose 2 \STAB or renounce its Claims
(costs 2 \STAB).
\aparag The first new Swedish Monarch after this event will be
\monarque{Charles IX}, with values 8/6/6 (but not a general) and random
duration (ignore \terme{Fragile health} and \terme{Dynastic crisis}.
Exception: if \monarque{Gustave Adolphe} was to be the new monarch due to
another event, use \monarque{Gustave Adolphe}.

\phdipl
\aparag If a war is declared because of its \CB, \SUE is now in Civil
Religious War (see \ref{chDiplo:Religious Civil War}). Apart from \POL, only
foreign intervention in the war is allowed.
\aparag The first time a war is declared due to Dynastic Claims, \POL gains
the military control of one province owned by \SUE, chosen by \POL (the
capital is forbidden). This effect is not applied for subsequent wars.

\phadm
\aparag \POL can recruit troops in Swedish provinces that are under its
military control, at double price (because those are not normal recruitment
provinces).
\aparag \POL can use outside its own territory only land forces paid with
ducats and not paid with free maintenance (mercenaries only). There is no such
restriction for naval forces, nor if the kings of \SUE are \PROTRIG in which
case the war is not limited for \POL. Note that it is not mandatory to use the
free maintenance.

\phpaix
\aparag If \SUE wins the war, a valid peace term is to ask for renouncement to
Dynastic Claims (equivalent of one province).
\aparag If \POL wins the war with a peace of level 3 or more, or forces an
unconditional peace, the Monarch of \POL becomes ruler of \SUE as one of the
Victory conditions (instead of one province).
\bparag The Monarch of \SUE is executed; now \SUE uses the values of the
Monarch of \POL. \SUE is considered Catholic during the Union (in every
aspects).
\bparag \SUE has a mandatory offensive alliance with \POL in which she is
complied to answer any call.
\bparag \SUE can not declare war without a \CB or the agreement of \POL.  It
can not declare war against \POL.
\bparag The alliance is in question when the Monarch of \POL dies or if \POL
refuses to answer a call for defensive war (not offensive war), or if \POL
declares a war against \SUE.  A new monarch is rolled for \SUE. \POL having
still Dynastic Claims over Sweden, it can renew the war to impose its ruler
but it renews the Union if \POL wins a peace of any level against \SUE. As
long as the war continues, the union exists for Victory Conditions, if not in
its consequences.
\bparag Note that if \ref{pIII:Union Poland Sweden} is rolled for a new time
when the Union exists, \SUE is Catholic and \ref{pIII:Religious War Sweden} is
thus applied.



\event{pIII:Union Lublin}{III-14}{Union of Lublin}{1}{PB}

\history{1569}

\condition{}
\aparag This event is described in \ref{pII:Union Lublin}.
\aparag If it has already occurred, mark off and apply either
\ref{pIII:Oprichnina} or \ref{pIII:Times of Troubles}.



\event{pIII:Oprichnina}{III-15 (1)}{Oprichnina}{1}{PB}

\history{1565-1572}
\dure{as long as there is a \REVOLT in Russia.}

\condition{}
\aparag If \monarque{Ivan IV} has not been yet Monarch of \RUS, do not mark
off and re-roll.
\aparag If \monarque{Ivan IV} is already dead, mark off and apply \RD the
first time (with a \REVOLT in \RUS), the second event the next time.

\phevnt
\aparag \RUS is in Civil War for the duration of the event.
\aparag \REVOLT are placed in \provinceMoscou and \provinceNovgorod; their
force is randomly decided.
\aparag Another \REVOLT is rolled for in Russia.
\aparag The Russian leader \leaderKurbsky is withdrawn from game as long as
\monarque{Ivan IV} rules in \RUS and can not be used.

\phadm
\aparag \RUS is not restricted by limits of land building this turn only, and
has no penalty for doing so. %
% Jym, 05/2011 No more restricted area for RUS.  As \provinceMoscou is
% occupied by a \REVOLT through, it has to pay double cost to build troops
% somewhere else.
However, the cost for building new troops is doubled for the duration of this
event.

\phmil
\aparag \monarque{Ivan IV} must take the field and lead a land stack as long
as this event last, respecting the usual hierarchy rules.
\aparag The land force of \monarque{Ivan IV} pillages every province it is in
at the end of each round.
% Bertrand is tied, brought before the cine-club and must watch in extenso
% Ivan the Terrible of Eisenstein, and this once per turn as long as the event
% is active. Sylvain has the moral duty of commenting the movie.

\phpaix
\aparag If at the peace phase there is no \REVOLT left in \RUS, one Russian
\ARMY (one counter and the equivalent of 4 \DT) is destroyed by \RUS and \RUS
gains {\bf 1} in \STAB.



\event{pIII:Times of Troubles}{III-15 (2)}{The Time of Troubles in
  Russia}{1}{PB}

\history{1605-1613}

\condition{}
\aparag If \ref{pIII:Oprichnina} is still in effect, mark off and apply \RD.
\aparag If not, apply \ref{pIV:Times of Troubles}.



\event{pIII:War Siberia}{III-16}{War in Siberia}{1}{Risto}

\history{non-historical}

\condition{Can occur only after the elimination of \payssiberie.  Otherwise
  re-roll.}

\phevnt
\aparag Place a Turkish controlled \REVOLT \facemoins in each Russian \COL/\TP
in \continent{Siberia}.

\phadm
\aparag Native forces within the revolted provinces return to their full
strength and are activated.
\aparag Furthermore, during the first turn only, an unmodified die-roll is
made for rebel reinforcements in offensive attitude. Troops thus received fill
the former \payssiberie counters and can be placed in any of the revolted
provinces.

\phmil
\aparag Rebels using \payssiberie counters draw supplies from native
territories (the same way as natives do), but can only do so either if there
is no \RUS controlled forts/fortresses in the province, or from the \REVOLT
counters, which they can use as supply bases.
\aparag Rebels using \payssiberie counters can move also outside their
original provinces.
\aparag Rebel natives and \payssiberie units automatically try to destroy
Russian \COL/\TP in provinces they occupy at the end of a full round, if these
are not protected by Russian units or fortresses. Roll one die: on 7 or more,
the \COL/\TP is destroyed.

\phinter
\aparag \REVOLT caused by this event never extend during the redeployment
phase.
\aparag During the native attacks phase count each \REVOLT \facemoins counter
as 2 native \DT when counting the modifications to the attack die-roll, and
rebel forces using \payssiberie counters are also used.



\event{pIII:Creation Arkhangelsk}{III-17}{Arkhangelsk and the Muscovy Trade
  Company}{1}{Risto}

\history{1584}

\condition{Requires permission from \RUS and \ENG to take effect. Otherwise
  re-roll.}

\phevnt

\aparag The port of Arkhangelsk (to the north of the European map) is created.
It cannot be accessed by any units, but still meets the requirement of having
a port along the Atlantic Ocean for purposes of placing commercial fleets.

\aparag \CTZ Russia is created, but its monopoly bonus remains 5 until the
\xnameref{chSpecific:Russia:St-Petersburg}.

\aparag English commercial fleet of 4 levels is placed in \CTZ Russia.

\aparag Muscovy Trade Company provides \ENG automatically with 10 \VP and
50\ducats.

\phadm

\aparag Until the \nameref{chSpecific:Russia:St-Petersburg}, \ENG can use both
its \DTI and \FTI as modifiers to all commercial actions in \CTZ Russia.
\aparag \RUS may ignore restriction of~\ref{chExpenses:Pioneering} for this
turn.



\event{pIII:Persian Safavids}{III-18}{Persian Safavids}{1}{PB}

\history{1590-1722}

\phevnt

\condition{}
\aparag If main provinces of \paysperse are conquered, activate a
\xnameref{chSpecific:Persia:uprising}.
\aparag Else, apply only the following effects.

\phevnt
\aparag \paysperse obtains the general \leader{Abbas Shah} that will stay for
6 turns.

\effetlong
\aparag \paysperse has now the same technological level as \TUR.  Its armies
are of class \CAI and it has 3 \ARMY available.
\aparag \paysperse can now send armies through regions in \ROTW belonging to
no one during wars, without activation of Natives. They are constrained by the
supply rules. They can assail and burn \TP or \COL (as if \TP) military
occupied at the end of a turn.



\event{pIII:Revolt Ceylon}{III-19}{Revolts in \granderegionCeylan}{1}{Risto}

\phevnt
\aparag \ROTW area \granderegionCeylan declares war against the owner of a
\TP/\COL in it.
\aparag If this is a minor country, the \TP/\COL will be attacked by the
Natives at the end of the military turn, without any defence from Europe.
\aparag If this is a player, the war proceeds as a normal war against natives.



\event{pIII:Mughal Akbar}{III-20}{The Great Moghol Akbar}{2}{PB}

\history{1556-1605}

\phevnt
\aparag If the non-European minor country \paysMogol does not exist, it is
created now.  It can use 2 \ARMY\faceplus and leader \leaderwithdata{Akbar}.
\aparag If \paysMogol already existed, its ruler only is changed from the
\leader{Grand Moghol} to \leaderAkbar (until replaced by a further event).
\aparag The \paysMogol will try to invade \bonus{4} regions during the turn,
according to \ref{pII:Mughal Expansions}.
\aparag Even if the country has no region after the invasions, it still exists
(and can gain provinces with new events).
\aparag \granderegionBengale has from now on 2 \RES{Spices}, 2 \RES{Products
  of Orient} and 2 \RES{Cotton} available instead on 1 (representing the
change of commercial fluxes because of the Mughals).



\event{pIII:Fall Vijayanagar}{III-21}{Wars in India}{2}{PB}

\history{1565 / 1585-1594}

\phevnt
\aparag If it was still existing, minor country \paysVijayanagar is destroyed
(by internal fights).  Every \TP (not \COL) that are in the minor country
\paysVijayanagar at the time of its disappearance will face an attack by
Natives that are activated against every country this turn.
\aparag If \paysVijayanagar had already been destroyed, every \TP/\COL in
\continent{India} loses 1 level due to internal strife in India.
\aparag \granderegionKarnatika has from now on 2 \RES{Spices} and 2
\RES{Products of Orient} available instead on 1 (representing the change of
commercial fluxes from the north to the south because of the Mughals and the
destruction of the Indian Empire).
\aparag If the \paysMogol exist, they invade one province, the next in the
list according to \ref{pII:Mughal Expansions}.



\event{pIII:China Colonial Attitude}{III-22 (1)}{\paysChine colonial
  attitude}{1}{PB}

\history[Closure of China was the historical choice.]{1557}

\condition{}
\aparag If \paysChine has no \TP, apply \xnameref{pIII:CCA:Closure China}.
\aparag If \paysChine has any \TP left, roll 1d10 added to the number of \TP
it has. If the result if 6 or higher, commercial exclusivity policy in
\paysChine triggers the event \xnameref{pIII:CCA:Closure China}. If the result
is 5 or less, apply \xnameref{pIII:CCA:Commercial Dynamism China}.


\subevent[pIII:CCA:Closure China]{Closure of \paysChine}

\phevnt
\aparag Any country having a \TP in \paysChine may sign immediately a Treaty
with \paysChine, and so gains \dipAT. If accepted, only one \TP of the country
is kept in \paysChine; \TP in excesses are destroyed. If refused, \paysChine
declares an Overseas War against the power.
\aparag From now on, \dipAT allows each country to keep only one \TP in
\paysChine (and not one per region). The remaining \TP can be upgraded, and it
causes no reaction by \paysChine.
\aparag The basic forces and reinforcements of \paysChine are now its mainland
army only (no overseas garrisons of fleets).

\effetlong
\aparag From now on, no new \TP counter can be placed in any area belonging to
\paysChine by means of administrative actions.
\aparag No regular diplomacy is permitted on \paysChine.  The Activation level
of \paysChine becomes 11 (except for areas conquered that are not mainland
\paysChine, where the Activation is 6).

\aparag The only way to have a new \TP in \paysChine is to take control of the
\TP of another country (then the Treaty status is given to the new controller
of the \TP and lost by the previous one) or to force a Treaty on \paysChine by
means of a war against it.

\aparag From now on, the \terme{Manila Galleon} is
available. See~\ref{chSpecific:Manila Galleon}.


\subevent[pIII:CCA:Commercial Dynamism China]{Commercial dynamism of
  \paysChine}

\phevnt
\aparag \paysChine gains a \TP with level 6 in every coastal city of its
territories.  An automatic concurrence with any existing establishment is made
until only one \TP survives in each province. Its fleet in \stz{Formose} rises
to level 5 (and automatic concurrence might also be necessary).
\aparag Japanese \TP in \granderegionCorea and \granderegionFormose are
destroyed (by Chinese invasions).

\effetlong
\aparag \paysChine has a \FTI of 2 (raised to 3 from period V on) and a \DTI
of 3 and uses both \FTI and \DTI for concurrence in its own provinces. Form
now on, consider \stz{Formose} as its \CTZ.
\aparag \TP of \paysChine exploit the resources in their region and those are
counted as normal exploitation for monopolies and evolution of prices.
\aparag European countries having monopoly in \stz{Formose} may declare a
commercial embargo against \paysChine.  No \TP (not \COL) may exploit anything
in \paysChine as long as the embargo continues (both Chinese and European
\TP); so they are not counted in for monopolies and evolution of
prices. Moreover, no commercial fleet in \stz{Formose} gives any income. This
embargo gives an oversea \CB to every European country having a \TP in
\paysChine.
\aparag Each turn, all Chinese \TP in continental \paysChine gain one level
(with a maximum of 6), overseas \TP one level (with a maximum of 3) and
\paysChine gains one \TradeFLEET level in \stz{Formose} (with a maximum level
of 6). Destroyed \TP do not come back but the commercial fleet keeps coming
back even if destroyed.
\aparag Basic reinforcements are increased to one \ARMY\faceplus in mainland,
and 2 \LD, 2 \ND for the garrisons.



\event{pIII:Sultanate of Aceh}{III-22 (2)}{Sultanate of Aceh}{1}{PB}

\history{1565}

\phevnt
\aparag Create the Sultanate of \paysaceh. Place its \TP\facemoins with 3
levels in
% (Jym) putting Centre first as is control the strait for the CC GO.
\granderegionSumatra (in the first empty province: Centre, North then South;
if none, place it in the Northern one and make automatic concurrence).
\bparag It proposes a \dipAT to \TUR that has the choice to accept it or not
immediately.
\bparag Forces are deployed as per the Annex.

\effetlong
% (JCD) This is not a reminder
\aparag Before 1700 \paysaceh has a \TP action every turn (strong investment)
to increase its \TP up to the original level 3, if ever its level is less (or
was destroyed).

\aparag The \TP of \paysaceh may never be annexed at peace.

\aparag No other establishment (\COL or \TP) may be created in the province if
the \TP of \paysaceh is here.
\bparag Existing establishments, including those that would be created while
the \TP of \paysaceh is temporarily destroyed, stay without harm.


\event{pIII:Japanese Expedition}{III-23}{Japanese Expedition in
  \granderegionCorea}{1}{PB}

\history[Both invasions failed, historically.]{1592/1597}

\phevnt
\aparag Place a Japanese \TP in a province of \granderegionCorea,
\provinceSeoul if possible, \provincePyongyang if \provinceSeoul is occupied;
if both are occupied, this event is marked off but ignored.
\aparag The \TP has 3 levels and exploits all resources of \granderegionCorea
(other countries will have to take them by regular concurrence).
\aparag A Japanese colonial force of 1 \ARMY\faceplus defends the \TP; it may
gain \ARMY\facemoins in reinforcement each turn if needed.  This army does not
activate the Natives and an attack in this region may be aimed at the Japanese
only and so does not activate the Natives of \granderegionCorea. As soon as
the \TP is no more Japanese or destroyed, normal activation rules for Natives
apply and the colonial force is removed.



\subsection{Some Alternative History Events}



\event{pIII:Union Russia Poland}{III-A}{Union between \paysmajeurPologne and
  \paysmajeurRussie}{1}{PB}

\history{Alternative history}

\phevnt
\aparag The Polish Monarch dies and the Heir of Russia is elected in
Poland. \POL has now the Monarch \monarque{Dimitri}.  Its values and its reign
length are random, as if an heir from \RUS.
\aparag The Russian dynasty remains on the Polish throne until a Dynastic
crisis occurs in Poland or an event (or some elected specific general) changes
the Dynasty; this terminates the event. From now on, \RUS has Dynastic Claims
on \POL.

\activation{}
\aparag When the current Tsar of \RUS dies, \monarque{Dimitri} becomes the
Monarch of \RUS for its remaining reign length.
\aparag He can choose to abandon the Polish crown; that costs {\bf 1} \STAB to
\RUS, a new dynasty is elected in \POL (as if after a Dynastic Crisis, or a
general-monarch may be elected if one is available), and the event is ended.
\aparag It can choose to keep both crowns and \xnameref{pIII:URP:Effect of the
  Union} is now applied.

\effetlong
\aparag At each time there is a new Tsar in \RUS, beginning with
\monarque{Dimitri}, \POL can accept the Union or try to break it.
\bparag If the Union is accepted, the new Tsar becomes (or remains) the ruler
in \POL and \RUS gains 20 \PV each time.
\bparag If it is refused, a new Monarch is rolled for \POL, as if after a
Dynastic Crisis, or a general-monarch may be elected if one is available.  A
War for Dynastic Union might happen, see underneath.
\bparag Any other event calling for a change of Polish Monarch is impossible
when the Union holds; do not mark off this event and roll anew.


\subevent[pIII:URP:Effect of the Union]{Effect of the Union}

\effetlong
\aparag \RUS and \POL shares the same Monarch; \RUS has the control on the
Monarch (what he is doing, its values, and so on).
\aparag \POL has a mandatory offensive alliance with \RUS in which it is
complied to answer any call.
\aparag \POL may not declare war without a \CB or the agreement of \RUS. If it
has a \CB against \RUS, it can declare war to it and lose \STAB due to
breaking of alliance (but this one is renewed afterwards).
\aparag \RUS has no specific obligation regarding the alliance, and does not
lose \STAB if it doesn't answer the call. It can declare war to \POL but that
breaks the union and this war is now as described in \xnameref{pIII:URP:War
  Dynastic Union}.  Determine a new Polish Monarch.
\aparag \POL does not change of religious attitude because of the Union.


\subevent[pIII:URP:War Dynastic Union]{War for Dynastic Union}

\phdipl
\aparag If \POL has refused a continuation of the Union, \RUS has a free \CB
against \POL to be used immediately, and will lose {\bf 1} \STAB if it refuses
the \CB. In that case, \RUS renounces also to its Dynastic Claims on \POL.
\aparag If a war is declared, \POL is in Civil War against \RUS (see
\ref{chDiplo:Religious Civil War}). \RUS is permitted full intervention in
this war.
\aparag Roll for 2 \REVOLT in \POL when such a war erupts.

\phpaix
\aparag If \POL wins the war or signs a white peace, the Union and the
Dynastic Claims of \RUS are forfeited.
\aparag If \RUS wins the war with a peace of level 2 or more, the Monarch of
\RUS becomes ruler of \POL also as an victory condition (instead of 1
province).
\bparag The previous Monarch of \POL is executed; now \POL uses the values of
the Monarch of \RUS and the Union (see above) is renewed.



\event{pIII:Religious War Sweden}{III-B}{Religious War in Sweden}{1}{PB}

\history{Alternative history}

\condition{}
\aparag \SUE proposes an immediate white peace to every countries is at war
against.  Minor countries sign it, and Major Countries have the choice to sign
such a white peace or to sign an Armistice. If an Armistice is decided,
military occupation remains in provinces where the city is controlled (other
are evacuated), no combat is possible between the enemy sides, and Swedish
provinces that are occupied by enemies are out for the Religious War (see
\ref{chDiplo:Religious Civil War}). The Armistice will last until the end of
the Religious War and causes no loss of \STAB at the end of each turn.

\tour{Turn 1}

\phevnt
\aparag Roll for 4 \REVOLT in \SUE. Those \REVOLT has to be all in Swedish
provinces and in different provinces. The force of the \REVOLT is random but
they all control the city.  This forms the side of Rebels. They are opposed to
Loyalists.
% (JCD) We do not want loyalists keeping to Finland only and resisting forever
% only from there. Possibly do something about that.
\aparag The player of \SUE chooses its side:
\bparag If his initial choice was Catholic, he must play the Loyalists;
\bparag If \SUE is Catholic because of Union with \POL or because of Forced
Conversion, the player can choose Loyalists or Rebels.
\bparag If the player chooses to play Rebels, a new Monarch is rolled for on
the last column for values, with a random reign length (ignore Dynastic
Crisis). The characteristics of the previous Monarch has to be written down
(in case of victory of Loyalists) and this Monarch can be used as a general by
Loyalists.
\aparag A test is made for each military unit (per counter deployed), each
leader and each \COL or \TP with 1d10:
\begin{modlist}
\item[1--5] controlled by Loyalists;
\item[6--10] controlled by Rebels.
\end{modlist}
\aparag The side not played by \SUE is controlled by:
\bparag \POL if this is the Loyalists and \POL is Catholic;
\bparag \HAB if this is the Loyalists and \POL is not Catholic (Protestant or
Orthodox);
\bparag \ENG if this is the Rebels and \ENG is Protestant;
\bparag \MAJHOL if this is the Rebels and \ENG is Catholic.
\aparag During the Religious War, \SUE may not declare any war, nor make
diplomacy on minors (except in reaction). Events calling for an intervention
of \SUE are played as if \SUE makes an immediate Armistice or White Peace.
\aparag Foreign countries can be involved in this war only by foreign
intervention, excepted for what is listed below.

\phdipl
\aparag If \POLpol is Catholic, it has a \CB against the Rebels to join war
alongside Loyalists. \POLmin always uses this \CB.
\aparag If \DANdan is Protestant, it has a \CB against the Loyalists to join
war alongside Rebels. \DANMin uses this \CB only if \POL uses one.

\tour{as long as the war continues}

\phadm
\aparag The side played by \SUE uses the normal rules for Major Powers. It
controls the province where its owns the city and, if playing the Rebels,
disregards any \REVOLT (they don't affect its income because they are allied
to it).
% (JCD) Adapted to new accounting
\bparag Its initial treasury is at most two thirds of the treasury at the end
of the event phase. The loss is of at least 50\ducats.
\begin{oldcompta}
  \bparag The initial treasury is 2/3 of the treasury at the end of the event
  phase.
\end{oldcompta}
\aparag The other side has a basic maintenance equal to that of \SUE in the
current period and receives reinforcements as a minor country.  It uses the
fully controlled provinces (minus \REVOLT for the side of Loyalists) as their
basic income (for the modifier).
\aparag Each side has only a minimum of one general (and has any general
coming from the initial test).

\phmil
\aparag If \POL is at war, it can not have more than one stack in National
provinces of \paysmajeurSuede and provinces of \regionNorvege.

\phinter
\aparag The \REVOLT extend as usual.

\phpaix
\aparag Only unconditional surrender is permitted to Loyalists and Rebels. If
there are no \REVOLT left and no cities owned by Rebels, the Rebels surrender
(whether played by \SUE or as a minor).  If there are no national provinces of
Sweden not in \REVOLT or controlled by the Rebels, a minor Loyalists surrender
automatically.
\bparag If the Loyalists win, \SUE remains Catholic and has its Monarch ruling
before the event.
\bparag If the Rebels win, \SUE becomes \PROTTOL (with a new ruler if they
were not played by \SUE).
\aparag[Consequences for Poland]
\bparag If \POL was at war and the Loyalists win, \POL gains 40 \PV.
\bparag If \POL was at war and the Rebels win, the war continues as a normal
war between \POL and \SUE (a peace can be signed now at the same turn).
\aparag[Consequences for \paysDanemark]
\bparag If \DANdan was at war and the Rebels win, a province of \SUE is given
to \DANdan (choice of \SUE, if possible a province that was once owned by
\DANdan).
\bparag If \DANdan was at war and the Loyalists win, the war continues as a
normal war between \DANdan and \POL/\SUE.  A peace can be proposed at the same
turn.
\aparag The player of \SUE on the losing side loses 20 \PV.



\event{pIII:Religious War Poland}{III-C}{Religious War in Poland}{1}{PBNew}

\history{Alternative history}

\activation{Replaces \ref{pIII:Union Poland Sweden} if \POL is protestant.
  The Swedish heir is elected as king of Poland, but remains protestant. He
  must fight a religious war in its new kingdom. Will be a variation on
  \ref{pIV:Polish Civil War}.}

\clearpage

%% *-* latex-mode *-*



\event{pIII:FWR Detailed}{III-D}{Religious Wars in France}{5}{PBNew}

\history{1562-1598}{The wars are fragmented in 5 parts. \\
  (1) First, Second and Third wars (1562-1570) with many truces broken by one
  side or the other. \\
  (2) Fourth and Fifth wars (1570-1575), where the Massacre
  of the Saint-Barth\'el\'emy heightens the intensity of the war.\\
  (3) Sixth and Seventh wars (1575-1580) where the Catholic League and the
  Duke of Guise seem almighty, and a background announced Dynastic Crisis. \\
  (4) Eighth war (1585-1598) that is the war of Succession for the French
  Crown. \\
  (5) Alternative history: more troubles if France is not Conciliant (mainly
  with foreign support).  }

\dure{until the end of \ref{pIII:FWR Last Stand} or \ref{pIII:FWR Succession}
  (as specified in these events) or at the end of period III.}

\activation{This event is composed by many sections describing first the
  general conditions under which the wars are fought, then specifics of the
  evolution of the Wars: from a set of strictly Religious Wars that go harder
  and harder to a War of Succession. The passage from one event to another is
  described hereafter.}
\aparag This event can not happen before turn 11 (1540). If the turn if 10 or
before, re-roll and do not mark off.
\aparag Only one \ref{pIII:FWR} can be rolled and marked off each turn. If a
second one is obtained, do not mark off and re-roll.
\aparag After the end of this event, \ref{pIII:FWR} triggers an event \RD, and
the box is marked.
\bparag If \FRA is \CATHCO, its Monarch will have a malus of \bonus{+2} to his
Survival Test next turn.
\bparag If \FRA is \CATHCR or Protestant, the \REVOLT is rolled on the table
of \FRA.
\aparag From the first to the end of the last event, \FRA is in religious
Civil War and is limited in many aspects.

\phevnt
\aparag[The states within the State] Two minor countries, \paysHuguenots and
\paysLigue are created for this event. No diplomacy is authorised on them;
they have the same technology and military features as \FRA.
% \bparag When at peace with \FRA, all their provinces and counters (armies,
% leaders,...) belong to \FRA and are used as such.
% \bparag When at war against \FRA, \FRA still earns the income of their
% provinces (except where there are \REVOLT !)  and those minor countries
% never pillage the provinces in \FRA.
\aparag[Les Huguenots]
\bparag \hug has the following provinces (if in \FRA): \provinceCaux,
\provinceTouraine, \provincePoitou, \provinceQuercy, \provinceGuyenne,
\provinceLanguedoc, \provinceBearn, \provinceDauphine, \provinceCevennes
(those provinces have a white shield border).
\bparag \hug is protestant.
\bparag Its main controller is \ENG (if Protestant) or \HOL (if it exists) or
\SUE (if Protestant), else \MAJHOL. This major power will be noted \HUG (and
the minor \hug); it may change at each turn (depending on the changes of
religion).
\aparag[La Ligue]
\bparag \lig has the following provinces (if in \FRA):
% \provinceArmor, \provinceFinistere, \provinceMorbihan,
\provinceNormandie, \provinceMaine, \province{Ile-de-France},
\provinceOrleanais, \provincePicardie, \provinceChampagne, \provinceBerry,
\provinceBourgogne, \provinceLyonnais, \provinceProvence (those provinces have
a yellow shield border).
\bparag \lig is \CATHCR.
\bparag Its main controller is the Sole Defender of the Catholic Faith (if it
is not \FRA), \SPA (if \CATHCR), \ENG (if Catholic), or \SPA (\CATHCO) in the
last possibility.  This major power will be noted \LIG (and the minor \lig);
it may change at each turn (depending on the changes of religion).
\aparag The Loyalists are \FRA and its allies. The Rebels are the revolted
minor country (\lig or \hug) and its allies. \REB is the Major Power that
controls the Rebels (\LIG or \HUG).
\aparag The Catholic side is the one of \lig else of Catholic \FRA.
\aparag The Protestant side is the one of \hug else of Protestant \FRA.

\aparag[Military units]
\bparag Basic forces of \FRA drops to \ARMY \facemoins (or \ARMY \facemoins,
\LD if in period \period{II}). Counters limit for \FRA drops to 3 \ARMY (and 2
\ARMY for each minor).
\bparag Basic forces of the new minors is \ARMY \facemoins, \LD (or \ARMY
\faceplus if in period \period{II}) if it has not the same religion than \FRA
and \ARMY \facemoins (\ARMY \facemoins, \LD if in period \period{II}) if it
has the same religion than \FRA.
\bparag If the minor is at war against \FRA, then it is controlled by its main
controller (either \HUG or \LIG). Else, if \FRA is at war (even civil war
against the other minor) then \FRA may use its troops as if they were french
troops.
\bparag If \FRA is at peace, the main controller of each minor may declare a
limited intervention (following usual rules) of this minor in any existing war
during the diplomatic phase. If the minor has the same religion than \FRA,
this can only be done if \FRA agrees to.  The main controller plays the troops
of the minor and pay for its campaign or reinforcements.
\bparag If \FRA is at peace, and the main controller doesn't want to use the
troops of the minor (or can't), then \FRA may use them as if they were its own
troops.
\bparag If \FRA is at peace, it may build troops of any of the two minors at
regular cost. This counts toward purchase limit of the turn.
\bparag If the minor is not used by somebody else, \FRA has to pay the
maintenance of any troops in addition to the basic maintenance of the minor.
\bparag If \FRA is at peace and the minor has less than its basic forces and
is not used in another war by its main controller, then \FRA has to build
troops of the minor. It is not complied to buy more than the turn limit or to
go bankruptcy, but it must build troops for the minor prior to any other
administrative action. If both minors lack troops, \FRA must start building
troops of the minor having a different religion than its own.
\bparag If \FRA is at peace with the minor, it cannot voluntary dismiss
(i.e. by not paying upkeep) troops of the minor below what was left at the end
of the last civil war. Yet, if the loss is due to any other reason (such as
being used in another war or by its main controller in a foreign
intervention), \FRA is not complied to buy new troops up to this value (just
up to the basic maintenance of the minor).

\aparag[Incomes]
\bparag If \FRA is at war against the minor, then it get no land income from
the provinces of the minor (this also may change the industrial and commercial
incomes of \FRA). Manufactures in these provinces do not provide income
either.
\bparag If \FRA is at peace, the provinces of the minor having the same
religion as \FRA are treated exactly like french provinces: they provide full
land income, manufactures and gold mines provide also full income.
\bparag If \FRA is at peace, the provinces of the minor having different
religion than \FRA only provide half their regular income: land income is
halved (this also change industrial and commercial income), manufactures
provide only half their facial value and half their percentage, gold mines
provide only 10\ducats, \ldots
\bparag If \FRA is in civil war (but not against the minor), provinces of the
minor only provide half their regular income (as above).
\bparag The (land) income not perceived by \FRA does not increase its foreign
trade.
\bparag If \FRA is at peace, it only gets 75\% of its colonial income if its
catholic.

\aparag[Military control]
\bparag If \FRA is not at war against the minor, then both may use provinces
belonging to both of them as supply sources.
\bparag If \FRA is at war against the minor, then supply may go through any
province not containing an unbesieged hostile troop or \REVOLT .

\effetlong
\aparag[Fragile Health of the Valois]
\bparag From the beginning of the event, and as long as the French Monarch is
a Valois, it adds \bonus{+3} to its Survival Test.
\aparag[Lack of Heirs]
\bparag An additional test of Dynastic Crisis is made at the beginning of each
turn (at the Monarch Survival Phase). A malus of \bonus{-1} is applied for
each \ref{pIII:FWR} rolled since the beginning of the game.
\bparag If a Dynastic Crisis occurs (because of the previous test or of a
normal test after the death of the Monarch), apply directly \ref{pIII:FWR
  Succession}.  If a Dynastic Crisis occurs without the death of the Monarch,
the rules of the event use the historical name \anchormonarque{Henri III} to
designate the current Monarch of \FRA.

\aparag[Mandatory Change of Religious Attitude] \FRA can be complied to change
its Religious choice during the war because of a Coup (\ref{pIII:FWR
  Succession}), or an unconditional surrender caused by foreign powers. The
following points occur (but not if the change is voluntary when designating an
Heir of the Valois).
\bparag \FRA goes down to {\bf -3} in \STAB, loses \bonus{-1} in \FTI, and
loses 30 \PV.
\bparag The controller of the side imposing its Heir by a Coup, or the
countries that force a unconditional surrender gain 30 \PV each time a
mandatory change is made.

\begin{digressions}[General troubles in France each time an event happens]


  \digression[pIII:FWR:Politic Crisis]{Politic crisis}

  \phevnt
  \aparag \FRA loses {\bf 2} \STAB.
  \aparag The diplomacy of \FRA is lowered by \bonus{-2} (minimum of 3).
  \aparag \FRA and its adversaries make a mandatory white peace (exception:
  see \ref{pIII:FWR Last Stand}).
  \aparag \FRA is involved in religious civil war when at war against
  Rebels. No-one can declare a war to \FRA at those times, but \MAJ may do
  \terme{Foreign Intervention} in the war each time the war resumes (new event
  or broken Truce) excepted if explicitly forbidden.


  \digression[pIII:FWR:Economic Crisis]{Economic crisis}

  \phevnt
  \aparag On the first event, the Royal Treasury of \FRA is diminished by half
  and loses at least 50\ducats.  On subsequent events, the Royal Treasury of
  \FRA is diminished by 50\ducats if greater than 50\ducats, goes to 0\ducats
  if greater than 20\ducats, and be diminished by 20\ducats if less than
  20\ducats.

  \bparag If \FRA makes a bankruptcy while at war against the rebels, they
  will receive \ARMY\facemoins extra reinforcement (\LD each if there are two
  rebels).
  % On the first event, the Royal Treasury of \FRA is halved and loses at
  % least 50\ducats if \FRA has not enough in its royal treasury, then it lose
  % everything, lose and 1 \STAB and rebels will receive \ARMY \facemoins
  % extra reinforcement (or \DT each if there are two rebels). On the
  % following ones, the Royal treasury of \FRA is halved with a maximal lose
  % of 50\ducats.
  % \aparag It receives no commercial income, no industrial income, and
  % no colonial income during the wars; however if a war resumes at the
  % end of a Truce, \FRA receives half of those industrial, commercial
  % and colonial income during the first turn after the Truce, as the
  % Truce is broken after some months (or years) of relative peace.
  \aparag \FRA (and also \hug and \lig) makes a mandatory trade refusal
  against all other countries. This does not provide CB or entail loss of
  stability and only last while \FRA is in civil war.
  % \aparag Industrial income is halved (after a reduction due to \hug
  % or \lig being at war with \FRA).
  \bparag \FRA only gets 75\% of its colonial income if protestant, 50\% if
  \CATHCO and 25\% if \CATHCR.
  \aparag \FRA can make no economic action (\COL, \TP, \TFI, \CONC) during the
  wars (even if the Truce was broken this turn), except as a reaction to
  concurrence.
  \aparag A \PIRATE\faceplus is placed in \CTZ of \FRA; at most one \PIRATE
  can be here due to this event.
  \aparag \FRA has to pay separate campaigns for any troop going in the \ROTW
  or whose movement end on the \ROTW map (so, it can bring back troops from
  the \ROTW without penalty).


  \digression[pIII:FWR:Uprisings]{Uprisings in France}

  \phevnt
  \aparag If \FRA is \CATHCR or \CATHCO, the Rebels are \hug.  If \FRA is
  Protestant, the Rebels are \lig. \FRA is at war against the Rebels (it is
  not a declaration of war by the Rebels).
  \aparag If \FRA is \CATHCR or \CATHCO, roll 1d10 and place \REVOLT
  \facemoins in the following provinces, excepted in the first province where
  the \REVOLT is \faceplus:
  \bparag result odd: \provincePoitou, \provinceQuercy, \provinceGuyenne,
  \provinceLanguedoc, \provinceAuvergne;
  \bparag result even: \provinceCaux, \provincePoitou, \provinceGuyenne,
  \provinceTouraine, \provinceVendee.
  \aparag If \FRA is \CATHCR, add a \REVOLT \faceplus in \provinceDauphine and
  a \REVOLT \facemoins in \provinceArmor.
  \aparag If the die-roll was 9 or 10 (between 7 and 10 if \FRA is \CATHCR),
  place a \REVOLT \facemoins on a randomly chosen colony (or \TP if no colony
  is available).
  \aparag If \FRA is Protestant, place a \REVOLT \faceplus in
  \province{Ile-de-France}, a \REVOLT \facemoins in \provinceLyonnais and roll
  1d10 for the other ones (the \REVOLT is \faceplus in the first province of
  the list and \facemoins in the others):
  % \bparag result even: \provinceQuercy, \provincePoitou,
  % \provinceDauphine, \provinceTouraine, \provinceCaux;
  \bparag result even: \provinceProvence, \provinceNormandie, \provinceMaine,
  \provinceTroyes, \provinceVendee;
  \bparag result odd: \provinceOrleanais, \provinceChampagne,
  \provinceTouraine, \provinceCaux, \provincePicardie.
  \aparag If the die-roll was 10, place a \REVOLT \facemoins on a randomly
  chosen colony (or \TP if no colony is available).
  \aparag The Rebels receive 2 minor unnamed generals to be placed on \REVOLT
  (they can only lead \REVOLT , not forces of the Rebels, and are eliminated
  when the \REVOLT is finally suppressed).
  \aparag The Rebels own its provinces %and any province of \FRA where
  % there is a \REVOLT .


  \digression[pIII:FWR:Military Troubles]{Military Troubles}

  \phevnt
  \aparag On the first event, only the basic forces of \FRA are kept (\ARMY
  \faceplus, \ARMY \facemoins, \DT), in veteran status.  If \FRA has less than
  this, it will receive less troops than stated. The rebels takes their forces
  first, then the non-rebelled minors and lastly \FRA.
  \aparag Roll 1d10:
  \bparag result even: \FRA keeps \ARMY \facemoins and \DT; the Rebels have
  \ARMY \facemoins; the minor of the same religion as \FRA has \ARMY
  \facemoins;
  \bparag result odd; \FRA keeps \DT, the Rebels have \ARMY \faceplus; the
  minor of the same religion as \FRA has \ARMY \facemoins.
  % \aparag The Rebels may already have some forces (remaining after a
  % favourable Truce) that are added to this forces. All those forces are
  % veterans.
  \aparag If the current turn is in period II, \FRA adds \ARMY \facemoins to
  its forces and the minor sharing its religion add \DT.
  \aparag If \FRA is Emperor of the \HRE, it can use the \ARMY of \HRE as a
  help in this war.
  \aparag Minor country \paysLorraine is activated and allied of the Catholic
  side. It gives 1 \DT, both sides can pass or stop in its provinces but the
  \REVOLT never extend in those.
  \aparag The forces of the Rebels are deployed in their provinces that are in
  \REVOLT .  The forces of \FRA are placed in any province of \FRA that does
  not belong to the Rebels.
  \aparag The naval forces of \FRA may defect as follows. Roll 1d10.
  \bparag result 1-8: \FRA keeps all the naval forces.
  \bparag result of 9: 1 \DN is given to the Major Power controlling the
  Rebels and the rest are Rebel forces.
  \bparag result of 0: 1 \DN is given to the first Protestant country of the
  list: \HOL, \ENG, \SUE, \POL, or to the Major Power controlling the rebels
  if there is none, and the rest are Rebel forces.
  \bparag Naval forces of the Rebels have to go in a port of Rebels.  When, at
  the end of a round, there is no port left to Rebels, the navy comes back in
  the ownership of \FRA.

  \phadm
  \aparag \FRA can build reinforcements as usual and deploys them in provinces
  not owned by the Rebels.
  \aparag The Rebels gain reinforcements in offensive mode on the minor table,
  with a bonus of \bonus{+2} and some other modifiers (see the various steps
  of the events). It gains only the \LD written in the table, not the F, CM or
  leaders.
  \bparag If \FRA is not \CATHCO, add \bonus{+1} to the roll.
  \bparag The Rebels receive 1\fortress if the result is even, or 2\fortress
  if the result is equal to 11 or higher.
  \bparag The reinforcements of the Rebels are deployed in provinces in
  \REVOLT , and the fortresses can only be deployed in provinces with \REVOLT
  \faceplus.
  \aparag[Leaders] After the building of forces, the loyalty of the leaders is
  tested.
  \bparag \leaderMontmorency is always loyal to \FRA.
  \bparag \lig receives \leader{Henri de Guise}.
  \bparag \hug receives \leaderColigny, \leaderConde and, beginning with
  \ref{pIII:FWR Barthelemy}, \leaderNavarre.
  \bparag Every other named leader is checked by rolling 1d10: used by the
  Catholic side if result 1-7; used by the Protestant side if the result is
  8-10.
  \bparag Each side should have at least two leaders. If one has less, it
  receives an unnamed general from those of \FRA.
  \bparag Neither the Loyalists nor the Rebels can use mercenary generals.
  \bparag This repartition is made once for all the following wars; but \FRA
  can use all its leaders (whether from \lig or \hug) during Truces.


  \digression[pIII:FWR:Military Operations]{Military operations during the
    wars}

  \phmil
  \aparag The Rebels control all cities of provinces with \REVOLT at start.
  It draws supply from all provinces of the rebel minor country and from
  cities it controls.
  \aparag \FRA controls all cities of provinces not in \REVOLT .  It draws
  supply from any such provinces.
  \aparag French Leaders of both side are only killed in battles if the
  die-roll was a natural 1. Else, if they would be killed (due to modifiers),
  they are Captured instead and are freed when a Truce happens.
  \aparag The Rebels and the minor countries that are involved in the war have
  a simple campaign each turn. Their controller may pay for a more important
  campaign (by spending the cost of the campaign minus 20\ducats).
  \aparag A city owned by the rebel minor country makes an immediate voluntary
  surrender if besieged by a land stack that is commanded by a named rebel
  general and that sets a siege with at least one \ARMY \faceplus.
  % \aparag If \FRA proposes a Truce favourable to the Rebels at the end
  % of any military round, it is signed immediately and the military
  % operations for this war ceases for the rest of the turn.  This is
  % not possible during events \ref{pIII:FWR Succession} and
  % \ref{pIII:FWR Last Stand}.
  % (Jym) (4) and (5) (no Truce). TODO deal with (1)


  \digression[pIII:FWR:Truces]{Truces during the Wars of Religion}

  \phpaix
  % \aparag Each war from \ref{pIII:FWR Beginning} to
  % \ref{pIII:FWR League} last only one turn, then ends with a
  % Truce favouring one side of the other. The war can resume the next
  % turn or a following one. \ref{pIII:FWR Succession} and
  % \ref{pIII:FWR Last Stand} are full Civil War were no Truces are
  % admitted.
  % \aparag[Favoured side in the Truce]
  % \bparag Excepted if a Truce favourable to the Rebels was proposed by
  % \FRA, determine the side favoured by the Truce according to the
  % following calculus of their respective positions.
  % \bparag A side obtains 1 point per victory in land battle, 2 per
  % major victory, 1 per siege made (but none for automatic surrenders),
  % and the Rebels gain 1 per \REVOLT remaining in France at the end of
  % turn (before any extension).
  % \bparag Whomever has the highest total obtains a favourable Truce;
  % The Rebels are favoured if it is a draw.
  \aparag At the end of any turn, \FRA may propose peace to the rebelled
  minor. This is treated as a regular peace with minor. This can not be done
  during \ref{pIII:FWR Succession} and \ref{pIII:FWR Last Stand} who have
  specific ending conditions.
  % \bparag The initial situation is the one at the beginning of the
  % military phase. Any \REVOLT crushed adds a +1 bonus to the french
  % die roll. Automatic surrenders of cities do not provide any peace
  % differential. (Jym) rather do the opposite => no differential for
  % \REVOLT , but differential for sieges, even automatic ? would be more
  % in line with the rest of the game...
  \bparag The initial situation is the one at the beginning of the military
  phase. \REVOLT do not count toward the peace differential, but provinces
  taken (including automatic surrender) count.
  \bparag Money may not be asked/given as a peace condition.
  \bparag A valid peace condition is the establishment or demolition of a
  safety place. If a safety place is granted, the minor may put a level 3
  fortress in an owned province. If possible it must be put in a province
  initially in \REVOLT \faceplus.
  \bparag The first peace condition must be a safety place (if possible).
  \bparag Any colonial establishment still having a \REVOLT when peace is
  signed immediately lose one level (and may thus be destroyed).
  \aparag If no truce is granted, \REVOLT do not extend as normal but \FRA
  loses stability for both the \REVOLT and the duration of war.
  \aparag Two white peaces count as a losing truce toward french objectives
  (but a single white peace has no effect).
  \aparag If \FRA removes all the \REVOLT counters and retakes all rebel
  fortresses, it may ask for an (automatic) unconditional surrender. It
  immediately gains 20 \VP and 2 stability. The next \ref{pIII:FWR} rolled
  will be played as \ref{pIII:FWR Succession} (even if \FRA is \CATHCO) after
  which the civil wars will permanently stop.
  \aparag If \FRA is for two consecutive turns of the same war at -3 in
  stability and does not manage to sign a peace, it must surrender
  unconditionally and suffer a mandatory change of religion.
  \aparag If \FRA sign a favourable peace, it gain 1 stability.

  \aparag[Effect of a Truce] All \REVOLT are suppressed in \FRA; the naval
  forces are back in the ownership of \FRA (except the \DN that might have
  been seized by foreigners).
  % \bparag If the Truce favours \FRA, \FRA annexes one province of the
  % rebel minor country; all fortresses of the Rebels are withdrawn; all
  % remaining Rebel land forces are given to \FRA; \FRA gains {\bf 1}
  % \STAB.
  % \bparag If the Truce favours the Rebels, they annexe one province
  % (from \FRA or the opposed minor country; if possible a province they
  % once owned); one of its provinces becomes a Place of Safety and
  % gains a level 3 fortress (if possible, a province having a \REVOLT
  % \faceplus before the Truce); the Rebels keep half of its land forces
  % (round up), the rest is dismantled and \FRA will have to build
  % forces anew to its basic forces before the next war; all fortresses
  % in the provinces of the rebel minor country remain.
  % \aparag During Truces, \FRA earn half of its colonial, industrial
  % and commercial income. This includes turns were a Truce is broken by
  % the following mechanism.

  \phdipl
  \aparag During Truces, \FRA is not limited in its diplomatic and
  administrative actions, and can also be involved in external wars (using its
  forces as well as those of \lig and \hug). This does not include turns where
  a Truce breaks down. Remember that both \lig and \hug may be used by their
  controllers.
  \aparag The Truce can be questioned at the beginning of any phase of
  Diplomacy:
  \bparag Roll 1d10 + %the difference between the position value for
  % Rebels at the end of last war and the value for \FRA + \STAB of \FRA
  the level of the peace (in favour of the rebel) \bonus{-1} per turn since
  the beginning of the Truce. If the result is 4 or below, the Rebels will
  break the Truce.
  \bparag Else, if \FRA did not have a favourable Truce and wants to break it,
  it is automatic.
  \bparag If the Truce is broken, apply \ref{pIII:FWR:Politic Crisis},
  \ref{pIII:FWR:Economic Crisis}, \ref{pIII:FWR:Uprisings},
  \ref{pIII:FWR:Military Troubles} at the end of the Diplomatic phase.
\end{digressions}



\event{pIII:FWR Beginning}{III-D (1)}{The first 3 Wars of Religion}{1}{PB}

\tour{Turn 1}

\phevnt
\aparag The Wars of Religion begin; apply the general conditions and the
lasting effects on the Valois as found in \numberref{pIII:FWR Detailed}.
\aparag Apply the full effects of \ref{pIII:FWR:Politic Crisis},
\ref{pIII:FWR:Economic Crisis}, \ref{pIII:FWR:Uprisings} and
\ref{pIII:FWR:Military Troubles}.
\aparag[Michel de l'Hospital]\label{pIII:FWR:Hospital} If \FRA is \CATHCR, it
can now decide to play the rest of the event as \CATHCO.  Its religion changes
immediately, using only the lasting effects of the \ref{pI:Reformation}; the
initial \REVOLT are played as \CATHCR though.
\aparag For each \REVOLT that should be placed, roll a die: the \REVOLT
actually happens only if the result if 6 or higher. Add 1 to the die roll if
\FRA is not \CATHCO (do not add if \FRA just changed its attitude due to
Michel de l'Hospital, but still use the \CATHCR line for placing \REVOLT ).

\phdipl
\aparag No Foreign intervention allowed on the first turn.
\aparag \REB can make a very limited intervention in the war, only with naval
forces (in order to install or break a blockade; no naval movement of Rebel
land forces), that costs no \STAB.
\aparag If \FRA is \CATHCR, \LIG can make a limited intervention as an ally of
\FRA.

\begin{digressions}[Specific conditions of the first event]


  \digression[pIII:FWR:War1 Military]{Military operations during the first
    event}

  \phmil
  \aparag Use the general rules of \ref{pIII:FWR:Military Operations}.
  \aparag If all the leaders of on side are captured, wounded or killed, this
  side signs a level 1 peace in favour of its enemy at the end of the round.

  \aparag At the beginning of each military round (except the first), a new
  \REVOLT is rolled for in France.
  \bparag This revolt is always rolled on the table for \FRA in period III,
  even if this is not the current period. Moreover, if \FRA is catholic,
  \textbf{subtract} its \STAB from the localisation die roll rather than
  adding it.
  \bparag If this \REVOLT is in the rebel minor country and has no \REVOLT nor
  Loyalist land force in it, place a new \REVOLT \facemoins which takes the
  city.
  \aparag A city in \FRA that had not a \REVOLT \faceplus at the beginning of
  the current war nor is a safety place, makes an immediate voluntary
  surrender if besieged by a land stack of \FRA (or its allies) that sets a
  siege with at least one \ARMY \faceplus and there is no more \REVOLT in the
  province (including if the \REVOLT was just crushed this round).


  \digression[pIII:FWR:Peace1]{Peace during the first event}

  \phpaix
  % \aparag A Truce is necessarily signed, and the favoured side is determined
  % as explained in \ref{pIII:FWR:Truces}. All the effects
  % explained here are applied (so the \REVOLT are withdrawn before extension
  % or Stability loss).
  \aparag No peace of level higher than 2 can be signed during this first war,
  especially no unconditional surrender can happen.
  \aparag If \LIG was in limited intervention, allied to a \CATHCR\ \FRA, it
  wins 15 \PV if the Truce is in favour of \FRA and \LIG had forces in at
  least one battle or one siege (including voluntary surrender) against the
  Rebels.
  \aparag \FRA may choose to commit \xnameref{pIII:FWR Barthelemy} on any
  later turn. Consider that \numberref{pIII:FWR Barthelemy} is one of the four
  events rolled this turn and apply all the relevant effects.

  \tour{Turn 2 and following: Extension of the War}


  \digression[pIII:FWR:Continuation1 War]{Extension of the war}

  \phevnt
  \aparag[\REVOLT extension]
  \bparag For each two \REVOLT still existing in France (including colonial
  empire), roll die on the \REVOLT table for \FRA. If the province is neither
  occupied by loyalist troops or part of the non-rebelling minor, place a
  \REVOLT \facemoins which takes the city there.
  \bparag Roll a die. Add 2 if \FRA is \CATHCR, subtract 2 if \FRA is
  protestant. On a roll of 6 or more, place a \REVOLT \facemoins in a randomly
  chosen french colony (if there is no french colony or all have 2 \REVOLT
  \faceplus, in a randomly chosen \TP).

  \phadm
  \aparag Rebel will receive reinforcement as on turn 1.

  \phdipl
  \aparag Foreign interventions are now permitted.
  \aparag \REB can make a limited intervention as an ally on the Rebels (and
  it is not limited to naval forces only from now on).
  \aparag \HOL can make a limited intervention as an ally of a rebel \hug.
  \aparag \SPA can make a limited intervention as an ally of a rebel \lig.

  \phmil
  \aparag[Intervention of \paysPalatinat]\label{pIII:FWR:Palatinate} If
  inactive, \paysPalatinat makes an intervention as an ally of the Rebels (it
  is a mercenary army). It is played by \REB. The intervention force is
  \leader{Jean-Casimir}, one \ARMY \faceplus and 1 \DT.  If the
  \nameref{pII:Schmalkaldic League} or the \nameref{pIII:League Nassau}
  exists, and the Rebels are \hug, this intervention is made with 2 \ARMY
  \faceplus.  \leader{Jean-Casimir} is a general of \paysPalatinat (and serves
  this country if it is at war elsewhere) that will stay as long as
  \ref{pIII:FWR Barthelemy} is not finished.  After that, \paysPalatinat is
  without leader (for intervention) or has normal generals (for other wars).

  \tour{Turn 2 and following: Breaking of Truces}


  \digression[pIII:FWR:Continuation1 Truce]{Breaking of Truces}

  \phevnt
  \aparag If a Dynastic Crisis occurs, \ref{pIII:FWR Succession} will happen
  at this turn. If \numberref{pIII:FWR} is rolled for at this turn, mark off
  the box and consider that it triggers \numberref{pIII:FWR Succession}.
  \aparag As long as a new \numberref{pIII:FWR} is not rolled for, the Truce
  can be broken as explained in \ref{pIII:FWR:Truces}. A war begins anew, as
  explained there.
  \aparag If a new \shortref{pIII:FWR} is rolled for in the Political Event
  Phase, the next phase of \shortref{pIII:FWR} begins (\numberref{pIII:FWR
    Barthelemy}, \numberref{pIII:FWR League} or \numberref{pIII:FWR
    Succession}). Go to this event.
  \aparag If none of this happens, \FRA is in civil peace, and has its
  activity limited by \ref{pIII:FWR:Truces} only.

  \phadm
  \aparag If the Truce has been broken, apply the full effects of
  \ref{pIII:FWR:Politic Crisis}, \ref{pIII:FWR:Economic Crisis},
  \ref{pIII:FWR:Uprisings} and \ref{pIII:FWR:Military Troubles}, and the
  following points.

  \phdipl
  \aparag Foreign interventions are now permitted.
  \aparag \REB can make a limited intervention as an ally on the Rebels (and
  it is not limited to naval forces only from now on).
  \aparag \HOL can make a limited intervention as an ally of a rebel \hug.
  \aparag \SPA can make a limited intervention as an ally of a rebel \lig.

  \phmil
  \aparag The war is prosecuted according to \ref{pIII:FWR:Military
    Operations}, and \ref{pIII:FWR:War1 Military}.
  \aparag[Intervention of \paysPalatinat] If inactive, \paysPalatinat makes an
  intervention as an ally of the Rebels (it is a mercenary army). It is played
  by \REB. The intervention force is \leader{Jean-Casimir}, one \ARMY
  \faceplus and 1 \DT.  If the \nameref{pII:Schmalkaldic League} or the
  \nameref{pIII:League Nassau} exists, and the Rebels are \hug, this
  intervention is made with 2 \ARMY \faceplus.  \leader{Jean-Casimir} is a
  general of \paysPalatinat (and serves this country if it is at war
  elsewhere) that will stay as long as the \ref{pIII:FWR Barthelemy} is not
  finished. Beginning with next event, \paysPalatinat is back to normal (no
  leader for intervention or normal generals for other wars).

  \phpaix
  % \aparag A Truce is necessarily signed, and the favoured side is determined
  % as explained in \ref{pIII:FWR:Truces}. All the effects
  % explained here are applied (so the \REVOLT are withdrawn before extension
  % or Stability loss).
  \aparag If a Major Power makes a limited intervention and the side it helps
  obtains a Truce in its favour, the Major Power gains 10 \PV if it had land
  forces in at least one battle or one siege (including voluntary surrender)
  against the enemy side.

\end{digressions}



\event{pIII:FWR Barthelemy}{III-D (2)}{The Saint-Barthelemy}{1}{PB}

\tour{Turn 1}

\phevnt
\aparag A new war breaks out. Apply the full effects of \ref{pIII:FWR:Politic
  Crisis}, \ref{pIII:FWR:Economic Crisis}, \ref{pIII:FWR:Uprisings} and
\ref{pIII:FWR:Military Troubles}.
\aparag \leaderNavarre is available as a \paysHuguenots general.

\phdipl
\aparag No Foreign intervention is allowed.
\aparag \REB can make a somewhat limited intervention in the war, only with
naval forces (in order to make or break blockade; no naval movement of Rebel
land forces) or with land forces in coastal besieged provinces of the Rebels,
in order to stop the siege; afterwards it can withdraw or remain in this
province only.
\aparag The Rebels control all cities in the rebel minor country (and not only
those with a \REVOLT in there).
\aparag \FRA can then announce an attempt of
\xnameref{pIII:FWR:Saint-Barthelemy}, and resolves this odious deed. This is
of course mandatory if this event happen due to \FRA's choice during
\ref{pIII:FWR Beginning}.
\aparag If \FRA is \CATHCR, \LIG can make a limited intervention as an ally of
\FRA.

\begin{digressions}[Specific conditions of the second event]


  \digression[pIII:FWR:Saint-Barthelemy]{Massacre of the Saint-Barth\'el\'emy}

  \phdipl
  \aparag 1d10 is rolled for every rebel leader, excepted \leader{Henri de
    Guise} and \leaderNavarre. An even result means that the leader was killed
  in the Massacre.
  \aparag Each city in the rebel minor country is taken by \FRA by rolling
  1d10 higher than the level of the fortress; one die is rolled for each
  city. The cities taken this way are military controlled by \FRA but still
  owned by the rebel minor country.
  \aparag The Rebels will have a malus of \bonus{-1} to receive its
  reinforcements at this turn.
  \aparag The Rebels can no longer make a limited intervention in
  \ref{pIII:Dutch Revolt}.
  \aparag \FRA loses {\bf 1} \STAB.
  \aparag The Survival roll of the French Monarch is modified by an additional
  \bonus{+1} until the end of the Wars of Religion.


  \digression[pIII:FWR:War2 Military]{Military operations after the
    Saint-Barth\'el\'emy}

  \phmil
  \aparag Use the general rules of \ref{pIII:FWR:Military Operations}.
  \aparag If all the leaders of on side are captured, wounded or killed, this
  side signs a level 1 peace in favour of its enemy at the end of the round.
  \aparag At the beginning of each military round (except the first), a new
  \REVOLT is rolled for in France. If this \REVOLT is in the rebel minor
  country and has no \REVOLT nor Loyalist land force in it, place a new
  \REVOLT \facemoins which takes the city.
  \aparag \FRA (and its allies) have a bonus of \bonus{+1} to suppress \REVOLT
  in France and perform automatic surrenders of rebel fortresses as in the
  previous war.
\end{digressions}

\phpaix
% \aparag A Truce is necessarily signed at the end of the turn, and the
% favoured side is determined as explained in \ref{pIII:FWR:Truces}. All the
% effects explained here are applied (so the \REVOLT are withdrawn before
% extension or Stability loss).
\aparag If \LIG was in intervention, allied to a \CATHCR\ \FRA, it wins 15 \PV
if the Truce is in favour of \FRA and \LIG had forces in at least one battle
or one siege (including voluntary surrender) against the Rebels.

\tour{Turn 2 and following}

\phevnt
\aparag The event goes on as described in \ref{pIII:FWR:Continuation1 Truce},
except that the military operations follow the rules of \ref{pIII:FWR:War2
  Military}, or as in \ref{pIII:FWR:Continuation1 War} if no peace was signed.



\event{pIII:FWR League}{III-D (3)}{The Rise and Fall of the League}{1}{PB}

\tour{Turn 1}

\phevnt
\aparag A new war breaks out. Apply the full effects of \ref{pIII:FWR:Politic
  Crisis}, \ref{pIII:FWR:Economic Crisis}, \ref{pIII:FWR:Uprisings} and
\ref{pIII:FWR:Military Troubles}.
\aparag If \REB spends 50\ducats, the Rebels will have a bonus of \bonus{+1}
to their reinforcement roll.
\aparag If \FRA is \CATHCR or \CATHCO, \LIG may give finances to \lig.  It
spends 100\ducats and takes the control of the stack commanded by
\leader{Henri de Guise} (he can take new forces during the military rounds as
long as the hierarchy is respected). One purpose of this is to attempt a Coup
by the League (as explained in \ref{pIII:FWR:League Coup}).

\phdipl
\aparag Usual Foreign interventions are permitted (even during the first
turn).

\begin{digressions}[Specific conditions of the third event]


  \digression[pIII:FWR:War3 Military]{Military operations during the League}

  \phmil
  \aparag Use the general rules of \ref{pIII:FWR:Military Operations}.
  \aparag At the beginning of each military round (except the first), a new
  \REVOLT is rolled for in France. If this \REVOLT is in the rebel minor
  country and has no \REVOLT nor Loyalist land force in it, place a new
  \REVOLT \facemoins which takes the city.
  \aparag \FRA (and its allies) have a bonus of \bonus{+2} to suppress \REVOLT
  in France and perform automatic surrenders of rebel fortresses as in the
  previous wars.


  \digression[pIII:FWR:League Coup]{Guise Coup and assassination}

  \phpaix
  % \aparag A Truce is necessarily signed at the end of the turn, and the
  % favoured side is determined as explained in
  % \ref{pIII:FWR:Truces}. All the effects explained here are
  % applied (so the \REVOLT are withdrawn before extension or Stability loss).
  \aparag If \LIG has taken control of \leader{Henri de Guise} and this
  general is not Captured, it may attempt a Coup that will make \leader{Henri
    de Guise} the Heir of the kingdom, by spending 100\ducats more.
  \aparag If \FRA is \CATHCO, or if \LIG has taken control of \leader{Henri de
    Guise}, \FRA may attempt to murder this pretender, even if \LIG does not
  attempt a Coup.
  \aparag Both those operations are described in the following event,
  \ref{pIII:FWR Succession} and are resolved as described in
  \xnameref{pIII:FWR:Coup Murder Pretender}.
  \bparag If the Coup is successful, \ref{pIII:FWR Succession} begins the very
  next turn, with \monarque{Henri de Guise} as the mandatory Heir (see
  afterwards).
  \bparag If \leader{Henri de Guise} was murdered and no event
  \shortref{pIII:FWR} happens (by Dynastic Crisis or rolled event), the Truce
  is broken by the \lig who is the Rebel for one particular war. Apply the
  procedure for a Truce broken, with \lig as the Rebels.
\end{digressions}

\tour{Turn 2 and following}

\phevnt
\aparag The event goes on as described in \ref{pIII:FWR:Continuation1 Truce},
except that the military operations follow the rules of \ref{pIII:FWR:War3
  Military}, or as in \ref{pIII:FWR:Continuation1 War} if no peace was signed.
\bparag If \leader{Henri de Guise} was murdered the previous turn and no
\shortref{pIII:FWR} happens (either by Dynastic Crisis or rolled event), the
Truce is now broken by the \lig who is the Rebel for this particular war.
Apply the procedure for the breaking of a Truce, with \lig as the Rebels.
\lig receives the general \leaderwithdata{Mayenne}.
\bparag Else, the Rebels are those of the previous war if the Truce is broken.
\aparag[Foreign limited interventions] (added to those already allowed).
\bparag Some limited interventions are allowed here; a country can help only
the first at-war country listed, or none at all.
\bparag \HOL can help \hug else a non \CATHCR\ \FRA.
\bparag \ENG Protestant or \CATHCR can help \hug else a non \CATHCR\ \FRA.
\bparag \ENG \CATHCR can help \lig, else a non Protestant \FRA.
\bparag \SPA can help \lig, else a non Protestant \FRA.



\event{pIII:FWR Succession}{III-D (4)}{War of Succession}{1}{PB}

\activation{This events is activated by a Dynastic Crisis during the Wars of
  Religion, or as the fourth event of \numberref{pIII:FWR}, or after a
  successful Coup by \leader{Henri de Guise}.}

\tour{Turn 1}

\phevnt
\aparag \hug and \lig revolt and will fight to impose their pretender on the
French Crown. Every one is sure now that there is no direct Heir of the last
Valois Monarch, \monarque{Henri III}.
\aparag If the French Monarch \monarque{Henri III} died at the beginning of
this turn, \FRA has to choose its Heir. Apply now the effects of
\xnameref{pIII:FWR:Designation Heir}, followed by the effect of the new
Religious attitude.
\aparag If a Coup was successful at the previous turn, the designated Heir is
now the one of the side having made this Coup. Apply his choice of Religious
Attitude.
\aparag Otherwise, apply only the event corresponding to the current Religious
attitude of \FRA; \FRA will have the opportunity to modify the would-be Heir
at the time of the death of the last Valois Monarch.
\aparag Only a Coup or a mandatory change of religion can change the Heir once
he is appointed.
\aparag Apply the full effects of \ref{pIII:FWR:Politic Crisis},
\ref{pIII:FWR:Economic Crisis}, \ref{pIII:FWR:Uprisings} and
\ref{pIII:FWR:Military Troubles}. Also apply \ref{pIII:FWR:War4 Military} and
\ref{pIII:FWR:War4 Peace}.
\begin{digressions}[The choice of the Heir]


  \digression[pIII:FWR:Designation Heir]{Designation of the Heir}

  \phevnt
  \aparag There are three possible Heirs.  Each one is linked to the choice of
  a Religious attitude, and \FRA can not change completely its attitude on its
  own: \CATHCR can not choose Protestant and a Protestant \FRA can not choose
  \CATHCR.  Any other choice is permitted.  \FRA can be forced to change its
  attitude because of a Coup.
  \aparag[\CATHCR] The Heir is \monarque{Henri de Guise}.  If \leader{Henri de
    Guise} is alive, the general is also the Heir; if not it's a cousin with
  random military capacities. The Heir has values 6/9/7.  When the Monarch is
  \monarque{Henri de Guise}, \FRA gains a free maintenance for one \ARMY
  \faceplus, event if it is still in Civil War. \FRA immediately annexes
  \provinceLorraine.
  \aparag[\CATHCO] The Heir would be \monarque{Henri IV}, that is a converted
  \monarque{Henri de Navarre}. If \leaderNavarre is alive, the general is also
  the Heir; if not it's a cousin with random military capacities. The Heir has
  values 9/9/9. When the Monarch is \monarque{Henri IV}, \FRA gains a free
  maintenance for one \ARMY \faceplus, event if it is still in Civil War.
  \aparag[Protestant] The Heir is \monarque{Henri de Navarre} who remains
  Protestant. If \leaderNavarre is alive, the general is also the Heir; if not
  it's a cousin with random military capacities. The Heir has values 9/9/9.
  \aparag[A new religious attitude] The designation of an Heir changes
  immediately the Religious Stand of \FRA. The Heir is Crowned now if the king
  is dead, or assists the king and will be crowned at the time of its
  death. If the Heir dies, another of the same family (and same
  characteristics) will stand forward. An Heir does not make Survival Test
  before its crowning; it will last 5 turns beginning with the turn of its
  crowning.
  \aparag
  Apply one of \xnameref{pIII:FWR:France is Protestant},
  \xnameref{pIII:FWR:France is Counter-Reformation} or
  \xnameref{pIII:FWR:France is Conciliant}.


  \digression[pIII:FWR:France is Protestant]{France is Protestant}

  \phevnt
  \aparag \lig rebels, following the general rules.
  \aparag If \monarque{Henri III} is dead and the Heir is crowned, \LIG can
  make a limited intervention from the first turn of the war.  Moreover, \lig
  will have a bonus of \bonus{+2} to its reinforcement roll.
  \aparag If \leader{Henri de Guise} is dead, \lig receive the general
  \leaderMayenne (B.2.2.2).
  \aparag \LIG can always make a limited intervention from the second turn of
  the war onward.
  \aparag \hug is immediately annexed by \FRA: its provinces become french
  provinces (and provide income as such) and its units (armies, leaders)
  become french units. Both the counter limits and free maintenance of \FRA
  resumes their regular values.


  \digression[pIII:FWR:France is Counter-Reformation]{France is \CATHCR}

  \phevnt
  \aparag \hug rebels, following the general rules.
  \aparag If \monarque{Henri III} is dead and the Heir is crowned, \HUG and
  \HOL can make a limited intervention from the first turn of the war.
  \aparag \HUG and \HOL can always make a limited intervention from the second
  turn of the war onward.
  \aparag \lig is immediately annexed by \FRA: its provinces become french
  provinces and its units become french units. Both the counter limit and
  maintenance of \FRA resume their regular values.


  \digression[pIII:FWR:France is Conciliant]{France is \CATHCO}

  \phevnt
  \aparag[If the king is \monarque{Henri III}, a Valois]
  \bparag Both \lig and \hug rebel, and a three-sided war begins between \FRA
  and the two Rebels.
  % \bparag The initial repartition of French forces is: \FRA has \ARMY
  % \faceplus, \lig has \ARMY \facemoins and \hug has \DT.
  % \bparag If the naval forces desert, decide at random if it is to
  % join the \hug or the \lig.
  \bparag \leaderNavarre is a possible Heir but is hesitant.  He is used as a
  general by \FRA, excepted if \hug controls or besieges \ville{Paris}.  He
  will go the side of \FRA as soon as he is chosen as Heir at the death of
  \monarque{Henri III}, or could go back to the Protestant side if
  \monarque{Henri de Navarre} is the chosen Heir, or if a Protestant Coup is
  made.
  \aparag Notice that as soon as \monarque{Henri III} die, one of the minor
  (the one having the chosen heir) will sign peace with \FRA and be
  immediately annexed.
  % Leader \leaderNavarre is a possible Heir but is hesitant.  Neither
  % the \hug nor \FRA can use it. He will go the side of \FRA as soon as
  % he is chosen as Heir at the death of \monarque{Henri III}, or could
  % go back to the Protestant side if \monarque{Henri de Navarre} is the
  % chosen Heir, or if a Protestant Coup is made.
  \aparag[If the king is the Heir,] (brand-new catholic \monarque{Henri IV}).
  \bparag \lig rebels, following the general rules.
  \aparag If \leader{Henri de Guise} is dead, \lig receive the general
  \leaderMayenne (B.2.2.2).
  \aparag If \leader{Henri de Guise} is alive, \lig will have a bonus of {\bf
    +2} to its reinforcement roll.
  \aparag \hug is immediately annexed by \FRA.
\end{digressions}

\phdipl
\aparag Foreign intervention are allowed.

\phadm
\aparag \FRA gets full income of all non-revolted, controlled provinces,
including those belonging to a revolted rebel or in the \ROTW.
\aparag As soon as the last Valois dies, \FRA is no more restricted in
administrative actions.
\aparag[Reinforcements of Rebels]
\bparag If \LIG spends 50\ducats, the \lig will have a bonus of \bonus{+1} to
their reinforcement roll.
\bparag If \HUG spends 50\ducats, the \hug will have a bonus of \bonus{+1} to
their reinforcement roll.

\tour{Turn 2 and afterwards}

\phevnt
\aparag Except for what follows, use the same rules as turn 1.
\aparag If the French Monarch \monarque{Henri III} died at the beginning of
some turn, \FRA has to choose its Heir (if no Coup has imposed an Heir). Apply
the effect of \ref{pIII:FWR:Designation Heir}, and then the effect of the
(possibly new) Religious attitude that follows. The revolted side receives new
\REVOLT according to \ref{pIII:FWR:Uprisings}.
\aparag Else, if a Coup was successful, apply \ref{pIII:FWR:Uprisings} to roll
for new \REVOLT of the now rebel side. The war resumes with rebels depending
on the new religious attitude.
\aparag If a pretender was murdered on the previous turn, new \REVOLT are
rolled for according to \ref{pIII:FWR:Uprisings} for this side only.
\aparag \paysSavoie will make a limited intervention as an ally of \lig (or
\FRA if \CATHCR), with an \ARMY \faceplus and one unnamed minor general.

\phadm
\aparag[Reinforcements of Rebels]
\bparag Reinforcements will be received for the rebel side(s) according to
\ref{pIII:FWR:Military Troubles} but the initial repartition of forces is not
made anew (it has already been done).
\bparag If \LIG spends 50\ducats, the \lig will have a bonus of \bonus{+1} to
their reinforcement roll.
\bparag If \HUG spends 50\ducats, the \hug will have a bonus of \bonus{+1} to
their reinforcement roll.

\begin{digressions}[Specific conditions of the War of Succession]


  \digression[pIII:FWR:War4 Military]{Military operations during the War of
    Succession}

  \phmil
  \aparag Use the general rules of \ref{pIII:FWR:Military Operations}.
  \aparag \FRA and its allies have a bonus of \bonus{+3} to suppress \REVOLT
  in France.
  \aparag \paysPalatinat makes an intervention as an ally of the side of
  \leaderNavarre or \monarque{Henri de Navarre} with \ARMY\faceplus, \LD and a
  random general.  If the Monarch is \monarque{Henri III} with \monarque{Henri
    IV} as the chosen Heir, \paysPalatinat makes no intervention.
  \aparag \FRA draws supply from any province in France (including those of
  \lig and \hug), except those in \REVOLT
  \aparag \lig and \hug draw supply only from the provinces they control.
  \aparag[Voluntary surrender]
  \bparag A city besieged by \FRA with at least one \ARMY \faceplus,
  voluntarily surrenders if there was no \REVOLT \faceplus in it at the
  beginning of the turn, nor is it a Place of Safety and there is no more
  \REVOLT in the province (including if the \REVOLT was just crushed this
  round).
  \bparag A city besieged by \lig with at least one \ARMY \faceplus,
  voluntarily surrenders if it is in the territory owned by \lig.
  \bparag A city besieged by \hug with at least one \ARMY \faceplus,
  voluntarily surrenders if it is in the territory owned by \hug.


  \digression[pIII:FWR:War4 Peace]{How to end the War of Succession?}

  \phpaix
  \aparag If there are only 2 sides in this war, the War of Succession ends if
  \FRA control Paris and wins a Major Victory over Rebel forces (at least 3
  \DT of Rebels) or if all Rebel forces and \REVOLT have been eliminated.
  \bparag \FRA has to spend 100\ducats to stop the war; no Coup or
  Assassination can happen. Apply \ref{pIII:FWR:End of the War of Succession}.
  \aparag If there are only 2 sides in this war, the War of Succession ends if
  \FRA has no land forces left and the Rebel controls the city of Paris. A
  Coup in favour of the Rebels is automatically made with no possible murder
  attempt by \FRA. A mandatory change of Religious attitude is imposed on \FRA
  and the new Monarch is the Heir of the winning side. Apply \ref{pIII:FWR:End
    of the War of Succession}.
  \aparag \FRA ends as barely victorious if this is the end of the first turn
  of period IV (then no Coup is permitted). Apply now \ref{pIII:FWR:End of the
    War of Succession} and \ref{pIII:FWR Final}.
  \aparag If \lig is in rebellion, controls the city of Paris, and
  \leader{Henri de Guise} is alive, then \LIG can spend 100\ducats for an
  attempt of Counter-Reformation Coup.
  \aparag If \hug is in rebellion, controls the city of Paris, and
  \leaderNavarre is alive, then \HUG can spend 100\ducats for an attempt of
  Protestant Coup.
  \aparag If a Coup is attempted, \FRA can try to murder the pretender
  (\leader{Henri de Guise} or \leaderNavarre).
  \aparag If no Coup is attempted, \FRA can try to murder one pretender of
  revolted \lig or \hug (\leader{Henri de Guise} or \leaderNavarre).
  \aparag \FRA loses no \STAB because of the \REVOLT but loses \STAB as in an
  usual war (1 the first turn, 2 the second, \dots)
  \aparag The war keeps on until one side is victorious; there is no Truce.


  \digression[pIII:FWR:Coup Murder Pretender]{Coup and Murder of the
    Pretender}

  \phpaix
  \aparag The side attempting the Coup (\LIG or \HUG) has to spend 100\ducats
  then rolls 1d10 and adds \bonus{+2} if \FRA is \CATHCO; \bonus{+2} if the
  \ref{pIII:FWR:Saint-Barthelemy} was not perpetrated; \bonus{+3} if the
  \ref{pIII:FWR:Saint-Barthelemy} was made against the religious faction of
  the coup's side; \bonus{+2} if \FRA makes no Murder attempt; \bonus{+1} per
  victory of the pretender's minor country with at least one \ARMY \faceplus.
  \aparag[Failure of the Coup] If the result of the Rebels is 9 or lower, the
  Coup is failed. It may succeed if the result is 10 or higher.
  \aparag If \FRA attempts to murder the pretender, it rolls 1d10, and add
  \bonus{+2} for each point of \STAB that it spends (it has to have those
  points); and \bonus{+3} is no Coup attempt was made.
  \aparag[Result of Assassination] If the result of \FRA is 9 or lower, the
  murder is failed. It may succeed if the roll is 10 or higher.  \FRA loses
  {\bf 1} \STAB, and the Valois \monarque{Henri III} will have
  an additional permanent malus of \bonus{+3} to its Survival Test until his death. \\
  \centerline{\textit{"Il est encore plus grand mort que vivant."}}
  \aparag[If both a Coup and a Murder succeed]
  \bparag If the result of \FRA is higher of equals to the result of the Coup,
  the Coup actually fails; the Pretender is murdered.
  \bparag Else (if the result of Rebels is higher than the result of \FRA),
  the Coup succeeds. \FRA makes a mandatory change of Religious attitude and
  of designated Heir. The pretender is not killed (miraculously saved!) and
  becomes the new Heir.
  \aparag[Successful Coup]
  \bparag The new mandatory Heir is the one (\monarque{Henri de Guise} or
  \monarque{Henri de Navarre}) of the side doing the Coup and the Religious
  attitude of \FRA is changed according to this new Heir.
  \bparag When \monarque{Henri III} dies, the Heir is crowned as the French
  King.
  \bparag If this case, on the next turn, a Civil War with the new sides
  depending of the new Religious attitude continues, or begins if the Coup was
  during event (3).
  \aparag In addition, \FRA has a mandatory defensive alliance with the
  controller of the side having done the Coup, and this power can now make
  full intervention in the war until the end of \nameref{pIII:FWR}.


  \digression[pIII:FWR:End of the War of Succession]{End of the War of
    Succession}

  \phinter
  \aparag \STAB of \FRA is raised by {\bf 2}.
  \aparag The new Monarch is the last designated Heir (\monarque{Henri III} is
  pushed aside if he is still alive...)
  \aparag All \REVOLT and forces of minor countries \hug and \lig are
  removed. But they continue to exist (they can rebel one more time if \FRA is
  not \CATHCO).
  \aparag[Intervention of Foreign countries]
  \bparag Minor countries having forces left in \FRA propose an immediate
  white peace to \FRA. If it is accepted, they withdraw and are at peace with
  \FRA. Else, they are now in a regular war with \FRA (but no one is victim of
  a declaration of war).
  \bparag Any Major power having forces left in \FRA has to sign a white
  peace, or are from now on in regular war with \FRA.  Their military activity
  is no more limited; nobody is victim of a declaration of war (but \FRA and
  its enemies are at war), and regular call to allies will be possible on the
  next turn.  This war causes normal loss of \STAB, beginning with a loss of
  {\bf 1} \STAB this turn.
  \bparag The only specificity of this war is that, if a unconditional peace
  is forced on \FRA, the winning power must change the Monarch of \FRA to the
  Heir of its Religious Attitude.  In this case this is the only condition of
  the peace, and \FRA has a mandatory defensive alliance with the winners
  during the reign of the new Monarch.
  \aparag As soon as \FRA is at peace at an end-of-turn and \CATHCO,
  \ref{pIII:FWR Final} is applied.
\end{digressions}



\event{pIII:FWR Last Stand}{III-D (5)}{Last Stand of the Heretics}{1}{PB}

\history{Alternate history}

\condition{}
\aparag If \ref{pIII:FWR Succession} is not finished, do not mark off and
reroll.
\aparag If \FRA is \CATHCO and no unconditional surrender was obtained by \FRA
against \hug in a previous war, mark off the event, play \RD instead and the
French king will have a malus of \bonus{+2} to his Survival test for the next
turn.
\aparag If \FRA is \CATHCO but did force an unconditional surrender of \hug in
a previous war, \hug rebels itself.
\aparag If \FRA is Protestant or \CATHCR at the end of \ref{pIII:FWR
  Succession} and \ref{pIII:FWR Final} was not applied, the rest of the event
happens.
\aparag If \FRA is Protestant or \CATHCR but \ref{pIII:FWR Final} already
occurred, play \RD instead with the \REVOLT on the table of \FRA.

\phevnt
\aparag One of \lig or \hug rebels itself depending on the religion of
\FRA. Apply the full effects of \ref{pIII:FWR:Politic Crisis},
\ref{pIII:FWR:Economic Crisis}, \ref{pIII:FWR:Uprisings} and
\ref{pIII:FWR:Military Troubles}.  Also apply \ref{pIII:FWR:War5 Military} and
\ref{pIII:FWR:War5 Peace}.
\aparag If the revolting minor was already annexed by \FRA (this may happen if
a mandatory religious change is then forced on \FRA), recreate it
immediately. It will get no troops at beginning.
\aparag If the non-rebelling minor still exists, it is immediately annexed by
\FRA: its provinces become regular French provinces and its units become
french units.
\aparag \REB is not obliged to do a white peace with \FRA.
\bparag If it chooses to continue a war, it can make a full military
intervention in the Civil War. But it will continue to suffer a normal loss of
\STAB at the end of turns, whereas \FRA will lose at most {\bf 2} \STAB each
turn during the Civil War.
\bparag If it chooses to sign a white peace, or if it was at peace, \REB can
make a limited intervention in the war.
\aparag \LIG can make a limited intervention as an ally of a \CATHCR\ \FRA.
\aparag \HOL can make a limited intervention as an ally of Protestant
\FRA. Else it can make a limited intervention as an ally of \hug.

\phdipl
\aparag Usual foreign interventions are allowed.

\begin{digressions}[Specific conditions of the last event]


  \digression[pIII:FWR:War5 Military]{Military operations during the fifth
    event}

  \phmil
  \aparag Use the rules of \ref{pIII:FWR:Military Operations}.
  \aparag \FRA and its allies have a bonus of \bonus{+4} to suppress \REVOLT
  in France.
  \aparag A city in \FRA that had not a \REVOLT \faceplus at the beginning of
  the turn, makes an immediate voluntary surrender if besieged by a land stack
  of \FRA (or its allies) that sets a siege with at least one \ARMY \faceplus
  and there is no more \REVOLT in the province (including if the \REVOLT was
  just crushed this round).


  \digression[pIII:FWR:War5 Peace]{How to end the Last Stand?}

  \phpaix
  \aparag \FRA loses no \STAB because of the \REVOLT . It loses at most {\bf
    2} \STAB per turn because of the war.
  \aparag No Truce happens ever in this civil war. It keeps going until one
  side wins.
  \aparag The War ends if \FRA controls Paris and wins a Major Victory over
  Rebel forces (at least 3 \DT of Rebels) or if all Rebel forces and \REVOLT
  have been eliminated.
  \aparag The War ends if \FRA has no land forces left and the Rebel controls
  the city of Paris. An change of Heir in favour of the Rebels is
  automatically made (with no possible murder attempt by \FRA) that causes a
  mandatory change of Religious attitude. The new Monarch of \FRA is the Heir
  of the winning side.
  \aparag \FRA ends as barely victorious if the last turn of period III has
  ended (now or previously).


  \digression[pIII:FWR:End of the Last Stand]{End of the Last Stand}

  \phpaix
  \aparag \STAB of \FRA is raised by {\bf 2}.
  \bparag The new Monarch is the last designated Heir (if it did change; the
  former one is pushed aside)
  \bparag All \REVOLT %and forces of minor countries \hug and \lig
  are removed.
  \bparag Minor countries having forces left in \FRA propose an immediate
  white peace to \FRA. If it is accepted, they withdraw and are at peace with
  \FRA. Else, they are now in a regular war with \FRA (but no one is victim of
  a declaration of war).
  % \bparag Minor countries having forces left in \FRA withdraw and are at
  % peace with \FRA.
  \bparag Any Major power having forces left in \FRA has to sign a white
  peace, or are from now on in regular war with \FRA.  Their military activity
  is no more limited; nobody is victim of a declaration of war (but \FRA and
  its enemies are at war), and regular call to allies will be possible on the
  next turn.  This war causes normal loss of \STAB, beginning with a loss of
  {\bf 2} \STAB this turn for everyone.  The Sole Defender of the Catholic
  Faith could impose a change of Religion, but by normal rules and not by
  specific rules of this event.
  \bparag When this War ends, apply \ref{pIII:FWR Final}.
\end{digressions}



\event{pIII:FWR Final}{III-D (Final)}{End of the Wars of Religion}{1}{PB}

\activation{}
\aparag This event is applied when the fifth event \xref{pIII:FWR} is at last
resolved.
\aparag This event is applied also as soon as \FRA is at peace and \CATHCO
after the end of the fourth event.
\aparag At the end of the last turn of the period III (or the first turn of
period IV if \ref{pIII:FWR Succession} is happening), this event is applied
regardless of other conditions.

\phinter
\aparag The Wars of Religion are ended. Further events \numberref{pIII:FWR}
cause \RD.
\aparag The Monarch should be the designated Heir, or the Heir is crowned
right now.
\aparag Minor countries \hug and \lig are immediately annexed by \FRA. All
their provinces are now regular provinces of \FRA. All their land forces
become french land forces. \FRA gets back its regular counter limit and
maintenance.  The navy is given back to \FRA. If alive, \leaderConde,
\leaderColigny, \leaderMayenne, \leaderNavarre and \leader{Henri de Guise}
retire (excepted the now Monarch); all other french leaders are now regular
french leaders.
\aparag If the king is \monarque{Henri de Guise} or \monarque{Henri IV}, \FRA
gains a free maintenance of one \ARMY \faceplus until the end of his
reign. This is not the case if the Monarch is \monarque{Henri de Navarre}.
\aparag[Victory Points]
\ENG, \HOL and \SPA win each 25 \PV if they have been allied at least once to
the side of the Heir that won finally the wars. They lose 25 \PV if they have
fought against this winning side.
\aparag[Economic consequences] Roll 1d10 and add \bonus{+1} for each
favourable truce conceded to the rebels, \bonus{+1} if \FRA has been complied
to change its Religious attitude, and \bonus{+1} is \FRA is \CATHCR.
\bparag Result 1-3: 1 level of French \TradeFLEET is lost to \HOL;
\bparag Result 4-5: 1 level of French \TradeFLEET is lost to \HOL, and 1 to
\ENG;
\bparag Result 6-10: 2 levels of French \TradeFLEET are lost to \HOL, and 1 to
\ENG; the \FTI of \FRA is diminished by \bonus{-1};
\bparag Result 11+: 2 levels level of French \TradeFLEET are lost to \HOL, and
2 to \ENG; both \FTI and \DTI of \FRA are diminished by \bonus{-1};
\bparag \HOL chooses first the \TradeFLEET it takes, then \ENG chooses.
\bparag If \FRA is \CATHCR, the level chosen by \HOL are lost but not received
by \HOL; \ENG gains the levels if it is Catholic, if not those levels are lost
for everyone.
\bparag if \ENG is \CATHCR and \FRA is not, \SUE chooses and gains the levels
instead of \ENG.
% \aparag Roll 1d10. Add \bonus{+2} if \CATHCR, \bonus{-2} if Protestant. On a
% result of 6+, one french \COL or \TP (decide at random between all those
% that exist) is lost.
\aparag[Undesired policy]
\bparag If the chosen Heir was Protestant but \FRA is no more Protestant at
the end of the Wars of Religion, \FRA has a malus of \bonus{-2} to all its
colonial actions during the period IV and its \FTI and \DTI are diminished by
a further 1.
\bparag If the chosen Heir was \CATHCO but \FRA is \CATHCR at the end of the
Wars of Religion, \FRA has a malus of \bonus{-1} to all its colonial actions
during the period IV.  Each event \RD obtained in period IV has a chance to
make appear a second \REVOLT \faceplus in \FRA. Roll 1d10: 1-3
\provincePoitou, 4-6: \provinceGuyenne, 7-10: none.
\bparag If the chosen Heir was \CATHCR but \FRA is no more \CATHCR, \FRA has a
malus of \bonus{-2} to all its Technological actions during the period IV.
Each event \RD obtained in period IV has a chance to make appear a second
\REVOLT \faceplus in \FRA. Roll 1d10: 1-3 \provinceArmor, 4-6:
\provinceOrleanais, 7-10: none.

% Local Variables:
% fill-column: 78
% coding: utf-8-unix
% mode-require-final-newline: t
% mode: flyspell
% ispell-local-dictionary: "british"
% End:

% LocalWords: pIII FWR PBNew malus Ligue Ile de unbesieged l'Hospital pII
% LocalWords: Barthelemy Casimir Schmalkaldic Mayenne Conciliant reroll Jym
% LocalWords: TODO


\stopevents

% Local Variables:
% fill-column: 78
% coding: utf-8-unix
% mode-require-final-newline: t
% mode: flyspell
% ispell-local-dictionary: "british"
% End:

% LocalWords: pII pIV pIII VOC Oxenstierna FWR Barthelemy Lublin Oprichnina
% LocalWords: Safavids Mughal RistoMod pI hollande Auld Stadhouder Compagnie
% LocalWords: Vereenigde Oostindische PBNew TYW Risto EIC POR Teutoniques de
% LocalWords: Kurland Duche Mancha malus interphase WPT Jym Prusse Zygmunt
% LocalWords: Vasa Moghol Mughals Petersbourg Petersburg CCA Formose URP JCD
%  LocalWords:  Moriscos
