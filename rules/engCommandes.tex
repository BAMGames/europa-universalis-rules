% Structure generale :
% \chapter{Titre}		I	saut page
% \section{Titre}		I.1	parfois saut page
% \subsection{Titre}		I.1.1	titre separe	
% \subsubsection[Titre]		I.1.1.1	titre sur la ligne
% \aparag[Titre]		A.	nouveau parag si titre,meme par sinon
% \bparag			A.1	

% Sauf pour les evenements :
% \chapter{Evenements}
% \section*{Foreword}
% \section{Periode I}			nouvelle page
% \subsection{Generalites}
%	dont le resume des events
% \phasesection{Titre}
% \ephase[Titre]			A.	nouveau paragraphe
% \fphase				A.1	nouveau par si titre avant.

% Commands for events:
% \tour{Text} produces the yellowish-gradient thingie
% \phasesection{Something} (also \dphase) produces the correct numbering of
%        phases, plus the text. Some commands are predefined with bold text
%        behind the phase number. They are, e.g.
% \condition \activation \consequences (those three commands admit an optional
%        argument that will be written on the same line than the text)
%        Note that this is really not necessary, since after any command
%        paragraph is not finished. However, font could be special.
% \phevnt \phdipl \phadm \phmil \phmvt \phpaix \phinter \effetlong
%        are others (no opt. argument)
% \history[details]{date} & \dure{Duration} simply inserts some italic
%        not-numbered text.
% \defsubevent{Something} resets the phase counter (and all subhierarchical
%        ones, puts a \bfintercale{Something}. 1st style subevent.
%        Changes \thephase to {\thesubevent\thephase}
% \begin{setsevent} \  Defines a set of subevents, second style. If used as 
% \end{setsevent}   /  as \begin{setother}[Something], puts a \bfintercale.
% \defsubeventbis{Something} Defines a subevent. PHASE is kept, where as
%        EVPARA and SUBEVPARA are saved, reset, and restored at end.
%        Changes \thephase to {\thesubevent\thephase}
% \ephase is the paragraph level numbering. Uses EVPARA counter.
% \fphase is the subparagraph level numbering. Uses EVSUBPARA counter.

% \subeventlabel \eventlabel pour marquer les noms des evenements
% \rulelabel pour marquer les noms des regles
% II-1 (Ghibli) => eventref
% Ghibli => eventrefname
% II-1 => eventrefshort
% a (Totoro) of event Ghibli => subeventrefbyname
% a (Totoro) of event II-1 (Ghibli) => subeventref
% a (Totoro) of event II-1 => subeventrefshort
% a (Totoro) => subeventreflocal
% rule I.1.1 (Titre) => ruleref
% A.1 of rule I.1.1 => ruleref
% A.1 => localruleref

%\oldref, \tableref, \figref

% Serie de commandes avec mise en forme et traduction eventuelle automatique
% Villes, provinces, Noms de pays majeurs (ne devrait pas etre trop utilise)
% Zone de mer, Grande region (ROTW), Sous-Continent (ROTW), Leader, Monarque
% nomme, Minister nomme, Region (Europe, ensemble de provinces)
% \ville \province \pays \paysmajeur \seazone \granderegion \continent
% \leader \monarque \ministre \region

% One can always use for these commands (example with ville):
% \shortville (short form of name)
% \longville (long name)
% \longlongville (definition of the term, bold in index)
% \sectionville (Used in section titles: the name appears in bookmarks)
% Only ville and sectionville support aliases (others have to use
% the main name)
% Please remark that aliases should disappear

%\leaderdef(Cle)[Tours](pays)(Type)(Caracs){Commentaires}
%\leaderdefdouble(Cle)[Tours](pays)(Type)(Caracs)(Type2)(Caracs2){Commentaires}
%\leaderdefother(Cle)[Tours](pays)(Type)(Caracs)(nom2)(pays2){Commentaires}
%\leaderdefothermajeur(Cle)[Tours](pays)(Type)(Caracs)(nom2)(liste)(Pays2){Commentaires}
% Pour un majeur, liste est une variation sur le nom du pays
%\leaderdef(Cle)[Tours](liste)(Type)(Caracs){Commentaires}

%Tableau grise: 
%{\graytabular \begin{tabular}...\endtabular}

%Graphiques
%\faceplus \facemoins \undemi \ducats \textetoile \anonyme

%Environnements:
%note, designnote, histoire, exemple, tablehere, figurehere
%listnbre, enumeration, figure, table, colonnes

%% Abreviations en vrac
\TUabbrevdefs{ADM}
\TUabbrevdefs{MIL}
\TUabbrevdefs{DIP}
\TUabbrevdefs{CB}
\TUabbrevdefs[OCB]{Overseas CB}
\TUabbrevdefs[RD]{R/D}
\TUabbrevdefs{VP}
\TUabbrevdefs{VPs} % Victory points (plural)
\TUabbrevdefs{ROTW}
\TUabbrevdefs{CC}
\newcommand{\CCs}[1][\null]{\CC~\textit{#1}\xspace}
\TUabbrevdefs{MNU}
\TUabbrevdefs{Neutral}
\TUabbrevdefs{RM}
\TUabbrevdefs{SUB}
\TUabbrevdefs{MA}
\TUabbrevdefs{EC}
\TUabbrevdefs{DC}
\TUabbrevdefs{EW}
\let\MR\RM
\let\CE\EC
\let\AM\MA
\let\EG\EW
\TUabbrevdefs[VASSAL]{VA}
\TUabbrevdefs[ANNEXION]{AN}
\TUabbrevdefs{MP}
\TUabbrevdefs{LoS}
\TUabbrevdef[ARMY]{A}{\bbold{A}}
\TUabbrevdef[FLEET]{F}{\bbold{F}}
\TUabbrevdef[corsaire]{P}{\bbold{P}}
\TUabbrevdef[GD]{D}{\bbold{D}}     % Generic Detachment
\TUabbrevdef{LDE}{L\bbold{D}E}   % Land Detachment of Exploration
\TUabbrevdef{LD}{L\bbold{D}}     % Land Detachment
\TUabbrevdef{ND}{N\bbold{D}}     % Naval Detachment (any kind)
\TUabbrevdef{NDE}{N\bbold{D}E}   % Naval Detachment of Exploration(warships)
\TUabbrevdef{NWD}{NW\bbold{D}}   % Naval Warships Detachment
\TUabbrevdef{NGD}{NG\bbold{D}}   % Naval Galley Detachment
\TUabbrevdef{NTD}{NT\bbold{D}}   % Naval Transport Detachment
\TUabbrevdef[VGD]{Venetian GD}{Venetian G\bbold{D}} % Venetian Galeasses Detachments
\TUabbrevdef[LDND]{LD/ND}{L\bbold{D}/N\bbold{D}}
\TUabbrevdef[LDENDE]{LDE/NDE}{L\bbold{D}E/N\bbold{D}E}

\TUabbrevdef[PO]{PO}{\RES{PO}}
\TUabbrevdef[PA]{PA}{\RES{PA}}

\TUabbrevdefs[TradeFLEET]{TF}
\TUabbrevdefs{TFI}
\TUabbrevdefs{TP}
\TUabbrevdefs{COL}
\TUabbrevdefs{REB}
\TUabbrevdefs{HUG}
\TUabbrevdefs{LIG}
\TUabbrevdefs{STZ}
\TUabbrevdefs{CTZ}
\TUabbrevdefs{DTI}
\TUabbrevdefs{FTI}
\TUabbrevdefs[dipNR]{NR}
\TUabbrevdefs[dipFR]{FR}
\TUabbrevdefs[dipAT]{AT}
\TUabbrevdefs{MAJ}
\TUabbrevdefs{MIN}
\TUabbrevdefs{RT}
\TUabbrevdef{EcoRS}{\emph{ERS}}
%Nouveaux acronymes a ajouter ici
\TUabbrevdef[fortress]{f}{\bbold{f}}

\TUabbrevdef[LeaderA]{A}{{\fontencoding{U}\fontfamily{eur}\fontseries{m}\selectfont{A}}}
\TUabbrevdef[LeaderC]{C}{{\fontencoding{U}\fontfamily{eur}\fontseries{m}\selectfont{C}}}
\TUabbrevdef[LeaderG]{G}{{\fontencoding{U}\fontfamily{eur}\fontseries{m}\selectfont{G}}}
\TUabbrevdef[LeaderE]{E}{{\fontencoding{U}\fontfamily{eur}\fontseries{m}\selectfont{E}}}
\TUabbrevdef[LeaderMis]{Mis}{{\fontencoding{U}\fontfamily{eur}\fontseries{m}\selectfont{M\kern -0.1em i\kern -0.1em s}}}
\TUabbrevdef[LeaderGov]{Gov}{{\fontencoding{U}\fontfamily{eur}\fontseries{m}\selectfont{G\kern -0.1em o\kern -0.1em v}}}
\TUabbrevdef[Leaderd]{D}{{\fontencoding{U}\fontfamily{eur}\fontseries{m}\selectfont{D}}}
\TUabbrevdef[LeaderI]{Eng}{{\fontencoding{U}\fontfamily{eur}\fontseries{m}\selectfont{E\fontfamily{eur}\selectfont\kern -0.1em n\kern -0.1em g}}}
\TUabbrevdef[LeaderK]{K}{{\fontencoding{U}\fontfamily{eur}\fontseries{m}\selectfont{K}}}
\TUabbrevdef[LeaderP]{P}{{\fontencoding{U}\fontfamily{eur}\fontseries{m}\selectfont{P}}}
\let\Leaderg\LeaderGov

%Abreviations simples
\def\Man{Manoeuvre\xspace}
\newcommand{\Timar}{\terme{T{\i}marlar}\xspace}
\newcommand{\Janissaire}{\terme{Yeni\c{c}eriler}\xspace}
\newcommand{\Pasha}{\terme{Pasha}\xspace}
\newcommand{\Pashas}{\terme{Pashas}\xspace}
\newcommand{\Presidios}{\terme{Pr\ae sidios}\xspace}
\newcommand{\Presidio}{\terme{Pr\ae sidio}\xspace}
\newcommand{\StraitFort}{\terme{Strait fortifications}\xspace}
\newcommand{\Area}{\terme{Area}\xspace}
\newcommand{\STAB}{Stability\xspace}
\newcommand{\CONC}{Concurrency\xspace}
\newcommand{\TPaction}{\TP placement\xspace}
\newcommand{\COLaction}{\COL placement\xspace}
\newcommand{\hug}{\pays{Huguenots}\xspace}
\newcommand{\lig}{\pays{Ligue}\xspace}
\newcommand{\parl}{\pays{parliament}\xspace}	% for English Civil War
\newcommand{\royal}{\pays{Royalists}\xspace}		% for English Civil War
\newcommand{\alliance}{\textsc{Alliance}\xspace}
\newcommand{\ligue}{\textsc{League}\xspace}
\newcommand{\stz}{\STZ~\seazone}
\newcommand{\ctz}{\CTZ~\paysmajeur}
\newcommand{\POSPICE}{\RES{PO} or \RES{Spices}\xspace}

% \lignebudget#1 \intercale#1 \bfintercale#1
% \CAI, \CAII, \CAIII, \CAIV, \CAIM, \CAIIM, \CAIIIM, \CAIVM, \CAA



% Commandes semantiques restant a traiter
\newcommand{\terme}[1]{\emph{#1}} % terme de regle
%Provisoires :
\newcommand{\oldref}[1]{\textit{#1}}
\newcommand{\newpb}{\textit{new ! }}
\newcommand{\prop}[1]{\texttt{#1}}
\newcommand{\future}[1]{\textit{#1}}

%Anciennes commandes a supprimer

\let\DT\LD
\let\DN\ND
\let\de\NDE
\let\PV\VP
\def\LeaderR{\blop}
\def\Leaderc{\vlop}

\let\rulelabel\label

\newcommand{\period}[1]{#1\xspace}

\newcommand{\technologie}[1]{\terme{#1}}

% Macros plus specifiques a Europa
\def\ducatsx{%
  D\llap{\raise 0.1em\hbox{|\kern -.05em|\hskip .1em}}%
}
\def\faceplusx{%
  \tikz[baseline=-.85ex,line width=.2ex,black]{%
    \draw[fill=black!10] (0,0) circle (1ex);%
    \draw (-.55ex,0)--(.55ex,0) (0,-.55ex)--(0,.55ex);}}
\def\facemoinsx{%
  \tikz[baseline=-.85ex,line width=.2ex,black]{%
    \draw[fill=black!10] (0,0) circle (1ex);%
    \draw (-.55ex,0)--(.55ex,0);}}
\def\anonymex{%
  \tikz[baseline=(X.base),line width=.2ex,black]{%
    \draw[fill=black!10] (0,0) circle (1.2ex);%
    \node(X)[anchor=center] {?};}}

\TUabbrevdef[Ducats]{D\textdollar}{\hspace*{.2em}\ducatsx}
\TUabbrevdef[Faceplus]{(+)}{\faceplusx}
\TUabbrevdef[Facemoins]{(-)}{\facemoinsx}
\TUabbrevdef[anonyme]{(?)}{\anonymex}
\TUabbrevdef[REVOLT]{\{Revolt\}}{\countermark{Revolt}}
\TUabbrevdef[PIRATE]{\{Pirate\}}{\countermark{Pirate}}
\TUabbrevdef[PILLAGE]{\{Pillage\}}{\countermark{Pillage}}

\def\xducats{\Ducatsnsp}
\def\ducats{\Ducatsusp}
\def\faceplus{\Faceplususp}
\def\facemoins{\Facemoinsusp}

\def\textetoilex{\includegraphics[height=1ex]{etoile}}
\def\textetoile{\textetoilex\xspace}
\def\textnumber{\#}
\def\f{\textonehalf}
\def\tu{\textonethird}
\def\td{\texttwothird}

\def\TTER{\technologie{Tercios}\xspace}
\def\TMED{\technologie{Medieval}\xspace}
\def\TREN{\technologie{Renaissance}\xspace}
\def\TARQ{\technologie{Arquebus}\xspace}
\def\TMUS{\technologie{Muskets}\xspace}
\def\TBAR{\technologie{Baroque}\xspace}
\def\TMAN{\technologie{Manoeuvre}\xspace}
\def\TL{\technologie{Lace}\xspace}

\def\TVGA{\technologie{Galleasses}\xspace}
\def\TGA{\technologie{Galley}\xspace}
\def\TCAR{\technologie{Carrack}\xspace}
\def\TGLN{\technologie{Nao-Galeon}\xspace}
\def\TGF{\technologie{Galleon-Fluyt}\xspace}
\def\TLS{\TGF} % Latin Sail. TODO: track and replace.
\def\TBAT{\technologie{Battery}\xspace}
\def\TVE{\technologie{Vessel}\xspace}
\def\TTD{\technologie{Three-decker}\xspace}
\def\TSF{\technologie{74's guns}\xspace}

% Jym, aout 2010
% Symbole de religion pour inclure dans les regles et pas seulement dans les
% annexes ou il y a des workaround specifiques.
% Marche pas ???
\def\techlatin{\includegraphics[height=0.8\baselineskip]{catholique.pdf}}
\def\techortho{\includegraphics[height=0.8\baselineskip]{orthodoxe.pdf}}
\def\techrotw{\includegraphics[height=0.8\baselineskip]{autrereligion.pdf}}
% \baselineskip dans un tabular compile mais cause une erreur enregardant le
% pdf (???)
% Dans un tabular, le graphique est trop haut par rapport a la ligne (???) le
% \smash ne corrige pas mais evite au moins de decaler le reste de la ligne...
\def\techrotwtab{\smash{\includegraphics[height=4mm]{autrereligion.pdf}}}
\def\techislam{\includegraphics[height=0.8\baselineskip]{sunnite.pdf}}

% Major countries and surronding commands
% Now made as shortcuts of paysmajeur
\TUdefshortcut{paysmajeur}{Portugal}{POR}{POR}
\TUdefshortcut{paysmajeur}{Espagne}{HIS}{HIS}
\TUdefshortcut{paysmajeur}{Hollande}{HOL}{HOL}
\TUdefshortcut{paysmajeur}{France}{FRA}{FRA}
\TUdefshortcut{paysmajeur}{Angleterre}{ANG}{ANG}
\TUdefshortcut{paysmajeur}{Turquie}{TUR}{TUR}
\TUdefshortcut{paysmajeur}{Russie}{RUS}{RUS}
\TUdefshortcut{paysmajeur}{Pologne}{POL}{POL}
\TUdefshortcut{paysmajeur}{Prusse}{PRU}{PRU}
\TUdefshortcut{paysmajeur}{Venise}{VEN}{VEN}
\TUdefshortcut{paysmajeur}{Suede}{SUE}{SUE}
\TUdefshortcut{paysmajeur}{Autriche}{AUS}{AUS}
\TUdefshortcut{paysmajeur}{Danemark}{DAN}{DAN}

% Plan: PORpor VENven HOLhol POLpol PRUbra (=PRUpru) AUShab (=AUSaus = HAB)
% => displayed as POR* VEN* HOL* POL* PRU* AUS* and are either MAJ or MIN
% PORMin VENLin ...
% => displayed as Minor Portugallia Minor Venetia ... and are MIN
% PORmin VENmin ...
% => displayed as minor Portugallia minor Venetia ... and are MIN
% POR VEN ...
% => displayed as POR VEN ... and are MAJ
% MAJHAB
% => displayed as [AUS/SPA] and designates the controller of \pays{autriche} 
% hab = AUSmin, HABmin=AUSmin, HABMin=AUSMin

\newcommand{\PORmin}{minor \pays{portugal}\xspace}
\newcommand{\PORMin}{Minor \pays{portugal}\xspace}
\TUdefshortcut{pays}{portugal}{*}{minporsc}
\newcommand{\PORpor}{{\POR}\minporsc}

\newcommand{\VENmin}{minor \pays{venise}\xspace}
\newcommand{\VENMin}{Minor \pays{venise}\xspace}
\TUdefshortcut{pays}{venise}{*}{minvensc}
\newcommand{\VENven}{{\VEN}\minvensc}

\newcommand{\HOLmin}{minor \pays{hollande}\xspace}
\newcommand{\HOLMin}{Minor \pays{hollande}\xspace}
\TUdefshortcut{pays}{hollande}{*}{minholsc}
\newcommand{\HOLhol}{{\HOL}\minholsc}

\newcommand{\POLmin}{minor \pays{pologne}\xspace}
\newcommand{\POLMin}{Minor \pays{pologne}\xspace}
\TUdefshortcut{pays}{pologne}{*}{minpolsc}
\newcommand{\POLpol}{{\POL}\minpolsc}

\newcommand{\SUEmin}{minor \pays{suede}\xspace}
\newcommand{\SUEMin}{Minor \pays{suede}\xspace}
\TUdefshortcut{pays}{suede}{*}{minsuesc}
\newcommand{\SUEsue}{{\SUE}\minsuesc}

\newcommand{\PRUmin}{minor \pays{brandebourg}\xspace}
\newcommand{\PRUMin}{Minor \pays{brandebourg}\xspace}
\TUdefshortcut{pays}{brandebourg}{*}{minprusc}
\newcommand{\PRUpru}{{\PRU}\minprusc}

\newcommand{\AUSmin}{minor \pays{habsbourg}\xspace}
\newcommand{\AUSMin}{Minor \pays{habsbourg}\xspace}
\TUdefshortcut{pays}{habsbourg}{*}{minaussc}
\newcommand{\AUSaus}{{\AUS}\minaussc}

\newcommand{\DANmin}{minor \pays{danemark}\xspace}
\newcommand{\DANMin}{Minor \pays{danemark}\xspace}
\TUdefshortcut{pays}{danemark}{*}{mindansc}
\newcommand{\DANdan}{{\DAN}\mindansc}

\def\PRUbra{\PRUpru}

\def\MAJHAB{[{\AUS}/{\HIS}]\xspace}
\def\MAJHOL{[{\HOL}/{\VEN}/{\DAN}]\xspace}
\def\MAJHOLx{[{\VEN}/{\DAN}]\xspace}
\def\hab{\AUSmin}
\def\HABmin{\AUSmin}
\def\HABMin{\AUSMin}
\def\HAB{\AUSaus}
\def\AUShab{\AUSaus}

\TUdefshortcut{pays}{german-empire}{GE}{GE}
\TUdefshortcut{pays}{saint-empire}{HRE}{HRE}


\def\AUT{\AUS}
\def\ENG{\ANG}
\def\SPA{\HIS}


\def\bonus#1{\expandafter\bonushelper#1\nil}
\def\bonushelper#1#2\nil{\bgroup\bf\if#1-\relax--\else#1\fi#2\egroup}

\def\CATHCR{Catholic/Counter-Reformation\xspace}
\def\CATHCO{Catholic/Conciliatory\xspace}
\def\PROTANG{Protestant/Anglican\xspace}
\def\PROTPUR{Protestant/Puritan\xspace}
\def\PROTRIG{Protestant/Rigorous\xspace}
\def\PROTTOL{Protestant/Tolerant\xspace}
\def\ORTHCHA{Orthodox/Champion\xspace}
\def\ORTHTol{Orthodox/Tolerant\xspace}

\def\Veteran{\emph{Veterans}\xspace}
\def\Conscripts{\emph{Conscsripts}\xspace}

% Jym, août 2012
% short commande for blason inclusion in section titles
\newcommand{\blasonJ}[1]{\includegraphics[width=12bp]{shield_#1}}
