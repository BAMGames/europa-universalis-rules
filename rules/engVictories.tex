% -*- mode: LaTeX; -*-

\definechapterbackground{Winning the game}{winning}
\chapter{Winning the game}\label{chapter:Victories}

% \section{Victory conditions}

% RaW: [22,52]

\begin{designnote}
  Even if the spirit of the game is mainly to simulate the whole modern
  history, a complete victory points (\VPs) system is explained here. It
  allows to designate a winner of the campaign, and also to know who is
  doing well and who isn't.

  The \VPs system is currently unbalanced. It will only be balanced once
  sufficiently many test games with stable rules will have been played. Thus,
  it should not be taken too seriously. A 1 \VP difference at the end of the
  game hardly qualifies as a ``Victory''. It is not only within the margin of
  error of the system but also within the margin of counting errors during the
  game (are you sure you did not forget a 1 \VP discovery at some point in the
  game?) Of course, if you're playing an EU championship, a 1 \VP difference
  is all it takes to be champion rather than vice-champion\ldots

  Thus, \VPs should be considered more as a guideline at what actions
  should be attempted even if they seem silly. Many monarchs of the
  period took decisions that in retrospective are considered stupid and
  no gamer would make the same error of spending that many efforts
  toward a useless goal. \VPs are an incentive to pursue some of those
  goals. This is especially true for some of the 'Mandatory' objectives
  that are often extremely hard to achieve but nonetheless were the main
  concern of historical monarchs (the Spanish ``Conversion of a
  Protestant major'' objective is a famous example of such a waste of
  efforts) .

  So, any action that is rewarded (or punished) by 50 \VPs should be
  considered as something extremely important and a huge goal to achieve for
  the players. A 10 \VPs bonus is more of a secondary objective, or a penalty
  that may be payed once or twice.
\end{designnote}

\begin{designnote}
  All in all, players are expected to gain around 2000-2500 \VPs along the
  game. The end of period check should be around 100-150 \VPs per period, more
  for countries in their period of Glory, less for decaying countries. The
  period objectives should be around 100-200 \VPs per period depending on your
  success on the field.
\end{designnote}

\begin{playtip}
  Some period objectives, as well as the end of game objectives, must be
  prepared in advance. Often, if one start trying a period objective
  only at the period it brings \VPs, it is too late. The ``Mediterranean
  Trade Center'' objective for \FRA is notorious for this. Thus, you
  should probably look your objectives in advance, as part of a long
  term strategy.
\end{playtip}

\begin{playtip}
  Get one player in charge of counting the \VPs. Other players won't need to
  know the \VPs per turn as well. That player should get a notebook and tally
  \VPs along the game. Since there are \VPs every turn, there is a lot of
  things to write.

  End of period (and end of game) \VPs are best checked with two players. One
  is reading the objectives and end of period checks while the \VPs accountant
  note the result. Others players should be around ready to answer questions
  (typically for computing Wealth).
\end{playtip}

\section{How to win the game ?}

\aparag The \VP system is composed of 3 parts.
\bparag[\VPs per turn:]
Those \VPs are earned by players performing special actions and
discoveries. This corresponds approximately to one sixth to one quarter of the
final \VPs.
\bparag[End of period \VPs:]
At the end of each period, players earn \VPs corresponding to their Wealth and
Prestige Expenses during the past turns and to the objectives they chose at
the beginning of the period. This corresponds approximately to one half to two
third of the final \VPs.
\bparag[End of game \VPs:]
At the end of the game, after counting \VPs of the end of seventh period, a
check-up of the the situation of each player is made, and \VPs are earned
according to it. For players who change country mid-game, a Transfer check is
also computed. This corresponds approximately to one sixth to one quarter of
the final \VPs.




\section{VPs per turn}\label{chVictories:Turn VP}
\aparag Players earn or lose \VPs each turn for regular or particular
situations, and also for particular discoveries.
\bparag Even if these \VPs are earned at different moments of the turn, it is
often easier to count them all together at the end of the turn.

\aparag Note that some other actions (\emph{e.g.} some events, \ldots) may
also give or take \VPs during turn and are not all recalled here.

\subsection{\VPs earned during each phase}
\aparag[During the Diplomacy phase:]~\\
\begin{modlist}
\item[-10] for each declaration of war without CB;
\item[+?] the income value of annexed provinces (Dowries);
\item[-?] twice the income value of lost provinces (Dowries).
\end{modlist}

\aparag[During the Administrative phase:]~\\
\begin{modlist}
\item[-5/15/30] for a Small/Major/Complete bankruptcy.
\end{modlist}

\aparag[During the Military phase:]~\\
\begin{modlist}
\item[+5] per major battle won.
\end{modlist}

\aparag[During the Peace phase:]~\\
\begin{modlist}
\item[+?] the income value of annexed provinces;
\item[-?] twice the income value of lost provinces;
  % \item[$\pm$?] half the income of occupied provinces (divided by 2
  %   for \TP/\COL, including Resources)
\item[+2] per peace level of a war won against at least one major power;
\item[+1] per peace level of a war won against only minor powers;
\item[-2] per peace level of a lost war;
\item[-20] for a forced religious conversion (unless another \VPs
  penalty is already stated);
\end{modlist}

\aparag[During the Interphase:]~\\
\begin{modlist}
  % \item[\textpm S] the \STAB
\item[+1] per partial \terme{monopoly} of Exotic Resource
\item[+3] per total \terme{monopoly} of Exotic Resource
\end{modlist}

\aparag[Definitions of \terme{monopolies}]
\label{chVictories:MonopolyExoticResources} for exploitation of each
exotics resources
% \bparag \STZ/\CTZ : a player has a partial \terme{monopoly} in a \CTZ or a
% \STZ when he is the only possesser of a face \Faceplus fleet counter in this
% zone. He has a total \terme{monopoly} when he has a 6 levels fleet, and there
% is no other fleet in the zone.
\bparag For each resource, a country has a partial \terme{monopoly} if
it produces at least 6 units and at least half of the world production
of this resource. Note that two countries may have a partial monopoly of
the same resource in the rare case where both produce exactly one half
of the total.
\bparag A country has a total \terme{monopoly} of a given resource if it
produces at least 6 units of that resource, and if no more than 2 units
of that resource are produced by other countries.
\bparag Resources produced by minor countries are counted as usual when
computing monopolies. Especially, resources exploited by a minor in
\dipAT are counted as if exploited by the major gaining the
corresponding income.
\bparag If a \ROTW minor country happen to have a monopoly
in one resource, no \VPs are gained (because \ROTW minors do not have
diplomatic patron).

\aparag[\VPs of minor powers]
The players also earn (or lose) half of the \VPs earned (or lost) by a minor
country which they control.
\bparag This is both for allied minors and for Neutral minors controlled by
the player (in order to avoid the players to voluntarily play poorly).

\subsection{\VPs earned for discovering the World}
\aparag[Discoveries]~\\
% TBD??? Keep all of them or not ???? JC, Jym ???  PB: seems to have been done
\begin{modlist}
\item[+50] the 1st round-the-World trip (if it is completed in a single turn)
\item[+20] the 1st round-the-World trip (if it is completed in 2 turns)
\item[+20] the 2nd round-the-World trip
\item[+10] the \seazoneHorn sea zone
% (jym) removing, POR should do it at T1 and HIS should go to America:
% \item[+10] the \seazoneTempetes sea zone
\item[+3] the \seazoneHudson sea zone
\item[+3] the \granderegionQuebec area
\item[+2] the \granderegion{Grands Lacs} area
\item[+3] the \granderegionRocheuses area
\item[+3] the \granderegionAlaska area
\item[+5] the \granderegionPanama area by the West
% (jym) removing, RUS already has incentive for doing it:
% \item[+5] the \granderegionKamchatka area by land from Europa
% \item[+5] the \granderegionAlaska area via \continentAsia
\item[+1] per province of the \granderegionAmazonia area
\item[+1] per province bording Mississippi river
\end{modlist}

\aparag[Discoveries] \VPs are earned when the discovery is made
(successful exploration roll), even if the stack is later destroyed
before bringing it back home (that is, some rumours about it reach the
home country and Europe anyway).
\bparag If several countries are able to claim \VPs for the same
discovery during the same round, they are considered as moving in order
of initiative: the first country to do the discovery, and actually gain
the \VPs, is the one with the higher initiative.
\bparag Discoveries \VPs of \Areas are given to the first country who
discovers at least one province in the \Area.
\bparag Discoveries \VPs of provinces and sea zones are given to the first
country who discovers it. Each province of \granderegionAmazonia as well as
each province bordering the Mississippi river is worth 1\VP individually.
\bparag List of provinces bordering the Mississippi: all provinces of
\granderegionMississippi, the two central provinces of \granderegionIllinois
and the two Eastern provinces of \granderegionKansas.

\aparag[The way is more important than the destination.]
\bparag Discovery of \granderegionPanama by the West is landing one stack in
\granderegionPanama from the Pacific Ocean. Both the land stack and the naval
stack carrying it must have cross Cape Horn (or used the special movement
of~\ref{chMilitary:Movement:Port Multiple Coasts} to avoid it).
% \bparag Discovery of \granderegionKamchatka by land is done using the special
% movement of~\ref{chBasics:Secret Passage:Bering}. The stack doing it may not
% have used naval transport since it left the European map.
% \bparag Discovery of \granderegionAlaska via \continentAsia is done by landing
% a stack in \granderegionAlaska, the naval stack carrying it must have left
% from a port in \continentAsia or \continentSiberia.
\bparag For this discovery, the \VPs are earned once the condition on
the way is met, even if the province is already known (including by the
power doing the discovery).

\begin{exemple}
  \HIS first discovers \granderegionPanama from the East
  (\seazoneMexique) and build a \COL there. Later \HIS goes round
  America and lands in the same, already colonised province from
  \seazonePanama. Since \HIS already knows the province, no discovery
  roll is needed. \HIS still gets the ``\granderegionPanama by West''
  \VPs (unless another country already grabbed these \VPs, of course).
\end{exemple}


\aparag[Circumnavigation] Round-the-World trip are completed when one
naval stack (possibly a single leader) goes from a port on the European
map, back to a port on the European map (possibly the same) after going
all the way around the World (use common sense).
\bparag Seriously, use common sense. Don't make me write the rules for
this.

% \aparag[Circumnavigation] (precision for the insanes)
% \bparag Round-the-World trips are completed when at least 1 \LDE or leader
% goes back to a port it already went in after going an odd number of time in
% each of the following zones: \seazone{Bonne-Esperance}, \seazoneTempetes,
% \seazonePacifique and \seazoneChili.

% (Jym)
% Can't define circumnavigation properly
% "At least once in each" is no good because one could do BE/Horn/Pac/Horn and
% skip Indian ocean.
% "Odd number" is no good because one could do BE/Horn/Pac/BE and
% circumnavigate with 2 BE.


\section{End of period \VPs}
\aparag The end of period \VPs are decomposed in 3 parts: Wealth and Prestige
\VPs; a check up of the situation of each country; and the verification of the
objectives chosen at the beginning of the period.

\subsection{Prestige}
\aparag[Wealth and Prestige \VPs] for each country are a certain percentage of
the average Wealth of the period. It is computed as follow.
\bparag The base Wealth is \lignebudgetlong{Period wealth}.
\bparag In case of transfer or end of game computation, modify the base Wealth
as follow:
\begin{modlist}
\item add the \RT if positive (\lignebudgetlong{RT at end});
\item remove twice the \RT if negative (\lignebudget{RT at end});
\item remove twice the amount of ongoing loans (\lignebudgetlong{National
    loans at end} plus the amount of international loan not yet refunded).
\end{modlist}
\bparag Divide this result by the number of turns in the period (do not round
yet). In case of transfer or end of game computation, only count the number of
turns actually played.
\bparag Multiply this average Wealth by the percentage indicated in the
table below and round down. This is the amount of Wealth and Prestige
\VPs gained by the country.

\aparag Period percentage: \par

\begin{center}
  \begin{tabular}{|*{10}{c|}}\hline
    \multicolumn{10}{|c|}{Percentages for Wealth and Prestige \VPs} \\
    \hline
% Do not auto format!
                 &\ANG&\FRA& \POL& \POR&\RUS&\HIS&\TUR& \VEN&\HOL\\
                 &    &    &+\PRU&+\SUE&    &    &    &+\AUS& \\
    \hline
    p\period{I}  & 25 & 25 & 100 & 40  &100 & 25 & 30 & 25  & na \\
    p\period{II} & 20 & 25 &  75 & 20  & 75 & 20 & 25 & 25  & na \\
    p\period{III}& 20 & 20 &  60 & 60  & 60 & 20 & 20 & 25  & 20 \\
    p\period{IV} & 15 & 20 &  50 & 50  & 40 & 15 & 15 & 50  & 15 \\
    p\period{V}  & 10 & 15 &  50 & 30  & 40 & 10 & 15 & 40  & 10 \\
    p\period{VI} & 10 & 10 &  50 & 25  & 30 & 10 & 15 & 25  & 10 \\
    p\period{VII}& 10 & 10 &  50 & 25  & 30 & 10 & 15 & 25  & 15 \\
    \hline
  \end{tabular}
\end{center}

\begin{exemple}
  At the end of period \period{III}, \TUR managed to have a total Wealth of
  6012\ducats (\lignebudget{Period wealth} at turn 25). Period \period{III} is
  11 turns long and \TUR percentage is 20\% in this period. Hence \TUR scores
  $6012 / 11 \times 0.20 = 109.3$ rounded down to 109 Wealth and Prestige
  \VPs.

  At the same time, \VEN has 5696\ducats of Wealth, with 26\ducats in \RT but
  an ongoing national loan of 148\ducats. \VEN percentage is 25\% and this is
  a transfer computation as \VEN switch to \AUS in period \period{IV}. Thus,
  its Prestige and Wealth \VPs are $(5696 + 26 - 2 \times 148) / 11 \times
  0.25 = 123.3$ rounded down to 123\VPs.
\end{exemple}

\subsection{End of period check up}
\aparag Each country earns \VPs at the end of each period according to their
overall situation (colonial, territorial and diplomatic).
\bparag Not all countries gain \VPs this way and some countries have special
modifiers representing the historical policies that were pursued.

\aparag[Colonial situation.] The countries indicated below (and only
these countries) gain 5 \VPs for each \COL or \TP in the specified
continents (or country) and periods (any time if no period specified).
\bparag \ANG: \continentIndia in periods \period{IV} to \period{VII};
\granderegionOceania in period \period{VII}.
\bparag \FRA: \continentAmerica always; \continentIndia in periods
\period{VI}, \period{VII}; \granderegionOceania in period \period{VII}.
\bparag \POR: \continentAmerica only if there is 3 or more \POR \COL counters
(whatever their side) in \continentBrazil; in this case, each \COL\faceplus
counts as 2 \COL (and earns 10\VPs).
\bparag \RUS: \paysChine, \continentAmerica; each \COL\faceplus counts as 2
\COL (and earns 10\VPs).
\bparag \SUE: \continentAmerica.

\begin{designnote}
  If \ANG or \FRA has \COL in \granderegionOceania in period \period{V}, they
  do not earn any \VPs for these. They will only bring \VPs in period
  \period{VII}. This represent the fact that most of this area was discovered
  by the expeditions of \leaderCook, \leaderBougainville or \leader{La
    Perouse} and thus were only known to Europeans in the very late game and
  colonised in the 19th Century.

  Similarly, French \COL in \continentIndia only earns \VPs in the end game
  while English \COL there start earning \VPs in period \period{IV}. This
  represents the different colonial policies (and power) of these countries
  and the early start of \ANG in \continentIndia (with the East India Company)
  while \FRA focused its efforts toward colonising Canada (especially during
  the reigns of Louis XIII and Louis XIV).
\end{designnote}

\begin{exemple}
  At the end of period \period{I}, \POR as a \COL\faceplus in
  \granderegionRecife W., a \COL \facemoins in \granderegionBelem E. and a
  \COL \facemoins in \granderegionAmazonia SE. Since that's only 2 \COL
  counters in \continentBrazil, it does not earn \VPs for Colonial situation.

  Suppose now that the third \COL is in \granderegionRio S. instead of
  \granderegionAmazonia. Now that's 3 counters in \continentBrazil and since
  the \COL\faceplus counts as 2, that's a total of 4 \COL and 20\VPs.
\end{exemple}

\aparag[Territorial gains.] Each country gains \VPs equal to the income value
of each province annexed during the period with following modifiers. Note that
these are in addition to the \VPs gained when annexing the province.
\bparag \FRA: former provinces of \paysBourgogne \textmultiply 0 during
periods \period{I} and \period{II}, \textmultiply2 during periods
\period{V} to \period{VII}; provinces in \regionItalie \textmultiply2
in periods \period{I} to \period{III}.
\bparag \SPA: provinces in \regionItalie \textmultiply2 in periods
\period{I} to \period{III}.
\bparag \ENG: provinces bordering \regionMediterranee \textmultiply5;
other continental provinces \textmultiply2.
\bparag \TUR: provinces taken from \paysPerse \textmultiply2; provinces
lost against \paysPerse count negative; provinces bordering
\regionMediterranee \textmultiply2.
\bparag \SUE: provinces bordering \regionBaltique \textmultiply3.
\bparag \RUS: provinces taken from \POL or \paysPologne \textmultiply2 ;
provinces lost against \POL or \paysPologne count negatively.
\bparag \POL: provinces taken from \RUS, \SUE or
\paysSuede\textmultiply2.
\bparag \AUS: former provinces of \paysHongrie or \paysMoldavie,
\provinceSerbia \textmultiply3.
\bparag \PRU: provinces of the \HRE or provinces take from \paysPologne
\textmultiply2.
\bparag \VEN: lost islands in \regionMediterranee count negatively.

\aparag Some countries have special modifiers for provinces taken from
(or lost to) a specific country (\emph{e.g.} \RUS versus \POL). For this
modifier to apply, the province must be owned by the specified country
just before being owned by the other.
\bparag Thus, \emph{e.g.}, if \RUS annexes a former Polish province
which is currently owned by \SUE, it does not trigger its special
modifier.
\bparag Other countries have special modifiers for specific provinces,
whoever owned them prior to the annexation.

\begin{designnote}
  Here also, the variations along the game represent different policies of
  countries. For example, \FRA focused its effort in Italy in the early game
  and only tried to grab parts of the Burgundian legacy after its Wars of
  Religion, with the policies of Richelieu and Louis XIV, or the ``\emph{pré
    carré}'' of Vauban.
\end{designnote}

\aparag[Diplomatic situation]
\bparag Each country earns {\bf 6 \VPs} for each minor country in {\bf
  \VASSAL, \ANNEXION} or {\bf \dipAT}, as well as special \EW (those
with no diplomacy allowed), and {\bf 3 \VPs} for each other controlled
minor country, with following modifiers:
\bparag \HIS: minors in \regionItalie \textmultiply2; \paysChevaliers
\textmultiply2 ; do not count autonomous Habsbourg states ; do not count
\paysChevaliers if they are still in \provinceRhodos.
\bparag \TUR: Muslims: only 4 \VPs for \VASSAL, \ANNEXION or \dipAT, 2\VPs for
other status; do not count Non-Muslims minors except \paysTransylvanie.
\bparag \FRA: minors in \regionItalie \textmultiply2 during periods
\period{I} to \period{III} ; minors of the \HRE \textmultiply2 in periods
\period{IV} to \period{VII}

\aparag[Special]
\bparag If \paysChevaliers is still in \provinceRhodos (whatever its
diplomatic status), \HIS gains 20\VPs.
% Avoid \HIS wanting the knights out of Rhodes to start earning VPs for them.
% (© MKL pouitage)

\subsection{Period's objectives}
\aparag[Overview]
\bparag At the beginning of each period, each player chooses global
orientations for its country by selecting 3 out of 5 possible objectives.
\bparag Once the period ends, objectives that were successfully achieved earn
\VPs.
\bparag The list of all objectives, per country, is given
in~\ref{chVictories:objectives}. A comprehensive table is also provided as a
player's aid. Note that in case of ambiguity or contradiction, the long list
in the rules is correct and the table is wrong.
\bparag Each objective is associated with a \VPs value. Sometimes it is a
yes/no objective with a single value (\emph{e.g.} ownership of a specific
province) and sometimes it is a ``for each'' objective with a value for each
item and a maximum value for the objective (\emph{e.g.} some \VPs for each
\TradeFLEET counter of the country in play).

\aparag[Choosing objectives]
\bparag At the beginning of each period (before the first turn of the
period), each player must secretly pick 3 of the possible 5 objectives
for its country.
\bparag One of the 3 must be marked as \textbf{main objective}.
\bparag Each player should write the 3 objectives on some paper, and then all
these papers should be put together, \emph{e.g.} in an envelop (sealed if you
don't trust your fellow players\ldots)

\aparag[Scoring objectives]
\bparag At the end of each period (during the end-of-period \VPs check),
objectives are revealed and checked, and players earn \VPs according to
this.
\bparag A \VPs value is indicated for each objective. If a player
chooses an objective and completes it, he earn its \VPs value, if this
is the \textbf{main objective}, he earns twice this value instead.

\aparag[Mandatory objectives]
\bparag Some objectives are marked on the tables with {\bf M} ('M'alus, or
'M'andatory objectives), they are objectives with malus.
\bparag If a player \textbf{both} does not choose this objective, \textbf{and}
does not complete it, he loses the associated \VPs.
\bparag If this is a ``for each'' objective, having a single one of the
required elements is enough to avoid the penalty. However, having none
of them result in a penalty equal to the maximum potential value of the
objective.


% (jym) most 'E' events seem to make much more sense if checked
% immediately rather than waiting for the next period to finish.
\aparag[Event objectives]
\bparag Some objectives depend on the occurence of an event.
\bparag If the corresponding event has not finished %or occurred
when the period ends, the check for the success of the objective is
postponed until the event %occurs and
terminates.
% \bparag If the event can not happen anymore (due to being 2 periods before),
% then the objective is considered as being half realised, entitling the player
% to gain half the \VPs value.
\bparag If the event did not occur yet, even if it may occur later, then
the objective is considered as being half realised, entitling the player
to gain half the \VPs value.

\begin{designnote}
  The event may occur later (in the next period) without changing the
  \VPs gained by the objective, whatever its outcome.

  In some case, it is also possible that the event occured one period
  early and is already resolved when the choice of objective has to be
  made, resulting in a ``free'' objective that is already fulfilled and
  is guaranteed to bring \VPs. Enjoy!
\end{designnote}

\begin{designnote}
  {\bf M} objectives are usually hard goals that historical monarchs
  pursued with a lot of efforts but often did not succeeded and resulted
  in a loss of these efforts. In retrospective, wargamers would like to
  ignore these and focus their efforts on goals more likely to
  succeed. The system forces the players to not completely ignore
  these. If they choose not to pick the {\bf M} objective and ignore it,
  they will loose \VPs. If they pick up the objective, since only three
  objectives can be picked, ignoring it is equivalent to picking only
  the two others thus forfeiting potential \VPs for a third one\ldots
  Since {\bf M} objectives are usually among the ones with highest \VPs
  value, it is normally best to choose it. Of course, the question of
  choosing it as \textbf{main objective} or not remains open.
\end{designnote}

\begin{exemple}
  At the start of period \period{I}, \ANG chooses objectives ``Calais''
  (as \textbf{main objective}), ``Hundred years war'' (the {\bf M}
  objective, representing a policy of continuing the Hundred Years War)
  and ``Pacified Ireland*''.

  At the end of the period, \ANG still owns a \Presidio in \provincePicardie
  (representing Calais) but does not own \provinceGuyenne. Good revolt rolls
  resulted in \regionIrlande without revolt for 5 out of the 6 turns. During
  the same time, \paysEcosse was \VASSAL for 2 turns of the period.

  Since Scotland was not an objective, \ANG does not earn any \VP for
  it. Since the maximum possible gain for the Irish objective is 40\VPs, \ANG
  only gains these 40\VPs and not 50 as would be expected for 5 turns without
  revolt. Thus, the objective VPs for \ANG are: 2 \textmultiply 45 + 40 =
  130\VPs.

  Note that if \ANG had chosen the Scottish objective instead of the Hundred
  Years War one, then it would have failed a {\bf M} objective and lost the
  associated \VPs, resulting in a total of 2  \textmultiply 45 + 40 + 20 - 50
  = 100 \VPs only.

  If Ireland was the \textbf{main objective}, then the maximum is computed
  before doubling, that is it would result in 2 \textmultiply 40 = 80 \VPs.
\end{exemple}

\begin{exemple}
  Still in period \period{I}, suppose that \HIS does not choose its {\bf M}
  objective ``Barbary Coast''. If \HIS nonetheless manage to have at least one
  \Presidio on the Barbary Coast, the objective is considered successful and
  no \VP is lost. If no \Presidio is there, then the objective is failed and
  \HIS loses the full 50 \VPs of it.
\end{exemple}

\section{Period objectives per country}
\label{chVictories:objectives}

\subsection{Explanation of some objectives}

% (jym) apparently, no more "MNU" obj. Probably replaced by the
% "Industrial development" one.

% \aparag[MNU.] Some objectives are explained as \MNU: those objectives
% consist in having at the end of the period a number of Manufacture units
% strictly superior to the period limit.
% \bparag In order to achieve it, one either has to build extra \MNU in
% advance, and take associated risks (\ref{chThePowers:Exceeding Limits}),
% or to make a last attempt in the last turn of the period, and risk
% failing it.

\aparag[Commercial Domination in the \region{Baltique}.] Each turn, one
country may have Commercial Domination in the
\region{Baltique}. Commercial Domination is attributed, in decreasing
order of precedence, to the \MAJ who
\begin{itemize}
\item levy Sund taxes (\ref{chSpecific:Sund Levies});
\item has diplomatic control of the minor with right to levy Sund taxes
  (whether there are taxes or free trade);
\item has Commercial Monopoly in \stz{Baltique}
  (\ref{chIncomes:Commercial Monopoly}) and nobody is levying the Sund
  taxes.
\end{itemize}

\begin{designnote}
  Note that if the minor with right on the Sund is not Neutral, then the
  Domination is awarded to its Diplomatic patron whatever the status of
  the taxes. On the other hand, if it is Neutral, and does not levy the
  taxes, then having Commercial Monopoly is enough.

  Similarly, if the right on the Sund is owned by a \MAJ and it's not
  levying taxes, then Commercial Domination is awarded to the country
  with Commercial Monopoly.
\end{designnote}

\begin{histoire}
  The taxes where initially in the hand of \paysDanemark but the Swedish
  independence and conquest of \provinceSkane switched this important
  source of income to other hands. \HOL, having a large trade activity
  in \stz{Baltique}, wasn't happy with foreign taxes and preferred free
  trade. This resulted in several Dutch implications in the Northern
  wars, often using the Dutch fleet to reinforce the Danish one.
\end{histoire}

\aparag[Dominium Marii Baltici.] \SUE has the DMB if all provinces
bordering the \region{Baltique} are owned by either \SUE,
\pays{Brandebourg} (or \PRU) or \pays{Danemark}.

\begin{histoire}
  This represents the Swedish expansion towards the Southern shores of
  the sea. Including the frequent struggles with \POL, the will to
  destroy \paysHanse and the ongoing conflict with \RUS around
  \provinceIngria.
\end{histoire}

\aparag[Orient Income] \VEN has objectives depending on its
\terme{Orient Income}. Orient Income is computed over each period and is
the sum of all income from any of the following source:
\bparag \CCs{Grand Orient} or \CCs{Tempete};
\bparag convoy of \bazar{Izmir} or \terme{East Indies} convoy;
\bparag resources exploited through an \dipAT with \paysaden, \paysoman
or \paysgujarat;
\bparag total income of each \COL/\TP producing at least one unit of
\RES{PO}, \RES{Spice} or \RES{Silk};
\bparag \TradeFLEET in any \STZ of the \CCs{Tempete}.

\aparag[\TradeFLEET and Trade Centres.]
For period objectives only, consider that the Trade Centres are given
according to the \emph{maximum} levels of \TradeFLEET and not according
to the \emph{current} level as per normal rules.
\bparag This does not move the counters nor changes the ownership of the
Centres for any other purposes.
\bparag Similarly, when counting the number of \TradeFLEET counters
owned by a country (and their side), consider that the counters are here
according to their \emph{maximum level}, not to their \emph{current
  level}. This does not change any \TradeFLEET level or counter.
\bparag When counting Commercial Monopolies (\ref{chIncomes:Commercial
  Monopoly}) consider any \TradeFLEET of \emph{maximum level} 6 as a
total monopoly and any \TradeFLEET of \emph{maximum level} 4 or 5 as a
partial monopoly. This may result in several countries having monopoly
in the same \CTZ/\STZ for objective purposes.
\bparag All this \textbf{only} applies for period objectives. At any
other times, use the \emph{current level} for placing counters, counting
monopolies, triggering automatic competition or placing Trade Centres.

\begin{designnote}
  This avoids last instant backstabs and lessens the ``end of period''
  effect. Especially, \corsaire can cause a lot of temporary loses in
  one turn and it would be unfair to bet a lot of objectives \VPs on
  this.
\end{designnote}

\begin{exemple}
  At the end of period \period{III}, after a long commercial struggle
  and several Barbaresque's raids, the situation in the Mediterranean is
  as follows (current level/maximum level):
  \begin{itemize}
  \item \stz{Lion}: \HOL (3), \HIS (1/3), \VEN (2/5), \FRA (2/3).
  \item \stz{Ionienne}: \HOL(4/5), \VEN (0/4).
  \item \ctz{Venise}: \VEN (6).
  \item \ctz{Turquie}: \TUR (5), \VEN (3), \HOL (3).
  \item \stz{Noire}: \HOL (5), \VEN (3), \TUR (2), \FRA (3)
  \item \stz{Caspienne}: \VEN (3), \HOL (4).
  \end{itemize}
  Thus, \HOL as a total of 19 current levels and 20 maximum levels while
  \VEN has a total of 17 current levels and 24 maximum levels. Hence,
  even if the \CCs{Mediterranee} is currently located in Holland with 19
  levels, for objectives purpose (only), \VEN is considered has having
  it, thus fulfilling its objective (while \HOL fails it). That is, \HOL
  should have planned its attack earlier.

  If the number of \TradeFLEET were needed for objectives purpose, then
  \VEN is considered as having a \TradeFLEET\faceplus both in \stz{Lion}
  and \stz{Ionienne} even if the first one is currently \Facemoins and
  the second is not here. Similarly, still for objectives purposes, both
  \HOL and \VEN are considered as having a partial monopoly in
  \stz{Ionienne}.
\end{exemple}

\aparag[No province lost] objectives mean that the country did not loose
ownership of any province it owned at the beginning of the period.
\bparag If the country annexes new provinces during the period and
looses them later during the same period, the objective is still
successful. Only provinces that were owned at the beginning of the
period are checked against this objective.

\begin{designnote}
  Thus, countries with a ``No provinces lost'' objective may still
  pursue an aggressive policy. Any province annexed early is a province
  that can be relinquished later without penalty rather than one more
  province to defend at any cost.
\end{designnote}

\aparag[Commercial monopolies.] When objectives ask for Monopoly in \STZ
or \CTZ or in production of resource without precision, any monopoly
(partial or total) counts toward fulfilling the objective.

\aparag[Independence wars] \label{chVictories:Explanation:Independence}
\ref{pVII:Independence War} may occur several times, resulting in several
rebellions. The associated objective is checked as follows, in
decreasing order of precedence:
\begin{itemize}
\item If at least one rebellion war occured in the country's colonies
  and all independence wars in the country's colonies were crushed, full
  success.
\item If at least one successful rebellion war occured in the country's
  colonies, the objective is failed.
\item If no rebellion war occured at all, half-success.
\item If at least one rebellion war occured in another country's
  colonies, and the major helped the rebels in all rebellion wars, and
  the rebels were successful every time, full success.
\item If at least one rebellion war occured in another country's
  colonies and the major helped at least one successful rebellion,
  half-success.
\item Otherwise, failure.
\end{itemize}
\bparag Note that most of the time, the war occurs only once, thus the
objective is simply to take part in the war and win it.

\aparag[Duration.] Objectives for actions only concern what happens
during the period just ended.
\bparag For example, a ``per turn'' objective in period \period{II} can
only be fulfilled with actions done during period
\period{II}.
\bparag Similarly, a ``per province annexed'' objective in period
\period{IV} only scores \VPs for provinces that were annexed during
period \period{IV}.
\bparag However, objectives for situation only check the current
situation, no matter when it was settled.
\bparag For example, a ``per province owned'' objective in period
\period{IV} scores for each province owned at the end of period
\period{IV} no matter whether it was annexed during period \period{I} or
\period{IV}.

\aparag[Minor provinces.] When an objective refers to the provinces of a
minor country (\paysCrimee, \paysGeorgie, \paysHanse, \paysHongrie,
\paysMoldavie, \paysPerse, \paysNaples, \paysprovincesne,
\paysValachie), it consists in all the provinces barring the
corresponding solid or blurred shield, and owned by the country at some
point in the game, whatever their current owner (especially if the
country was destroyed).
\bparag However, when an objective refers to provinces ``taken from''
a specific country, then the province must have been owned by that
country just before it was transferred (either by peace treaty,
diplomacy, \ldots)
\bparag List of Crimean provinces (\blasonsmall{crimee}):
\provinceKhadzhibei, \provinceZaporozhye, \provinceCrimee,
\provinceAzov, and possibly \provinceCaffa, \provinceKuban.
\bparag List of Georgian provinces (\blasonsmall{georgie}):
\provinceGeorgie, \provinceKuban.
\bparag List of Hanseatic provinces (\blasonsmall{hanse}):
\provinceBremen, \provinceLubeck, \provinceHolstein,
\provinceMecklenburg.
\bparag List of Hungarian provinces (\blasonsmall{hongrie}):
\provinceMures, \provinceErdely, \provinceBukovina, \provinceKarpatok,
\provinceSzlovakia, \provinceBalaton, \provinceCarniola,
\provinceKapela, \provinceCroatie, \provincePecs, \provinceMagyarorszag,
\provinceBanat.
\bparag List of Moldavian provinces (\blasonsmall{moldavie}):
\provinceMoldova, \provinceBasarabia.
\bparag List of Napolitan provinces (\blasonsmall{naples}):
\provinceAbruzzo, \provinceCampania, \provinceBasilicata,
\provincePuglia, \provinceCalabria, \provinceSicilia, \provincePalermo,
\provinceSaldigna.
\bparag List of Persian provinces (\blasonsmall{perse}): \provincePars,
\provinceMeshhed, \provinceBam, \provinceIsfahan, \provinceKermanshah,
\provinceAzarbayadjan, \provinceKordistan, \provinceVan,
\provinceArmenie, \provinceShirvan, \provinceDagestan.
\bparag List of provinces of the North-East (\blasonsmall{hollande}):
\provinceFriesland, \provinceGelderland, \provinceHolland,
\provinceOverijssel, \provinceUtrecht, \provinceZeeland.
\bparag List of Walachian provinces (\blasonsmall{valachie}):
\provinceValahia.
\bparag Note that \provinceKuban may be both a Georgian and Crimean
province and thus bring \VPs for both reasons. It may, however, only
count once for each objective.

\begin{exemple}
  In period \period{III}, \RUS has an objective for provinces taken from
  \paysCrimee. Only provinces that were owned by \paysCrimee just before
  being owned by \RUS count. Other provinces with a \blasonsmall{crimee}
  shield that were owned by, say, \paysUkraine before Russian annexation
  do not count.

  On the other hand, in period \period{VII}, \RUS has an objective for
  provinces of \paysGeorgie or \paysPerse. Now, any province with either
  a \blasonsmall{georgie} or \blasonsmall{perse} shield counts, even if
  it was annexed from, typically, \TUR.
  % Since it is a single objective, \provinceKuban only counts as one
  % province, even if it is both a (former) Georgian and Crimean
  % province.
\end{exemple}

\aparag[Ownership and Control]
\bparag Most objectives requiring ownership of several provinces don't
care about who is the controller of the province. That is, if a war is
going on when objectives are checked, only the rightful owner of the
province may claim it for these objectives.
\bparag On the other hand, most objectives requiring ownership of a
single province require both ownership and control when the objective is
checked and only provide half success if owner and controller are
different. Thus, a war may quickly change some \VPs.

\begin{designnote}
  This lessens an end-of-period effect where players tend to stay in
  lost wars one more turn to score the full \VPs value of a specific
  province as an objective before signing a peace and relinquishing that
  province on the very next turn.
\end{designnote}

\begin{todo}
  The English objectives of \paysEcosse Vassal are incompatible with the
  idea of the Auld alliance forbidding a \ANG-\paysEcosse vassalship
  before Elisabeth.

  Anyway, objectives for \ANG in the early game need to be seriously
  revamped into something more interesting and interactive\ldots
\end{todo}

% -*- mode: LaTeX; -*-

%  \EUobjective{M?}{short name (table)}
%    {XXX}{VP per XXX/half success}{max VP/total VP}
%    {long name (rules)}

\makeatletter
\newcommand{\EUcurrmaj}{\paysmajeur{\EUcurrcoun}\xspace}
\newcommand{\testempty}[3]{% arg empty not-empty
  \ifx#1\relax\relax#2\xspace\else#3\xspace\fi}
\newcommand{\ifnotempty}[2]{\testempty{#1}{}{#2}}

\newcounter{EU@objhalving}
\newcommand{\EU@objhalf}[1]{%
  \setcounter{EU@objhalving}{#1/2}%
  \theEU@objhalving}


% Macros for repeated explanations. As much as possible, if an objective is
% present at least twice it deserves a specific macro to ensure that all the
% texts are exactly the same.
%% Generic ones
%%% Commercial
\newcommand{\EU@objIndustrial}[2][]{% M? VPs.
  \EUobjective{#1}{Industrial development}{}{}{#2}{\EUcurrmaj has at least as
    many \MNU\faceplus as its period limit, and Commercial Monopoly in
    \ctz{\EUcurrcoun}}}%
\newcommand{\EU@objMonopolyZoneFull}[4][]{% M? each max precision
  \EUobjective{#1}{Trade monopolies}{monop\EUAbbr{oly}}{#2}{#3}{%
    Each Commercial \terme{Monopoly} in any \STZ/\CTZ#4}}
\newcommand{\EU@objMonopolyZone}[3][]{% M? each max
  \EU@objMonopolyZoneFull[#1]{#2}{#3}{}}
\newcommand{\EU@objMonopolyRES}[4][]{% M? RES per_turn max
  \EUobjective{#1}{\RES{#2} Monopoly}{turn}{#3}{#4}{Each turn \EUcurrmaj has
    \RES{#2} Monopoly}}%
\newcommand{\EU@objEastIndiesConvoy}[3][]{% M? per_turn max
  \EUobjective{#1}{\emph{East Indies}}{turn}{#2}{#3}{Each turn \EUcurrmaj owns
    the \emph{East Indies} convoy and sails successfully at least 1\NTD of it
    to Europe}}%
\newcommand{\EU@objTradeCentre}[3][]{% M? CC VPs
  \EUobjective{#1}{\CCs{#2}}{}{}{#3}{\EUcurrmaj owns the \CCs{#2}}}%
\newcommand{\EU@objEachCC}[3][]{% M? per_CC max
  \EUobjective{#1}{Each \CC}{CC}{#2}{#3}{Each \terme{Commercial Centre} owned
    by \EUcurrmaj}}%
\newcommand{\EU@objBalticTrade}[3][]{% M? turn max
  \EUobjective{#1}{\region{Baltique} Trade}{turn}{#2}{#3}{Each turn of
    Commercial Domination in the \regionBaltique}}%
\newcommand{\EU@objEachItemText}[2][]{Each #2 of \EUcurrmaj%
  \ifnotempty{#1}{#1}}%
\newcommand{\EU@objEachItem}[6][]{% M? complement? item short? per max
  \testempty{#4}{%
    \testempty{#2}{%
      \EUobjective{#1}{Each #3}{#3}{#5}{#6}{\EU@objEachItemText[#2]{#3}}}{%
      \EUobjective{#1}{#2}{#3}{#5}{#6}{\EU@objEachItemText[#2]{#3}}}}{%
    \EUobjective{#1}{#4}{#3}{#5}{#6}{\EU@objEachItemText[#2]{#3}}}}
\newcommand{\EU@objEachCOL}[4][]{% complement? short? per max
  \EU@objEachItem[]{#1}{\COL}{#2}{#3}{#4}}
\newcommand{\EU@objEachCOLTP}[4][]{% complement? short? per max
  \EU@objEachItem[]{#1}{\COL or \TP}{#2}{#3}{#4}}


%%% Religious
\newcommand{\EU@objSDCF}[2][]{% M? VPs
  \EUobjective{#1}{SD of Catholic Faith}{}{}{#2}{\EUcurrmaj is \SDCF}}%

%%% Political + Territorial
\newcommand{\EU@objEmperor}[3][]{% M? nationality VPs
  \EUobjective{#1}{#2 King Emperor}{}{\EU@objhalf{#3}}{#3}{#2 King was elected
    to the Imperial Throne at least once since the beginning of the
    game. Half-success if no election was held}}%
\newcommand{\EU@objOuterProvincesText}{Each non-national province owned by
    \EUcurrmaj, no matter who controls them\xspace}%
\newcommand{\EU@objOuterProvinces}[3][]{% M? per max
  \EUobjective{#1}{Outer provinces}{prov\EUAbbr{ince}}%
  {#2}{#3}{\EU@objOuterProvincesText}}%
\newcommand{\EU@objOuterProvincesAbove}[4][]{% M? per max above
  \EUobjective{#1}{Outer provinces}{prov\EUAbbr{ince}-#4}%
  {#2}{#3}{\EU@objOuterProvincesText (above 3)}}%
\newcommand{\EU@objSilesie}[2][]{% M? VP
  \EUobjective{#1}{Silesia and Lausitz}{}{}{#2}{\provinceSilesie and
    \provinceLausitz are both owned by \EUcurrmaj, no matter who currently
    controls them}}%
\newcommand{\EU@objAlliance}[6][]{% M? turn max alliance? ally short
  \EUobjective{#1}{#6}{turn}{#2}{#3}{Each turn in #4 alliance with #5}}%
\newcommand{\EU@objAllianceAny}[3][]{% M? turn max
  \EU@objAlliance[#1]{#2}{#3}{military}{any \MAJ}{Alliance with \MAJ}}
\newcommand{\EU@objEachProvinceOtherText}[2][]{Each province owned by #2
  \ifnotempty{#1}{among #1}, no matter who controls them\xspace}%
\newcommand{\EU@objEachProvinceText}[1][]{%
  \EU@objEachProvinceOtherText[#1]{\EUcurrmaj}}%
\newcommand{\EU@objEachProvinceListAbove}[6][]{% M? list per max short above?
  \EUobjective{#1}{#5}{prov\EUAbbr{ince}*\ifnotempty{#6}{-#6}}{#3}{#4}{%
    \EU@objEachProvinceText[#2]\ifnotempty{#6}{ above #6}}}%
\newcommand{\EU@objEachProvinceList}[5][]{% M? list per max short
    \EU@objEachProvinceListAbove[#1]{#2}{#3}{#4}{#5}{}}
\newcommand{\EU@objEachProvinceAbove}[6][]{% M? above country per max short
  \EUobjective{#1}{#6}{prov\EUAbbr{ince}-#2}{#4}{#5}{%
    \EU@objEachProvinceOtherText{#3}, above #2}}%
\newcommand{\EU@objCountryExists}[3][]{% M? country VPs
  \EUobjective{#1}{\pays{#2} exists}{}{}{#3}{%
    \pays{#2} exists and owns at least one province; full success if it was
    destroyed and later recreated}}%

\newcommand{\EU@objNoRUSBaltic}[2][]{% M? full
  \EUobjective{#1}{\paysmajeurRussie
    contained}{}{\EU@objhalf{#2}}{#2}{\paysmajeurRussie does not own any
    province bordering the \regionBaltique; only half-success if
    \paysmajeurRussie controls a province bordering the \regionBaltique}}%
\newcommand{\EU@objSpecificProvince}[4][]{% M? prov VP short
  \EUobjective{#1}{#4}{}{\EU@objhalf{#3}}{#3}{\province{#2} is owned and
    controlled by \EUcurrmaj; half-success if only controlled or if owned but
    not controlled}}%
\newcommand{\EU@objCountryDestroyed}[4][]{% M? country VPs short?
  \EUobjective{#1}{#4 of \pays{#2}}{}{}{#3}{\pays{#2} does not exist any more
    ; failure if it was destroyed and recreated later}}%
\newcommand{\EU@objCountryConquest}[3][]{% M? country VPs
  \EU@objCountryDestroyed[#1]{#2}{#3}{Conquest}}
\newcommand{\EU@objNoProvincesLost}[2][]{% M? VPs
  \EUobjective{#1}{No provinces lost}{}{}{#2}{}}%
\newcommand{\EU@objSwedenContainedFull}[2][]{% M? full
% Seem to be always with the same number of provinces...
  \EUobjective{#1}{\paysmajeurSuede contained}{}{\EU@objhalf{#2}}{#2}{%
    Number of provinces owned by \paysmajeurSuede adjacent to the
    \regionBaltique (\regionSuede and \regionFinlande excepted); 3 or less:
    success; 4: half-success; 5 or more: failure}}%
% (jcd) Probably remove \regionNorvege too?
% (jym) Nope. \regionNorvege is not adjacent to the \regionBaltic...
\newcommand{\EU@objSwedenContained}{
% Actually, seem to be always with no malus and the same VPs...
  \EU@objSwedenContainedFull{40}}

%%% Specific Wars
\newcommand{\EU@objLouisXIV}[3][]{% M? half full
  \EUobjective{}{Louis XIV's wars}{war}{#2}{#3}{For each victory
    in either \ref{pV:Devolution War}, \ref{pV:Chamber of Reunion} or
    \ref{pV:League Augsburg}. If none of these events occur, half-success
    (#2 \VPs) ; If \FRA refuses the war after the event is rolled, it counts as
    a defeat for \FRA and a victory for all others}}%

% always used with the same values (HIS has a special one):
\newcommand{\EU@objWoSSfull}[2][]{% M? VP
  \EUobjective{#1}{Spanish Succession}{}{\EU@objhalf{#2}}{#2}{Either
    \EUcurrmaj is victorious in \ref{pV:WoSS}, or there is no war and
    \EUcurrmaj received part of the inheritance. Half-success if the event
    does not occur. Failure if there is no war and \EUcurrmaj received no part
    of the inheritance or if \EUcurrmaj was on the loosing side of the war}}%
\newcommand{\EU@objWoSS}{\EU@objWoSSfull{50}}%
\newcommand{\EU@objPOLVictoryText}{\EUcurrmaj is protector of \paysPologne
  (permanent \EW); half-success if \ref{pVI:WoPS} never occured}%
\newcommand{\EU@objSYW}[2][]{% M? VP
  \EUobjective{#1}{Seven Years War}{}{\EU@objhalf{#2}}{#2}{\EUcurrmaj
    victorious in \ref{pVII:Seven Years War} (signs a peace of level \geq
    1). Half-success if the event does not occur, failure if the event occur
    but \EUcurrmaj is not part of the war}}%
\newcommand{\EU@objIndependanceWars}[2][]{% M? VP
  \EUobjective{#1}{Independence Wars}{}{\EU@objhalf{#2}}{#2}{Victory in
    \ref{pVII:Independence War}
    (see~\ref{chVictories:Explanation:Independence})}}%
\newcommand{\EU@objFranceRoyalist}[2][]{% M? VP
  \EUobjective{#1}{\FRA Royalist}{}{\EU@objhalf{#2}}{#2}{French Revolution is
    crushed and the King is back on the throne. Half success if
    \ref{pVII:French Revolution} does not occur}}%
% Was:
% {\paysmajeurFrance
%     does not own all its ``natural frontiers'' during the Revolution
%     (\ref{pVII:Revolution:Natural Frontiers}); half-success if the event does
%     not occur. Cannot be the main objective.}}%


%% Country specific
%%% ANG
\newcommand{\EU@objEcosseVassal}[3][]{% M? per_turn max
  \EUobjective{#1}{\paysEcosse Vassal}{turn}{#2}{#3}{Each turn \paysEcosse is
    \VASSAL of \paysmajeurAngleterre}}%
\newcommand{\EU@objIrelandPacified}[3][]{% M? per_turn max
  \EUobjective{#1}{Pacified Ireland}{turn}{#2}{#3}{Each turn without any
    \REVOLT in provinces owned by \paysmajeurAngleterre in \regionIrlande}}%
% always used with the same values:
\newcommand{\EU@objMonopolyZoneANGfull}[4][]{% M? STZ CTZ max
  \EUobjective{#1}{Trade Monopoly}{zone}{#2/#3}{#4}{Each English Commercial
    \terme{Monopoly} in any \STZ (#2)/\CTZ (#3)}}%
\newcommand{\EU@objMonopolyZoneANG}{\EU@objMonopolyZoneANGfull{5}{10}{30}}
\newcommand{\EU@objCalaisANG}[2][]{% M? VPs
  \EUobjective{#1}{Calais}{}{}{#2}{\paysmajeurAngleterre owns
    \provincePicardie or a \Presidio in it}}%
\newcommand{\EU@objNoConversionText}[1]{\paysmajeur{#1} was not forced to
  change its religion}%
\newcommand{\EU@objNoConversion}[2][]{% M? VPs
  \EUobjective{#1}{No forced conversion}{}{}{#2}{%
    \EU@objNoConversionText{\EUcurrcoun}}}
% Must keep country as arg because of cross objectives ANG/HOL.
\newcommand{\EU@objNoConversionOther}[4][]{% M? VPs country nationality
  \EUobjective{#1}{No forced #4 conversion}{}{}{#2}{%
    \EU@objNoConversionText{#3}}}
\newcommand{\EU@objCaraibesANG}[4][]{% M? ctrl own max
  \EUobjective{#1}{\continentCaraibes}{island}{#2/#3}{#4}{Each \TP or \COL
    controlled (#2)/owned and controlled (#3) in \continentCaraibes}}%

%%% FRA
\newcommand{\EU@objCalaisFRA}[2][]{% M? VP
  \EUobjective{#1}{Calais}{}{}{#2}{\paysmajeurFrance owns \provincePicardie
    and there is no \Presidio in it}}%
\newcommand{\EU@objArtoisFRA}[1][]{% M?
  \EU@objSpecificProvince[#1]{Artois}{30}{French \provinceArtois}}%


%%% HAB
\newcommand{\EU@objHalfHungary}[2][]{% M? VPs
  \EUobjective{#1}{Half of \payshongrie}{}{}{#2}{At least half of the provinces
    of \paysHongrie are owned by either \payshongrie or \AUSaus (\geq 6
    provinces)}}%
\newcommand{\EU@objNoTURHungaryFull}[4][]{% M? max minus compl
  \EUobjective{#1}{\payshongrie}{\TUR prov\EUAbbr{ince}}{#2-#3}{#2}{#2 \VPs
    minus #3 \VPs per province of \paysHongrie owned by a non-Christian
    country or a Turkish minor ally#4}}%
\newcommand{\EU@objNoTURHungary}[3][]{% M? max minus
  \EU@objNoTURHungaryFull[#1]{#2}{#3}{}}
% includes Christian Transylvanie as TUR Vassal, and Neutral Crimea.
\newcommand{\EU@objGermanEmpire}[2][]{% M? full
  \EUobjective{#1}{\pays{German Empire}}{}{\EU@objhalf{#2}}{#2}{Creation of
    \pays{German Empire}; if it was created and destroyed later, still counts
    as a success. Half-success if \ref{pIV:TYW} did not occur}}%
\newcommand{\EU@objBigAustria}[3][]{% M? per max
  \EU@objEachProvinceAbove[#1]{11}{\paysmajeurAutriche}%
  {#2}{#3}{\paysmajeurAutriche}}%
\newcommand{\EU@objSpanishNetherlands}[3][]{% M? per max
  \EUobjective{#1}{Low Countries}{prov\EUAbbr{ince}}{#2}{#3}{
  \EU@objEachProvinceOtherText{\paysmajeurEspagne}, in \regionBelgique}}%

%%% HIS
\newcommand{\EU@objNoFrenchVassal}[2][]{% M? VPs
  \EUobjective{#1}{No \FRA \VASSAL in \regionItalie}{}{}{#2}{No minor of
    \regionItalie is \VASSAL of \paysmajeurFrance}}%
\newcommand{\EU@objNoFrenchItaly}[2][]{% M? VPs
  \EUobjective{#1}{\FRA not in \regionItalie}{}{}{#2}{\paysmajeurFrance owns no
    provinces in \regionItalie}}%
\newcommand{\EU@objPresidiosHIS}[3][]{% M? per max
  \EUobjective{#1}{Barbary Coast}{\Presidio}{#2}{#3}{Each Christian \Presidio
    in a province owned by \Barbaresques}}%
\newcommand{\EU@objSpanishWorld}[2][]{% M? full
  \EUobjective{#1}{Spanish World}{}{\EU@objhalf{#2}}{#2}{No non-Spanish \COL
    in \continent{South America} and \continentCaraibes; half-success if no
    non-Spanish \COL in \continent{South America} and \granderegionFlorida}}%
\newcommand{\EU@objPortugalAnnexed}[2][]{% M? VPs
  \EUobjective{#1}{Annexed \paysportugal}{}{}{#2}{\paysportugal in \ANNEXION
    status}}%

%%% HOL
\newcommand{\EU@objNoActNavigation}[2][]{% M? full
  \EUobjective{#1}{Abolition Act of Navigation}{}{\EU@objhalf{#2}}{#2}{Act of
    Navigation (\ref{pIV:Act Navigation}) is not in effect; half-success if
    the event did not occur}}%
\newcommand{\EU@objEastMalaccaText}{Christian non-Dutch \TP east of
  \granderegionMalacca (\continent{Extreme Orient}, \continentIndonesia,
  \granderegionMalacca, \granderegionAyutthaya and \granderegion{Dai Viet})}
\newcommand{\EU@objEastMalacca}[4][]{% M? per full compl
  \testempty{#2}{%
    \EUobjective{#1}{East of \granderegionMalacca}{}
    {#2}{#3}{#4 \EU@objEastMalaccaText}}{%
    \EUobjective{#1}{East of \granderegionMalacca}{\TP}
    {#2}{#3}{#4 \EU@objEastMalaccaText}}}%
\newcommand{\EU@objAmericaHOL}[2]{% per max
  \EU@objEachCOL[in \continentAmerica (\continentBrazil
  excepted)]{\continentAmerica}{#1}{#2}}

%%% POL
\newcommand{\EU@objCountryExistsNoTUR}[3][]{% M? country VPs
  \EUobjective{#1}{\pays{#2} exists}{}{}{#3}{%
    \pays{#2} exists, owns at least one province and is not a \VASSAL of
    \paysmajeurTurquie; full success if it was destroyed and later
    recreated}}%

%%% RUS
\newcommand{\EU@objPorts}[4][]{% M? per max region
  \EUobjective{#1}{Ports on \region{#4}}{port}{#2}{#3}{Each port bordering the
    \region{#4} owned by \EUcurrmaj}}%
\newcommand{\EU@objEachAnnexed}[6][]{% M? short per max from compl?
  \EUobjective{#1}{#2 Annexations}{prov\EUAbbr{ince}*}{#3}{#4}{%
  Each province taken from #5\ifnotempty{#6}{#6}}}%
\newcommand{\EU@objEachOwned}[6][]{% M? short per max from above?
  \EUobjective{#1}{#2}{prov\EUAbbr{ince}\ifnotempty{#6}{-#6}}{#3}{#4}{%
    Each province of #5 owned by \EUcurrmaj\ifnotempty{#6}{ above #6}}}%


%%% SUE
\newcommand{\EU@objDMB}[2][]{% M? VPs
  \EUobjective{#1}{DMB}{}{}{#2}{\paysmajeurSuede has the \terme{Dominium Marii
      Baltici}}}%

%%% TUR
\newcommand{\EU@objPresidiosTUR}[3][]{% M? max penalty
  \EUobjective{#1}{Barbary Coast}{\Presidio}{#2-#3}{#2}{#2\VPs, minus #3 \VPs
    for each Christian \Presidio in any provinces with the shield of any
    \Barbaresques}}%
\newcommand{\EU@objSpiceTUR}[3][]{% M? per max
  \EU@objEachItem[#1]{producing \POSPICE; establishments of minors in \dipAT
  providing their resources count}{\COL or \TP}{\POSPICE}{#2}{#3}}
\newcommand{\EU@objRhodos}[2][]{% M? VPs
  \EU@objSpecificProvince[#1]{Rhodos}{#2}{\provinceRhodos}}%
\newcommand{\EU@objViennaFallen}[2][]{% M? VPs
  \EUobjective{#1}{\villeVienne fallen}{}{}{#2}{%
    \provinceOsterreich/\villeVienne was captured during this period by
    \paysmajeurTurquie}}%
\newcommand{\EU@objValMol}[2][]{% M? both
  \EUobjective{#1}{\paysValachie/\paysMoldavie}{}{\EU@objhalf{#2}}{#2}{%
    Each of \paysValachie and \paysMoldavie is either \VASSAL of \EUcurrmaj or
    conquered (all of its provinces are owned by \EUcurrmaj, whoever controls
    them). Half-success if only one of the two is \VASSAL or conquered}}%
\newcommand{\EU@objCrimeaTURFull}[8][]{% M? VPa VPb VPc a+ b c d-
  \EUobjective{#1}{Defence of \payscrimee}{}{#4/#3}{#2}{Number of provinces
    owned by \payscrimee; #5 or more: success; #6: #3\VPs; #7: #4\VPs; #8 or
    less: failure}}%
% always same VPs and decreasing number of provinces
\newcommand{\EU@objCrimeaTUR}[2][]{% M? a+
  \EU@objCrimeaTURFull[#1]{40}{30}{20}{#2}{\the\numexpr#2-1\relax}{\the\numexpr#2-2\relax}{\the\numexpr#2-3\relax}}
\newcommand{\EU@objBalkansTUR}[4][]{% M? per max above?
  \EUobjective{#1}{\regionBalkans}{prov\EUAbbr{ince}\ifnotempty{#4}{-#4}}
  {#2}{#3}{Each province of \regionBalkans owned by \paysmajeurTurquie
    \ifnotempty{#4}{ above #4}, whoever controls it}}%

%%% VEN
\newcommand{\EU@objItaliSanMarco}[3][]{% M? per max
  \EUobjective{#1}{Italia e San Marco}{\MIN}{#2}{#3}{%
    Each Italian minor in \EG or better (2 provinces of \pays{naples} owned
    count as 1 minor for this purpose)}}%
\newcommand{\EU@objBalkansVEN}[4][]{% M? per max except?
  \EUobjective{#1}{Balkans}{prov\EUAbbr{ince}}{#2}{#3}{%
    Each province in \regionBalkans \ifnotempty{#4}{(except #4)} owned by
    \paysmajeurVenise, whoever controls it}}%
\newcommand{\EU@objOrientIncome}[3][]{% M? income VPs
  \EUobjective{#1}{Orient income\geq #2\ducats}{}{}{#3}{%
    Total Orient Income of the period is #2\ducats or more}}%

%%% POR
\newcommand{\EU@objMarocVassal}[3][]{% M? per_turn max
  \EUobjective{#1}{\paysMaroc Vassal}{turn}{#2}{#3}{Each turn \paysMaroc is
    \VASSAL of \paysmajeurPortugal}}%
\newcommand{\EU@objMarocAnnexion}[2][]{% M? VPs
  \EUobjective{#1}{Annexation in \paysmaroc}{}{}{#2}{At least one province
    annexed from \paysMaroc during the period}}%
\newcommand{\EU@objMonopolyPOR}[3][]{% M? min VPs
  \EUobjective{#1}{Trade monopolies\geq #2}{}{}{#3}{At least #2 \CTZ/\STZ
    monopolies}}%


%%%%%%%%%%%%%%%%%%%%%%%%%%%%%%%%%%%%%%%%%%%%%%%%%%%%%%%%%%%%%%
\EUcurrentcountry{Angleterre}
%%%%%%%%%%%%%%%%%%%%%%%%%%%%%%%%%%%%%%%%%%%%%%%%%%%%%%%%%%%%%%
%
\EUcurrentperiod{I}%
\EU@objEcosseVassal{10}{30}%
\EU@objCalaisANG{45}%
\EU@objIndustrial{25}%
%\EU@objSpecificProvince[M]{Guyenne}{50}{Hundred years war}%
\EUobjective{M}{Hundred years war}{occupation}{25}{50}{[BLP] Each occupation
  marker in French national provinces ; or full success if \provinceGuyenne is
  owned, no matter who controls it}
\EU@objIrelandPacified{10}{40}%
%
\EUcurrentperiod{II}%
\EU@objEcosseVassal{10}{40}%
\EU@objCalaisANG[M]{45}%
\EUobjective{}{\COL in \continentAmerica}{}{}{30}{%
  \paysmajeurAngleterre has at least one \COL in \continentAmerica}%
\EUobjective{}{Victory against \FRA}{}{}{40}{%
  \paysmajeurAngleterre has signed at least one peace of level \geq 2 against
  \paysmajeurFrance this period}%
\EU@objIrelandPacified{7}{35}%
%
\EUcurrentperiod{III}%
\EU@objEcosseVassal{5}{30}%
\EU@objNoConversion[M]{45}%
\EU@objIndustrial{30}%
\EU@objNoConversionOther{35}{Hollande}{Dutch}% yes, \HOL this is not an error
\EU@objMonopolyZoneANG%
%
\EUcurrentperiod{IV}%
\EU@objEcosseVassal{5}{30}%
\EU@objNoConversion[M]{40}%
\EUobjective{}{Commercial Centre}{}{}{50}{%
  \paysmajeurAngleterre owns at least one \terme{Commercial Centre}}%
\EUobjective{}{English Civil War}{}{20}{40}{%
  Duration of \ref{pIV:English Civil War}; four turns or less: full success;
  five turns or no event: half-success; six turns or more: failure}%
\EU@objMonopolyZoneANG%
%
\EUcurrentperiod{V}%
\EU@objCaraibesANG{7}{15}{45}%
\EU@objNoConversion{40}%
\EU@objTradeCentre[M]{Atlantic}{40}%
\EUobjective{}{Independent Portugal}{}{15}{30}{%
  \paysPortugal in not in \ANNEXION of \paysmajeurEspagne; half success if
  neither \ref{pIV:Portuguese Revolt} nor \ref{pV:WoSS} occured}%
\EU@objMonopolyRES{Fish}{5}{40}%
%
\EUcurrentperiod{VI}%
\EU@objCaraibesANG{5}{10}{50}%
\EUobjective{}{Jacobite rebellion}{war}{15}{30}{%
  Each victory in \ref{pVI:Jacobite Rebellion} (15 \VPs if none occur)}%
>>>>>>> comptav2
% (jcd) I added the second condition. Original was unclear.
\EU@objTradeCentre[M]{Atlantic}{40}%
\EU@objWoSS%
\EU@objEastIndiesConvoy{4}{40}%
%
\EUcurrentperiod{VII}%
\EU@objIndependanceWars[M]{50}%
\EU@objFranceRoyalist{30}%
\EU@objEachCC{15}{45}%
\EU@objSYW{30}%
\EU@objEastIndiesConvoy{4}{40}%
%
%%%%%%%%%%%%%%%%%%%%%%%%%%%%%%%%%%%%%%%%%%%%%%%%%%%%%%%%%%%%%%
\EUcurrentcountry{France}
%%%%%%%%%%%%%%%%%%%%%%%%%%%%%%%%%%%%%%%%%%%%%%%%%%%%%%%%%%%%%%
%
\EUcurrentperiod{I}%
\EU@objCalaisFRA{40}%
\EU@objArtoisFRA%
\EU@objSDCF{40}%
\EU@objIndustrial{25}%
\EUobjective{M}{War in Italy}{}{25}{50}{%
  \paysmajeurFrance signed a peace of level \geq 2 in at least one War in
  Italy. Half-success if none occur}%
%
\EUcurrentperiod{II}%
\EU@objCalaisFRA[M]{50}%
\EU@objArtoisFRA%
\EU@objEmperor{French}{50}%
\EU@objIndustrial{40}%
\EUobjective{}{\provinceLombardia or \provinceCampania}{}{15}{30}{%
  One of these provinces is owned by \paysmajeurFrance. Half-success if no War
  in Italy occurred since the beginning of the game}%
%
\EUcurrentperiod{III}%
\EU@objNoProvincesLost{30}%
\EU@objEachProvinceList{\provinceArtois, \provinceBresse,
  \province{Franche-Comte} and \provinceRoussillon}{10}{40}{Specific
  possessions}%
\EUobjective{}{No change of religion}{}{}{50}{%
  \paysmajeurFrance did not change religion, except due to the use of
  \ref{pIII:FWR:Hospital}}%
\EUobjective{}{\geq 3 \COL/\TP}{}{}{30}{%
  \paysmajeurFrance owns at least 3 \COL or \TP}%
\EUobjective{M}{\leq 2 unfavourable truces}{}{}{40}{%
  \paysmajeurFrance does not sign more than 2 unfavourable truces during
  \ref{pIII:FWR}}%
%
\EUcurrentperiod{IV}%
\EUobjective{}{\ctz{France} Monopoly}{}{}{30}{%
  \paysmajeurFrance has a Commercial Monopoly in \ctz{France}}%
\EUobjective{}{Trade Monopoly}{zone}{15/20}{40}{%
  Each partial (15)/total (20) Commercial Monopoly in any \STZ/\CTZ except
  \ctz{France}}%
\EUobjective{M}{No \GE/Southern \HRE}{}{25}{50}{%
  Neither \GE nor the Southern \HRE alliance exists. Success if one was
  created but is now destroyed. Half-success if \ref{pIV:TYW} never occured}%
\EU@objEachCOL{}{5}{30}%
\EUobjective{}{No Northern \HRE}{}{20}{40}{%
  There is no Northern \HRE alliance. Success if it was created but is now
  destroyed. Half-success if \ref{pIV:TYW} never occured}%
%
\EUcurrentperiod{V}%
\EU@objTradeCentre{Mediterranee}{40}%
\EUobjective{M}{Adjacent to \HOL}{}{}{40}{%
  \paysmajeurFrance owns at least one province adjacent to \paysmajeurHollande
  national territory}%
\EUobjective{}{Glorious Revolution}{}{20}{40}{%
  During \ref{pV:Glorious Revolution}, \paysmajeurFrance wins after either
  controlling the Rebels or making an intervention; half-success if the event
  does not occur ; failure if \paysmajeurFrance neither controls the Rebels
  nor intervene}%
\EU@objMonopolyRES{Fish}{5}{30}%
\EU@objLouisXIV{20}{40}%
%
\EUcurrentperiod{VI}%
\EU@objTradeCentre{Mediterranee}{40}%
\EU@objWoSS%
\EUobjective{}{Austrian Succession}{}{20}{40}{%
  \paysmajeurFrance is victorious in \ref{pVI:WoAS} without any territorial
  gain. Half-success if the event does not occur}%
\EU@objTradeCentre[M]{Atlantic}{50}%
\EUobjective{}{Polish Succession+\FRA}{}{25}{50}{%
  \EU@objPOLVictoryText; half-success if \paysmajeurSuede is protector of
  \paysPologne}%
%
\EUcurrentperiod{VII}%
\EU@objCountryExists[M]{Pologne}{50}%
\EU@objEachCC{20}{40}%
\EU@objIndependanceWars{45}%
\EUobjective{}{Colonial expansion}{}{}{40}{%
  \paysmajeurFrance has more \TP plus \COL than any other country in one of
  \continent{India}, \continent{North America}, or \continent{Caraibes}}%
\EUobjective{}{Natural frontiers}{}{15}{30}{%
  \paysmajeurFrance owns all its ``natural frontier'' during the Revolution
  (\ref{pVII:Revolution:Natural Frontiers}); half-success if the event does
  not occur}%
%
%%%%%%%%%%%%%%%%%%%%%%%%%%%%%%%%%%%%%%%%%%%%%%%%%%%%%%%%%%%%%%
\EUcurrentcountry{Espagne}
%%%%%%%%%%%%%%%%%%%%%%%%%%%%%%%%%%%%%%%%%%%%%%%%%%%%%%%%%%%%%%
%
\EUcurrentperiod{I}%
\EU@objNoFrenchVassal{50}%
\EU@objNoFrenchItaly{40}%
\EU@objSDCF{25}%
\EU@objPresidiosHIS[M]{15}{50}%
\EUobjective{}{\paysprovincesne annexed}{}{}{25}{%
  All provinces of \paysprovincesne are owned by \paysmajeurEspagne}%
%
\EUcurrentperiod{II}%
\EU@objNoFrenchVassal{40}%
\EU@objNoFrenchItaly{30}%
\EU@objEmperor{Spanish}{50}%
\EU@objPresidiosHIS[M]{15}{50}%
\EUobjective{}{Religious calm in \HRE}{}{25}{50}{%
  Full success if either \ref{pI:Reformation2} did not occur; or Schmalkaldic
  league destroyed without religious liberty (\ref{pII:Schmalkaldic League})
  and \ref{pIV:TYW} hasn't occured yet; or \ref{pIV:TYW} won (dominant
  position at the final peace). Half success if \ref{pII:Schmalkaldic League}
  did not occur}%
%
\EUcurrentperiod{III}%
\EU@objHalfHungary{50}%
\EUobjective{M}{Forced conversion}{\MAJ}{50}{--}{%
  Each time a Protestant Major Country is converted due to a war in which
  \paysmajeurEspagne was fighting against it. Not possible if
  \paysmajeurEspagne is \CATHCO; Religious and Civil wars count; Limited and
  foreign interventions of \SPA count. No maximum \VPs value}%
\EUobjective{}{\TUR stopped in Ionian}{}{}{40}{%
  No Turkish possesions nor \VASSAL west of \seazoneIonienne (\provinceTripoli
  excluded, \provinceMalta or any province in \regionItalie included)}%
%\EUobjective{}{Habsburg-Sultan peace}{}{}{30}{%
%  No more than one turn of war between \paysHabsbourg and \paysmajeurTurquie}%
\EUobjective{}{Habsburg-Sultan peace}{}{}{30}{%
  [BLP] 40 -5/turn where \ref{chSpecific:Little war} is active (max 30); or
  full success (30\VPs) if no \TUR Occupation in \paysHongrie}%
\EUobjective{}{Trade expansion}{}{}{30}{%
  Monopoly in \ctz{Espagne} and in at least 2 other \CTZ/\STZ}%
%
\EUcurrentperiod{IV}%
\EU@objHalfHungary{40}%
\EU@objGermanEmpire[M]{50}%
\EU@objBigAustria{10}{30}%
\EU@objPortugalAnnexed{50}%
\EU@objMonopolyZoneFull{8}{40}{; \ctz{Espagne} counts as two}%
%
\EUcurrentperiod{V}%
\EU@objNoTURHungary[M]{40}{10}%
\EU@objSpanishNetherlands{10}{50}%
\EU@objSpanishWorld{40}%
\EU@objPortugalAnnexed{40}%
\EU@objMonopolyZone{8}{40}%
%
\EUcurrentperiod{VI}%
\EUobjective{M}{Spanish Succession}{}{}{50}{%
  Victory of \paysmajeurEspagne in \ref{pV:WoSS}; full success if the event
  occurs but there is no war}%
\EU@objOuterProvinces{10}{40}%
\EU@objSpanishWorld{45}%
\EU@objPresidiosHIS{10}{50}%
% (jcd) What about annexations of provinces? Vassals?
% (jym) they're not presidios... We don't want HIS to go full berserk against
% Algeria.
\EU@objMonopolyZone{8}{40}%
%
\EUcurrentperiod{VII}%
\EUobjective{}{Spanish Asiento}{}{}{35}{%
  \paysmajeurEspagne has a commercial policy of Exclusive
  Asiento (\ref{chSpecific:Spain:Asiento})}%
\EU@objOuterProvincesAbove{10}{40}{3}%
\EU@objSpanishWorld[M]{50}%
\EU@objIndustrial{25}%
\EU@objIndependanceWars{30}%
%
%%%%%%%%%%%%%%%%%%%%%%%%%%%%%%%%%%%%%%%%%%%%%%%%%%%%%%%%%%%%%%
\EUcurrentcountry{Hollande}
%%%%%%%%%%%%%%%%%%%%%%%%%%%%%%%%%%%%%%%%%%%%%%%%%%%%%%%%%%%%%%
%
\EUcurrentperiod{III}%
\EU@objTradeCentre{Mediterranee}{40}%
\EUobjective{}{Recognition of Independence}{}{}{50}{%
  Dutch Independence recognised by Spain}%
\EUobjective{}{Protestant \paysmajeurFrance}{}{}{35}{%
  \paysmajeurFrance is Protestant}%
\EUobjective{}{Protestant \paysmajeurAngleterre}{}{}{40}{%
  \paysmajeurAngleterre is Protestant}%
\EU@objEastIndiesConvoy{10}{40}%
%
\EUcurrentperiod{IV}%
\EU@objTradeCentre{Mediterranee}{40}%
\EU@objBalticTrade{5}{30}%
\EUobjective{}{\payshanse/Northern \HRE}{}{}{40}{%
  Either \ref{pIV:TYW} occured and there is a Northern \HRE alliance; or
  \payshanse still exists and is \VASSAL of \paysmajeurHollande (including
  after \ref{pIV:TYW})}%
\EU@objNoActNavigation[M]{40}%
\EU@objEachCOL[in \continentBrazil]{\continentBrazil}{10}{40}%
%
\EUcurrentperiod{V}%
\EU@objTradeCentre[M]{Atlantic}{40}%
\EUobjective{}{Safe from \paysmajeurFrance}{}{}{50}{%
  \paysmajeurFrance does not own any province adjacent to \paysmajeurHollande
  national territory}%
\EU@objEastMalacca{}{40}{No}% Nobody
\EU@objNoActNavigation{50}%
\EU@objMonopolyRES{Spices}{5}{40}%
%
\EUcurrentperiod{VI}%
\EU@objEachCC[M]{25}{50}%
\EU@objWoSS%
\EU@objEastMalacca{40-10}{40}{Remove 10 \VPs for each}%
\EU@objAmericaHOL{15}{45}%
\EU@objEastIndiesConvoy{4}{40}%
%
\EUcurrentperiod{VII}%
\EU@objMonopolyZone{5}{30}%
\EUobjective{M}{Batavian Revolution}{rev\EUAbbr{olution}}{20}{40}{%
  Per victory in \ref{pVII:Batavian Revolution}; half-success if the even
  never occur}%
\EU@objFranceRoyalist{35}%
\EU@objAmericaHOL{15}{45}%
\EU@objEachCOLTP[in \continentIndia]{\continentIndia}{15}{50}%
%
%%%%%%%%%%%%%%%%%%%%%%%%%%%%%%%%%%%%%%%%%%%%%%%%%%%%%%%%%%%%%%
\EUcurrentcountry{Autriche}
%%%%%%%%%%%%%%%%%%%%%%%%%%%%%%%%%%%%%%%%%%%%%%%%%%%%%%%%%%%%%%
%
\EUcurrentperiod{IV}%
\EU@objHalfHungary{40}%
\EU@objGermanEmpire[M]{50}%
\EU@objBigAustria{10}{30}%
\EUobjective{}{To the \regionBaltique}{turn}{15}{45}{%
  Each turn where a port on the \regionBaltique or in \payshanse is controlled
  during the Inter-phase}%
\EUobjective{}{Southern \HRE}{}{15}{30}{%
  Southern \HRE alliance exists; half-success if~\ref{pIV:TYW} did not occur}%
%
\EUcurrentperiod{V}%
\EU@objNoTURHungary{50}{10}%
\EU@objSpanishNetherlands[M]{10}{40}%
\EUobjective{}{Defending \paysvenise}{prov\EUAbbr{ince}*-2}{20}{50}{%
  Possessions of \pays{Venise}: Each Mediterranean island or province in
  \region{Balkans}, above 2}%
\EU@objLouisXIV{20}{40}%
\EUobjective{}{Absolutist \paysmajeurPologne}{}{}{25}{%
  \paysmajeurPologne is absolutist (either \ref{pIV:Liberum Veto} did not
  happen or \ref{pIV:Polish Civil War} was won by the Absolutists)}%
%
\EUcurrentperiod{VI}%
\EU@objNoTURHungary{50}{20}%
\EU@objWoSS%
\EUobjective{}{Austrian Succession}{}{20}{40}{%
  \AUS is victorious in \ref{pVI:WoAS}. Half-success if it did not occur}%
\EUobjective{}{No Royal \paysmajeurPrusse}{}{}{30}{%
  \paysmajeurPrusse has not received the Royal Dignity}%
\EU@objSilesie[M]{50}%
%
\EUcurrentperiod{VII}%
\EUobjective{}{H\EUAbbr{ungaria}/\provinceBosna/\provinceSerbia}{%
  prov\EUAbbr{ince}*}{20}{50}{No Turkish provinces in \payshongrie,
  \provinceBosna and \provinceSerbia; each of \payshongrie and the two
  provinces counts as one item (20\VPs)}%
\EU@objEachProvinceAbove[M]{3}{\paysNaples}{10}{50}{\paysNaples}%
\EU@objAllianceAny{5}{30}%
\EU@objFranceRoyalist{30}%
\EU@objSilesie{50}%
%
%%%%%%%%%%%%%%%%%%%%%%%%%%%%%%%%%%%%%%%%%%%%%%%%%%%%%%%%%%%%%%
\EUcurrentcountry{Prusse}
%%%%%%%%%%%%%%%%%%%%%%%%%%%%%%%%%%%%%%%%%%%%%%%%%%%%%%%%%%%%%%
%
\EUcurrentperiod{VI}%
\EU@objEachProvinceAbove{9}{\EUcurrmaj}{10}{40}{Expansion}%
\EU@objAllianceAny{5}{40}%
\EUobjective{}{Royal \paysmajeurPrusse}{}%
{}{30}{\paysmajeurPrusse has received the Royal Dignity}%
\EUobjective{}{No Austrian Emperor}{}%
{20}{40}{\paysmajeurAutriche has lost the imperial throne; half-success if
  \ref{pVI:WoAS} did not occur}%
\EU@objSilesie[M]{50}%
%
\EUcurrentperiod{VII}%
\EUobjective{}{Further expansion}{prov\EUAbbr{ince}}%
{15}{50}{15 \VPs per province annexed during the period}%
\EU@objAllianceAny{5}{40}%
\EU@objSYW{40}%
\EU@objFranceRoyalist{30}%
\EU@objSilesie[M]{50}%
%
%%%%%%%%%%%%%%%%%%%%%%%%%%%%%%%%%%%%%%%%%%%%%%%%%%%%%%%%%%%%%%
\EUcurrentcountry{Suede}
%%%%%%%%%%%%%%%%%%%%%%%%%%%%%%%%%%%%%%%%%%%%%%%%%%%%%%%%%%%%%%
%
\EUcurrentperiod{III}%
\EU@objBalticTrade{5}{30}%
% (jcd) Should be M ?
% (jym) Probably not.
\EU@objEachCOLTP[in \continentAmerica]{\continentAmerica}{10}{30}%
\EU@objEachProvinceList{\provinceNeva, \provinceLivonija and
  \provinceEstland}{20}{50}{Livonian Annexations}%
\EU@objEachProvinceList{\provinceSkane, \provinceVastergotland,
  \provinceGotland or any of \regionNorvege}{10}{40}{Swedish Annexations}%
\EUobjective{}{No Polish Claim}{}{25}{50}{%
  \paysmajeurPologne has renounced to its claim to the throne of Sweden (given
  by \ref{pIII:Union Poland Sweden}); half-success if the event did not
  occur. If \paysmajeurPologne is protestant or Supporter of the Orthodoxy,
  this objective cannot be chosen}%
%
\EUcurrentperiod{IV}%
\EU@objBalticTrade{5}{30}%
\EUobjective{M}{No \pays{German Empire}}{}{20}{40}{%
  \GE does not exist; success if it was created and is now
  destroyed. Half-success if \ref{pIV:TYW} did not occur}%
\EU@objCountryDestroyed{Hanse}{30}{Dissolution}%
\EU@objDMB{50}%
\EUobjective{}{No Polish King}{}{20}{40}{%
  King of \paysmajeur{Pologne} never was on the Swedish throne; half-success
  if \ref{pIII:Union Poland Sweden} did not occur. If \paysmajeurPologne is
  protestant or Supporter of the Orthodoxy, this objective cannot be chosen}%
%
\EUcurrentperiod{V}%
\EU@objBalticTrade{5}{30}%
\EU@objEachCOLTP{Colonisation}{5}{40}%
\EU@objNoRUSBaltic{30}%
\EU@objDMB[M]{50}%
\EUobjective{}{No Polish Absolutism}{}{}{30}{%
  \paysmajeurPologne is not absolutist (\ref{pIV:Liberum Veto} occured and
  there has been no absolutist victory in \ref{pIV:Polish Civil War})}%
%
\EUcurrentperiod{VI}%
\EU@objBalticTrade{5}{40}%
\EU@objEachCOLTP{Colonisation}{7}{40}%
\EUobjective{M}{Only \ville{Saint-Petersbourg} lost}{}{}{40}{%
  No province bordering the \regionBaltique lost during the period, except the
  one where \ville{Saint-Petersbourg} is built}%
\EU@objDMB{40}%
\EUobjective{}{Polish Succession+\SUE-\FRA}{}{25}{50}{%
  \EU@objPOLVictoryText; failure if \paysmajeurFrance is protector of
  \paysPologne}%
%
\EUcurrentperiod{VII}%
\EU@objBalticTrade{5}{40}%
\EU@objEachCOLTP{Colonisation}{10}{40}%
\EU@objNoProvincesLost{40}%
\EUobjective{}{Out of Scandinavia}{prov\EUAbbr{ince}}{15}{50}{%
  Per province not in \regionNorvege, \regionDanemark, \regionFinlande or
  \regionSuede owned by \paysmajeurSuede}%
\EU@objCountryExists{Pologne}{40}%
%
%%%%%%%%%%%%%%%%%%%%%%%%%%%%%%%%%%%%%%%%%%%%%%%%%%%%%%%%%%%%%%
\EUcurrentcountry{Russie}
%%%%%%%%%%%%%%%%%%%%%%%%%%%%%%%%%%%%%%%%%%%%%%%%%%%%%%%%%%%%%%
%
\EUcurrentperiod{I}%
\EUobjective{}{\payspskov/\paysryazan}{prov\EUAbbr{ince}}{20}{40}{%
  Per principality conquered (\payspskov or \paysryazan)}%
\EU@objSpecificProvince[M]{Smolenska}{50}{Russian \provinceSmolenska}%
\EU@objNoProvincesLost{25}%
\EUobjective{}{Conquest of one \terme{Khanate}}{}{}{50}{%
  At least one \terme{Khanate} (other than \paysSteppes) has been destroyed as a
  result of a war against \paysmajeurRussie; \terme{Khanates} are: \paysKazan,
  \paysAstrakhan, \paysCrimee and \paysCosaquesdon}%
\EU@objCountryConquest{Steppes}{35}%
%
\EUcurrentperiod{II}%
\EUobjective{}{Control of Orthodoxy}{\MIN}{10}{50}{%
  Each diplomatic control (or annexation) of orthodox \MIN. Destroyed
  countries with all provinces no owned by \RUS count toward this objective}%
\EUobjective{}{Nat\EUAbbr{ional}\ Territory and \provinceSmolenska}{}{}{40}{%
  \paysmajeurRussie owns all its national provinces and \provinceSmolenska}%
\EUobjective{M}{Forward to the \regionBaltique}{}{25}{50}{%
  \paysmajeurRussie owns and controls a port bordering the \regionBaltique;
  half-success if a port is only controlled or only owned}%
\EU@objCountryConquest{Kazan}{35}%
\EU@objCountryConquest{Astrakhan}{50}%
%
\EUcurrentperiod{III}%
\EU@objMonopolyRES{Furs}{10}{50}%
% (jcd) I added _total_ monopoly
% (jym) Removed. total is almost impossible due to the North American
% furs. Partial is already not completely trivial.
\EU@objEachAnnexed{Polish}{10}{40}{either \paysmajeurPologne,
  \paysmajeurLithuanie or \paysUkraine}{}%
\EU@objEachProvinceList{\provinceNeva, \provinceLivonija and
  \provinceEstland}{15}{45}{Livonian Annexations}%
\EU@objEachAnnexed[M]{Crimean}{20}{40}{\paysCrimee}{}%
\EU@objCountryConquest{Siberie}{35}%
%
\EUcurrentperiod{IV}%
\EU@objMonopolyRES{Furs}{5}{45}%
\EUobjective{}{National integrity}{}{15}{30}{%
  Number of Russian national provinces not owned by \paysmajeurRussie; 0 or 1:
  success; 2: half-success; 3 or more: failure}%
\EU@objSwedenContained
\EU@objPorts[M]{20}{50}{Noire}%
\EUobjective{}{Time of Troubles}{}{15}{30}{%
  \RUS victorious in \ref{pIV:Times of Troubles}; half-success if the event
  does not occur}%
%
\EUcurrentperiod{V}%
\EU@objMonopolyRES{Furs}{5}{35}%
\EU@objCountryConquest{Astrakhan}{40}%
\EUobjective{}{Building \ville{Saint-Petersbourg}}{}{}{50}{%
  Construction of \ville{Saint-Petersbourg} completed}%
\EU@objPorts{20}{50}{Noire}%
\EU@objEachCOLTP[in \granderegionAmour or \granderegionBaikal]{%
  \granderegionAmour/\granderegionBaikal}{10}{30}
%
\EUcurrentperiod{VI}%
\EU@objEachCOLTP[in \granderegionAmour, \granderegionBaikal,
    \granderegionAfghanistan, \granderegionPerse, or
    \continentIndia]{\emph{Silk Road}}{10}{40}%
\EUobjective{}{Polish Succession-\FRA-\SUE}{}{}{30}{%
  \paysPologne has no protector}%
\EU@objPorts{15}{40}{Baltique}%
\EU@objEachOwned[M]{\paysCrimee/\paysMoldavie}{15}{50}{either \paysCrimee or
  \paysMoldavie}{4}%
% (jcd) Does this mean at least 5 _new_ provinces or at least 5 _owned_
% provinces?
% (jym) only _owned_ makes some sense. There are 8 such provinces total.
\EU@objEachCOL[\granderegionAlaska]{}{10}{30}%
%
\EUcurrentperiod{VII}%
\EU@objEachOwned{\paysGeorgie/\paysPerse}{10}{40}{either \paysGeorgie or
  \paysPerse}{}%
\EU@objCountryConquest{Pologne}{50}%
\EU@objPorts{10}{40}{Baltique}%
\EU@objCountryConquest[M]{Crimee}{40}%
\EU@objEachAnnexed{Turkish}{15}{45}{\paysmajeurTurquie}{(excepted former
  provinces of \paysgeorgie or \paysperse)}%
%
%%%%%%%%%%%%%%%%%%%%%%%%%%%%%%%%%%%%%%%%%%%%%%%%%%%%%%%%%%%%%%
\EUcurrentcountry{Pologne}
%%%%%%%%%%%%%%%%%%%%%%%%%%%%%%%%%%%%%%%%%%%%%%%%%%%%%%%%%%%%%%
%
\EUcurrentperiod{I}%
\EU@objSpecificProvince{Smolenska}{40}{Polish \provinceSmolenska}%
\EUobjective{M}{At most \provinceSmolenska lost}{}{}{40}{%
  No province other than \provinceSmolenska lost}%
\EU@objCountryExists{Hongrie}{50}%
\EU@objCountryExistsNoTUR{Moldavie}{35}%
\EU@objCountryExistsNoTUR{Valachie}{35}%
%
\EUcurrentperiod{II}%
\EU@objSpecificProvince[M]{Smolenska}{50}{Polish \provinceSmolenska}%
\EU@objNoProvincesLost{40}%
\EU@objCountryExists{Hongrie}{50}%
\EU@objNoRUSBaltic{40}%
\EUobjective{}{Eastern expansion}{prov\EUAbbr{ince}-2}{20}{50}{%
  If \Xcatholique, each province above 2 in Polish Ukraynia is 20\VPs, max 50;
  if \Xorthodoxe, one \COL in \continentSiberia is a full success}%
%
\EUcurrentperiod{III}%
\EU@objEachProvinceList{\provinceKurland, \provinceMemel and
  \provincePreussen}{20}{50}{Baltic Annexations}%
\EU@objNoProvincesLost{40}%
\EUobjective{}{Union of Lublin}{}{15}{30}{%
  Union of Lublin is in effect; half-success if \ref{pIII:Union Lublin} did
  not occur}%
\EU@objNoRUSBaltic{40}%
\EUobjective{}{Polish Claim to \paysmajeurSuede}{}{20}{40}{%
  \POL has still its claim to the throne of Sweden (given by \ref{pIII:Union
    Poland Sweden}); half-success if the event did not occur}%
%
\EUcurrentperiod{IV}%
\EU@objSwedenContained
\EUobjective{M}{No non-Ukrainian provinces lost}{}{}{40}{%
  Provinces of \paysUkraine may be lost without hampering the objective}%
\EUobjective{}{\villeVienne never fell to \TUR}{}{}{40}{%
  \provinceOsterreich/\ville{Vienne} was never captured by
  \paysmajeurTurquie since the beginning of the game}%
\EU@objNoRUSBaltic{40}%
\EUobjective{}{Union with \paysmajeurSuede}{}{}{50}{%
  Union between \paysmajeurPologne and \paysmajeurSuede was active at least
  once since the beginning of the game. This may not be the main objective if
  it is already fulfilled at the beginning of the period (the Union was or is
  active)}%
%
\EUcurrentperiod{V}%
\EU@objSwedenContained
\EUobjective{}{Few nat\EUAbbr{ional} prov\EUAbbr{inces} lost}{}{20}{40}{%
  Number of provinces of the Polish or Lithuanian national territory lost;
  0: full sucess; 1: half-success;2 or more: failure. Only check provinces
  owned at the beginning of the period, not provinces annexed (and then lost
  again) during the period}%
\EUobjective{}{Absolutism}{}{}{40}{Absolutism established}%
\EU@objNoRUSBaltic{40}%
\EU@objNoTURHungaryFull[M]{50}{10}{ plus 20\VPs if \villeVienne was never
  controlled by \paysmajeurTurquie since the beginning of the game}%
%
%%%%%%%%%%%%%%%%%%%%%%%%%%%%%%%%%%%%%%%%%%%%%%%%%%%%%%%%%%%%%%
\EUcurrentcountry{Turquie}
%%%%%%%%%%%%%%%%%%%%%%%%%%%%%%%%%%%%%%%%%%%%%%%%%%%%%%%%%%%%%%
%
\EUcurrentperiod{I}%
\EUobjective{}{Mamluk Conquest}{}{}{35}{%
  Both \paysDamas and \paysEgypte have been destroyed}%
\EUobjective{}{Monopoly in \ctz{Turquie}}{}{}{30}{}%
\EU@objAlliance{10}{30}{defensive}{\paysmajeurFrance}{French alliance}%
\EU@objRhodos{30}%
\EU@objValMol{25}%
%
\EUcurrentperiod{II}%
\EU@objCountryDestroyed{Hongrie}{40}{Collapse}%
\EU@objViennaFallen{50}%
\EU@objCountryExists{Astrakhan}{35}%
\EU@objRhodos{35}%
\EU@objSpiceTUR[M]{10}{40}%
%
\EUcurrentperiod{III}%
% \EUobjective{M}{War or Peace in \payshongrie}{}{}{35}{%
%   Either there is no more than one turn of war between \AUSaus and
%   \paysmajeurTurquie, or at least 9 provinces of \payshongrie are owned by
%   \paysmajeurTurquie (whoever controls them)}%
\EUobjective{M}{War or Peace in \payshongrie}{}{}{35}{%
  [BLP] Either there is no more than one turn of formal war between \AUSaus
  and \paysmajeurTurquie, or at least 9 provinces of \payshongrie are owned or
  occupied by \paysmajeurTurquie}%
\EU@objViennaFallen{40}%
\EU@objCountryExists{Astrakhan}{35}%
\EUobjective{}{Med\EUAbbr{iterranean} Islands}{prov\EUAbbr{ince}*}{20}{50}{%
  \EU@objEachProvinceText[\provinceKreta, \provinceMalta, and
  \provinceChypre]; plus 10\VPs if \provinceCyclades is owned by
  \paysmajeurTurquie}%
\EU@objValMol{25}%
%
\EUcurrentperiod{IV}%
\EUobjective{}{Peace in \payshongrie}{}{}{35}{%
  No more than two turns of war between \AUSaus and \TUR ; interventions do
  not count}%
\EU@objPresidiosTUR{40}{10}%
\EU@objCrimeaTUR{6}%
\EU@objEachProvinceList{\provinceCorfou, \provinceKreta, \provinceMalta and
  \provinceChypre}{10}{40}{Med\EUAbbr{iterranean} Islands}%
\EU@objMonopolyZone{15}{45}%
%
\EUcurrentperiod{V}%
\EUobjective{}{\paysHongrie}{prov\EUAbbr{ince}}{8}{50}{%
  Each province of \paysHongrie owned by \paysmajeurTurquie, whoever controls
  it}%
\EU@objViennaFallen[M]{50}%
\EU@objCrimeaTUR{5}%
\EU@objCountryExists{Astrakhan}{30}%
\EU@objSpiceTUR{6}{30}%
%
\EUcurrentperiod{VI}%
\EUobjective{}{\paysHongrie or \paysTransylvanie}{}{}{40}{%
  Either \paysmajeurTurquie owns at least one province of \paysHongrie; or
  \paysTransylvanie exists and is on the Turkish Diplomatic track}%
\EU@objPresidiosTUR{50}{10}%
\EU@objCrimeaTUR[M]{4}%
\EU@objBalkansTUR{10}{45}{}%
\EU@objSpiceTUR{10}{30}%
%
\EUcurrentperiod{VII}%
\EUobjective{}{\paysMamelouks \VASSAL or annexed}{}{}{30}{%
  \paysMamelouks either does not exists (including if \ref{pVII:Mameluks
    Revolt} did not happen); or is \VASSAL of \paysmajeurTurquie}%
\EUobjective{}{Turkish Reforms}{reform}{5}{50}{%
  Each successful Reform since the beginning of the game
  (\ref{chSpecific:Turkey:Reform})}%
\EU@objCountryExists[M]{Crimee}{50}%
\EU@objBalkansTUR{15}{45}{4}%
\EUobjective{}{\paysgeorgie/\paysperse}{prov\EUAbbr{ince}-3}{15}{50}{%
  Each province of either \pays{Georgie} or \pays{Perse} owned by
  \paysmajeurTurquie above 3, whoever controls it}%
%
%%%%%%%%%%%%%%%%%%%%%%%%%%%%%%%%%%%%%%%%%%%%%%%%%%%%%%%%%%%%%%
\EUcurrentcountry{Venise}
%%%%%%%%%%%%%%%%%%%%%%%%%%%%%%%%%%%%%%%%%%%%%%%%%%%%%%%%%%%%%%
%
\EUcurrentperiod{I}%
\EU@objItaliSanMarco{15}{50}%
\EU@objEachProvinceList{\provinceHellas and
  \provinceMoreas}{20}{40}{\provinceHellas/\provinceMoreas}%
\EU@objBalkansVEN{10}{30}{\provinceHellas/\provinceMoreas}%
\EU@objCountryExists[M]{Mamelouks}{50}%
\EU@objOrientIncome{200}{40}%
%
\EUcurrentperiod{II}%
\EU@objItaliSanMarco{10}{50}%
\EU@objSpecificProvince{Moreas}{40}{\provinceMoreas}%
\EU@objBalkansVEN{10}{30}{\provinceMoreas}%
\EUobjective{M}{No \TUR islands}{}{}{50}{%
  No Mediterranean Island is owned by \paysmajeurTurquie (except
  \provinceRhodos)}%
\EU@objOrientIncome{250}{30}%
%
\EUcurrentperiod{III}%
\EU@objTradeCentre{Mediterranee}{40}%
\EUobjective{}{\regionBalkans/Barbary}{\Presidio}{10}{30}{%
  Each Venetian \Presidio in \Barbaresques, or \regionBalkans}%
\EU@objBalkansVEN{15}{45}{}%
\EU@objEachProvinceListAbove[M]{\provinceChypre, \provinceKreta,
  \provinceCyclades, \provinceCorfu, \provinceMalta,
  \provinceRhodos}{20}{50}{Islands}{2}
% (jcd) I added Rhodos here. Won't hurt.
\EU@objOrientIncome{300}{30}%
%
%%%%%%%%%%%%%%%%%%%%%%%%%%%%%%%%%%%%%%%%%%%%%%%%%%%%%%%%%%%%%%
\EUcurrentcountry{Portugal}
%%%%%%%%%%%%%%%%%%%%%%%%%%%%%%%%%%%%%%%%%%%%%%%%%%%%%%%%%%%%%%
%
\EUcurrentperiod{I}%
\EUobjective{}{Indian city}{}{20}{40}{%
  \paysmajeurPortugal owns a \COL on a city of the coast of \continentIndia;
  half-success if it is in \granderegionCeylan}%
\EU@objMarocVassal{10}{40}%
\EU@objMarocAnnexion{30}%
\EU@objMonopolyPOR{2}{30}%
\EUobjective{M}{Colonisation}{}{}{50}{%
  \paysmajeurPortugal has at least one \COL in \continentBrazil; \textbf{and}
  there is no more than 2 \COL/\TP producing \POSPICE owned by countries not
  in \dipAT with \paysmajeurPortugal (this includes \COL/\TP of other \MAJ)}
% (jcd) Should be split in two (objectives about Morocco can be merged)
% (jym) No. For Morocco, there is a choice between diplo and military. For
% colonisation, there is no choice between America and India.
%
\EUcurrentperiod{II}%
\EUobjective{M}{\TP in \paysChine and \paysJapon}{}{}{50}{%
  \paysmajeurPortugal has a \TP both in \paysChine and in \paysJapon}%
\EU@objMarocVassal{10}{40}%
\EU@objMarocAnnexion{30}%
\EU@objMonopolyPOR{3}{30}%
\EUobjective{}{Orient Trade}{}{}{50}{%
  There is no non-portuguese \COL/\TP in Asia (producing \POSPICE); excepted
  in \granderegion{Philippines}, and excepted establishments belonging to
  countries in \dipAT with \paysmajeurPortugal, \paysChine or \paysJapon}%

\makeatother

% Local Variables:
% fill-column: 78
% coding: utf-8-unix
% mode-require-final-newline: t
% mode: flyspell
% ispell-local-dictionary: "british"
% End:


% JC magic
\def\EUobjectives#1{%
  \EUcurrentcountry{#1} \csname subsection\endcsname {Objectives of
    \paysmajeur{#1}} \EUobjectivesa{I} \EUobjectivesa{II} \EUobjectivesa{III}
  \EUobjectivesa{IV} \EUobjectivesa{V} \EUobjectivesa{VI} \EUobjectivesa{VII}
} \def\EUobjectivesa#1{ \EUcurrentperiod{#1}
  \EUcasevalue{objective@\EUcurrcoun @\EUcurrper @1@short}{}{ \csname
    subsubsection\endcsname {Period \period{#1}} \EUobjectivesb{1}
    \EUobjectivesb{2} \EUobjectivesb{3} \EUobjectivesb{4} \EUobjectivesb{5} }}
\def\EUobjectivesb#1{ \EUcasevalue{objective@\EUcurrcoun @\EUcurrper
    @#1@short}{}{ \def\RES##1{##1}%
    \def\EUAbbr##1{##1}
    \aparag
    \EUvalue{objective@\EUcurrcoun @\EUcurrper @#1@short}{}%
    \EUcasevalue{objective@\EUcurrcoun @\EUcurrper @#1@letters}{%
    }{%
      (\EUvalue{objective@\EUcurrcoun @\EUcurrper @#1@letters}{})%
    }: %
    \EUcasevalue{objective@\EUcurrcoun @\EUcurrper @#1@per}{%
      \EUvalue{objective@\EUcurrcoun @\EUcurrper @#1@max}{}~\VPs%
      \EUcasevalue{objective@\EUcurrcoun @\EUcurrper @#1@score}{%
      }{%
        ~(\undemi: \EUvalue{objective@\EUcurrcoun @\EUcurrper
          @#1@score}{}~\VPs)%
      }%
    }{%
      \EUvalue{objective@\EUcurrcoun @\EUcurrper @#1@score}{} per
      \EUvalue{objective@\EUcurrcoun @\EUcurrper @#1@per}{} ~(max
      \EUvalue{objective@\EUcurrcoun @\EUcurrper @#1@max}{})%
    }%
    \EUcasevalue{objective@\EUcurrcoun @\EUcurrper @#1@desc}{%
    }{%
      ~--- \EUvalue{objective@\EUcurrcoun @\EUcurrper @#1@desc}{}.%
    }%
  } }
% end JC magic

\EUobjectives{Angleterre} % ANG
% DAN + move HOL here.
\EUobjectives{France} % FRA
\EUobjectives{Espagne} % HIS
\EUobjectives{Hollande} % HOL
\EUobjectives{Pologne} % POL-(PRU)
\EUobjectives{Prusse} % (POL)-PRU
\EUobjectives{Portugal} % POR-(SUE)
\EUobjectives{Suede} % (POR)-SUE
\EUobjectives{Russie} % RUS
\EUobjectives{Turquie} % TUR
\EUobjectives{Venise} % VEN-(AUS)
\EUobjectives{Autriche} % (VEN)-AUS


\section{End of game \VPs}\label{chVictories:End Game}
\aparag At the end of the game, a global check-up of each country is
done.
\bparag Additionally, the same check-up is performed for countries that
cease to be played (\POR, \HOL, \VEN, \POL) at the time of the transfer.

\aparag During this check-up, each country earns \VPs for fulfilling
objectives. All the objectives listed here are checked. There is no
choice among them (contrary to the end-of-period objectives).

\aparag Additionally, each country loses twice the income value of each
national province it does not own (whoever controls them).

\aparag[Provinces] \VPs are awarded for ownership and control of some
provinces. These \VPs depend on the income of the province.
\bparag If a province is only owned and not controlled, it it worth only
\td\ of the listed \VPs.
\bparag If a province is only controlled and not owned, it it worth only
\tu\ of the listed \VPs.
\bparag Round the \VPs total down once all the computation are done.
\bparag Each province may only count once for each country. Namely, a
province listed in a ``X times the income of'' objective does not count
for the ``each non-national province'' objective.
\bparag Provinces owned by \VASSAL count toward objective that
specifically list them. They do not count toward the ``each non-national
province'' objective.

\aparag Neither Exotic resources nor gold count for \COL and \TP
income. Namely, count only the sum of \lignebudget{Colonies} and
\lignebudget{Trading posts} for these objectives.

\aparag The verification for the rank of army or fleet are made at the
beginning of the last turn and not at the end and are counted respectivly in
\LD and in \NWD.
% \\
% TODO : réduite ou supprimer ces \PV ??
\aparag Alliances are also checked a the beginning of the last turn

\subsection{All powers}
\aparag -2 times the income value of each lost national province.

\subsectionJ{\sectionpaysmajeur{Portugal}}{\blason{portugal}}
\aparag[Territory.]
\bparag 1 time the income value of \shortprovince{Tanger},
\shortprovince{JebelTubqal}, \shortprovince{SidiIfni}, \shortprovince{Rif},
and \shortprovince{Magrib}.

\aparag[Trade.]
\bparag 1 \VP per level of commercial fleet.
\bparag 20 \VPs: total monoply in \stz{Indien}.
\bparag 10 \VPs: partial monoply in \stz{Indien}.

\aparag[Colonisation.]
\bparag 1 \VP per level of \COL or \TP.
\bparag 20 \VPs: \COL in a coastal city in \continent{India}.%\ville{Goa}.

\subsectionJ{\sectionpaysmajeur{Venise}}{\blason{venise}}
\aparag[Territory.]
\bparag 1 times the income value of each non national province.

\aparag[Trade.]
\bparag 1 \VP per level of commercial fleet.
\bparag 30 \VPs: Mediterranean \terme{Commercial Center}.

\aparag[Military.]
\bparag 10 \VPs: military alliance with \paysmajeur{Espagne}.

\subsectionJ{\sectionpaysmajeur{Pologne}}{\blason{pologne}}
\aparag[\region{Duche de Prusse}.] Provinces of \region{Duche de
  Prusse} that \paysmajeur{Pologne} voluntarily gave to
\pays{brandebourg} are counted as if \paysmajeur{Pologne} still
control them.

\aparag[Territory.]
\bparag 4 times the income value of each non-national province
(Provinces of \region{Ukraine} are considered as non-national).
% \bparag -2 times the income value of all lost national provinces, not
% including Ukrainian provinces.

\aparag[Industry.]
\bparag 20 \VPs: at least 10 levels of manufactures.

\aparag[Military.] Only count the best case among the three.
\bparag 30 \VPs: military alliance with \paysmajeur{France}, or
\bparag 20 \VPs: military alliance with \paysmajeur{Autriche}, or
\bparag 10 \VPs: military alliance with \paysmajeur{Suede}.

\subsectionJ{\sectionpaysmajeur{Hollande}}{\blason{hollande}}
%\bparag ({\it divide total by 2 if it is a Transfer check})
\aparag[Territory.]
\bparag 2 times the income value of \shortprovince{Hainaut},
\shortprovince{Flandre}, \shortprovince{Vlaandern}, \shortprovince{Luxemburg},
\shortprovince{Brabant}, \shortprovince{Limburg}
\bparag 3 times the income value of \shortprovince{Oldenburg},
\shortprovince{Artois}, \shortprovince{Picardie}, \shortprovince{Bremen}.
\bparag 1 time the income value of each non-national province, and of
all \COL and \TP (without exotic resources).
\bparag 50 \VPs: \FRA not adjacent to national territory.

\aparag[Trade.]
\bparag 1 \VP per level of \terme{commercial fleet}.
\bparag 50 \VPs: Atlantic \terme{Commercial Center}.

\aparag[Military.] Only count the best case among the two.
\bparag 30 \VPs: First or second largest fleet, or
\bparag 15 \VPs: At least 15 \NWD.

\subsectionJ{\sectionpaysmajeur{Angleterre}}{\blason{angleterre}}
\aparag[Territory]
\bparag 2 times the income value of each province of \pays{ecosse}.
\bparag 3 times the income value of \shortprovince{Picardie},
\shortprovince{Baleares}, \shortprovince{Corfu}, \shortprovince{Malta},
\shortprovince{Bremen}, \shortprovince{Hannover}, \shortprovince{Osnabruck},
and \shortprovince{Oldenburg}.
\bparag 1 time the income value of each non-national province, and of
all \COL and \TP (without the resources).
\bparag 50 \VPs: \shortprovince{Gibraltar}

\aparag[Trade.]
\bparag 1 \VP per level of commercial fleet.
\bparag 30 \VPs: Atlantic \terme{Commercial center}.
\bparag 30 \VPs: Mediterranean \terme{Commercial center}.

\aparag[Military.]
\bparag 30 \VPs: the largest fleet.
\bparag 10 \VPs: the largest army.
\bparag 50 \VPs: having 4 adjacent unrevolted level 6 \COL in
\continent{America}.

\subsectionJ{\sectionpaysmajeur{France}}{\blason{france}}
\aparag[Territory.]
\bparag 3 times the income value of \shortprovince{Hainaut},
\shortprovince{Vlaandern}, \shortprovince{Brabant}, \shortprovince{Luxemburg},
\shortprovince{Catalunya}, \shortprovince{Lombardia}, \shortprovince{Nice}.
\bparag 2 times the income value of \shortprovince{Lorraine},
\shortprovince{Alsace}, \shortprovince{Artois}, \shortprovince{Flandre},
\shortprovince{Picardie}, \shortprovince{Bresse},
\shortprovince{Franche-Comte}, \shortprovince{Rosselo},
\shortprovince{Corsica}.
\bparag 1 time the income value of each non-national province, and of
all \COL and \TP (without the resources).

\aparag[Trade.]
\bparag 1 \VP per level of commercial fleet.
\bparag 50 \VPs: Atlantic \terme{Commercial Center}.
\bparag 30 \VPs: Mediterranean \terme{Commercial Center}.

\aparag[Military.]
\bparag 30 \VPs: the largest fleet.
\bparag 10 \VPs: the largest army.

\subsectionJ{\sectionpaysmajeur{Espagne}}{\blason{espagne}}
\aparag[Territory.]
\bparag 3 times the income value of
\shortprovince{Friesland},\shortprovince{Overijssel},
\shortprovince{Gelderland}, \shortprovince{Utrecht}, \shortprovince{Zeeland},
\shortprovince{Brabant}, \shortprovince{Hainaut}, \shortprovince{Luxemburg},
\shortprovince{Franche-Comte},
\shortprovince{Vlaandern},\shortprovince{Lombardia}, \shortprovince{Campania},
\shortprovince{Sicilia}, \shortprovince{Palermo}
\bparag 2 times the income value of \shortprovince{Flandre},
\shortprovince{Rosselo}, \shortprovince{Calabria}, \shortprovince{Basilicata},
\shortprovince{Puglia}, \shortprovince{Abruzzo}, \shortprovince{Oran},
\shortprovince{Algerie},\shortprovince{Annabah}, \shortprovince{Tunisie}
\bparag 1 time the income value of each non-national province, and of
all \COL and \TP (without the resources).
\bparag {\bf -30} \VPs: loss of \shortprovince{Gibraltar}

\aparag[Military.]
\bparag 30 \VPs: the largest fleet.
\bparag 15 \VPs: the second largest fleet.
\bparag 20 \VPs: the largest army.

\aparag[Trade.]
\bparag 1 \VP per level of commercial fleet.

\aparag[Diplomacy.]
\bparag 20 \VPs: having \pays{venise} in \AM or more.

\aparag[Colonisation.] Only count the best case among the two.
\bparag 50 \VPs: no non-Spanish, non-portuguese \COL in
\continent{America}, or
\bparag 20 \VPs: no non-Spanish \COL in \continent{South America}.

\subsectionJ{\sectionpaysmajeur{Autriche}}{\blason{habsbourg}}
\aparag[Territory.]
\bparag 2 times the income value of \shortprovince{Magyarorszag},
\shortprovince{Pecs}, \shortprovince{Erdely}, \shortprovince{Karpatok},
\shortprovince{Mures}, \shortprovince{Banat}, \shortprovince{Croatie},
\shortprovince{Kapela}.
\bparag 50 \VPs: \paysmajeurTurquie owns no former province of
\paysHongrie (\blasonsmall{hongrie}).
% plus 50 \VPs if all these provinces are owned (whoever controls them).
\bparag 2 times the income value of \shortprovince{Campania},
\shortprovince{Sicilia}, \shortprovince{Palermo}, \shortprovince{Calabria},
\shortprovince{Basilicata}, \shortprovince{Puglia}, \shortprovince{Abruzzo}.
\bparag 3 times the income value of \shortprovince{Serbia},
\shortprovince{Bosna}, \shortprovince{Dalmacija}, \shortprovince{Ragusa},
\shortprovince{Lombardia}, \shortprovince{Valahia},
\shortprovince{Malopolska}, \shortprovince{Lublin}, \shortprovince{Wolyn},
\shortprovince{Podolie}.
\bparag 4 times the income value of \shortprovince{Lorraine},
\shortprovince{Alsace}, \shortprovince{Silesie}, \shortprovince{Lausitz}.
\bparag If \paysmajeurAutriche inherited \regionBelgique: 3 times the
income value of \shortprovince{Hainaut}, \shortprovince{Vlaandern},
\shortprovince{Luxemburg}, \shortprovince{Brabant},
\shortprovince{Limburg} and 2 times the income value of
\shortprovince{Artois}, \shortprovince{Flandre}
\bparag 1 time the income value of each non-national province, and of
all \COL and \TP (without the resources).

\aparag[Diplomacy.]
\bparag 20 \VPs: having \pays{venise} in \AM or more.

\subsectionJ{\sectionpaysmajeur{Russie}}{\blason{russie}}
\aparag[Territory.]
\bparag 3 times the income value of \shortprovince{Finland},
\shortprovince{Nyland}, \shortprovince{Georgie}, \shortprovince{Armenie},
\shortprovince{Mazowia}, \shortprovince{Wielkopolska}
\bparag 2 times the income value of \shortprovince{Karelen},
\shortprovince{Estland}, \shortprovince{Livonija}, \shortprovince{Kurland},
\shortprovince{Memel}, \shortprovince{Prypec}, \shortprovince{Lietuva},
\shortprovince{Baltarusija}, \shortprovince{Zemaitija},
\shortprovince{Severia}, \shortprovince{Moldova}, \shortprovince{Basarabia},
\shortprovince{Valahia}, \shortprovince{Savo}, \shortprovince{Crimee},
\shortprovince{Kuban}, \shortprovince{Caffa}, \shortprovince{Poltava},
\shortprovince{Azov}, \shortprovince{Podolie}, \shortprovince{Ukraine},
\shortprovince{Zaporozhye}, \shortprovince{Dagestan}, \shortprovince{Shirvan},
\bparag 50 \VPs: \paysmajeurRussie owns the initial territory of all
the Khanates (\paysSteppes (\blasonsmall{steppes}), \paysKazan
(\blasonsmall{kazan}), \paysCosaquesdon (\blasonsmall{cosaquesdon}),
\paysAstrakhan (\blasonsmall{astrakhan}), \paysCrimee
(\blasonsmall{crimee})).
\bparag 50 \VPs: \paysmajeurRussie owns \provinceNeva,
\provinceLietuva, \provinceEstland and all of \regionFinlande.
\bparag 1 time the income value of each non-national province, and of
all \COL and \TP (without the resources).

\aparag[Military.]
\bparag 20 \VPs: the largest army.
\bparag 15 \VPs: the first or 2nd largest fleet.

\aparag[Industry.]
\bparag 30 \VPs: at least 12 levels of manufactures.

\aparag[Development.]
\bparag 30 \VPs: \ville{Saint-Petersbourg} has been built.

\subsectionJ{\sectionpaysmajeur{Turquie}}{\blason{turquie}}
\aparag[Territory.]
\bparag 5 times the income value of \shortprovince{Malta},
\shortprovince{Sicilia}, \shortprovince{Corfu}, \shortprovince{Rhodos},
\shortprovince{Kreta}, \shortprovince{Chypre},
\bparag 2 times the income value of \shortprovince{Baleares},
\shortprovince{Saldigna}, \shortprovince{Oran}, \shortprovince{Algerie},
\shortprovince{Annabah}, \shortprovince{Tunisie}, \shortprovince{Pecs},
\shortprovince{Magyarorszag}, ,\shortprovince{Erdely},
\shortprovince{Karpatok}, \shortprovince{Wolyn}, \shortprovince{Malopolska},
\shortprovince{Armenie}, \shortprovince{Azarbayadjan},
\shortprovince{Kordistan}, \shortprovince{Van} \shortprovince{Irak},
\shortprovince{Serbia}, \shortprovince{Croatie}, \shortprovince{Bosnia},
\shortprovince{Carniola}, \shortprovince{Balaton}, \shortprovince{Szlovakia},
\shortprovince{Zaporozhye}, \shortprovince{Crimee}, \shortprovince{Azov},
\shortprovince{Kuban}, \shortprovince{Arabie}, \shortprovince{Egypte}
\shortprovince{Nil}, \shortprovince{Tanger}.
\bparag 1 time the income value of each non-national province, and of
all \COL and \TP (without the resources).

\aparag[Military.]
\bparag 15 \VPs: the first or 2nd largest fleet.

\aparag[Industry.]
\bparag 30 \VPs: at least 12 levels of manufactures.

\aparag[Trade.]
\bparag 50 \VPs: Mediterranean \terme{Commercial Center}.

\aparag[Reforms.]
\bparag 15 \VPs per reform.

\subsectionJ{\sectionpaysmajeur{Prusse}}{\blason{prusse}}
\aparag[Territory.]
\bparag 5 times the income value of \shortprovince{Silesie},
\shortprovince{Lausitz}, \shortprovince{Wielkopolska},
\shortprovince{Mazowia}, \shortprovince{Danzig}
\bparag 3 times the income value of \shortprovince{Berg},
\shortprovince{Anhalt}, \shortprovince{Holstein}, \shortprovince{Lubeck},
\shortprovince{West Preussen}
\bparag 1 time the income value of each non-national province, and of
all \COL and \TP (without the resources).

\aparag[Military.]
\bparag 30 \VPs: the biggest army.

\subsectionJ{\sectionpaysmajeur{Suede}}{\blason{suede}}
%\bparag ({\it divide total by 2 if it is a Transfer check})
\aparag[Territory.]
\bparag 5 times the income value of each baltic coastal province, except
\shortprovince{Sjaelland}.
\bparag 50 \VPs: \paysmajeurSuede has the \terme{Dominium Mari
  Balticii}.
\bparag 3 times the income value of \shortprovince{Savo},
\shortprovince{Tavastland}
\bparag 1 time the income value of each non-national province, and of
all \COL and \TP (without the resources).

\aparag[Trade.]
\bparag 1 \VP per level of commercial fleet.

% Local Variables:
% fill-column: 78
% coding: utf-8-unix
% mode-require-final-newline: t
% mode: flyspell
% ispell-local-dictionary: "british"
% End:



% \section{Objectives of period \period{I}}

% % TODO : déplacer les objectifs

% {\graytabular
% \begin{tabular}{|p{.8\columnwidth}|c|}
%     \hline
%     {\paysmajeur{Angleterre}}&\VP\\\ghline\hline
%     per turn of \pays{ecosse} as \VASSAL (max 30) &10\\
%     \ghline \hline
%     \shortprovince{Picardie} english presidio&45\\
%     \ghline \hline
%     \shortprovince{Guyenne} english provinces {\bf M}&50\\
%     \ghline \hline
%     MNU and Monopoly in  \CTZ{} England &25\\
%     \ghline \hline
%     Each turn without any revolts in \pays{Virlande} (max 40) &10\\
%     \ghline \hline
%   \end{tabular}
% }

%   {\graytabular
%   \begin{tabular}{|p{.8\columnwidth}|c|}
%     \hline
%     {\paysmajeur{France}}&\VP\\\ghline\hline
%     No presidio in \shortprovince{Picardie} &40\\\ghline\hline
%     \shortprovince{Artois} french province&30\\\ghline\hline
%     Sole defensor of the catholic faith&40\\\ghline\hline
%     MNU and Monopoly in  \CTZ France &25\\\ghline\hline
%     Victory in at least one War in Italy (Peace lvl. $\ge 2$) {\bf E}{\bf M}&50\\\ghline\hline
%   \end{tabular}
% }

%   {\graytabular
%   \begin{tabular}{|p{.8\columnwidth}|c|}
%     \hline
%     {\paysmajeur{Espagne}}&\VP\\\ghline\hline
%     No french \VASSAL in \region{Italie}&50\\\ghline\hline
%     Sole defensor of catholic faith&25\\\ghline\hline
%     No french annexion in \region{Italie}&40\\\ghline\hline
%     Each \Presidio on the barbary coast (max 50) {\bf M}&15\\\ghline\hline
%     All \pays{provincesne} provinces annexed&25\\\ghline\hline
%   \end{tabular}
% }

%   {\graytabular
%   \begin{tabular}{|p{.8\columnwidth}|c|}
%     \hline
%     {\paysmajeur{Pologne}}&\VP\\\ghline\hline
%     No provinces lost (except \shortprovince{Smolenska}) {\bf M}&40\\\ghline\hline
%     \shortprovince{Smolenska} polish province&40\\\ghline\hline
%     \pays{hongrie} still exist&50\\\ghline\hline
%     \pays{moldavie} neither \VASSAL of, nor conquered by
%     \paysmajeur{Turquie}&35\\\ghline\hline
%     \pays{Valachie} neither \VASSAL of, nor conquered by
%     \paysmajeur{Turquie}&35\\\ghline\hline
%   \end{tabular}
% }

%   {\graytabular
%   \begin{tabular}{|p{.8\columnwidth}|c|}
%     \hline
%     {\paysmajeur{Portugal}}&\VP\\\ghline\hline
%     \COL on a city of the coast of \continent{India}
%     (\granderegion{Ceylan} count 1/2 PV value only) &40\\
%     \ghline\hline
%     %     \shortprovince{Tanger} portuguese province&25\\\ghline\hline
%     Each turn \pays{maroc} is \VASSAL (max 40) &10\\\ghline\hline
%     One provinces annexed on \pays{maroc} &30\\\ghline\hline
%     Two Commercial \terme{monopoly} in any \STZ or \CTZ &30\\\ghline\hline
%     No more than 2 \COL/\TP in \continent{Asia} (producing \POSPICE) of
%     country without treaty and at least one \COL in \continent{Brazil} {\bf
%       M}&50\\\ghline\hline
%   \end{tabular}
% }

%   {\graytabular
%   \begin{tabular}{|p{.8\columnwidth}|c|}
%     \hline
%     {\paysmajeur{Russie}}&\VP\\\ghline\hline
%     Each province: \pays{pskov}, \pays{ryazan} (max 40) &20\\\ghline\hline
%     \shortprovince{Smolenska} conquered {\bf M}&50\\\ghline\hline
%     Conquest of \pays{Steppes} &35\\\ghline\hline
%     No province lost&25\\\ghline\hline
%     Conquest of one \terme{Khanate} &50\\\ghline\hline
%   \end{tabular}
% }

%   {\graytabular
%   \begin{tabular}{|p{.8\columnwidth}|c|}
%     \hline
%     {\paysmajeur{Turquie}}&\VP\\\ghline\hline
%     \pays{damas} and \pays{egypte} conquered&35\\\ghline\hline
%     Commercial \terme{monopoly} in \CTZ{Turquie}&30\\\ghline\hline
%     Each turn of Defensive alliance with \paysmajeur{France} (max 30) &10\\\ghline\hline
%     \shortprovince{Rhodos} conquered&30\\\ghline\hline
%     \pays{Valachie} and \pays{moldavie} \VASSAL or conquered&25\\\ghline\hline
%   \end{tabular}
% }

%   {\graytabular
%   \begin{tabular}{|p{.8\columnwidth}|c|}
%     \hline
%     {\paysmajeur{Venise}}&\VP\\\ghline\hline
%     Italia e San Marco: Each minor in \EG or better
%     (2 \pays{naples}'s provinces=1 minor) (max 50) &15\\\ghline\hline
%     Each province: \shortprovince{Hellas} or \shortprovince{Moreas} Venitian (max 40)&20\\\ghline\hline
%     Per (other) Venitian provinces in the Balkans (max 30) &10\\\ghline\hline
%     \pays{egypte} not conquered by Turquie {\bf M}&50\\\ghline\hline
%     Orient income total at least 200 \ducats on the period&40\\%\ghline\hline
%     (from \terme{Commercial Center} Gd Orient, Smyrna convoy, Alliance with \pays{aden}, \pays{oman} or
%     \pays{gujarat})&\\\ghline\hline
%   \end{tabular}
% }

%   \section{Objectives of period \period{II}}

%   {\graytabular
%   \begin{tabular}{|p{.8\columnwidth}|c|}
%     \hline
%     {\paysmajeur{Angleterre}}&\VP\\\ghline\hline
%     Each turn of \pays{ecosse} in \VASSAL (max 40)&10\\\ghline\hline
%     \shortprovince{Picardie} English presidio {\bf M}&50\\\ghline\hline
%     Win a war against \FRA (level $\ge 2$) (or \SPA ??)&40\\\ghline\hline
%     Each turn with no revolt in \pays{Virlande} (max 35)&7\\\ghline\hline
%     A \COL in \continent{America}&30\\\ghline\hline
%   \end{tabular}
% }

%   {\graytabular
%   \begin{tabular}{|p{.8\columnwidth}|c|}
%     \hline
%     {\paysmajeur{France}}&\VP\\\ghline\hline
%     No presidio in \shortprovince{Picardie} {\bf M}& 50\\\ghline\hline
%     French king on the Imperial throne {\bf E}&50\\\ghline\hline
%     \shortprovince{Artois} french province&30\\\ghline\hline
%     \shortprovince{Lombardia} or \shortprovince{Naples} french province&40\\\ghline\hline
%     \MNU and \terme{Monopoly} in \CTZ France&30\\\ghline\hline
%   \end{tabular}
% }

%   {\graytabular
%   \begin{tabular}{|p{.8\columnwidth}|c|}
%     \hline
%     {\paysmajeur{Espagne}}&\VP\\\ghline\hline
%     \pays{habsbourg} on the imperial Throne{\bf E}&35\\\ghline\hline
%     No french \VASSAL in \region{Italie}&40\\\ghline\hline
%     No french annexion in \region{Italie}&30\\\ghline\hline
%     Each \Presidio on the Barbary coast (max 50) {\bf M} &15\\\ghline\hline
%     No Religious problems in \HRE (Either no 2nd Reform event, or
%     Schmalkaldic league destroyed without religious liberty
%     (\eventref{pII:Schmalkaldic League}), or \eventref{pIV:TYW} won {\bf
%       E}&50\\\ghline\hline
%   \end{tabular}
% }

%   {\graytabular
%   \begin{tabular}{|p{.8\columnwidth}|c|}
%     \hline
%     {\paysmajeur{Pologne}}&\VP\\\ghline\hline
%     \shortprovince{Smolenska} polish province {\bf M}&50 \\\ghline\hline
%     No province lost&40\\\ghline\hline
%     \paysmajeur{Russie} non-adjacent to \region{Baltique}&40\\\ghline\hline
%     \pays{hongrie} still exist&50\\\ghline\hline
%     if Orthodox: at least one \COL in \continent{Siberia}&50\\\ghline\hline
%     Else: each province above 2 in Polish Ukraynia (max 50)&20\\\ghline\hline
%   \end{tabular}
% }

%   {\graytabular
%   \begin{tabular}{|p{.8\columnwidth}|c|}
%     \hline
%     {\paysmajeur{Portugal}}&\VP\\\ghline\hline
%     %     \shortprovince{Tanger} portuguese province&30\\\ghline\hline
%     Each turn \pays{maroc} is \VASSAL (max 40) &10\\\ghline\hline
%     One provinces annexed on \pays{maroc} &30\\\ghline\hline
%     Three commercial \terme{Monopoly} in any \STZ or \CTZ&30\\\ghline\hline
%     No non-portuguese \TP (or with \dipAT) in Asia (producing \POSPICE),
%     excepted in \granderegion{Philippines}, and excepted Chinese and Japanese \TP&50\\\ghline\hline
%     \TP in \pays{chine} and in \pays{japon} {\bf M} &40\\\ghline\hline
%   \end{tabular}
% }

%   {\graytabular
%   \begin{tabular}{|p{.8\columnwidth}|c|}
%     \hline
%     {\paysmajeur{Russie}}&\VP\\\ghline\hline
%     \pays{kazan} conquered&35\\\ghline\hline
%     \pays{astrakhan} conquered&50\\\ghline\hline
%     Possession of all national provinces and \shortprovince{Smolenska}&40\\\ghline\hline
%     \paysmajeur{Russie} has a port bordering \region{Baltique} {\bf
%       M}&50 \\\ghline\hline
%     Each diplomatic control (or annexion) of orthodox minor&10\\\ghline\hline
%   \end{tabular}
% }

%   {\graytabular
%   \begin{tabular}{|p{.8\columnwidth}|c|}
%     \hline
%     {\paysmajeur{Turquie}}&\VP\\\ghline\hline
%     \pays{hongrie} conquered&40\\\ghline\hline
%     \ville{Vienne} captured during the period&50\\\ghline\hline
%     \pays{astrakhan} still exist&35\\\ghline\hline
%     \shortprovince{Rhodos} Turkish province&35\\\ghline\hline
%     Each \COL or \TP producing \POSPICE\  (max 40) {\bf M}
%     &10\\\ghline\hline
%   \end{tabular}
% }

%   {\graytabular
%   \begin{tabular}{|p{.8\columnwidth}|c|}
%     \hline
%     {\paysmajeur{Venise}}&\VP\\\ghline\hline
%     Italia e San Marco: Each minor in \EG or better
%     (2 \pays{naples}'s provinces=1 minor) (max 50) &10\\\ghline\hline
%     Commercial \terme{Monopoly} in \CTZ Turkey&30\\\ghline\hline
%     Orient Income total at least 250 \ducats in the period&30\\%\ghline\hline
%     (from \terme{Commercial Center} Gd Orient, Smyrna convoy, Alliance with \pays{aden},
%     \pays{oman} or \pays{gujarat})&\\\ghline\hline
%     \shortprovince{Moreas} Venitian provinces&40\\\ghline\hline
%     No Turkish Island in East mediteranean sea
%     (\shortprovince{Chypre}, \shortprovince{Kreta}, \shortprovince{Cyclades},
%     \shortprovince{Corfu}, \shortprovince{Malta})  {\bf M}&50\\\ghline\hline
%   \end{tabular}
% }

%   \section{Objectives of period \period{III}}

%   {\graytabular
%   \begin{tabular}{|p{.8\columnwidth}|c|}
%     \hline
%     {\paysmajeur{Angleterre}}&\VP\\\ghline\hline
%     Each turn of \pays{ecosse} in \VASSAL (max 30)&5\\\ghline\hline
%     \paysmajeur{Angleterre} not forced to religion's change {\bf
%       M}&45\\\ghline\hline
%     MNU objective, and Monopoly in  \CTZ England&30\\\ghline\hline
%     \paysmajeur{Hollande} not convert by \paysmajeur{Espagne}&35\\\ghline\hline
%     Each Commercial \terme{Monopoly} in \CTZ/\STZ (max 30) &5/10\\\ghline\hline
%   \end{tabular}
% }

%   {\graytabular
%   \begin{tabular}{|p{.8\columnwidth}|c|}
%     \hline
%     {\paysmajeur{France}}&\VP\\\ghline\hline
%     No provinces lost&30\\\ghline\hline
%     Each province of \shortprovince{Artois}, \shortprovince{Bresse}, \shortprovince{Franche-Comte}, \shortprovince{Roussillon} (max 40) &10\\\ghline\hline
%     \paysmajeur{France} didn't change his religion (except due to L'Hospital) &50\\\ghline\hline
%     \paysmajeur{France} does not sign more than 2 unfavorable truce during the
%     Wars of Religion {\bf M} {\bf E}&40\\\ghline\hline
%     At least 3 \COL or \TP &30\\\ghline\hline
%   \end{tabular}
% }

%   {\graytabular
%   \begin{tabular}{|p{.8\columnwidth}|c|}
%     \hline
%     {\paysmajeur{Espagne}}&\VP\\\ghline\hline
%     Possession of at least half of the \pays{hongrie} by
%     \pays{habsbourg} (6+ provinces) &50
%     \\\ghline\hline
%     Each Protestant Major Countries converted (not possible if conciliant)
%     {\bf M} &50\\\ghline\hline
%     No Turkish possesions nor \VASSAL west to Ionian Sea (\shortprovince{Malta}
%     included)&40\\\ghline\hline
%     No more than one turn of war between \pays{habsbourg} and
%     \paysmajeur{Turquie}&30\\\ghline\hline
%     Commercial \terme{monopoly} in \CTZ Hispania and in at least 2 other
%     \CTZ/\STZ&30 \\\ghline\hline
%   \end{tabular}
% }

%   {\graytabular
%   \begin{tabular}{|p{.8\columnwidth}|c|}
%     \hline
%     {\paysmajeur{Pologne}}&\VP\\\ghline\hline
%     King of \paysmajeur{Pologne} have pretention on swedish throne
%     {\bf E}&40\\\ghline\hline
%     Each of those provinces \shortprovince{Kurland}, \shortprovince{Memel}, \shortprovince{Preussen} (max 50) &20\\\ghline\hline
%     No province lost&40\\\ghline\hline
%     \paysmajeur{Russie} non adjacent to \region{Baltique}&40\\\ghline\hline
%     Union of Lublin in effect {\bf E}&30\\\ghline\hline
%   \end{tabular}
% }

%   {\graytabular
%   \begin{tabular}{|p{.8\columnwidth}|c|}
%     \hline
%     {\paysmajeur{Suede}}&\VP\\\ghline\hline
%     Each turn of Commercial Domination on the \region{Baltique}(max 30) &5\\\ghline\hline
%     King of \paysmajeur{Pologne} don't pretend to Swedish throne {\bf
%       E}&50 \\\ghline\hline
%     Each province \shortprovince{Skane}, \shortprovince{Vastergotland}, \shortprovince{Gotland}
%     or of \region{Norvege} (10 only) (max 45)
%     &15\\\ghline\hline
%     Each province \shortprovince{Neva}, \shortprovince{Estland} and \shortprovince{Livonija} (max 50) &20\\\ghline\hline
%     Each \COL or \TP in \continent{America} (max 30) &10\\\ghline\hline
%     %     Conquest of \region{Norvege}&30\\\ghline\hline
%   \end{tabular}
% }

%   {\graytabular
%   \begin{tabular}{|p{.8\columnwidth}|c|}
%     \hline
%     {\paysmajeur{Russie}}&\VP\\\ghline\hline
%     Each turn of Furs \terme{monopoly} (max 50)&10\\\ghline\hline
%     Each province taken in \pays{Pologne}, \pays{Lithuanie}, \pays{Ukraine} (max 40)&10\\\ghline\hline
%     Each province \shortprovince{Neva}, \shortprovince{Estland} and \shortprovince{Livonija} &15\\\ghline\hline
%     Each province taken on \pays{crimee} (max 40) {\bf M}&20\\\ghline\hline
%     \pays{siberie} conquered&35\\\ghline\hline
%   \end{tabular}
% }

%   {\graytabular
%   \begin{tabular}{|p{.8\columnwidth}|c|}
%     \hline
%     {\paysmajeur{Turquie}}&\VP\\\ghline\hline
%     Each of the ollowing provinces that is turkish: \shortprovince{Malta} (20), \shortprovince{Cyclades} (10), and \shortprovince{Chypre}
%     or \shortprovince{Kreta} (20) (max 50)&10/20\\\ghline\hline
%     \ville{Vienne} captured during the period&40\\\ghline\hline
%     No more than one turn of war between \paysmajeur{Turquie} and
%     \pays{habsbourg}, or at least 9 provinces of \pays{Hongrie} {\bf M}&35\\\ghline\hline
%     \pays{astrakhan} still exist&30\\\ghline\hline
%     \pays{Valachie} and \pays{moldavie} conquered or \VASSAL&25\\\ghline\hline
%   \end{tabular}
% }

%   {\graytabular
%   \begin{tabular}{|p{.8\columnwidth}|c|}
%     \hline
%     {\paysmajeur{Venise}}&\VP\\\ghline\hline
%     %     Italia e San Marco (at least 5 minors in \EG or
%     %     better)&50\\%\ghline\hline
%     %     2 \pays{naples}'s provinces=1 minor for the purpose of this
%     %     objective&\\\ghline\hline
%     (??) Orient Income total at least 300 \ducats in the period&30\\%\ghline\hline
%     (from \terme{Commercial Center} Gd Orient, Smyrna convoy, Alliance with \pays{aden},
%     \pays{oman} or \pays{gujarat})&\\\ghline\hline
%     Mediterranean \terme{Commercial Center} &40\\\ghline\hline
%     Each island not Turkish Island East mediteranean sea above 2, from
%     (\shortprovince{Chypre}, \shortprovince{Kreta}, \shortprovince{Cyclades},
%     \shortprovince{Corfu}, \shortprovince{Malta})  (max 50) {\bf M}&20\\\ghline\hline
%     Each Venitian province in the Balkans (max  45)&15\\\ghline\hline
%     Each \Presidio (in \terme{Barbaresques}, or \region{Balkans} (not in owned province)) (max 30) &10\\\ghline\hline
%   \end{tabular}
% }

%   {\graytabular
%   \begin{tabular}{|p{.8\columnwidth}|c|}
%     \hline
%     {\paysmajeur{Hollande}}&\VP\\\ghline\hline
%     Mediterranean \terme{Commercial Center} &25\\\ghline\hline
%     Each turn of Oriental Indies convoy (max 40) {\bf M}&10\\\ghline\hline
%     \paysmajeur{Espagne} has recognized
%     \paysmajeur{Hollande}&50\\\ghline\hline
%     \paysmajeur{Angleterre} Protestant or Anglican and not
%     converted&40\\\ghline\hline
%     Protestant won the French wars of religions&35\\\ghline\hline
%   \end{tabular}
% }

%   \section{Objectives of period \period{IV}}

%   {\graytabular
%   \begin{tabular}{|p{.8\columnwidth}|c|}
%     \hline
%     {\paysmajeur{Angleterre}}&\VP\\\ghline\hline
%     \pays{ecosse} in \VASSAL: each turn (max 30) &5\\\ghline\hline
%     \paysmajeur{Angleterre} not forced to change religion {\bf M}&40\\\ghline\hline
%     Having one \terme{Commercial Center}& 50\\\ghline\hline
%     Each Monopoly in \CTZ/\STZ (max 30)&5/10\\\ghline\hline
%     English Civil War: 4 turns of less = 40; 5 turns = 20; 6+=failed {\bf E}&40\\\ghline\hline
%   \end{tabular}
% }

%   {\graytabular
%   \begin{tabular}{|p{.8\columnwidth}|c|}
%     \hline
%     {\paysmajeur{France}}&\VP\\\ghline\hline
%     No \pays{German Empire} nor Southern \HRE\ {\bf E} {\bf M}&50
%     \\\ghline\hline
%     No Northern \HRE alliance {\bf E}&40\\\ghline\hline
%     Each \COL (max 30)&5\\\ghline\hline
%     Commercial \terme{monopoly} in \CTZ France&30\\\ghline\hline
%     Each Commercial \terme{monopoly} (Total/Partial) (excepted \CTZ France)  (max 40) & 20/15 \\\ghline\hline
%   \end{tabular}
% }

%   {\graytabular
%   \begin{tabular}{|p{.8\columnwidth}|c|}
%     \hline
%     {\paysmajeur{Autriche}}&\VP\\\ghline\hline
%     Half of \pays{hongrie} controled by \pays{habsbourg} (6+ provinces)&40\\\ghline\hline
%     Creation of \pays{German Empire} {\bf M}&50\\\ghline\hline
%     Each province above 11 in \pays{habsbourg} (max 30)&10\\\ghline\hline
%     Southern HRE alliance exists {\bf E}&30\\\ghline\hline
%     Each turn of control of a port on the \region{Baltique} or in \pays{Hanse} (max 45)&15\\\ghline\hline
%   \end{tabular}
% }

%   {\graytabular
%   \begin{tabular}{|p{.8\columnwidth}|c|}
%     \hline
%     {\paysmajeur{Espagne}}&\VP\\\ghline\hline
%     Half of \pays{hongrie} controled by \pays{habsbourg} (6+ provinces)&40\\\ghline\hline
%     Creation of \pays{German Empire} {\bf M}&50\\\ghline\hline
%     Each province above 11 in \pays{habsbourg} (max 30)&10\\\ghline\hline
%     \pays{portugal} in \ANNEXION &50\\\ghline\hline
%     Each Commercial \terme{monopoly} (\CTZ Hispania counts as 2) (max 40)&8\\\ghline\hline
%   \end{tabular}
% }

%   {\graytabular
%   \begin{tabular}{|p{.8\columnwidth}|c|}
%     \hline
%     {\paysmajeur{Pologne}}&\VP\\\ghline\hline
%     \paysmajeur{Suede}-\paysmajeur{Pologne} union has been effective during the period
%     (not available as Prioritary objective if already effective at beginning of pIV) &50\\\ghline\hline
%     \ville{Vienne} never captured by \paysmajeur{Turquie}&40\\\ghline\hline
%     No non-Ukrainian provinces lost {\bf M}&40\\\ghline\hline
%     Number of swedish provinces adjacent to Baltic Sea (\region{Suede} and
%     \region{Finlande} excepted): 3 or less= 40; 4 = 20; 5+=failed &40\\\ghline\hline
%     \paysmajeur{Russie} non adjacent to \region{Baltique}&40\\\ghline\hline
%     \FTI, \DTI, and \MNU at maximum level&35\\\ghline\hline
%   \end{tabular}
% }

%   {\graytabular
%   \begin{tabular}{|p{.8\columnwidth}|c|}
%     \hline
%     {\paysmajeur{Suede}}&\VP\\\ghline\hline
%     No \pays{German Empire} {\bf E} {\bf M}&40\\\ghline\hline
%     \terme{Dominium Marii Baltici}&50\\\ghline\hline
%     Each turn of Commercial Domination on the \region{Baltique} (max 30) &5\\\ghline\hline
%     \pays{Hanse} does not exist any more&30\\\ghline\hline
%     King of \paysmajeur{Pologne} never on the swedish throne {\bf E} &40\\\ghline\hline
%     %     \paysmajeur{Russie} non adjacent to
%     %     \region{Baltique}&40\\\ghline\hline
%   \end{tabular}
% }

%   {\graytabular
%   \begin{tabular}{|p{.8\columnwidth}|c|}
%     \hline
%     {\paysmajeur{Russie}}&\VP\\\ghline\hline
%     Each turn of Furs \terme{monopoly} (max 45) &5\\\ghline\hline
%     Victory during the Time of Troubles {\bf E} &30\\\ghline\hline
%     Number of swedish provinces adjacent to Baltic Sea (\region{Suede} and
%     \region{Finlande} excepted): 3 or less= 40; 4 = 20; 5+=failed &40\\\ghline\hline
%     Number of national province that are not russian: 0 or 1=30, 2 = 15; 3+=failed &(30) \\\ghline\hline
%     Each Russian port on \region{Noire} (max 50) {\bf M} &20\\\ghline\hline
%   \end{tabular}
% }

%   {\graytabular
%   \begin{tabular}{|p{.8\columnwidth}|c|}
%     \hline
%     {\paysmajeur{Turquie}}&\VP\\\ghline\hline
%     No more than 2 turn of war with \pays{habsbourg} or \AUS&35\\\ghline\hline
%     Each province of \shortprovince{Kreta}, \shortprovince{Chypre}, \shortprovince{Corfou}, \shortprovince{Malta}  (max 40)&10\\\ghline\hline
%     Each \Presidio on the \terme{Barbaresques}: -10 (none=40) &40\\\ghline\hline
%     Each Commercial Monopoly in \STZ or \CTZ (max 45) &15 \\\ghline\hline
%     Number of provinces in \pays{crimee}: 6=40; 5=30; 4=20; 3-=failed&(40)\\\ghline\hline
%   \end{tabular}
% }

%   {\graytabular
%   \begin{tabular}{|p{.8\columnwidth}|c|}
%     \hline
%     {\paysmajeur{Hollande}}&\VP\\\ghline\hline
%     Each \COL in \continent{Bresil} (max 40)&10\\\ghline\hline
%     Alliance with Northern \HRE if \eventref{pIV:TYW} occured, or
%     \pays{hanse} still exists and in \VASSAL&40\\\ghline\hline
%     Mediterranean \terme{Commercial Center}&30\\\ghline\hline
%     Abolition of Navigation Act {\bf E} {\bf M}&40\\\ghline\hline
%     Each turn of Commercial Domination on the \region{Baltique} (max 30) &5\\\ghline\hline
%   \end{tabular}
% }

%   \section{Objectives of period \period{V}}

%   {\graytabular
%   \begin{tabular}{|p{.8\columnwidth}|c|}
%     \hline
%     {\paysmajeur{Angleterre}}&\VP\\\ghline\hline
%     Each island with a \TP or \COL controled/owned in
%     \continent{Caraibes} (max 45) &7/15\\\ghline\hline
%     Each turn of Fish \terme{monopoly} (max 30) &5\\\ghline\hline
%     \paysmajeur{Angleterre} not forced to change religion&40\\\ghline\hline
%     \pays{Portugal} is independent&30\\\ghline\hline
%     Atlantic \terme{Commercial Center} {\bf M}&40\\\ghline\hline
%   \end{tabular}
% }

%   {\graytabular
%   \begin{tabular}{|p{.8\columnwidth}|c|}
%     \hline
%     {\paysmajeur{France}}&\VP\\\ghline\hline
%     Each turn of Fish \terme{monopoly} (max 30) &5\\\ghline\hline
%     Mediterranean \terme{Commercial Center}&40\\\ghline\hline
%     \paysmajeur{France} adjacent to \paysmajeur{Hollande} national
%     territory {\bf M}&50\\\ghline\hline
%     Victory in the Glorious Revolution {\bf E}&40\\\ghline\hline
%     Each victory in the following wars: Devolution, Chamber of Reunion, League of Augsbourg {\bf E} (max 40)&20\\\ghline\hline
%   \end{tabular}
% }

%   {\graytabular
%   \begin{tabular}{|p{.8\columnwidth}|c|}
%     \hline
%     {\paysmajeur{Autriche}}&\VP\\\ghline\hline
%     Each turkish province in \pays{hongrie}: -10 (none=50) &(50)\\\ghline\hline
%     Each province in Spanish Netherland (max 40) {\bf M}&10\\\ghline\hline
%     Possessions of \pays{Venise}: Each Mediterranean island, or province in \region{Balkans} above 2 (max 50)  &20\\\ghline\hline
%     Absolutism established in \POL &25\\\ghline\hline
%     Each victory in the following wars: Devolution, Chamber of Reunion, League of Augsbourg {\bf E} (max 40)&20\\\ghline\hline\end{tabular}
%     % }

%   {\graytabular
%   \begin{tabular}{|p{.8\columnwidth}|c|}
%     \hline
%     {\paysmajeur{Espagne}}&\VP\\\ghline\hline
%     Each turkish province in \pays{hongrie}: -10 (none=40) {\bf M}&(40)\\\ghline\hline
%     Each province in Spanish Netherland (max 50)&10\\\ghline\hline
%     \pays{Portugal} is in \ANNEXION &40\\\ghline\hline
%     No non-spanish \COL south of \granderegion{Florida} (\continent{Caraibes} included;
%     \continent{Brazil} excepted)= 40, or same objective with \continent{Caraibes} and  \continent{Brazil} excepted = 20 &40\\\ghline\hline
%     Each Commercial \terme{monopoly} (max 40)&8 \\\ghline\hline
%   \end{tabular}
% }

%   {\graytabular
%   \begin{tabular}{|p{.8\columnwidth}|c|}
%     \hline
%     {\paysmajeur{Pologne}}&\VP\\\ghline\hline
%     \ville{Vienne} never captured by \paysmajeur{Turquie} {\bf M}&50\\\ghline\hline
%     Number of swedish provinces adjacent to Baltic Sea (\region{Suede} and
%     \region{Finlande} excepted): 3 or less= 40; 4 = 20; 5+=failed &(40)\\\ghline\hline
%     \paysmajeur{Russie} non adjacent to \region{Baltique}&40\\\ghline\hline
%     National province of \pays{Pologne} or \pays{Lithuanie} lost: 0=40, 1=20, 2+=failed&(40)\\\ghline\hline
%     Each turkish province in \pays{hongrie}: -10  (none=50) {\bf M}&(50)\\\ghline\hline
%   \end{tabular}
% }

%   {\graytabular
%   \begin{tabular}{|p{.8\columnwidth}|c|}
%     \hline
%     {\paysmajeur{Suede}}&\VP\\\ghline\hline
%     \terme{Dominium Marii Baltici} {\bf M}&50\\\ghline\hline
%     %     \pays{hanse} does not exist any more&40\\\ghline\hline
%     \paysmajeur{Russie} non adjacent to \region{Baltique}&40\\\ghline\hline
%     Each turn of Commercial Domination on the \region{Baltique} (max 30) &5\\\ghline\hline
%     Absolutism is not established in \POL\  &30\\\ghline\hline
%     Each \COL or \TP (max 40)&5\\\ghline\hline
%   \end{tabular}
% }

%   {\graytabular
%   \begin{tabular}{|p{.8\columnwidth}|c|}
%     \hline
%     {\paysmajeur{Russie}}&\VP\\\ghline\hline
%     Contruction of \ville{Saint-Petersbourg} completed&50\\\ghline\hline
%     Each turn of Furs \terme{monopoly} (max 35)&5\\\ghline\hline
%     Each \COL or \TP in \granderegion{Amour} or \granderegion{Baikal} (max
%     30)&10\\\ghline\hline
%     \pays{Astrakhan} destroyed&40\\\ghline\hline
%     Each Russian port on \region{Noire} (max 50) &20\\\ghline\hline
%   \end{tabular}
% }

%   {\graytabular
%   \begin{tabular}{|p{.8\columnwidth}|c|}
%     \hline
%     {\paysmajeur{Turquie}}&\VP\\\ghline\hline
%     Each \TP producing \POSPICE (max 30)&6\\\ghline\hline
%     \ville{Vienne} captured during the period {\bf M}&50\\\ghline\hline
%     Each turkish province in \pays{hongrie} (max 50)&8\\\ghline\hline
%     Number of provinces in \pays{crimee}: 5 or 6=40; 4=30; 3=20; 2-=failed&(40)\\\ghline\hline
%     \pays{Astrakhan} still exists &30\\\ghline\hline
%   \end{tabular}
% }

%   {\graytabular
%   \begin{tabular}{|p{.8\columnwidth}|c|}
%     \hline
%     {\paysmajeur{Hollande}}&\VP\\\ghline\hline
%     \paysmajeur{France} not adjacent to national
%     territory&50\\\ghline\hline
%     Abolition of Navigation Act {\bf E}&50\\\ghline\hline
%     Atlantic \terme{Commercial Center} {\bf M}&40\\\ghline\hline
%     No Christian non-dutch \TP east of \granderegion{Malacca} (included,
%     \continent{Siberia} excepted) &40\\\ghline\hline
%     Each turn of Spices \terme{monopoly} (max 40) &5 \\\ghline\hline
%   \end{tabular}
% }

%   \section{Objectives of period \period{VI}}

%   {\graytabular
%   \begin{tabular}{|p{.8\columnwidth}|c|}
%     \hline
%     {\paysmajeur{Angleterre}}&\VP\\\ghline\hline
%     Each island with a \TP or \COL controled/owned in \continent{Caraibes} (max 50) &5/10\\\ghline\hline
%     Victory in the Jacobite rebellion {\bf E}&30\\\ghline\hline
%     Victory in the war of Spanish Succession {\bf E}&50\\\ghline\hline
%     Each turn of control of Oriental India convoy (max 40)&4\\\ghline\hline
%     Atlantic \terme{Commercial Center} {\bf M}&40\\\ghline\hline
%   \end{tabular}
% }

%   {\graytabular
%   \begin{tabular}{|p{.8\columnwidth}|c|}
%     \hline
%     {\paysmajeur{France}}&\VP\\\ghline\hline
%     Victory in the war of Spanish Succession {\bf E}&50\\\ghline\hline
%     Victory in the war of Austrian Succession without any territorial gain {\bf E}&40\\\ghline\hline
%     Polish King supported by \FRA\ {\bf E} & 50 \\\ghline\hline
%     Mediterranean \terme{Commercial Center}&40\\\ghline\hline
%     Atlantic \terme{Commercial Center} {\bf M}&50\\\ghline\hline
%   \end{tabular}
% }

%   {\graytabular
%   \begin{tabular}{|p{.8\columnwidth}|c|}
%     \hline
%     {\paysmajeur{Autriche}}&\VP\\\ghline\hline
%     Control of \pays{hongrie}: 50 -20 per Turkish province &50\\\ghline\hline
%     \shortprovince{Silesie} and \shortprovince{Lausitz} Austrian provinces {\bf M}
%     &50\\\ghline\hline
%     Victory in the war of Spanish Succession {\bf E}&50\\\ghline\hline
%     Victory in the war of Austrian Succession  {\bf E}&40\\\ghline\hline
%     \PRU has not the Royal Dignity&30\\\ghline\hline
%   \end{tabular}
% }

%   {\graytabular
%   \begin{tabular}{|p{.8\columnwidth}|c|}
%     \hline
%     {\paysmajeur{Espagne}}&\VP\\\ghline\hline
%     Victory in the war of spanish succession {\bf E} {\bf
%       M}&50\\\ghline\hline
%     Each Commercial \terme{monopoly} (max 40)&10 \\\ghline\hline
%     No non-spanish \COL south of \granderegion{Florida}  (\continent{Caraibes} included;
%     \continent{Brazil} excepted)= 45, or same objective with \continent{Caraibes}
%     and  \continent{Brazil} excepted = 20 &45\\\ghline\hline
%     Each \Presidio on the \terme{Barbaresques} (max 50)& 10\\\ghline\hline
%     Each non national provinces (max 40)&10\\\ghline\hline
%   \end{tabular}
% }

%   {\graytabular
%   \begin{tabular}{|p{.8\columnwidth}|c|}
%     \hline
%     {\paysmajeur{Prusse}}&\VP\\\ghline\hline
%     \shortprovince{Silesie} and \shortprovince{Lausitz} Prussian provinces {\bf
%       M}&50\\\ghline\hline
%     \paysmajeur{Autriche} is not on the imperial throne any more {\bf
%       E}&40\\\ghline\hline
%     Each turn of Military alliance with a major (max 40)&5\\\ghline\hline
%     Having the Royal Dignity&30\\\ghline\hline
%     Each province above 9 (max 40)&10\\\ghline\hline
%   \end{tabular}
% }

%   {\graytabular
%   \begin{tabular}{|p{.8\columnwidth}|c|}
%     \hline
%     {\paysmajeur{Suede}}&\VP\\\ghline\hline
%     \terme{Dominium Marii Baltici}&40\\\ghline\hline
%     Polish King supported by \SUE (50), or no France-supported king in \POL (25) {\bf E}& 25/50 \\\ghline\hline
%     No provinces lost (except \shortprovince{Neva}) {\bf M}&40\\ \ghline\hline
%     Each turn of Commercial Domination on the \region{Baltique} (max 40) &5\\\ghline\hline
%     Each \COL or \TP (max 40)&7\\\ghline\hline
%   \end{tabular}
% }

%   {\graytabular
%   \begin{tabular}{|p{.8\columnwidth}|c|}
%     \hline
%     {\paysmajeur{Russie}}&\VP\\\ghline\hline
%     Each \COL in \granderegion{Alaska} (max 30)&10\\\ghline\hline
%     \POL does not have a King suported by \FRA or \SUE (need a Peace of level 3+ against \POL) {\bf E}&30\\\ghline\hline
%     Each port of the \region{Baltique} (max 40) &15\\\ghline\hline
%     Each conquered province of \pays{Crimee} and \pays{Moldavie} above 4 (max 50)  {\bf M}&15\\\ghline\hline
%     Each \COL/\TP in china (\granderegion{Amour}, \granderegion{Baikal}),
%     or on indian
%     road (\granderegion{Afghanistan}, \granderegion{Perse},
%     \continent{India}) (max 40) &10\\\ghline\hline
%   \end{tabular}
% }

%   {\graytabular
%   \begin{tabular}{|p{.8\columnwidth}|c|}
%     \hline
%     {\paysmajeur{Turquie}}&\VP\\\ghline\hline
%     Each province in \region{Balkans} (max 45)&10 \\\ghline\hline
%     Each \TP with \POSPICE (max 30)&10\\\ghline\hline
%     Number of provinces in \pays{crimee}: 4+=40; 3=30; 2=20; 1-=failed {\bf M}&(40) \\\ghline\hline
%     Number of \Presidio on the \terme{Barbaresques}: 50-10/\Presidio&(50)\\\ghline\hline
%     \pays{transylvanie} still exists, or control of at least one province of \pays{hongrie} &40\\\ghline\hline
%   \end{tabular}
% }

%   {\graytabular
%   \begin{tabular}{|p{.8\columnwidth}|c|}
%     \hline
%     {\paysmajeur{Hollande}}&\VP\\\ghline\hline
%     Victory in the war of Spanish Succession {\bf E}&50\\\ghline\hline
%     Each \terme{Commercial Center} (max 50) {\bf M} &25\\\ghline\hline
%     Each \COL in \continent{America} (excepted \continent{Bresil}) (max 45)&15\\\ghline\hline
%     Each Christian non-dutch \TP east of \granderegion{Malacca} (included): -10  &40\\\ghline\hline
%     Each turn of control of Oriental India convoy (max 40)&4\\\ghline\hline
%   \end{tabular}
% }

%   \section{Objectives of period \period{VII}}

%   {\graytabular
%   \begin{tabular}{|p{.8\columnwidth}|c|}
%     \hline
%     {\paysmajeur{Angleterre}}&\VP\\\ghline\hline
%     Victory in the Seven Years War {\bf E}&30\\\ghline\hline
%     Victory in the Independance War (victory of insurgent if the war
%     occured in another country) {\bf E} {\bf M}&50\\\ghline\hline
%     Each \terme{Commercial Center} (max 45) &15\\\ghline\hline
%     Each turn of control of Oriental India convoy (max 40)&4\\\ghline\hline
%     \paysmajeur{France} does not reach its ``natural frontier'' during the Révolution
%     (cannot be the prioritary objective) {\bf E} &30\\\ghline\hline
%   \end{tabular}
% }

%   {\graytabular
%   \begin{tabular}{|p{.8\columnwidth}|c|}
%     \hline
%     {\paysmajeur{France}}&\VP\\\ghline\hline
%     \pays{pologne} still exists {\bf M}&50\\\ghline\hline
%     Each \terme{Commercial center} (max 40)&20\\\ghline\hline
%     Victory in the Independance War (victory of insurgent if the war
%     occured in another country) {\bf E}&45\\\ghline\hline
%     \paysmajeur{France} has more \TP plus \COL than any other country in any one region of
%     \continent{India}, \continent{North America}, or \continent{Caraibes}&40 \\\ghline\hline
%     \paysmajeur{France} reaches its ``natural frontier'' during the Révolution
%     (cannot be the prioritary objective) {\bf E} &30\\\ghline\hline
%   \end{tabular}
% }

%   {\graytabular
%   \begin{tabular}{|p{.8\columnwidth}|c|}
%     \hline
%     {\paysmajeur{Autriche}}&\VP\\\ghline\hline
%     \shortprovince{Silesie} and \shortprovince{Lausitz} Austrian provinces &50\\\ghline\hline
%     Each turn of Military alliance with a major  (max 30)&5\\\ghline\hline
%     Ownership of \pays{hongrie} (no turkish province), \shortprovince{Bosna} and \shortprovince{Serbia} (max 50) &20\\\ghline\hline
%     Each provinces of \pays{naples} above 3 (max 50) {\bf M}&10\\\ghline\hline
%     \paysmajeur{France} does not reach its ``natural frontier'' during the Révolution
%     (cannot be the prioritary objective) {\bf E} &30\\\ghline\hline
%   \end{tabular}
% }

%   {\graytabular
%   \begin{tabular}{|p{.8\columnwidth}|c|}
%     \hline
%     {\paysmajeur{Espagne}}&\VP\\\ghline\hline
%     Spanish Asiento &35 \\\ghline\hline
%     Each non national provinces above 3 (max 40) &10\\\ghline\hline
%     No non-spanish \COL south of \granderegion{Florida}  (\continent{Caraibes} included;
%     \continent{Brazil} excepted)= 50, or same objective with \continent{Caraibes}
%     and  \continent{Brazil} excepted = 20 {\bf M} &50\\\ghline\hline
%     Objective MNU plus \FTI, \DTI at maximum level&25\\\ghline\hline
%     Victory in the Independance War (victory of insurgent if the war
%     occured in another country) {\bf E}&30\\\ghline\hline
%   \end{tabular}
% }

%   {\graytabular
%   \begin{tabular}{|p{.8\columnwidth}|c|}
%     \hline
%     {\paysmajeur{Prusse}}&\VP\\\ghline\hline
%     \shortprovince{Silesie} and \shortprovince{Lausitz} Prussian provinces {\bf
%       M}&50\\\ghline\hline
%     Each turn of Military alliance with a major (max 40)&5\\\ghline\hline
%     Victory in the Seven Years War {\bf E}&40\\\ghline\hline
%     Each province annexed during the period (max 50)&15\\\ghline\hline
%     \paysmajeur{France} doesn't reach its ``natural frontier'' (cannot be
%     the prioritary objective) &30\\\ghline\hline
%   \end{tabular}
% }

%   {\graytabular
%   \begin{tabular}{|p{.8\columnwidth}|c|}
%     \hline
%     {\paysmajeur{Suede}}&\VP\\\ghline\hline
%     \pays{pologne} still exist&40\\\ghline\hline
%     No province lost &40\\\ghline\hline
%     Each province outside \region{Suede}, \region{Finlande},
%     \region{Denmark} and\region{Norvege} (max 50) &15\\\ghline\hline
%     Each turn of Commercial Domination on the \region{Baltique} (max 40) &5\\\ghline\hline
%     Each \COL or \TP (max 40)&10\\\ghline\hline
%   \end{tabular}
% }

%   {\graytabular
%   \begin{tabular}{|p{.8\columnwidth}|c|}
%     \hline
%     {\paysmajeur{Russie}}&\VP\\\ghline\hline
%     \pays{pologne} does not exist any more &50\\\ghline\hline
%     \pays{crimee} does not exist any more {\bf M} &40\\\ghline\hline
%     Each port of the \region{Baltique} (max 40) &10\\\ghline\hline
%     Each province of national territory of \pays{Georgie} or \pays{Perse} (max 40) &10 \\\ghline\hline
%     Each province taken from \paysmajeur{Turquie} (excepted national territory of \pays{Georgie} or \pays{Perse} )(max 45) &15 \\\ghline\hline
%   \end{tabular}
% }

%   {\graytabular
%   \begin{tabular}{|p{.8\columnwidth}|c|}
%     \hline
%     {\paysmajeur{Turquie}}&\VP\\\ghline\hline
%     Each province in \region{Balkans} above 4 (max 45)&15 \\\ghline\hline
%     \pays{crimee} still exists {\bf M}&50\\\ghline\hline
%     Each Success in Reforms since the beginning &5\\\ghline\hline
%     Each province of national territory of \pays{Georgie} or \pays{Perse} above 3 (max 50) &15\\\ghline\hline
%     \pays{Mamelouks} in \VASSAL or annexed&30\\\ghline\hline
%   \end{tabular}
% }

%   {\graytabular
%   \begin{tabular}{|p{.8\columnwidth}|c|}
%     \hline
%     {\paysmajeur{Hollande}}&\VP\\\ghline\hline
%     Victory in \eventrefname{pVII:Batavian Revolution} {\bf E} {\bf M} &40\\\ghline\hline
%     Each Commercial \terme{Monopoly} in \CTZ or \STZ (max 30) & 5 \\\ghline\hline
%     Each \COL or \TP in \continent{India} (max 50) &15\\\ghline\hline
%     Each \COL in \continent{America} (excepted \continent{Bresil}) (max 45)&15\\\ghline\hline
%     \paysmajeur{France} does not reach its ``natural frontier'' during the Révolution
%     ({\bf may} be the prioritary objective) {\bf E} &35\\\ghline\hline
%   \end{tabular}
% }
