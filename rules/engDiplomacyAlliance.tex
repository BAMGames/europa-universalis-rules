% -*- mode: LaTeX; -*-

\section{Agreements between Major Powers}

% RaW: [32,34]



\subsection{Negotiations}


\subsubsection{Negotiations between Players}
\aparag
Players can negotiate freely between them to get into various kinds of
agreements, as long as they respect the letter and the spirit of the
rules. Players' diplomatic relationships may however be "officialized" in
alliances, or may be broken.
\aparag
Players negotiate between them, freely. It is advised that the time of
negotiations be limited to at most 10 minutes on an average (5 is counselled,
but not always possible or realistic).


\subsubsection{Outcome of Agreements}
\aparag When negotiations are closed, players announce their agreements:
informal agreement, or formal agreements: alliance (by specifying which), or
some possible trade refusal.
\bparag
This is done during the Diplomatic Phase on the fourth segment (the
Announcement Segment), after Declarations of Wars caused by events, but before
the declarations of War and any Diplomacy on minor countries.
\bparag Formal Agreements should be decided before the Announcement
Segment. Then they are made made loudly in the order of the initiative. As the
Agreements need not be written beforehand, a player could change his mind just
when doing announcements: this is allowed but no negotiation can take place at
this time.
\bparag
The simple public announcement of the agreement suffices to validate it. This
public agreement bears treaty value.
\bparag A formal agreement can be written down during the phase of
negotiations. If this is the case and one player refuses to make the
announcements, his power loses {\bf 1} \STAB.
\bparag Formal agreements can be kept secret: they have value only if written
down and signed by all allies. They can be used later, but with reduced value.
% (Jym).  Informal agreement were for small money transfer. Seems to be
% completely useless since we now allow loan treaty with as few money as one
% wants. Removing.

% \aparag[Informal Agreements]
% An informal agreement allows all types of actions and understandings that
% does not require specifically a formal alliance (ex. right of passing
% through player's provinces, of supply through). The breach to such an
% agreement entails no penalty for the player in cause.
% \bparag Are explicitly forbidden in informal agreement: change of ownership
% of provinces, \COL or \TP, transfers of diplomatic control, of troops.
% \bparag[Informal Money Transfer]\label{chDiplo:Informal Money Transfer}
% Money can be transfered freely by informal agreements, only up to 25 \ducats
% lent to of from another Power, and up to 50 \ducats lent to or from other
% Powers.
% \bparag Diplomatic supports is another kind of Informal Agreement that is
% discussed in \ruleref{chDiplo:Diplomatic Support}, which can creates money
% transfers up to 30 \ducats.

\aparag There exist several types of Announcements: Alliances of different
kinds, each corresponding to a precise agreement, and Trade Refusal. The type
of alliance must always be publicly announced to all other players, or kept
secret and written down.


\subsubsection{Alliances}
\aparag Only players possessing a determined alliance can co-operate in the
various domains considered hereafter. Alliances are of 4 different levels:
\bparag Dynastic Ties
\bparag Loan Treaty
\bparag Defensive Alliance
\bparag Offensive Alliance

\aparag[Generalities]
Alliances are concluded between two or more players. A player can conclude as
many different alliances as he desires with the same player, and/or with
different players, with the restrictions given for each type of alliance as
described hereafter.
\bparag A Formal Agreement (except Loan Treaty) is valid for this turn, the
two following ones, and the very beginning of the next turn, until the
beginning of the segment of Announcements (at which point the Formal Agreement
that ends could be signed again).
\bparag Secret agreements must specify the type of alliance, the powers
involved, the first turn of the alliance, or would be void. They last 3 turns
(like announced alliances). Dynastic Ties are always public and can not be
kept secret (secret Dynastic Ties are void).

\aparag[Dynastic Ties]\label{chDiplo:Alliance:Dynastic Alliance}
A pair of players may conclude a marriage between the ruling families of their
realms, so as to create family ties. They can no longer declare war on each
other without Casus Belli (\CB). This alliance lasts for the whole duration of
the next 2 consecutive turns, except when specific events occurs, forcing its
cancellation.

\bparag To conclude this marriage, one of the two players has to offer a dowry
to the other. The dowry has to be 100 \ducats (minimum, more can be offered up
to the gross income from previous turn of the Power), or consists of one
single province, \COL or \TP, immediately ceded to the other, receiving, party
upon conclusion of the agreement, at the end of the Diplomatic Phase. Note
that the province is still owned by its former controller for the following
segment of Declaration of Wars, the transfer would be latter, at the end of
the phase.
\bparag Money transferred is recorded on \lignebudget{Gifts and loans between
  players}.
\bparag The ceded province, Colony or Trading Post must be owned and
controlled by the ceding player, i.e. it is not possible to cede any territory
in revolt or occupied by another player at the time of the dynastic treaty.
\bparag
The two players are authorised in addition to exchange an extra province, \COL
or \TP. This exchange may be made in addition to the dowry (e.g., exchange of
one province + dowry of a province/or 100+ \ducats), but it is not compulsory
and may never involve national provinces. The previous condition on control
holds.
\bparag
The dynastic alliance can be cancelled at any given time. The party that
cancels it loses 2 \STAB levels.
\bparag
Only a dynastic alliance allows players to cede or exchange a province, \COL
or \TP. Each ceded possession has to be specified at the time of the alliance
conclusion.
\bparag Each ceding of a province, \COL or \TP, costs 1 level of \STAB to the
ceding party.

\bparag[War of Successions.]\label{chDiplo:succession}
The player that pays the dowry can benefit from a War of Succession inside the
other player's country, if a dynastic Crisis occurred in the country that
received the dowry. After Dynastic Ties are established, the rights in case of
War of Succession are valid for 8 turns. When a dynastic Crisis happens, the
power is allowed to declare war on that country as if he had a \CB, or on the
contrary he is allowed to enter as an ally of that same country, as if he had
a defensive alliance with it. See~\ruleref{chSpecific:War of Succession} about
the conditions of this war.

\bparag A dynastic alliance cannot be renewed with the same player less than 3
complete turns after the official end (i.e. be it after two turns or earlier
be-cause it was previously broken) of the previous alliance.

\bparag A dynastic alliance cannot be formed with a player of a different
religion unless a 2 \STAB level loss is incurred for doing so. This applies
until the end of \terme{Religious Enmities} between Protestant, Catholic and
Orthodox countries. It always applies between all Christians and Muslims.
\bparag%[Historical Option]
No Dynastic alliance can be formed by \TUR with any other player.

\aparag[Loan Treaty]\label{chDiplo:Alliance:Loan Treaty}
Only players that have agreed on a Loan treaty can lend money from one to the
other. One is referred to as the "lender", the other as the "borrower".
\bparag The sole possibilities for a player to give money to another are by
Dynastic Ties (as a dowry), by Peace Resolution or by a Loan Treaty.%  Small
% amounts of money can also be transferred
% when buying a Diplomatic support.
\bparag Money transferred by loan treaty is recorded on \lignebudget{Gifts and
  loans between players}.
\aparag[Restrictions on loans]
\bparag Powers having different religions and signing Loan Treaty lose 1 \STAB
if they transfer 50 \ducats or more to the same borrower in one turn.
% (Jym) religion != standing, removing:
%  Catholic, be they conciliatory or counter-reform, share the same
% religion for this rule.
\bparag[Exceptions.] \FRA, if Catholic/Conciliatory, and \ENG beginning with
Period IV, may lend money to any \MAJ with no penalty for Religion. \HOL,
after being recognised by \SPA (see \ref{pIII:Dutch Revolt}), may also lend
money to any \MAJ with no penalty for Religion.
% \bparag To be valid, the moneylender has to transfer at least 30 \ducats to
% the borrower on the turn the treaty is concluded.
\bparag The lender can not give more than 150 \ducats per turn to a given
borrower. %at the diplomatic phase;
\bparag[Exception.] \HOL or \ENG if it has created its Stock Exchange
(\ref{pIII:Amsterdam Stock Exchange} and \ref{pIV:London Stock Exchange}) can
transfer up to 250 \ducats per Loan.
% PB:Loans during turns are not needed with v2 comptability
% \bparag
% The moneylender may transfer additional funds during the Military Phase (at
% end of round), up to 50 \ducats (or 100 \ducats for \HOL or \ENG after
% events \eventref{pIII:Amsterdam Stock Exchange} and \eventref{pIV:London
% Stock Exchange}).
\bparag Restriction: during one turn, the lender is forbidden to lend more
than his gross income when adding all the transfers made.
\bparag A given Power can not be both borrower and moneylender in different
Loan Treaties at the same time.
\aparag[Modalities of refunding]
\bparag Modes of pay-back and interest are left to the discretion of
players. The "loan" can be even a gift without refund.
\bparag The treaty remains valid as long as the borrower has not paid back all
received ducats. Other loans can be concluded on following turns, but always
in the same way (moneylender to borrower). No new, additional, loan treaty can
be concluded between these two players as long as that one remains valid.
Loans that are gifts end at the end of turn.
\bparag The borrower can break the treaty at any time, and refuse to pay
interest and/or the capital owed to the lender. In such a case, he loses
immediately {\bf 1} \STAB level and receives a negative modifier
% of \bonus{-2} on the Loans table until the end of the period currently
% played.  (Jym) to compta v2. Same as bankrupcy.
of \bonus{-1} on the Exchequer test during 5 turns.
\bparag The moneylender may freely abandon the Loan and transform it in a gift
at any Declaration Phase of a following turn. This ends the Treaty.
\bparag If an event releases a \CB between the moneylender and the borrower
and that the war is declared between them, the treaty is immediately broken
without penalty. In such a case, no back payment or reimbursement is to be
made by the borrowing country.

\aparag[Defensive Alliance]\label{chDiplo:Alliance:Defensive Alliance}
A player linked to another player by a defensive alliance may has to declare
war on any other country that attacks his co-signer. He benefits from a \CB
for this specific declaration.

\bparag The Alliance is effective to be used on the turn of its contracting.

\bparag The player can either enters the war by its own will or if the
co-signer ask him to honour the alliance.

\bparag If the player is called by his ally and refuses to declare war along
with his co-signer, he immediately loses {\bf 2} \STAB levels and the alliance
is cancelled. The co-signer also receives a temporary \CB against the
defaulting player.

\bparag The co-signer player may also prefer not to call for his Ally (or
Allies). In this case, the allied player is left free to declare his
participation in this war (with a \CB) or not. If the Ally chooses not to
participate, he suffers no penalties and the Alliance is not considered as
broken.

\bparag If a secret alliance is called for and the co-signer refuses to
declare the war in response of this alliance, the loss is reduced to {\bf 1}
\STAB instead of 2. The betrayed power still has a temporary \CB against the
defaulting power.

\bparag This Alliance lasts for the duration of the next 2 turns, except when
and if cancelled by events or voluntary cancellation by one (or even both) of
the co-signer.

\bparag All declarations of war by this way cost only 1 \STAB level (whatever
number of declarations in the current turn).

\bparag When the players are forming an "Alliance", they have to sign together
the same peace with their enemies. With a minor country: count all the
modifiers enemy minor/all allies. If peace is accepted, the allies must share
the gains. With a major country: as for minor country, except for allies make
an average of their Stabilities (rounded down).

\bparag If at war against the same enemy, all allied players move and play
together (at the lowest player's Initiative rating rank).

\bparag If an ally is twice at {\bf -3} \STAB at the phase of Peace, he must
sue for peace, and sign a separate peace. In this case, his Alliance is not
considered as voluntarily broken (and there is no \CB).

\aparag[Offensive Alliance]\label{chDiplo:Alliance:Offensive Alliance}
Same as for the preceding type of alliance, except that it applies also in the
case where the co-signer is at the origin of the war declaration on another,
third-party player or minor.
\bparag The details are the same as for a Defensive Alliance.


\subsubsection{The Trade Refusal}
\aparag A player can refuse the access of his market to the foreign trade of
another player, even in peace, but that costs him 1 \STAB level at the moment
he announces his decision. Once taken, the decision can be maintained from one
turn to the other (without any additional decline in Stability); the decision
can be repelled later by the power at no cost.

\aparag[Reaction of the Other Player]
\bparag The other player whose trade has been denied then receives a temporary
Commercial \CB against the player refusing him trade. This \CB is to be used
in the segment of Declarations of Wars and starts a new war.
\bparag Alternatively, he may refuse his own trade in reaction and
reprisal. He then suffers from the same effects (loss of 1 level of \STAB, and
Commercial \CB to the enemy). This is to be announced immediately.

\aparag[Value of Trade Refused]

\bparag When a player is refused the trade of another, a foreign trade loss is
assigned: it is calculated on the basis of the refusing player's European
Trade value, i.e. the income of the refusing player's provinces, including
vassals. This value is added to the amount of the European Market that is
denied as foreign trade.

\bparag The player being refused trade gains also no income from Commercial
fleets in the own \CTZ of the refusing \MAJ (neither regular nor monopoly
incomes).

\bparag The player being refused trade gains half of its usual income (trade
plus monopoly income if he has the monopoly) in some \STZ, depending of the
\MAJ that refuses Trade:
\begin{itemize}
\item \TUR: \stz{Caspienne}, \stz{Noire W}, \stz{Ionienne};
\item \VEN: \stz{Lion}, \stz{Noire W}, \stz{Ionienne};
\item \POR: \stz{Canaries}, \ctz{Espagne};
\item \POL and \SUE: \stz{Baltique}.
\item \HOL and \ENG: \stz{Nord}.
\end{itemize}

\bparag These \TradeFLEET still count toward ownership of trade centres and
the income of trade centres is not affected.
% (Jym). Otherwise there could be an easy way to get the CC Med for TUR.

\aparag A Trade Refusal breaks any past Loan Treaty between the two Powers
with no further penalties. It forbids any Loan Treaty as long as the Trade
Refusal continues.


\subsubsection{Others Announcements}
\aparag Others announcements can be made during the Diplomatic phase.
\bparag Most of them come from events specifying that a given choice must be
made ``as a diplomatic announcement''.
\bparag Speculation on exotic resources is made as a diplomatic
announcement. See~\ref{chAdministration:Speculation} for the effect.
\bparag Trade of wood is decided as a diplomatic
announcement. See~\ref{chIncomes:Wood} for the effect.

% Local Variables:
% fill-column: 78
% coding: utf-8-unix
% mode-require-final-newline: t
% mode: flyspell
% ispell-local-dictionary: "british"
% End:
