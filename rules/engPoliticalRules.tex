% -*- mode: LaTeX; -*-

\section{General Political Rules}

\subsectionJ{Instability of the \sectionregion{Balkans}}{\blason{balkans}}
\label{chSpecific:Balkans}

\aparag The following provinces are affected by specific rules:
\province{Alabania}, \province{Hellas}, \province{Moreas} (controlled by \TUR
in 1492), \province{Dalmacija}, \province{Corfu} (controlled by \VEN in 1492)
\province{Bosna}, \province{Serbia}, \province{Montenegro} (\ville{Ragusa})
(independent in 1492).
\aparag The ownership of each province is given at the beginning of the phase
of Peace to the Major Power that controls it.
\aparag Those provinces are a zone of permanent war. Any Major Power can send
troops herein and attack armies and cities, build \Presidios, etc.  without
declaration of war.
\bparag Sieges cannot be continued from one turn to the other (excepted if
there is a regular war). Besieging forces has to retreat (but may pillage, and
build \Presidios).
\bparag The \hab may also campaign in this zone, even if active elsewhere.
\bparag \pays{Hongrie}, if currently inactive, can make a limited intervention
with up to one \ARMY\faceplus in the zone. The intervention is decided and
resolved by its diplomatic controller. Its forces has to retreat at the end of
the turn and if it controls a province of the \region{Balkans} at that moment,
the province becomes independent.
\aparag \TUR may use Privateers in \stz{Ionienne}, \stz{Egee} and \ctz{Venise}
against any Christian countries without declaring war.  Conversely, Christian
countries may fight against those Privateers.
\aparag The provinces of the \region{Balkans} are in the zone allowed to
prosecute Overseas Wars.

\aparag[End of the Specific Status]
\bparag This rule ends when the period II ends or if \pays{Hongrie} falls
apart according to \eventref{pI:Fall Hungary} (but not if only
\eventref{pI:Habsburg Hungary} has happened).
\bparag At that time, the regular ownership of each province is given to the
power that controls it. Independent provinces are given to \pays{Hongrie}, or
\pays{Transylvanie} if it is no more, or to \HAB if only \eventref{pI:Habsburg
  Hungary} happened.



\subsection{The Religious Struggles}

The religious aspect of most conflict is important, in particular in terms of
victory objectives for the players. The following rules give explanations to
that part of the game.


\subsubsection{Sole Defender of Catholic Faith}\label{chSpecific:Catholic
  Faith}
\aparag That title is also a period objective for some Catholic countries and
is defined as follows.

\aparag If \FRA, \SPA, \ENG or \POL is the only Catholic major country (do not
count \POR or \VEN), it is automatically the Sole Defender of the Catholic
faith.
\bparag Alternatively, if there is only one Catholic \MAJ that is
Counter-Reformation, it is automatically the Sole Defender of the Catholic
Faith.

\aparag Else, if more than one country is Catholic, a Catholic \MAJ becomes
the Sole Defender of the Catholic Faith when the conditions below are
fulfilled:
\bparag Control of the \pays{Papaute} diplomatic marker for at least 3 turns
in the last 5 turns just elapsed.
\bparag Participated in all \terme{Crusades} that happened in the last 5
turns, with at least one Simple Campaign per round and no separate peace with
\TUR.

% \aparag[Defender of the Faith]
% The \terme{Sole Defender of the Catholic faith} has a permanent \CB against
% \TUR, until the End of Religious Struggles.

\aparag[Restoration of Catholicism after the Reform] Each time a player
(either \SPA (if Counter-Reformation) or the Sole Defender of the Catholic
Faith) declares war on a Protestant country and obtains an unconditional peace
from this country, the player may decide to abandon all province(s) transfer
and ask as sole peace condition the restoration of Catholicism (of the same
attitude of the power imposing Catholicism) in this country.
\bparag This clause is mandatory if the power is Counter-Reformation.

\bparag[Gain on Conversions]
For each Protestant country thus forcibly converted, the victorious Catholic
player receives 10\VP if the loser is a minor country and 20\VP if it is a
player (or the value indicated in the period objective, if any, which takes
precedence).  Also, if an event gives different values, apply them and ignore
the above.

\aparag[Effect of the Reconversion]
All reciprocal permanent \CB between these players and/or minors are cancelled
following the restoration of Catholicism.  A major country (i.e. a player) who
is reconverted more than two turns after its initial change of religion
suffers from the following side effects, each turn during the next 10 turns,
if it maintains the new (forcibly imposed) religion:
\bparag The country loses 1 additional level of Stability each turn.
\bparag Every even-numbered turn, the player must roll on the revolt table, in
addition to any revolt mandated by events or the rules.

\aparag[Revert to the Reformation]
\bparag If a major country reverts to its religion as it was before the
reconversion, it loses 1 Stability level and grants a temporary CB to the
country that imposed the change of religion on it. In such a case, side
effects as per above are cancelled.
\bparag On the other hand, a forcibly reconverted minor country returns to the
Protestant faith at the start of the turn following its reconversion without
any side effect.


\subsubsection{End of Religious Struggles}
\aparag The Religious Struggles between Protestant and Catholic end in 1664
(Interphase of turn 35 and 36, beginning of period V), or when
\eventref{pIV:TYW} is ended and the year is 1615 or after (turn 26, period
IV). This time is name \terme{End of Religious Enmities} in the rules and
tables.
\aparag The Religious Struggles between other religions (Catholics,
Protestants, Orthodoxes, and Islam) end in 1614 (Interphase of turn 25 and 66,
beginning of period IV), excepted between Shiites and Sunnites.
\aparag All the above rules no longer apply from this time onwards, as well as
some other rules or modifiers.


\subsubsection{The Islamic Schism}\label{chSpecific:Islam}
\aparag As Defender of the Sunni Islam, \TUR has a permanent \CB against
\pays{Perse}, \pays{egypte} and \pays{damas} which are Shiite Muslim
countries.
\aparag \TUR can make no diplomacy on \pays{Perse} or \pays{Ormus} until 1615
(turn 26, period III).

\subsectionJ{The Ottoman advance}{\blason{turquie}}
\label{chSpecific:Crusades and Vienna}


\subsubsection{Crusades}\label{chSpecific:Crusades}
\aparag During periods \period{I}-\period{III}, each turn \TUR annexes a
Christian province, a test for Crusade occurs at the end of the turn.
\bparag See~\ref{chPeace:Crusade} for the details.

\subsubsection{Turkish Capture of \sectionville{Wien}}\label{chSpecific:Fall
  Vienna}
\aparag If the Turks capture \ville{Wien}, the following effects are
activated:
\bparag \HAB loses 1 \STAB immediately. \TUR receives 25 VP for the capture of
\ville{Vienne}, but only once in a game.
\bparag Any Catholic power may do an immediate limited intervention in the war
on the side of \HAB if not at war against them, on no \STAB loss.
\bparag If the Turks still hold the city at the end of the turn, \VEN, \FRA,
\ENG, \SPA, \AUT and \POL (if Catholic) lose 1 \STAB level each.  Other
countries are not affected.  \TUR receives 25 VP more for the capture and
holding of \ville{Vienne}, but only once in a game and \HAB loses 25 VP
because of the same event.

\aparag[\sectionville{Wien} and the Crusade]
The capture of \ville{Wien} gives a bonus of +5 to the Crusade die-roll.

\aparag[Turkish Control of  \sectionville{Wien}]
Each turn where \ville{Vienne} stays under Turkish control, \HAB loses 1
Stability level per turn in addition to all other losses of turn-end.
\bparag \MAJHAB receives a bonus of +3 for all its diplomatic actions (and
Entry in War tests) with minor countries that have a common frontier with
\AUSaus. This effect remains even if \ville{Vienne} is ceded to Turkey during
a peace, and this until \ville{Vienne} is Habsburg/Austrian again.

\aparag[Transfer of the Austrian Capital]
If the province \province{Osterreich} is ceded to Turkey, the capital of the
\pays{Habsbourg} minor country is transferred to any other city in a
\pays{Habsbourg} province, at the choice of the \HAB player.

\bparag If so, \HAB ceases to lose 1 Stability level per turn.

\bparag The new capital can be again conquered by the Turkish player, but in
that case its capture brings no special VP bonus to the Turkish player. It
also does not cost any special VP to \HAB/\AUS.

\bparag \ville{Vienne} become automatically and immediately the capital of the
\pays{Habsbourg} again if the province of \province{Osterreich} is
re-conquered by the \HAB player. The province is immediately annexed without
need for Peace.


\subsubsection{Turkish Capture of \villeRoma}\label{chSpecific:Fall Roma}
\aparag If the Turks capture \ville{Roma}, the following effects are
activated:
\bparag Any Catholic power may do an immediate limited intervention in the war
against \TUR if not at war allied to them, on no \STAB loss.
\bparag If the Turks still hold the city at the end of the turn, each Catholic
power loses 1 \STAB level.

\aparag[\villeRoma and the Crusade] The capture of \villeRoma creates an
immediate call for Crusade in periods \period{I} and \period{II} and gives a
bonus of \bonus{+5} to the Crusade die-roll afterwards.

\aparag[Turkish Control of  \villeRoma]
Each turn where \villeRoma stays under Turkish control, the \SDCF loses 1
\STAB level per turn in addition to all other losses of turn-end.

\aparag[Transfer of the pope]
If the province \provinceLazio is ceded to \TUR, the pope is transferred to
\VENven, see~\ruleref{chSpecific:Venice:Pope Venice}.
\bparag If so, the \SDCF ceases to lose 1 Stability level per turn.
\bparag \villeRoma become automatically and immediately the capital of
\payspapaute again if the province of \province{Lazio} is re-conquered by any
Catholic power. The province is immediately annexed without need for Peace.



\subsection{The Wars of Succession}

\label{chSpecific:War of Succession}
% RaW: 53.35 RaW: receive dowry + DC => WoS.  Jym: converting into DC =>
% existing wars can be turned into WoS if both christian.  received dowry + DC
% => WoS (both christian).


\subsubsection{Conditions}
\aparag Wars of Succession may occur whenever a country suffers from a
dynastic crisis. The country suffering from the crisis (and, potentially, from
the War) is called here the victim.
% Jym: addendum

\aparag If the victim is christian and at war against at least one other
christian major country, then its enemies can decide to turn the on going war
into a war of succession (thus supporting a dissident monarch).
\bparag In that case, one of the christian enemies of the victim is designed
(by its alliance) as the pretending power.
\bparag In the case when several separate alliances are at war against the
victim, each can decide to support a separate pretending power (several people
are pretending to the throne, supported by different powers).

\aparag Any country (any religion) that gave a dowry to the victim in a
Dynastic Alliance signed less than 8 turns ago has, in reaction, a free \CB
against any alliance that turned the war into a war of succession. If it is
used, this country is called the supporting power.
\bparag The supporting power is automatically allied to the victim country for
the current wars of succession.
\bparag If the supporting power is part of an opposing alliance, it can still
choose to support the new monarch of the victim by breaking its alliance (and
paying the usual cost in \STAB).
\bparag There can be at most one supporting country. If several meet the
conditions, the victim country can ask one (and only one) to support its new
monarch.

\aparag If the victim received a dowry in a Dynastic Alliance signed less than
8 turns ago then any christian country that gave the dowry has a normal \CB to
declare a war of succession on the victim this turn and become a pretending
power.
\bparag If one (or more) other countries also choose to declare a war of
succession, this may lead to several different pretending powers.


\subsubsection{Results}
\aparag In addition to any other peace conditions, dynastic ties with the
victim are added has a possible compensation for the war of succession.

\aparag[Dynastic ties] If a country obtains dynastic ties with the victim, the
following apply immediately:
\bparag Both countries sign a defencive alliance. The victim must answer to
this alliance whenever called for. The country that has the ties can refuse to
answer the defencive alliance, but this voids the ties.
\bparag The victim may not declare war to the country obtaining dynastic ties,
unless with a \CB given by event, for the next $5$ turns.
\bparag If the country having the ties declares war to the victim, this voids
the ties.

\aparag[Claiming dynastic ties] Dynastic ties may be granted either to
pretending or supporting country.
\bparag The pretending country (only) can ask for dynastic ties as a peace
condition. This cancel any status of supporting country of the victim that may
exist.
\bparag If the victim country wins the war of succession (peace of level 1
minimum), then the supporting country (if still at war) automatically gains
dynastic ties.

\aparag[Multiple pretending countries] If there are several pretending
countries, then when claiming Dynastic ties, the following conditions are
added:
\bparag The country claiming the ties becomes the new supporting power and
gains a free \CB against all alliances currently at war of succession against
the victim.
\bparag Refusing to use this \CB voids the Dynastic ties.



\subsection{Using mercenaries}\label{chSpecific:Mercenaries}

\begin{histoire}[Condottieri]
  In the thirteenth and fourteenth centuries Italian city-states were becoming
  enriched by their trade with the Orient. These cities, such as Venice,
  Florence, and Genoa, had woefully small armies and were increasingly
  becoming targets of attack by foreign powers as well as envious
  neighbours. The noblemen ruling the cities soon resorted to hiring companies
  of mercenaries known as condotta (``contract'') to defend their
  territories. Each condotta was led by a condottiere, a term which soon
  became synonymous with ``captain''. The condotierri were the key forces in
  the Italian wars. Later, they were overwhelmed with other forces such as the
  Swiss pikemen, German Landsknechts, English musketeers, French cavalry or
  Spanish tercios, but the use of mercenary forces remained in strong use.

  This term is used there for all the mercenary leaders that can be recruited
  by larger powers during the games.
\end{histoire}
\aparag It is possible to buy mercenary generals, . This has to be done in the
logistics segment.
\bparag All the countries willing to buy mercenaries announce their intention
of doing so.
\bparag A die roll is made on \tableref{table:Condottieri} to see how many
mercenary generals are available for sale.
\bparag The mercenaries are drawn randomly and kept hidden in the pool of
mercenary generals. Each one is sold before the next one is drawn.
\bparag Each interested country makes a hidden bid for the mercenary. All the
bids are revealed simultaneously.
\bparag The highest bidder gains the mercenary general for his service and
pays the corresponding price. In case of equality, a second round of bids is
made among the highest bidders. In case of a second equality, the mercenary is
no more available.
\bparag The mercenaries are revealed only after all mercenaries have been
sold.
\aparag It is also possible to recruit a mercenary explorer or
conquistador. For each of both, all the interested countries have to follow
the same procedure as for the generals.
\bparag However, the number of explorer or conquistador is at most 1
(see~\tableref{table:Condottieri}).  \GTtable{mercenariestable}
% \begin{tablehere}\centering
%   \mercenaire
% \end{tablehere}



\subsection{Use of Missions and Missionaries}\rulelabel{chSpecific:Missions}


\subsubsection{Availability}
\aparag \SPA, \POR, \FRA and \ANG receive Missionaries that help them in
colonial activities. See the specific rules of each power for the number of
Missionaries.
\bparag Missionaries always make the appearance in Europe during the
Interphase.
\aparag Each Missionary is a leader with values [3.0.0] and a bonus (from 1 to
3). He has no turn limit. Missionaries have no stacking limits but their
bonuses are not cumulative.
\bparag A Missionary can lead units in the \ROTW (to employ his \Man) and be
used for movements and exploration; but a replacement general should be rolled
for if a battle happens. He does not obey the Hierarchy and can only be used
if there is no other leader to command the stack. A Missionary always test for
possible death when involve in a battle.
\aparag A Missionary can set a Mission at the phase of Redeployment. The
counter is returned to its ``Mission'' face in the province where it is and
will never revert back to the Missionary face.
\bparag There can only be one Mission in each province. A Mission can not be
set in a province where there are enemy \COL, \TP, forts or Missions. If a
Mission set in a province where there is a friendly fort, this fort is
removed.


\subsubsection{Missionaries}
\aparag A Missionary gives a modifier equal to their modifier (+1 to +3) to
any \TP/\COL placement attempt in the province they are if there is no
established settlement on \Faceplus.
\bparag This bonus is not cumulative with the effects of Conquistadors,
Governors or Missions.
\aparag A Missionary gives a bonus of {\bf -1} when rolling on the
Conquistador table against Natives.


\subsubsection{Missions}
\aparag A Mission is a kind of fort with inherent colonial militia, and cannot
move. As it is a fort, a Mission cost 1\ducats to be maintained each turn.
\bparag In \COL, a Mission gives a \LDE added to the colonial militia.
\bparag If there is no other kind of fortification in the settlement, the fort
of the Mission has to be conquered in order to control the settlement.
\aparag A Mission gives a bonus of {\bf +1} to every \TP/\COL placement
attempt in the Area they are set in.
\bparag This bonus is not cumulative with the effects of Conquistadors,
Governors or Missionaries.
\aparag A Mission gives a bonus of {\bf +1} when testing for a possible
reaction of Natives of Minor countries in the area for their country.
\aparag A \TP with a Mission may be transform in a \COL (even if there is no
city), according to \ruleref{chAdministration:TP to Col}.
\aparag A Mission that is not deployed on a \COL may be destroyed voluntarily
by their owner at the Phase of Redeployment. The now free counter could be
used on the next turn (but not this one).
\bparag A Mission may also be destroyed when conquered by a country of a
different religion.
\bparag If a colonial settlement is destroyed by natives or minor countries, a
Mission therein is destroyed.
\bparag If a colonial settlement is annexed by another power of the same
religion, it may replaced it by one of its own. If by a power with a different
religion, the Mission is destroyed.



\subsection{Commercial specificities}
\label{chSpecific:Commercial specificities}

\subsubsectionJ{Levies of the Sund}{\blason{oresund}}
\label{chSpecific:Sund Levies}
\begin{designnote}
  The fight for the commercial levies collected on trade fleets crossing the
  Sund, the \ville{Copenhague} strait, nourished the conflict between Denmark,
  Sweden and the commercial nations such as England and Holland that dominated
  the trade of the area.
\end{designnote}
\aparag One country has the Rights to make Levies on the trade passing through
the Sund; the effect is explained here.
\bparag At the phase of Diplomacy, the country has to announce whether it will
take those Levies or will let trade free.
\bparag If a \MAJ was taking the levies and chose to let them, it immediately
lose one \STAB.
\bparag The country can only take the Levies if, added to the Rights, it
military controls one province among \province{Skane},
\province{Vastergotland} or \province{Sjaelland}.
\bparag A minor country that has the Rights on the Levies on the Sund, will
take them if it is fully at war. It may take them at controller's choice in
others cases.
\aparag[Effects of the Levies]
\bparag A \MAJ earns 5\ducats plus 1\ducats for each level of Commercial Fleet
in \stz{Baltique}.
\bparag Only the \MAJ raising taxes, or the diplomatic patron of a \MIN
raising the taxes, may receive the income and \PV for a monopoly in
\stz{Baltique}. If the Monopole belongs to another \MAJ, this \MAJ has a
Commercial or normal \CB (his choice) against the country levying the taxes
this turn.
\bparag If \pays{Danemark} levies the taxes, it adds one \LD to its
reinforcements this turn, and has a second \ARMY counter at its disposal.
\bparag If another \MIN levies the taxes, it gains nothing.
\aparag[Taking the Rights on the Sund]
\bparag The Rights to do levies on the Sund are obtained as one condition of
Peace (in place of a province), or as equivalent to one province in Dynastic
Ties.
\bparag In Peace, the Rights count as 2 Peace Conditions excepted if the
winner of the Peace will own at least one province among \province{Skane},
\province{Vastergotland} or \province{Sjaelland} at the conclusion of the war,
or has monopoly in \stz{Baltique} -- in that case it counts as 1 Peace
Condition.
\aparag[The Sund and \sectionpays{Danemark}]
\bparag In 1492, \pays{Danemark} has the Rights on the Levies on the Sund.
\bparag Whenever \pays{Danemark} signs a victorious peace, it takes back the
Rights on the Sund, even though if this condition is not part of the Peace
Treaty. In this case, the previous owner of those Rights has a free \CB
against \pays{Danemark} on the following turn if it was not on the losing side
of the Peace.
\bparag The country having the Rights on the Sund can give them back to
\pays{Danemark} as a diplomatic announcement. The country gains a {\bf +2} on
diplomatic actions on \pays{Danemark} this turn.
\bparag \pays{Danemark} is the only minor country that consider taking the
Rights on the Sund as a valid condition of peace.


\subsubsection{Commerce and Wars in the Baltic Sea}
\aparag Raise the \terme{Blocked trade} (\ruleref{chIncomes:Foreign Trade}) by
75\ducats when \SUE (or \pays{Suede}) and \pays{Danemark} are at war against
each other.
\aparag This effect is not applied to any country that is involved in this war
(because this is then already accounted for).

\subsubsectionJ{Control of the Scheldt (L'Escaut)}{\blason{flandrebrabant}}
\rulelabel{chSpecific:Scheldt} %L'Escaut
\aparag If a \MAJ owns and controls militarily \province{Vlaanderen} and
\province{Brabant}, the \ctz{Hollande} is considered as a \CTZ for him.
\bparag He also gains each turn one fleet or concurrence action in either
\stz{Nord}, \ctz{Hollande}, \ctz{France} or \ctz{Angleterre} (at the player's
choice).  \bparag The \CTZ and \STZ concerned by this rule are marked on the
map by a mark that is also in the \province{Vlaanderen} and \province{Brabant}
provinces.
\begin{designnote}
  Historically, \SPA and \HOL reached an agreement during the Utrecht Treaty
  so that \SPA would not use this possibility, but would be helped by \HOL to
  defend those provinces.
\end{designnote}


\subsubsection{Ragusa}\label{chSpecific:Ragusa}
\aparag Until the end of period III, the owner of \province{Montenegro} (or
controller before the End of \ruleref{chSpecific:Balkans}), has one of the
following advantages due to the commercial fleet of \ville{Ragusa} (player's
choice at the Administrative Phase):
\bparag He receives a free \corsaire\facemoins to be used as its own (using
the Ragusa counter), only in \seazone{Adriatique} (to attack trade in
\ctz{Venise} or \stz{Ionienne});
\bparag He receives an additional\terme{Commercial Fleet Implantation} action
(\terme{Basic investment} only) in any \STZ or \CTZ of the Mediterranean Sea.


\subsubsection{Occupation of the
  Caribbean}\label{chSpecific:Occupation:Caribbean}
\aparag A Power may decide to stay and continue an occupation of a \COL or \TP
in \continent{Caraibes}, and even after a Peace ends the war when it took
control of the place. This is possible only it the power controls the \COL or
\TP and it must leave at least one \LD there else the occupation is
discontinued an voided immediately (with any \de destroyed).
\bparag When occupied, a \TP or \COL counts exactly as if owned by its
occupant (for incomes, monopolies,...)
\bparag The legitimate owner has a free Overseas \CB against the occupying
power.


\subsubsection{The Manila Galleon}\label{chSpecific:Manila Galleon}
\aparag In order to benefit from the \terme{Manila Galleon}, a \MAJ must
fulfil all five following conditions:
\bparag Event \ref{pIII:CCA:Closure China} has been played.
\bparag The \MAJ has a \COL in \granderegionPhilippines.
\bparag The \MAJ has a \COL exploiting gold in \continent{America}.
\bparag The \MAJ knows a path between these two \COL through
\seazone{Pacifique} and no enemy-controlled provinces.
\bparag The \stz{Formose} must contain a \TradeFLEET (any level) of either the
\MAJ or \paysChine.

\aparag Only one power can have the Galleon. If several countries claim the
Galleon, it is given to whoever controlled it the previous turn. If nobody, to
the first power in the following list: \HIS, \POR, \HOL, \ANG, \FRA.
\bparag As a diplomatic announcement, a country having the Galleon may release
it at no cost. It is then given to another country meeting the conditions.

\aparag Each turn, if some resources of \granderegionNankin and
\granderegionCanton are not used, then one (and only one) \COL in
\granderegionPhilippines of the country having the Galleon may exploit these
resources as if they were located here.
\bparag The \COL may thus exploit resources from two or three different areas.



\subsection{Independence of Revolted
  Principalities}\label{chSpecific:Peace:Independence Revolt}

\aparag A \MAJ may grant independence to a group of provinces as a whole,
thereby creating a new minor country. This independence simulates the freedom
and liberties acquired for a group of provinces that are not exactly the
national provinces of the \MAJ.
\aparag This rule does not apply to any group of provinces. The groups are
given hereafter for each country:
\bparag \pays{Vbelgique} for \SPA and \AUS is composed of
\theminorprovincesshort{Vbelgique}
\bparag \pays{Vfinlande} for \SUE and \RUS is composed of
\theminorprovincesshort{Vfinlande}
\bparag \pays{Virlande} for \ANG is composed of
\theminorprovincesshort{Virlande}
\bparag \pays{Vhollande} for \SPA is composed of
\theminorprovincesshort{Vhollande}
\bparag \pays{Vliflandie} for \SUE is composed of
\theminorprovincesshort{Vliflandie}
\bparag \pays{Vlithuanie} for \POL and \RUS is composed of
\theminorprovincesshort{Vlithuanie}
\bparag \pays{Vpommeranie} for \SUE is composed of
\theminorprovincesshort{Vpommeranie}
\bparag \pays{Vnorvege} for \SUE is composed of
\theminorprovincesshort{Vnorvege}
\bparag \pays{Vukraine} for \POL and \RUS is composed of
\theminorprovincesshort{Vukraine}
\bparag \pays{VEastPrussia} for \PRU is composed of provinces that used to be
in \POL (\region{Duche de Prusse} plus some others).
\bparag A group is available for a single country only if this country own 3
or more provinces of the group. Owning the complete group is not required.
\aparag[Granting the independence] A \MAJ may give the independence to a group
if all the provinces of the group he owns (except at most one) have a revolt
\facemoins. This announce is made during the diplomatic phase.
\bparag The country granting the independence loses 2 \STAB.
\bparag A revolted principality can be created several times.
\aparag[The new country] The newly independent country is a minor country that
is put in \RM of the \MAJ that just granted independence to it (or \Neutral if
it was granted independence by two \MAJ at the same time).
\bparag The \MIN has no capital, accepts diplomacy, and may use an \ARMY
counter, 2\LD and a basic force of 1\ARMY\facemoins. These countries are
described in the appendix. The \MAJ that granted independence is always first
in the diplomatic preference (draw at random for simultaneous grant of
independence).
\bparag The new country always uses its reinforcements in \terme{defensive}
attitude (never \terme{offensive}). Since there is no capital, an
unconditional peace can be obtained only through a level 5 peace (or see
below).
\aparag[Relationships with the Granter] Three specific rules may apply to the
relations between the \MIN and the \MAJ that granted the independence:
\bparag The \MAJ has a normal \CB against the \MIN (free \CB if the \MIN owns
national provinces of the \MAJ)
\bparag The \MAJ cannot ask for war compensations at the end of the war (only
provinces)
\bparag The \MAJ may impose an unconditional peace to the \MIN if he
militarily controls all the provinces of the \MIN. In this case, he may annex
all the provinces of the \MIN, even if there are more than 3, and the \MIN
ceases to exist.
\aparag[Independent Holland] \pays{Vhollande} can only exist before
\eventref{pI:Reformation2}, or after dissociation (by
\eventref{pV:WoSS}. Granting independence to (or existence of)
\pays{Vhollande} in between is equivalent to a premature roll of
\eventref{pIII:Dutch Revolt}.
\bparag In this special case, \HOL will use the periods III limits during
periods I and II.

% Local Variables:
% fill-column: 78
% coding: utf-8-unix
% mode-require-final-newline: t
% mode: flyspell
% ispell-local-dictionary: "british"
% End:

% LocalWords: Vlaanderen Sund Copenhague Skane Vastergotland Danemark Baltique
% LocalWords: Scheldt l'Escaut Hollande Nord Angleterre Brabant Interphase TP
% LocalWords: Habsbourg Osterreich Vienne Wien Habsburg Papaute pI pIII HRE
% LocalWords: Perse Ormus Sancta Hellas Rhodos Moreas Lubnan Alep Venise
% LocalWords: Trakya Hongrie Parme pre Toscane Ionienne Egee Transylvanie
% LocalWords: Alabania Dalmacija Ragusa Bosna Kreta Chypre Condottieri pikemen
% LocalWords: condotta condottiere Landsknechts tercios
