\definechapterbackground{Inter-turns Phase}{interphase}
\chapter{Inter-turns Phase}\label{chapter:Inter}

\InterPhase

% RaW: [21, 51]

This is the last phase of the turn and is played simultaneously.  The Players'
State Prosperity is verified because it can influence positively or negatively
their Stability; the Stability is modified by ongoing wars or limited
interventions.  Remove military leaders scheduled to leave the game and take
those scheduled to arrive on the following turn.  Remove all un-named leaders
of major powers, and those of minor powers that are now at peace.  In this
phase, Inflation is determined and affects all players' treasuries. Inflation
increases according to the quantity of gold repatriated during this turn from
the \ROTW.

\begin{todo}
  + replenish natives in \ROTW.
\end{todo}

\aparag[Sequence.]
\InterDetails



\section{Stability adjustment}

\aparag[Effects of ongoing wars]
During a war, at the end of each turn, the \STAB of each participating player
is reduced 1 cumulative level each turn (i.e.. the player loses 1 level on the
first turn, 2 others on the second, 3 on the third and so
on... etc.). However, after 4 consecutive war turns, the loss is limited to -4
levels per turn, and this until the peace is made.
\aparag[Effects of Overseas Wars]
The same reduction of \STAB is applied in Overseas Wars excepted that the loss
is limited to -2 levels per turn.
\aparag[Multiple Wars] If a power is involved in more than one war, only the
hardest lost is applied.
\aparag[Continuing Limited Intervention]
A limited intervention in a war may be continued from one turn to another, but
this costs -1 level in \STAB for the intervening power.
\aparag[Turkey and the Knights]
If \TUR is neither at war, nor Anti-Prosperous, it loses -1 level in \STAB if
the pirate of The \pays{chevaliers} managed to inflict losses on Turkish
commercial fleets.
\aparag[Vienna]
% \aparag[\ville{Vienne}]
If \TUR controls \ville{Vienne}, \HAB (\SPA or \AUS) loses 1 additional level
of \STAB.

% \aparag[Prosperity]\label{chPeace:Prosperity}
% The player checks on his economic form the evolution of his \terme{Gross
% Income} on the last two consecutive turns (including the one just
% played). This evolution affects the player's Stability.
% \bparag[Prosperous power]
% If the gross income has progressed 2 consecutive turns, the Stability
% increases 1 level.
% \bparag[Anti-Prosperous power]
% If it has decreased 2 consecutive turns, the Stability declines 1 level.  PB
% 07/2008: changed to a modifier to Stab Improvement



\section{Placement of leaders}

\aparag[Ransom of Leaders]
Captured Monarchs are given back at this time.
\bparag For Major Powers, it costs them 2 \STAB and they give 200\ducats to
the ransoming power (this is mandatory).
\bparag For Minor Powers, their Monarch is ransomed against 50\ducats.
\bparag Alternativeley, a ransome Monarch of a minor power can be used at the
Peace Segment, to increases the modifier for Peace by a +2.

\begin{todo}
  DEPLACER la majorité de ça !!!!
\end{todo}
\aparag Generals and Admirals to be received (those stated as available from
the start of the following turn) are placed normally.
\bparag[Dates on Military Leader counters]
The date of arrival (turn number) is the figure located on the upper left-hand
side of the leader silhouette.  The date of departure (turn number) figure is
located on the lower left-hand side of the leader silhouette.
\bparag Placement of Newly arrived Leaders These leaders are placed on
military stacks of their owner, located in friendly unbesieged cities or
friendly province.
\bparag Leaders Removal from Game Leaders are removed from the game if their
date of departure corresponds to the turn just played (i.e. they leave the
game at the end of their scheduled turn of departure).
\bparag Remove all un-named leaders of Major Powers; leave those of Minor
Powers at war or those on Revolts (but keep all military forces of Minor
powers in play, even if at peace, for usage on next turn if they are again
involved in a war).

\aparag[Conquistadors and Explorers]
The Conquistadors and Explorers may either be placed in a national province of
the player, or on the \ROTW map, in a \COL/\TP of the player or a province
were a players' leader of the same type was at the end of the turn.
\bparag Explorers may use their full values when traveling from the European
map to the \ROTW map but cannot stay on the European map at the end of a
round, unless as the result of an interception.

\aparag[Governors, Overseas Generals and Admirals]
Generals and Admirals with \$ or @ symbol, or Governors, may only be placed in
existing \COL or \TP of the power. If there is none, they are delayed until
they can be placed.
\bparag Leaders bearing a symbol \$ may only be place in \continent{America}.
\bparag Leaders bearing a symbol @ may only be place in \continent{Asia}.



\section{Inflation}

\aparag The Gold counter placed on the ``Resources and Prices'' track
indicates the percentage of inflation currently in force. The maximum
inflation rate is 33\%.
\bparag This percentage of inflation is the proportion of the Royal Treasure
that is lost by each power at the end of the turn. It is reported in
\lignebudget{Inflation}.
\aparag[Increase of Inflation] The increase of inflation is controlled by one
die-roll. If successful (7 or more), this die-roll will increase the inflation
by 1 box to the right (unless inflation is already at its maximum).
\bparag Inflation can also increase due to specific economic conditions (see
\ruleref{chIncomes:EconomicInflation}) or various events.
\bparag \label{chInter:Inflation Gold ROTW} If the quantity of gold in the
\ROTW that was produced this turn exceeds 100\ducats, the inflation counter is
turned on its other side before the die-roll (success on 3 or more) for the
current turn.
\bparag Inflation increase is done before discounting inflation from the \RT.
\aparag[Limited Inflation] \label{chInter:InflationGold} Players that do not
produce gold in \continent{America} are affected by the inflation as if the
inflation marker was found one box to the left of the box it currently
occupies.
\bparag However, the minimum inflation is at least 5\% for all players.



\section{Trade centres}\label{chInter:TradeCentres}

\aparag The Trade Centres may be moved during the interphase.
\aparag The \emph{Great Orient} centre does not move from \shortprovince{Nil}
as long as both \pays{damas} and \pays{egypte} are not conquered by \TUR.
\bparag At this point, the convoy of \bazar{Izmir} appears, and the
\emph{Great Orient} centre is moved to any coastal province of
\paysmajeur{Turquie} bordering the Mediterranean sea.
\bparag If the province is ceded by \TUR, the \emph{Great Orient} centre has
to be moved to another such province.
\aparag The \emph{Mediterranean} centre is given to the country that has the
greatest total number of fleets in the following zones: \stz{Caspienne},
\stz{Noire}, \stz{Lion}, \stz{Ionienne}, \ctz{Turquie}, \ctz{Venise}.
\bparag If the owner changes or if the province in which the Trade Centre
resides is ceded, the Trade Centre has to be moved to a national port of the
controller (bordering the Mediterranean Sea if there is any).
\bparag If no such province is eligible, the Trade Centre is temporarily not
available.
\aparag The \emph{Indian Ocean} centre is given to the country that has the
greatest total number of fleets in the following zones: \stz{Tempetes},
\stz{Oman}, \stz{Indien}, \stz{Formose}.
\bparag It follows the same rule as above (placed in a National port of the
controller).
\aparag The \emph{Atlantic} centre is given to the country that has the
greatest total number of fleets in all other \STZ and \CTZ.
\bparag It follows the same rule as above, but with the Atlantic Ocean instead
of Mediterranean sea.

% Local Variables:
% fill-column: 78
% coding: utf-8-unix
% mode-require-final-newline: t
% mode: flyspell
% ispell-local-dictionary: "british"
% End:
