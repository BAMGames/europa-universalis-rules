\definechapterbackground{Inter-turns Phase}{interphase}
\chapter{Inter-turns Phase}\label{chapter:Inter}

%\InterPhase

% RaW: [21, 51]

This is the last phase of the turn and is played simultaneously. It mostly
consists in some cleanup of the past turn and preparation for the next one:
Handling old and new military leaders, moving Trade centres, rebuilding some
military assets and giving some \VPs to players.


\begin{todo}
  + replenish natives and Militia in \ROTW.
\end{todo}

\aparag[Sequence.]
\InterDetails

\section{Removal and Placement of leaders}\label{chInter:Leaders}
\begin{designnote}
  Note that removal of leaders must physically occur before placement of new
  ones because it is possible that the same \anonyme counter is immediately
  reused (at the same place or elsewhere). However, they are considered to
  happen simultaneously, especially for the replacement of \LeaderC/\LeaderE
  in the \ROTW.
\end{designnote}

\subsection{Ransoms}
\aparag[Return of the kings] Captured Monarchs return to their countries.
\bparag Major monarch can use their value again.
\bparag The military counter (whether Major or Minor) is placed together with
the new leaders arriving next turn.

\subsection{Removal of leaders}
\aparag[Death]
Each named leader who reached its last turn of activity is removed from the
game.
\bparag For leaders with turns on the counter, it happens if this turn is the
same as the second turn written on the counter.
\bparag For leaders with event on the counter, or other special conditions,
check the description of the event to known how long the leader lasts.
\bparag Named major monarchs are removed during the Monarch Survival phase
(\ref{chEvents:Survival}) at the time where the Monarch dies.
\bparag Note that leaders may be removed earlier due to death in battles.

\begin{designnote}
  Turns of ``life'' may either represent the actual life of historical people
  or their period of military activity. ``Death'' of a leader (whether
  scheduled or during battle) may be either actual death, retirement of old
  age or after a severe wound, change of career (often to become minister),
  fall in disgrace, \ldots
\end{designnote}

\begin{exemple}[Removing named leaders]
  It is Inter-turns phase of turn 46. \leaderwithdata{Marlborough} is
  scheduled to live for turns 43-46. Since this is his last turn of activity
  his counter is removed from the game (historically, he died in 1722, in the
  middle of turn 47).

  \smallskip

  At the beginning of turn 26, \ref{pIV:Bohemian Revolt} occurs. As per Event
  description, \paysBaviere receives \leaderwithdata{Tilly} for 4 turns. Thus,
  he is considered as having turns 26-30 on his counter. At the end of turn
  30, if he is still alive, \leaderTilly is removed from game (historically,
  he died facing \leader{Gustav-Adolf} at the battle of Rain in 1632, during
  turn 29).
\end{exemple}

\aparag[Anonymous]
\bparag All \anonyme leaders of major countries return to their respective
pool.
\bparag Exception: besieged leaders stay on the map.
\bparag All mercenaries leaders return to the pool of mercenaries.
\bparag \anonyme leaders of minors countries fully at peace return to their
respective pool.
\bparag \anonyme leaders of minors at war stay on the map.

\aparag[Free redeployment]
Each player may choose to redeploy any of its named leader still alive. The
counter is removed from the map and will be replaced immediately as a new
leader arriving this turn.
\bparag Exception: besieged leaders must stay on the map.
\bparag Exception: Leaders in the \ROTW with unknown discoveries must stay in
place (hint: you should have redeploy them to an establishment during
voluntary redeployment).

\begin{playtip}
  In other words, named leaders have a free ``teleportation'' movement at the
  end of turn. Use this either to change your frontlines or to ensure
  hierarchy is respected for the next turn. This is also the occasion to
  redeploy \LeaderGov to other Areas.
\end{playtip}

\subsection{Placement of leaders}
\aparag[New leaders]
Each leader who is schedule to be active starting with the next turn is placed
on the map by its controller.
\bparag Each leader that was removed due to free redeployment is also placed.
\bparag Leaders that were wounded but did not recover during the turn are also
placed.

\begin{exemple}[New leader]
  This is the inter-turn phase of turn 42. \leaderwithdata{Marlborough} is
  active starting with turn 43, thus he is placed on the map now. \ANG chooses
  where to place him (see some restriction below).
\end{exemple}

\aparag[Anonymous] Major countries check their minimum leader limit.
\bparag For each category of leaders (\LeaderG, \LeaderA, \LeaderE, \LeaderC,
\LeaderGov) where a country as less leader than its minimum (for the next
turn), draw as many leaders as necessary to reach the minimal value.
\bparag If a country has more leaders than its minimal value, none is
received.
\bparag Beware that you are preparing next turn, hence must check the minimal
leaders value for next turn (this does change at the end of periods).

\begin{exemple}[Minimum leaders]
  At the end of turn 1, \RUS has one \LeaderG (\leaderwithdata{Shchenya},
  provided he did not die during turn 1) but at turn 2 (for period
  \period{I}), \RUS has a minimum limit of 2\LeaderG. So, \RUS gets one
  \anonyme\LeaderG at the inter-turns phase of turn 1 (to be used during turn
  2).

  \smallskip

  At the end of turn 3, suppose \leaderShchenya is still alive. \RUS receives
  \leaderwithdata{I Vorotynsky} as he is scheduled for turn 4. Thus, \RUS now
  has 2 \LeaderG, equal to its minimum limit, and does not receive any
  \anonyme\LeaderG.
\end{exemple}

\aparag[Missionaries] Some countries receive \LeaderMis. Check the specific
rules of the country to know when.
\bparag If this is the case, new \LeaderMis are placed now.

\aparag[Placement] All leaders deemed to be placed at a given turn are placed
simultaneously. That is, one first draws all its \anonyme leaders before
placing any.

\aparag[Where to place?]
\bparag \LeaderMis must be place in an owned, controlled and unbesieged
province of the European map.
\bparag \LeaderGov must be place in an owned, controlled and unbesieged
establishment (\COL, \TP or fort) in the \ROTW, possibly in a \COL of level 6.
\bparag Other leaders (\LeaderG, \LeaderA, \LeaderE, \LeaderC) may be placed
either with any unbesieged troop of the same country or in any owned,
controlled and unbesieged province.
\bparag Additionally, \LeaderE and \LeaderC may be placed in any province or
seazone where another leader of the same category was just removed, even if
there are no more counter of the country here and even if the discovery of
this province or seazone has still not been brought back home (the expedition
is too small to be represented, but there are still some members to take the
lead once the initial leader is dead).
\bparag Note that \LeaderE or \LeaderC may be placed in Europe.

\begin{exemple}[Replacing a \LeaderC]
  At the end of turn 3, \HIS decides to let the lone \leaderwithdata{Colon} in
  \granderegionCuba. At the end of turn 3, \leaderColon dies and the counter
  is removed from game. The province is now empty. However,
  \leaderwithdata{Solis} is scheduled to arrive at turn 4, hence he is placed
  now. Since both are \LeaderE, \leaderSolis may be placed exactly where
  \leaderColon was. Thus, a \COL attempt may be performed here on turn 4 with
  the bonus for the presence of a \LeaderE.

  \smallskip

  \ANG leaves a lone \anonyme\LeaderC on the cost of \continentAmerica. Since
  this is an \anonyme leader, it must be removed at the end of turn. However,
  \ANG receives another \anonyme \LeaderC for the next turn and may choose to
  place him at the same place. Note that since \ANG has only 3
  \anonyme\LeaderC, there is 33\% chance that the new one is the same as the
  old one.
\end{exemple}

\aparag[Geographic restrictions] Some leaders have Geographic restrictions as
where to be placed (America, Asia or Mediterranean).
\bparag In addition to other rules, these leaders must be placed in a province
or seazone where they are allowed.
\bparag That is, \emph{e.g.}, a \LeaderG with a \$ (America) restriction must
be place with a stack or establishment in \continentAmerica and may not be
placed in Europe or Asia.
\bparag Note that leaders without capacity to go in the \ROTW are \emph{de
  facto} restricted to Europe and thus must be placed on the European map.

\aparag[Hierarchy]
After placement of leaders, hierarchy must be respected.
\bparag If not, you should probably have use free redeployment to solve the
problem.

\aparag[\Pashas]
\TUR receives new \Pashas each turn as
per~\ref{chSpecific:Turkey:Pashas}. They are placed now.
\bparag The correct procedure of placement is (i) decide a province where a
\Pasha will be placed ; (ii) draw a random \Pasha and place it here ; (iii)
repeat until all new \Pashas are place.
\bparag That is, \TUR may not wait to see the actual values of a new \Pasha
before deciding where to place him.
\bparag Placement of \Pashas may break hierarchy.

\section{Trade centres}\label{chInter:Trade Centres}

\aparag The Trade Centres may be moved during the interphase.

\aparag[Great Orient] As long as \paysEgypte exists, the \CCs{Grand Orient}
stays in \provinceNil. In the rare case where the province is ceded, the
centre is relocated in any other province of \paysEgypte.
\bparag As soon as \paysEgypte is destroyed, the \CCs{Grand orient} is placed
in \provinceIzmir (if owned by \TUR) or any other Turkish province bordering
the \regionMediterrannee (otherwise).
\bparag If the centre is in \TUR, the convoy of \bazar{Izmir} is available for
next turn, appearing in the province of the centre.

\aparag[Other centres] are attributed to a country.
\bparag They must be placed in any coastal national province of the owner.
\bparag If this is not possible, they may be placed in any coastal province of
the owner.
\bparag There is no limit to the number of centre in any given province.

\aparag[Attribution] Each centre is attributed to the country with the largest
number of \TradeFLEET levels in a given set of \STZ/\CTZ (sum the levels of
all these zones).
\bparag Count the \textbf{current} levels of the \TradeFLEET, not the
\textbf{maximum} levels. Hence, piracy may temporarily change ownership of the
centre.
\bparag In case of tie, if the current owner is amongst the tied countries, it
keeps the centre.
\bparag In case of tie, if the current owner is not amongst the tied
countries, the centre is attributed at random amongst the tied countries.

\aparag[Mediterranean]
\bparag The \CCs{Mediterranee} is attributed to the country with the largest
number of \TradeFLEET levels in \stz{Caspienne}, \stz{Noire}, \stz{Lion},
\stz{Ionienne}, \ctz{Turquie} and \ctz{Venise}.
\bparag If possible, it must be placed in a province bordering
\regionMediterrannee. National non-Mediterranean provinces still have higher
priority that non-national Mediterranean provinces.


\aparag[Indian] The \CCs{Indian} is attributed to the country with the largest
number of \TradeFLEET levels in \stz{Tempetes}, \stz{Oman}, \stz{Indien} and
\stz{Formose}.

\aparag[Atlantic] The \CCs{Atlantic} is attributed to the country with the
largest number of \TradeFLEET levels in all other \STZ and \CTZ.

\section{Natives, Militias and Fortresses}
\label{chInter:Natives, Militias, Fortresses}

\aparag[Natives] In each \ROTW province, the number of natives is replenished
to its maximum.
\bparag Exception: in \continentAmerica and \continentSiberia, if a province
was reduced to 0 natives, it does not replenish (in these areas, natives may
be permanently exterminated).
\bparag Simply remove all temporary \pays{natives} counters.

\aparag[Appeasement] In each \ROTW province, natives are appeased and are no
more activated.

\aparag[Militia] In each owned, controlled and unbesieged establishment, the
number of militia is replenished to its maximum.
\bparag Besieged militias, as well as militia in occupied provinces are not
replenished
\bparag Simply remove all temporary white militia counters.

\aparag[Fortresses] Remove all white level 1 fortress counter in provinces of
countries that are not at war (either at peace or in intervention).

\section{\VPs}
\aparag Some \VPs are earned each turn and are tallied at the end of turn.
\bparag They are described together with all the \VPs in the next
Chapter. See~\ref{chVictories:Turn VP} for details.

% Local Variables:
% fill-column: 78
% coding: utf-8-unix
% mode-require-final-newline: t
% mode: flyspell
% ispell-local-dictionary: "british"
% End:
