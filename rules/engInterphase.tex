% -*- mode: LaTeX; -*-

\definechapterbackground{Inter-turns Phase}{interphase}
\chapter{Inter-turns Phase}\label{chapter:Inter}

% RaW: [21, 51]

\section{Overview}
\aparag This is the last phase of the turn and is played
simultaneously. It mostly consists in some cleanup of the past turn and
preparation for the next one: Moving Trade centres, rebuilding some
military assets and giving some \VPs to players.

\aparag[Sequence.]
\InterDetails

\section{Trade centres}\label{chInter:Trade Centres}

\aparag The Trade Centres may be moved during the interphase.

\aparag[Great Orient] As long as \paysEgypte exists, the \CCs{Grand Orient}
stays in \provinceNil. In the rare case where the province is ceded, the
centre is relocated in any other province of \paysEgypte.
\bparag As soon as \paysEgypte is destroyed, the \CCs{Grand orient} is placed
in \provinceIzmir (if owned by \TUR) or any other Turkish national province
bordering the \regionMediterrannee (otherwise).
\bparag If the centre is in \TUR, the convoy of \bazar{Izmir} is available for
next turn, appearing in the province of the centre.

\aparag[Other centres] are attributed to a country (see below).
\bparag They must be placed in any coastal national province of the owner.
\bparag If this is not possible, they may be placed in any province of the
owner.
\bparag There is no limit to the number of centres in any given province.

\aparag[Attribution] Each centre is attributed to the country with the largest
number of \TradeFLEET levels in a given set of \STZ/\CTZ (sum the levels of
all these zones).
\bparag Count the \textbf{current} levels of the \TradeFLEET, not the
\textbf{maximum} levels. Hence, piracy may temporarily change ownership of the
centre.
\bparag In case of tie, if the current owner is amongst the tied countries, it
keeps the centre.
\bparag In case of tie, if the current owner is not amongst the tied
countries, the centre is attributed at random amongst the tied countries.

\aparag[Mediterranean]
\bparag The \CCs{Mediterranee} is attributed to the country with the largest
number of \TradeFLEET levels in \stz{Caspienne}, \stz{Noire}, \stz{Lion},
\stz{Ionienne}, \ctz{Turquie} and \ctz{Venise}.
\bparag If possible, it must be placed in a province bordering
\regionMediterrannee. National non-Mediterranean provinces still have higher
priority that non-national Mediterranean provinces.


\aparag[Indian] The \CCs{Indian} is attributed to the country with the largest
number of \TradeFLEET levels in \stz{Tempetes}, \stz{Oman}, \stz{Indien} and
\stz{Formose}.

\aparag[Atlantic] The \CCs{Atlantic} is attributed to the country with the
largest number of \TradeFLEET levels in all other \STZ and \CTZ.

\section{Monarchs, Natives, Militias and Fortresses}
\label{chInter:Natives, Militias, Fortresses}

\aparag[Natives] In each \ROTW province, the number of natives is replenished
to its maximum.
\bparag Exception: in \continentAmerica and \continentSiberia, if a province
was reduced to 0 natives, it does not replenish (in these areas, natives may
be permanently exterminated).
\bparag Simply remove all temporary \pays{natives} counters.

\aparag[Appeasement] In each \ROTW province, natives are appeased and are no
more activated.

\aparag[Militia] In each owned, controlled and unbesieged establishment, the
number of militia is replenished to its maximum.
\bparag Besieged militias, as well as militia in occupied provinces are not
replenished
\bparag Simply remove all temporary white militia counters.

\aparag[Fortresses] Remove all white level 1 fortress counter in provinces of
countries that are not at war (either at peace or in intervention).
\bparag Exception: Do not remove the counters in the \ROTW on \COL of level
6.

\aparag[Return of the kings] Captured Monarchs return to their countries.
\bparag Major monarch can use their value again.
\bparag The military counter (whether Major or Minor) is placed together with
the new leaders arriving next turn.

\aparag[No lasting wounds] Leaders that were wounded but did not have time to
recover during the turn are healed.
\bparag They are placed together with the new leaders arriving next turn.

\section{\VPs}
\aparag Some \VPs are earned each turn and are tallied at the end of turn.
\bparag They are described together with all the \VPs in the next
Chapter. See~\ref{chVictories:Turn VP} for details.

% Local Variables:
% fill-column: 78
% coding: utf-8-unix
% mode-require-final-newline: t
% mode: flyspell
% ispell-local-dictionary: "british"
% End:
