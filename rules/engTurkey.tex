\sectionJ{\anchorpaysmajeur{Turquie}}{\blasonJ{turquie}}
\subsection{Internal affairs}
\subsubsection{The Policy of Grand Orient}
\aparag[Trade of Grand Orient.] In 1492, the \CCs{Grand Orient} is in
\ville{Alexandrie}. As long as it is the case:
\bparag \TUR receives half of its income if it owns \ville{Damas}, or if
it has \pays{damas} on its diplomatic track.
\bparag See \ruleref{chSpecific:Mamluks} for the fall of \pays{damas} and
\pays{egypte} and the beginning of the convoy of \ville{Smyrna}.
\bparag See \ruleref{chIncomes:Levant Convoy} about the specific rules for the
convoy of \ville{Smyrna}.

\aparag[Colonial Expansion.] \TUR may only place \COL by land contacts,
i.e., in a province (not an \Area) adjacent to its territory in Europe
or to an existing \COL, or through \seazone{Caspienne}.
\bparag \TUR always ignores restrictions
of~\ref{chExpenses:Pioneering}.
\bparag \TUR has no such restrictions regarding \TP placements.

\aparag If \TUR has a \TP in the \ROTW or an \dipAT with a minor
having a \TP and it has no \LeaderA allowed to go in Asia ('R' or '@'),
then its lowest ranking \LeaderA which is not restricted to the
Mediterranean gain the ability to go in Asia ('@').
\subsubsection{Turkish Military system}
% Jym, 05/2011
% Removed. The yearly campaigns now have another simulation (attrition
% of Timars).
%\aparag[Recruitment Area.]\label{chSpecific:Turkey:Recruitment Area} The
%\terme{Recruitment Area} of \TUR is limited to its capital province
%\province{Trakya} and \province{Angora}.

\aparag[The initial system of \Timar.]
Land counters if the Turkish forces are of two different kinds:
\Janissaire (or equivalent professional forces) and \Timar.

\bparag[Janissaries] \Janissaire are the normal forces of \TUR (same
color, name). They function like the forces of any other power.
Initially, \TUR uses at most 2 \Janissaire \ARMY counters.  Note that
these armies have augmented artillery, and increase the losses in siege
assaults (add \td\ if there is such an \ARMY\faceplus involved, during
periods I--III only).

\bparag[Timars] Other forces are \Timar: counters with the \Timar
mention (different color) and \Pashas units.  \Timar are limited
initially to the \TARQ Technology, and are considered so if \TUR has a
higher level.  \Timar are always \terme{Conscripts} (even those
maintained in the \terme{basic forces}).  If \Timar units are stacked in
battle with other units of a higher Technology, the morale of the stack
is always that of the \Timar (i.e., has the morale of conscript from the
\Timar Technology).  Finally, \Timar units may never be moved by sea
transport.

\bparag \Janissaire and \Timar are not the same kind of forces.  Basic
Maintenance is separated between \Janissaire and \Timar.  Also they can not
absorb \LD from the other kind.

\bparag[Technology limitation]\label{chSpecific:Turkey:Army Tech}
\TUR can not go beyond \terme{Land Technology} \TMUS, and \terme{Naval
  Technology} \TBAT. Its markers must stop before entering a higher Technology
Level.  It has malus of {\bf -1} to the die to raise its \terme{Naval
  Technology}, and {\bf -1} to the die to raise its \terme{Land Technology}
unless if it is currently less than \TARQ.

%   v1.0
%\bparag[Yearly Campaigning] All Turkish forces suffer a malus of {\bf
%  +2} to every Attrition test (movement, \LOS or besieging). Exception:
%  force besieged. 

%   v2.0
%\bparag[Yearly Campaigning] Any stack containing at least one \Timar
%counter (\ARMY, \LD or \Pasha) not in a controlled province at the end
%of a \terme{Winter round} must do an attrition test in the next supply
%segment (or redeployment segment if this is the last round of the turn).

%   v3.0, [not satisfactory in 08/2008]
%\bparag[Yearly Campaigning] Any stack containing at least one \Timar
%counter (\ARMY, \LD or \Pasha) double the length of its Line of Supply.

%   v4.0, after session 08/2008 
\aparag[Yearly Campaigning.]\label{chSpecific:Turkey:Yearly Campaigning}
At the passing of each Winter box (end of it, or if bypassing), there
is a specific attrition test on all stacks containing \Timar. 
\bparag Roll 1d10+ 2 times the distance in provinces to the National
Territory of \TUR (Note: count through provinces controlled or owned,
\pays{Egypte} and \pays{damas} counts as National Territory here once
owned).
\bparag Read the result on the Attrition table, crossed with the number
of \Timar detachments (only) and ignore the P results.
\bparag Each loss is a \LD of \Timar that goes home.  Regular \Timar \LD
that go home are given back freely (above all construction limits) at
the beginning of next turn, or can be raised at a following round at
half cost (not counting in the limits).  \Pasha units that go home will
be raised in addition to usual reinforcements at the beginning of next
turn.


\begin{designnote}
  The preceding rule simulates the limit in supply of their
  kind-of-feudal forces that were to withdraw almost every winter.
\end{designnote}

\aparag[Pashas]\label{chSpecific:Turkey:Pashas} 
\TUR has a certain number of \Pasha units at his disposal.  
Each one is similar to a general, with
military values and a hierarchical rank, that is accompanied by their
own intrinsic troops. A \Pasha has a standard military force of one
\Timar \LD for each number of force increment.
\bparag[New \Pashas.] \TUR has a maximal number of \Pashas in play
equals to its number of owned provinces divided by 3 (rounded down).
Each turn, it receives new \Pashas up to this limit.
\bparag During the administrative phase of each turn, \TUR receives a
number of new \Pasha units equal at most to its \STAB, plus 1 if
\pays{Hongrie} has fallen according to \eventref{pI:Fall Hungary}, plus
1 if \pays{egypte} has been conquered. If this number is negative, no
new \Pasha is received (but none lost).
\bparag \Pasha counters are taken randomly among those not yet placed on
the map.
\bparag If \TUR controls more provinces than he has available \Pashas to
rule them, the extra is lost and placement cancelled.
\bparag[Placement of \Pashas.] New \Pashas can only be placed in owned
provinces where there is no \Pasha nor in any adjacent province, and
that is not \TUR capital. They must be placed in \TUR national
provinces, except that one at most can be placed directly in former
provinces of \pays{damas} or \pays{irak}, and one can be placed in
\region{Balkans} or former provinces of \pays{Hongrie}.

\aparag[Corruption Cost of \Pashas.]\label{chSpecific:Turkey:Cost Pashas} 
%\bparag \TUR always uses the maximal inflation (even if it exploits no
%Gold in \ROTW).
\bparag \TUR always uses the inflation as if it was exploiting gold in
\continent{America}.
\bparag Some \Pashas may become corrupted. They are flipped on their
corrupted side. On this side, a \Pasha cannot move, it has no intrinsic
force, it is not a military leader. Its only effect is to nullify all
the incomes coming from the province it is in.

\bparag[Decadence.] One \Pasha becomes corrupted when one or more of the
following situations occur:
\begin{enumerate}
\item \TUR raises exceptional taxes (see \ruleref{chIncomes:Exceptional
    Taxes})
\item \TUR suffers bankruptcy (see \ruleref{chExpenses:bankruptcy}); 2
  corrupted \Pasha if large or complete bankruptcy
\item \TUR exceeds its \MNU limits (see \ruleref{chThePowers:Exceeding
    Limits})
\end{enumerate}
\bparag Some \Pashas become corrupted also because of economical events
(\subeventref{eco:Fiscal:Pashas Corruption}) or of failed attempts at
Reform.
\bparag The newly corrupted \Pashas are chosen randomly among those that
are not. The \SDCF places each of them in the province it is in, or any
adjacent province where there is no \Pasha; or, if the \Pasha is not in
a Turkish owned province, it has to be placed in any Turkish owned
province where there is no \Pasha in, and which is not adjacent to
another \Pasha unit. Once placed, a corrupted \Pasha can be moved in
only two instances: the \Pasha is dismissed, or the ownership of the
province is lost by \TUR (in which case the corrupted \Pasha is replaced
as above).

\bparag[Dismissal of \Pashas.] The Turkish player can dismiss (or
impale...!) a \Pasha at any given time during the Redeployment phase
(replace the removed \Pasha among those not yet in play). It is not
possible if \TUR is at war or has a negative \STAB.  Every time a \Pasha
is removed, the Turkish player loses 1 \STAB level for each \Pasha that
is dismissed.

\aparag[\Pashas as military units.]
\bparag When at peace, \TUR must move its \Pashas so that there is a
maximum of one \Pasha per province at the end of the turn. In addition,
no \Pasha may ever finish its move in the Turkish capital province.
\bparag When at war, the \Pashas can be moved without any constraint of
placement but must respect hierarchical rank constraints. They can not
go in \ROTW.
\bparag Being \Timar units, \Pashas are always \terme{conscripts} (and
their stack also, disregarding the presence of other \terme{Veteran}
units), are limited initially to \TARQ, and can not move by sea.
\bparag[Stacking of \Pashas] The Turkish player can stack up to 2
\Pashas in addition to the other military units in a stack. This is an
exception to the rule that limits to 3 the number of units in a
stack. They are counted as their value in \LD for attrition and battle
purpose (but not for activation and hierarchy).

\bparag[\Pashas and Hierarchy] Even if \Pashas can be used as generals,
they are disregarded as generals to enforce the hierarchy. They can only
command if there is no regular general.
\bparag The force content in \LD of a \Pasha cannot be incorporated in
any other military unit (and conversely).
\bparag Losses undergone during combat or attrition can be attributed to
an engaged \Pasha, up to the number of \LD part of that \Pasha counter,
using the normal rules of assignation of losses.  In this case, whatever
is the supported loss, the \Pasha is lost. %(to be replaced randomly by
%another \Pasha during the redeployment phase of the first turn following
%peace).

\subsubsection{Turkish Reformation}
\aparag[Attempts of Turkish Reformation.] \TUR may attempt to reform the
government and the military system during the play.  The 7 steps of the
reformation are divided in two groups: government and military, and in
three levels of progression. Each level of progression (both groups) has
to be finished for any attempts on a higher level to be allowed.
\bparag This is an Administrative Domestic operation (and takes the
place of the allowed Domestic operation of the turn,
see~\ruleref{chExpenses:Administrative Limits}) that has a fixed cost of
100\ducats.
\bparag A given Sultan can make only one attempt of reform during his
whole reign.
\bparag No reformation attempt can be made before the death of
\monarque{Suleyman}. That is, the first sultan allowed to attempt a
reform is the heir of \monarque{Suleyman}.
\bparag A its last scheduled turn of life, no Sultan may attempt an
Administrative Reform.

\aparag A test is made on \tableref{table:Administrative Actions},
using the column (\MIL or \ADM)+\DTI-9, with the following die
modifiers:
\begin{modlist}
\item[\textpm?] Stability of country 
\item[-?] the Level of the attempted reformation
\item[-?]  the number of corrupted \Pashas
\end{modlist}
\bparag A ``S'' result is a success: the attempted reform is activated,
-1 in \STAB, roll for one revolt in \TUR and the Turkish monarch has a
malus of {\bf +2} to its Survival die roll next turn.
\bparag A ``\undemi'' result is a failure, -1 in \STAB and roll 1d10
against \FTI:
\begin{itemize}
\item if higher than \FTI, the monarch is killed, 1 \Pasha is corrupted
  and the next monarch will not be allowed to attempt a reform of the
  same group,
\item else roll for one revolt in \TUR and the Turkish monarch has a
  malus of {\bf +2} to its Survival die roll next turn.
\end{itemize}
\bparag A ``F'' result is a complete failure: death of the monarch and
dynastic crisis, 2 \Pashas are corrupted and the next monarch will not
allowed to attempt any reform.

\aparag[Government Reformation]
\bparag[Level 1: Elder Succession]
The effect of dynastic crisis (\ruleref{chEvents:Dynastic Crisis}) for
\TUR is reduced from now on: the only effect is a -1 in \STAB.  When
rolling for a new Sovereign in \tableref{table:Reign}, \DC and
\terme{Fragile Health} are always ignored and the length of reign of new
monarchs is changed : for die results 1 to 7, the result is divided by 2
(rounded down), and results 8, 9 and 10 are for (respectively) teen,
child and baby monarch lasting 6, 7 or 7 turns.
\bparag[Level 1: Reforms against corruption.]
Exceptional taxes are no more causes for corruption of \Pasha
anymore. When this reform is achieved, the Turkish player may remove up
to 4 corrupted \Pasha units at no cost.
\bparag[Level 2: End of feudality]
All corrupted \Pashas are removed and there can be no new corruption of
Pasha anymore. From now, \TUR uses normal inflation. Gives a bonus 
of {\bf +1} to all further attempts of Reformation.

\aparag[Military Reforms] The effects are summarized on a table on the
Turkish aid of play.

\bparag[Level 1: Development of the \Janissaire Corps]
The number of \Janissaire \ARMY available is now 4, and the number of
\Timar \ARMY decreases to 2. The troop pruchase limit is reduced by one \LD.
Basic Maintenance changes: add \ARMY\facemoins to \Janissaire and
remove \ARMY\faceplus to \Timar.
The \Janissaire forces lose their increased casualty
in assault and their size is now of the smaller reformed type.
\\
Regarding Technology, the new limit is \TBAR and the malus to improve
it is applied only if \terme{Land Technology} is \TMUS or better.

\bparag[Level 1: Reforms of the Sipahi and of the Navy]
Basic Maintenance changes: remove \ARMY\faceplus to \Timar.
\\
Regarding Technology, the new limit is \TMUS and \Timar forces can now
be \TMUS.  The malus to improve terme{Land Technology} is applied only
if it is \TMUS or better.  The \terme{Naval Technology} \TBAT is now
accessible.

\bparag[Level 2: Reduction of \Timar]
All Turkish forces are now \Janissaire forces.  Counters of \Timar \ARMY
are no longer in use, \Timar \LD are considered as regular \Janissaire
and there are 6 \Janissaire\ARMY available.  Note that the basic
maintenance of \Timar is now irrelevant as it cannot be used.  All
forces are of smaller reformed size, and Turkish forces lose their
cavalry bonus.  In addition, \Pasha counters are no more military units
(nor Leaders). They still may be corrupted.
\\
Regarding Technology, \TMAN and \TTD are now accessible.  The malus to
improve it is applied to \terme{Land Technology} if currently \TBAR or
better, and to \terme{Naval Technology} if currently \TBAT or better.

\bparag[Level 3: Modernisation of the Army]
All technologies are now accessible, and the malus to increase
Technology is cancelled.

\aparag The effects of all these reforms are cumulative.

\subsection{Relations with foreigners}
\subsubsection{Diplomacy}
\aparag \TUR has a \CB against all Christian countries, and against
\pays{Perse}, in period I to V.

\aparag \TUR is prohibited to make offensive alliances in period I to V.

\aparag[Turkish Conquests.] The Turkish player can annex the capital
province of a conquered country. This is valid only if the conquered
province is an island or if it is adjacent to a Turkish province 
% (Jym, 06/2013) removing to allow conquest of Italy from the
% South. Especilly, conquest of Napoli should not require going through
% Venise but must be allowed from the South.
%
% This version prevent annexion of Napoli on the first war. First, a
% foothold must be established and at the second war, it can be
% annexed.
%
% and connex by land to its National Territory
and the province is occupied by a Turkish military unit (and not by a
Turkish minor ally or vassal). This may destroy the country.

\aparag[Relations with the Knights.] The \pays{Chevaliers} are in
permanent semi-Overseas war against \TUR. The reverse is true. This war
allow for naval battles, and attack by and against Privateers. It does
not cause automatic \STAB loss at the end of turn.
\bparag Each turn that the pirate of The Knights inflicts losses on
Turkish commercial fleets, \TUR loses 1 \STAB level if at peace.

\aparag See also \ruleref{chSpecific:Islam}.

\aparag[Crusades and Turkish occupation of Vienna] See rules
\ruleref{chSpecific:Crusades and Vienna}.


\subsubsection{Relations with the Barbaresque countries}
\aparag Depending on several events, \TUR may have geopolitical malus to
all diplomacy attempts against all \Barbaresques.
\bparag Initially (before event \eventrefname{pII:Alignment of
  Barbaresques}), \TUR has a {\bf -3} malus to all diplomacy attempts
against all \Barbaresques.
\bparag This malus is cancelled when \eventrefname{pII:Alignment of
  Barbaresques} occurs of at the death of \leader{Barbaros} if
\eventrefname{pII:Algeria Vassalisation} occurred.
\bparag Event \eventref{pIV:Morocco} puts back a {\bf -3} malus to all
diplomacy attempts against \pays{Maroc}.
\bparag Event \eventref{pVI:Barbaresques} sets a uniform {\bf -3} malus
to all diplomacy attempts against all \Barbaresques (including
\pays{Maroc}).
\\
PB 07/2008: MORE TO DO
\begin{designnote}
These rules simulate both the clear trend toward inpendence of those regions,
the occasional in-fighting that are not expliciteley dealt with, but also leave
open the historical window of Turkish domination over those countries. 
\end{designnote}

\aparag[Pirates and Ottoman admirals]
\leader{Barbaros} and \leader{Dragut} may be used as Turkish leader if
their country is a \VASSAL of \TUR. They can then lead both Turkish
units and units from their own country.
\bparag[\sectionleader{Barbaros}] The first time \leader{Barbaros} is
reputed dead due to battle loss or attrition, he is in fact unavailable
for the rest of the turn but returns back in play at the beginning of
the following turn.

\subsubsection{Discoveries and Activities in the Indian Ocean}
\aparag[Discoveries of the Mamelouks] \TUR may gain discoveries made
by \pays{Egypte} due to event \eventref{pI:War Roads Spices}.
\aparag[Admirals in the Indian Ocean]
From period III to period IV,  if \TUR has a \TP in  \continent{Asia} (his own or thanks
to a \dipAT), the admiral (that is not restricted to the Mediterranean Sea)
of the lowest rank, has the possibility to go in \continent{Asia}.

\subsubsection{Facing the Ottomans}
\aparag Before 1560, any player or minor country entering combat (on
land or at sea) against Turkish units suffers a malus of {\bf -2} to
both his shock (on land) or boarding (at sea) die-rolls.
\aparag This malus applies for a power only in the first battles, until
after the first combat were at least one \ARMY or \FLEET of the power is
engaged (exception: if a power has neither \ARMY or \FLEET counter, any
force engaged is counted).  This malus does not apply to fire combat
(either on land or at sea). This malus does not apply against mere
\VASSAL or \TUR, only against units of \TUR.
\aparag
The Venetian player is immune to this malus, as well as the following
minors: \pays{Hongrie}, \pays{damas}, \pays{Egypte}, \pays{Genes} and
\pays{Chevaliers}, \pays{Perse}.

\subsection{\sectionpaysmajeur{Turquie} in play}
%\aparag TODO : mettre Raguse dans le territoire controle par \TUR en 1492
%ou plutôt le mettre Neutre avec Présidio Venitien.
%\aparag + statut special de Raguse ? (voir MinorCountries)

Check~\ruleref{chMilitary:Strait Fortifications} for the defence of
\seazone{Marmara}. Check~\ruleref{chSpecific:Knights},
\ruleref{chSpecific:Barbaresques}, \ruleref{chSpecific:Mamluks},
\ruleref{chSpecific:Hungary}, \ruleref{chSpecific:Persia},
\ruleref{chSpecific:Balkans}, \ruleref{chSpecific:Islam},
\ruleref{chSpecific:Crusades and Vienna} and \ruleref{chSpecific:Ragusa}
for other points of interest.

\subsubsection{Sultans and Viziers of Turkey}
\aparag[\anchormonarque{Bayezid II}] is the sultan in 1492. He has
values 7/5/6 and is due to last until the end of turn 6.

\aparag[\anchormonarque{Selim I}.] If \monarque{Bayezid II} dies before
the end of turn 6, his successor is \monarque{Selim I}. If there had
been a dynastic crisis at that time, it is nullified. \monarque{Selim I}
has values 7/5/8 and is due to last 3 turns.

\aparag[\anchormonarque{Suleyman}.] The successor of \monarque{Bayezid
  II} (if at the beginning of turn 7), or of \monarque{Selim I} is
\monarque{Suleyman}. He has values 7/9/8 and will last 9 turns. He does
not test for survival during the 5 first turns. \TUR gains a free
maintenance of one \ARMY\faceplus \Janissaire during his reign. He is
also a general \leaderdata{Suleyman}.
\bparag Note that the absence of survival test automatically prevents the
specific Turkish \REVOLT of~\ref{chEvents:Survival:TUR revolt}.

\aparag[Istanbul rebellions (\emph{İstanbul İsyanları}).] If the Turkish
Monarch has not at least 7 in Military value, add {\bf +1} to the
die-roll testing survival.
\bparag \TUR uses the effects of the second column of the survival test,
that may cause revolts or dynastic crisis.

\aparag[\anchorministre{Sadrazam}] does assist the Monarch and can be
used as a general \leaderwithdata{Sadrazam} at any turn. He is added
to the minimum number of generals (as a Monarch used as general). He is
second only to the Monarch of \TUR.
\bparag \label{chSpecific:Turkey:Vizier}Two kind of counters carry the monarch
symbol for \paysmajeur{Turquie}. Two of them are sultans:
\leader{Suleyman} and the generic Sultan counter. The other ones are
special great viziers, and the \leaderSadrazam counter can not be
used when they are in play on their Monarch-bearing side.

\aparag[\anchorministre{Koprulu}] (in fact, the dynasty of Viziers) may
be named minister through \eventref{pV:Koprulu}. They have values
8/9/7 and remain 8 turns; they are not dismissed if the sultan dies. The
next monarch's values determination gets no modifier at all (not
positive nor negative).
\bparag They are also generals \leaderwithdata{Koprulu}, and this
counter replaces usual \leaderSadrazam.

\subsubsection{Available counters}
\aparag[Military] 10\ARMY (4 marked \Timar), 6\FLEET, 2\corsaire, 10\LDND
(5 marked \Timar on \LD side), 10\LD (5 marked \Timar), 22 \Pashas, 4\NTD,
6\LDENDE, 5 fortresses 1/2, 10 fortresses 2/3, 2 fortresses 3/4, 1
fortress 4/5, 5 forts.
\aparag[Economical] 5\COL, 6\TP, 9\MNU, 9\TradeFLEET, 2\ROTW treaty
counters.



% LocalWords: Turquie Mediterranee malus feudality Trakya Barbaresque
