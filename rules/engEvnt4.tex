% -*- mode: LaTeX; -*-
\definechapterbackground{Political Events of Period IV}{vanLeydenRaidMedway}
\chapter{Political Events of Period IV}
%\section{Period IV}
\label{events:pIV}



\subsection*{Event Table of Period IV}

\begin{eventstable}[Period IV events table]
  \tabcolsep=4.5pt\centering%
  \begin{tabular}{|l|*{6}{c}|l|}
    \hline
    1\up{st}\textarrow& 1-3 & 4-5 & 6 & 7 & 8 & 9 & 10 \\ \hline
    1 & 1  &  1 & 13 & R4 & R19 & 7   & \textbullet~1--2: \\
    2 & 12 & 14 & 15 & R5 & 18  & 8   & +1 then\\
    3 & 17 & 15 & 9  & 6  & 17  & R9  & \nameref{events:pIII}\\
    4 & 18 & 16 & 10 & R7 & 16  & R17 & \textbullet~3--10:\\
    5 & 10 &  4 & R11& 8  & 14  & R18 & \nameref{events:pIII}\\
    6 & 3  &  2 & 12 & 9  & 1   & 19  & \\
    7 & 7  &  6 & 1  & 11 & R5  & R20 & \\
    8 & 22 & R4 & 2  & 12 & 21  & R4  & \\
    9 & 5  & R7 & 3  & 13 & R22 & 8   & \\ \hline
    10 & \multicolumn{7}{l|}{\nameref{events:pV}} \\ \hline
  \end{tabular}
\end{eventstable}

\eventssummary{%
  pIV:Bohemian Revolt|,%
  pIV:Augsburg Revocation|,%
  pIV:Olivares|,%
  pIV:Unity HRE|,%
  pIV:War Persia Turkey|,%
  pIV:Persian Safavids|O{pIII:Persian Safavids},%
  % Change to Morocco/Portuguese Revolt ?
  pIV:Portuguese Revolt|,%
  pIV:Morocco|,%
  pIV:Act Navigation|,%
  pIV:Union Scotland|,%
  pIV:English Civil War|,%
  pIV:English Restoration|,%
  pIV:London Stock Exchange|,%
  pIV:Amsterdam Stock Exchange|O{pIII:Amsterdam Stock Exchange},%
  pIV:Dutch Colonial Dynamism|E/E/E,%
  pIV:Liberum Veto|,%
  pIV:Great Elector|,%
} \eventssummary{%
  pIV:Oxenstierna|O{pIII:Oxenstierna},%
  pIV:Union Poland Sweden|O{pIII:Union Poland Sweden},%
  pIV:Torstensson|,%
  pIV:Swedish Nobles Unrest|,%
  pIV:La Rochelle|E/O{pIV:Morocco},%
  pIV:Richelieu|,%
  pIV:Fronde|,%
  pIV:Times of Troubles|,%
  pIV:Revolt Cossacks|,%
  pIV:Extension Moghol|E/E,%
  pIV:Wars India|E/E,%
  pIV:Revolt Singala|E/E,%
  pIV:China Colonial Attitude|O{pIII:China Colonial
    Attitude}/S{pIV:CCA:Vassalisation Korea},%
  pIV:Japan Colonial Attitude|,%
  pIV:Deluge|,%
  pIV:Koprulu|,%
  O|,%
  pIV:TYW|,%
  pIV:Polish Civil War|,%
}

\newpage\startevents



\event{pIV:Bohemian Revolt}{IV-1 (1)}{Bohemian Revolt}{1}{PBNew}

\history{1618-1621}[This event describes the War for \paysBoheme, whereas the
break out of a general German conflict (that historically followed this event)
is dealt with in \ref{pIV:TYW}.]

\phevnt
\aparag[The Winter King] The minor country \paysBoheme is created / separated
/ breaks alliance (depending on its previous status) from its current
allegiance (even a \GE), and allies itself with \paysPalatinat (which would
also breaks from an existing \GE). The first Major power in the list: \FRA
(except if \CATHCR), \POL (if Protestant), else \SUE (even if Catholic)
controls both those countries and have them placed in \EG on its diplomatic
track.
\aparag \paysBaviere and \hab declares war to these two countries. This is a
Religious Civil War (see \ref{chDiplo:Religious Civil War}) in the \HRE.
\bparag \AUS has a free \CB against \paysBoheme and must use it or lose {\bf
  2} \STAB.
% (JCD) probably, \paysBaviere should be controlled by \MAJHAB, not
% \SPA. Changing.
\bparag If \HAB declares war, \paysBaviere is placed in \EG of \HAB and is
controlled by \MAJHAB.
\bparag \SPA is allowed a limited intervention in the war as an ally of
\HAB. Other countries are constrained by usual rules.

\aparag[The Revolt of Bethlén]
\bparag A \REVOLT \faceplus is placed in a randomly chosen province belonging
to \HAB in \paysHongrie. It controls the city. The \REVOLT is controlled by
\RUS.
\bparag The military forces of the Revolt of Bethlén can use up to 1 \ARMY and
2 \LD of the Hungarian counters (and the \HAB can use at most one \ARMY and 2
\LD from Hungarian counters).
\aparag \TUR cannot declare war against \HAB at this turn.

\phadm
\aparag \AUSMin receives its usual forces and reinforcements.
\aparag \paysBaviere has 1 \ARMY \faceplus, 3 \LD (all Veterans), 1 \fortress
and is commanded by \leaderwithdata{Tilly} (lasting 4 turns).  It has 2
Multiple Campaigns. \paysBaviere has 2 \ARMY counters at its disposition
during the whole length of this event.
\bparag[Tilly's training] [BLP] Troops of \paysBaviere (not its allies)
stacked with \leaderTilly are always \TTER.
\aparag \paysBoheme has 1 \ARMY \faceplus, 1 \LD (Conscripts) and 1 \fortress.
\aparag \paysPalatinat has 1 \ARMY \faceplus (Veterans) leaded by
\leaderwithdata{Mansfeld} (lasting 3 turns). It has 1 Multiple Campaign.
\aparag The Revolt of Bethlén consists of one Hungarian \ARMY \faceplus
(Conscripts) and \leaderwithdata{Bethlen} (lasting 4 turns) placed in the
province of the \REVOLT .
\aparag None of \paysBaviere, \paysBoheme, \paysPalatinat and the Revolt of
Bethlén receive reinforcements on the first turn. They receive normal
reinforcements beginning with the second turn of the war.
\aparag The reinforcements of the Revolt of Bethlén are based only on the
provinces in \payshongrie that he controls or that are in \REVOLT . If there
are none, or if \leaderBethlen is not in play (dead or wounded), it receives
no reinforcements.

\phmil
\aparag \leaderTilly may lead any stack of \paysBaviere or its allies.
\aparag \leaderMansfeld may lead any stack of \paysPalatinat or its allies. It
can retreat with 1 \LD (only) in any neutral Protestant or mixed \HRE country
and remain there (after a battle or a retreat before battle).

\aparag[Destruction of Bohemia]\label{pIV:BR:Destruction}
If \villePrague is captured, \paysBoheme is destroyed at the end of the
current round. All its provinces are now owned by \HAB.  Its military forces
are disbanded and its provinces not yet military controlled by \HAB are
considered controlled by rebels (use Control markers of \paysBoheme); they
surrender as soon as an \ARMY \faceplus besieges them, or by regular siege
with smaller forces.

\aparag[Bethlén]
\bparag The forces of \leaderBethlen are always in restricted supply in the
national provinces of \paysHongrie (provinces with Hungarian shield). They use
the \REVOLT in \paysHongrie and the cities they control as regular supply
sources.
\bparag A force lead by \leaderBethlen can withdraw in \paysTransylvanie or
national provinces of \payshongrie owned by \TUR (by retreat or movement). If
he retreats there, he must stay there until the end of turn but may go out on
any following turn.
\bparag If \leaderBethlen and/or its forces are in \paysTransylvanie
or national provinces of \payshongrie owned by \TUR, \TUR may make a foreign
intervention against both the revolt of \leaderBethlen and \HAB. Declaring the
intervention cost 1\STAB to \TUR.
% \bparag As long as \leaderBethlen and/or its forces are in \paysTransylvanie
% or national provinces of \payshongrie owned by \TUR, \TUR has a free \CB
% against \leaderBethlen. If \TUR uses it, the forces of \leaderBethlen are
% also at war against \TUR and \TUR may enter, besiege and attack any force in
% national provinces of \payshongrie.

\phpaix
\aparag Before any peace is made, roll a test for the possible breakout of a
Religious war, according to \ref{pIV:TYW}.  A \bonus{-2} is applied to this
roll.
\aparag Add a Revolt \facemoins in a national province of \payshongrie if
\leaderBethlen is therein.
\aparag[Survival of \paysBoheme]\label{pIV:BR:Survival} If no such war occurs
and \villeVienne is controlled by the enemies of \HAB, the war end as a
victory of \paysBoheme.  The minor country is fully recreated ; \HAB has a
mandatory peace for 3 turns with \paysBoheme. \HAB loses {\bf 1} \STAB and 30
\PV ; The controller of \paysBoheme gains 30 \PV. \HAB gains the permanent
right to make the complete conquest of \paysBoheme.  \paysBoheme and
\paysPalatinat are placed in \AM of their controlling \MAJ.
\aparag \HAB and \paysBoheme stop war only when \paysBoheme is destroyed or if
\villeVienne is occupied by enemies.  Other countries use normal peace rules
(but are allied to \HAB and \paysBoheme and subjects to Separate Peace
modifiers if any).
\aparag[Victory conditions if the war becomes the TYW]
\bparag if the \nameref{pIV:TYW:Peace Prague} favours the \alliance and they
control \villeVienne, apply \ref{pIV:BR:Survival} if \paysBoheme still exists.
\bparag If the \nameref{pIV:TYW:Peace Prague} favours the \ligue, \paysBoheme
is destroyed as in \ref{pIV:BR:Destruction}.
\bparag Else, \paysBoheme remains at war after the \nameref{pIV:TYW:Peace
  Prague} and will survive the \nameref{pIV:TYW:Peace Westphalie} if not
destroyed before that during the war.

\tour{Turn 2 and following}

\phdipl
\aparag If this event does not evolve in \xnameref{pIV:TYW} (because there has
already been one, or an Appeasement of the religious fight was obtained), the
controller of \paysBoheme and \paysPalatinat may make a full intervention in
the war.



\event{pIV:Augsburg Revocation}{IV-1 (2)}{Revocation of the Truce of
  Augsburg}{1}{PBNew}

\history{Alternative history}

\condition{Check the conditions in the given order until one is found true.}
\aparag If events \xref{pIV:Bohemian Revolt}, \xref{pIV:Unity HRE} or
\xref{pIV:TYW} are happening now, do not mark off and re-roll.
\aparag If there is a \GE, apply the \xnameref{pIV:Augsburg:GE vs Northern
  Alliance}.
\aparag If \eventref{pIV:TYW} has not yet happened, apply the
\xnameref{pIV:Augsburg:HRE vs Augsburg}.
\aparag Else, apply \xnameref{pIV:Augsburg:Troubles HRE}.


\subevent[pIV:Augsburg:GE vs Northern Alliance]{Revolt of a Northern Alliance}

\phevnt
\aparag A Northern Alliance of countries of the \HRE is created.  The
countries \paysOldenburg, \paysHanovre, \paysHesse, \paysHanse, and \paysBerg
breaks free from the \GE and are allied.
\aparag \GE and \hab declare war to all those countries (and are controlled by
\SPA). The \GE is in Civil War (see \ref{chDiplo:Religious Civil War}).
\bparag \AUS has a free \CB against the whole Northern Alliance (to be used
immediately, or forfeited at the cost of {\bf 2} \STAB).
\aparag The Northern Alliance is controlled by the first Protestant \MAJ in
the list that accepts the alliance: \HOL, \ENG, \SUE, \FRA, \POL.  It has a
\CB to enter the war. If it declines, the next country in the list has the
opportunity to do the same (and so on).  The \MAJ gains all the \MIN of the
Alliance in \EG.  If nobody enters the war, \SUE controls the Northern
Alliance (but is not involved and does not gain diplomatic control).
\aparag If \HOL controls the Northern Alliance, it gains the advantages of
\ref{pIV:TYW:Northern HRE Alliance}, as long as the Alliance exists.

\phdipl
\aparag \SPA can make a full intervention as an ally of \HAB.
\aparag If they have not declined control of the Alliance, \FRA and \ENG (if
they are not Counter-Reformation) and \SUE can make a limited intervention in
the war alongside the Northern alliance.

\phpaix
\aparag If no \MAJ entered the war to control the Northern Alliance, it is
dealt with as one country for the peace in this war (except attempts of
Separate Peace), with a malus of \bonus{-4} to make peace.
\aparag A peace of level 3 or higher against the \MAJ in control (or the
Northern Alliance itself if there is none) would dissolve the Alliance in
addition to the peace.
\aparag If the war ends and the Alliance is not dissolved:
\bparag The \MIN are now normal independent countries that are no more part of
the \GE.
\bparag If the \MAJ was \HOL, it gains the benefits of \ref{pIV:TYW:Northern
  HRE Alliance}. Otherwise, the Alliance is dissolved for game purpose.
\aparag Remember that, according to \ref{pIV:TYW:German Empire}, a peace of
level 3 or higher against the \MAJHAB may dissolve the \GE.  Conversely, any
Unconditional Peace against a country once part of the \GE forces is back in
the \GE.


\subevent[pIV:Augsburg:HRE vs Augsburg]{War of Revocation of the Truce of
  Augsburg}

\phevnt
\aparag The Emperor of the HRE has the possibility to revoke this Truce (even
if it was not given in game terms). If he declines to do so, his country loses
{\bf 2 \STAB} and the event terminates.  If the Truce of Augsburg is revoked,
alliances are created in the \HRE and the \HRE is in Civil War.
\aparag[Northern Alliance] If a Northern Alliance already exists, skip this
paragraph.
\bparag A Northern Alliance of countries of the \HRE is created.  The
countries are \paysOldenburg, \paysHanovre, \paysHesse, \paysHanse, and
\paysBerg (if they exist). If there was no \terme{Truce of Augsburg} at the
beginning of the event, \pays{Hesse} and \pays{Berg} are not in the Alliance.
\bparag The Northern Alliance is controlled by the first Protestant \MAJ in
the list that accepts the alliance: \HOL, \ENG, \SUE, \FRA, \POL.  It has a
\CB to enter the war. If it declines, the next country in the list has the
opportunity to do the same (and so on).  The \MAJ gains all the \MIN of the
Alliance at \EW.  If nobody enters the war, \SUE controls the Northern
Alliance (but is not involved and does not gain diplomatic control).
\bparag If \HOL controls the Northern Alliance, it gains the advantages of
\ref{pIV:TYW:Northern HRE Alliance}, as long as the Alliance exists.
\aparag[Southern Alliance] If a Southern alliance already exists, skip this
paragraph.
\bparag A Southern HRE Alliance is created by association of \paysBaviere,
\paysMayence, \paysAlsace, \paysBade and \paysWurtemberg (if they exist).
\bparag The Southern Alliance is controlled by the first Catholic \MAJ in the
list that accepts the alliance: \AUS, \SPA, \POL, \FRA.  It has a \CB to enter
the war. If it declines, the next country in the list has the opportunity to
do the same (and so on).  The \MAJ gains all the \MIN of the Alliance at \EW.
If nobody enters the war, \MAJHAB controls the Southern Alliance (but is not
involved and does not gain diplomatic control).
\bparag \AUSMin gains the \MIN on its track, not \SPA, if \SPA accepts the
alliance.
\bparag If \MAJHAB controls the Southern Alliance, it gains the advantages of
\ref{pIV:TYW:Southern HRE Alliance}, as long as the Alliance exists.
\aparag Both Alliances are at war against each other.  The \HRE is in Civil
War.

\phdipl
\aparag \SPA can make a limited or full intervention alongside the Southern
Alliance (excepted if it declined the control and involvement).
\aparag \SUE can make a limited intervention alongside the Northern Alliance
(excepted if it declined the control and involvement).

\phpaix
\aparag A test to begin a Religious War in HRE is made at the end of the first
turn of this war, with a \bonus{-4} modifier.  See \ref{pIV:TYW} for the
result of the test and the possibility of this Religious War, and the renewal
or not of the test on following turns.  If no such war occurs, peace can be
made as usual.
\aparag{The alliances after the war}
\bparag If \HOL was controlling the Northern Alliance, the Alliance may last
after the war at the condition of \ref{pIV:TYW:Northern HRE Alliance}.
\bparag If \MAJHAB was controlling the Southern Alliance, the Alliance may
last after the war at the condition of \ref{pIV:TYW:Southern HRE Alliance}.
\bparag In other cases, the Alliances would not last after the end of the war.


\subevent[pIV:Augsburg:Troubles HRE]{Troubles in the Holy Roman Empire}

\condition{This event may happen twice, once because of \xnameref{pIV:Augsburg
    Revocation} and another time because of \xnameref{pIV:Unity HRE}}

\phevnt
\aparag \AUS, \HOL and \HIS rolls for one \REVOLT .
\aparag The effect of a diplomatic event is made on every minor country that
is part of the \HRE (fidelity/religion):
\FEforeachlist{listofminorsHRE}{\pays{\loopitem}
  (\theminorfid{\loopitem}/\theminorreligionshort{\loopitem})}{, }.



\event{pIV:Olivares}{IV-2 (1)}{Olivares}{1}{Risto}

\history{1621-1643}
\dure{as long as \strongministre{Olivares} remains the excellent minister}

\condition{}
\aparag \SPA can refuse this event if it so wishes. In that case mark-off as
played.
\aparag \SPA can freely remove \ministre{Olivares} from office at the end of
any following monarch survival phase and the event terminates.

\phevnt
\aparag \SPA receives the excellent minister \ministre{Olivares}, with values
8/9/7.  These minister values supersede the current values of the Monarch (if
they are inferior). This Minister will last for a random length of Excellent
Minister, see \ref{eco:Excellent Minister}.
\aparag If \SPA monarch dies while the this event is still in effect, use the
minister values as a basis for rolling for the values of the new
monarch. Otherwise the monarch returns with its original values when the
minister dies and play continues normally.
\aparag \SPA may receive an additional Art manufacture of level {\bf 2} (if
available, and if \SPA wants so) placed according to normal rules, and also 2
additional \TradeFLEET levels placed in any eligible trade zone (even if it
had no \SPA commercial fleet counter before, and may be in different zones).
\bparag \SPA may now move the \RES{Cloth} \MNU without any drawback
(see~\ref{chSpecific:Spain:Cloth}).
\aparag The \ctz{Espagne} can no more be attacked by Pirates and Privateers in
the Mediterranean Sea. Attacks are to be made from the Atlantic.
\aparag The malus for foreign occupation for Stability improvement is changed
from \bonus{-3} to \bonus{-5} in national provinces only, and none for other
provinces (normal rule).

\aparag The reference level of \paysGenes in \CTZ \SPA is reduced to 0 if the
Spanish player chooses so.
\aparag From now on, \SPA can raise a second privateer that can go in any \STZ
of the \CCs{Atlantic} (in Europe or in the \ROTW).



\event{pIV:Unity HRE}{IV-2 (2)}{War for the Unity of the HRE}{1}{PBNew}

\history{alternative history}

\condition{Check the conditions in the given order until one is found true.}
\aparag If events \xref{pIV:Bohemian Revolt}, \xref{pIV:Augsburg Revocation}
or \xref{pIV:TYW} are happening now, do not mark off and re-roll.
\aparag If \ref{pIV:TYW} finished during the current period, mark off and roll
for one \REVOLT in each of the following countries: \AUT, and \FRA.
\aparag If there is a \GE, apply the \xnameref{pIV:HRE:GE vs Prussian
  Alliance}.
\aparag If \ref{pIV:TYW} never happened, apply the
\xnameref{pIV:HRE:Brandenburg vs Bavaria}.
\aparag Else, use \xnameref{pIV:Augsburg:Troubles HRE}.


\subevent[pIV:HRE:GE vs Prussian Alliance]{Revolt of Brandenburg and allies}

\phevnt
\aparag A Northern Alliance of countries of the \HRE is created. \PRUMin,
\paysSaxe and \payspalatinat are created anew and break free from the
\GE. They are allied.
\aparag \GE and \hab declare war to all those countries (and are controlled by
\SPA). The \GE is in Civil War.
\bparag \AUS has a free \CB against the whole Northern Alliance (to be used
immediately, or forfeited at the cost of {\bf 2} \STAB).
\aparag This countries are controlled by the first Protestant \MAJ in the list
that accepts the alliance: \SUE, \ENG, \HOL, \FRA, \POL.  It has a \CB to
enter the war. If it declines, the next country in the list has the
opportunity to do the same (and so on).  The \MAJ gains all the \MIN of the
Alliance in \EG.  If nobody enters the war, \SUE controls the Northern
Alliance (but is not involved).

\phdipl
\aparag \SPA can make a full intervention as an ally of \HAB.
% (JCD) What about HOL?
\aparag If they have not declined control of the Alliance, \FRA and \ENG (if
they are not Counter-Reformation) and \SUE can make a limited intervention in
the war alongside the Northern alliance.

\phpaix
\aparag If no \MAJ entered the war to control of the \MIN involved, they are
dealt with as one country for the peace in this war (except attempts of
Separate Peace), with a malus of \bonus{-4} to make peace.
\aparag A peace of level 3 or higher against the \MAJ in control (or the
Alliance itself if there is none) would dissolve the Alliance.
\aparag If the war ends and the Alliance is not dissolved, the \MIN are now
normal separate countries that are no more part of the \GE. The Alliance is
then dissolved.
\aparag Remember that, according to \ref{pIV:TYW:German Empire}, a peace of
level 3 or higher against the \MAJHAB may dissolve the \GE.  Conversely, any
Unconditional Peace against a country once part of the \GE forces is back in
the \GE.


\subevent[pIV:HRE:Brandenburg vs Bavaria]{War between \paysBrandebourg and
  \paysBaviere}

\phevnt
\aparag \paysBrandebourg declares a war to \paysBaviere.  \paysSaxe and
\payspalatinat are allied to \paysBrandebourg and declares also a war to
\paysBaviere.
\aparag \AUSMin declares a full war against the enemies of \paysBaviere.
\bparag \AUS has instead a free \CB to enter war as an ally of \paysBaviere,
and will lose {\bf 2} \STAB if it does not use it.

\phdipl
\aparag Each \MAJ that controls one of the involved countries may react as per
the usual rules to enter in limited intervention (only).
\aparag \SPA may make a limited intervention as an ally of the side of
\paysBaviere.

\phpaix
\aparag A test to begin a Religious War in HRE is made at the end of the first
turn of this war, with a \bonus{-2} modifier.  See \ref{pIV:TYW} for the
result of the test and the possibility of this Religious War, and the renewal
or not of the test on following turns.  If no such war occurs, peace can be
made as usual.



\event{pIV:War Persia Turkey}{IV-3 (1)}{War between Turkey and
  Persia}{1}{Risto}

\history{1606-1639}

\condition{This event has the same conditions and effects as \ref{pIII:War
    Persia Turkey}. It is nonetheless a different event (thus both can happen
  separately).}



\event{pIV:Persian Safavids}{IV-3 (2)}{Persian Safavids}{1}{PB}

\history{1590-1722 -- The high tide of Shah Abbas.}

\condition{}
\aparag This event is the same as \ref{pIII:Persian Safavids}.  If it did not
happen, apply immediately its effects. Apply additionally \ref{pIV:Persian
  Safavids:Fall Ormus}.
\aparag If it happened and main provinces of \paysperse are conquered,
activate a Persian Uprising as per the rules.
\aparag Else, apply the following events.

\phevnt
\aparag[Fall of Ormus]\label{pIV:Persian Safavids:Fall Ormus} \dipAT and
\dipFR status with \paysOrmus are immediately broken to \dipNR. This might
cause an Activation of \paysperse against a \TP in \paysOrmus.
\aparag[Conquest of Oman] \paysperse attacks \paysOman and that results in
breaking all diplomatic status of \paysOman. This applies also to military
\dipAT imposed by \PORmin (troops are redeployed).
\aparag[Submission of \granderegion{Afghanistan}]
\bparag \granderegion{Afghanistan} is no more part of the \paysMogol or
\paysAfghans (which is destroyed at this point), but submitted to
\paysperse. The Natives are used by \paysperse in this region.
\bparag As long as \paysPerse masters \granderegion{Afghanistan}, the Silk
resources of this region may be exploited by \provinceOrmus (usual concurrence
with \TP or \COL in \granderegion{Afghanistan}).
\bparag Persian units can go in \granderegion{Afghanistan} and have supply in
every provinces. But only \villeHerat and the European provinces of \paysperse
are supply sources.
\bparag \RUS and \TUR have \OCB against \paysperse as long as it owns
\granderegion{Afghanistan}.
\bparag \granderegion{Afghanistan} can be conquered later by \paysMogol, or
can become \paysAfghans again by \ref{pVI:India:Afghan Empire}.
\bparag \paysperse also loses the area in a losing Peace of level 2 or higher
(in regular or Overseas war) that has no other condition of peace. In Overseas
Wars, the occupation of a province without city counts as a province
occupied. In every war, the control of \villeHerat and its province counts as
for a Persian province.
\bparag When \paysperse loses the area, all the effects described here are
nullified.



\event{pIV:Portuguese Revolt}{IV-4 (1)}{National Revolt of the
  Portugal}{1}{Risto}

\begin{todo}
  Province Tanger should go to Morocco. Helper of POR should gain a predisio
  on Tanger + a \TP of POR in case of victory (no reannexion). HIS should not
  be able to attack if at war otherwise. Helper should be first in order at
  Methuen. Helper should be Catholic?

  Maybe swap Portuguese revolt with Alaouite dynasty and re-add Portuguese
  revolt as secondary event in pV (typically of WoSS which is four times in
  the table). The real war only started in 1660, the turning point between pIV
  and pV.
\end{todo}

\history{1640-1668}

\condition{Occurs only if \xnameref{pIII:POR Ann.:Portugal Annexed} is
  currently in effect.}
\aparag Else, if \ref{pIII:Portuguese Annexation} was never rolled for, do not
mark off and re-roll.
\aparag Otherwise treat as a \RD instead, with a \REVOLT in \SPA.

\phevnt
\aparag All effects of the \xnameref{pIII:POR Ann.:Portugal Annexed} are
cancelled and \paysPortugal returns to play as a minor country. \paysPortugal
receives the same provinces it had at the time of its annexation to \SPA
notwithstanding who currently owns such provinces. It also receives all
Portuguese \COL/\TP, missions, forts/fortresses, commercial fleets etc. that
are currently in Spanish hands.
% (Jym) Included \provinceTanger and other provinces given to SPA during
% annexation (JCD) Yes, why all of the provinces?
\aparag All non-Portuguese \COL in \continent{Brazil} receive a \REVOLT
\faceplus controlled by \paysPortugal. They can't extend outside the regions
of \continent{Brazil}.
\aparag All non-Portuguese troops inside its territories are removed as per
normal peace phase.
\aparag All Portuguese troops are removed from the map, as \paysPortugal is
initially at peace (keep the basic forces in the \ROTW where needed).
\aparag \ENG may accept \paysportugal in \EG; if \ENG declines, same to \FRA,
then to \SUE; if no country accepts, it remains neutral.

\phdipl
\aparag All players who are forced to cede provinces to Portugal by this event
receive a temporary free \CB to be used this turn.
\aparag Players who want to fight against Portuguese \REVOLT in their own \COL
have to declare an Overseas war against \paysPortugal and have a free \OCB to
do so.  Else, their \COL is freely given to \paysportugal (no loss of \STAB or
\VPs).
\aparag \SPA receives a free \CB that lasts until the end of the next period
and can be used multiple times.

\phpaix
\aparag The Portuguese Revolt in a \COL causes the loss of at most {\bf 1}
\STAB to each \MAJ.
\aparag Any \COL having 2 Portuguese \REVOLT \faceplus in it is immediately
annexed by \paysPortugal.
\aparag If \SPA uses its free \CB against \paysPortugal and wins an enforced
unconditional surrender over Portugal, it can reapply \xnameref{pIII:POR
  Ann.:Portugal Annexed}. All Portuguese possessions as they are now are
reannexed to \SPA as described there. Reannexation of Portugal as by
\numberref{pIII:POR Ann.:Portugal Annexed} is only possible in wars \SPA
started using its free \CB. In addition, \SPA gains a \Presidio in
\provinceTanger if the province was Portuguese.
\aparag If \SPA uses its free \CB against \paysPortugal, but does not win an
enforced unconditional surrender over it, the controller of \paysPortugal
receives 30 \VP when peace is made. This can occur several times.
refshort{pIII:POR Ann.:Portugal Annexed} is only possible in wars \SPA started
using its free \CB.  + \fphase In addition, \SPA gains a \Presidio in
\province{Tanger} if + the province was Portuguese.  \ephase If \SPA does use
its free \CB against \pays{Portugal}, but does not win an enforced
unconditional surrender over it, the controller of \pays{Portugal} receives 30
\VP when peace is made. This can occur several times.
\aparag Whatever the result of the war, if \ENG was intervening in the war, it
gains \provinceTanger if the province was Portuguese.



\event{pIV:Morocco}{IV-4 (2)}{Alaouite dynasty in
  \paysmaroc}{1}{PB+JymNew}

\begin{todo}
  Maybe here for giving back Tangier (except presidio?) to Morocco?
\end{todo}

\history{1631}
\dure{Until the end of the game}

\effetlong
\aparag \TUR has a malus of \bonus{-3} to diplomacy with \paysmaroc.
\aparag \paysmaroc loses its \corsaire counter.
\aparag Fidelity of \paysmaroc is now 10.



\event{pIV:Act Navigation}{IV-5}{Act of Navigation}{1}{RistoMod}

\history{1651}
\dure{until English defeat in a war caused by this event, or by event
  \ref{pV:Glorious Revolution}}

\condition{}
\aparag Can occur only if \ENG is currently \PROTANG. Otherwise re-roll.
\aparag Can occur only if \ref{pIV:English Civil War} has already occurred
(not necessarily ended). Otherwise re-roll.
\aparag \ENG can refuse the event, in which case it is marked off and \RD is
applied instead.

\phevnt
\aparag All non-English commercial fleet counters in \ctz{Angleterre} are
eliminated and \ENG receives 2 \TradeFLEET levels in \ctz{Angleterre} (up to 6
levels). All powers that lose their counters as a result of this, receive a
\CB against England until the end of current period.
\aparag From now on, only \ENG can place \TradeFLEET levels in
\ctz{Angleterre}.
\aparag From now on, all \MAJ have an \OCB against \ENG, usable once each
period.

\phadm
\aparag \ENG may ignore restriction of~\ref{chAdministration:Pioneering} for
this turn.

\phpaix
\aparag If a \CB against \ENG received through this event has been used by a
player, and if such a player wins at least a level 4 victory against \ENG, he
may renounce the effects of this event instead of any other peace conditions
(all the allies must agree with this as per normal peace procedure).
\aparag In such a case all non-English \TradeFLEET levels in \ctz{Angleterre}
lost due to this event are returned and \ENG \TradeFLEET in \ctz{Angleterre}
is reduced to 1 whatever its current level.



\event{pIV:Union Scotland}{IV-6}{Personal Union between England and
  Scotland}{1}{Risto}

\history{1603}
\dure{until \ref{pVI:Act Union} occurs or the union is dismantled by
  \ref{pIV:English Civil War}}

\condition{Can occur only if \paysecosse is at peace with \ENG.  Otherwise
  re-roll and do not mark off as played.}

\phevnt
\aparag \paysecosse becomes a permanent \VASSAL of \ENG whatever its current
situation.
\aparag If \paysecosse is currently at war, its opponent must immediately
either accept a white peace with it, or declare war to \ENG with a free \CB.
Normal call for allies can be made for such a war at this point.



\event{pIV:English Civil War}{IV-7 (1)}{English Civil War}{1}{RistoMod}

\history{1642-1648}

\condition{}
\aparag If \monarque{Elisabeth I} rules in \ENG, do not mark off and reroll.
\aparag If \ENG is \CATHCR or \CATHCO and period is III, do not mark off and
reroll.
\aparag If \ENG is currently at war, it offers an immediate white Peace or
Armistice to all its enemies, and will renew the offer at the end of each
turn.
\bparag This event is activated as soon as \ENG is at peace or in armistice
with every other \MAJ (\MIN automatically accept an Armistice).

\phevnt
\aparag A Religious Civil War (\ref{chDiplo:Religious Civil War}) erupts in
\ENG between the \parl and the \paysRoyalists.

\aparag[Which side is played ?]
\bparag If \ENG is \CATHCR by choice in \ref{pI:Reformation2}, the player
keeps playing \royal.
\bparag If \ENG is \CATHCO, or \CATHCR by forced conversion, the player
chooses the side he will play.
\bparag If \ENG is Anglican or Protestant, the player plays \parl.
\bparag The other side will be the Rebels; the \MAJ controlling the Rebels
will be called \REB.
\bparag\label{pIV:ECW:Monarchs} The \royal are governed by the English Monarch
before the event (and he can be used as a general). The \parl are ruled by a
Monarch \monarque{Parliament} of values 5/8/8 that makes no test of Survival.
It gives a bonus of \bonus{+2} to the rolls for all administrative actions
(except exceptional taxes). It may not be used as a general.

\aparag If not played by \ENG, \parl is played by the first Protestant country
in the list: \HOL, \FRA, \SUE or else by \POL.

\aparag Three \REVOLT are rolled for in \ENG. These \REVOLT are hostile to
both sides and controlled by \TUR.

\aparag If not played by \ENG, \royal are played by the first \CATHCR \MAJ in
the list: \SPA, \FRA, \HOL, \VEN else by the first Catholic \MAJ in the list:
\SPA, \FRA, \VEN, \SUE, \POR, \POL.  Failing that it is played by \RUS.

\aparag[Initial position]
\bparag \royal control \provinceMidlands, \provinceCornwall, \provinceDurham
and 1d10/3 (round down) provinces adjacent to \provinceMidlands (to be chosen
by their controller). Add \bonus{+2} to the roll if \ENG was
Counter-Reformation or Protestant. \royal controls all (non-revolted)
provinces in \regionIrlande.
\bparag The \parl control all other (non-revolted) provinces in \ENG.
\bparag \royal and the \parl receive up to the equivalent of basic land forces
of \ENG; the Rebels take the forces before (so they can take everything is
there is not enough). Additional forces are removed.
\bparag The Rebels add 1\LD (Veteran) in any controlled province, and 1\LD
(Conscript) in \provinceDurham (if \royal) or \provinceWessex (if \parl).
\bparag \ENG loses 1 \ND, and the rest is controlled by the \parl.
\bparag All named leaders in play are controlled by the \parl.

\aparag[Economic consequences]
\bparag \ENG loses one-third of its treasury, and at least 50\ducats (this
might cause a Bankruptcy).
\bparag Two \PIRATE\faceplus are placed in \CTZ England.
\bparag All \TP, \COL, \TradeFLEET, etc., remain under the control of \ENG.

\aparag If \ref{pIV:Union Scotland} is in effect, apply \xnameref{pIV:ECW:War
  Scotland} in addition.

\phdipl
\aparag If \ENG was \CATHCR, \SPA if also \CATHCR may make a full intervention
on the side of the \royal.
% (JCD) I added /Anglican there.
\aparag If \ENG was \PROTANG, \SPA if \CATHCR may make a limited intervention
on the side of the \royal.
\aparag If \ENG was not \CATHCR, \HOL if Protestant may make a limited
intervention on the side of the \parl.

\phadm
\aparag[Reinforcements]
\bparag The Rebels roll for reinforcement with offensive status, or naval
status at \bonus{-3}, during all the war. On the first turn, they roll for
offensive with a modifier of \bonus{+4} if \ENG was Protestant, \bonus{+2} if
it was \CATHCR or \PROTANG, of \bonus{0} otherwise (\CATHCO).
\bparag On following turns, they receive a modifier of \bonus{+1} for every 2
provinces they control, with a maximum of \bonus{+4}.
\bparag If the Rebels are the \parl, they can take up to 2 \LD as \ND instead.
\bparag The Rebels have as many counters as \ENG available.
\aparag \ENG uses normal purchase rules, except that its purchase limits are
doubled during the Civil War.
\aparag The \royal receive the general \leaderwithdata{Rupertroy} on the first
turn of the war; he will last 7 turns.% He has a double usage as an admiral.
\aparag The \parl receive the general \leaderwithdata{Cromwell} at the end of
the first turn of the war (before the Peace Segment).  He will last for the 5
following turns. The \parl benefits from a Military Revolution at that point
(\textit{The New Model Army}, see rules \ruleref{chAdministration:Military
  Revolutions}, that is to take immediately any Land Technology available at
most in 4 turns, and in the mean time, is raised to the highest Technology
available at that time).
\aparag The \royal have the Land Technology of \ENG at the beginning of the
event. If played by \ENG, they may raise their technology as per usual rules;
else their Land technology is raised by {\bf 1} each turn of the war beginning
with the second.

\phpaix
\aparag The Civil War ends only if either party achieves both following
conditions:
\bparag Military control of \province{East Anglia} and five other English
National provinces with at least 3 ports.
\bparag Elimination of all enemy army counters, or at least two major
victories against them.
\aparag If the \royal win, \ENG is ruled by its previous Monarch and becomes
\CATHCR (exception: if \ENG was \CATHCO, it remains so). \leaderRupertroy is
kept as a general; land technology of \ENG is at the level reached by the
\royal.
% (JCD): I added (or remains)
\aparag If the \parl wins and \ENG was Catholic or \PROTANG, \ENG becomes (or
remains) \PROTANG.  It is ruled by the \monarque{Parliament}
(see~\ref{pIV:ECW:Monarchs}).
\bparag If \leaderCromwell is in play at the end of the war, it becomes Lord
Protector of the Kingdom, and is an English Monarch that raises the values of
the \monarque{Parliament} to 8/8/9. His Reign is to last the number of turns
remaining for the general (of the initial 5 turns).  A test of survival has to
be done for him. As long as his reign continues, \ENG gains a free maintenance
of one \ARMY\faceplus.
\bparag When \leaderCromwell dies, or at the beginning of the sixth turn after
the end of the Civil War, apply \ref{pIV:English Restoration} as one of the
event of the turn.
\bparag \leaderRupertang becomes an admiral only, kept by \ENG as one of its
own.

\aparag If \ENG was Protestant (not \PROTANG) and the \parl wins, \ENG remains
so. It is rules by the \monarque{Parliament} (see~\ref{pIV:ECW:Monarchs}).
\bparag If \leaderCromwell is in play at the end of the war, he stays as a
general only. \leaderRupertang is not used by \ENG.
\bparag At the beginning of the sixth turn after the end of the Civil War,
roll for a new Monarch on columns 9 for the three values. \ENG is ruled by a
Protestant Republic lead by some strong Lord Protector of the Commonwealth
(represented by the Monarch).
\aparag Regardless of the winner, \leaderMonck and \leaderBlake are admirals
from now on.


\subevent[pIV:ECW:War Scotland]{War with Scotland}

\phevnt
\aparag \paysecosse declares war against the \royal and becomes neutral.
\paysecosse is controlled by \FRA, but no allies can ever take part in this
war. This declaration of war does not trigger a truce in the civil war as per
normal rules.

\phadm
\aparag \paysecosse rolls for reinforcements in offensive status.  It has a
minor general added to its base forces.

\phmil
\aparag Scottish units may not enter England during the first 2 rounds of
their war.
% (Jym): What? (Pierre) I do insist
\aparag On the turn following their entrance in England, \royal gain as added
reinforcements \leaderwithdata{Montrose}, 2 \LD and control of one mountainous
province in \paysecosse of their choice.

\phpaix
\aparag When the Civil War ends, \ENG may decide to continue an on-going war
against \paysecosse (it will be counted as the second turn of the war).
\aparag If \ENG (\royal or, after the end of the Civil War, the \parl) scores
an enforced unconditional victory over \paysecosse during this war, Scotland
is restored to permanent \VASSAL of \ENG as per \ref{pIV:Union Scotland}.  In
all other cases, it reverts to a normal minor.



\event{pIV:English Restoration}{IV-7 (2)}{The Parliament and the English
  Kings}{1}{PBNew}

\history{1660}

\condition{May not happen if the \xnameref{pIV:English Civil War} is not
  finished yet. Re-roll and do not mark off.}

\phevnt
\aparag If \ENG is \PROTANG or \CATHCO, apply \xnameref{pIV:ENG
  Restoration:Restoration}.
\aparag If \ENG is \CATHCR, apply \xnameref{pIV:ENG Restoration:Asking
  Reforms}.
\aparag If \ENG is Protestant, apply \xnameref{pIV:ENG Restoration:War Against
  Puritanism}.


\subevent[pIV:ENG Restoration:Restoration]{The Restoration of the English
  Kings}

\phevnt
\aparag \ENG has the choice of crowning now the Pretender (an Heir of the
Monarch overthrown by \ref{pIV:English Civil War}); if not, \ref{pV:Glorious
  Revolution} is applied now (with worsened consequences).
\aparag If the Pretender is crowned, roll for his values using those of the
Monarch overthrown by the \nameref{pIV:English Civil War}. The effects of
\monarque{Cromwell} or the \monarque{Parliament} are ended (and
\leaderCromwell is put out of play).
\aparag \ENG receives the general (also usable as admiral) \leader{Duke of
  York} that will stay for 5 turns (note: he actually became king in 1685 but
we choose to ignore this and separate the general from the king).


\subevent[pIV:ENG Restoration:Asking Reforms]{The Parliament asks for more
  reforms}

\phevnt
\aparag \ENG has to choose one of the 2 following attitude.
\aparag[Reforms granted]
\bparag \ENG becomes Anglican. It loses {\bf 2} \STAB and rolls for 2 \REVOLT
.
\bparag \SPA, if \CATHCR, has a free \CB against \ENG.
\aparag[Refusal]
\begin{todo}
  ``CHANGE'' (Pierre's notes).
\end{todo}
\bparag Apply \ref{pV:Glorious Revolution} now (with worsened consequences).


\subevent[pIV:ENG Restoration:War Against Puritanism]{Civil War between
  Protestants and Puritans}

\phevnt
\aparag \ENG is now in Civil war (\ref{chDiplo:Religious Civil War}) between
two sides: the (Puritans and Calvinist) Rebels (possibly with Orange
Partisans) and the (Protestant) Royalists. Catholics rebel against both sides.
\bparag The Rebels are controlled by a Protestant \FRA, or \HOL (or \SUE if
there is no \HOL). They use the \paysroyalistes counters.
\bparag The (Protestant) Royalists are controlled by \ENG and use \ENG
counters; all \ENG leaders are with them.
\aparag Four Rebel \REVOLT are rolled for in England (re-roll until in English
owned territory). They control all the fortresses.
\bparag A Rebel \ARMY \facemoins and a \LeaderG are placed in one of these
provinces.
\aparag Catholic \REVOLT \faceplus are placed both in \provinceConnacht and
\provinceMumhan and the \REVOLT control both fortresses.

\phdipl
\aparag The controller of the Rebels have a \CB against \ENG to make a limited
intervention against \ENG this turn, that can become a full intervention on
the second turn.

\phadm
\aparag The Rebels roll for reinforcements in offensive or naval status (but
with \bonus{-2} for naval).
\aparag All reinforcements must be placed in a province with existing rebel
units, allied units, or controlled cities (\REVOLT are not enough). If none,
no reinforcements are received.

\phpaix
\aparag Peace is determined with usual rules except that:
\bparag The Rebels surrender unconditionally if they have no forces nor
\REVOLT left (fortresses do not count).
\bparag If the English King is overthrowned by \REVOLT, it also surrenders
unconditionally to the Rebels and their controller.
\aparag If the Rebels win, \ENG will have a \terme{Dynastic Crisis} next turn,
and loses 50 \PV.
\bparag \eventref{pVI:Act Union} is broken. If it did not happen yet, may
occur later.
\aparag If the Rebels win unconditionally and their controller was involved in
full intervention, additional consequences are:
\bparag \ENG makes a mandatory Dynastic Alliance with the controller of the
Rebels and must give a \COL or \TP as dowry.
\bparag \ENG makes a mandatory offensive alliance with the controller of the
Rebels for 2 turns. It cannot declare war against it (except with \CB from
events; then, the alliance has to be broken with the usual cost in \STAB).
\bparag \eventref{pVI:Act Union} is broken. If it did not happen yet, it may
not occur later.



\event{pIV:London Stock Exchange}{IV-8 (1)}{Creation of the London Stock
  Exchange}{1}{Risto}

\history{1554}

\condition{\ENG chooses to apply this event or \ref{pIII:East Indian
    Company}. Mark the one that is chosen.}

\phadm
\aparag \ENG may ignore restriction of~\ref{chAdministration:Pioneering} for
this turn.

\effetlong
\aparag \ENG can from now on lend 250\ducats per turn to other countries.
% in the Diplomacy phase, plus 100\ducats during the turn (instead of
% 100\ducats plus 50\ducats).

\aparag From now on, \ANG receives a bonus for its International Loan rolls
and Bankruptcy rolls.

\aparag From now on, \ANG is more resilient to exceeding limits in \MNU.

% \begin{oldcompta}
%   \aparag From now on \ENG receives a bonus equal to its \DTI to all die-rolls
%   on international loan amount and interest (not length) in the loans table
%   % (Jym) Table just said "international" and usually more up to date
%   \aparag \ENG is also more resistant to Bankrupt and more tolerant to
%   trespassing of commercial limits.
%   % (Jym) According to the rule of chapter 3, it is actually less resilient
% \end{oldcompta}



\event{pIV:Amsterdam Stock Exchange}{IV-8 (2)}{Creation of the Amsterdam Stock
  Exchange}{1}{Risto}

\history{1608}

\condition{This event is the same as \ref{pIII:Amsterdam Stock Exchange}.}



\event{pIV:Dutch Colonial Dynamism}{IV-9}{Dutch Colonial Dynamism}{3}{Risto}

\condition{\HOL chooses to apply this event or \ref{pIII:VOC}. Mark the one
  that is chosen.}

\phevnt
\aparag \HOL receives an additional commercial fleet level to any eligible
\STZ zone in \ROTW map (even if none existed before).

\phdipl
\aparag For this turn \HOL receives a bonus of \bonus{+2} to all diplomatic
actions made on countries from the \ROTW map.

\phadm
\aparag \HOL receives an additional and free strong \TP placement attempt.
\aparag For this turn \HOL receives a bonus of \bonus{+1} to all
administrative actions made in \ROTW map.
\aparag \HOL may ignore restriction of~\ref{chAdministration:Pioneering} for
this turn.



\event{pIV:Liberum Veto}{IV-10 (1)}{Liberum Veto}{2}{PB}

\history{1652}

\phevnt
\aparag The conditions of the \textit{Liberum Veto} (see
\ref{chSpecific:Poland:Liberum Veto}) are now enforced.

\phadm
\aparag If \POL is at peace after the diplomacy phase of this turn, it loses 2
in \STAB.

\effetlong
\aparag Each time a new dynasty is elected in \POL, it can decide to impose
Absolutism in the Republic. This decision is made at the phase of the monarch
survival (before the events) ; it causes an additional event, \ref{pIV:Polish
  Civil War}. There can be no further additional event at this turn.



\event{pIV:Great Elector}{IV-11}{The Great Elector Friedrich-Wilhelm of
  Prussia}{1}{PB}

\history{1640-1688}

\phevnt
\aparag \leader{Friedrich-Wilhelm} is now the ruler of \paysBrandebourg and a
general [A.2.3.3]. He will last 6 turns.  The basic force of this country is
now one \ARMY\faceplus, 1 \LD, 1 \fortress and 1 general. Its counter limits
are 2 \ARMY and 5 \LD.  The fidelity of the country is 9 from now on.
\aparag \paysBrandebourg claims the \region{Duche de Prusse}:
\provincePreussen, \province{Ost Pommern} and \provinceMemel.
\bparag Minor countries cede those provinces immediately to \paysBrandebourg.
\bparag Major countries have the possibility to cede them or not (and lose \PV
normally).

\phdipl
\aparag If a country declares a war against a \MAJ that owns one of those
territories, he can ask for a full intervention of \paysBrandebourg as an ally
(which is put in \EW immediately).
\aparag If \POL owns provinces of the \region{Duche de Prusse}, it can cede
all of them to \paysBrandebourg in exchange for an alliance with
\paysBrandebourg. \POL does not lose the \PV.  \paysBrandebourg signs a white
peace, is put in \EW of \POL and may be called as ally by \POL in any war it
is currently involved in.
\bparag \POL is now the first power in the list of preference of
\paysBrandebourg.

\phpaix
\aparag In any war involving \paysBrandebourg, only this country may annex a
province of the Duchy of Prussia if its alliance wins; if its alliance wins,
it asks for one province or refuses the peace (so that the other powers must
break their alliance and make a separate peace).



\event{pIV:Oxenstierna}{IV-12 (1)}{Oxenstierna}{1}{PBNew}

\condition{Same event as \ref{pIII:Oxenstierna}.}



\event{pIV:Union Poland Sweden}{IV-12 (2)}{Union between \paysmajeurPologne
  and \paysmajeurSuede}{1}{PB}

\condition{Same event as \ref{pIII:Union Poland Sweden}.}



\event{pIV:Torstensson}{IV-13 (1)}{Torstensson's War}{1}{PB}

\history{1643-1645}

\phevnt
% (JCD) TODO: adapt to DANdan
\aparag \SUE has a mandatory free \CB against \paysDanemark at this turn (even
if their are allied in another war).
\aparag If \SUE refuses the \CB, it loses 2 \STAB.



\event{pIV:Swedish Nobles Unrest}{IV-13 (2)}{Agitation of the Swedish
  Nobles}{1}{PBNew}

\history{1650's}

\phevnt
\aparag If \SUE is Catholic and \ref{pIII:Religious War Sweden} did not happen
yet, it occurs now.
\aparag If \SUE is \PROTRIG, roll for two \REVOLT in \SUE.
\aparag If \SUE is \PROTTOL and at war, rolls for one \REVOLT in \SUE, \SUE
loses 2x \STAB and its monarch changes (abdication of the Queen Kristin).
\aparag If \SUE if \PROTTOL but not at war, roll for 4 \REVOLT in \SUE (do not
place the \REVOLT if not inside \SUE, but do not reroll either) and a Revolted
\ARMY appears in one of those provinces with a general.
\aparag The resulting \REVOLT are controlled by \SPA.


\event{pIV:La Rochelle}{IV-14}{Revolt of La Rochelle}{1}{RistoMod}

\history{1626}
\dure{Until the suppression of the \REVOLT in \provincePoitou and the conquest
  of \ville{La Rochelle}.}

\condition{}
\aparag If \ref{pIII:FWR} is not finished yet, do not mark off and re-roll.
\aparag If the owner of \provincePoitou is Protestant, roll on its Revolt
table and place a \REVOLT\Faceplus if this is a Catholic province, and a
\REVOLT\Facemoins otherwise. The event is marked off and considered as played.

\phevnt
\aparag Place 2 \REVOLT \faceplus and a \LD in \provincePoitou.  Retreat all
other units from the province.
\bparag Roll for two Revolts in \FRA. Place a \REVOLT\Facemoins if the
province is Protestant (or mixed if \FRA is \CATHCR) and nothing otherwise.
% \aparag Place 2 Revolts \facemoins in other protestant provinces (randomly),
% or in Protestant or Mixed provinces if \FRA is \CATHCR.
\aparag The fortress of La Rochelle is controlled by the Rebels and upgraded
to the highest level available to the owner of the province.
\aparag Place a \PIRATE\faceplus in \CTZ of \FRA.
\aparag The Rebels/\REVOLT are controlled by \ENG, or \FRA if \ENG owns the
province. This war is a Religious Civil War (see \ref{chDiplo:Religious Civil
  War}) between Protestants and Catholics and normal Foreign interventions are
allowed.

\phadm
\aparag As long as the event lasts, the owner of \provincePoitou has a malus
of \bonus{-1} to all its administrative actions in the \ROTW.

\phmil
\aparag If a Foreign power enters a land province in the power at war that is
not \provincePoitou during its intervention, it loses 1 \STAB.
\aparag If the owner of \provincePoitou is \FRA and \ministre{Richelieu} is in
the game, consider that the port of the fortress is under blockade if a french
army besieges it.

\phpaix
\aparag If the fortress is controlled by the Rebels, it counts has a \REVOLT
\facemoins for the loss of \STAB by the owner of \provincePoitou due to
\REVOLT .
\aparag The owner of \provincePoitou may cede the province to the controller
of the \REVOLT , losing 30 \PV for doing this.
\aparag The controller of the \REVOLT earns 5 \PV at the end of each turn that
the Rebels exist (\REVOLT or fortress in \provincePoitou).



\event{pIV:Richelieu}{IV-15}{Richelieu}{1}{RistoMod}

\history{1624-1642}
\dure{as long as \strongministre{Richelieu} remains the excellent minister}

\condition{}
\aparag If \ref{pIII:FWR} is not finished yet, do not mark off and re-roll.
\aparag \FRA can refuse this event if it so wishes. In that case mark-off a
played.
\aparag \FRA can freely remove \ministre{Richelieu} from office at the end of
any following monarch survival phase and the event terminates.

\phevnt
\aparag \FRA receives automatically the excellent minister
\ministre{Richelieu}, with values 9/8/7.  These minister values supersede the
current values of the Monarch (if they are inferior). This Minister will last
for a random length of Excellent Minister, see \ref{eco:Excellent Minister}.
\aparag \FRA gains one level of \TradeFLEET in any \CTZ or \STZ of its choice.

\phadm
\aparag As long as \ministre{Richelieu} lives, \FRA has a bonus of \bonus{+2}
to any die-roll for External Administrative Actions and to improve its \FTI.
\aparag \FRA may ignore restriction of~\ref{chAdministration:Pioneering} for
this turn (only).

\phinter
\aparag When \FRA monarch dies, his successor is \monarque{Louis XIV}.



\event{pIV:Fronde}{IV-16}{The Fronde}{1}{PB}

\history{1648-1653}
\dure{3 turns or as long as \monarque{Louis XIV} is not adult (whichever is
  the latest). In any case, it ends after the turn of revolts.}

\phevnt
\aparag If \monarque{Louis XIV} has not already been king of \FRA, the current
king of \FRA dies and the new king is \monarque{Louis XIV}.
\bparag \monarque{Louis XIV} has values 7/6/9, scheduled to last 12 turns and
starts as a baby.
\bparag He'll become adult at the beginning of the fourth turn of reign, thus
ending the event.
\aparag If due to \xnameref{pIV:Richelieu}, \ministre{Richelieu} was still in
charge, then during the first two turns of reign of \monarque{Louis XIV},
\ministre{Mazarin} will be minister with values 7/8/7.
\aparag If \monarque{Louis XIV} is already king, or if his reign is already
finished, then the event lasts for 2 turns.

\phdipl
\aparag Until the end of the event, \FRA may only offer a white or losing
peace to all minors, and peace based on the peace differential to each major
countries, with a maximum level of 1 in the favour of \FRA.
\bparag At each turn, \FRA offer and cannot refuse Armistices with opponents.
\bparag Neutral minor countries always accept that peace.
% \bparag Other countries (either majors or minors) choose their attitude
% freely toward that proposal: accept, refuse or armistice.
\bparag At the third turn of the event, if \ministre{Mazarin} is minister,
then major countries cannot refuse an armistice.
\aparag \FRA may not declare war as long as the event lasts (except
\xnameref{pIV:TYW} and \xnameref{pV:WoSS}).
\aparag If, at the end of a diplomacy phase, \FRA is not at war (don't count
armistices), the Fronde happens.

\tour{Turn of the revolts}

\phdipl
\aparag Half of French troops in Europe become rebel. \FRA choose a stack of
troops staying loyal, thus taking up to half the total number of \LD (rounded
down). The rest becomes the troops of the Fronde.
\bparag If in play, \leaderConde becomes leader of the Fronde. Otherwise, a
randomly chosen general (a named one if there is one in play) becomes leader
of the Fronde.
\aparag The Fronde is controlled by a country currently at armistice with
\FRA. If none exists, the order of priority to control the Fronde is: \HIS,
\ANG, \HOL, \AUS, \POL, \RUS, \SUE, \TUR.
\aparag Naval forces, admirals, everything in the \ROTW as well as
administrative counters (\MNU, \ldots) stay in the control of \FRA.

\phadm
\aparag \FRA collects neither land nor vassals income this turn. \FRA does get
other incomes.
\aparag The Fronde rolls for reinforcements with offensive attitude and no
modifier.
\aparag No side may get reinforcements such that its total force (in Europe)
is above the basic force of \FRA for the current period.

\phmil
\aparag Countries in armistice with \FRA can enter the civil war on the side
they want.
\aparag Fleet of \FRA may stay in ports controlled by the rebels without
penalties.
\aparag Except for the capital of \FRA, fortresses in France are friendly to
both sides.
\bparag A province is controlled by one side if it has an army in the province
and there is no enemy troop besieged in the fortress.
\bparag Other provinces are considered friendly to both sides for supply or
movement cost.
\aparag The capital of \FRA is always controlled by the loyalists until the
Fronde takes the fortress.

\phpaix
\aparag The side controlling the capital of \FRA at the end of turn wins.
\bparag If the Rebels win, \monarque{Louis XIV} (and \ministre{Mazarin}) is
overthrown. During the next turn, there will be a dynastic crisis in \FRA. The
player controlling the Fronde gains 10 \VP.
\bparag In any case, both the loyalist and Fronde's units become units of \FRA
as soon as the event is finished.



\event{pIV:Times of Troubles}{IV-17 (1)}{The Times of Troubles in
  Russia}{1}{PB}

\history{1605-1613}

\condition{}
\aparag If \monarque{Ivan IV} is not dead yet, do not mark-off and re-roll.
\aparag If \RUS chose \terme{Religious Tolerance}, mark off and use \RD
instead.
\aparag If \RUS is at war, the event is pending. It will activate at the
beginning of the first turn where \RUS is at peace and a roll of 6 or higher
is obtained on 1d10.

\tour{Turn 1}

\phevnt

\aparag The Russian monarch dies and is replaced by \monarque{Boris Godunov}.
His values are 5/8/4 and he will reign 5 turns; he is a general
\leaderwithdata{Godunov}.

\aparag \RUS is now in Religious Civil War (see \ref{chDiplo:Religious Civil
  War}). Rebels are Catholic; loyalists (\RUS) are Orthodox.

\aparag Roll for 6 \REVOLT in Russia. Only provinces actually in \RUS revolt,
other rolled-for are ignored. The \REVOLT are controlled by \POL.

\aparag Rebels gain one \ARMY\faceplus in one province in \REVOLT, and the
control of the city.

\aparag Rebels own any revolted province with no Russian armies in there
(except \province{Moscou}) and provinces they control. These provinces are
their supply sources.

\aparag \RUS owns all non-revolted provinces they control. They are their
supply sources.

\aparag All other provinces are disputed. Supply of both sides may cross those
provinces if there is no enemy force within.

\phdipl

\aparag During the event, \RUS may ask for help of \SUE. The condition is the
cession of one Russian province to \SUE; if this province is revolted, it
becomes Swedish only when it is no more revolted and its is controlled by \RUS
or \SUE. During the rest of the event, this province (even Swedish) can be
entered and attacked by all belligerents.

\bparag If an intervention of \SUE is agreed upon, \SUE has to commit at least
4 \LD in Russia, following the conditions of limited intervention.  \SUE can
not withdraw any force sent in Russia.

\aparag Major countries may make \terme{Foreign Intervention} in this war.

\phadm

\aparag Rebels receive offensive reinforcements at each turn, using the
provinces they own.

\aparag Rebels have the general \leaderDmitry (until he dies) for 5 turns.

\phpaix

\aparag See the explanations hereafter, valid for all turns.

\tour{Turn 2 and afterwards}

\phevnt

\aparag \monarque{Boris Godunov} has a malus of \bonus{+3} to his survival
roll. If he dies, a period of anarchy follows and \RUS has values 3/3/3 as a
monarch. On the next turn, \monarqueRomanov (in fact Fyodor and Michael) is
the new monarch 6/5/6; as this monarch represents the whole family, do not
roll for his survival (it is automatic).

\aparag if \monarque{Boris Godunov} is dead (on this turn or a previous one),
\leaderDmitry also rolls for survival with a \bonus{+3} as sole modifier
during the event.

\aparag As long as the event continues, roll for 3 \REVOLT in \RUS (that occur
only is in a Russian owned province).

\phdipl

\aparag \POL may make a full or limited intervention as ally of the Rebels. It
has a \CB to do so, or a free \CB is \leaderDmitry is alive. This
intervention is not affected by excessive foreign intervention.

\aparag If \POL was involved in this war on the previous turn and \SUE is
making an intervention allied to the loyalists, \POL may generalise the war
with a free \CB in a full war between \SUE and \POL. This does not change the
terms of their respective interventions in the Civil War.

\phpaix

\aparag \REVOLT in provinces that are controlled or occupied by \POL do not
extend and do not count for the conditions of victory of this event.

\aparag If half (round-up) of the Russian national provinces are in \REVOLT ,
\monarque{Boris Godunov} is overthrown and killed with no further
consequences.

% TODO exception to ??

\aparag A side fulfils the military condition of victory if it won a major
victory against the enemy or if it controls all cities in national provinces,
or if the enemy (not their foreign allies) has no \ARMY left.

\aparag The event ends as a victory for the Rebels or the Loyalists under the
following conditions.

\bparag Rebels win if \monarque{Boris Godunov} is dead and they control
\villeMoscou and they fulfil the military condition of victory; or they win if
\monarque{Boris Godunov} is dead and Loyalists surrender willingfully.

\bparag The Loyalists win if all the \REVOLT are eliminated in owned national
provinces and they fulfil the military condition of victory.

\bparag When the victory is obtained, all the \REVOLT and the Rebel armies are
destroyed.

\bparag The intervention of \SUE ends; \RUS has now a free \CB (one use)
against \SUE until the end of the period.

\aparag If the Loyalists win, \leaderDmitry is eliminated.

\bparag If \monarque{Boris Godunov} is alive, he is now legitimate ruler of
Russia. He has now values 6/8/5.  \RUS gains 10\PV.

\bparag If he is not, the new ruler is \monarqueRomanov for 5 turns, with
values 6/5/6. Russian \STAB is increased by 1.

\aparag If the Rebels win, \monarque{Boris Godunov} is eliminated.

\bparag If \leaderDmitry is alive, he becomes tsar \monarqueDimitri with
values 4/7/5 (and the turns left). \RUS loses 3 in \STAB.  If \POL is still
intervening in the war, \RUS is now in mandatory defensive alliance with \POL
during \monarqueDimitri's reign. In addition, \POL gains one province in \RUS
that it currently controls or occupies (its choice).

\bparag If \leaderDmitry is dead, the new ruler is \monarqueRomanov for 5
turns, with values 6/5/6. Russian \STAB is decreased by 2.  If \POL is still
intervening in the war, it gains one province in \RUS that it currently
controls or occupies (its choice).

\bparag In both cases, \POL gains 10 \PV and signs a white peace with \RUS.



\event{pIV:Revolt Cossacks}{IV-17 (2)}{Revolt of the Cossacks}{1}{PB}

\history{1654-1667}
\dure{until the end of the wars caused by the event.}

\condition{If the religious attitude of \POL is Tolerance of Orthodoxy, the
  event does not occur. Mark off and play \RD instead.}

\tour{Turn 1}

\phevnt
\aparag One province of \paysUkraine belonging to \POL (if none, belonging to
\HAB) secedes and create the new minor \paysukraine. The province is
\provincePoltava if available, else, the closest to this one (chosen by the
new protector or controller of \paysUkraine). Units of other countries inside
are immediately redeployed.
\aparag The new minor is a special \VASSAL of its protector. No diplomacy is
allowed on it.
\bparag The protector stops being protector if it declares war to
\paysukraine. The next possible protector in the list becomes the new
protector.
% (Jym) Originally also if does not come to help, but vassal means war to the
% minor is war to the major, so no way not to help
\aparag \paysUkraine never makes a separate peace without its protector and
must be included in the same peace treaty.
\aparag Possible protectors are (in order): \POL (if Orthodox), \RUS, \TUR,
\POL (if not Orthodox). If there are no (more) protectors, \paysUkraine
becomes a normal minor country.

\phdipl
\aparag \POL has a free \CB against \paysUkraine if it loses at least one
province during the formation of that country.
% (JCD): the event does not happen if \POL is orthodox! removing
% \aparag If \POL is Orthodox and does not declare the war, roll for two
% \REVOLT in \POL and ignore the rest of the event.
\aparag If \paysUkraine (as a special Polish \VASSAL) owns a province of
\paysCrimee (a province with a Crimean shield, even blurred), then \POL may
ask for a limited intervention of \paysCrimee in this war.
\bparag This does not change the diplomatic status of \paysCrimee nor its
controller. \paysCrimee is played by its usual controller decided by the usual
rules.
\bparag If \POL wins after an intervention of \paysCrimee, it must give one
province back to it.

\phadm
\aparag If \POL is at war against another \MAJ during the event, \HAB can make
a limited intervention as an ally of \POL.

\effetlong
\aparag \ref{chSpecific:Poland:Polish Ukraine} is no more valid.

\tour{Turn 2 and after}

\phevnt
\aparag If \POL is at war against \paysukraine, \SUE has a free \CB against
\POL.
\bparag If \SUE is at war against \POL, \RUS has a free \CB against \SUE (can
be used in reaction).
\bparag If \RUS uses this \CB and \paysdanemark is either inactive or already
at war with \SUE, then \paysdanemark is put in \EG of \RUS and enters war
against \SUE (if not already at war).

\phpaix
\aparag Normal rules for peace apply, except that allies of \POL cannot annex
provinces of \paysukraine that they didn't own before the event.



\event{pIV:Extension Moghol}{IV-18}{Extension of the Moghol Empire}{2}{PB}

\history{1635-1638 / 1653-1657}

\phevnt
\aparag If the non-European minor country \paysMogol does not exist, it is
created now. Its ruler is now \leader{Grand Moghol} (if period is IV or later,
it replaces \leaderAkbar if he was in play).
\aparag The \paysMogol will try to invade \textbf{3} regions during the turn,
according to \ref{pII:Mughal Expansions}.
\aparag Even if the country has no region after the invasions, it still exists
(and can gain provinces with new events).
\aparag \granderegionBengale and \granderegionKarnatika become rich region,
with 2 resources of each kind shown on the map (instead of 1).



\event{pIV:Wars India}{IV-19}{Wars in India}{2}{PB}

\history{1631-1635 / 1656-1659}

\phevnt
\aparag If it was still existing, minor country \paysVijayanagar is destroyed
(by internal fights).  Every \TP (not \COL) that is in the minor country
\paysVijayanagar at the time of its disappearance will face an attack by
Natives that are activated against every country this turn.
\aparag If \paysVijayanagar had already been destroyed, choose randomly 2 \TP
and/or \COL in \continent{India} that will be attacked by the Natives in the
region, due to internal strife in India.
\aparag \granderegionKarnatika has from now on 2 \RES{Spices} and 2
\RES{Products of Orient} available instead on 1 (if not yet done).
\aparag If the \paysMogol exist, they invade one province with a modifier of
\bonus{-2}, the next in the list according to \ref{pII:Mughal Expansions}.



\event{pIV:Revolt Singala}{IV-20}{Revolts in
  \granderegionSingala/\granderegionFormose}{2}{PB}

\history{1630}

\condition{If there is no \TP/\COL in \granderegionSingala nor
  \granderegionFormose, do not mark off and re-roll.}

\phevnt
\aparag Choose randomly the province of the revolt between
\granderegionSingala or \granderegionFormose if both contain a \TP/\COL. If
not, the chosen province is the one containing the \TP/\COL.
\aparag Place a \REVOLT \facemoins in the chosen region. This \REVOLT is not
connected to the Natives but military forces sent there to suppress it may
have to confront the Natives if they react.



\event{pIV:China Colonial Attitude}{IV-21}{\paysChine Colonial
  Attitude}{1}{PB}

\history{1557 / 1637}

\condition{This event is the same as \ref{pIII:China Colonial
    Attitude}. Exception: if \xnameref{pIII:CCA:Closure China} is already
  effective, apply \xnameref{pIV:CCA:Vassalisation Korea} instead.}


\subevent[pIV:CCA:Vassalisation Korea]{Vassalisation of Korea}

\phevnt
\aparag Two Chinese armies and the natives of \granderegionCorea attack any
\TP/\COL that are in the area (even Japanese \TP).
\aparag \granderegionCorea is now part of \paysChina.

\phpaix
\aparag If a \TP has survived, \pays{Chine} concedes a new \dipAT to the owner
of the \TP, if it didn't have any. The owner still has to pay as for usual
\dipAT with \paysChine.



\event{pIV:Japan Colonial Attitude}{IV-22}{\paysJapon Colonial
  Attitude}{1}{PB}

\history[Tokugawa's Commercial Restrictions in history]{1638}

\condition{}
\aparag If \paysJapon has no \TP, use \xnameref{pIV:JCA:Closure Japan}.
\aparag If \paysJapon has a \TP on the map (in \granderegionCorea), use
\xnameref{pIV:JCA:Japan Commercial Dynamism}.


\subevent[pIV:JCA:Closure Japan]{Tokugawa's Commercial Restrictions}

\phevnt
\aparag One country having a \TP in \paysJapon may sign immediately a Treaty
with \paysJapon, and so gains \dipAT. If more than one country have a \TP in
\paysJapon, all owners (except minor powers) make a secret bidding (minimum
bid is 50 \ducats).  The largest bidder wins and gains the \dipAT; all the
bids are lost and all other \TP are removed from \paysJapon.
\aparag When the \dipAT is accepted, only one \TP of the country is kept in
\paysJapon; excess \TP are destroyed. If refused, \paysJapon declares an
Overseas War against the power.
\aparag From now on, \dipAT allow one country to keep only one \TP in
\paysJapon (and not one per region). The remaining \TP can be upgraded, and it
causes no reaction by \paysJapon.
\aparag The basic forces and reinforcements of \paysJapon are now its mainland
army only (no overseas garrisons or fleets).

\effetlong
\aparag From now on, no new \TP counter can be placed in any area belonging to
\paysJapon by means of administrative actions.
\aparag No regular diplomacy is permitted on \paysJapon.  The Activation level
of \paysJapon becomes 11.


\subevent[pIV:JCA:Japan Commercial Dynamism]{Commercial dynamism of
  \paysJapon}

\phevnt
\aparag \paysJapon gains a \TP with level 6 in \provinceSeoul,
\provincePyongyang and with level 3 in \granderegionFormose, if those
provinces do not contain foreign \TP. \paysJapon has a \FTI of 2, raised to 3
from period V on.
\aparag If there are \TP in any of those provinces, \paysJapon declares an
Overseas War against all the country having those. This war may not be ended
by an automatic white peace.

\phadm
\aparag Basic forces of \paysJapon are increased to 2 \ARMY\faceplus in
\paysJapon, plus 1 \ARMY\faceplus (in \granderegionCorea at start), 2 \LD and
1 \FLEET\facemoins overseas.
\aparag Basic reinforcements are increased to one \ARMY\faceplus in mainland,
and 1 \ARMY\facemoins, 1 \ND for the garrisons.
\aparag If \paysJapon has a \TP counter, it gains 1 level, up to level 6 in
\provinceSeoul and \provincePyongyang, and level 3 in
\granderegionFormose. Choose one randomly for this increase if there are
several \TP.  These \TP exploit the resources in the region and are counted as
normal exploitation for monopolies and evolution of prices.

\phmil
\aparag Japanese forces outside \granderegionJapan do not activate the Natives
and an attack in regions with Japanese \TP may be aimed at the Japanese only
and so does not activate the Natives of the region. As soon as the \TP is no
more Japanese or destroyed, normal activation rules for Natives apply.

\phpaix
\aparag If \paysJapon does not lose the war and there is no \TP in any of the
3 provinces claimed, it places a \TP in there of level 1.

% Jym, 05/2013, Pierre's notes (2007).
\event{pIV:Deluge}{IV-y}{Swedish Deluge}{1}{PBNotEvenWritten}

\history{1648 (Khmelnytsky Uprising)-1667 (Truce  of
  Andrusovo)}[Russo-Swedo-Polish wars, Second Northern war]
\dure{2 turns.}

If \POL is at war, fortresses in \paysLithuanie let enemy supply go through
their province.

Should appear either during IV-17(2), or as IV-10(2).


% Jym, 05/2013, Pierre's notes (2007).
\event{pIV:Koprulu}{IV-z}{\ministreKoprulu}{1}{RistoMod}

Same as~\ref{pV:Koprulu}. Should appear late in the table only. (Jym):
Possibly as IV-17(3) or IV-11(2).

\vfill \pagebreak

%% *-* latex-mode *-*



\event{pIV:TYW}{IV-A}{Thirty Years' War}{1}{PB}

\history{1618-1648}

\activation{This war is a consequence of some religious fighting in the
  \HRE. If \ref{pV:WoSS} has already begun, this event is not possible
  anymore. Ignore it.}
\aparag It might be triggered by \xnameref{pII:Schmalkaldic League},
\xnameref{pIII:League Nassau}, \xnameref{pIV:Bohemian Revolt},
\xnameref{pIV:Augsburg Revocation} or \xnameref{pIV:Unity HRE}.  This event
may happen only once; before that, at the end of the first turn of a war
caused by one of the previous event, make the following test.
\aparag Roll 1d10 and add the modifiers:\par
\begin{modlist}
\item[\bonus{+2}] in period II and III
\item[\bonus{-2}]for each turn of the current war before this turn
\item[\bonus{-1}] if the peace modifier of the \HAB is >0
\item[\bonus{+2}] if \monarque{Charles V} rules \SPA
\item[\bonus{+2}] if \SPA has chosen \CATHCO
\item[\bonus{+2}] if \villeVienne is not owned and controlled by \HAB
\item[\bonus{+4}] is test during \xnameref{pII:Schmalkaldic League}
\item[\bonus{+2}] if test during \xnameref{pIII:League Nassau} and \SPA is
  \CATHCR
\item[\bonus{-2}] if test during \xnameref{pIV:Bohemian Revolt}
\item[\bonus{-2}] if test during \xnameref{pIV:Unity HRE}
\item[\bonus{-4}] if during \xnameref{pIV:Augsburg Revocation}
\item[\bonus{\textplusminus1}] if \ministre{Richelieu} or \ministre{Mazarin}
  are still present (choice of \FRA)
\item[\bonus{+1}] If \nameref{pIII:FWR} have yet to happen
\item[\bonus{+3}] If \nameref{pIII:FWR} are happening now
\item[\bonus{-1}] If Protestant won in \nameref{pIII:FWR}
\item[\bonus{+1}] If Counter-Reformation won in \nameref{pIII:FWR}
\end{modlist}

\aparag Result:\par
\begin{modlist}
\item[\geq 11] Appeasement of the religious fight
\item[7--10] Agitations in the \HRE
\item[\leq 6] Eruption of the Religious War
\end{modlist}
\aparag[Appeasement of the religious fight] The current war does not
degenerate in a general Religious War. No further test will be made for this
war.
\aparag[Agitations in the \HRE]
\bparag One \MIN enemy of \HAB will have a bonus of \bonus{+2} to its
reinforcement roll next turn (Alliance's choice).
\bparag \paysSaxe joins the enemy side of the \HAB in full intervention (or
\paysBrandebourg if \paysSaxe is already at war).
\bparag At the end of the next turn, roll this test anew to see if a Religious
War breaks.
\aparag[Eruption of the Religious War] The rest of the event will be applied
as one of the 4 regular events of the next turn.  No peace is made for the war
of this turn in the \HRE (except for specific rules of this war about
conquered minor countries).  The Thirty Years' War is now about to begin.

\phevnt
\begin{digressions}[War setup]


  \digression[pIV:TYW:Creation of the Germanic Alliances]{Creation of the
    Germanic Alliances}
  \aparag Two German sides are made up for this war: the (German) Catholic
  \ligue and the Protestant \alliance (more properly called: \emph{Protestant
    Union} or \emph{League of Evangelical Union}).  All minor countries of the
  \HRE at war will be part of one or another. When a minor country joins one
  alliance, it is placed in Neutral diplomatic position and will change of
  status before the end of the war only if specified by this event or another
  political event. The \HRE is now in Civil and Religious War
  (see~\ref{chDiplo:Religious Civil War}), with all the usual restrictions.
  \bparag The \alliance is formed by all the German minor countries that were
  enemies of the \HAB during the previous turn.
  \bparag \HAB and its German allies (minor countries at war with it) form the
  \ligue. \AUSMin is part of the \ligue as any other minor. \paysBaviere
  automatically joins this alliance.
  \bparag The stability of both sides is placed on \bonus{+2}, modified by any
  Major Victory of the preceding turn of their side (battles with troops of
  German minor countries or \HAB).  This stability will evolve during the turn
  because of the major victory/defeat of any forces in their alliance that is
  in any province of the \HRE (even if there are only forces of non Germanic
  major powers).
  \aparag[Attitude of the Netherlands] If \HOL is not a Major Power, the
  following conditions apply:
  \bparag If \payshollande is either owned by \SPA or is \paysprovincesne or
  \paysVhollande, apply \ref{pIII:Dutch Revolt}. This gives a new status to
  \payshollande (it may trigger the following points if still a \MIN).
  \bparag If \payshollande is a \VASSAL of \SPA (special or regular),
  \payshollande breaks its special status with \SPA. \SPA has an immediate
  free \CB against \payshollande ; if used, \payshollande revolts against the
  Spanish Crown, (re)apply \numberref{pIII:Dutch Revolt} and \HOL is now a
  Major Power. If it does not use it, apply \ref{pIV:TYW:VEN transfer}. For
  the rest of the event \HOLhol is neutral, and may not be involved in any
  manner in the incoming war. Ignore any reference to \HOLhol hereafter for
  this event.
  \bparag If \payshollande is a normal minor country, apply \ref{pIV:TYW:VEN
    transfer}. \HOLhol is involved in the war.
  \aparag[Transfer to \HOL]\label{pIV:TYW:VEN transfer} If
  \payshollande is liberated by the preceding paragraph, \VEN may be allowed
  to choose between incarnating \AUS or \HOL according to the rules of the
  Grand Campaign.
  \bparag If \VEN chooses \AUSMin (which becomes \AUS), \payshollande is now a
  normal minor country.
  \bparag If \VEN chooses \HOLmin (which becomes \HOL), \HOL is created with
  no Revolt (using the current position of \HOLmin).
  \bparag TODO: establish full starting position of non-revolted \HOL.
  \aparag The \alliance is controlled according to the order of preference (a
  player may not refuse control): \HOL, \ENG (Protestant), \FRA (Protestant),
  \SUE (Protestant), \RUS.
  \aparag The \ligue is controlled according to the order of preference (a
  player may not refuse control): \SPA (Counter-Reformation), \AUS (if it
  exists), \SPA (Conciliatory).
  \aparag If the \nameref{pII:Schmalkaldic League} or the \nameref{pIII:League
    Nassau} still do exist, the countries part of the League immediately join
  the Protestant \alliance and the Leagues are dissolved.
  \aparag If the period IV has not begun yet, the Major Powers: \SPA, \HOL,
  \SUE, \FRA and \AUT have to choose immediately if they take or not the
  Objectives relevant to this war. The Objective are conditions to be true at
  the end of period IV (and not especially this war).


  \digression[pIV:TYW:Extension Alliances]{Extension of the alliances}
  \aparag Every minor country of the \HRE that is not part of the war is
  checked for war entry at the beginning of each turn. One rolls 1d10, added
  to the \STAB of the side it could join, the current turn of the war
  (\bonus{+1} this first turn), and specific modifier for some countries.  On
  a result of {\bf 6 or higher}, this country enters the war.
  \aparag The list of the countries of the \HRE is given in
  \ref{table:TYW:Extension table}, with the side they will join and their
  starting force.  All those forces are conscripts, except where indicated.
  It is possible that, given the peculiar conditions of the war triggering the
  Religious War, a country ends up in a different side of the one which should
  be expected.
  \begin{table}\centering
    \begin{tabular}{l|l|c|p{.5\textwidth}}
      Country & Side & Mod. & Forces \\\hline
      \paysBaviere & \ligue & Auto. & \ARMY\faceplus, \LD, \fortress and at
      least 1 General (see below);
      % if none available, use \leader{Mercy} that will stay
      % until his death or when this war ends
      may use 2 \ARMY counters for all
      the duration of the war; starting forces are Veterans.\\
      \paysCologne & \ligue & & \LD, 1 \fortress\\
      \paysLiege & \ligue & & \fortress\\
      \paysMayence & \ligue & & \fortress\\
      \paysTreves & \ligue & & \fortress\\
      \paysAlsace & \ligue & --2 & \LD, \fortress\\
      \paysLorraine & \ligue & --4 & \LD\\
      \paysWurtemberg & \ligue & --2 & 2 \LD\\
      \paysThuringe  & \ligue & --2& none\\
      \paysBade &\alliance& & 2 \LD and \LeaderG (Georg Friedrich of
      Baden)\\
      \paysPalatinat &\alliance&& \ARMY\facemoins and \fortress\\
      \paysBerg &\alliance& --2& \LD\\
      \paysBrandebourg &\alliance& --2&\ARMY\facemoins and \LeaderG\\
      \paysBrunswick &\alliance& &\ARMY\facemoins and \LeaderG (Christian
      of Brunswick)\\
      \paysHanovre & \alliance & & \LD and \fortress\\
      \paysOldenburg & \alliance & --2 & \fortress\\
      \paysHanse&\alliance && \LD, \DN\\
      \paysHesse& \alliance& --2& \ARMY\facemoins and \fortress\\
      \paysSaxe&\alliance &--4& \ARMY\facemoins, \LD  and \fortress\\
      \paysBoheme &\alliance && \ARMY\facemoins and \LD\\
    \end{tabular}
    \caption{Extension of the Alliances during the Thirty Years' War}%
    \label{table:TYW:Extension table}
  \end{table}

  \aparag[Mercy] If there is no named \LeaderG of \paysBaviere in play, it
  receives \leaderMercy.
  \bparag If there is one, as soon as he dies (wound is not enough),
  \paysBaviere immediately receives \leaderMercy.
  \bparag \leaderMercy stays in play for 4 turns. If he arrives in the middle
  of a turn (due to death of his predecessor), this turn fully counts as his
  first turn of activity.

  \aparag The forces written may be inferior to the basic forces of the
  country (representing the confused situation).  They are only used when the
  country join the alliance. If already at war a previous turn, a country
  keeps all that is deployed and gains nothing new.
  \aparag If \AUSmin joins war at this time, they receive their basic force
  plus 1 \ARMY \faceplus (but no supplementary random reinforcement ; that
  will be part of those of the \ligue) as Veterans.
  \aparag No intervention (full or limited) of foreign countries are allowed
  if it is not explicitly written in this event.
  \aparag \paysSaxe may be used as mercenaries during this event once it
  surrendered all its home territory to the enemy. Its army is available to
  the side that controls its home territories; if this side loses subsequently
  part of the provinces, it still uses the army but can no more recruit
  Saxons; if it loses all the provinces, the Saxon forces are removed (and
  available now to the enemy).
\end{digressions}

\tour{Turn 1 (1624--1629)}

\phevnt
\aparag From now, and until the war is ended by the \xnameref{pIV:TYW:Peace
  Westphalie}, no Diplomacy is possible on minor countries of the \HRE, no
attempt to have them enter in a war also, and no declaration of war against
them is possible outside the rules of this event.
\aparag After the creation and the extension of both German sides in the war,
some foreign countries can be involved in it also.
\aparag The controller of each alliance can declare war to German minor
countries that refused to be in war this turn, precipitating them in the enemy
alliance (regardless of their religion).
\aparag \SPA enters the war as an ally of \ligue. This is not a formal
declaration of war and costs no \STAB.
\aparag \HOLhol enters war as an ally of \alliance.  This is not a formal
declaration of war and costs no \STAB. \HOLMin receives its full basic forces,
has a separate die-roll for reinforcements, is allied to the \alliance but not
part of it (for the conditions specifying that the \alliance sues for peace).
\aparag \ENG can do a limited intervention. Its side is the \alliance if \ENG
is Protestant, the \ligue if it is \CATHCR, or the one of its choice if it has
chosen \CATHCO.
\aparag \SUE, if \PROTRIG, can do a limited intervention as an ally of the
\alliance.
\aparag The Emperor of the \HRE, if he is not \HAB, can begin a limited
intervention in the War as an ally of the \ligue.
\aparag Any Major Power that was doing a limited intervention during the
previous turn (as defined in the original war) can continue this limited
intervention to help the same side.
% (JCD): TODO adapt to DAN
\aparag[The Danish Crusade]\label{pIV:TYW:Danish Crusade}
\DANMin makes a mandatory white peace with all its adversaries.  It then
enters the war as an ally of \alliance (but not part of it). It has 2 \ARMY
\faceplus (Veteran), 1 \FLEET \facemoins, 1 \fortress, 2 Multiple Campaigns
and is led by its general-king \leader{Christian IV} present for 4 turns.  It
does not receive reinforcements on this turn. \DANMin is played by \ENG.
\aparag All those alliances and interventions during the whole war are made
with the German alliances; the foreign countries are not allied with each
other except if they decide to sign a specific alliance. Else, they are not
obliged to continue the fight together (no penalty to sign peace) and only
separate peace from the German alliance is required.

\begin{digressions}[Specific rules for the war]


  \digression[pIV:TYW:Turkish Frontier]{The Turkish frontier}
  \aparag As long as there are 2 \ARMY\faceplus of \HAB in \villeVienne or any
  province once owned by \paysHongrie and a \LeaderG, \TUR may not declare a
  war to \HAB (but may continue one). For the first turn, this restriction is
  enforced if \HAB has this force available anywhere in the \HRE instead.
  \aparag If \villeVienne is conquered by the \alliance, or the previous
  condition is not respected at the Diplomatic Phase, \TUR has no such
  restriction.
  \aparag If \TUR takes \villeVienne, the \ligue will concede a winning peace
  to the \alliance at the end of the turn. A Crusade might then happen.
  \aparag \TUR is entitled to make a Foreign Intervention against the \ligue
  in this war if otherwise at peace and \paysTransylvanie is \VASSAL or
  annexed.


  \digression[pIV:TYW:German Reinforcements]{German reinforcements}

  \phadm
  \aparag Reinforcements for both \alliance and \ligue are determined globally
  for all German minor countries involved in an alliance.
  \aparag The \alliance is due to receive 4 \LD and the result of random
  reinforcements in defensive attitude with a global modifier of \bonus{+2}.
  \aparag The controller of the \alliance can pay 50\ducats to give a further
  \bonus{+1} to the reinforcement roll, or 100\ducats for a \bonus{+2}. If it
  does not pay, \SUE has the opportunity to do so and in this case will
  control \alliance for this turn only.
  \aparag The reinforcements of the \alliance are lowered by 1 \LD for each
  one of the following cities that have been conquered by the enemies (even if
  liberated later on): \villeMagdeburg and:
  \begin{itemize}
  \item \villeStuttgart, \villeErfurt if the war follows
    \xnameref{pII:Schmalkaldic League},
  \item \villeMunster, \villeRostock if the war follows \xnameref{pIII:League
      Nassau},
  \item \villeSpeyer, \villePrague if the war follows \xnameref{pIV:Bohemian
      Revolt}
  \item \villeFrankfurt, \villeErfurt if the war follows
    \xnameref{pIV:Augsburg Revocation} or \xnameref{pIV:Unity HRE}.
  \end{itemize}
  \aparag The reinforcements of the \alliance are also lowered by 1 \LD for
  each two cities in the following list that have been conquered by the
  enemies (even if liberated later on): \villeHannover, \villeCassel,
  \villeDresden, \villeBerlin, \villeLubeck, \villeHamburg.
  \aparag If \AUSmin is part of the \ligue, the \ligue is due to receive 3 \LD
  and the result of random reinforcements in defensive attitude with a global
  modifier of \bonus{+2}. Else (\AUS is a \MAJ), the \ligue receives only a
  random reinforcements with a global modifier of \bonus{+2}.  The \ligue uses
  the \ARMY counter of the \HRE regardless of who the Emperor is.
  \aparag The controller of the \ligue can pay 50\ducats to give a further
  \bonus{+1} to the reinforcement roll, or 100\ducats for a \bonus{+2}.
  \aparag The reinforcements of the \ligue are lowered by 1 \LD for each one
  of the following cities that have been conquered by the enemies (even if
  liberated later on): \villeVienne, \villeSalzburg and \villeMunich.
  \aparag{Placement: \alliance then \ligue}
  \bparag The reinforcements obtained are freely distributed among the
  countries part of the alliance. \AUS as a Major power buys its own
  reinforcements but may take up to 2 \LD from the \ligue as its own
  reinforcements.
  \bparag They can only be placed in provinces not pillaged, not controlled by
  the enemy and free of enemy forces.
  \bparag They have to be placed in a province of their nationality, or with
  at least one \LD of the same nationality if their country is nor completely
  occupied by the enemy.
  \aparag[Wallenstein] \HAB may hire mercenary general
  \leaderwithdata{Wallenstein}. He costs 40\ducats (payed by the controller of
  \ligue) to recruit him for one turn.
  \bparag If \leader{Wallenstein} is not hired at turn 1 or 2 of this war, he
  will not be available later. He can not be hired anew after the
  \xnameref{pIV:TYW:Peace Prague}.  The first time \leader{Wallenstein} is
  hired, he appears anywhere in a friendly province of \payshabsbourg or
  \paysBoheme with one Veteran \ARMY\faceplus (use an \AUS or \HRE counter).
  \bparag \leader{Wallenstein} can command any stack of the \ligue (including
  \HAB) but no Bavarian counter.
  \bparag If at the end of a turn the \STAB of the \ligue is positive or its
  situation favourable, \leader{Wallenstein} is dismissed.  He can be hired
  again on the round and/or turn after \ligue suffered a Major Defeat.
  \bparag \MAJHAB can assassinate \leader{Wallenstein} at any time.  He is
  eliminated and \ligue (and \AUT) gain immediately {\bf 1} in \STAB.
  \bparag After the \nameref{pIV:TYW:Peace Prague}, \leader{Wallenstein} is no
  more available.
  \aparag Three mercenary generals are available each turn of this war.  They
  can be recruited by the \ligue or the \alliance. A general is recruited for
  one turn only. He can lead any stack of the alliance (including allied
  \MAJ); by paying 10\ducats more, he can lead a stack even if there is a
  general with higher rank.


  \digression[pIV:TYW:Condition War]{General conditions of the war}

  \phmil
  \aparag Each alliance has a Simple Campaign available each round.  Major or
  Multiple Campaign could be paid for by the controller of the alliance (cost
  lowered by 20\ducats).
  \aparag Each alliance and their allies draw supply in the \HRE from any
  province controlled by their side that is not pillaged or that has an
  unblockaded port.
  \aparag Supply can be traced through any neutral province, or controlled
  province (pillaged or not).
  % (JCD): Neither \HOL nor \FRA nor \POL?
  \aparag Both alliances can freely cross any neutral \HRE minor countries ;
  this is also permitted to \DANdan, \SUE, \ENG in limited intervention, \HAB
  of course and \SPA but not to other allies.
  \aparag Alternatively, a side may, before its movement, declare war against
  any neutral country of the \HRE. Its forces are immediately deployed.
  \aparag All pillages of the \ligue and of the \alliance are decided by their
  controller and goes in their Treasury.
  \aparag A Major Victory involving forces of one or both alliances adjust the
  \STAB of this side accordingly of the usual rules.


  \digression[pIV:TYW:Winning War]{Who is winning the war ?}

  \phpaix
  \aparag No minor country of an alliance ever makes a regular peace (even
  unconditional) outside of the peace of its alliance.
  \aparag One side may be in favoured position depending on the military
  control of the following cities.
  \bparag The \alliance is awarded 2 points for the control of \villeVienne.
  \bparag One point is awarded for each of those: \villeSpeyer, \villePrague,
  \villeMunich, \villeFreiburg, \villeStrasbourg, \villeHannover, \villeKleve,
  \villeCassel, \villeMagdeburg, \villeBerlin, \villeDresden, \villeFrankfurt
  and \villeBrunswick
  \bparag \textonehalf point is awarded for each of these: \villeKoln,
  \villeStuttgart, \villeUlm, \villeMainz, \villeTrier, \villeHamburg,
  \villeMunster and \villeErfurt
  % (Pierre): may be add \villeWiesbaden
  \bparag A side has a favoured position of it has at least 3 points more than
  the other alliance.
  \aparag Both the \alliance and the \ligue lose each {\bf 2} \STAB.
  \aparag Then if a side is favoured, it gains {\bf 1} \STAB.
  \aparag \SPA, \HOL and \AUS lose {\bf 1} \STAB if they were not in the
  original war (in full intervention, not just a limited one) on the previous
  turn.
  % (even if it was a war that lasted since more than one turn
  % ; this war counts as one turn of the current one): their second turn
  % of war just ended.
  \aparag \SPA, \HOL, \AUS lose {\bf 2} \STAB if they were at war (full
  intervention) on the previous turn (even if it was a war that lasted since
  more than one turn ; this war counts as one turn of the current one): their
  second turn of war just ended.
\end{digressions}
\aparag[Result of the Danish Crusade]
\bparag If \DANdan wins a battle against at least 1 \ARMY\faceplus of the
\ligue (or its allies) in the \HRE, is never routed in battle and has forces
left in \HRE at the end of the turn, then its Crusade is successful.
\bparag Thus the \alliance gains {\bf 1} \STAB ; \DANmin is placed in \EG of
\ENG, annexes immediately \provinceLubeck and \provinceHolstein (or
\provinceMecklenburg if it owns already both) and will continue its
intervention until the end of the war, or when it signs any separate peace (in
this war or another). It will not receive reinforcements \emph{per se}, but
some can be given from those of the \alliance.
\bparag If the Danish Crusade failed, \DANmin makes a white peace and
withdraws from the war. \leader{Christian IV} remains as a Danish general for
the full 4 turns.

\tour{Turn 2 -- The Lion of the North (1629--1632)}

\phevnt
\aparag Check for a possible extension of each alliance, see
\ref{pIV:TYW:Extension Alliances}.
\aparag \SUE has to enter the war as an ally of the Protestant \alliance.  If
it is Catholic, roll for 2 \REVOLT in \SUE and it loses {\bf 1} \STAB ;
nothing happens if it is Protestant -- no \CB is necessary and this is not a
declaration of war.
\aparag[Military revolution] \SUE receives \monarque{Gustave Adolphe}. He is
due to last 7 turns.
\bparag If the current Monarch has 1 or 2 turns of life left,
\monarque{Gustave Adolphe} would be his heir. If \monarque{Gustave Adolphe}
dies (in battle) before the current Monarch, \SUE will use the columns 7 to
roll its next Monarch.
\bparag If the current Monarch has more than 2 turns left, \monarque{Gustave
  Adolphe} replaces him entirely and will last for the remaining of the 7
turns as a Monarch (but a death in battle).
\bparag \monarque{Gustave Adolphe} is a military genius, a general
\leaderwithdata{Gustav-Adolf}. As long as the war goes on for \SUE, it
benefits from a Military Revolution (see \ref{chExpenses:Military
  Revolutions})
\bparag[\leader{Sachsen-Weimar}]\label{pIV:TYW:Saxe-Weimar}
\leaderwithdata{Sachsen-Weimar} joins \SUE for 7 turns also.
\bparag If \monarque{Gustave Adolphe} dies, \FRA (if allied to \SUE) may hire
\leader{Saxe-Weimar} as a mercenary general to fight in the present war.  It
costs 30\ducats the first turn, then 20\ducats to keep \leader{Saxe-Weimar};
when \leader{Saxe-Weimar} is not paid one turn, he is eliminated (he does not
go back to \SUE). \leader{Saxe-Weimar} takes command of one German stack of
the \alliance when he goes to \FRA; at each following turns, \FRA can take
half (round down) of the reinforcements of the \alliance (up to 4\LD) to be
placed with \leader{Saxe-Weimar}.  If he dies the forces go back to normal
status in the \alliance.
\aparag \FRA, if Protestant, can begin a limited intervention in the war on
the side of the \alliance.
\aparag Any \MAJ that was doing a limited intervention during the previous
turn (as defined in the original war) can continue this limited intervention
to help the same side.
\aparag \xnameref{pIV:TYW:Turkish Frontier} is in effect this turn.

\phadm
\aparag Roll for reinforcements as in the first turn, see
\xnameref{pIV:TYW:German Reinforcements}.

\phmil
\aparag The war is conducted according to \xnameref{pIV:TYW:Condition War}.
\aparag \SUE takes the control of the forces of one minor country of the
\alliance (its choice). This country can change from one turn to the other and
is chosen at the beginning of any military round of the turn.
\aparag \SUE may force a minor country to enter the war in the \alliance if it
is one of the countries that could join the \alliance and \SUE has at least 1
\ARMY\faceplus and \monarque{Gustave Adolphe} in a province of the country.
\aparag If \SUE makes a siege of allied or neutral \provinceMecklenburg,
\province{Ost Pommern} or \province{West Pommern} with at least one \ARMY
\faceplus, then the city surrenders without fighting at the end of the round.
\aparag All cities taken (by siege, assault or automatic surrender) with at
least one Swedish \ARMY, or only Swedish troops, have now Swedish garrisons
(and the town counts as Swedish presence in the \HRE).  Other Major powers put
their garrison if the city is taken with only their own forces (else, German
garrisons are in charge).

\phpaix
\aparag The balance of the war is checked as in \xnameref{pIV:TYW:Winning
  War}.  The losses of \STAB are applied except that now there is one turn
more:
\bparag Both the \alliance and the \ligue lose each {\bf 3} \STAB.
\bparag Then if a side is favoured, it gains {\bf 2} \STAB.
\bparag Any Major Power in its second turn of war lose {\bf 2} \STAB.
\bparag \SPA, \HOL, \AUS lose {\bf 3} \STAB if they are in their third turn of
war.
\bparag \SUE and \ENG if continuing their intervention lose {\bf 1} \STAB.
\aparag[Suing for peace]\label{pIV:TYW:Suing turn 2}
\bparag A German alliance sues for the \xnameref{pIV:TYW:Peace Prague} when it
is at \bonus{-3} in \STAB at the end of two consecutive turns, and the
position in the \HRE is not in its favour. The enemy side grants necessarily
this peace.
\bparag If both alliances are at \bonus{-3} in \STAB at the end of any turn,
their controllers can agree to a Status Quo and sign the
\nameref{pIV:TYW:Peace Prague}.

\bparag When the \nameref{pIV:TYW:Peace Prague} is signed, the German
alliances are partly dissolved; their stability will not be recorded further
and most of the minor countries in these alliances make a peace.  The
alliances want to stop the war and sign a peace so, from now on, all foreign
countries have no constraint to sign peaces also. It would not be a separate
peace from the German alliance point of view (but could be from another
country...)

\bparag However, if some Major Powers want to keep fighting in the \HRE and
refuse to sign the \nameref{pIV:TYW:Peace Prague}, see \ref{pIV:TYW:War after
  Prague}. Keeping fighting means that the Major power does not sign treaty of
peace with every enemy (that are \MAJ, the enemy German alliance, and possibly
\HOLmin and \DANmin); moreover this country is not allowed to sign a Truce
next turn. \AUSMin signs or not the \nameref{pIV:TYW:Peace Prague} alongside
\SPA.
\bparag If no Major Power contests the \nameref{pIV:TYW:Peace Prague} by
continuing the fight, apply now the \xnameref{pIV:TYW:Peace Westphalie}.

\tour{Turn 3 (1632--1636) and after: a Foreign War}
\history{Turn 4: 1637--1641 (first turn after the Peace of Prague); Turn 5:
  1642--1648 (from Rocroi and Jankov to Lens); Turn 6: 1648-1654 (La Fronde);
  Turn 7: 1654--1660.}

\phevnt
\aparag Check for a possible extension of each alliance, see
\xnameref{pIV:TYW:Extension Alliances}.
\aparag No limited intervention of the previous turn can be carried on.
\aparag At any turn, \FRA and \ENG can enter the war as an ally of the side
they chose. They have a \CB against a side which has not their Religious
Stand, and none against an alliance having the same Religious Attitude; the
\alliance is Protestant and the \ligue is \CATHCR.
\aparag At any turn, \POL (unless it is Orthodox) can make a full or limited
intervention in the war as an ally of any side. \POL can do such an
intervention only once during the war. It has a \CB only against an alliance
that has not the exact same Religious Attitude (relative to Catholicism) as
itself.

\phadm
\aparag Roll for reinforcements as in the first turn, see
\xnameref{pIV:TYW:German Reinforcements}.
\aparag Two turns after a Military Revolution caused by \SUE, the Land
Technology of the Latin minor countries reaches this new Technology.

\phmil
\aparag The war is conducted according to \xnameref{pIV:TYW:Condition War}.
\aparag \SUE takes the control of the forces of one minor country of the
\alliance (its choice). This country can change from one turn to the other and
is chosen at the beginning of any military round of the turn.
\aparag On the third turn only (not after), if \SUE makes a siege of allied or
neutral \provinceMecklenburg, \province{Ost Pommern} or \province{West
  Pommern} with at least one \ARMY \faceplus, then the city surrenders without
fighting at the end of the round.
\aparag All cities taken (by siege, assault or automatic surrender) with at
least one Swedish \ARMY, or only Swedish troops, have now Swedish garrisons
(and the town counts as Swedish presence in the \HRE).  Other Major powers put
their garrison if the city is taken with only their own forces (else, German
garrisons are in charge).

\phpaix
\aparag The balance of the war is checked as in \xnameref{pIV:TYW:Winning
  War}.  The losses of \STAB are applied with one turn more. This war can not
cause a loss more than {\bf 4} \STAB at the end of turn.  On turn 3 of the
Religious War, the losses should be:
\bparag the \alliance and the \ligue lose {\bf 4} \STAB;
\bparag any Major Power in its third turn of war lose {\bf 3} \STAB.
\bparag \SPA, \HOL, \AUS lose {\bf 4} \STAB if they were at war before the
Religious War in the \HRE.
\bparag \SUE loses {\bf 2} \STAB.
\bparag Any other Major Power intervening in the war at this turn lose {\bf 1}
\STAB.
\aparag[Suing for peace] As described in \ref{pIV:TYW:Suing turn 2}.
\aparag If \SUE, \ENG or \POL (in full intervention) do not hold any city nor
have any land forces left in the \HRE, they make a mandatory white peace
against all its enemies in this war. This will count as a losing position in
\xnameref{pIV:TYW:Peace Westphalie}.
\aparag If \POL is doing a limited intervention and wins a battle against at
least one \ARMY\faceplus of the enemy side (any nationality) in the \HRE, then
loses no battle in the \HRE, the alliance it helps gains {\bf 1} in \STAB
(\AUS also). \POL may then annex \province{Ost Pommern} or any province in the
\HRE that once was Polish. Its limited intervention lasts only one turn.

\begin{digressions}[Between Prague and Westphalie]


  \digression[pIV:TYW:Peace Prague]{Peace of Prague}
  \aparag If the \ligue is favoured by the Peace:
  \bparag \paysBaviere gains permanently its second \ARMY and \paysPalatinat
  loses its own; \paysBaviere is now an Electorate.  It also gains a permanent
  \bonus{+1} to its reinforcement rolls.
  \bparag \paysBaviere annexes \provinceOberPfalz, except if this war follows
  \xnameref{pII:Schmalkaldic League}, in which case it annexes
  \provinceSchwaben.
  \bparag \paysBaviere is now in \AM with \HAB (move its diplomatic marker
  accordingly).
  \bparag A Total Victory of the \ligue in the \xnameref{pIV:TYW:Peace
    Westphalie} is possible.
  \bparag Any specific consequence given by the victory of the side of the
  \ligue in the war having caused \ref{pIV:TYW} is applied.
  \bparag The Truce of Augsburg is revoked.
  \bparag \SPA and \AUS gain 30 \PV, \SUE loses 10 \PV.
  \aparag If the Peace is a Status Quo:
  \bparag \paysBaviere keeps its second army for the continuation of this war
  (but not permanently).
  \bparag The Truce of Augsburg is in effect.
  \bparag No side can achieve Total Victory in the \xnameref{pIV:TYW:Peace
    Westphalie}.
  \aparag If the \alliance is favoured by the Peace:
  \bparag The Truce of Augsburg is in effect.
  \bparag A \xnameref{pIV:TYW:Northern HRE Alliance} is created and allied to
  \HOL.
  \bparag \paysOldenburg, \paysHanovre, \paysHesse, \paysHanse and \paysBerg
  are placed in \EG of \HOL.
  \bparag A Total Victory of the \alliance is now possible.
  \bparag \HOL and \SUE gain 30 \PV.


  \digression[pIV:TYW:War after Prague]{The War after Prague}
  \aparag Only some minor countries continue the war. All other minor
  countries of the \HRE surrender: their forces are withdrawn and their cities
  are considered as taken for the reinforcements.
  \bparag On the side of the \ligue: \HAB and, if the Peace is not in favour
  of the \alliance, \paysBaviere.
  \bparag On the side of the \alliance: the controller is now \SUE and it
  chooses 2 countries, or only 1 in case of unfavourable Peace, that will
  continue the fight from the following list: \paysHesse, \paysHanovre,
  \paysPalatinat, \paysSaxe.
  \bparag If the Peace is favourable to the \ligue, \paysSaxe reverses its
  alliance and enters war with the Catholics.  All its forces are withdrawn
  from the map, and the cities of \paysSaxe surrender immediately to the
  Catholics; Protestant forces in the provinces are withdrawn.
  \bparag \paysBrandebourg will continue (or enter) the war as an ally of the
  Protestant if \SUE gives up its claims on \province{West Pommern} to
  \paysBrandebourg in \xnameref{pIV:TYW:Peace Westphalie}.
  \bparag If \FRA hires \leader{Saxe-Weimar} at this turn (continuing from a
  previous turn or not), he keeps one stack of any one protestant
  country. This country remains at war (until it surrendered unconditionally
  or \leader{Saxe-Weimar} is no more at the service of \FRA). It will receive
  reinforcements for this stack (using the mechanism for the stack of
  \leader{Saxe-Weimar}).
  \aparag The minor countries that continue the war are allied in their
  alliance, and with the Major countries in the war. But they want peace so
  they will stop fighting as soon as all foreign minor/major countries do
  likewise.
  \bparag A minor country of the \HRE can now be ejected from its alliance and
  from the war, but only by imposing an unconditional surrender on it; other
  regular peaces are not possible.
  % (Jym): theta-N.1.b systematically gives control to SUE. Commenting out.
  % \bparag If the \nameref{pIV:TYW:Peace Prague} was not favourable to the
  % \alliance, the controller of the \alliance is now \SUE in priority
  % (whatever its Religion is).%, then \HOL (and so on).
  \aparag All other minor countries that were in both alliances are now at
  peace; they all have now a Neutral diplomatic status.  All the cities in
  those countries are considered conquered in order to check for
  reinforcements.
  \aparag Foreign minor country \DANmin stops the war whereas \HOLmin
  continues. A regular peace has to be obtained against it.
  \aparag Do not forget that this war causes at most a loss of {\bf 4} \STAB
  for each country at the end of turn.  If the War caused by the Revolt of the
  United Provinces continue, it resumes its normal loss in \STAB only if an
  Armistice is made (at least 1 turn) between \SPA and \HOL at the end of the
  present war; else the present war has to continue and so does the loss of
  {\bf 4} \STAB each turn.


  \digression[pIV:TYW:Peace Westphalie]{Peace of Westphalie}
  \aparag This Peace is signed at the end of a turn, beginning with the turn
  of the \nameref{pIV:TYW:Peace Prague}, if all Major countries in the war
  agree to end the war, that is to sign Peaces or Armistices between them. The
  following effects are implemented as further consequences of the regular
  Peace Treaties.
  \aparag The Emperor of the \HRE is now \HAB if this was not, for the rest of
  the game.
  \aparag The Major Countries that can be involved in the war are \SPA (and
  \AUSmin), \AUS, \FRA, \HOL, \SUE, \ENG and \POL.
  \bparag A Major Power that stops the war (it has signed Peaces or Armistices
  with all other Major Powers at the end of some turn) before the end has a
  losing position for this Peace; it has also this position if it signs a
  mandatory white peace (for any reason).
  \bparag A Major Power has a dominant position if it signs only winning
  Treaties of Peace with countries of the other side (no Armistices or White
  Peaces either) on the last turn of this war.
  \bparag A Major Power has a losing position if it signs only losing Treaties
  of Peace with countries of the other side (no Armistices or White Peaces
  either) on the last turn of this war.
  \bparag In other cases, the position is neutral.
  \aparag[Spain or Austria]
  \bparag These specific conditions are for \MAJHAB.
  \bparag A \AUSmin will continue to fight with \SPA until the end of the war
  (except by unconditional surrender, following the rules for all minor
  countries from the \HRE still at war after the \nameref{pIV:TYW:Peace
    Prague}).
  \bparag If \HAB is in dominant position and a Catholic Total Victory was
  possible, the \pays{German Empire} is created (see \shortref{pIV:TYW:German
    Empire}).
  \bparag If \HAB is in dominant position but no Catholic Total Victory was
  possible, a \xnameref{pIV:TYW:Southern HRE Alliance} is associated to \HAB.
  The countries in this alliance are put in \EG of \HAB: \paysBaviere,
  \paysTreves, \paysAlsace, \paysBade and \paysWurtemberg.
  \bparag \HAB in neutral or losing position: nothing more.
  \aparag[Spain] If \SPA is in dominant position and there is a Major \AUS,
  the \xnameref{pV:WoSS} will only concern provinces in the Low Countries and
  northern Italy (those in southern Italy are not part of the Inheritance and
  remain Spanish).
  \aparag[Austria] If \AUS is in neutral position, it gains a permanent {\bf
    +1} bonus in Diplomacy on Catholic countries of the \HRE.
  \aparag[The Netherlands]
  \bparag If \HOLhol has a dominant position and a Protestant Total Victory is
  possible, \paysHanse annexes \provinceOldenburg and \HOL gains \paysHanse as
  a permanent \VASSAL.  Eliminating the \xnameref{pIV:TYW:Northern HRE
    Alliance} will now need a Peace of level 5 against \HOL.
  \bparag If \HOLhol has a dominant position (but without possible Protestant
  Total Victory), it gains \paysHanse as a normal \VASSAL and \paysHanse
  annexes \provinceOldenburg. The \xnameref{pIV:TYW:Northern HRE Alliance} is
  created and allied to \HOLhol with the corresponding effects.
  \bparag If \HOL has a neutral position, it has the choice to allow or not to
  the destruction of \paysHanse (its controller in the case of a \HOLmin).
  \bparag Else, if \HOL (or minor \payshollande) is in losing position, the
  \paysHanse is destroyed and the \xnameref{pIV:TYW:Northern HRE Alliance} is
  dissolved.
  \aparag[Sweden]
  \bparag If \SUE has a dominant position, it annexes \provinceMecklenburg,
  then \province{West Pommern} if it has not renounced its claims on this
  province (else it gains \paysBrandebourg in \EG) and \provinceBremen or
  \provinceLubeck (its choice). It then chooses one Protestant minor country
  (or 3 minor countries if a Protestant Total Victory was possible) of the
  \HRE that is (are) placed in \EG on its Diplomatic chart.
  \bparag If \SUE is in neutral position, it annexes \provinceMecklenburg,
  then \province{West Pommern} if it has not renounced its claims on this
  province; else it gains \paysBrandebourg in \EG. It then chooses one
  Protestant minor country of the \HRE that is placed in \EG on its Diplomatic
  chart.
  \bparag If \SUE is in losing position, it gains nothing.
  \aparag[France]
  \bparag If \FRA is in dominant position, it gains a \bonus{+1} bonus for
  Diplomacy on countries of the \HRE until the end of the period.
  \bparag If \FRA is in dominant or neutral position, it gains \paysAlsace as
  a \VASSAL and \paysCologne in \EC.
  \aparag[England] If \ENG is in dominant position, it gains a \bonus{+1}
  bonus for Diplomacy on countries of the \HRE until the end of period V.  It
  also gains a minor country of its choice, having the same religion as \ENG,
  that is placed in \EG on its chart.
  \aparag[Poland] If \POL is in dominant position after a full intervention,
  it gains a \bonus{+1} bonus for Diplomacy on countries of the \HRE until the
  end of period V.  It also gains a minor country of its choice, having the
  same religion as \POL, that is placed in \EG on its chart.
  \aparag When a major country can take a the diplomatic control of a minor
  country, the order of choice is the order written here, and a power can only
  choose neutral minor country of the \HRE (not those already allied to
  someone else).
  \aparag \paysBrandebourg annexes \province{Ost Pommern} if it is in
  \paysHanse.
  \aparag Then, if \paysHanse has to be destroyed, its remaining provinces are
  now given as follows: \SUE takes \provinceBremen, \paysBrandebourg takes
  \province{West Pommern} and \provinceMecklenburg, then \DANmin all the
  remaining ones.
  \aparag From now on, any major power that owns a province in \HRE or
  adjacent to a province of the \HRE may, when at war, enter and remain in any
  neutral province of the \HRE. The cost in \MP is the same as enemy
  territory. The neutral provinces can not be pillaged, besieged nor give
  supply (but supply lines can cross those if there are no enemy force
  within).
  \bparag
  \aparag[Victory Points]
  \bparag A Major Power in dominant position at the end of the war wins 30 \PV
  (added to those of the treaties of Peace).
  \bparag A Major Power in losing position at the end of the war loses 30 \PV.
\end{digressions}

\begin{digressions}[German alliances emerging from the war]


  \digression[pIV:TYW:Northern HRE Alliance]{Northern \HRE Alliance}

  \effetlong
  \aparag When this alliance exists, it is allied to \HOLhol.  It represents
  treaties between \paysOldenburg, \paysHanovre, \paysHesse, \paysHanse and
  \paysBerg.
  \aparag \HOL has a permanent bonus of \bonus{+2} in Diplomacy on these
  countries.
  \aparag \HOL gains also a income of 10\ducats for each coastal city in
  \paysHanse if it is on his diplomatic track.
  \bparag This Northern alliance is dissolved when \HOL signs a losing Peace
  of level 3 or higher, or when it controls no country of the alliance. The
  bonuses are permanently lost.


  \digression[pIV:TYW:Southern HRE Alliance]{Southern \HRE Alliance}

  \effetlong
  \aparag A Southern \HRE alliance is associated to \HAB, composed by the
  following countries: \paysBaviere, \paysMayence, \paysAlsace, \paysBade and
  \paysWurtemberg.
  \aparag Each of these countries on the \HAB or \MAJHAB diplomatic chart will
  give an income of 10\ducats to \MAJHAB.
  \aparag \MAJHAB gains a \bonus{+1} bonus in Diplomacy on every Catholic
  countries in the \HRE.
  \aparag This Southern alliance is dissolved when \MAJHAB signs a losing
  Peace of level 3 or more, or when neither \MAJHAB nor \HAB controls any
  country of the Alliance.  The bonuses are permanently lost.
  \aparag When a \GE is created, the Southern alliance is also dissolved (and
  becomes part of the \GE).


  \digression[pIV:TYW:German Empire]{German Empire}

  \effetlong
  \aparag All minor countries of the \HRE (except \HAB which remains
  independent) are associated in one minor country, called the \pays{German
    Empire}. This country is a permanent \VASSAL of \MAJHAB. It can use 4
  \ARMY counters, and 12 \LD (for practical ease, use the counter of the \HRE
  and any counter of some part of the empire, with no notion of nationality --
  there are all from the \GE).  Its basic forces are one \ARMY\faceplus and
  one \ARMY\facemoins. It has a modifier of \bonus{+2} for reinforcements and
  always makes peace with \MAJHAB.
  \aparag \MAJHAB receives an income of 100\ducats from the \HRE (and not the
  exact value of the country) and can use its port on the Baltic Sea.
  \aparag When the \pays{German Empire} exists, the Dynastic Alliance between
  \AUSmin and \SPA is both defensive and offensive.
  \aparag Some events may dissolve part of the \pays{German Empire} by
  creating a League (\xnameref{pII:Schmalkaldic League}, \xnameref{pIII:League
    Nassau}, \xnameref{pIV:Bohemian Revolt}, \xnameref{pIV:Augsburg
    Revocation}, \xnameref{pIV:Unity HRE}) which ceases to be in the Empire,
  and is (depending on the event) at war with the Emperor. An unconditional
  peace of the Emperor on any of those countries bring it back in \pays{German
    Empire}.
  \aparag Event \xref{pV:Kingdom Prussia} liberates \paysBrandebourg from
  \pays{German Empire} (and it can't be forced back in).
  \aparag When any province with a capital of \pays{German Empire} is lost as
  the result of a Peace, the minor country having this capital is renewed as a
  free country, having status \EG or \VASSAL (if possible) with the \MAJ that
  liberated it (player's choice). \HAB can force the \MIN back in the
  \pays{German Empire} by means of an unconditional peace on it.
  % (Jym) For the Hansa, is it enough to free one capital?
  \aparag Some events (\xnameref{pIV:Augsburg Revocation}, \xnameref{pIV:Unity
    HRE} and \xnameref{pV:Devolution War}) can cause Civil War in \pays{German
    Empire} that foreign countries can help in order to dissolve \pays{German
    Empire}.
  \aparag The \xnameref{pV:WoSS} may separate the Spanish dynasty from the
  Austrian dynasty because of a Crisis of Succession.
  \bparag If \SPA chooses a \AUSmin Heir, the \pays{German Empire} fights
  along their side with no Dynastic Separation.
  \bparag If \SPA chooses another Heir than a \AUSmin, the \pays{German
    Empire} is dissolved but \paysBaviere, \paysMayence, \paysLorraine,
  \paysBade and \paysWurtemberg are placed in \AM of \HAB and enters war at
  its side; and \HAB gains the benefits of \xnameref{pIV:TYW:Southern HRE
    Alliance}. All other countries that are recreated at this time are
  Neutral.
  \bparag \AUS (if major) keeps the \pays{German Empire}.
  \bparag See the other conditions in this event.
  \aparag The \pays{German Empire} ceases to exist as soon as its controller
  is forced to sign any peace of level 3 or more.  In addition to the normal
  peace conditions, \pays{German Empire} is dissolved: all minor countries of
  the HRE are back to previous frontiers, and are neutral.
\end{digressions}

% Local Variables:
% fill-column: 78
% coding: utf-8-unix
% mode-require-final-newline: t
% mode: flyspell
% ispell-local-dictionary: "british"
% End:

% LocalWords: defensive se Hansa offensive pIII FWR Schmalkaldic pIV TYW pII
% LocalWords: Ost Pommern HRE Westphalie unblockaded JCD Sachsen Weimar Quo
% LocalWords: pV WoSS Jym Rocroi Jankov


\vfill \pagebreak



\event{pIV:Polish Civil War}{IV-B}{Civil War in Poland}{1}{PB}

\history{\textit{alternative history}}
\dure{Until the end of the war}

\phevnt
\aparag Can only happen once, either as explained in \ref{pIV:Liberum Veto} or
in \ref{pV:Saxon King Poland}.
\aparag \POL is now in civil war. One side, called ``Absolutists'' remain
loyal to the King and try to impose Absolutism in \POL while the other side,
called ``Rebels'' is lead by the great nobles of the kingdom trying to keep
the Republic and the elective monarchy.
\bparag The player plays the Absolutists.
\aparag If they have a province bordering \POL, the following countries can
enter a full war against any of the side: \RUS, \SUE, \HAB, \PRU.
\bparag They have a free \CB this turn against both sides of the civil war.
\bparag Other countries can only make a foreign intervention as per normal
rules.
\aparag \textbf{Economic and Political crisis}: The \RT of \POL is diminished
by half and loses at least 50\ducats. \POL loses 2 \STAB.
% \begin{oldcompta}
%   \bparag Do not take into account the minimal loss of 50\ducats.
% \end{oldcompta}
\aparag The Rebels control the following provinces:
\bparag \provinceMalopolska, \provinceLietuva ;
\bparag one other province randomly chosen in \paysmajeurPologne;
\bparag two other provinces randomly chosen in \paysmajeurLithuanie.
\bparag The 5 provinces must be different and all possessed by \POL at the
beginning of the event.
% (JCD): Controlling cities is the same as controlling provinces! removing
% \bparag The Rebels control the cities in these 5 provinces.
\aparag Roll for two \REVOLT in \POL. There are \facemoins and do not control
the cities.
\aparag If \ref{pIV:Times of Troubles} already happened but not
\ref{pIV:Revolt Cossacks} and the religious attitude of \POL is not Tolerance
of the Orthodoxy, \nameref{pIV:Revolt Cossacks} happens immediately.
\aparag The Rebels side is played by the first country at war against the
Absolutists in the following list: \RUS, \SUE, \HAB, \TUR, \HOL, \ANG, \FRA,
\PRU.
\bparag If none is at war against the Absolutists, then the Rebels are played
by the first country in the same list which is not at war as an ally of the
Absolutists.

\phadm
\aparag Lands forces of \POL equal to the basic forces for the period
(excluding Ukraine) become Rebels.
\bparag If \POL does not have enough troops raised, an immediate levy happens,
paid for by the treasure of \POL (even if this causes a bankruptcy).
\aparag The basic upkeep for the Absolutists is the one of \paysPologne only
(\ARMY\faceplus).
\bparag The player may use the counters of \paysPologne and two \ARMY (they
can be taken from any unused country, and are similar to any other Polish
Army)
\bparag Absolutists receive normal income from the provinces they control.
\bparag Absolutists troops in rebel provinces (at the beginning of the war)
are retreated normally.
\bparag Fleet stay loyal to the Absolutists.
\bparag The king of \POL must be used as a general of the Absolutists, except
if he is \monarque{August II}.
\aparag The Rebels side uses the counters of \paysLithuanie as well as two
revolts \ARMY.
\bparag He does not get reinforcement at the first turn of the war.
\bparag At the first turn of the war, the Rebels forces can be freely
redeployed in the controlled provinces.
\bparag If a named general (other than \leaderPatkul when \monarque{August II}
is king) is in play, he takes side for the Rebels. Otherwise, the Rebels are
lead by a random mercenary general and get an extra random general.
\aparag \REVOLT in \POL are friendly to the Rebels.
\bparag A rebel general can lead a \REVOLT . A \REVOLT \facemoins count as
2\LD for hierarchy rules.
\aparag Starting with the second turn of the war, Rebels get reinforcement
either in offensive or defensive attitude based on the income of the province
they control (control the city with no absolutist army in the province).
\aparag If \paysukraine is not in revolt or independent due to \ref{pIV:Revolt
  Cossacks}, the Ukrainian army can be used by the Absolutists (but without
the basic upkeep for it).
\aparag If the king is member of the dynasty of \paysSaxe, he can use the
forces of the minor as per the rules of \ref{pV:Saxon King Poland}.
\bparag In that case, \paysSaxe is at war against the Rebels and their allies
can freely cross the \HRE and wage war in \paysSaxe.

\phmil
\aparag Absolutists and Rebels get supply from the cities they control.
\bparag They can cross enemy provinces without besieging the city.
\bparag This is only true for polish forces. Not for the foreign allies.
\bparag The Absolutists cannot cross freely the provinces with a \REVOLT .

\phpaix
\aparag Victory in the civil war occurs as soon as one side gets two out of
the following three conditions:
\bparag controlling the capital (controlling \provinceMalopolska and, if
\villeVarsovie has been made capital, \provinceMazowia) ;
\bparag controlling the country (military control of at least 60\% of the
provinces, that is controlling the city without enemy presence ; provinces
with a \REVOLT and the city still controlled by the Absolutists count for
nobody) ;
\bparag military victory (having one more major victory than the other side
this turn, or the other side as no more \ARMY in play).
\aparag The war lasts as long as no side wins.
\aparag Wars with foreign countries can be ended by separate peaces.
\bparag If the Absolutists are not fully at war against another major country,
\POL does not lose \STAB due to the war (but does so due to \REVOLT ).
\bparag A (foreign) peace in the civil war is also a peace with \POL (if
another war was occurring), or a separate peace with loss of 2 \STAB for
breaking the alliance with the side of the civil war the foreign country was
allied to.
\aparag[Absolutists victory]
\bparag The effect of \ref{chSpecific:Poland:Liberum Veto} are cancelled.
\bparag Events \xref{pVI:Great Northern War}, \xref{pVII:Bar Confederation},
\xref{pVII:First Partition Poland}, \xref{pVII:Second Partition Poland} and
\xref{pVII:National Revival of Poland} are modified.
\bparag Any country fully allied with the Absolutists who accept the peace
annexes a province of \POL (\POL choose which).
\bparag The Rebels armies are eliminated.
\bparag The \REVOLT stay in place.
\aparag[Rebels victory]
% (Jym): added condition on election
\bparag A dynastic crisis occurs and a new king is elected (this is a change
of polish dynasty), a general cannot be elected king unless he took the side
of the Rebels.
\bparag A Polish provinces is given to each \MAJ who was fully at war against
the Absolutists (choice is made by the \MAJ receiving the province, in order
of initiative).
\bparag The \STAB of \POL immediately becomes -1.
\bparag The \REVOLT and the Absolutists armies are removed.

\stopevents

% Local Variables:
% fill-column: 78
% coding: utf-8-unix
% mode-require-final-newline: t
% mode: flyspell
% ispell-local-dictionary: "british"
% End:

% LocalWords: pIV TYW HRE offensive reroll minister Olivares Safavids pIII pV
% LocalWords: Liberum Oxenstierna Torstensson Moghol Singala CCA PBNew JCD de
% LocalWords: Vassalisation Bethlén Mansfeld Westphalie Ausgsburg Schmalkaldic
% LocalWords: malus Risto Espagne HOL Ormus POR Jym reannexed Reannexation
% LocalWords: RistoMod Angleterre ECW Montrose Duche Prusse Pommern Ost JCA
% LocalWords: Torstensson's Fyodor defensive Formose pII Alaouite pVI pVII
% LocalWords: PBNotEvenWritten willingfully
