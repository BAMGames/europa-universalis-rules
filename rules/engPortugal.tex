\sectionJ{\anchorpaysmajeur{Portugal}}{\blason{portugal}}
\aparag For the transfer to \paysmajeur{Suede}, see
\ruleref{chSpecific:Campaign:Transfer Portugal}

\subsection{The Overseas Empire}
\subsubsection{Viceroys of the India}\label{chSpecific:Portugal:Viceroys}
\aparag Several leaders of \POR are designated as Viceroys (red symbol
instead of a black one): \listleadershort{leadersporviceroy}.
% \leader{Da Gama}, \leader{Almeida}, \leader{Albuquerque},
% \leader{Albergaria}, \leader{de Castro}, \leaderNoronha.
They give \POR the following advantages.

\aparag[Autonomy of the Viceroys.]
\bparag As long as there is a VR in play, \POR has permanent free
Overseas \CB against any non-European country.
\bparag The presence of a VR in any region of \ROTW gives a bonus of
{\bf +2} to \CONC attempt on \TP and \COL in the region.
\aparag \POR may raise exceptional taxes if engaged in \terme{Overseas
  Wars}.

\aparag \POR may also raise Exceptionnal Levies (see
\ruleref{chMilitary:Exceptional Levies}) if engaged in \terme{Overseas
  Wars}, or a war against a \ROTW power with modified condiyions
\bparag Conditions: having a Viceroy; having suffered a major defeat in
naval or land battle in \ROTW this round; or having suffered a defeat in
naval or land battle in \ROTW this round withe the VR and lose an
additional 1 \STAB.
\bparag In that case, \POR may recruit land forces by Exceptional
Levies, without any reduction of its land recruitment limit, but only in
\ROTW.

\aparag[The \anchorconstruction{Goa}
colony.]\label{chSpecific:Portugal:Goa Colony}
\bparag If there is a VR in \continent{India} and \POR has an \dipAT
with \pays{Vijayanagar}, it can attack a city in a province where there
is a \TP without declaring war to \pays{Vijayanagar}.
\bparag If the \TP is controlled by \POR (its own, or conquered), the
control of the city allow \POR to try to transform the \TP in a \COL or
\POR, as per \ruleref{chAdministration:TP to Col}.
\bparag Neither \pays{Vijayanagar} nor \pays{Moghol} will ever react to
the presence of a Portuguese \COL in its territory.

\aparag[Occupation of \sectionpays{Aden} and \sectionpays{Oman}.]
\bparag Any VR can enter the \COL of \pays{Oman} and \pays{Aden} with
military forces if at peace with the country (passive campaign), in a
attempt to submit it. A test of reaction is made for this country
immediately.
\bparag If there is a reaction, an immediate Overseas War begins (with
no formal declaration of war). The forces of the \MIN are deployed and
there is an immediate battle between their forces and the stack of the
VR. Any country having \dipAT with the \MIN can freely joins this
Overseas war at the same time.
\bparag If there is no reaction, the \MIN is submitted, signs an
\dipAT with \POR and breaks any other status with other powers. [BLP]
Place a Portuguese occupation here. As long as a Portuguese occupation
is on the \COL of the \MIN, \POR exploits the resources of the \COL as
its own (and also the \TP of \province{Zanzibar} if \pays{Oman} is
occupied). \POR can built fort or fortress on the \COL/\TP to support
its occupation, but may not use the \MIN as an ally. The \dipAT can
not be broken by usual diplomacy.
\bparag Enemies of \POR can enter the submitted \MIN and attack
Portuguese forces. The \dipAT is lost by \POR if the occupation is
lost (\emph{i.e.} if the garrison is destroyed), but could be renewed
at the same conditions by a VR.

\aparag[The trading post in \sectionpays{Ormus}.]
\bparag The first time a Portuguese VR is in \province{Ormus} at the
beginning of a phase of Diplomacy, \POR raises its overseas relations
with \pays{Ormus} --- actually \pays{Perse} --- by 1 (from \dipNR to \dipFR,
or from \dipFR to \dipAT). It still can use a diplomatic action to raise
it further this turn.

\subsubsection{Portuguese Missions and Missionaries}
\aparag See \ruleref{chSpecific:Missions} for the general rules.

% \aparag Portuguese Missionaries appear at fixed turns (written on the
% counter). They appear during the Interphase of the previous turn (so
% that they are active during the turn of appearance shown).
% \bparag A destroyed Missionary is available in the following Interphase
% to be placed in Europe.
% \bparag A Jesuit with bonus {\bf 1} is available on turn 3, a Jesuit of
% bonus {\bf 2} on turn 4, and St Francis Xavier (bonus {\bf 3}) on turn
% 10.

\aparag Portuguese Missions give a bonus of {\bf +2} (instead of +1)
to improve \TP and \COL in \continent{Asia}, and to improve \COL in
\continent{Brazil}.

\aparag At the end of each period, \POR loses 10\VPs for each \COL
that is neither in \continent{Brazil} nor in \granderegion{Cabo Verde}
with no Mission on it (in the same province).

\aparag Installed missions are kept when \pays{Portugal} becomes a minor
power; missionaries are lost and no further missionaries will be
received.

\aparag[The \anchorconstruction{Kongo} mission.] At the start of the
game \POR has a Mission already in place. If this mission is destroyed, it
is removed form the game and may not be rebuilt.

\begin{designnote}
  It represents the contacts made by Henry the Navigator with the
  kingdom of Kongo.
\end{designnote}

\subsubsection{Portuguese colonial militia}
\aparag Portuguese Colonial Militia are more numerous: one \LDE for each
level of \COL and are always Veterans.

\subsubsection{Exclusivity on Portuguese discoveries}
\aparag \POR is not allowed to sell, give or trade any of his
discoveries, Colonies (except those that may be concerned with the
Tordesillas Treaty application, see \eventref{pI:Tordesillas}) or
Trading Post with any other player.
\aparag[Exclusive trading] \POR may not give the authorisation of trade
to other countries in any sea zone where it has a \COL/\TP.

\subsubsection{The African gold}\label{chSpecific:Portugal:African Gold}
\aparag[The Gold in \anchorconstruction{Elmina}.] The Portuguese \TP\
\construction{Elmina} in \granderegion{Cotedor} that exists in 1492,
exploits two \RES{Gold Mines} (for an income of 40\ducats) that have
the same status as European Mines. This does not counts as gold from
the \ROTW for Inflation. It can also exploit \RES{Slaves} in the
region. The \RES{Gold Mines} disappear when \POR is no more a \MAJ, or
if the \TP is destroyed or given to another country.

\subsubsection{Portuguese Explorers}
\aparag Some Portuguese leaders have two sides (\leader{Da Gama},
\leader{Almeida}, \leader{Albuquerque}, \leader{Albergaria},
\leader{Pinto}).
\bparag\label{chSpecific:Portugal:Explorers} Contrarily to
\ruleref{chMilitary:Double Sided Leaders}, these leaders can be switched
at will on one side or another (even change during a round). Thus, they
can lead fleets using their \Man as an admiral, and explore a province
with their full \Man as a conquistador.
\bparag The category they count in is marked by a {\textetoile} on one
of the sides.
\aparag[Foreign trade index] \POR has a specific \FTI for \ROTW
operations, that is different from its \FTI (see
\ruleref{chAdministration:Special FTI}).
\bparag This \FTI is no more used when \POR is a minor country.


\subsection{\sectionpaysmajeur{Portugal} in play}
\subsubsection{Portuguese Monarchs}
\aparag[\anchormonarque{Joao II} and \anchormonarque{Manuel I}] are the
first two monarchs in 1492. \monarque{Joao II}, with values 8/6/7, dies
at the beginning of turn 2. His heir is \monarque{Manuel I}, with values
8/6/8, scheduled to die at the beginning of turn 7.
\subsubsection{Available counters}
\aparag[Military] 1\ARMY, 1\FLEET, 1\corsaire, 7\LDND, 3\LD, 4\NTD,
8\LDENDE, 3 fortresses 1/2, 5 fortresses 2/3, 2 fortresses 3/4, 4 forts,
2 Arsenals 2/3, 2 Arsenals 3/4, 3 Missions.
\aparag[Economical] 12\COL, 12\TP, 6\MNU, 8\TradeFLEET, 6 \ROTW treaty
counters.

\subsection{\sectionpaysmajeur{Portugal} as a minor
  country}\label{chSpecific:Portugal:Minor}
\aparag See \ruleref{chSpecific:Campaign:Transfer Portugal} for the conditions
of the transfer proper.
\aparag Before \eventref{pIII:Portuguese Disaster}, Portugal has 1\TFI,
1 \TPaction and 1 \COLaction.
\bparag This is lowered to 1\TFI and 1 \TP or 1 \COLaction after
\eventref{pIII:Portuguese Disaster}.
\bparag This is lowered to 1\TradeFLEET or 1 \TP or 1 \COLaction after
\eventref{pVI:Treaty Methuen}.
\bparag During annexation by \SPA, there are no actions (but \SPA has a
specific number of actions for \pays{Portugal}). All those actions are mandatory.
\bparag If \pays{Portugal} is \Neutral, \SPA plays these actions. Else,
the patron has this duty.
\aparag \pays{Portugal} has commercial fleets and a base \DTI and \FTI
of 3, or 4 in periods IV to VII.
\aparag \pays{Portugal} only gives authorisation of implantation of
Commercial fleets in \STZ adjacent to its \COL/\TP on the following
occasions:
\bparag To \SPA when it is annexed by this power;
\bparag To \HOL if, by setting the peace at the end of
\subeventref{pIII:DR:War Holland Portugal}, \HOL takes the right by
renouncing to take one \COL/\TP that it could annexe;
\bparag To \ENG, when the \eventref{pVI:Treaty Methuen} is signed.

% LocalWords: Da Almeida de Albergaria VR Moghol Ormus Perse Elmina os pI
% LocalWords: pIII pVI TP FTI Interphase Tordesillas Cotedor Joao
% LocalWords: Tejo Methuen Tras Montes Cabo Beira Alentejo
% LocalWords: Algarve Tanger Acores Praya malus Gambie Arguin
% LocalWords: Bonne Esperance Cameroun Guinee
