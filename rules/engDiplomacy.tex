% -*- mode: LaTeX; -*-

\definechapterbackground{Diplomacy}{diplomacy}
\chapter{Diplomacy}
\label{chapter:Diplo}




\section{Diplomatic phase}

\aparag[Overview.]
This phase is played simultaneously. Players may negotiate to establish
agreements between them (official alliances or informal agreements).  During
this phase also, players may declare wars (between them or against minor
countries).  They end this phase by making diplomatic actions with minor
countries to control them (or to lower the level of control of other
players). These levels of control are appreciated to various degree (Royal
Marriage, Subsidies, Military Alliance... etc.).
\aparag[Sequence.]
\DiploDetails

% -*- mode: LaTeX; -*-

\section{Agreements between Major Powers}

% RaW: [32,34]



\subsection{Negotiations}


\subsubsection{Negotiations between Players}
\aparag
Players can negotiate freely between them to get into various kinds of
agreements, as long as they respect the letter and the spirit of the
rules. Players' diplomatic relationships may however be "officialized" in
alliances, or may be broken.
\aparag
Players negotiate between them, freely. It is advised that the time of
negotiations be limited to at most 10 minutes on an average (5 is counselled,
but not always possible or realistic).


\subsubsection{Outcome of Agreements}
\aparag When negotiations are closed, players announce their agreements:
informal agreement, or formal agreements: alliance (by specifying which), or
some possible trade refusal.
\bparag
This is done during the Diplomatic Phase on the fourth segment (the
Announcement Segment), after Declarations of Wars caused by events, but before
the declarations of War and any Diplomacy on minor countries.
\bparag Formal Agreements should be decided before the Announcement
Segment. Then they are made made loudly in the order of the initiative. As the
Agreements need not be written beforehand, a player could change his mind just
when doing announcements: this is allowed but no negotiation can take place at
this time.
\bparag
The simple public announcement of the agreement suffices to validate it. This
public agreement bears treaty value.
\bparag A formal agreement can be written down during the phase of
negotiations. If this is the case and one player refuses to make the
announcements, his power loses {\bf 1} \STAB.
\bparag Formal agreements can be kept secret: they have value only if written
down and signed by all allies. They can be used later, but with reduced value.
% (Jym).  Informal agreement were for small money transfer. Seems to be
% completely useless since we now allow loan treaty with as few money as one
% wants. Removing.

% \aparag[Informal Agreements]
% An informal agreement allows all types of actions and understandings that
% does not require specifically a formal alliance (ex. right of passing
% through player's provinces, of supply through). The breach to such an
% agreement entails no penalty for the player in cause.
% \bparag Are explicitly forbidden in informal agreement: change of ownership
% of provinces, \COL or \TP, transfers of diplomatic control, of troops.
% \bparag[Informal Money Transfer]\label{chDiplo:Informal Money Transfer}
% Money can be transfered freely by informal agreements, only up to 25 \ducats
% lent to of from another Power, and up to 50 \ducats lent to or from other
% Powers.
% \bparag Diplomatic supports is another kind of Informal Agreement that is
% discussed in \ruleref{chDiplo:Diplomatic Support}, which can creates money
% transfers up to 30 \ducats.

\aparag There exist several types of Announcements: Alliances of different
kinds, each corresponding to a precise agreement, and Trade Refusal. The type
of alliance must always be publicly announced to all other players, or kept
secret and written down.


\subsubsection{Alliances}
\aparag Only players possessing a determined alliance can co-operate in the
various domains considered hereafter. Alliances are of 4 different levels:
\bparag Dynastic Ties
\bparag Loan Treaty
\bparag Defensive Alliance
\bparag Offensive Alliance

\aparag[Generalities]
Alliances are concluded between two or more players. A player can conclude as
many different alliances as he desires with the same player, and/or with
different players, with the restrictions given for each type of alliance as
described hereafter.
\bparag A Formal Agreement (except Loan Treaty) is valid for this turn, the
two following ones, and the very beginning of the next turn, until the
beginning of the segment of Announcements (at which point the Formal Agreement
that ends could be signed again).
\bparag Secret agreements must specify the type of alliance, the powers
involved, the first turn of the alliance, or would be void. They last 3 turns
(like announced alliances). Dynastic Ties are always public and can not be
kept secret (secret Dynastic Ties are void).

\aparag[Dynastic Ties]\label{chDiplo:Alliance:Dynastic Alliance}
A pair of players may conclude a marriage between the ruling families of their
realms, so as to create family ties. They can no longer declare war on each
other without Casus Belli (\CB). This alliance lasts for the whole duration of
the next 2 consecutive turns, except when specific events occurs, forcing its
cancellation.

\bparag To conclude this marriage, one of the two players has to offer a dowry
to the other. The dowry has to be 100 \ducats (minimum, more can be offered up
to the gross income from previous turn of the Power), or consists of one
single province, \COL or \TP, immediately ceded to the other, receiving, party
upon conclusion of the agreement, at the end of the Diplomatic Phase. Note
that the province is still owned by its former controller for the following
segment of Declaration of Wars, the transfer would be latter, at the end of
the phase.
\bparag Money transferred is recorded on \lignebudget{Gifts and loans between
  players}.
\bparag The ceded province, Colony or Trading Post must be owned and
controlled by the ceding player, i.e. it is not possible to cede any territory
in revolt or occupied by another player at the time of the dynastic treaty.
\bparag
The two players are authorised in addition to exchange an extra province, \COL
or \TP. This exchange may be made in addition to the dowry (e.g., exchange of
one province + dowry of a province/or 100+ \ducats), but it is not compulsory
and may never involve national provinces. The previous condition on control
holds.
\bparag
The dynastic alliance can be cancelled at any given time. The party that
cancels it loses 2 \STAB levels.
\bparag
Only a dynastic alliance allows players to cede or exchange a province, \COL
or \TP. Each ceded possession has to be specified at the time of the alliance
conclusion.
\bparag Each ceding of a province, \COL or \TP, costs 1 level of \STAB to the
ceding party.

\bparag[War of Successions.]\label{chDiplo:succession}
The player that pays the dowry can benefit from a War of Succession inside the
other player's country, if a dynastic Crisis occurred in the country that
received the dowry. After Dynastic Ties are established, the rights in case of
War of Succession are valid for 8 turns. When a dynastic Crisis happens, the
power is allowed to declare war on that country as if he had a \CB, or on the
contrary he is allowed to enter as an ally of that same country, as if he had
a defensive alliance with it. See~\ruleref{chSpecific:War of Succession} about
the conditions of this war.

\bparag A dynastic alliance cannot be renewed with the same player less than 3
complete turns after the official end (i.e. be it after two turns or earlier
be-cause it was previously broken) of the previous alliance.

\bparag A dynastic alliance cannot be formed with a player of a different
religion unless a 2 \STAB level loss is incurred for doing so. This applies
until the end of \terme{Religious Enmities} between Protestant, Catholic and
Orthodox countries. It always applies between all Christians and Muslims.
\bparag%[Historical Option]
No Dynastic alliance can be formed by \TUR with any other player.

\aparag[Loan Treaty]\label{chDiplo:Alliance:Loan Treaty}
Only players that have agreed on a Loan treaty can lend money from one to the
other. One is referred to as the "lender", the other as the "borrower".
\bparag The sole possibilities for a player to give money to another are by
Dynastic Ties (as a dowry), by Peace Resolution or by a Loan Treaty.%  Small
% amounts of money can also be transferred 
% when buying a Diplomatic support.
\bparag Money transferred by loan treaty is recorded on \lignebudget{Gifts and
  loans between players}.
\aparag[Restrictions on loans]
\bparag Powers having different religions and signing Loan Treaty lose 1 \STAB
if they transfer 50 \ducats or more to the same borrower in one turn.
% (Jym) religion != standing, removing:
%  Catholic, be they conciliatory or counter-reform, share the same
% religion for this rule.
\bparag[Exceptions.] \FRA, if Catholic/Conciliatory, and \ENG beginning with
Period IV, may lend money to any \MAJ with no penalty for Religion. \HOL,
after being recognised by \SPA (see \ref{pIII:Dutch Revolt}), may also lend
money to any \MAJ with no penalty for Religion.
% \bparag To be valid, the moneylender has to transfer at least 30 \ducats to
% the borrower on the turn the treaty is concluded.
\bparag The lender can not give more than 150 \ducats per turn to a given
borrower. %at the diplomatic phase;
\bparag[Exception.] \HOL or \ENG if it has created its Stock Exchange
(\ref{pIII:Amsterdam Stock Exchange} and \ref{pIV:London Stock Exchange}) can
transfer up to 250 \ducats per Loan.
% PB:Loans during turns are not needed with v2 comptability
% \bparag
% The moneylender may transfer additional funds during the Military Phase (at
% end of round), up to 50 \ducats (or 100 \ducats for \HOL or \ENG after
% events \eventref{pIII:Amsterdam Stock Exchange} and \eventref{pIV:London
% Stock Exchange}).
\bparag Restriction: during one turn, the lender is forbidden to lend more
than his gross income when adding all the transfers made.
\bparag A given Power can not be both borrower and moneylender in different
Loan Treaties at the same time.
\aparag[Modalities of refunding]
\bparag Modes of pay-back and interest are left to the discretion of
players. The "loan" can be even a gift without refund.
\bparag The treaty remains valid as long as the borrower has not paid back all
received ducats. Other loans can be concluded on following turns, but always
in the same way (moneylender to borrower). No new, additional, loan treaty can
be concluded between these two players as long as that one remains valid.
Loans that are gifts end at the end of turn.
\bparag The borrower can break the treaty at any time, and refuse to pay
interest and/or the capital owed to the lender. In such a case, he loses
immediately {\bf 1} \STAB level and receives a negative modifier
% of \bonus{-2} on the Loans table until the end of the period currently
% played.  (Jym) to compta v2. Same as bankrupcy.
of \bonus{-1} on the Exchequer test during 5 turns.
\bparag The moneylender may freely abandon the Loan and transform it in a gift
at any Declaration Phase of a following turn. This ends the Treaty.
\bparag If an event releases a \CB between the moneylender and the borrower
and that the war is declared between them, the treaty is immediately broken
without penalty. In such a case, no back payment or reimbursement is to be
made by the borrowing country.

\aparag[Defensive Alliance]\label{chDiplo:Alliance:Defensive Alliance}
A player linked to another player by a defensive alliance may has to declare
war on any other country that attacks his co-signer. He benefits from a \CB
for this specific declaration.

\bparag The Alliance is effective to be used on the turn of its contracting.

\bparag The player can either enters the war by its own will or if the
co-signer ask him to honour the alliance.

\bparag If the player is called by his ally and refuses to declare war along
with his co-signer, he immediately loses {\bf 2} \STAB levels and the alliance
is cancelled. The co-signer also receives a temporary \CB against the
defaulting player.

\bparag The co-signer player may also prefer not to call for his Ally (or
Allies). In this case, the allied player is left free to declare his
participation in this war (with a \CB) or not. If the Ally chooses not to
participate, he suffers no penalties and the Alliance is not considered as
broken.

\bparag If a secret alliance is called for and the co-signer refuses to
declare the war in response of this alliance, the loss is reduced to {\bf 1}
\STAB instead of 2. The betrayed power still has a temporary \CB against the
defaulting power.

\bparag This Alliance lasts for the duration of the next 2 turns, except when
and if cancelled by events or voluntary cancellation by one (or even both) of
the co-signer.

\bparag All declarations of war by this way cost only 1 \STAB level (whatever
number of declarations in the current turn).

\bparag When the players are forming an "Alliance", they have to sign together
the same peace with their enemies. With a minor country: count all the
modifiers enemy minor/all allies. If peace is accepted, the allies must share
the gains. With a major country: as for minor country, except for allies make
an average of their Stabilities (rounded down).

\bparag If at war against the same enemy, all allied players move and play
together (at the lowest player's Initiative rating rank).

\bparag If an ally is twice at {\bf -3} \STAB at the phase of Peace, he must
sue for peace, and sign a separate peace. In this case, his Alliance is not
considered as voluntarily broken (and there is no \CB).

\aparag[Offensive Alliance]\label{chDiplo:Alliance:Offensive Alliance}
Same as for the preceding type of alliance, except that it applies also in the
case where the co-signer is at the origin of the war declaration on another,
third-party player or minor.
\bparag The details are the same as for a Defensive Alliance.


\subsubsection{The Trade Refusal}
\aparag A player can refuse the access of his market to the foreign trade of
another player, even in peace, but that costs him 1 \STAB level at the moment
he announces his decision. Once taken, the decision can be maintained from one
turn to the other (without any additional decline in Stability); the decision
can be repelled later by the power at no cost.

\aparag[Reaction of the Other Player]
\bparag The other player whose trade has been denied then receives a temporary
Commercial \CB against the player refusing him trade. This \CB is to be used
in the segment of Declarations of Wars and starts a new war.
\bparag Alternatively, he may refuse his own trade in reaction and
reprisal. He then suffers from the same effects (loss of 1 level of \STAB, and
Commercial \CB to the enemy). This is to be announced immediately.

\aparag[Value of Trade Refused]

\bparag When a player is refused the trade of another, a foreign trade loss is
assigned: it is calculated on the basis of the refusing player's European
Trade value, i.e. the income of the refusing player's provinces, including
vassals. This value is added to the amount of the European Market that is
denied as foreign trade.

\bparag The player being refused trade gains also no income from Commercial
fleets in the own \CTZ of the refusing \MAJ (neither regular nor monopoly
incomes).

\bparag The player being refused trade gains half of its usual income (trade
plus monopoly income if he has the monopoly) in some \STZ, depending of the
\MAJ that refuses Trade:
\begin{itemize}
\item \TUR: \stz{Caspienne}, \stz{Noire}, \stz{Ionienne};
\item \VEN: \stz{Lion}, \stz{Noire}, \stz{Ionienne};
\item \POR: \stz{Canaries}, \ctz{Espagne};
\item \POL and \SUE: \stz{Baltique}.
\item \HOL and \ENG: \stz{Nord}.
\end{itemize}

\bparag These \TradeFLEET still count toward ownership of trade centres and
the income of trade centres is not affected.
% (Jym). Otherwise there could be an easy way to get the CC Med for TUR.

\aparag A Trade Refusal breaks any past Loan Treaty between the two Powers
with no further penalties. It forbids any Loan Treaty as long as the Trade
Refusal continues.


\subsubsection{Others Announcements}
\aparag Others announcements can be made during the Diplomatic phase.
\bparag Most of them come from events specifying that a given choice must be
made ``as a diplomatic announcement''.
\bparag Speculation on exotic resources is made as a diplomatic
announcement. See~\ref{chAdministration:Speculation} for the effect.
\bparag Trade of wood is decided as a diplomatic
announcement. See~\ref{chIncomes:Wood} for the effect.

% Local Variables:
% fill-column: 78
% coding: utf-8-unix
% mode-require-final-newline: t
% mode: flyspell
% ispell-local-dictionary: "british"
% End:


% -*- mode: LaTeX; -*-

\section{Diplomacy with European Minor Powers}

% RaW: [33]



\subsection{Presentation}
\subsubsection{Actions and control}
\aparag[Informal Overview]
After having negotiated between them, players may "negotiate" with minor
countries. Each player has 1 to 6 diplomatic actions per turn. This number is
given for each country and each period, as per the Limits table located on
Players' Aides.  Each diplomatic attempt against one minor country uses 1 such
action and an investment in ducats which can be basic, medium or strong.
Actions and diplomatic expenses have to be written on \lignebudget{Diplomatic
  actions}.
Results of those actions are assessed: each is solved with the help of three
dice. In case of success, the influence that the player exerts on the minor is
adjusted.  Each minor country that is influenceable by the diplomacy of
players has a diplomatic status marker displaying the relevant indications for
the diplomatic game.  Each such counter is placed on the diplomatic track
located on the Rest-of-the-world map. Such a counter must be found in
permanence placed in a square corresponding to its attitude towards a player
or in the square reserved to the neutrals.

\aparag[Levels of Diplomatic control]
The principle of the diplomacy with European minor countries is that there can
be only one influence of any one single player on a given minor, meaning that
this player has a preponderant influence, or diplomatic control of the minor
country; he is also names the "Patron" of the minor country. This influence is
divided into different levels of increasing importance, which are:
\begin{deflist}
  \listingabbrev{Neutral}{Neutral (not really a status, rather the fact of
    being independent)} \listingabbrev{RM}{Royal Marriage (dynastic ties unite
    the reigning families of the two countries)} \listingabbrev{SUB}{Subsidies
    (the countries share economic ties and have mutual debts)}
  \listingabbrev{MA}{Military Alliance (the two countries have concluded
    military alliances and may help each other during wars)}
  \listingabbrev{EC}{Expeditionary Corps (the minor country is susceptible of
    sending larger armed forces)} \listingabbrev{EW}{Entry in war (the minor
    country may be called for a full participation in a war)}
  \listingabbrev{VASSAL}{Vassal (the minor country is effectively dependant on
    the authority of the major country, and will participate in wars)}
  \listingabbrev{ANNEXION}{Annexation (the minor country has really become
    part of the major country in some form, and counts for many things as
    such)}
\end{deflist}

\aparag[Limit]
This influence may be limited sometimes to a maximum level for some specific
minor countries or for some particular players. It is even possible that a
player could not make diplomatic action against a particular minor (e.g. the
Turkish player against \paysPerse).

\aparag[The Diplomatic Track]
Each player has a line of his own on the diplomatic track situated on the
Rest-of-the-world map. Columns indicate the different diplomatic status that
the player can achieve on a minor, as described immediately above.

\aparag[Diplomatic Counters]
Each counter (front/back) regroups information concerning the minor country
mentioned on that counter.  All this information also figures in the Annexes
dealing with minor countries.

\subsubsection{Other}
\aparag It is possible to give a province to a minor country if either this is
a province formerly owned by it (at any point during the game) or it has a
blurred shield of the minor.
\bparag This is not an action, this does not count toward the limit of actions
per turn.

\aparag If the minor is not existing anymore, it is immediately recreated as a
\VASSAL of the major giving a province.
\bparag If the minor cannot be \VASSAL, put it on the highest possible
diplomatic level allowed for it instead.

\subsection{Diplomatic actions}\label{chDiplo:Diplomatic Actions}


\subsubsection{Principles of diplomatic actions}
\aparag A player has a number of diplomatic actions which is limited according
to the period in play (from 1 to 6 actions per turn). Even though the
Diplomatic Actions are resolved after the Declarations of Wars, the rules are
explained here (because Diplomatic control is helpful to understand the wars).
\bparag The action is aimed at increasing the level of control of the player
on that minor country, or decreasing the level of control of another power on
a minor country.
\bparag The player registers on his monarch sheet all his diplomatic actions
of the current turn, by specifying which minor countries are aimed at. He must
pay the cost of each action (written on \lignebudget{Diplomatic actions}) and
indicate on his monarch sheet the level of investment placed in that action
(either basic, medium or strong).
\bparag[Diplomatic Supports]\label{chDiplo:Diplomatic Support}
The player can also declare that he is supporting one action of another
player. This support is a diplomatic action of the player by itself (it has to
be paid as a basic investment diplomatic action), and must be written on the
supporting player monarch sheet.
\bparag Supports can be discussed and established as an informal agreement
between the player granting support and the one receiving it.
% \bparag In all cases, supports can never be sold to another player more than
% 30 \ducats each.
\bparag ``Selling'' supports is possible by contracting a loan treaty at the
same time, but remember the limits on loan treaties.

\aparag[Writing actions] When deciding which actions to make, a player should
write all of them in details on his monarch sheet: the turn at which the
action occurs, the country targeted, the amount of money spent (investment)
and the resulting bonus to die roll (as explained below). Writing all this
before actually resolving any of the actions will greatly speed up and
smoothen play.

\aparag A player can make only one action on a given minor country per turn.

\aparag No diplomatic action is allowed on a European minor country that is
fully involved in any war (even a Civil War) even by a major country that is
not part of the war. The only "diplomatic" action allowed on minors at war
with the player is separate peace. There is no such restriction for minors in
limited interventions.

\aparag{Cost of Diplomatic actions}
The costs are the following:
\bparag Basic investment: 20 \ducats
\bparag Medium Investment (+2 to the die-roll): 50 \ducats
\bparag Strong investment (+5 to the die-roll): l00 \ducats
\bparag Support (+1 to the die-roll): 20 \ducats

\aparag Actions must have been written down to be considered as valid.


\subsubsection{Resolution of an action}
\aparag[Order of Resolution] Intended actions are first written down by all
players, then they are announced and then solved, minor by minor, the order of
which being of no importance (choice of minor according to the initiative if
contentions between players), in the following order:
\bparag players decide of their reactions;
\bparag resolve opposed actions (on minor countries already controlled by a
power, or if two powers aim at the same Minor);
\bparag resolve remaining unopposed actions.
\bparag Note that all actions should be announced first, then all reaction
should be decided and only after should the action be resolved. If you start
resolving your actions earlier, don't complain that your opponent bases his
ones actions or reactions on the results of your actions.

\aparag[Reactions by Another Player on a Minor it controls]
\bparag When an action is made on a minor already on the track of a player,
this power may react depending on whether it was also making an action to
increase his own level of control, or not.
\bparag[If the Patron is doing an action] There is no "reaction" investment to
be paid by the controlling player excepts that the player may decide to
immediately raise his level of investment and pay the difference. This level
of investment is paid for his own action and the action will be considered at
the same time as a the "reaction".
\bparag[If the controlling player did not plan to make any action on that
minor] He is then allowed to take a "reaction" on that minor by paying the
investment required. This reaction is in addition to the actions he is
normally entitled for the current turn.
\bparag[If the controlling (i.e. defending) power refuses to make any
reaction] by not paying any investment in reaction, the minor country is
immediately placed in \Neutral position and defends itself according to his
new \Neutral stance.
\bparag Note that the defending player benefits from a bonus applicable to the
die roll according to the degree of control that he exerts on the minor.  This
bonus is reminded to the player's attention at the top of the Diplomatic track
on the map.
\bparag Money spent for reactions (if any) is recorded on
\lignebudget{Diplomatic reactions}.

\aparag[Resolution of Opposed Actions]
If several powers are doing actions (including reaction) on the same minor,
these actions are resolved together at the same time (each player rolls his
die-roll and modifies it). The player that obtained the best result (i.e. the
highest modified result) is selected to proceed further.
\bparag Solve ties by competitive unmodified die-rolls, but the original
result will be used for the resolution.
\bparag If a reaction (that was not originally an action) is the best result,
do not proceed further (no progression point can be gained, the reaction only
served to keep the minor).

\aparag[Resolution of the Action]
The power selected with the best result compares its result to the following
score:
\bparag the score in reaction (even if it was originally a normal action) of
the controlling power if it was opposing the attempt and did not achieve the
best result (only the controlling players can use his score here, not another
player attempting an action on the same minor);
\bparag otherwise, the sum of 2d10 in all other cases.
\bparag The player earns a number of progression points equal to the
difference between his (modified) die roll and this latter result.
\bparag If the difference is null or negative, it does get any points of
progression (there is no "negative" progression).

\aparag[Modifiers] Any player that rolls for this Minor Diplomacy has his
die-roll modified as follows: \diplomod
\aparag[How to read the Diplomatic Values of each Minor]
Each Status (i.e. box) on the diplomatic track has a variable cost of
progression, according to the level of control (status name is printed at the
top of the track) and the concerned minor country.
\bparag Political status \Neutral, \RM and \SUB cost always 1 point of
progression. Exception: to enter the \SUB box for \paysSuisse costs 3 points.
\bparag The cost is variable for the other status according to minor
countries. It is indicated on their diplomatic marker, as well as in the list
of minor countries located in the Appendix handbook
\bparag If a \textetoile figures on the diplomatic marker for a particular
status, it indicates that this political status is not achievable with this
minor country.
\bparag If initials appear instead of a figure (cost), they indicate that only
the country having these initials can reach this political status, under the
restriction that a specific event allowing it has occurred.

\aparag[Diplomatic Markers Adjustment]
Costs of progression indicate the minimum number of points of progression to
advance the counter of the concerned minor on the diplomatic track.
\bparag When all diplomatic actions have taken place, the minor country
diplomatic marker is moved according to the number of points of progression
obtained for that minor and the costs to enter the various status boxes, in
favour of the player having obtained the success on this minor.
\bparag Advancing a diplomatic counter is never mandatory. A player may always
stop the marker progression even if sufficient progression points remain.
\bparag Moving back a marker is mandatory. If the marker reaches the \Neutral
box while doing so and some remaining points of progression are still
available, the marker can then progress in favour of the player that has
succeeded in the action as explained below.
\bparag All points of progression balance that do not suffice to enter into
the box is lost and not applicable.
\bparag The diplomatic marker of a minor country is moved on the track until
it reaches a political status box, as allowed by the number of points of
progression and the various costs to enter those boxes.  If the marker has
progressed, intermediate boxes indications are ignored. Apply only the result
and benefits of the status corresponding to the box where the marker is
located.

\aparag[Handling reactions]
When an action is opposed by a reaction (or in case of a competitive action
lost by the controller), the score need to be compared both to the reaction
score and later to 2d10 (as per regular minor).
\bparag Comparing the action score with the reaction gives a number of
progressions points used to reach \Neutral.
\bparag Once the minor is \Neutral, roll 2d10 for it. Compare the (original)
score of the action with them to get a number of progression points, then
subtract the number of points previously used to reach \Neutral. The result
(if positive) is the number of progression points used to raise the

\begin{exemple}[A simple action]
  At turn 1, \FRA tries to do some diplomacy on \paysSavoie which is already
  in \MA, the French monarch is \monarque{Charles VIII} with a \DIP of 9 and
  he chooses to make a basic investment only. Both \FRA and \paysSavoie share
  the same religion (Catholicism). Thus, the total modifier for \FRA is
  \bonus{+12} (\bonus{+9} for \DIP, \bonus{+1} for religion and \bonus{+2} for
  control).

  \FRA rolls a 3, for a net result of 15. Someone else rolls 2d10 for
  \paysSavoie and gets 6 and 5 for a result of 11. \FRA thus scores 4
  progression points. \paysSavoie is already in \MA, the next box is
  \EC. According to the diplomatic value (in the Appendix), it costs 2 points
  to raise \paysSavoie to \EC. There are still 2 points left. However, raising
  \paysSavoie to \EW would cost 3 extra progression points which \FRA doesn't
  have. So, \paysSavoie stops in \EC and the 2 extra progression points are
  lost.
\end{exemple}

\begin{exemple}[A competitive action]
  At turn 1, both \ANG and \HIS want to make an action on \paysPalatinat
  (which is \Neutral). The three countries are Catholic (\paysPalatinat will
  become Protestant later but it begins Catholic). Both \ANG and \HIS choose
  to make a basic action, their respective \DIP is 7 and 6, thus giving
  modifiers of \bonus{+8} for \ANG and \bonus{+7} for \HIS.

  \ANG rolls 4 for a final result of 12 while \HIS rolls 7 for a final result
  of 14. Thus, only \HIS is allowed to do an action. Someone rolls two dice
  for \paysPalatinat and gets 4 and 9 for a total of 13, to the amusement of
  \ANG. \HIS thus only scores 1 progression point, enough to get
  \paysPalatinat in \RM but no further.
\end{exemple}

\begin{exemple}[A reaction]
  At turn 1, \FRA also wants to try and get \paysPapaute out of Spanish
  hands. Thus, he makes his second diplomatic action on it, still with a basic
  investment resulting in a \bonus{+10} modifier.

  \HIS did not plan any action on \paysPapaute and shocks when he learns about
  the French villainous move, claiming that he is the most Catholic king out
  there and should morally be the only one with ties to the Pope (after all,
  the soon to be elected Alexander VI is Spanish\ldots) \FRA smiles and calmly
  asks if \HIS wants to react to this action or forfeit his illegitimate
  claims on Rome (adding that bringing the Papacy back in Avignon looks like a
  promising idea).

  If \HIS chooses not to react, then \paysPapaute will immediately becomes
  \Neutral and the French action is then resolved normally. However, \HIS
  wants to keep his lead on \paysPapaute and thus chooses to react. It has to
  decide at which investment. Since its \DIP is only 6, a basic investment
  will yield in a \bonus{+8} modifier (\bonus{+1} for religion and \bonus{+1}
  for control), somewhat smaller than the French \bonus{+10}. So, \HIS decides
  to limit the risks and use a medium investment, thus spending an extra
  50\ducats but reaching a \bonus{+10} modifier.

  Both roll a die. \FRA rolls 7 for a total of 17 while \HIS only rolls 1 for
  a total of 11. Thus, \FRA gets 6 progression points. The first one is used
  to bring \paysPapaute back to \Neutral. Then, the rest of the action is
  resolved as against a normal \Neutral (and the extra progression points
  against \HIS are lost). \HIS swear to take his revenge and quickly grab two
  dice, rolling 6 and 8 for a total of 14. \FRA initial total was 17, so he
  has 3 progression points against \paysPapaute, however, one is considered to
  have been already used against \HIS, so there are 2 left, just enough to
  bring \paysPapaute in \SUB.

  \smallskip

  Even if \HIS had initially rolled 9, for a total of 19, higher than \FRA, he
  could not have raised his control on \paysPapaute because this was a
  reaction and not a planed action.
\end{exemple}

\begin{playtip}
  It is more efficient to have all the players simultaneously write down all
  the diplomatic actions they want to do this turn, including the computation
  of the bonus ; then have pair of players (as soon as they are finished) roll
  for their actions (with the other rolling for the minor) and write done the
  result (number of progression points) ; and lastly implement the results
  (going to the diplomatic track and moving the markers, maybe rolling for
  subsidies or dowries.

  This avoids numerous back and forth journeys to the diplomatic track to
  implement the results and speeds the rolling process by pre-computing
  everything (thus requiring less time overall).

  Note also that the influence of the diplomatic actions of other players on
  the immediate other phases (incomes and expenses) is almost null. So, as
  soon as one has resolved ones diplomatic actions, one can begin computing
  ones incomes and thinking about expenses. Only the military phase will
  require further synchronisation between players.
\end{playtip}

\aparag[Reading markers] The cost for entering the different boxes is
specified in the Appendix. Additionally, it is written on the diplomatic
counters for easy reference during game. The front of the counter shows values
for dowry, subsidies and \MA while the back (with the ``at war'' strip) shows
values for \EC, \EW, \VASSAL and \ANNEXION.



\subsection{Effects of the Diplomatic control}


\subsubsection{Royal Marriage}
\aparag The Royal Marriage (\MR) box gives the advantage of a bonus of +1
during any ulterior diplomatic phase as long as the player controlling the
minor country retains this status.

\aparag[The Dowry] When the minor country diplomatic marker reaches the \MR
box by advancing (not by moving back), the player rolls one die. If the result
is:
\bparag Even the player receives the sum of the dowry in ducats as indicated
on the diplomatic marker.
\bparag Odd: the player has to pay the dowry.
\aparag This sum (positive or negative) is written on \lignebudget{Subsidies
  and dowries}.

\aparag if the player refuses to pay the dowry, the marker is returned
immediately to the \Neutral box.


\subsubsection{Subsidies}
\aparag The position of a diplomatic marker on the Subsidy (\SUB) box gives a
bonus of +l during any ulterior diplomatic phase for the player controlling
the minor country.

\aparag[Payment of Subsidies.]
When the minor country diplomatic marker reaches the \SUB box by advancing
(not by moving back), the player rolls a 1d100 He modifies the obtained
die-roll result by the Subsidy modifier (always negative) indicated on the
minor country marker.  If the result is:
\bparag positive: it indicates the number of ducats that the player receives
from the minor;
\bparag negative: it indicates the number of ducats that the player has to pay
to the minor.
\aparag This sum (positive or negative) is written on \lignebudget{Subsidies
  and dowries}.
\aparag If the player refuses to pay, the marker is immediately and directly
returned to the \Neutral box.

\aparag The positive net amount obtained by Subsidies can never exceed 50
\ducats, except explicit precision of the contrary as explained in some
events.

\aparag When a players pays the subsidies, the ducats thus transferred to the
minor are deducted from the player treasury (and just marked-off i.e. there is
no such thing as "minor country treasury").

% \aparag Diplomatic "income" (i.e. positive subsidy) is credited to the
% player TR at the end of the Diplomatic phase.


\subsubsection{Military Alliance}
\aparag The position on the Military Alliance (\AM) box gives a bonus of +2
during any ulterior diplomatic phase for the player controlling the minor
country.

\aparag[Alliance.] As it is an alliance, if the \MIN is declared war upon or
if it declares war, it will call for its patron, that is also an ally.

\aparag[Limited Intervention in wars.]
Conversely, the \MIN is allowed to be involved in a limited way in the wars of
their patron. This declaration is a Reaction, and is shown by placing the
forces of the \MIN on the map. Additionally and as an exception to the rules
of reaction, a limited intervention can be declared at the instant a status of
\AM (or better) is obtained, so at the end of the phase of Diplomacy (and not
at the usual segment where reactions are allowed).
\bparag A limited intervention of a minor country is made only with its basic
forces. It can draw supply only from its own provinces (and so can not go
further than 12 MP from its country).
\bparag Units can not go out of the European map if the minor country has no
\TP/\COL on the \ROTW map. They can not participate in discoveries if it is
not specified for this minor power (mainly \pays{Portugal} and \pays{Hollande}
are allowed).
\bparag In \AM, the intervention is at most of one land stack and one naval
stack outside the minor country.
\bparag The \MIN receives reinforcements each turn in the administrative
phase. The base reinforcement is given in the Appendix. These reinforcements
are only used to recreate the basic force of the \MIN, should they be
diminished.
\bparag The \MIN has a free active campaign each turn, and free passive
campaign each other round. Its Patron may increase the level of the campaign
by paying for this.
\bparag The \MIN is in fact out of the war: its territories can not be
attacked or trespassed if it is only in limited intervention.  The \MIN is not
part of the Peace Treaty that will end the war. The \MIN may withdraw from the
war if its diplomatic status changes.
\bparag A \MIN that is announced in limited intervention in a war offers a
free \CB to the enemy alliance to involve fully the \MIN in the war.

\aparag[Full involvement in wars.] Some events, or declaration of wars may
involve fully the minor country in a war.
\bparag In this case, the status is shown by by putting the Diplomatic marker
of the \MIN on the side reading "At War" and the Diplomatic position is
increased to Entry in War (\EW).


\subsubsection{Expeditionary Corps}
\aparag The position on the Expeditionary Corps (\CE) box gives a bonus of +2
during any ulterior diplomatic phase for the player controlling the minor
country.
\aparag[Alliance.] As it is an alliance, if the \MIN is declared war upon or
if it declares war, it will call for its patron, that is also an ally.
\aparag[Limited Intervention in wars.]
Conversely, the \MIN is allowed to be involved in a limited way in the wars of
their patron. The conditions are the same as in \AM, except that the \MIN in
\CE add one \LD or \ND (controller's choice) to its reinforcements each turn.
\aparag[Full involvement in wars.] Some events or declaration of wars, may
involve fully the minor country in a war.  The conditions are the same as in
\AM.


\subsubsection{Entry in war}\label{chDiplo:EW Effects}
\aparag The position on the Entry in War (\EW) box gives a bonus of +3 during
any ulterior diplomatic phase for the player controlling the minor country.
\aparag[Alliance.] As it is an alliance, if the \MIN is declared war upon or
if it declares war, it will call for its patron, that is also an ally.
\bparag Additionally, the Patron may ask for a full entry in war on the minor
country, as an ally fully involved in the war. This is done during the
announces of Reactions to a declaration of war (as if calling for alliances of
\MAJ).  To participate, a minor must be rolled for and a modified result of 6
or more must be obtained on 1d10.
\bparag
Modifiers to this entry die-roll depend on the country the player wants his
minor to declare war upon. They are the following: \diplowar
\bparag Failure to this test lowers the diplomatic control to \CE immediately,
and forbids the Major power to declare a limited intervention of this Minor
country at the current turn in this war.

\aparag[Limited Intervention in wars.]
Conversely, the \MIN is allowed to be involved in a limited way in the wars of
their patron. The conditions are the same as in \AM, except that the \MIN in
\EG add one \LD or one \ND (controller's choice) to its reinforcements each
turn.

\aparag[Full involvement in wars.]\label{chDiplo:Entry War Minor}
Some events or declaration of wars may involve fully the minor country in a
war.
\bparag In this case, there is no restriction to the manner that the \MIN
conducts the war. The status is shown by putting the Diplomatic marker of the
\MIN on the side reading "At War".
\bparag It maintains up to its Basic Force at the begining of each turn.
Additional forces can be maintained by their Patron.
\bparag It receives reinforcements based on a roll on the Reinforcement Table.
It has, for free, an active campaign for each round, plus some major (or
multiple) campaigns given by the reinforcements table. The Patron may complete
the cost of those to a higher level of activity if need be.
\bparag It will have to sign a Peace Treaty to cease the war (a Separate Peace
or the common Peace Treaty).


\subsubsection{Vassalisation}
\aparag The position on the vassalisation (\VASSAL) box gives a bonus of +3
during any ulterior diplomatic phase for the player controlling the minor
country.

\aparag[Income] Vassal income from provinces, colonies, Trading Posts, exotic
resources and commercial fleets is included in the controlling player's
income, both during war or peacetime.
%\bparag Vassal provinces do not contribute to Manufacture percentage income.
\bparag Their income is added to blocked foreign trade for Foreign trade
income, and count for domestic income.

\aparag The territory of a \VASSAL country is always open to its controlling
power. The allies of this powers and its enemies can pass through the \VASSAL
if (and only if) the Patron has been in it before during the current turn.
\bparag Movements, supply passing through, staying in and battles are
permitted to those countries. The territory is friendly to the controlling
power and its allies, and enemies to others.
\bparag No siege or pillage are possible. The cities are supply sources only
to the Vassal minor country.
\bparag Fortresses may be maintained by the Patron.

\aparag[Alliance.] A \VASSAL is tightly associated to its Patron.
\bparag The controlling power may decide to fully use its \VASSAL in war, or
to declare only a limited intervention, or do nothing (except that the
territory of the \VASSAL is accessible as said above). All those declarations
can be made as reactions at any turn of the war. Once a \VASSAL is fully
involved in a war, it stays so until a Peace is signed.
\bparag The enemies of the Patron can declare during the diplomatic phase that
they fully include any \VASSAL in an existing war: the \VASSAL is now in full
war.  Also, a declaration of war against a \VASSAL is actually a joint
declaration of war against the Patron.
\bparag A \VASSAL can only be involved in war (full or limited way) if the
\VASSAL is at the distance of 12 MP or 4 sea zones from one enemy province of
a Power fully involved in the war.

\aparag[Limited Intervention in wars.] The \VASSAL can be involved in a
limited way in the wars of their patron. The conditions are the same as in
\AM, except that the \MIN in \VASSAL gain no free reinforcement each turn,
including its own basic reinforcements.
\bparag Instead, the Patron may pay for reinforcements, on his own treasury,
to raise troops up to the basic forces of the country.  The maximal
reinforcements so raised are the basic reinforcements indicated in the
Annexes, plus 2 detachments (\LD or \ND).
\bparag All the basic forces of the \MIN can be used.

\aparag[Full involvement in wars.] Some events or declaration of wars may
involve fully the minor country in a war.  Additionally, its Patron or the
enemies of this power may declare at any Diplomatic Phase that the \VASSAL is
now fully involved in the war.
\bparag The conditions are the same as in \EW.
\bparag[Vassals and Separate Peace] A vassal ally never accept to sign a
separate peace unless its capital is under enemy control (and unbesieged by
friendly forces), or it is forced to accept an unconditional peace (when
totally conquered), or its monarch is captured and ransomed for the right to
attempt a separate peace.


\subsubsection{Annexation}
\aparag The position on the Annexation (\ANNEXION) box gives a bonus of +3
during any ulterior diplomatic phase for the player controlling the minor
country.  When the minor country marker is in the AN box of a player, that
country is considered as annexed by the player.
\aparag[Units and Income of Annexed Minors]
All force of an annexed minor are removed, and provinces of that minor are
annexed, although they cannot be considered as national provinces of the
annexing player.
\bparag The player receives all income from annexed provinces as if they were
his own, including for Manufacture percentage income.
\bparag He may build units there as in other non-national provinces.
\bparag For military operations, the annexed country is part of the
controlling power.
\aparag[Condition of Annexation]
To be annexed diplomatically, a minor country has to be adjacent to a province
already controlled by the annexing player, otherwise the diplomatic counter of
this minor cannot be move up to the Annexation box.

\aparag[Dis-annexion]
An diplomatically annexed minor can be dis-annexed if another player succeeds
in moving the diplomatic marker of the \MIN on the diplomatic track, away from
the \ANNEXION box.
\bparag A minor can also be dis-annexed by a Diplomatic Agitation during the
event phase, by a change that could make the marker's present position be
moved one or more boxes.
\bparag Destroyed minor countries (possible by some events, or by rules on
Turkish, Russian or Polish Annexions) are not annexed for this rule: their
diplomatic marker is not put in Annexion and the Diplomatic Agitations do not
affect them.

% Local Variables:
% fill-column: 78
% coding: utf-8-unix
% mode-require-final-newline: t
% mode: flyspell
% ispell-local-dictionary: "british"
% End:

% LocalWords: influenceable minor's


\section{Diplomacy with non-European countries}

\subsection{Diplomacy status in \ROTW} 
\aparag The following minor countries are on \ROTW map.  \paysInca,
\paysAzteque, \paysGujarat, \paysVijayanagar, \paysMogol, \paysChine,
\paysJapon, \paysSiberie, \paysOman, \paysAden, \paysSoudan,
\paysMysore, \paysHyderabad, \paysIroquois, \paysAfghans, \paysOrmus (a
special part of \paysPerse).  The relations between European Major
Powers and those countries are governed by different diplomatic rules.
\aparag[Generalities]
A Major Power has a specific status regarding  each one of those countries:
\begin{deflist}
\listingabbrev{dipNR}{No relation}
\listingabbrev{dipFR}{Formal relation}
\listingabbrev{dipAT}{Alliance Treaty}
\end{deflist}
\bparag \dipNR is not recorded; 
\bparag \dipFR and \dipAT are recorded by placing a \ROTW diplomatic
counter of the Major Power in the diplomatic status box of the relevant
minor country, that is found on the \ROTW map, on the side showing
\dipFR or \dipAT as needed.
\bparag Note that the number of \ROTW Diplomatic counters provided to
each \MAJ is limited by design. A Major Power may always decide to lose
a relation in order to free a needed counter. Each counter allow for one
\dipFR (front) or one \dipAT (back).%  Limits by Major Power: \POR: 6
%counters ; \FRA, \HOL and \ENG : 4 counters ; \SPA, \TUR, \RUS, \SUE,
%\POL and \VEN: 2 counters.
\aparag Diplomatic status is achieved by doing diplomatic actions, as
described in the \ruleref{chDiplo:Diplomatic Actions}. A diplomatic action
on a country in the \ROTW counts as one of the allowed actions, but it
is resolved differently.



\subsection{Diplomatic actions in the ROTW}
\aparag[Conditions to attempt actions.]
In order to attempt a diplomatic action on a \ROTW minor country,
a Major Power needs to have discovered at least one province of the
minor country, and needs to
\bparag either have a \TP/\COL in an area owned by, or adjacent to the
country, or adjacent to the same seazone,
\bparag or have a Commercial fleet in a seazone bordering that country,
\bparag or have an emissary in the minor country at the diplomatic
phase,
\bparag or be \TUR attempting action on \pays{Oman}, \pays{Aden} or
\pays{Soudan},
\bparag or be \VEN after \subeventref{pI:WRS:War Indian Sea}, attempting
action on \pays{Aden}, \pays{Oman} and \pays{Gujerat}
\bparag No diplomatic action is allowed if the power is fully at war
against the minor country of the \ROTW.

\aparag[Emissaries]
An emissary is a Conquistador (or an Explorator used as a C, with values
divided by 2), a Governor, a Missionary, or a Mission. To be helpful, an
emissary has to be in the target minor country, or in an adjacent
region, or in a province bordering the same (discovered) sea as the
minor country.

\aparag[Resolution of diplomatic actions in \ROTW]
The result of the action is always given by the difference between 1d10
rolled by the \MAJ (plus bonuses below) and the resistance given by the
sum of 2d10.
\bparag as for actions on European minors, the actions (and final bonus)
has to be written on the monarch sheet and the cost is recorded on
\lignebudget{Diplomatic actions}.
\diplorotw

\bparag An adjusted roll strictly higher than the resistance (2d10) plus
one raises the diplomatic status of one level (from \dipNR to \dipFR, or
from \dipFR to \dipAT), or of two if the difference is 5 or higher (all
the way to \dipAT).
\bparag Going to a higher level of relations is always voluntary and can
be declined.
\bparag More than one power can make a diplomatic action on a country in
the \ROTW at the same time. The attempts are not in opposition. Several
major countries may have \dipFR or \dipAT with the same minor at the
same time.
\bparag An adjusted roll less or equal to the resistance causes nothing,
except that a Missionary that served as an Emissary is killed (and may
come back afterwards).

\aparag[Reaction]
Any \MAJ sharing an \dipAT with the \MIN has the opportunity to react.
It uses the same condition and modifiers as diplomatic action in \ROTW.
As a reaction, the \MAJ pays the action (according to the investment),
this is recorded in \lignebudget{Diplomatic reactions}, but the action
is not counted as one of its own at this turn. If the roll of the
reacting player is higher than the resistance (sum of 2d10), the result
of the action is given by the comparison with his roll.
% Jym: useless because one needs an AT to react already...
%The
%reacting player may not augment his own status as a result of the
%reaction.

\aparag[Opposing to other countries' relations.]
A Diplomatic action may be aimed at diminishing the diplomatic relations
of some or all Major Powers with the minor. This counts as one
diplomatic attempt and is allowed provided the power satisfies the
conditions to make diplomacy on this minor country.  The opposed \MAJ(s)
is/are announced before the action and they defend their status as
usual, by paying the cost of a Diplomatic action (that is not counted as
one of their permitted actions for the turn), if they have no action
planned.
\bparag Both opposing \MAJ make a roll of 1d10, modified as above.  If
the acting \MAJ obtains a higher roll than an other \MAJ opposing the
action, the result is that this \MAJ lowers its diplomatic status of one
level (from \dipAT to \dipFR, from \dipFR to \dipNR).

\begin{exemple}[Diplomatic action]
  During turn 2, \leader{Da Gama} lands in India and stays inside the
  territory of \paysVijayanagar at the end of turn. Thus, he may act as
  an emissary during the diplomatic phase of turn 3. The special \FTI
  for \ROTW of \POR is 5 and the player chooses to make a small
  investment only. Thus, the final bonus is \bonus{+11} (\bonus{+6} for
  the Manoeuvre and \bonus{+5} for the \FTI) which is already rather
  good\ldots

  \POR rolls 6, for a total of 17 while the minor rolls 4 and 8 for a
  total of 12. The difference between the two is 5 which is enough to go
  directly to \dipAT. \POR now has to pay 2d10\ducats as presents to the
  local Rajahs (see below).
\end{exemple}

\begin{exemple}[Diplomatic reaction]
  At turn 8, \TUR manages to get an \dipAT with \paysAden, allowing it
  to get part of the spice trade. Since \ref{pI:WRS:War Indian Sea}
  occurred earlier, \VEN, always eager to get more hold on the spice
  trade, attempts some diplomacy on \paysAden and \TUR decides to
  react. None of them has emissary in the country. The \FTI are 4 for
  \VEN and 3 for \TUR. \VEN chooses to make a medium investment for a
  final bonus of \bonus{+4} (\FTI, \bonus{+2} for the investment but
  \bonus{-2} for the religious difference) while \TUR only reacts with a
  small investment for a final bonus of \bonus{+5} (\FTI, \bonus{+2} for
  being both Muslims).

  \VEN rolls 8, for a total of 12. \paysAden rolls 3 and 2 for a total
  on 5. If the action was not opposed, this would be enough to get an
  \dipAT! However, \TUR rolls 5, for a total of 10. Thus, the Turkish
  roll is taken into account rather than the minor one and \VEN only
  gets a difference of 2. Still enough to go to \dipFR.
\end{exemple}

\begin{exemple}[Hampering another status]
  It is turn 53 (1750). Both \FRA and \ANG have an \dipAT with
  \paysMysore. Sensing that colonial tensions may arise in a state of
  war sooner or later, the East Indian Company decides to play on the
  intra-indian struggles and sends \leaderClive in a attempt to convince
  \paysMysore to break its alliance with \FRA. The \emph{Compagnie des
    Indes Orientales} learns about it and quickly sends \leaderDupleix
  to try and counter the English deeds.

  \ANG makes an action on \paysMysore, specifically to lower the
  relation with \FRA, with a \FTI of 5, a manoeuvre of 4 for
  \leaderClive and a medium investment, thus getting a final modifier of
  \bonus{+11}. \FRA also has a \FTI of 5 and a manoeuvre of 4 for
  \leaderDupleix but only reacts with a small investment (after all,
  India can't be more important than the sugar Islands of the Caribbean,
  says the King) for a final modifier of \bonus{+11} (\bonus{+2} for
  defending its \dipAT).

  \ANG rolls 8, for a final result of 19 while \FRA rolls 7, for a final
  result of 18. Since the English result is higher, the diplomatic
  status of \FRA is lowered by one level and goes to \dipFR.
\end{exemple}

\subsection{Consequences of "Formal Relations"}
\aparag In the provinces of the minor country, neither Native Activation
(during each round), nor reaction of the \MIN due to the presence of
military forces, will be made if only stacks of one \LD would be
responsible of the test.
\bparag The presence of more than one \LD in any one province, or of an
\ARMY may still cause such activation.
\aparag Any war (normal or overseas) between the powers and the minor
country will break the status to \dipNR. Native reaction in a province
of the minor country is not a war and changes nothing.

\subsection{Consequences of an "Alliance Treaty"}
\subsubsection{Generalities}
\aparag For \paysInca, \paysAzteque, \paysGujarat, \paysMogol,
\paysChine, \paysJapon and \paysAfghans, the \MAJ has to pay 1d100
\ducats immediately, or the status remains \dipFR only.
\aparag For \paysVijayanagar, \paysSiberie, \paysOman, \paysAden,
\paysSoudan, \paysMysore, \paysHyderabad, \paysIroquois, and \paysOrmus,
the \MAJ has to pay 2d10\ducats immediately, or the status remains
\dipFR only.
\aparag The effect of \dipFR on lone \LD is still applied.
\aparag
Supplementary effects vary according to each \MIN.
\aparag Having an \dipAT is analogue to a \VASSAL status for Victory
Conditions.

\subsubsection{\paysJapon and \paysChina}\label{chDiplo:Diplo-japon}\label{chDiplo:Diplo-chine}
\aparag The \MAJ can have a \TP in each area of the minor country that
will not cause a test of reaction of the Native country at the beginning
of the turn.

\aparag[Closure of China or Japon.] Events \ref{pIII:CCA:Closure China}
and \ref{pIV:JCA:Closure Japan} close respectively \paysChine and
\paysJapon for the following effects:
\bparag The reaction level of the country is raised to 11 (so a reaction
is automatic if the conditions are met); the fidelity is raised to 16.
\bparag The country refuses any diplomacy, except as detailed
afterwards; existing diplomatic status remain so (and other powers are
forbidden to try opposing existing relations);
\bparag \dipAT allow each country to keep only one \TP in \paysChine or
\paysJapon, and not one per area (that \TP causes no reaction of the
minor country);
\bparag No new \TP counter can be placed in any area belonging to the
country, by means of administrative actions, except in
\granderegionFormose or \granderegionCoree;
\bparag The only way to have a new \TP is to take control of the \TP of
another country (by military means, and a peace, or by placing a new \TP
in the same province and using automatic concurrence to try to replace
the existing \TP) in which case the Treaty status is given to the new
controller of the \TP and lost by the previous one.%  Another way is to
% force a Treaty on the minor by means of a victorious war against it.
\bparag New areas that would be conquered later by \paysChine or
\paysJapon would suffer from the same restrictions, but existing \TP or
\COL are not destroyed immediately (unless the event says so).
Moreover, for the new areas controlled, the Activation level is 6 only
(and not automatic). In these area, it is possible to create new \TP by
administrative action, but the rest of the restrictions apply.
\aparag[Treaty of Nerchinsk.] Event \ref{pV:Treaty Nerchinsk} results in
the annexion by \paysChine of area \granderegionAmour, and some
provinces in \granderegionBaikal.
\bparag The Activation level of \paysChine is 6 herein.
\bparag Powers having a \COL/\TP in this area are allowed to attempt
diplomatic actions on \paysChine. If they manage an \dipAT status, they
can have and keep up to 2 \COL/\TP in \granderegionAmour, or (exclusive)
keep one existing in the rest of \paysChine (as per the previous rule ;
note that such a \TP can not be created) that will not cause reaction of
the minor.

\subsubsection{\sectionpays{Vijayanagar}}\label{chDiplo:Diplo-vijayanagar}
\aparag \pays{Vijayanagar} will not react to the presence of \TP in its
provinces. It will react to the presence of \COL.
\bparag Exception: with an \dipAT of \POR, \pays{Vijayanagar} will never
to the presence of a Portuguese \COL in its territory.
\aparag Neither \pays{Vijayanagar} nor natives in its territory will
react to the presence (movements or remaining) of stacks of at most one
\ARMY\faceplus in its territories.

\subsubsection{\sectionpays{Mogol}, \sectionpays{Siberie}, \sectionpays{Soudan}, \sectionpays{Afghans}}
\label{chDiplo:Diplo-mogol}
\label{chDiplo:Diplo-siberie}
\label{chDiplo:Diplo-soudan}
\label{chDiplo:Diplo-afghans}
\aparag The concerned minor country will not react to the presence of
\TP\facemoins in its provinces. It will react to the presence of \COL or
of \TP\faceplus.
\bparag Exception 1: with a Treaty, \paysAfghans will not react to the
presence of a \COL in \provinceHerat.
\bparag Exception 2: with a Treaty, \paysSoudan will not react to the
presence of \COL of \TUR.
\bparag Exception 3: with a Treaty, \paysMogol will never react to the
presence of a Portuguese \COL in its territory.
\aparag Neither the minor country nor natives in its territory will
react to the presence (movements or staying there) of stacks of at most
one \ARMY\faceplus in its territories.

\subsubsection{\paysGujerat, \paysOman, \paysAden,
  \paysaceh}\label{chDiplo:Diplo-aden}\label{chDiplo:Diplo-oman}\label{chDiplo:Diplo-gujarat}\label{chDiplo:Diplo-aceh}
\aparag Neither the minor country nor natives in its territory will
react to the presence (movements or staying there) of stacks of at most
one \ARMY\faceplus in its territories.
\aparag \label{chDiplo:AdenOmanExoticResources} If there is only one
power having \dipAT with the country, the resources produced by the
\TP/\COL of the minor country are given to this power (it gains the
income and count those resources as its own to obtain a monopoly).
\aparag The minor country can be used as an ally in wars.
\aparag They do not react to \COL of \TUR, except \paysaceh.
\aparag \paysOman controls \provinceSocotra if no power has an
establishment (fort, \TP or \COL) in the province.

\subsubsection{\sectionpays{Mysore}, \sectionpays{Hyderabad}}\label{chDiplo:Diplo-mysore}\label{chDiplo:Diplo-hyderabad}
\aparag The minor country will not react to the presence of \TP in its
provinces. It will react to the presence of \COL.
\aparag Neither the minor country nor natives in its territory will
react to the presence (movements or staying there) of stacks of at most
one \ARMY\faceplus in its territories.
\aparag The minor country can be used as an ally in wars.

\subsubsection{\sectionpays{Iroquois}}\label{chDiplo:Diplo-iroquois}
\aparag \pays{Iroquois} will not react to the presence of \TP\facemoins
in its provinces. It will react to the presence of \COL or of
\TP\faceplus.
\aparag Neither \pays{Iroquois} nor natives in its territory will react
to the presence (movements or staying there) of stacks of at most one
\ARMY\faceplus in its territories.
\aparag The minor country can be used as an ally in wars.
 

\subsubsection{\sectionpays{Ormus}, part of \sectionpays{perse}}\label{chDiplo:Diplo-ormus}
\aparag[Specifics of Ormus.] 
\province{Ormus} is a \ROTW province in \seazone{Persique} belonging to
\pays{perse}.
In general, \province{Ormus} is dealt with as a normal \ROTW province
(allowing forces to enter in it without war declaration, placement of \TP, etc.),
with usual Native reaction, or country \pays{Ormus} reaction.
\bparag No \COL can ever be placed in the province (but a \TP may be).
\bparag A reaction of the minor \pays{Ormus}
is actually a declaration of Overseas war by \pays{perse},
as is a war declaration against \pays{Ormus}.
\bparag A country at war against \pays{perse} and the owner of  forces or \TP in
\province{Ormuz} is allowed to attack it from the European map also.
\bparag The fortress in \province{Ormus} acts as a \Presidio against \province{Bam}.
\bparag See also~\ruleref{chBasics:Provinces:Ormus}.
\aparag[Effects of a Treaty.]
\dipAT with \pays{Ormus} allows a player to have a \TP in
\province{Ormus} that attracts no reaction from \pays{Ormus}, as long as
the \dipAT holds.
\bparag The power can also enter this province with military forces, or
fortify the \TP. This draw no reaction from \pays{Ormus}.
\aparag[Afghanistan.] \pays{perse} may also own
\granderegion{Afghanistan} because of some event. It will not react to
the presence in this area of \TP\facemoins of a power having a \dipAT
with \pays{Ormus}. It will react to the presence of \COL or of
\TP\faceplus.
\bparag \pays{perse} will also not react to the presence (movements or
staying there) of stacks of at most one \ARMY\faceplus in
\granderegion{Afghanistan}, if those are owned by a power having a
\dipAT with \pays{Ormus}.  Neither would natives react under this
condition.

\subsubsection{\sectionpays{Inca} and \sectionpays{Azteque}}\label{chDiplo:Diplo-inca}\label{chDiplo:Diplo-azteque}
\aparag[Permanent \dipAT of Incas and Aztecs.]
In 1492, \pays{Inca} and \pays{Azteque} are always in \dipAT with every
power.  This can change because of event \eventref{pII:American
  Resistance}, or when a power besieges their capital.
\aparag[Effect of \dipAT.]
\bparag The concerned minor country will never react, neither to
military forces, nor presence of \TP/\COL.
\bparag Natives in the area of the country can be attacked with no
declaration of war. The capitals of the empires can also be attacked
without war against the country (but Natives has to be attacked first
for assault or siege).
\aparag[Fall of the American empires.]
\bparag If its capital is controlled by a power at the end of a turn, an
American empire is destroyed. The number of Natives in each province is
now 2 \LD (instead of 20 \LD).
\bparag Place immediately a \COL of level 3 on the city, owned by the
power controlling the city. If this power is \SPA, it must immediately
place a mission there, either by drawing an available mission in the
pool, or by moving a deployed mission that is in the same area; then the
highest rank Conquistador present in the region is nominated as Vice-Roy
of the area.
\aparag[Attack on capital]
Whenever the capital of \pays{Inca} and \pays{Azteque} is attacked, a
test of reaction is made at the end of the round (after the result of
siege or assault). If there is a reaction, the concerned minor country
declares an immediate Overseas war against the aggressor.
\bparag Its troops are deployed (even in occupied provinces) and Natives
in all its provinces are activated for the war.
\bparag If this is the last round of the turn, the Fall of the Empire is
suspended for this turn (but may happen on the future turn).

\subsection{Countries from the \ROTW as ally}
\aparag Some countries from the \ROTW in \dipAT can be used as ally in
wars: \pays{Aden}, \pays{Oman}, \pays{Gujerat}, \pays{Mysore},
\pays{Hyderabad} and \pays{Iroquois}.  The power having the \dipAT can
ask for a limited intervention.  This is a declaration in reaction, and
is shown by placing the forces of the \MIN on the map.
\bparag If more than one power have \dipAT, all that want can ask for
limited intervention. Then they all roll 1d10, modified by the modifiers
for diplomatic actions in the \ROTW. The power that rolls highest gains
the intervention for this turn (in case of ties, no intervention). This
test should be renewed at each turn, and the side of intervention thus
may change.
\bparag[Reciprocal alliance.] As it is an alliance, if the \MIN is declared war
upon or if it declares war, it will call for its patron, that is also an
ally. If the power does not respond the alliance (at the least a
limited intervention), the status is broken to \dipFR. 
\aparag[Conditions of the Limited intervention in \ROTW.]
\bparag A limited intervention of a minor country is made only with its
basic forces. It draws supply only from its own provinces (and so can
not go further than 12 MP from its country). Its units can not go on the
European map.
\bparag The intervention is at most of one land stack and one
naval stack outside the provinces of the minor country. 
\bparag The \MIN receive reinforcements each turn in the administrative
phase. The base reinforcement is given in the annexes. These
reinforcements are only used to recreate de basic force of the \MIN,
should they be diminished.
\bparag All campaign costs for the \MIN are paid by its ally.
\bparag In the provinces that it controls, the \MIN is allowed to attack
forces of enemies, but the Natives are not activated (only the basic
forces may attack). During the end of turn, the forces can do "Native
attack" on \TP/\COL of an enemy power that is in an area the \MIN
controls, but this does not use also the Natives (unless specified in
the description of the country).
\bparag The \MIN is in fact out of the war. The \MIN is not part of
Peace Treaty.  But its territories could be crossed as it is usually
permitted.

\subsection{Military Diplomacy and  Treaty}
\aparag A power at war (normal or overseas) against a country in the
\ROTW signing a
victorious peace treaty of level 2 or higher, and forfeiting all other
conditions of peace, may do the following:
\bparag reducing any or all \dipAT and \dipFR of other powers,
to respectively \dipFR and \dipNR.
A power that has its diplomatic status broken this way gains a temporary
free Overseas \CB against the responsible power;
\bparag and, sign a \dipFR with the \ROTW country, or
upgrade a \dipFR in \dipAT.
 
\aparag A power at war (normal or overseas) against a country in the
\ROTW achieving a peace of level 4 or higher, and forfeiting all other
conditions of peace, may do the following:
\bparag break any or all \dipAT and \dipFR of other powers to \dipNR.
A power that has its diplomatic status  broken this way gains a temporary
free Overseas \CB against the responsible power;
\bparag and upgrade its position by imposing a \dipAT to the \ROTW country.

\aparag Note that Allies in this victorious war can each apply the
previous effects (excepted to break or reduce \dipFR or \dipAT of
Allies in the same war).

\subsection{Activation of \ROTW minors}
\aparag At the end of the diplomatic phase, a test is made for each
\ROTW minor to see whether it declares war against countries inside its
territory.
\bparag a \ROTW minor may react against any or all countries having
either troops (including \LDE, forts of fortresses) or colonial
establishment (\COL or \TP) inside its territory (the areas it owns and
the provinces with its own colonial establishments).
\bparag \dipFR and \dipAT may allow some troops and/or establishment
inside the territory of a minor without triggering activation, as
explained above.
\bparag Leaders alone (with no troops) never cause minor activation.

\aparag For each minor and each country that can cause activation of the
minor, roll one die.
\bparag If the roll is strictly smaller than the Activation level of the
minor, it declares an oversea war against the offending country (and
breaks an eventual \dipAT or \dipFR to \dipNR).
\bparag Otherwise, nothing happens.
\bparag Activation levels are given in the Appendix (in the description
of the minor) and recalled on the \ROTW diplomatic track (on the \ROTW
map).
\bparag Remember that for some countries (eg \paysChina), the activation
level may depend on the province where the troops or establishment are
located.
\bparag It is completely possible for a minor country to declare war
this way against one offending country  but nor against another, even in
the same turn. The test is made for each offending country separately
(in decreasing order of initiative in case the order is relevant).

\aparag Reactions after these declarations of war happen as usual.

\aparag Activation of \ROTW minor should not be confounded with
activation of the natives.
\bparag The former is the whole country declaring war, it is done in the
diplomatic phase and result in diplomatic announcements.
\bparag The later is local population reacting, it is done during each
military round and does not causes a new war or change the diplomatic
status. Moreover only one province is concerned each time.
\bparag Colonial establishments usually to not cause native activation
(the local population is rather happy to trade) while it may cause minor
activation (the government is not happy to see its trade regulation
broken by European).
\bparag The same troop, however may both cause minor activation and
native activation (and thus must roll both in the diplomatic phase and
each military round as long as the condition for activation exists).

\begin{designnote}
  Since the activation happens at the end of the diplomatic phase, you
  have one attempt to get a good diplomatic status after landing
  troops. This typically occurs in two cases:
  \begin{itemize}
  \item At the end of a military phase, an emissary lands in a
    country. During the upcoming diplomacy phase, the emissary has one
    attempt to establish diplomatic status with the country before the
    troops he might have with him cause minor activation.
  \item During the event phase, a \RD causes the diplomatic status of a
    \ROTW minor to decrease. You have one attempt to re-establish it
    before seeing your trade burnt to the ground (or more if by chance
    the minor is not activated this turn\ldots)
  \end{itemize}
\end{designnote}

%\aparag[Ormus.] The only possibility to impose an \dipAT on \pays{Ormus}
%is by conquest of the province \province{Bam} from \pays{perse}. As long
%as the province is owned by another power, an \dipAT is enforced between
%this power and \pays{Ormus} when they are not at war; no other power can
%achieve diplomatic status with \pays{Ormus} when \province{Bam} is not
%Persian.


% -*- mode: LaTeX; -*-

\section{On wars}

% RaW: [34]



\subsection{How Wars Begin}

Wars take place due to independent decisions of any player or players
(announced during the Diplomatic phase) or may be started by events.


\subsubsection{Wars caused by events}
\aparag Some wars may be caused by events, offering a \CB to some \MAJ, or
telling that some \MIN declares a war.
\bparag The description of political events may offer a \CB to some
countries. The \CB that are described under the "Event Phase" part are used
during the second step of the Diplomacy Phase, before formal Agreements are
made. By order of Initiatve, all players announce which declaration(s) of war
allowed by events they use, or not.
% Jym : this is in contradiction with the sequence of play.
\bparag The reaction on wars breaking down this way are resolved at that
time. Note that no new Formal Agreement could have been signed at this turn,
but Alliances of a past turn are usable (they finish in the next segment
only).
\bparag If an event gives several \CB, all countries using these \CB are
automatically allied for this war (only), unless the event specifically speaks
of distinct wars being possible.

\aparag[Wars continuing other wars] If a war should begin between two
countries already at war against each other, the exact meaning of this depends
on the nature of the war about to begin for the country declaring the war:
mandatory, incompatible with other wars, or provoked by the country. Most
events are mandatory; the other ones are explicitly mentioned in the event.
\bparag[Mandatory war] The new conditions of war described in the event are
added to already existing conditions. A \MAJ can announce at the diplomatic
phase that an already running conflict becomes the new war. Calls for allies
are made at this point (according to the conditions of the new war) because
the war's motives change. The only thing that should be ignored is the initial
declaration of war, since the country is already at war (a \CB for this turn
is deemed to have been used).
\bparag[Incompatibility] The new war can be made incompatible with wars
between the two countries about to begin the new one. Usually, the event calls
for a replacement event (the event did not happen at all, and another one is
rolled for instead). However, a war with incompatibilities can be followed by
a mandatory war.
\bparag[Controlled war] The new war is indicated as being controlled by a
country. It may delay the event (which, as above, did not happen at all and is
replaced by another one), or accept the event and apply it as if it were
mandatory.
\bparag[Armistices] An Armistice may not be signed for an ongoing war that is
transformed by either a controlled or a mandatory war.


\subsubsection{Wars by voluntary declarations}
\aparag Wars are also declared during the Diplomatic phase by the attacking
player, in the fourth segment of the phase, after the segment of Announcements
of Formal Agreements. No private negotiation is permitted between the
Announcements and the Declarations of Wars.
\aparag War using \CB described under the section "Diplomacy Phase" of an
event, has to be declared at that time.
\aparag A whole segment of reactions following these declarations of wars is
then made.
\aparag[Restriction on Wars]
A War is usually declared against an Alliance that is either a power currently
at peace, or an Alliance already formed in an ongoing war.
\bparag The only way to declare against only one power of a warring Alliance
(instead of the whole Alliance) is if the attacker has a valid \CB and uses it
against this power.



\subsection{Casus Belli}

\aparag
A Casus Belli (\CB) allows declaring war by losing only {\bf 1} \STAB level,
without any loss of victory points (\VP). \CB are of two different types,
permanent or temporary, and may be usual or free. Free \CB allows declaring of
wars without loss of \STAB.

\aparag[Temporary Casus Belli]
The temporary \CB is provided by events, or by the rules. Usually it may be
used only once and is then cancelled; a temporary \CB is valid for 6 turns,
excepted if specified differently in the description of the \CB.  Some
temporary \CB are linked to the existence of a condition: the \CB is valid as
long as the condition is met; if the \CB is used and the war terminates, the
\CB could still be valid if the condition is satisfied.

\aparag[Permanent Casus Belli]
Here are the permanent \CB:
\bparag Following the event \eventref{pI:Reformation}, all Catholic countries
have a permanent \CB against all Protestant countries (and vice versa). This
is no longer valid after the end of \terme{Religious Enmities}.
\bparag \SPA has a permanent \CB against all Pagan or Muslim countries.  This
is no longer valid after 1700, included.
\bparag \TUR has a permanent \CB against all Christian countries, against
\pays{perse},
% Jym. Added Mameluks.
against \paysEgypte and against \paysDamas. This is no longer valid after
1700, included.
\bparag A player has a permanent \CB against any country (player or minor)
that has has annexed a national province of the player.



\subsection{Cost of a War Declaration}

\aparag A declaration of war costs VP, as well as a loss of Stability,
according to whether the player has a \CB or not.

\aparag Cost in Victory Points
\bparag No \VP: with \CB
\bparag -10 \VP: without \CB, against a player or a minor country vassal of a
player.
\bparag -5 \VP: without \CB, against a minor (except vassal minor country -
see above).

\aparag Cost in Stability
\bparag none: with a Free \CB.
\bparag -1 level: with \CB.
\bparag -2 levels: without \CB.
\bparag[Note]
Cost in lost \STAB may be altered by existing treaties and alliances between
players, or also by event description. Especially, breaking and alliance
(either defensive or offensive) costs 2 extra levels of \STAB.

\aparag[Wars and reduction of Trade]
The war forces all belligerent players to refuse mutually the trade access to
their market. This influences the calculation of their foreign trade income as
follows:
\bparag The European market value of each power is decreased by the amount of
Income of the enemy player's provinces (including vassals).
\bparag Other commercial income sources (commercial fleets, exotic resources,
etc...) are not affected directly by the state of war.
\bparag Note that this reduction of Trade does not affect the commercial
fleet, as would do a Trade Refusal declaration (but a declaration for this
effect can be added to the war).



\subsection{Overseas Wars}\label{chDiplo:Overseas Wars}


\subsubsection{Commercial and Overseas \CB}
\aparag Some \CB are obtained to wage a restricted kind of war that is called
an \terme{Overseas War}. They are called Commercial \CB or Overseas \CB and
may be free, permanent or temporary as usual. Some events, or conditions in
the rules, give other Commercial or Overseas \CB, as indicated in their
description.  \overseascb

\aparag A Commercial/Overseas \CB may be used to initiate an Overseas
War. Declaring an Overseas War without a Commercial/Overseas \CB is not
allowed.
\aparag When an Overseas War is declared, reactions caused by the war may be
made as usual.


\subsubsection{Permanent State of Overseas War}
\aparag[Barbaresques.]
\terme{Barbaresques} countries are \pays{Cyrenaique}, \pays{Tripoli},
\pays{Tunisie}, \pays{Algerie} and \pays{Maroc}. They are always in a state of
restricted Overseas War against every Christian countries.
\bparag It allow them to use Privateers and naval forces (no land forces) to
attack Christian countries. Christian countries can use their own naval forces
or \Presidios to fight against the \terme{Barbaresques}.
\bparag As an exception, Privateers of the \terme{Barbaresques} may loot
European provinces adjacent to the \STZ they attack, even if they are European
provinces usually outside the scope of Overseas Wars.
\bparag \TUR plays the \terme{Barbaresques} that are neutral, and the
diplomatic patrons play those that are not. The specific rules tell the \STZ
that are attacked by the Privateer.
\bparag This state of war causes no loss of \STAB.
\bparag[Reinforcements] They receive some reinforcements each turn:
\pays{Algerie} gains a \corsaire\facemoins each turn; in periods I to III it
receives also a \ND or 2 \NGD (player's choice) and in periods IV and after,
only one \NGD or a \NDE. Other countries gain only a \corsaire\facemoins 2
turns after their Privateer has been destroyed.
\bparag \textit{Exception.} Whenever \leader{Dragut} is in play and if it used
in its Privateer leader role, a \corsaire\facemoins of \pays{Tunisie} is
raised (even if eliminated at previous turn).
\bparag[Mandatory Sea Sortie] The Privateers usually have to go out at sea
each turn, except if their Patron decides against it: a test is made at the
beginning of the 2nd round if the Privateer is not at sea, by rolling 1d10 for
each country the Patron wants to keep the Privateer at port.  This is
permitted if the result is lower or equal to the number of the current period
plus the Diplomatic status bonus and the geopolitical bonus.
\aparag[The Knights.] The \pays{chevaliers} is always in a state of restricted
Overseas War against \TUR.
\bparag It allow them to use Privateer and naval forces (no land forces) to
attack \TUR. \TUR can use their own naval forces to fight against them.
\bparag The diplomatic patron of the \pays{chevaliers} play this forces, or
\SPA if it is neutral.
\bparag The annexes specify the reinforcements gained by the \pays{chevaliers}
each turn: a \corsaire\facemoins (or \Faceplus if in \province{Rhodos}), and a
\NGD or a \NDE.
\bparag This state of war does not cause automatic \STAB loss at the end of
turn.  But, at each turn that the pirate of The Knights inflicts losses on
Turkish commercial fleets, \TUR loses at least 1 \STAB level (that is, the
Knights' privateer causes a loss of \STAB if and only if \TUR does not already
loose \STAB for another reason at the end of turn (war, revolts, \ldots))


\subsubsection{Restriction in Overseas Wars}
\aparag[Reaction of the victim.] A country that has an Overseas war declared
upon gains a temporary \CB against the attacker to declare a regular war.
\bparag If/When this \CB is used, the war changes and causes a whole new set
of reactions allowed by this new full-blown war. The state of Overseas war is
no more.
\bparag This \CB can be used in reaction as a free \CB on the first turn of
the war, or as a normal \CB to declare a full war on following turns (as long
as the Overseas war continues).
\aparag Reactions other than this case are restricted:
\bparag Calls of allies (Formal Alliance or Limited Alliance) are made as
usual excepted that they give only Overseas \CB;
\bparag No minor country may be involved completely in an Overseas war if it
was not the victim of the war, or if it is not a \VASSAL of an involved \MAJ;

\aparag[The course of the war.]
\bparag Overseas wars can cause no military action on the European mainland
(that is all land provinces on the European map), except provinces in
\terme{Barbaresques} countries, \pays{Egypte} and \pays{Irak}.
\bparag No trade refusal or reduction is applied (except if an added
declaration of Trade Refusal is made by one country).
\bparag An Overseas War is not exactly a state of War for the power.  If it is
its only war, a \MAJ would have to use the costs of Maintenance as if at
peace.
\bparag Minor countries in \EG cannot be called for a full intervention in the
war.
\bparag In any other aspect, except when specified, an Overseas War is
conducted as a regular war. For instance, any naval operation, attacks by
Privateers, fights in the \ROTW (\COL, \TP, in any provinces on the \ROTW map)
are allowed, as well as limited intervention of \MIN.

\aparag[Peace and Overseas wars.]
\bparag A minor country always accepts a proposed white peace to end an
Overseas War at the end of a turn.
\bparag A peace treaty ending an Overseas War may not involved change of
ownership of any province on the European map.
% PB - TBD: allow annexion on African Coast or not ? I favor not. Then
% Overseas wars against Barabresques may target Praesidios or Diplomatical
% Alignment.
\bparag Transfer of \TP (even \Facemoins) counts as a full province.
\bparag If an Overseas War is not finished at the end of a turn, the loss of
\STAB (due to this war) by involved countries is limited to 2 levels per turn
(instead of 4).



\subsection{Reactions to a Declaration of War} \label{chDiplo:diplo:Reactions}


\subsubsection{Generalities about Reactions}
\aparag On both segments allowing Declarations of wars, Reactions can be made
by any power, after all initial Declarations of War.  Going through in the
order of initiative, and then circling again until no-one has anything left to
declare, each power can make none, one, or several declarations in reaction.
\bparag Note that some reactions can only be made just following some initial
declaration (usually a new war, or mere new conditions due to events) -- at
the same turn and segment; whereas others can be made spontaneously at any
turn.  \diploreac


\subsubsection{Guidelines about successive declarations of wars.}
\aparag No new war can begin by reactions (excepted by reacting to a Trade
Refusal). Reactions are mere extensions of an existing war.  One can react
after a reaction, broadening further the scope of the war.
\aparag When a reaction puts a country in a war, this country has to join a
whole alliance and its thus at war against every enemies of this alliance. If
it is allied to countries in both sides of the war, it has to break one of the
alliances.
\aparag The sole possibilities to have multi-sided wars is then to have
different wars involving the same country(ies). All country that join the
alliance at war against several alliances at the same time will have to
declare war against all those alliances.
\bparag Conversely, entering the war at the side of an alliance B, when
alliance A is at war against B and C, is a war only against A and the
Neutrality is conserved regarding C, i.e. no co-operation, no supply, no
passing through provinces controlled or occupied by the other alliance.  Note
that this situation gives a \CB to alliances B and C against the other one, or
on the contrary, they could declare that they ally together in this war.
\bparag Three-sided wars (or more) where more than two alliances are at war
against each other are allowed.


\subsubsection{Signing an Alliance for Intervention}
\label{chDiplo:InterventionLimitee}
\aparag Alliances for Intervention are signed in reaction to a declaration of
war. Such an Alliance involves two Major powers, one at war and another
one. The second country enters then the war in a limited intervention at the
side of the alliance of the first power.
\bparag This is a kind of alliance and the intervening power uses a \CB given
by the alliance to enter the war in this limited way: it loses {\bf 1} \STAB.
\bparag Usually, only a country that is victim of a declaration of war (even
in reaction due to alliance, or by a minor country) can sign an Alliance for
Intervention.
\bparag Exception: \ENG and \PRU may always sign Alliances for Intervention
with attacking countries.
\bparag Signing an Alliance for Intervention is only possible on the first
turn of a war (or new developments), except if written otherwise in some
events.
\bparag Limited intervention is forbidden in Religious or Civil Wars, excepted
if the event explicitly says otherwise.

\aparag[Conditions of a limited intervention of a \MAJ.]
\bparag The power is not at war because of the intervention. It uses the costs
of Maintenance at peace (if not involved in another war).
\bparag The power can use up to one land stack and one naval stack to do
anything as part of the war. Once a land or naval stack has been committed, no
other land or naval (respectively) force of the power can be involved in this
war. These forces are the only one that can move in provinces at war, attack,
besiege, assault, do naval transport of forces at war, make a blocus, fight
against Privateers, and so on\ldots All conquests (including captured
monarchs) are made for the sake of the alliance at war (he chooses one
country, a \MAJ is possible). All pillages made by his stack go in his TR.
\bparag All other forces of the power doing a limited intervention are as if
at peace.  All provinces of the power are also not part of this war and only
its forces can enter them.
\bparag Minor countries controlled by the power are not part of the
intervention (this includes \VASSAL). Exception:
see~\ref{chSpecific:England:Minors at war}.
\bparag A power can do limited interventions at the same time in more than one
war. It cannot intervene on the side of enemy alliances.

\aparag[Continuation of a limited intervention.]
\bparag After the Truces, if the war is still going on, any power of the enemy
Alliance has first the possibility to declare Full war against the intervening
Power, having a \CB and paying 1 \STAB to do so.
\bparag Else, a limited intervention ends at the end of the turn, excepted if
the power doing the intervention spends 1 \STAB at the end of turn (after
\STAB improvement action), in addition to any other loss of \STAB.
\bparag If the intervention ends, the forces are redeployed as when signing a
white peace. There is no gain of \STAB.
\bparag If the intervention continues, the power will be able to send
reinforcements as long as those are stacked at the end of the first round with
the intervening stacks.
\bparag If the intervention continues, the enemy alliance has a free \CB at
the following Event Phase to declare a full war against the intervening power.

\begin{exemple}[Alliance going into flames]
  It is turn 10. \HIS, \VEN and \POL are allied in a holy Catholic league
  (defensive alliance) while \TUR and \FRA also have a defensive
  alliance. \TUR decides to send the \terme{Levant} convoy
  (see~\ref{chIncomes:Levant Convoy}) to \FRA, thus providing a commercial \CB
  to \VEN (who owns the Mediterranean centre of trade and thus believes he
  should get the convoy).

  \VEN decides to use this \CB (thus loosing 1 \STAB). \TUR reacts by turning
  the war into a full blown war, hoping to advance in the Balkans (no \STAB
  lost as this is a free \CB). Since \VEN has now been victim of a declaration
  of war, the Doge calls his Polish allies (to protect the Balkans) and \POL
  accepts and declares war on \TUR (cost 1 \STAB for \POL). \TUR then decides
  to call its minor \VASSAL, \paysCrimee, fully into the war to chop on the
  Polish flank.

  In the West, \HIS was not called into the war, however, \monarque{Charles V}
  decides that this is a good opportunity to try and seize
  \provinceTunis. Thus, \HIS uses the \CB provided by his alliance and declare
  war to \TUR and then to its \VASSAL, \paysTunisie (1 \STAB
  lost). \monarque{Francois I}, always eager to harm the Hapsburg, then uses
  its alliance to react to the Spanish aggression by also declaring war. He'd
  like to declare war only on \HIS but cannot as war must be declared against
  the full alliance, in this case \VEN, \HIS, \POL (and maybe some minors
  allies). This cost him 1 \STAB.

  \HIS would then like to call for a full war his ally, \paysPalatinat, in
  order to open a second front against \FRA. However, \paysPalatinat is only
  in \EW. Since \paysPalatinat is not adjacent to \FRA but nonetheless less
  than 6 MP away, and \HIS has no specific bonus on it, he must roll 6 or more
  on a die to successfully call it. \HIS rolls 7 and \paysPalatinat declares
  war on \FRA.

  Back in the East, \RUS believe that this could be an opportunity to weakens
  the Crimean. So, he react to the Turkish attack by signing an alliance for
  limited intervention with \HIS, \VEN and \POL (cost 1 \STAB).

  After Diplomatic actions on minors are made, both \paysKazan and
  \paysAstrakhan are on the Turkish diplomatic track, thus \TUR decides to
  call them for limited intervention in this full blown war (to defend
  \paysCrimee).

  Both \paysCrimee and \paysPalatinat are fully at war. They will thus receive
  reinforcements in the upcoming administrative phase. On the other hand,
  \paysKazan and \paysAstrakhan are only in limited intervention. They will
  only have their basic forces but are not part of the war (and thus cannot be
  entered by enemy troops). \RUS is also not fully at war. He will use the
  (more expensive) peace maintenance cost and cannot send more than one stack
  in the war ; moreover all his conquests will be made for the behalf of
  another major (for example \HIS), and count as his for peace purpose. But no
  enemy troops can enter Russia and besiege his fortresses.

  At the end of turn, \RUS can choose to stop its intervention. In this case,
  Russian troops go back in Russia but the fortresses he has conquered are not
  given back to \TUR (they are still controlled by \HIS). Alternatively, \RUS
  can choose to stay in intervention (loosing 1 \STAB). In this case, at turn
  11, \TUR can choose to generalise the war and fully imply \RUS in the war
  (with no \STAB lost, this is a free \CB to be used at the same time as \CB
  provided by events). If this is done, this new declaration of war can causes
  a full new set of reactions\ldots
\end{exemple}

\begin{exemple}[Three-sided wars]
  In 1700 (turn 42), \ref{pVI:Great Northern War} is rolled. As per event
  description, it provides both \RUS and \POL \CB against \SUE (plus some
  other conditions). Both \RUS and \POL separately decide to use them. So,
  there are two wars going on: \RUS (and eventual allies) against \SUE and
  \POL (and allies) against \SUE. However, Russian may not enter Poland or
  attack Polish troops and conversely as these countries are not in the same
  war. Swedish troops (and allies) can go both in Poland and Russia as \SUE is
  at war against both. Note that if a Swedish fortress is besieged and taken
  by \RUS, \POL cannot later go and besiege it as this would be an attack
  against a Russian fortress\ldots

  In turn 43, the war is going on. Since there are two alliances (namely \RUS
  and \POL) at war against the same third alliance (\SUE), they can do one of
  the following:
  \begin{itemize}
  \item Keep the wars separate and continue as the previous turn.
  \item Decide to join the wars. \RUS and \POL will then be allied for the
    duration of the war (only). They can now go in each other territory, stack
    troops together, \ldots but must sign a peace together.
  \item Declare war one to another. The alliance (\RUS or \POL) declaring the
    war loses 1 \STAB for this (normal \CB). Then, there will be a three-sided
    war between \SUE, \RUS and \POL. Each of them can go in each other
    territory, or attack each other troops. Polish troops can now besiege a
    Swedish fortress that was previously taken by \RUS and, in case of
    success, the fortress will be controlled by \POL (and count as such for
    peace). Three different peaces will need to be signed as there are 3 wars,
    each peace using specific differential for its own war\ldots
  \end{itemize}
\end{exemple}


\subsubsection{Armistice}
\aparag An armistice can be signed in any war that began in a previous turn
(but not if it begins this turn, or has new conditions due to an event or a
transformation from Overseas to full war).  All powers in both enemy alliances
has to agree the Armistice; if not, none is signed.
\bparag Usually, no Armistice is allowed in Religious or Civil Wars, excepted
if the event says otherwise.
\bparag Some events call for mandatory Armistices: no one has to agree\ldots
\aparag The countries stay at war for the turn but can make no offensive
action against the enemy alliance. All besieged provinces at the time of the
Armistice has to be freed on the first round. Provinces that are controlled by
the enemy stay so.
\bparag During the turn, it is forbidden to enter a province, \COL or \TP of
the enemy that was not controlled at the beginning of the turn.  Interception,
siege, attack by naval units or privateers are also forbidden.
\bparag Use of \Presidios or \StraitFort, however, is still allowed (as when
the countries are at peace).
\aparag At the end of the turn of the Armistice, if no peace is signed, the
enemy alliances lose 1 \STAB in addition to normal losses (after \STAB
improvement action), in remplacement of the \STAB losses normally caused by
this war. Moreover, this turn will not be counted as a turn of war to compute
the length of the war (and the \STAB loss associated).
\bparag The countries are still considered at war for attempts of \STAB
improvement
% Jym
and maintenance.


\subsubsection{Religious Wars, Civil Wars}\label{chDiplo:Religious Civil War}
\aparag Some wars caused by events are said Religious Wars, or Civil Wars. In
a Religious War, any Major Power that shares the religion of one of the two
sides may intervene in the war to help the side having the same religion. In a
Civil War, any Major Power can intervene for one side or the other.
\bparag Those interventions are ruled by the \xnameref{chDiplo:Foreign
  Intervention} limits.
\bparag Several kinds of more important interventions (limited war or full
war) may be allowed in the precise description of the event.  Except for those
allowed, interventions, any other kind of war or attempts to be involved in a
Religious or Civil War implies the effects described in "Excessive Foreign
Implication".
\bparag Exception: during \eventref{pIII:Dutch Revolt}, wars against \HIS or
\HOL do not qualify as Excessive Foreign Implication if fought out of Holland
and the Spanish Netherlands.
% Jym: shouldn't Artois/Flandre/Hainaut be also allowed for an hypothetical
% action of FRA in South Belgium ?  excessive intervention should be limited
% to HOL national ter. + Vlaanderen/Brabant/Liege/Limburg.
\bparag[List of Religious Wars.]
% A ADAPTER :
\begin{todo}
  Double- or triple-check the list of religious and civil wars\ldots
\end{todo}
\eventref{pII:Schmalkaldic League}, \eventref{pIII:FWR Detailed},
\eventref{pIII:Dutch Revolt}, \eventref{pIII:League Nassau},
\eventref{pIV:TYW},
\eventref{pIII:Religious War Sweden}, \eventref{pIII:Religious War Poland},
\eventref{pIII:Times of Troubles}, \eventref{pIV:Bohemian Revolt},
\eventref{pIV:Augsburg Revocation}, \eventref{pIV:English Civil War},
\eventref{pIV:La Rochelle}
\bparag[List of Civil Wars.]
\eventref{pIV:Fronde}, \eventref{pIV:Unity HRE}, \eventref{pIV:Swedish Nobles
  Unrest}, \eventref{pIV:English Restoration},
\eventref{pV:WoSS}, \eventref{pV:Glorious Revolution},
\eventref{pVI:WoPS}, \eventref{pVI:WoAS}, \eventref{pVI:Jacobite
  Rebellion}, \eventref{pVI:Kurland}, \eventref{pVII:Pugatchev Revolt},
\eventref{pVII:Independence War}, \eventref{pVII:French Revolution},
\eventref{pVII:Bavarian Succession}.
\bparag Added to these lists, any War of Succession following a Dynastic
Crisis becomes a Religious Civil War before the end of \terme{Religious
  Enmities}, and a Civil War afterwards.

\aparag[Foreign Intervention]\label{chDiplo:Foreign Intervention}
Other countries may, without declaring a war on the country suffering the
civil war, send units to fight in that country. In Religious Civil Wars, the
intervention is necessarily on the side of a faction that shares same religion
as that of the intervening player.
\bparag This Foreign Intervention is not a war (nor a declaration of war) and
costs {\bf 1 \STAB} for each intervention.
% announced by a given power in a given war.  PB: changed from no \STAB
It is announced as a reaction during the Diplomatic Phase.
\bparag
This intervention is limited to a maximum of one land stack of at most one
\ARMY\faceplus, and/or one \FLEET counter per allied player.  (i.e. per
country, not group of countries).  These forces are the only one that can move
in provinces involved in the Religious/Civil War (including provinces of
powers that are fully involved in the war); movements or campaigns in the
\ROTW is not allowed (excepted if the event says otherwise).  All conquests
are made for the sake of the side supported in the war.  All pillages made by
his stack go in his own TR.
\bparag Minor countries controlled by the power are not part of the
intervention (this includes \VASSAL).
\bparag A power can do Foreign interventions at the same time in more than one
war. It cannot intervene at the same time on the side of enemy alliances.
\aparag[Continuation of a Foreign Intervention.]
\bparag A Foreign intervention ends at the end of the turn if no force of the
Foreign power stays in a province at war.
\bparag If the Foreign Intervention continues, no reinforcement can be send in
the war; no \STAB is lost by the intervening power.  It is possible to end an
intervention and resumes is afterwards (see next point) so that new forces are
sent.
\bparag A Foreign intervention can be resumed at any turn after it has ended
but this costs {\bf 1} \STAB to the Foreign power intervening. In Civil Wars,
the Foreign intervention could resume as an ally of the other side.

\aparag[Excessive Foreign Implication.]
No player can send more than one \ARMY\faceplus on the side of any one faction
in such a war, if a limited or full intervention of his power is not allowed
in the event.
\bparag If ever a power declares war on the country where the civil war rages,
the civil war stops temporarily in a mandatory Armistice. The victim country
may use units of both factions in his civil war to fight against the
invader(s). In addition:
\begin{enumerate}
\item Revolts do not incur any Stability loss during excessive foreign
  interventions.
\item Rebel and loyal units may not collaborate (i.e. transport, stack and/or
  fight together).
\item If an Excessive Implication occurs, events concerning the same Civil War
  are still marked off but their application is suspended. On any following
  turn when the intervention is over, such already marked off events (during
  the above intervention turns) will occur in addition of regular events on a
  even roll of 1d10 (no more than 1 per turn).
\end{enumerate}
\bparag However, the units of both factions are kept under the control of the
victim country until the peace is signed with all foreign invaders.

\bparag Once the Excessive Implication is over, the civil war is resumed and
the rebels receive reinforcements if they have lost 25\% or more of their
initial strength (proceed as per first turn of the civil war).

\begin{designnote}
  Excessive foreign intervention is not really meant to happen. If you start
  to think that it is often a good thing to do to achieve your goals, you're
  probably abusing some loophole in the rules. Typical games should not see
  more than one or two excessive foreign intervention (and most of the time,
  none should occur).

  Typically, trying to use excessive foreign intervention to artificially
  lengthen a civil war, lower the \STAB or your enemy or destroy loyal troops
  while keeping rebels alive to give them the edge are abuses.

  Excessive foreign intervention should only arise when another event is
  rolled and call for a new war with a country already in civil war.
\end{designnote}

\begin{todo}
  Add a (high) VP cost for EFI unless using a \CB provided by event to
  dissuade players from abusing it ??? -30VP should be enough to prevent
  abuses.
\end{todo}



\subsection{Call for ally by Minor countries}


\subsubsection{Generalities}
\aparag A minor country can be involved in various ways in a war:
\bparag Limited intervention, as per the previous rules; this intervention is
possible in a war of its Patron if the diplomatic status is \AM, \CE, \EG or
\VASSAL;
\bparag Full intervention if it was declared war upon, or if it declares
war. When a European minor country is fully involved in a war, no-one is
allowed diplomacy action on it.
\bparag In Overseas wars, the intervention are of the same kinds, but
constrained by the limits of Overseas wars.

\aparag A minor country can declare a war in the following occasions:
\bparag Some events (including \RD);
\bparag A \VASSAL is fully involved by its Patron, as a reaction. This costs
no additional \STAB.
\bparag The country is in \EG and its Patron tests for declaration of war by
the minor country (as explained in \ruleref{chDiplo:EW Effects}) and
successes.
\bparag A country in the \ROTW may declare an Overseas war due to reaction
against European presence.

\aparag A minor country can be declared war upon in the following occasions:
\bparag As a usual declaration of war (with, or without \CB; sometimes caused
by events);
\bparag If it is a \VASSAL, only as part of a declaration of war jointly
against its controlling country ; or as a generalisation of the war against
the patron.
\bparag If it is in limited intervention in a war and the enemy alliance
decides to fully involve the minor country in the war (this is done in
reaction).
\aparag Note that some specific alliances are dealt with different rules. That
is for instance the case of the alliance between \SPA and \hab, or of some
alliances forced by events.


\subsubsection{When a minor country is attacked}
\aparag A minor country that is attacked will call for some help according to
the rules explained here. Those calls are the first reactions resolved, in a
random order, before other kinds of reactions announced by Major powers.

\aparag[If the minor country is Neutral.]
The first power listed in the Appendix in the preference list, and that is not
at war against the \MIN, is called as an ally in the war.
\bparag The \MAJ can refuse any help, in which case it plays the minor country
but is by no means involved in this war and the \MIN stays "Neutral";
\bparag If it accepts, he makes a limited intervention (as if signing a
Alliance for Intervention) in the war, and the minor country is put in \AM of
the intervening power.
\bparag If the limited intervention ends before the war against the minor
country, this is a break of alliance: it costs 2 \STAB to the power breaking
the alliance, and the \MIN is put as "Neutral".

\aparag[If the minor country is in \RM or \SUB.]
\bparag The controlling power can refuse its help, in which case the
diplomatic status is broken, the \MIN is now "Neutral" and the previous
situation is applied, ignoring the \MAJ that just refused to help.
\bparag The controlling \MAJ may accept to do a limited intervention (as if
signing an Alliance for Intervention) in the war, and the minor country is put
in \AM of the intervening power.
\bparag If the limited intervention ends before the war against the minor
country, this is a break of alliance: it costs 2 \STAB to the power breaking
the alliance, and the \MIN is put as "Neutral".
\bparag The controlling \MAJ may accept to do a full intervention in the war,
that is to declare a war with a \CB against the attacking alliance; the minor
country is then put in \EG of the \MAJ.

\aparag[If the minor country is in \AM, \CE, \EG or \dipAT (in \ROTW).]
\bparag The controlling power can refuse its help, in which case the
diplomatic status is broken, the \MIN is now "Neutral" and the previous
situation is applied, ignoring the \MAJ that just refused to help.  If the
status was \EG or \dipAT, the \MAJ loses {\bf 1} \STAB (for the breaking of
this alliance).
\bparag The controlling \MAJ may accept to do a limited intervention (as if
signing an Alliance for Intervention) in the war, and the minor country is put
in \AM of the intervening power (or stays in \dipAT in the \ROTW).
\bparag If the limited intervention ends before the war against the minor
country, this is a break of alliance: it costs {\bf 2} \STAB to the power
breaking the alliance, and the \MIN is put as "Neutral".
\bparag The controlling \MAJ may accept to do a full intervention in the war,
that is to declare a war with a \CB against the attacking alliance; the minor
country is then put in \EG of the \MAJ (or stays in \dipAT in the \ROTW).

\aparag[If the minor country is a \VASSAL or in \ANNEXION.]
The declaration of war is only possible jointly against the controlling power,
or if a war against this power is already active.

\aparag Note that in the frequent case where the \MAJ is already at war when
one minor country it controls is declared war upon, the existence of the
existing war is sufficient to respond the alliance (and the minor is raised in
\EW if it had a lower status).


\subsubsection{When a minor country is declaring war.}
\aparag[If the minor country is Neutral.]
Excepted if an event says otherwise, the first power listed in the Annexe in
the preference list that is not at war against the \MIN, is called as an ally
in the war.
\bparag The \MAJ can refuse any help, in this case he will play the minor
power, but he is by no means involved in this war and the \MIN stays
"Neutral";
\bparag If the \MAJ accepts, he makes a limited intervention (as if signing an
Alliance for Intervention) in the war, and the minor country is put in \AM of
the intervening power.
\bparag If the limited intervention ends before the war against the minor
country, this is a break of alliance: it costs 2 \STAB to the power breaking
the alliance, and the \MIN is put as "Neutral".

\aparag[If the minor country is in \RM or \SUB.]
\bparag The controlling power can refuse its help, in which case the
diplomatic status is broken, the \MIN is now "Neutral" and the previous
situation is applied (ignoring the \MAJ that just declined).
\bparag The controlling \MAJ may accept to do a limited intervention (as if
signing an Alliance for Intervention) in the war, and the minor country is put
in \AM of the intervening power.
\bparag If the limited intervention ends before the war against the minor
country, this is a break of alliance: it costs 2 \STAB to the power breaking
the alliance, and the \MIN is put as "Neutral".

\aparag[If the minor country is in \MA, \EC, \EW or \dipAT (in ROTW).]
\bparag The controlling power can refuse its help, in which case the
diplomatic status is broken, the \MIN is now "Neutral" and the previous
situation is applied (ignoring the \MAJ that just declined intervention). If
the status was \EG or \dipAT, the \MAJ loses {\bf 1} \STAB (for the breaking
of this alliance).
\bparag The controlling \MAJ may accept to do a limited intervention (as if
signing an Alliance for Intervention) in the war, and the minor country is put
in \AM of the intervening power (or stays in \dipAT in the \ROTW).
\bparag If the limited intervention ends before the war against the minor
country, this is a break of alliance: it costs 2 \STAB to the power breaking
the alliance, and the \MIN is put as "Neutral".
\bparag The controlling \MAJ may accept to do a full intervention in the war,
that is to declare a war with a \CB against the attacking alliance; the minor
country is then put in \EG of the \MAJ (or stays in \dipAT in the \ROTW).

\aparag[If the minor country is a \VASSAL.]
The declaration of war by a \VASSAL gives a free \CB to the controlling power,
to be used now (in reaction), or at any following turn as long as the war
continues.

% Local Variables:
% fill-column: 78
% coding: utf-8-unix
% mode-require-final-newline: t
% mode: flyspell
% ispell-local-dictionary: "british"
% End:



\section{Conflicts against non-European}
\subsection{Generalities}
\aparag[Areas owned by minor countries.]
The Natives in areas owned by minor countries in the \ROTW, and the
cities, can not be attacked by a power if it is not at war against the
minor country. Exception: a reaction during the turn by Natives may
cause battles in such a province without involvement of the minor
country; in this case the power can continue to attack the Natives in
this province until the end of the turn, but not the cities.

\aparag[Wars in the \ROTW.]
An overseas war is sufficient to make a war against a country in \ROTW,
by definition of this kind of war.
\bparag Forces of a country in the \ROTW may never go on the European
map. They are deployed in any province they own (even if there is \COL
or \TP or enemy forces; in the last case, an immediate battle happens
before the first military round).
\bparag A country in the \ROTW always receives fixed reinforcements each
turn of limited or full war, as described in the Annexes. Those can only
raise their force to the basic forces of the country.
\bparag If a minor country is at peace during one whole turn, its basic
forces come back entirely.
\bparag The forces of a minor country are always in full supply in the
provinces of owned areas, and use those provinces as supply sources if
outside the area. A province where there is a \TP/\COL or a fort
controlled by an enemy can not be used as supply source to go outside
(but minor troops are still supplied within the province).
\bparag A country in the \ROTW uses all the Natives that are in the
areas that it controls. Natives are of moral ``conscript'' (exception:
Natives in \granderegion{Japan} are ``veteran'') and are added to
regular forces if there is any in the province. They never move.  They
will attack \TP and \COL in their provinces if they are at war against
the owning country.
\bparag Natives and regular forces of minor countries can do "Native
attack" in owned areas at the end of the turn to destroy \COL or \TP.
Additionally, regular forces can burn down controlled \TP as per normal
rules (Natives cannot).

\aparag[Areas with no minor countries.]
Some areas are less organised: no minor country owns them. A European
country can decide to attack Natives or cities in the corresponding
provinces without being at war, with no declaration beforehand.
\bparag If Natives are attacked in a given province, they will continue
to react (as defined afterwards) against the aggressor until the end of
the turn.
\bparag To assault or besiege a city, a power has first to attack the
Natives of the province (or they have to be already active).
\begin{designnote}
  By ``less organised'', we do not mean, of course, that areas such as
  South-East Asia or Indonesia were lacking states. Dai Viet,
  Ayutthaya, the sultanate of Borneo and other countries clearly
  exists. However, these countries were of a rather local importance and
  their relative strength and tolerance to the Europeans is directly
  represented by the values of the corresponding area. \ROTW countries
  correspond to large empires such as China or the Mogols, with a large
  territorial base or a powerful army.
\end{designnote}

\subsection{Reactions by countries in the \ROTW}
\aparag At the end of the phase of event, a test of reaction is made in
a country from the \ROTW where one of the conditions is met:
\bparag there is a military force in one of its province (excepted if
this force is in a foreign \COL settled in the province, or if allowed
by a \dipFR or \dipAT);
\bparag there is a European \COL or \TP that is not allowed by
diplomatic status (or a special rule).

\aparag The test is 1d10, compared to the Activation level of the
country. If it is strictly lower, the minor country declares an Overseas
war against any and all powers that satisfy one of the previous
conditions.
\bparag List of the Activation levels:
\begin{modlist}
\item[9/3] \pays{Moghol} before/after \eventref{pVI:Last Great Mughals}
\item[9/11] \pays{Chine} and \pays{Japon} before/after
  \subeventrefshort{pIII:CCA:Closure China} and
  \subeventrefshort{pIV:JCA:Closure Japan}, except in newly conquered areas
  (6)
\item[9] \pays{gujarat}
\item[8] \pays{Iroquois}, \pays{Soudan}
\item[4] \pays{Inca}, \pays{Azteque}, \pays{Vijayanagar}
\item[6] All others: \pays{Siberie}, \pays{Oman}, \pays{Aden},
  \pays{Mysore}, \pays{Hyderabad}, \pays{afghans}, \pays{ormus}
\end{modlist}

\subsection{Reactions by Natives during the rounds}
\aparag At the end of each military round, before the sieges, a test of
reaction is made in every province in the \ROTW where there is a
European military force that is
\bparag Neither in a \COL of a European power;
\bparag Nor allowed by some \dipFR or \dipAT in this province by a minor
country owning the area.
\bparag When a land stack moves also through a province where none of
the two previous conditions hold, a test of reaction is also made before
it leaves the province.
\bparag Finally some attempts of putting \TP or \COL in a province may
cause an automatic reaction of the Natives,
see~\ruleref{chAdministration:Colony:Critical failure} 
and~\ruleref{chAdministration:TP:Critical failure}.

\aparag The test of reaction is resolved by rolling 1d10. If it is
strictly inferior to the Tolerance level in the area, the Natives react.
When the Tolerance is "-", no reaction can happen.

\aparag[Effect of a reaction.]
\bparag The reaction is an attack of the Natives against the units that
caused the reaction, and all units of the same country in the province
(not area).
\bparag The attack is revolved immediately (as an interception if it is
caused by a movement, or a regular battle if it is at the end of the
round or due to botched \TP/\COL action).
\bparag
The reaction last until the end of the turn and the Natives will attack
any other force of the power causing the reaction that is in the
province. Only one battle is possible each round (at the time of the
first interception by reaction, or at the end of the round). Natives
will then attack \COL/\TP owned by the power at the end of the turn.
Note that if A has activated the Natives against him, and controls a
fort of fortress of the side B who has not, the Natives would attack A
and besiege its forces (attrition if A is withdrawn in the fortress) but
would not attack a \COL/\TP owned by B (even if controlled by A) at the
end of the turn.
\bparag If units of another player enter the province later in the turn, 
they can also provoke a reaction of the Natives against them. 


% \section{Conflicts against non-European}

% \section{TODO}
% PB 08/2007 - done - c'était deja rédigé comme il faut dans les règlesn en
% exceptant les 2 changements sur l'entretien par le Majeur et la limite en
% distance pour les vassaux.
%
% Précisions sur la logistique des mineurs en guerre données par Pierre sur le
% forum.

% Voici les clarifications attendues :

% \aparag tous les pays mineurs sauf annexés : si en paix pendant un tour
% complet, ils reconstruisent leur force de base.

% \aparag tous les pays mineurs sauf vassaux et annexés, en "limited war" :
% \bparag ils maintiennent au plus leur force de base en début de tour (non
% dépassable, jamais)
% \bparag ils reçoivent en renfort ce qui est mis en renforts de base (+ 1D
% (LD ou ND) en EC ou EW)

% \aparag tous les pays mineurs sauf vassaux et annexés, en "full war" :
% \bparag bah il sont forcément Neutre ou en EW si on lit attentivement (et si
% c'est bien rédigé, je n'ai pas vérifié) les rgles... La suite est donc pour
% un pays en EW en "Full war"
% \bparag ils maintiennent leur force de base gratuitement en début de tour ;
% le MAJ peut payer l'entretien du complément [note aux vieux d'EU8 : ce sont
% est un changement/clarification nouvelle, revenant aux règles originelles]
% \bparag ils reà§oivent des renforts par jet de dé sur la table des renforts.

% \aparag les vassaux :
% \bparag le MAJ peut tjs entretenir des forteresses en + de la force de base
% \bparag en limited war: entretien gratuit de leur force de base; entretien
% au-delà  de la force de base interdit (sauf forteresses); pas de renforts
% gratuits mais renforts payables par le MAJ jusqu'à  renforts de base +2D
% \bparag en full war: même chose sauf que l'entretien peut dépasser (en
% payant) la force de base ; renfort par jet de dé

% \aparag les annexés : les forces militaires du pays ne sont pas
% utilisables. Le MAJ peut construire ses propres troupes (comme en province
% non nationale, au coût double) et ses forteresses.

% \aparag La nouvelle règle qui apparaît sous vos yeux : on ne peut mettre un
% vassal en guerre que si il y a une province ennemie déjà  en guerre à  au
% plus 12 MP ou 4 zones de mer d'une de ses provinces.

% Local Variables:
% fill-column: 78
% coding: utf-8-unix
% mode-require-final-newline: t
% mode: flyspell
% ispell-local-dictionary: "british"
% End:
