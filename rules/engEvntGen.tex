% -*- mode: LaTeX; -*-

\definechapterbackground{Economical, diplomatic and revolts
  events}{economicalevents}
% Suggestion: one of Turner's storm (bad weather).  See
% http://www.william-turner.org/
\chapter{Economical, diplomatic and revolts events}
\label{chapter:Events:Eco}

% -*- mode: LaTeX; -*-

\section{Event Table of economical random events}

\begin{eventstable}[Random economical events]
  \centering{\graytabular\tabcolsep=4pt
    \begin{tabular}{|c|cccccccccc|}%
      \hline\ghline%
      {1\up{st}\rlap{>}}%
      &  1& 2& 3& 4& 5& 6& 7& 8& 9& 0\\\hline\ghline%
      1&29& 9&17&38&22& 7& 6&18& 4&45\\\ghline%
      2& 2&43&28&12&36&16&49&24& 3&15\\\ghline%
      3&42&33&18&45& 4&14&38& 7&46&10\\\ghline%
      4&22& 6&44&19&32&37&21& 7&40& 9\\\ghline%
      5&16&34& 8&24&13& 2&38&28&36&45\\\ghline%
      6&44&10&27&15&20&47&18& 6&14&30\\\ghline%
      7&23&38&17& 9& 5&43&11&41&26& 4\\\ghline%
      8& 8&35& 2&31&39&16&20&45&13&16\\\ghline%
      9&24& 7&19&14&12& 4& 5&25&35&48\\\ghline%
      0&38&17&37& 8&11& 9& 7&16&23& 1\\\hline%
    \end{tabular}}
\end{eventstable}
\eventssummary{%
  eco:Crisis of madness|,%
  eco:Excellent Minister|,%
  eco:Serious sickness|,%
  eco:Agricultural Crisis|,%
  eco:Naval losses|,%
  eco:Looting and insecurity|,%
  eco:Fiscal evasion|,%
  eco:Corruption|,%
  eco:Technological advance|,%
  eco:Mine Discovery|,%
  eco:Wave of obscurantism|,%
  eco:Piracy|,%
  eco:Development of warships|,%
}

\eventssummary{%
  eco:Military leader|,%
  eco:Drought|,%
  eco:Exceptional year|,%
  eco:Sales of honorary titles|,%
  eco:Epidemics|,%
  eco:Rush Colonists|,%
  eco:Refugees|,%
  eco:Gift to the State|,%
  eco:Scandal at the court|,%
  eco:Plots at the court|,%
  eco:Poor weather|,%
  eco:Death Heir|,%
  eco:Mine Depletion|,%
  eco:New ally|,%
  eco:Defection of an ally|,%
  eco:Desertions|,%
  eco:Death of a military leader|,%
  eco:Dynastic inheritance|,%
}\eventssummary{%
  eco:Inflation|,%
  eco:Offer of alliance|,%
  eco:Independence of a vassal|,%
  eco:Enthusiasm for the Army|,%
  eco:Renewal of popularity|,%
  eco:Enthusiasm for the Navy|,%
  eco:Agricultural technique development|,%
  eco:Reorganisation of the army or the fleet|,%
  eco:Conquistador|,%
  eco:Explorer|,%
  eco:Governor|,%
  eco:Diplomatic Preeminence|,%
  eco:Cultural expansion|,%
  eco:Deflation|,%
  eco:Economic Crisis|,%
  eco:Economic Boom|,%
  eco:Rectification|,%
  eco:Treachery|,%
} \newpage\startevents




\section{Description of Economical Events}



\event{eco:Crisis of madness}{E-1}{Crisis of madness}{1}{Orig}

Reduce all values of the monarch's characteristics by half for this turn
(rounded down). Modify next-turn survival die-roll by \bonus{+1}.



\event{eco:Excellent Minister}{E-2}{Excellent ministers}{3}{PBmod}

\phevnt
\aparag A Minister is appointed per \ref{chEvents:Excellent Ministers}. His
characteristics as ruler are rolled for the three values by 1d10 modified: a
die-roll of 1 becomes 7, of 2 becomes 8, 10 becomes 9.  Another die roll sets
the length of the Ministry:

\GTtable{excellentministres}

\aparag The office of the Minister include the current turn, and ends just
before the ``economical events'' segment of the events phase following the
last full turn of office.

\aparag A value of the Minister is used only if it is strictly superior to the
monarch's own characteristic.

\aparag If the Monarch dies when the Minister is still in office, a malus of
\bonus{-2} is applied to the characteristics determination die-rolls for the
monarch's successor, but only for a characteristic that was increased due to
the Minister by at least 2 above the Monarch value.



\event{eco:Serious sickness}{E-3}{Serious sickness}{1}{Orig}

% (JCD) What about monarchs that do no survival roll. Immune sudden death?
Reduce all characteristics of the monarch by 3 for this turn only, 1 being the
minimum value. In addition, roll a die. If the result is 10, the monarch
deceases immediately. Else, modify next-turn survival die-roll by \bonus{+1}.

If the current monarch did benefit from \ref{chEvents:Excellent Ministers},
the characteristics are only reduced by 1.

The monarch cannot lead armies or fleets during the turn
% (Jym) adding:
except if he must do so due to a political event.



\event{eco:Agricultural Crisis}{E-4}{Agricultural crisis}{4}{Orig}

The country has seen real trouble in crops and farming. The loss is of 50\%
(lowered by 10\% per unit of \RES{Cereals} \MNU already owned by the country)
of its income of provinces this turn (\lignebudget{Provinces income}). The
loss is registered in \lignebudget{Event}.

% (JCD) Should be added to the industrial income instead of RT?
Other countries that possess \RES{Cereals} \MNU gain immediately 10\ducats per
unit in their \RT, to be added from \lignebudget{RT at start of turn} to
\lignebudget{RT after Events}.



\event{eco:Naval losses}{E-5}{Naval losses}{2}{Orig}

Fires, storms and disasters spread at sea. Roll 2d10, and add \bonus{+2} if
the \MAJ has at least 3 \FLEET counters deployed at that time, or subtract
\bonus{-2} if it has only one (or none).  The number of \ND lost is given by
the result:

\centerline{\begin{tabular}{*{5}{c}} \textlessequal 4& 5--10& 11--15 & 16--19
    & \textgreatequal20\\\hline 0 & 1 & 2 & 3 & 4
  \end{tabular}}

The \ND can come from anywhere. \NGD count for half a loss only; \NTD can be
lost only if there are no warships or galleys left.



\event{eco:Looting and insecurity}{E-6}{Looting and insecurity}{3}{JCMod}

The country loses 10\% (rounded up) of its income of provinces this turn
(\lignebudget{Provinces income}). The loss is registered in
\lignebudget{Event}.

Place a \PIRATE\facemoins in the player's \CTZ (if any); in \stz{Baltique} if
the player has a port on this \STZ (and no \CTZ); in \stz{Adriatique} if the
player has a port on this \STZ (and no \CTZ). There may be no \PIRATE if there
are no such ports.



\event{eco:Fiscal evasion}{E-7}{Fiscal evasion}{5}{Orig}

\phevnt
Reduce the Royal treasury by 20\% of its absolute value (min. is 20\ducats)
this turn (from \lignebudget{RT at start of turn} to \lignebudget{RT after
  Events}). Furthermore, if \TUR receives this event, he has to check for
Pashas' corruption.


\digression[eco:Fiscal:Pashas Corruption]{Corruption of Pashas}

\phevnt
\aparag Two \xnameref{chSpecific:Turkey:Pashas} become corrupted (turn the
counters on their corrupted side). These Pashas are chosen by The Sole
defender of the Catholic Faith (or \MAJHAB is there is none).  Those pashas
must be in owned Turkish provinces; if none are available, displace the newly
corrupted pashas in any province (except the capital).



\event{eco:Corruption}{E-8}{Corruption}{3}{Orig}

\phadm
All costs of purchase double this turn (reinforcements and campaigns). Costs
of maintenance increase by 10\% (rounded up). In addition, \TUR suffers the
effects described in \ref{eco:Fiscal:Pashas Corruption}.



\event{eco:Technological advance}{E-9}{Technological advance}{4}{Orig}

The player can move one of his two technology marker (naval or land) a number
of boxes forward on the technology track determined by the roll of a die
(choice of the technology must be made before rolling the die):

\centerline{\begin{tabular}{*{3}{c}} \textlessequal 1--5 & 6--8 &
    9--10\\\hline 1 box & 2 boxes & 3 boxes
  \end{tabular}}



\event{eco:Mine Discovery}{E-10}{Discovery of mines}{2}{JCMod}

\phevnt
\aparag Place a \countermark{Gold Mine} counter in one national province of
the player (still controlled) in mountain terrain (or non-clear terrain if
none available, or clear terrain as a last resort), where there is not already
such a counter, and provided the country (not the player) did not benefit from
this event two times.
\bparag If the country had already benefited from this event two times, test
for \ref{eco:Depletion:Depletion America} instead.

If no controlled terrain is available, re-roll.



\event{eco:Wave of obscurantism}{E-11}{Wave of obscurantism}{2}{Orig}

Reduce the \STAB by {\bf 1} level if player is Protestant, and {\bf 2} levels
in all other cases.



\event{eco:Piracy}{E-12}{Pirates}{2}{JymMod}

\phevnt
\aparag This event is only resolved at during the economic situation segment
of the event phase.
\aparag Count the number of piracy events that must occur this turn, including
the one that may arise via the economic situation roll.
\bparag If there are two or more, then the target is ``Everywhere''.
\bparag If there is only one, and the target is not precised, roll one die:
1--5: America, 6--10: Asia.

\aparag For each \STZ in the target, in the order listed below, roll one die
and place a \PIRATE\facemoins if it is higher than the piracy level of the
\STZ. See~\ruleref{chEvents:Piracy Level} and following for the details of
piracy placements.

\aparag List of targeted \STZ and order of test:
\bparag Everywhere: \seazoneCaraibes, \seazone{Atlantique W}, \seazoneIndien,
\seazoneOman, \seazoneGuinee, \seazoneRecife, \seazonePerou, \seazoneFormose,
\seazonePatagonie, \seazoneTempetes, \stz{Canarias}.
\bparag America: \seazoneCaraibes, \seazone{Atlantique W}, \seazoneGuinee,
\seazoneRecife, \seazonePatagonie, \seazoneTempetes, \stz{Canarias}.
\bparag Asia: \seazoneIndien, \seazoneOman, \seazonePerou, \seazoneFormose,
\stz{Canarias}.



\event{eco:Development of warships}{E-13}{Development of warships}{2}{Orig}

The player advances his naval technology by 1 box.
% (JCD) Relation with \RES{Wood}? NO!



\event{eco:Military leader}{E-14}{Military leader}{3}{Orig}

Roll one die. If the result is even, draw a general, else draw an admiral. The
leader will be drawn from the anonymous pool of the player, and will not be
included in the minimum leaders limit for the period that the leader is
entitled to.

The leader is available for 1 turn if the result is between 1 and 5, 2 turns
(current and following) if it is between 6 and 10.



\event{eco:Drought}{E-15}{Drought}{2}{Orig}

The country loses 30\% (rounded up) of its income of provinces this turn
(\lignebudget{Provinces income}). The loss is registered in
\lignebudget{Event}.



\event{eco:Exceptional year}{E-16}{Exceptional year}{5}{Orig}

The country gains 10\% (rounded up) of its income this turn
(\lignebudget{Income}). The gain is registered in \lignebudget{Events}.



\event{eco:Sales of honorary titles}{E-17}{Sales of honorary titles}{3}{Orig}

The Major Power may opt to sell honorary titles. If it chooses so, roll 1d100.
The result gives the product of these sales in \xducats, added immediately to
\lignebudget{RT at start of turn} in \lignebudget{RT after Events}.  Then the
minimum number of generals of the power is lowered by one this turn (only).
If may opt to have none of these effects (before rolling the dies).



\event{eco:Epidemics}{E-18}{Epidemics}{3}{Orig}

The country loses 20\% (rounded up) of its income this turn
(\lignebudget{Incomes}). The loss is registered in \lignebudget{Events}.



\event{eco:Rush Colonists}{E-19}{Rush of colonists}{3}{JymMod}

If the country has no \COLaction or \TPaction, it may elect to ignore this
event and re-roll another one (to be decided immediately).

This event gives a bonus of \bonus{+3} to the die-roll of \COLaction, as well
as a supplementary and free \COLaction with small investment (30\ducats),
usable this turn or any other turn of the current period (lost if not used
before the end of the current period). Moreover, the country may ignore
restrictions of~\ref{chExpenses:Pioneering} for this turn.

If this is not period \period{I} also apply~\ref{eco:Rush:Minor colony}.


\digression[eco:Rush:Minor colony]{Minor country colonisation}

If this is not period \period{I}, roll on the following table; subtract {\bf
  3} in periods \period{II} and \period{III} and add {\bf 3} in periods
\period{VI} and \period{VII}.

\begin{modlist}[3em]
\item[-2] Destruction of a Minor establishment.
\item[-1] Creation of a Minor establishment in \continentBresil.
\item[0] Creation of a Minor establishment in
  \granderegionEcuador/\granderegionYucatan/\granderegionPanama.
\item[1--2] Creation of a Minor establishment in \continentCaraibes.
\item[3--4] Loss one side of a Minor establishment.
\item[5] Creation of a Pirate Haven in \continentCaraibes.
\item[6--7] Creation of a Minor establishment in a coastal province in the
  American zoom.
\item[8] Increase one Minor establishment.
\item[9] Creation of a Minor establishment in a coastal province in
  \continentIndia.
\item[10] Creation of a Pirate Haven in \granderegionMadagascar
\item[11] Creation of a Minor establishment in a coastal province in
  \continentIndia.
\item[12--13] Creation of a Minor establishment in \continentCaraibes.
\end{modlist}

\aparag[Creation of a Minor establishment.] Select one empty province at
random within the specified ones and put a Minor establishment \Facemoins in
it.
\bparag If there are no empty provinces in the specified ones or there are no
unused Minor establishment, turn this into a \terme{Increase of one Minor
  establishment} instead.

\aparag[Creation of a Pirate haven.] If one already exists in the specified
provinces, it is turned on level 2 (nothing happens if it is already level 2).
\bparag If there is no Pirate haven in the specified provinces, select an
empty one at random and put a Pirate haven of level 1 in it.
\bparag For \granderegionMadagascar, do not select the province at random. Use
\province{Madagascar N} if empty and \province{Madagascar S} otherwise.

\aparag[Destruction of a Minor establishment.] Select a Minor establishment at
random and remove it from the map.

\aparag[Loss of one side.] Select a Minor establishment at random.
\bparag If it is \Facemoins, remove it from the map.
\bparag If it is \Faceplus, turn it \Facemoins and select one of its exploited
resources at random which is no longer exploited.

\aparag[Increase of one Minor establishment.] Select one Minor establishment
\Facemoins at random and turn it \Faceplus.

\aparag[Creation/Increase of establishments.] Whenever a new side of Minor
establishment is created:
\bparag If there is at least one unexploited resource in the \Area, it
exploits one at random.
\bparag Otherwise, it exploits one of the existing resource at random,
stealing it from whoever exploits it.



\event{eco:Refugees}{E-20}{Refugees}{2}{JCMod}

If the country has no \COLaction or \TPaction, it may elect to ignore this
event and re-roll another one (to be decided immediately).

The player receives a free of charge strong investment that can be used for a
\TFI (but cannot be cumulative with another investment on the same \STZ/\CTZ).

This also gives in addition the same effect as \ref{eco:Rush Colonists}, but
with a bonus of \bonus{+2} only.



\event{eco:Gift to the State}{E-21}{Gift to the State}{1}{Orig}

The people make a gift of 1d100\ducats added immediately to \lignebudget{RT at
  start of turn} in \lignebudget{RT after Events}.



\event{eco:Scandal at the court}{E-22}{Scandal at the court}{2}{JCMod}

The player's monarch's Diplomatic value is reduced by 3 for this turn (to a
minimum of 1). The player also immediately loses 50\ducats, taken from
\lignebudget{RT at start of turn} into \lignebudget{RT after Events}.



\event{eco:Plots at the court}{E-23}{Plots at the court}{2}{Orig}

The player's monarch's Diplomatic value is reduced to 1 for this turn.  In
addition, he will add a modifier of \bonus{+2} to next turn's survival
die-roll for his monarch.



\event{eco:Poor weather}{E-24}{Poor weather}{3}{JymMod}

\phmil
\aparag This turn, add \bonus{+2} to each season continuation die roll. All
Winter round will be in bad weather.
\aparag[Frozen Sea] Moreover, if a Winter round happen after a die roll of 1
(before modifications), \seazoneOresund is frozen. No fleet can go through, in
or out of it (fleets in it at the beginning of the round stay there but suffer
no damage). Armies can cross it (it's an unfriendly rough terrain with no
effect on combat) but not stop in it. No battle or interception of any kind
may happen here. If retreat into \seazoneOresund is forced after a land
battle, the stack retreats one province further into solid ground but has a
malus of \bonus{+2} to it retreat die roll.



\event{eco:Death Heir}{E-25}{Death of the heir to the throne}{1}{Orig}

The player will receive a -1 malus to his die-roll for each one of the future
characteristics of his next monarch. This event may be drawn several times but
the malus will apply only once on the next monarch. This event has no effect
if the next monarch is a named monarch, including one whose characteristics
are not fixed but must be rolled.



\event{eco:Mine Depletion}{E-26}{Depletion of a mine}{1}{Orig}

Place a marker \countermark{Exhausted Mine} on a mine currently exploited by
the player (either in Europe or in the \ROTW), drawn at random, and check for
\ref{eco:Depletion:Depletion America}. If no mine qualifies, just do the
check.


\digression[eco:Depletion:Depletion America]{Depletion of mines in America}

\phevnt
\aparag Each time this is called for, all exploited mines in \continentAmerica
will be tested for depletion. This test is made at most once each turn.
\bparag The mines are tested in the following order: the Potosi mine (value
50), the Tenochtitlan mine (value 40), then the mines of the player exploiting
the largest number of mines in \continentAmerica (in an order chosen by the
player itself), then the next player, and so on.
\bparag A mine is depleted if a die-roll gives 1, or 1 or 2 in period V or
later.
% (JCD) Used to be turn 36, now 35
\aparag Only one mine per turn may be depleted this way. As soon as one as
been depleted this way, there is no further need to check the others.



\event{eco:New ally}{E-27}{New ally}{1}{Orig}

The player receives a modifier of \bonus{+3} in diplomacy on a minor of his
choice, valid for this turn. The choice of the minor has to be made
immediately and secretly. It will be revealed during the next Diplomacy phase.



\event{eco:Defection of an ally}{E-28}{Defection of an ally}{2}{PBMod}

One country in \VASSAL position that is not a special vassal (i.e. on which
diplomacy is possible) of the power, if any (chosen at random), is lowered by
3 boxes on the Diplomatic track.  If none qualifies, another country
determined at random among all the countries on Diplomatic track of the power
is lowered by 2 boxes.



\event{eco:Desertions}{E-29}{Desertions}{1}{Orig}

Desertions occur in the army. Roll 2d10, and add \bonus{+2} if the \MAJ has at
least 4 \ARMY counters deployed at that time, or subtract \bonus{-2} if it has
only one.  The number of \LD lost is given by the result:

\centerline{\begin{tabular}{*{5}{c}} \textlessequal 4& 5--10& 11--15 & 16--19
    & \textgreatequal20\\\hline 0 & 1 & 2 & 3 & 4
  \end{tabular}}



\event{eco:Death of a military leader}{E-30}{Death of a military
  leader}{1}{Orig}

Draw one leader at random in all military leaders of the player on the map.
The leader is removed from the game if it is a named one, returned to the pool
if it is an \anonyme one. The period limit is diminished by one for the turn.



\event{eco:Dynastic inheritance}{E-31}{Dynastic inheritance}{1}{Orig}

The player receives a \bonus{+5} bonus in his next diplomacy phase for a minor
country that may become a vassal. This minor must currently be located in the
\RM box or above on the player's diplomatic track. This minor has to be
nearest to the national territory of the player in term of number of provinces
(in case of tie, leave it to the player's choice).



\event{eco:Inflation}{E-32}{Inflation}{1}{JymMod}

Increase the level of inflation by 1 (without exceeding the maximum level). At
most one event among \xnameref{eco:Inflation} and \xnameref{eco:Deflation} can
take place in a single turn (treat as no event if a second one is rolled).



\event{eco:Offer of alliance}{E-33}{Offer of alliance}{1}{Orig}

The player receives a \bonus{+3} bonus in diplomacy to his die-roll for a
minor of his choice (to be decided immediately).



\event{eco:Independence of a vassal}{E-34}{Independence of a vassal}{1}{Orig}

A minor vassal that is not a special vassal (i.e. on which diplomacy is
possible) breaks its vassalisation and remains only an ally. The player has a
temporary \CB against this minor. Move the marker of the minor from the
\VASSAL box to the \MA box.



\event{eco:Enthusiasm for the Army}{E-35}{Enthusiasm for the Army}{2}{Orig}

The player may either receive 2 \LD free of charge, or increase his land
technology by 1 box.



\event{eco:Renewal of popularity}{E-36}{Renewal of popularity}{2}{Orig}

The player receives 20\ducats in his royal treasury (added immediately to
\lignebudget{RT at start of turn} in \lignebudget{RT after Events}). All the
following administrative operations: \TFI, \TPaction, \COLaction, \MNU
placement attempts, \DTI/\FTI improvement also receive an exceptional bonus of
\bonus{+2} to the die-roll for this turn.

On the other hand, a malus of \bonus{-10} to the die-roll is applied on the
\terme{Exceptional taxes raising} operation.



\event{eco:Enthusiasm for the Navy}{E-37}{Enthusiasm for the Navy}{2}{Orig}

The player may either receive 2 \NWD (or 4 \NGD) free of charge, or increase
his naval technology by 1 box.



\event{eco:Agricultural technique development}{E-38}{Agricultural technique
  development}{5}{Orig}

Increase the country's income by 2\ducats per controlled and owned province
(i.e. not including occupied, looted, controlled but still belonging to the
enemy, belonging to a vassal provinces) for this turn only. The gain is
registered in \lignebudget{Event}. In addition, for this turn only, the
country receives a bonus of \bonus{+3} to the die-roll for the
\terme{improvement of \DTI}, as well as all attempts to create a \RES{Cereals}
or \RES{Wine} manufacture.



\event{eco:Reorganisation of the army or the fleet}{E-39}{Reorganisation of
  the army or the fleet}{1}{Orig}

Gives a bonus of \bonus{+2} to the die-roll of either land or naval technology
improvement (the choice must be written down immediately). Also gives a 50\%
bonus discount to the unit reorganisation due to a new technology being
discovered.



\event{eco:Conquistador}{E-40}{Conquistador}{1}{Orig}

If the country has no \anonyme\LeaderC, it may elect to ignore this event and
re-roll another one (to be decided immediately).

The player receives a conquistador among the \anonyme \LeaderC markers still
available. It remains in play for this turn only.



\event{eco:Explorer}{E-41}{Explorer}{1}{Orig}

If the country has no \anonyme\LeaderE, it may elect to ignore this event and
re-roll another one (to be decided immediately).

The player receives an explorer among the \anonyme \LeaderE markers still
available. It remains in play for this turn only.



\event{eco:Governor}{E-42}{Governor}{1}{JCMod}

If the country has no \anonyme\LeaderGov, it may elect to ignore this event
and re-roll another one (to be decided immediately).

The player receives a governor among the \anonyme \LeaderG markers still
available, to be placed in a \TP or a \COL of the player.  It remains in play
for this turn only.



\event{eco:Diplomatic Preeminence}{E-43}{Diplomatic preeminence}{2}{Orig}

Gives the player a bonus of \bonus{+1} to the die-roll to all his diplomatics
actions on minors (either European or \ROTW), and a bonus of \bonus{+1} column
in his favour for all of his attempts of \TP and \COLaction for this turn
only.



\event{eco:Cultural expansion}{E-44}{Cultural expansion}{2}{Orig}

This gives a bonus of 20\ducats to any subsidies obtained by a minor vassal
reaching the \SUB diplomatic level. Any subsidies will yield at least
20\ducats, whatever the modifiers.  In addition, it has the same effect as
\ref{eco:Diplomatic Preeminence} above.



\event{eco:Deflation}{E-45}{Deflation}{4}{Orig}

Reduce the level of inflation by 1 (without exceeding the minimum level). At
most one event among \xnameref{eco:Inflation} and \xnameref{eco:Deflation} can
take place in a single turn (treat as no event if a second one is rolled).



\event{eco:Economic Crisis}{E-46}{Economic crisis}{1}{JCMod}

Demand for exotic resources decreases in Europe and prices fall. Adjust prices
as follows (without exceeding any normal limits, and only for already
available resources):
\begin{itemize}
\item \RES{Fish}, \RES{Salt}: no modification
\item \RES{Sugar}, \RES{Cotton}, \RES{Furs}: \bonus{-1} box
\item \RES{Slaves}, \RES{Spices}, \RES{Products of America}: \bonus{-2} boxes
\item \RES{Products of Orient}, \RES{Silk}: \bonus{-3} boxes
\end{itemize}
At most one event among \xnameref{eco:Economic Crisis} and
\xnameref{eco:Economic Boom} will take effect this turn. Re-roll if one was
already used.



\event{eco:Economic Boom}{E-47}{Economic boom}{1}{JCMod}

Demand for exotic resources increases in Europe and prices rise. Adjust prices
as follows (without exceeding any normal limits, and only for already
available resources):
\begin{itemize}
\item \RES{Fish}, \RES{Salt}: no modification
\item \RES{Sugar}, \RES{Cotton}, \RES{Furs}: \bonus{+1} box
\item \RES{Slaves}, \RES{Spices}, \RES{Products of America}: \bonus{+2} boxes
\item \RES{Products of Orient}, \RES{Silk}: \bonus{+3} boxes
\end{itemize}

At most one event among \xnameref{eco:Economic Crisis} and
\xnameref{eco:Economic Boom} will take effect this turn. Re-roll if one was
already used.



\event{eco:Rectification}{E-48}{Rectification}{1}{PBMod}

The monarch yields to pressures yielding to straighten the domestic and
foreign situation. The player can choose one option exactly among the three
following bonuses:
\begin{itemize}
\item Pay without overcosts all his land forces up to the triple of the normal
  limit.
\item Increase his construction limit for ships by 50\% (rounded up).
\item Obtain a bonus of \bonus{+5} to his die-roll concerning the action of
  \terme{improvement of Stability}.
\item Refund for free National Loans up to 200\ducats.
\end{itemize}
Choice must be written down immediately to be valid.



\event{eco:Treachery}{E-49}{Treachery}{1}{PBMod}

The player benefits from a treachery against one of his opponents with whom he
is \emph{already} at war (either a player or a minor country). The player can
choose one option immediately among the three following bonuses:
\begin{itemize}
\item Capture immediately an enemy fortress that he currently besieges, or
  obtain a one time bonus of \bonus{+4} to a \terme{siegeworks} action
  die-roll in the current turn (if he establishes a siege this turn).
\item Move himself one land stack of his opponent one time during his
  opponent's movement phase this turn, instead of his opponent. The player
  will pick the exact round. However, he cannot make this stack attack any
  units except a stack commanded by him, nor can he exceed 5 MP on land, or
  make a naval move with a modifier higher than +8 for attrition on sea.
\item Obtain a bonus of \bonus{+5} to one his diplomatic operations against a
  minor country whose marker is on his opponents diplomatic track, this turn
  only (the choice is announced along the diplomatic actions).
\end{itemize}

% (Pierre) Some choices have problems especially diplomacy during the war;
% convert the condition on naval move.

\stopevents

% Local Variables:
% fill-column: 78
% coding: utf-8-unix
% mode-require-final-newline: t
% mode: flyspell
% ispell-local-dictionary: "british"
% End:

% LocalWords: cccccccccc eco PBmod excellentministres malus JCMod Baltique
% LocalWords: Adriatique Atlantique Canarias JymMod turn's Potosi PBMod JCD
% LocalWords: vassalisation siegeworks overcosts Jym


\clearpage

% \section{Revolt and Diplomatic events tables}




\section{Diplomatic event tables}

\begin{tablehere}\centering
  \begin{tabular}{l|l}
    Roll & Result\\\hline
    1,4,7 & Also test for \xnameref{chEvents:diplomacy:uprising}\\
    1--3 & Catholics \Xcatholique\ (Christians \Xcatholique\Xorthodoxe\ %
    before the Reform)\\
    4--6 & Protestants \Xprotestant\ (Christians \Xcatholique\Xorthodoxe\ %
    before the Reform)\\
    7--9 & Muslims \Xsunnite\Xchiite\ \\
    10 & Other \Xautrereligion\ and a minor will possibly declare a war.
  \end{tabular}
  \caption{Troubled Religion table}\label{table:diplomatic event religion}
\end{tablehere}\par

\begin{tablehere}
  % No aliases, just the internal codes of countries
  \def\paysfid#1{\pays{#1}~(\theminorfid{#1})} \def\labelitemi{10.}
  \begin{enumerate}
  \item {\bf Northern Italy:} %
    1-\paysfid{genes} %
    2-\paysfid{montferrat} %
    3-\paysfid{modene} %
    4-\paysfid{lucca} %
    5-\paysfid{milan}.\\ %
    {\bf Balkans:} %
    6-\paysfid{hongrie} %
    7-\paysfid{moldavie} %
    8-\paysfid{valachie} %
    9-\paysfid{mazovie} %
    10-\paysfid{transylvanie}. %
  \item {\bf Southern Italy:} %
    1-\paysfid{papaute} %
    2-\paysfid{chevaliers} %
    3-\paysfid{toscane} %
    4-\paysfid{parme} %
    5-\paysfid{venise} %
    6-\paysfid{corse}.\\ %
    {\bf Middle East:} %
    7-\paysfid{arabie} %
    8-\paysfid{irak} %
    9-\paysfid{georgie} %
    10-\paysfid{mamelouks} %
    11-\paysfid{damas}. %
  \item {\bf Spanish road:} %
    1-\paysfid{suisse} %
    2-\paysfid{wurtemberg} %
    3-\paysfid{savoie} %
    4-\paysfid{treves} %
    5-\paysfid{cologne} %
    6-\paysfid{lorraine} %
    7-\paysfid{mayence} %
    8-\paysfid{liege}. %
  \item {\bf Northern \HRE:} %
    1-\paysfid{hollande} %
    2-\paysfid{hanovre} %
    3-\paysfid{hesse} %
    4-\paysfid{palatinat} %
    5-\paysfid{berg} %
    6-\paysfid{oldenburg}.\\ %
    {\bf America:} %
    7-\paysfid{iroquois} %
    8-\paysfid{inca} %
    9-\paysfid{azteque}. %
  \item {\bf Southern \HRE:} %
    1-\paysfid{baviere} %
    2-\paysfid{wurtemberg} %
    3-\paysfid{alsace} %
    4-\paysfid{bade} %
    5-\paysfid{thuringe} %
    6-\paysfid{habsbourg}. %
  \item {\bf Eastern \HRE:} %
    1-\paysfid{boheme} %
    2-\paysfid{brandebourg} %
    3-\paysfid{saxe} %
    4-\paysfid{brunswick} %
    5-\paysfid{pologne} %
    6-\paysfid{Vlithuanie} %
    7-\paysfid{Vpommeranie}.\\ %
    {\bf Asia:} %
    8-\paysfid{chine} %
    9-\paysfid{japon}. %
  \item {\bf Baltic shores:} %
    1-\paysfid{teutoniques1} %
    2-\paysfid{hanse} %
    3-\paysfid{danemark} %
    4-\paysfid{suede} %
    5-\paysfid{Vnorvege} %
    6-\paysfid{Vfinlande} %
    7-\paysfid{Vliflandie} %
    8-\paysfid{Veastprussia} %
    9-\paysfid{courlande} %
    10-\paysfid{pologne}.\\ %
    {\bf Atlantic shores:} %
    11-\paysfid{portugal} %
    12-\paysfid{hollande} %
    13-\paysfid{ecosse} %
    14-\paysfid{Virlande} %
    15-\paysfid{Vbelgique}. %
  \item {\bf Khanates:} %
    1-\paysfid{ryazan} %
    2-\paysfid{pskov} %
    3-\paysfid{steppes} %
    4-\paysfid{cosaquesdon} %
    5-\paysfid{kazan} %
    6-\paysfid{astrakhan} %
    7-\paysfid{crimee} %
    8-\paysfid{ukraine}.\\ %
    {\bf India:} %
    9-\paysfid{gujarat} %
    10-\paysfid{vijayanagar} %
    11-\paysfid{mysore} %
    12-\paysfid{hyderabad}. %
  \item {\bf North Africa:} %
    1-\paysfid{maroc} or (10) %
    2-\paysfid{algerie} %
    3-\paysfid{tunisie} %
    4-\paysfid{tripoli}%
    5-\paysfid{cyrenaique}.\\ %
    {\bf Semi-major countries:} %
    6-\paysfid{suede} %
    7-\paysfid{brandebourg} %
    8-\paysfid{danemark} %
    9-\pays{perse}+\paysfid{ormus} %
    10-\paysfid{portugal} %
    11-\paysfid{pologne}. %
  \item {\bf Eastern Muslims:} %
    1-\pays{perse}+\paysfid{ormus} %
    2-\paysfid{aden} %
    3-\paysfid{oman} %
    4-\paysfid{soudan} %
    5-\paysfid{mogol} %
    6-\paysfid{afghans} %
    7-\paysfid{maroc} or (10) %
    8-\paysfid{algerie} %
    9-\paysfid{tunisie}. %
  \end{enumerate}
  \caption{Diplomatic table}\label{table:diplomatic event}
\end{tablehere}

\clearpage

% \section{General revolt table (deprecated)}

% \aparag When a revolt event happens (\RD), roll in \ref{table:Revolt} to
% find in which country the \REVOLT occurs.
% \aparag The \REVOLT strength is rolled \ref{table:Revolt Strength}.
% \bparag The \leader{random} is pulled out of the generic leader pool.
% \bparag If the fortress is initially at the hand of the \REVOLT , its
% maximum level is 2 before turn 40, and 3 from turn 40 onwards. Lost levels
% are transformed in rebel \LD immediately.

% \begin{minipage}[t]{.5\linewidth}
%   \begin{tablehere}\centering\graytabular
%     \begin{tabular}{|c|p{8mm}p{8mm}p{8mm}p{8mm}p{8mm}|} \hline%
%       & I & II & III & IV, V& VI, VII\\\hline\ghline%
%       1 & \ANG & \ANG & \ANG & \ANG & \ANG\\\ghline%
%       2 & \FRA & \FRA & \FRA & \FRA & \FRA\\\ghline%
%       3 & \SPA & \SPA & \SPA & \SPA & \SPA\\\ghline%
%       4 & \TUR & \TUR & \TUR & \TUR & \TUR\\\ghline%
%       5 & \RUS & \RUS & \RUS & \RUS & \RUS\\\ghline%
%       6 & \POR & \POR & \SUE & \SUE & \SUE\\\ghline%
%       7 & \POL & \POL & \POL & \POL & {\PRU}\textddag \\\ghline%
%       8 & \VEN & \VEN & {\VEN}\textdag & \AUS & \AUS\\\ghline%
%       9 & \RUS & \RUS & \HOL & \HOL & \HOL\\\ghline%
%       0 & \POL & \ANG & \FRA & \POL & \POL\\\hline%
%     \end{tabular}\par
%     \begin{minipage}{\linewidth}
%       \textdag\ or \RUS if \VENmin.\\
%       \textddag\ or \FRA if \PRU not a major country and
%       \numberref{pV:Expulsion French Protestants} happened.
%     \end{minipage}
%     \caption{General revolt table}\label{table:Revolt}
%   \end{tablehere}
% \end{minipage}%
% \begin{minipage}[t]{.5\linewidth}
%   \begin{tablehere}
%     \begin{tabular}{c|p{55mm}}
%       1d10 & Revolt strength\\\hline
%       1--3 & \REVOLT \facemoins\\
%       4--7 & \REVOLT \facemoins and \leader{random}\\
%       8 & \REVOLT \faceplus\\
%       9 & \REVOLT \faceplus and \leader{random}\\
%       10 & \REVOLT \faceplus, \leader{random} and \LD inside the fortress
%       (which falls into the hands of the rebels).
%     \end{tabular}
%     Some events set a modifier to this roll.
%     \caption{Revolt Strength table}\label{table:Revolt Strength}
%   \end{tablehere}
% \end{minipage}

% \section{Countries revolt tables (deprecated)}
% \subsection{Revolt table for \FRA}

% \aparag When a \REVOLT occurs in \FRA, roll on this table to find the
% revolted province.

% \begin{tablehere}
%   \centerline{\graytabular\begin{tabular}{|c|cccccc|}\hline%
%     Result & pI,pII & pIII & pIV & pV & pVI & pVII \\\hline\ghline%
%     <0 & \province{Ile-de-France} & \province{Ile-de-France} &
%     \province{Ile-de-France} & \province{Ile-de-France} &
%     \province{Ile-de-France} & \province{Ile-de-France}\\\ghline%
%     0 & \provinceOrleanais & \provinceLanguedoc & \provinceOrleanais &
%     \provinceOrleanais & \provinceOrleanais & \provinceOrleanais\\\ghline%
%     1 & \provinceTroyes & \provinceOrleanais & \provinceTroyes &
%     \provinceCevennes & \provinceCevennes & \provinceLyonnais\\\ghline%
%     2 & \provinceAuvergne & \provinceDauphine & \provinceAuvergne &
%     \provinceAuvergne & \provinceAuvergne & \provinceProvence\\\ghline%
%     3 & \provinceLyonnais & \provinceTroyes & \provinceBerry &
%     \provinceBerry & \provinceBerry & \provinceBerry\\\ghline%
%     4 & \provinceQuercy & \provinceQuercy & \provinceQuercy &
%     \provinceQuercy & \provinceQuercy & \provinceQuercy\\\ghline%
%     5 & \provinceBearn & \provinceAuvergne & \provinceBearn &
%     \provinceAuvergne & \provinceBourgogne & \provinceBourgogne\\\ghline%
%     6 & \provinceDauphine & \provinceBearn & \provinceDauphine &
%     \provinceCevennes & \provinceCevennes & \provinceNormandie\\\ghline%
%     7 & \provinceProvence & \provinceBerry & \provinceProvence &
%     \provinceProvence & \provinceProvence & \provinceProvence\\\ghline%
%     8 & \provinceLanguedoc & \provinceProvence & \provinceLanguedoc &
%     \provinceLanguedoc & \provinceLanguedoc & \provinceLanguedoc\\\ghline%
%     9 & \provinceArmor & \provinceVendee & \provinceGuyenne &
%     \provinceGuyenne & \provincePicardie & \provincePicardie\\\ghline%
%     10 & \provinceFinistere & \provinceGuyenne & \provinceFinistere &
%     \provinceCevennes & \provinceCevennes & \provinceFlandre\\\ghline%
%     11 & \provinceNormandie & \provinceFinistere & \provinceNormandie &
%     \provinceNormandie & \provinceNormandie & \provinceNormandie\\\ghline%
%     12 & \provinceLimousin & \provinceNormandie & \provinceLimousin &
%     \provinceMorbihan & \provinceCaux & \provinceFinistere\\\ghline%
%     13 & \provinceChampagne & \provinceLimousin & \provinceVendee &
%     \provinceVendee & \provinceArtois & \provinceArtois\\\ghline\hline
%     \end{tabular}}
%     \caption{Revolt table for \FRA}\label{chEvents:table-revolt-france}
%   \end{tablehere}

% %   (JCD) Some observations: \provinceArmor \provinceChampagne
% %   \provincePoitou
% %   \provinceMaine 7 \provinceProvence 7 \provinceNormandie 6
% %   \provinceQuercy 6
% %   \provinceOrleanais 6 \provinceLanguedoc 6 \province{Ile-de-France} 6
% %   \provinceCevennes 6 \provinceAuvergne 5 \provinceBerry 4
% %   \provinceFinistere
% %   3 \provinceVendee 3 \provinceTroyes 3 \provinceLimousin 3
% %   \provinceGuyenne 3
% %   \provinceDauphine 3 \provinceBearn 2 \provincePicardie 2
% %   \provinceLyonnais 2
% %   \provinceBourgogne 2 \provinceArtois 1 \provinceMorbihan 1
% %   \provinceFlandre
% %   1 \provinceChampagne 1 \provinceCaux 1 \provinceArmor 0 \provincePoitou
% %   0
% %   \provinceTouraine 0 \provinceAlsace 0 \provinceLorraine 0
% %   \province{Franche-Comte} 0 \provinceRoussilon 0 \provinceBresse I
% %   suggest to
% %   stuff some generic regions instead of provinces for \FRA, it could be:
% %   1-\provinceProvence 2-\province{Ile-de-France} 3-\provinceNormandie
% %   4-\provinceOrleanais 5-\zoneAuvergne
% %   (\provinceCevennes,\provinceAuvergne)
% %   6-\zoneBourgogne (\provinceBourgogne,\provinceTroyes,\provinceChampagne)
% %   7-\zoneNord (\provincePicardie \provinceFlandre \provinceArtois
% %   \provinceHainaut) 8-\zoneProtestants (\provinceDauphine \provinceCaux
% %   \provinceLanguedoc \provinceQuercy) 9-\zoneBretagne (\provinceFinistere
% %   \provinceMorbihan \provinceArmor) 10-\zoneLyonnais (\provinceLyonnais
% %   \provinceBresse) 11-\zoneEspagnole (\provinceRoussillon
% %   \provinceFrancheComte \provinceLorraine \provinceAlsace)
% %   12-\zoneSudOuest
% %   (\provinceLanguedoc \provinceGuyenne \provinceQuercy \provinceBearn)
% %   13-\zoneRoyale (\provinceMaine \provinceBerry \provinceOrleanais)
% %   14-\zoneItalienne (\provinceNice \provinceCorsica \provinceMilano) I
% %   inserted 14 zones, with some redundancy, not all zones need to be
% %   available
% %   for one period.  Roll randomly through the zones; if first choice not
% %   French, roll a second time (and take it whatever the choice)

%   \subsection{Revolt table for Baltic countries, Portugal or Venice}

%   \aparag When a \REVOLT occurs in \POL, \SUE, \PRU, \POR or \VEN, roll on
%   this table to find the revolted province.

%   \aparag Remark that \seazoneAdriatique may be rolled as a revolted
%   province for \VEN.
%   \bparag In that case, roll for the strength as usual but put in a
%   \PIRATE\facemoins or \PIRATE\faceplus instead of a \REVOLT. Ignore the \LD
%   but use an admiral if a leader is needed.
%   \bparag The \corsaire attacks all trade fleet in \ctz{Venise}, not only
%   venetian ones.

%   \begin{tablehere}
%     \centerline{\graytabular\begin{tabular}{|c|ccccc|}\hline Result & \POL &
%       \SUE & \PRU & \POR & \VEN \\\hline\ghline%
%       <0 & \provinceMalopolska & \provinceSvealand &
%       \provinceBrandenburg&\ROTW & \provinceVeneto\\\ghline%
%       0 & \provinceMazowia & \provinceSvealand & \provinceBrandenburg &
%       \provinceTejo & \provinceVeneto\\\ghline%
%       1 & \provinceMalopolska & \provinceJamtland & \provinceAltmark&
%       \provinceAlentejo & \provinceMantova\\\ghline%
%       2 & \provinceBaltarusija & \provinceBergslagen & \provinceNeumark&
%       \province{Tras-os-Montes} & \provinceIstria\\\ghline%
%       3 & \provinceLietuva & \provinceGastrikland & \province{Ost Pommern}&
%       \provinceTanger & \provinceDalmacija\\\ghline%
%       4 & \provincePreussen & \provinceFinland & \provincePreussen&
%       \provinceTanger & \seazoneAdriatique\\\ghline%
%       5 & \provincePskov & \provinceNyland & \provinceMemel& \provinceTanger
%       & \provinceCorfu\\\ghline%
%       6 & \provincePolacak & \provinceKarelen & \provinceDanzig&
%       \provinceAcores & \provinceCyclades\\\ghline%
%       7 & \provinceLivonija & \provinceVastergotland & \provinceLausitz&
%       \provinceGaliza & \provinceMoreas\\\ghline%
%       8 & \provinceMemel & \provinceSkane & \provinceSilesie&
%       \provinceCaceres & \provinceKreta\\\ghline%
%       9 & \provincePoltava & \provinceEstland & \provinceAnhalt&
%       \provinceHuelva & \provinceChypre\\\ghline%
%       10 & \provinceUkrainya & \provinceSavo & \province{West Pommern}&
%       \provinceGranada & \provinceHellas\\\ghline%
%       11 & \provincePodolie & \provinceMemel & \provincePodolie&
%       \provinceTanger & \seazoneAdriatique\\\ghline%
%       12 & \provinceKaluga & \provinceKurland & \provinceMalopolska &
%       \provinceAlgarve & \provinceLombardia\\\ghline%
%       13 & \provinceNovgorod & \provinceLivonija & \provinceBoheme&
%       \provinceTanger & \provinceIzmir\\\hline\ghline%
%     \end{tabular}}
%     \caption{Revolt table for Baltic countries, Portugal,
%     Venice}\label{chEvents:table-revolt-baltic}
%   \end{tablehere}

%   \subsection{Revolt table for \SPA}

%   When a \REVOLT occurs in \SPA, roll on this table, in the column of the
%   current period, to find the revolted province.
%   \begin{tablehere}
%     \centerline{\graytabular\begin{tabular}{|c|cccccc|}\hline Result & I, II
%       & III & IV & V & VI & VII\\\hline\ghline%
%       <0 & \ROTW & \ROTW & \ROTW & \ROTW & \ROTW & \ROTW\\\ghline%
%       0 & \provinceGranada & \provinceGranada & \provinceGranada &
%       \provinceLombardia & \provinceLombardia & \provinceLombardia
%       \\\ghline%
%       1 & \provinceToledo & \provinceToledo & \provinceToledo &
%       \provinceSicilia & \ROTW & \ROTW \\\ghline%
%       2 & \provinceMurcia & \provinceMurcia & \provinceFlandre &
%       \provinceFlandre & \provinceFlandre & \provinceFlandre \\\ghline%
%       3 & \provinceAndalucia & \provinceAndalucia & \provinceAndalucia &
%       \provinceAndalucia & \provinceAndalucia & \provinceAndalucia
%       \\\ghline%
%       4 & \provinceExtremadura & \provinceHainaut & \provinceHainaut & \ROTW
%       & \ROTW & \ROTW \\\ghline%
%       5 & \provinceCatalunya & \provinceCatalunya & \provinceCatalunya &
%       \provinceCatalunya & \provinceCatalunya & \provinceCatalunya
%       \\\ghline%
%       6 & \provincePalermo & \provincePalermo & \provincePalermo &
%       \provincePalermo & \provincePalermo & \provincePalermo \\\ghline%
%       7 & \provinceSicilia & \provinceSicilia & \provinceSicilia &
%       \provinceSicilia & \provinceSicilia & \provinceSicilia \\\ghline%
%       8 & \provinceBasilicata & \provinceBasilicata & \provinceCatalunya &
%       \provinceCatalunya & \provinceCatalunya & \provinceCatalunya
%       \\\ghline%
%       9 & \provinceCampania & \provinceCampania & \provinceCampania &
%       \provinceCampania & \provinceCampania & \provinceCampania \\\ghline%
%       10 & \provinceValencia & \provinceLombardia & \provinceLombardia &
%       \provinceLombardia & \provinceLombardia & \ROTW \\\ghline%
%       11 & \provinceLombardia & \provinceVlaanderen & \provinceVlaanderen &
%       \provinceVlaanderen & \provinceVlaanderen & \provinceVlaanderen
%       \\\ghline%
%       12 & \province{Illes Balears} & \provinceFlandre & \provinceFlandre &
%       \provinceCatalunya & \provinceCatalunya & \provinceCatalunya
%       \\\ghline%
%       13 & \provinceSaldigna & \provinceLuxemburg & \provinceLuxemburg &
%       \provinceLuxemburg & \provinceVlaanderen & \ROTW\\\hline\ghline%
%     \end{tabular}}
%     \caption{Revolt table for \SPA}\label{chEvents:table-revolt-spain}
%   \end{tablehere}

%   \pagebreak

%   \subsection{Revolt table for \ENG}

%   When a \REVOLT occurs in \ENG, roll on this table, in the column of the
%   current period, to find the revolted province.
%   \begin{tablehere}
%     \centerline{\graytabular\begin{tabular}{|c|ccccc|}\hline Result & I & II
%       & III, IV, V & VI & VII\\\hline\ghline%
%       <0 & \province{East Anglia} & \province{East Anglia} & \province{East
%       Anglia} & \province{East Anglia} & \ROTW \\\ghline%
%       0 & \provinceKent & \provinceKent & \provinceCymru & \provinceMoray &
%       \ROTW \\\ghline%
%       1 & \provinceLincolnshire & \provinceWessex & \provinceCornwall &
%       \provinceLothian & \ROTW \\\ghline%
%       2 & \provinceMidlands & \provinceMidlands & \provinceLaighean &
%       \provinceLaighean & \provinceLaighean \\\ghline%
%       3 & \provinceCymru & \provinceBrega & \provinceMumhan &
%       \provinceMumhan & \provinceMumhan \\\ghline%
%       4 & \provinceCornwall & \provinceYorkshire & \provinceUladh &
%       \provinceUladh & \provinceUladh \\\ghline%
%       5 & \provinceGloucester & \provinceAyr & \provinceConnacht &
%       \provinceConnacht & \provinceConnacht \\\ghline%
%       6 & \provinceWessex & \provinceLincolnshire & \provinceBrega &
%       \provinceBrega & \provinceBrega \\\ghline%
%       7 & \provinceYorkshire & \provinceUladh & \provinceHighlands &
%       \provinceHighlands & \province{Illes Balears} \\\ghline%
%       8 & \provinceCumberland & \provinceConnacht & \provinceAlba &
%       \provinceAlba & \ROTW \\\ghline%
%       9 & \provinceBrega & \provinceBrega & \provinceGalloway & \provinceAyr
%       & \ROTW \\\ghline%
%       10 & \provinceLaighean & \provinceLaighean & \provinceLothian &
%       \provinceGalloway & \provinceGibraltar \\\ghline%
%       11 & \provinceUladh & \provinceUladh & \provinceUladh & \provinceUladh
%       & \provinceSaldigna \\\ghline%
%       12 & \provinceMumhan & \provinceMumhan & \provinceMumhan &
%       \provinceMumhan & \provinceMumhan \\\ghline%
%       13 & \provinceConnacht & \provinceConnacht & \provinceConnacht &
%       \provinceConnacht & \provinceConnacht \\\hline\ghline%
%     \end{tabular}}
%     \caption{Revolt table for \ANG}\label{chEvents:table-revolt-england}
%   \end{tablehere}

%   \subsection{Revolt table for \RUS}

%   When a \REVOLT occurs in \RUS, roll on this table, in the column of the
%   current period, to find the revolted province.

%   If \RUS owns provinces of the \regionPerse, check for
%   \xnameref{chSpecific:Persia:uprising}.

%   \begin{tablehere}
%     \centerline{\graytabular\begin{tabular}{|c|ccccc|}\hline%
%       Result & I, II & III, IV & V & VI & VII\\\hline\ghline%
%       <0 \textdag & \provinceMoskva & \ROTW & \ROTW & \ROTW & \ROTW
%       \\\ghline%
%       0 \textdag & \provinceMoskva & \provinceMoskva & \provinceMoskva &
%       \provinceMoskva & \provinceAstragan \\\ghline%
%       1 & \provincePskov & \provincePskov & \provincePskov & \provincePskov
%       & \provinceVyatka \\\ghline%
%       2 & \provinceNovgorod & \provinceNovgorod & \provinceNovgorod &
%       \provinceNovgorod & \ROTW \\\ghline%
%       3 & \provinceVyatka & \provinceVyatka & \provinceVyatka &
%       \provinceVyatka & \provinceVyatka \\\ghline%
%       4 & \provinceKazan & \provinceKazan & \provinceKazan & \provinceKazan
%       & \provinceKazan \\\ghline%
%       5 & \provinceAstragan & \provinceAstragan & \provinceAstragan &
%       \provinceAzov & \provinceAzov \\\ghline%
%       6 & \provinceCrimee & \provinceCrimee & \provinceUkrainya &
%       \provinceUkrainya & \provinceUkrainya \\\ghline%
%       7 & \provinceSeveria & \provinceSeveria & \provinceSeveria &
%       \provinceSeveria & \provinceSeveria \\\ghline%
%       8 & \provinceSmolenska & \provinceSmolenska & \provinceSmolenska &
%       \provinceSmolenska & \provincePrypec \\\ghline%
%       9 & \provinceKaluga & \provinceCheboksary & \provinceCheboksary &
%       \provinceCheboksary & \provinceLivonija \\\ghline%
%       10 & \provinceYaroslavl & \provinceStep & \provinceDonets &
%       \provinceDonets & \provinceNeva \\\ghline%
%       11 & \provinceDon & \province{Dikoe Pole} & \province{Dikoe Pole} &
%       \province{Dikoe Pole} & \provinceVyatka \\\ghline%
%       12 & \provinceKaluga & \provinceKaluga & \provinceKaluga &
%       \provinceDon & \provinceDonets \\\ghline%
%       13 & \provinceRyazan & \provinceLadoga & \ROTW & \ROTW & \ROTW
%       \\\hline\ghline%
%     \end{tabular}}
%     %     \centerline{\textdag: check for
%     %     \xnameref{chSpecific:Persia:uprising}}
%     \caption{Revolt table for \RUS}\label{chEvents:table-revolt-russia}
%   \end{tablehere}

%   \pagebreak

%   \subsection{Revolt table for \TUR}

%   When a \REVOLT occurs in \TUR, roll on this table, in the column of the
%   current period, to find the revolted province.

%   If \TUR owns provinces of the \region{Perse}, check for
%   \xnameref{chSpecific:Persia:uprising}.

%   \begin{tablehere}
%     \centerline{\graytabular\begin{tabular}{|c|cccccc|}\hline%
%       Result & I, II & III & IV & V & VI & VII\\\hline\ghline%
%       <0\textdag & \provinceTrakya & \provinceTrakya & \provinceTrakya &
%       \provinceTrakya & \provinceTrakya & \provinceEgypte \\\ghline%
%       0\textdag & \provinceBursa & \provinceBursa & \provinceBursa &
%       \provinceBursa & \provinceMagyarorszag & \provinceDelta \\\ghline%
%       1 & \provinceIzmir & \provinceIzmir & \provinceIzmir &
%       \provinceTrabzon & \provinceTrabzon & \province{Terra Sancta}
%       \\\ghline%
%       2 & \provinceMakedonya & \provinceMakedonya & \provinceMures &
%       \provinceMures & \provinceMures & \provinceNubie \\\ghline%
%       3 & \provinceHellas & \provinceHellas & \provinceHellas &
%       \provinceArmenie & \provinceArmenie & \provinceArmenie \\\ghline%
%       4 & \provinceCyclades & \provinceEgypte & \provinceEgypte &
%       \provinceEgypte & \provinceEgypte & \provinceEgypte \\\ghline%
%       5 & \provinceMoreas & \provinceSyrie & \provinceSyrie & \provinceSyrie
%       & \provinceSyrie & \provinceSyrie \\\ghline%
%       6 & \provinceAlabania & \provincePecs & \provincePecs & \provinceBanat
%       & \provinceBanat & \provinceCataractes \\\ghline%
%       7 & \provinceKosovo & \provinceLubnan & \provinceLubnan &
%       \provinceLubnan & \provinceBukovina & \provinceBukovina \\\ghline%
%       8 & \provinceSerbia & \provinceSerbia & \provinceSerbia &
%       \provinceSerbia & \provinceSerbia & \provinceSerbia \\\ghline%
%       9 & \provinceBulgaristan & \provinceAlep & \provinceKarpatok &
%       \provinceKarpatok & \provinceKarpatok & \provinceKarpatok \\\ghline%
%       10 & \provinceKilikya & \provinceKilikya & \provinceKilikya &
%       \provinceKordistan & \provinceKordistan & \provinceKordistan
%       \\\ghline%
%       11 & \provinceAnadolu & \provinceAnadolu & \provinceAnadolu &
%       \provinceAnadolu & \provinceAnadolu & \provinceSinai \\\ghline%
%       12 & \provinceAntalya & \provinceErdely & \provinceErdely &
%       \provinceErdely & \provinceErdely & \provinceErdely \\\ghline%
%       13 & \provinceBasarabia & \provinceCataractes & \provinceCataractes &
%       \provinceCataractes & \provinceCataractes & \provinceAlep
%       \\\hline\ghline%
%     \end{tabular}}
%     %     \centerline{\textdag: check for
%     %     \xnameref{chSpecific:Persia:uprising}}
%     \caption{Revolt table for \TUR}\label{chEvents:table-revolt-turkey}
%   \end{tablehere}

%   \subsection{Revolt table for Holland and Austria}

%   When a \REVOLT occurs in \HOL or \HAB, roll on this table, in the column
%   of the current period, to find the revolted province.
%   \begin{tablehere}
%     \centerline{\graytabular\begin{tabular}{|c|ccc|cc|}\hline%
%       Result & \HOL III, IV &\HOL V, VI &\HOL VII & \HAB IV, V, VI & \HAB
%       VII\\\hline\ghline%
%       <0 & \ROTW & \ROTW & \ROTW & \provinceOsterreich & \provinceMorava \\
%       \ghline%
%       0 & \provinceHolland & \ROTW & \ROTW & \provinceSalzburg &
%       \provinceSilesie \\\ghline%
%       1 & \provinceFriesland & \provinceFriesland &\provinceFriesland&
%       \provinceBoheme & \provinceBoheme \\\ghline%
%       2 & \provinceUtrecht &\provinceUtrecht & \ROTW& \provinceBoheme &
%       \provinceLausitz \\\ghline%
%       3 & \provinceGelderland & \provinceGelderland &\provinceGelderland&
%       \provinceLausitz & \provinceWolyn \\\ghline%
%       4 & \provinceZeeland & \provinceZeeland & \provinceZeeland&
%       \provinceSilesie & \provinceSilesie \\\ghline%
%       5 & \provinceOverijssel & \provinceOverijssel & \provinceOverijssel&
%       \provinceSteiermark & \provinceLublin \\\ghline%
%       6 & \provinceBrabant & \provinceBrabant & \provinceBrabant&
%       \provinceMorava & \provinceMorava \\\ghline%
%       7 & \provinceVlaanderen & \provinceVlaanderen & \provinceVlaanderen&
%       \provinceSzlovakia & \provinceMalopolska \\\ghline%
%       8 & \provinceLuxemburg & \provinceLuxemburg & \provinceOldenburg&
%       \provinceKarnten & \provinceKarnten \\\ghline%
%       9 & \provinceKoln & \provinceKoln & \ROTW& \provinceOsterreich &
%       \provinceBukovina \\\ghline%
%       10 & \provinceMunster & \provinceMunster & \provinceMunster&
%       \provinceCroatie & \provinceCroatie \\\ghline%
%       11 & \provinceHainaut & \provinceHainaut & \provinceHainaut&
%       \provincePecs & \provincePecs \\\ghline%
%       12 & \provinceHolland & \provinceHolland & \ROTW& \provinceBoheme &
%       \provinceBoheme \\\ghline%
%       13 & \provinceFriesland & \ROTW & \ROTW& \provinceLausitz &
%       \provinceBanat \\\hline\ghline%
%     \end{tabular}}
%     \caption{Revolt table for \HOL and
%     \HAB}\label{chEvents:table-revolt-holland}
%   \end{tablehere}

%   \clearpage




\section{Global revolts table}

% Rules moved in the corresponding chapter. Only reminder here.

\aparag Roll 2d10 and read the revolted country in the column of the current
period. The target country may be a \MIN or other abstract entity in which
case a pseudo-stability is provided in brackets.
\bparag Decrease this pseudo-stability of minors in the table by \bonus{-1}
if:
\begin{itemize}
\item This is \HOLhol and \SPA perceived the taxes at the preceding turn;
\item This is \PORpor at the turn of \ref{pIII:Portuguese Disaster} or after.
\end{itemize}
\aparag Roll 1d10+the \STAB (or modified pseudo-stability) on the target
country's table. Reroll in the description of groups below if needed.

\aparag Lastly, roll 2d10 in the last column of the table below to find the
strength of the revolt.

% Jym: I prefer to keep the "higher roll, harder revolt" paradigm than to have
% exact translation of the percentages I'd like. The translation of R-G might
% be somewhat harder, so its higher global percentage is OK.

% #faces of R = min rounds needed to crush. Troops make it harder because
% battle may be lost. No more than A- in order to prevent major victories and
% cheap STAB...

% old strength -> new strength 1-3 R-
% (30%) -> (36%) LD (1-2,3%) / A- (3-4,7%) / R- (5-9,26%)
% 4-7 R-G (30%) -> (36%) R-G (11,10%) / R-LD (10,9%) / R- A- (12,9%) / R- A-G
% (13,8%)
% 8 R+ (10%) -> (7%) R+ (14,7%) /
% 9 R+G (10%) -> (11%) R+G (15,6%) / R+ A- (16,5%)
% 0 R+GfLD (10%) -> (10%) R+GfLD (18-20,6%) / R+G A-G (17,4%)

% TODO: replace these by real commands with link to the revolt table rather
% than the specific rules.

\newcommand{\ANGrev}{\ANG} \newcommand{\AUSrev}{\AUS}
\newcommand{\DANrev}{\DAN} \newcommand{\FRArev}{\FRA}
\newcommand{\HISrev}{\HIS} \newcommand{\HOLrev}{\HOL}
\newcommand{\POLrev}{\POL} \newcommand{\PORrev}{\POR}
\newcommand{\PRUrev}{\PRU} \newcommand{\ROTWrev}{COL}
\newcommand{\RUSrev}{\RUS} \newcommand{\SUErev}{\SUE}
\newcommand{\TURrev}{\TUR} \newcommand{\VENrev}{\VEN}

\begin{designnote}
  The \ROTWrev revolt area is mutually exclusive with both
  \eventref{pIV:Revolt Singala}, \eventref{pV:Slave Revolts WI},
  \eventref{pVI:Slave Revolts WI} and \eventref{pVII:Revolt Indonesia}. If
  using the \ROTWrev revolt area, consider these events as \RD. If not, reroll
  the revolt area whenever \ROTWrev occurs.
\end{designnote}

\begin{tablehere}\centering\graytabular%
  \begin{tabular}{|c|ccccccc|c|} \hline%
    ~ & I & II & III & IV & V & VI & VII & Strength\\\hline\ghline%
    2 & \SUErev[0] & \PORrev & \FRArev & \FRArev & \PRUrev[0] & \PRUrev[0] &
    \ANGrev & \LD\\\ghline% 1%
    3 & \SUErev[0] & \PORrev & \FRArev & \AUSrev[-1] & \PORrev & \ANGrev &
    \POLrev[-2] & \LD\\\ghline% 2%
    4 & \AUSrev[-1] & \SUErev[-1] & \ANGrev & \PRUrev & \VENrev & \VENrev &
    \PRUrev & \ARMY\facemoins\\\ghline% 3%
    5 & \AUSrev[-1] & \SUErev[-1] & \SUErev & \PORrev & \PRUrev & \PRUrev &
    \ANGrev & \ARMY\facemoins\\\ghline% 4%
    6 & \PORrev & \PRUrev[+3] & \PRUrev[+3] & \HOLrev & \SUErev & \SUErev &
    \AUSrev & \REVOLT\facemoins\\\ghline% 5%
    7 & \ANGrev & \ANGrev & \SUErev & \PORrev[-1] & \POLrev & \POLrev[0] &
    \PRUrev & \REVOLT\facemoins\\\ghline% 6%
    8 & \VENrev & \VENrev & \VENrev & \VENrev[+2] & \AUSrev & \AUSrev &
    \SUErev & \REVOLT\facemoins\\\ghline% 7%
    9 & \FRArev & \HISrev & \HISrev & \HISrev & \HISrev & \HISrev & \HISrev &
    \REVOLT\facemoins\\\ghline% 8%
    10 & \HISrev & \FRArev & \PORrev[-1] & \FRArev & \ANGrev & \ANGrev &
    \POLrev[-2] & \REVOLT\facemoins/\LD\\\ghline% 9%
    11 & \HOLrev[-1] & \HOLrev[-2] & \HOLrev[-3] & \POLrev & \ROTWrev[0] &
    \ROTWrev[0] & \ROTWrev[+3] & \REVOLT\facemoins\LeaderG\\\ghline%10%
    12 & \ANGrev & \ANGrev & \ANGrev & \ANGrev & \RUSrev & \RUSrev &
    \POLrev[-2] & \REVOLT\facemoins/\ARMY\facemoins\\\ghline% 9%
    13 & \RUSrev & \POLrev & \POLrev & \RUSrev & \PORrev & \FRArev & \FRArev &
    \REVOLT\facemoins/\ARMY\facemoins\LeaderG\\\ghline% 8%
    14 & \TURrev & \TURrev & \RUSrev & \SUErev & \POLrev & \POLrev[0] &
    \HOLrev & \REVOLT\faceplus\\\ghline% 7%
    15 & \POLrev & \AUSrev[+1] & \AUSrev[+1] & \TURrev & \TURrev & \TURrev &
    \TURrev & \REVOLT\faceplus\LeaderG\\\ghline% 6%
    16 & \PORrev & \RUSrev & \TURrev & \AUSrev[+1] & \HOLrev & \HOLrev &
    \RUSrev & \REVOLT\faceplus/\ARMY\facemoins\\\ghline% 5%
    17 & \POLrev & \AUSrev[-2] & \AUSrev[-2] & \TURrev & \TURrev & \TURrev &
    \TURrev & \REVOLT\faceplus\LeaderG/\ARMY\facemoins\LeaderG\\\ghline% 4%
    18 & \TURrev & \TURrev & \RUSrev & \ROTWrev[-3] & \FRArev & \PORrev &
    \HOLrev & \REVOLT\faceplus\LeaderG\fortress\LD\\\ghline% 3%
    19 & \VENrev & \VENrev & \VENrev & \ROTWrev[-3] & \FRArev & \FRArev &
    \FRArev & \REVOLT\faceplus\LeaderG\fortress\LD\\\ghline% 2%
    20 & \HISrev & \FRArev & \PORrev[-1] & \AUSrev[-2] & \RUSrev & \RUSrev &
    \PRUrev & \REVOLT\faceplus\LeaderG\fortress\LD\\\hline\ghline% 1%
    % ANG FRA HIS POR SUE HOL AUS VEN TUR RUS POL PRU ROTW
    % _15 __8 _10 _10 __3 _10 __7 __9 _10 __8 _10 __0 ___0
    % _15 _10 __8 __3 __7 _10 _10 __9 _10 __5 __8 __5 ___0
    % _12 __3 __8 _10 _10 _10 _10 __9 __5 _10 __8 __5 ___0
    % __9 _10 __8 _10 __7 __5 __8 __7 _10 __8 _10 __3 ___5
    % __9 __5 __8 _10 __5 __5 __7 __3 _10 _10 _13 __5 __10
    % _11 _10 __8 __3 __5 __5 __7 __3 _10 _10 _13 __5 __10
    % __5 _10 __8 __0 __7 _10 __5 __0 _10 __5 _20 _10 __10
  \end{tabular}\par
  \caption{Revolt table: target area and
    strength}\label{table:alt-revolt-global}
\end{tablehere}

\begin{designnote}
  Here's the percentages of each country being rolled in each period.
  \begin{tabular}{|r|rrrrrrrrrrrrr|}
    \hline
    ~ & \ANGrev & \FRArev & \HISrev & \PORrev & \SUErev & \HOLrev & \AUSrev
    & \VENrev & \TURrev & \RUSrev & \POLrev & \PRUrev & \ROTWrev\\
    \hline
    \period{I} & 15 & 8 & 10 & 10 & 3 & 10 & 7 & 9 & 10 & 8 & 10 & 0 & 0\\
    \period{II} & 15 & 10 & 8 & 3 & 7 & 10 & 10 & 9 & 10 & 5 & 8 & 5 & 0\\
    \period{III} & 12 & 3 & 8 & 10 & 10 & 10 & 10 & 9 & 5 & 10 & 8 & 5 & 0\\
    \period{IV} & 9 & 10 & 8 & 10 & 7 & 5 & 8 & 7 & 10 & 8 & 10 & 3 & 5\\
    \period{V} & 9 & 5 & 8 & 10 & 5 & 5 & 7 & 3 & 10 & 10 & 13 & 5 & 10\\
    \period{VI} & 11 & 10 & 8 & 3 & 5 & 5 & 7 & 3 & 10 & 10 & 13 & 5 & 10\\
    \period{VII} & 5 & 10 & 8 & 0 & 7 & 10 & 5 & 0 & 10 & 5 & 20 & 10 & 10\\
    \hline
  \end{tabular}
\end{designnote}

\clearpage

% Jym: Some revolts are much harder to fight. Typically the ROTW one require
% more effort than in Europe.  Those should not be in the 13 line. The 13 line
% should only create "easy" revolts. 1/ Bonus. 2/ +3 STAB can be somewhat
% avoided thus preventing the hard revolts to occur (we don't want MKL abusing
% the system that way).

% I'm basically swapping the 12 and 13 lines where ever I think it should be
% done.




\section{Countries revolt tables}

\label{chapter:events:revolt:Country tables}



\subsection{Revolt table for \ANG}

\aparag When a \REVOLT occurs in \ANG, roll on this table, in the column of
the current period.

\begin{tablehere}
  \defgroup{CE}{\group{Central England}}{1.~\provinceKent,
    2--3.~\provinceLincolnshire, 4.~\provinceWessex,
    5--6.~\provinceGloucester, 7--10.~\province{East Anglia}}%
  \defgroup{WE}{\group{Western England}}{1--4.~\provinceCornwall,
    5--8.~\provinceCymru, 9--10.~\provinceMidlands}%
  \defgroup{NE}{\group{Northern England}}{1--3.~\provinceYorkshire,
    4--6.~\provinceCumberland, 7--9.~\provinceDurham,
    10.~\provinceLancashire}%
  \defgroup{PE}{\group{English Provinces}}{1.~\provinceLincolnshire,
    2.~\provinceWessex, 3.~\provinceGloucester, 4.~\provinceCornwall,
    5.~\provinceCymru, 6.~\provinceMidlands, 7.~\provinceYorkshire,
    8.~\provinceCumberland, 9.~\provinceDurham, 10.~\provinceLancashire}%
  \defgroup{II}{\group{Inner Ireland}}{1--5.~\provinceBrega,
    6--10.~\provinceLaighean}%
  \defgroup{OI}{\group{Outer Ireland}}{1--3.~\provinceMumhan,
    4--6.~\provinceConnacht, 7--10.~\provinceUladh}%
  \defgroup{LS}{\group{Low Scotland}}{1--4.~\provinceAyr,
    5--7.~\provinceLothian, 8--10.~\provinceGalloway}%
  \defgroup{HS}{\group{High Scotland}}{1--4.~\provinceHighlands,
    5--7.~\provinceAlba, 8--10.~\provinceMoray}%
  \defgroup{RE}{\group{Europe}}{A random English European province not in
    Great-Britain/Ireland (possibly including \payshanovre); if none,
    \ctz{Angleterre}}%
  \defgroup{Wy}{\provinceCymru}{}%
  \defgroup{Wc}{\provinceCornwall}{}%
  \defgroup{Cl}{\province{East Anglia}}{}%
  \defgroup{Sa}{\ctz{Angleterre}}{}%
  \defgroup{FS}{\group{French Soil}}{1.~\provinceGuyenne,
    2--4.~\provinceFinistere, 5--7.~\provinceArmor, 8--10.~\provincePicardie}%
  \defgroup{AS}{\group{Scotland}}{1--3.~\provinceAyr, 4--5.~\provinceLothian,
    6.~\provinceGalloway, 7--8.~\provinceHighlands, 9.~\provinceAlba,
    10.~\provinceMoray}%
  % \defgroup{Am}{\group{America}}{Any \TP/\COL of \ANG in
  % \continent{America}}%
  % Feedback for American Revolution victory/defeat.  Induced change :
  % replaced a OI by RO in pV,VI (was 4 OI) and a As by RO in pVII. Avoid safe
  % COL/TP elsewhere in America.
  \defgroup{Am}{\group{America}}{A random \COL/\TP (of any nationality) in the
    following area: 1--2.~\granderegionAmerica, 3--4.~\granderegionVirginia,
    5--6.~\granderegionCarolina, 7--8.~\granderegionAntilles,
    9--10.~\granderegion{Terre-Neuve} or \granderegionHudson}
  % America is both a continent and an area...  (should we fix this ?)
  \defgroup{As}{\group{Asia}}{A random \TP/\COL of \ANG not in continent
    \continent{America}}%
  \defgroup{RO}{\group{ROTW}}{A random \TP/\COL of \ANG; if none, \Sa}%
  \centerline{%
    \graytabular\begin{tabular}{|c|ccccc|}\hline%
      Result & I & II & III, IV & V, VI & VII\\\hline\ghline%
      <0& \CE & \Cl & \CE & \CE & \PE \\\ghline%
      0 & \CE & \CE & \NE & \PE & \AS \\\ghline%
      1 & \Wy & \Wy & \WE & \OI & \RE \\\ghline%
      2 & \Sa & \NE & \OI & \HS & \OI \\\ghline%
      3 & \Wc & \Wc & \WE & \LS & \II \\\ghline%
      4 & \CE & \CE & \PE & \II & \RO \\\ghline%
      5 & \NE & \NE & \LS & \OI & \Am \\\ghline%
      6 & \OI & \OI & \OI & \PE & \PE \\\ghline%
      7 & \II & \II & \II & \II & \II \\\ghline%
      8 & \NE & \Sa & \LS & \AS & \As \\\ghline%
      9 & \WE & \WE & \II & \RO & \Am \\\ghline%
      10& \FS & \LS & \OI & \RE & \RE \\\ghline%
      11& \OI & \FS & \HS & \As & \RE \\\ghline%
      12& \LS & \NE & \RO & \Am & \OI \\\ghline%
      13& \Wc & \OI & \OI & \OI & \OI \\\hline\ghline%
      % pI : CE 5>3 WE 3>4 NE 2=2 II 2>1 OI 3>2 LS 0>1 FS 0>1 CTZ 0>1
      % pII: CE 4>3 WE 1>3 NE 1>2 II 3>1 OI 5>2 LS 1>2 FS 0>1 CTZ 0>1
      % pIII:CE 1=1 WE 2=2 NE 0>1 II 2=2 OI 6>5 LS 3=3 HS 1=1
      % pVI: CE 1=1 PE 0>1 II 2=2 OI 6>5 LS 3>2 HS 3>3 RE 0>1
      % pVII:PE 0>1 RE 3=3 II 2=2 OI 5>4 RO 5>4
    \end{tabular}}
  \showgrouplist{Am,As,CE,PE,RE,FS,HS,II,LS,NE,OI,RO,AS,WE}{}
  \caption{Revolt table for \ANG}\label{table:alt-revolt-england}
\end{tablehere}

\clearpage



\subsection{Revolt table for \FRA}

\aparag When a \REVOLT occurs in \FRA, roll on this table, in the column of
the current period.

\smallskip

% small -> catholics area (incl Paris), large -> protestant area.  the roundly
% revolt of FWR only spans if in the correct minor.
\aparag For the roundly revolts caused by~\eventref{pIII:FWR}, always use the
column for period III (even if it occurs during another period).
\bparag Moreover, if \FRA is catholic, \textbf{subtract} its \STAB rather than
adding it to find the localisation of the revolts caused by this event.
% No Tr/Sa in pIII as III-D already span P+ in the CTZ.

\begin{tablehere}
  % Catalogne -> HIS table.
  \defgroup{Pa}{\province{Ile-de-France}}{}%
  \defgroup{BR}{\group{Brittany}}{1--4.~\provinceArmor,
    5--7.~\provinceFinistere, 8--10.~\provinceMorbihan}% FRA
  \defgroup{SE}{\group{Midi}}{1--3.~\provinceCevennes,
    4--6.~\provinceLanguedoc, 7--8.~\provinceDauphine,
    9--10.~\provinceProvence}% Hug+
  \defgroup{SR}{\group{Spanish Road}}{1--2.~\provinceBresse,
    3--4.~\province{Franche-Comte}, 5--6.~\provinceAlsace,
    7--8.~\provincePfalz, 9--10.~\provinceLuxemburg}% out pIV+
  \defgroup{BE}{\group{Belgium}}{1--3.~\provincePicardie,
    4--6.~\provinceArtois, 7--8.~\provinceFlandre,
    9--10.~\provinceHainaut}% mix
  \defgroup{SW}{\group{Aquitaine}}{1--3.~\provinceBearn,
    4--6.~\provincePoitou, 7--8.~\provinceGuyenne,
    9--10.~\provinceQuercy}% Hug
  \defgroup{CF}{\group{Central France}}{1--2.~\provinceLyonnais,
    3--4.~\provinceAuvergne, 5--6.~\provinceLimousin, 7--8.~\provinceTouraine,
    9--10.~\provinceBerry}% mix
  \defgroup{EF}{\group{East}}{1--2.~\provinceBourgogne, 3--4.~\provinceTroyes,
    5--6.~\provinceChampagne, 7--8.~\provinceLorraine,
    9--10.~\provinceAlsace}% Lig
  \defgroup{NW}{\group{North West}}{1--2.~\provinceVendee,
    3--4.~\provinceMaine, 5--6.~\provinceNormandie, 7--8.~\provinceCaux,
    9--10.~\provinceOrleanais}% Lig+
  \defgroup{It}{\group{Italy}}{1--2.~\provinceBresse, 3--4.~\provinceSavoia,
    5--6.~\provinceLombardia, 7--8.~\provinceNice,
    9--10.~\provinceCorsica}% out, pII-,pVII
  \defgroup{Sa}{\ctz{France}}{}%
  \defgroup{Am}{\group{America}}{A random \COL/\TP (of any nationality) in the
    following area: 1--2.~\granderegionQuebec, 3--4.~\granderegion{Grands
      Lacs}, 5--6.~\granderegionMississippi, 7--8.~\granderegionAcadie,
    9--10.~\granderegion{Terre-Neuve} or \granderegionHudson}
  % only specified areas so Caraibes excluded. Nice if SYW is
  % lost and Canada is no more FRA. Bad if SYW is won. Good feedback.
  \defgroup{RO}{\group{\ROTW}}{A random \COL/\TP of \FRA ; \ctz{France} if
    none}%
  \centerline{\graytabular\begin{tabular}{|c|cccccc|}\hline%
      Result & pI,pII & pIII & pIV & pV & pVI & pVII \\\hline\ghline%
      <0 & \Pa & \Pa & \Pa & \Pa & \Pa & \Pa\\\ghline%
      0 & \NW & \EF & \NW & \CF & \CF & \CF\\\ghline%
      1 & \Sa & \NW & \SE & \SW & \SW & \SW\\\ghline%
      2 & \CF & \NW & \CF & \NW & \NW & \NW\\\ghline%
      3 & \CF & \EF & \CF & \CF & \CF & \CF\\\ghline%
      4 & \It & \EF & \NW & \BE & \BE & \BE\\\ghline%
      5 & \SW & \CF & \SW & \SE & \SE & \SE\\\ghline%
      6 & \SE & \CF & \EF & \BR & \SR & \SR\\\ghline%
      7 & \NW & \CF & \CF & \SR & \SW & \SW\\\ghline%
      8 & \SE & \SE & \SE & \SW & \BR & \BR\\\ghline%
      9 & \BR & \BR & \BR & \RO & \Sa & \Sa\\\ghline%
      10 & \EF & \SW & \BE & \SE & \SE & \BE\\\ghline%
      11 & \It & \SE & \RO & \EF & \RO & \RO\\\ghline%
      12 & \BR & \SW & \Am & \Am & \Am & \Am\\\ghline%
      13 & \BE & \SE & \SR & \BR & \EF & \It\\\ghline\hline
    \end{tabular}}
  \showgrouplist{Am,SW,BE,BR,CF,EF,It,NW,RO,SR}{}
  \caption{Revolt table for \FRA}\label{table:alt-revolt-france}
\end{tablehere}

\clearpage



\subsection{Revolt table for \HIS}

\aparag When a \REVOLT occurs in \HIS, roll on this table, in the column of
the current period.

\begin{tablehere}
  \defgroup{HC}{\group{Central Castile}}{1--3.~\province{Castilla La Nueva},
    4--5.~\provinceToledo, 6--7.~\provinceSalamanca, 8.~\provinceLeon,
    9--10.~\province{Castilla La Vieja}}%
  \defgroup{HN}{\group{Northern Castile}}{1--2.~\provinceGaliza,
    3--4.~\provinceAsturias, 5--6.~\provinceVizcaya, 7--8.~\provinceNavarra,
    9--10.~\provinceBearn}%
  \defgroup{HS}{\group{Southern Castile}}{1--2.~\provinceCaceres,
    3--4.~\provinceExtremadura, 5--6.~\provinceHuelva,
    7--9.~\provinceAndalucia, 10.~\provinceGibraltar}%
  \defgroup{HE}{\group{Granada}}{1--4.~\provinceGranada,
    5--7.~\provinceCordoba, 8--9.~\provinceMurcia, 10.~\province{La Mancha}}%
  \defgroup{HA}{\group{Aragon}}{1--4.~\provinceAragon,
    5--8.~\provinceValencia, 9--10.~\province{Illes Balears}}%
  \defgroup{HB}{\group{Catalonia}}{1--5.~\provinceCatalunya,
    6--7.~\provincePirineos, 8--10.~\provinceRosselo}%
  \defgroup{Sa}{\group{Atlantic}}{1--6.~\ctz{Espagne}, 7--8.~\seazoneLion,
    9.~\seazoneCanarias, 10.~\province{Islas Canarias}}%
  \defgroup{Am}{\group{New Spain}}{A random \COL/\TP (of any nationality) in
    the following area: 1--3.~\granderegionAzteca, 4--6.~\granderegionInca,
    7--8.~\granderegionChichimeca, 9.~\granderegionCuba,
    10.~\granderegionGuyana. If some area is empty, it is replaced by \Sa.}%
  \defgroup{As}{\group{Asia}}{A random \TP/\COL of \HIS not in continent
    \continentAmerica; if none, \Sa}%
  \defgroup{RO}{\group{America}}{A random \TP/\COL of \HIS in
    \continentAmerica; if none, \Sa}%
  \defgroup{AA}{\group{Africa}}{1.~\provinceAlgerie, 2--3.~\provinceOran,
    4.~\provinceAnnabah, 5--7.~\provinceTunis, 8.~\provinceIfriqiya,
    9.~\provinceAures, 10.~\provinceAtlas and \provinceKabylie (Revolts
    strength at \bonus{-10}, possibly no revolt if Strength<2)}%
  \defgroup{It}{\group{Italy}}{1.~\provinceMonferrato, 2.~\provinceSavoia,
    3.~\provinceParma, 4.~\provinceLucca, 5.~\provinceToscana,
    6.~\provinceSiena , 7.~\provinceNice, 8.~\provinceLiguria,
    9--10.~\provinceLombardia}%
  \defgroup{IN}{\group{Naples}}{1.~\provinceUmbria, 2.~\provinceLazio,
    3.~\provinceUmbria, 4.~\provinceAbruzzo, 5.~\provincePuglia,
    6.~\provinceBasilicata, 7.~\provinceCalabria, 8--10.~\provinceCampania}%
  \defgroup{II}{\group{Islands}}{1--2.~\provinceCorsica,
    3--4.~\provinceSaldigna, 5--6.~\provincePalermo, 7--8.~\provinceSicilia,
    9--10.~\provinceMalta}%
  \centerline{%
    \graytabular\begin{tabular}{|c|ccccc|}\hline%
      Result & I & II & III, IV & V, VI & VII\\\hline\ghline%
      <0& \HC & \HC & \HC & \HA & \HS \\\ghline%
      0 & \HA & \HA & \HA & \HS & \HN \\\ghline%
      1 & \HS & \HS & \HS & \HN & \IN \\\ghline%
      2 & \HA & \HE & \II & \IN & \Am \\\ghline%
      3 & \HC & \IN & \HN & \Am & \RO \\\ghline%
      4 & \HE & \HE & \HE & \HE & \HE \\\ghline%
      5 & \HB & \HB & \HB & \HB & \HB \\\ghline%
      6 & \Am & \Am & \Am & \II & \II \\\ghline%
      7 & \IN & \IN & \AA & \It & \It \\\ghline%
      8 & \HN & \HN & \HN & \HB & \HB \\\ghline%
      9 & \RO & \RO & \RO & \RO & \RO \\\ghline%
      10& \As & \As & \As & \As & \As \\\ghline%
      11& \II & \II & \IN & \IN & \IN \\\ghline%
      12& \It & \It & \It & \It & \It \\\ghline%
      13& \AA & \AA & \AA & \AA & \AA \\\hline\ghline%
    \end{tabular}}
  \showgrouplist{AA,HA,RO,As,Sa,HB,HC,HE,II,It,IN,Am,HN,HS}{}
  \caption{Revolt table for \HIS}\label{table:alt-revolt-spain}
\end{tablehere}

\clearpage



\subsection{Revolt table for \POR, \SUE and COL}

\aparag When a \REVOLT occurs in \SUE or \POR, roll on this table, in the
column of the correct country and current period.

\bparag If \DANmin or \SUEmin have to fight a revolt, they will raise the Sund
taxes (see \ref{chSpecific:Sund Levies}).

\aparag Decrease the pseudo-stability of \PORpor by \bonus{-1} if
\ref{pIII:Portuguese Disaster} happened (at this turn or a previous one).

\begin{tablehere}
  \defgroup{SC}{\provinceSvealand}{}%
  \defgroup{SN}{\group{Northern Sweden}}{1--3.~\provinceJamtland,
    4--6.~\provinceBergslagen, 7--10.~\provinceGastrikland}%
  \defgroup{SS}{\group{Southern Sweden}}{1--2.~\provinceVastergotland,
    3--4.~\provinceSmaland, 5--7.~\provinceGotland, 8--10.~\provinceSkane}%
  \defgroup{FI}{\group{Finland}}{1--3.~\provinceFinland,
    4--5.~\provinceNyland, 6.~\provinceTavastland, 7--8.~\provinceKarelen,
    9--10.~\provinceKexholm}%
  \defgroup{SB}{\group{Baltic Sweden}}{1.~\seazoneBaltique,
    2--3.~\provinceEstland, 4--5.~\provinceLivonija, 6.~\provinceKurland,
    7.~\provinceDanzig, 8--9.~\provinceHinterpommern,
    10.~\provinceVorpommern}%
  \defgroup{SH}{\group{Hansa}}{1--2.~\seazoneBaltique, 3--4.~\provinceBremen,
    5--6.~\provinceHolstein, 7--8.~\provinceLubeck,
    9--10.~\provinceMecklenburg}%
  \defgroup{DN}{\group{Denmark}}{1--2.~\provinceSjaelland,
    3.~\provinceJylland, 4.~\provinceSlesvig, 5--6.~\provinceOstlandet,
    7.~\provinceVestfold, 8.~\provinceTrondelag, 9--10.~\provinceSkane}%
  \defgroup{RS}{\group{ROTW (\SUE)}}{A random \COL\faceplus of \SUE; if none,
    \stz{Baltique}}% was \ctz RUS but Baltique is worst for SUE.

  \defgroup{RP}{\group{ROTW (\POR)}}{A random \COL\faceplus of \POR; if none,
    \stz{Guinee}}% was \ctz SPA. Could be \stz Canarias
  \defgroup{PR}{\group{Portugal}}{1--4.~\province{Tras-os-Montes},
    5--7.~\provinceAlgarve, 8--10.~\provinceBeira}%
  \defgroup{PC}{\group{Tagus}}{1--5.~\provinceTejo, 6--10.~\provinceAlentejo}%
  \defgroup{PO}{\group{Overseas}}{1--8.~\provinceTanger,
    9--10.~\provinceAcores}%
  \defgroup{PM}{\group{Morocco}}{1--3.~\provinceTanger, 4--6.~\provinceMagrib,
    7.~\provinceGranada, 8.~\ctz{Espagne}, 9.~\provinceSouss,
    10.~\provinceRif}%
  \defgroup{PH}{\group{Spain}}{1--4.~\provinceGaliza, 5--7.~\provinceCaceres,
    8.~\provinceExtremadura, 9--10.~\provinceHuelva}%

  \defgroup{Si}{\group{Singala}}{\REVOLT\facemoins in a random \COL/\TP in
    \granderegionSingala or \granderegionFormose}%
  \defgroup{WI}{\group{Slaves}}{Each power with a \COL in either
    \granderegionCuba, \granderegionHaiti or \granderegionAntilles rolls a
    die. On 7 or more, place a \REVOLT\facemoins in a random \COL of this
    power in these areas.} %
  \defgroup{In}{\group{Indonesia}}{Place one \REVOLT \facemoins and one
    \REVOLT \faceplus in two randomly chosen \COL/\TP in areas
    \granderegionJava, \granderegionSumatra, \granderegionBorneo and
    \granderegionCelebes. Both \REVOLT can occur in the same place.}
  % STAB: pIV: -3 (Si) ; pV,pVI: 0 (WI) ; pVII: +3 (In).

  \centerline{%
    \graytabular\begin{tabular}{|c|c|c|cc|c|}\hline%
      Result & \POR & \SUE I--II & \SUE III--IV & \SUE V--VII &
      COL\\\hline\ghline%
      <0& \RP & \DN & \SC & \SC & \Si \\\ghline%
      0 & \PC & \SS & \SC & \SN & \Si \\\ghline%
      1 & \PC & \SC & \SN & \SN & \Si\\\ghline%
      2 & \PR & \DN & \DN & \DN & \Si\\\ghline%
      3 & \PR & \SH & \SS & \SH & \WI\\\ghline%
      4 & \PO & \SB & \SB & \SB & \WI\\\ghline%
      5 & \PO & \FI & \FI & \FI & \WI\\\ghline%
      6 & \PM & \DN & \SS & \RS & \WI\\\ghline%
      7 & \PH & \SS & \SS & \SS & \WI\\\ghline%
      8 & \PH & \SH & \SH & \FI & \WI\\\ghline%
      9 & \PH & \SS & \FI & \SB & \In\\\ghline%
      10& \PM & \FI & \FI & \FI & \In\\\ghline%
      11& \PO & \SB & \SB & \SB & \In\\\ghline%
      12& \PR & \RS & \RS & \RS & \In\\\ghline%
      13& \PM & \DN & \DN & \DN & \In\\\hline\ghline%
    \end{tabular}}
  \showgrouplist{PM,PO,PR,RP,PH,PC}{SB,DN,FI,SH,SN,RS,SS}\hrule%
  \showgrouplist{Si,WI,In}{}

  \caption{Revolt table for \POR, \SUE and
    COL}\label{table:alt-revolt-portugal-sweden}
\end{tablehere}

\clearpage



\subsection{Revolt tables for \HOL and \AUS}

\aparag When a \REVOLT occurs in \AUSaus or \HOL, roll on this table, in the
column of the correct country and current period.

\aparag Decrease the pseudo-stability of \HOLhol by \bonus{-1} if \HIS
perceived the taxes last turn.

\begin{tablehere}
  \defgroup{RO}{\group{ROTW}}{A random Dutch \COL; if none, \Sa}%
  \defgroup{NR}{\group{Rhine lands}}{1--4.~\provinceZeeland,
    5--10.~\provinceUtrecht}%
  \defgroup{NN}{\group{North lands}}{1--5.~\provinceFriesland,
    6--10.~\provinceOverijssel, }%
  \defgroup{NO}{\group{Outer lands}}{1--3.~\provinceLimburg,
    4--5.~\provinceBrabant, 6.~\provinceLiege, 7.~\provinceBremen,
    8.~\provinceOldenburg, 9.~\provinceGibraltar, 10.~\provinceBaleares}%
  \defgroup{NE}{\group{Netherlands}}{1--2.~\provinceHolland,
    3--4.~\provinceGelderland, 5.~\provinceZeeland, 6--7.~\provinceUtrecht,
    8--9.~\provinceOverijssel, 10.~\provinceFriesland}%
  \defgroup{NH}{\provinceHolland}{}%
  \defgroup{NG}{\provinceGelderland}{}%
  \defgroup{BF}{\group{Flanders}}{1--5.~\provinceVlaandern,
    6--10.~\provinceFlandre}%
  \defgroup{BB}{\group{Brussels}}{1--5.~\provinceBrabant,
    6--10.~\provinceLimburg}%
  \defgroup{BW}{\group{Wallonia}}{1--3.~\provinceLuxemburg,
    4--6.~\provinceHainaut, 7--10.~\provinceArtois}%
  \defgroup{GW}{\group{Westphalia}}{1--3.~\provinceBerg,
    4--5.~\provinceNassau, 6--8.~\provinceOldenburg,
    9--10.~\provinceOsnabruck}%
  \defgroup{Sa}{\ctz{Hollande}}{}%
  \defgroup{Am}{\group{America}}{A random \TP/\COL of \HOL in
    \continent{America}; if none, \Sa}%
  \defgroup{As}{\group{Asia}}{A random \TP/\COL of \HOL not in
    \continent{America}; if none, \Sa}%
  \defgroup{PW}{\group{Peasants War}}{After \shortref{pI:Reformation}, place
    3 % precise number TBD.
    random \REVOLT in provinces of the \HRE. The Emperor must crush these
    revolts that can extend in all the \HRE and cause loss of \STAB to the
    Emperor. Otherwise, \NN.}

  \defgroup{AA}{\group{Alps}}{1--3.~\provinceTrentino, 4--6.~\provinceTirol,
    7.~\provinceGraubunden, 8--9.~\provinceSchwaben, 10.~\provinceFriuli}%
  \defgroup{AI}{\group{Naples}}{1--2.~\provinceCampania, 3.~\provinceAbruzzo,
    4.~\provincePuglia, 5.~\provinceBasilicata, 6.~\provinceCalabria,
    7.~\provincePalermo, 8.~\provinceSicilia, 9.~\provinceMalta,
    10.~\provinceSaldigna}%
  \defgroup{AP}{\group{Poland}}{1--3.~\provinceBukovina, 4--5.~\provinceWolyn,
    6--7.~\provinceLublin, 8.~\provinceWielkopolska,
    9--10.~\provinceMalopolska}%
  \defgroup{AS}{\group{Slovenia}}{1--2.~\provinceIstria,
    3--5.~\provinceSlovenija, 6--8.~\provinceSteiermark,
    9--10.~\provinceKarnten}%
  \defgroup{AG}{\group{Germany}}{1--3.~\provinceOberPfalz,
    4--7.~\provinceSchwaben, 8--10.~\provinceAnhalt}%
  \defgroup{AD}{\group{Danube}}{1--5.~\provinceOsterreich,
    6--10.~\provinceSalzburg}%
  \defgroup{PM}{\group{Moravia}}{1--3.~\provinceMorava,
    4--6.~\provinceLausitz, 7--10.~\provinceSilesie}%
  \defgroup{PB}{\provinceBoheme}{}%
  \defgroup{HC}{\group{Croatia}}{1--2.~\provinceKapela,
    3--5.~\provinceCroatie, 6--7.~\provinceCarniola,
    8--10.~\provinceDalmacija}%
  \defgroup{HS}{\group{Slovakia}}{1--3.~\provinceSzlovakia,
    4--6.~\provinceBalaton, 7--10.~\provincePecs}%
  \defgroup{HF}{\group{Hungary}}{1--3.~\provinceKarpatok,
    4--5.~\provinceMagyarorszag, 6--8.~\provinceBanat, 9--10.~\provinceBosna}%
  \centerline{%
    \graytabular\begin{tabular}{|c|c|ccc|cc|}\hline%
      Result &\HOL I--II&\HOL III--IV&\HOL V--VI&\HOL VII%
      &\HAB I--VI&\HAB VII\\\hline\ghline%
      <0& \NH & \NH & \As & \NG & \AD & \AD \\\ghline%
      0 & \NR & \RO & \Am & \NN & \AA & \PM \\\ghline%
      1 & \NR & \NR & \NR & \NR & \AI & \AI \\\ghline%
      2 & \NN & \NN & \NN & \As & \PB & \PB \\\ghline%
      3 & \NG & \NG & \NG & \Am & \PM & \PM \\\ghline%
      4 & \PW & \NE & \NE & \NE & \PB & \AP \\\ghline%
      5 & \NO & \NO & \NO & \NO & \AS & \AS \\\ghline%
      6 & \BB & \BB & \BB & \BB & \PM & \PM \\\ghline%
      7 & \BF & \BF & \BF & \BF & \AP & \AP \\\ghline%
      8 & \BW & \BW & \BW & \BW & \AS & \AS \\\ghline%
      9 & \Sa & \NO & \NO & \NO & \AG & \HF \\\ghline%
      10& \GW & \GW & \GW & \Am & \HC & \HC \\\ghline%
      11& \NE & \NE & \Am & \RO & \HS & \HS \\\ghline%
      12& \Sa & \RO & \RO & \As & \PB & \PB \\\ghline%
      13& \BW & \BW & \NE & \GW & \PM & \HF \\\hline\ghline%
    \end{tabular}}
  \showgrouplist{Am,As,BB,BF,NE,NN,NO,PW,NR,RO,BW,GW}{AA,HC,AD,AG,HF,PM,AI,AP,HS,AS}

  \caption{Revolt table for \HOL and
    \AUSaus}\label{table:alt-revolt-holland-austria}
\end{tablehere}

\clearpage



\subsection{Revolt tables for \POL and \PRU}

\aparag When a \REVOLT occurs in \POL or \PRU, roll on this table, in the
column of the correct country and current period.

\begin{tablehere}
  \defgroup{PM}{\group{Capitals}}{1--3.~\provinceMalopolska,
    4--5.~\provinceLietuva (if no union of Lublin; \provinceMalopolska else),
    6--10.~\provinceMazowia. If union with \payssaxe, use rather
    1--3.~\provinceMalopolska, 4--5.~\provinceAnhalt, 6--7.~\provinceSachsen,
    8--10.~\provinceMazowia.}%
  \defgroup{PC}{\group{Central Poland}}{1--3.~\provinceWielkopolska,
    4--6.~\provinceWolyn, 7--10.~\provinceLublin}%
  \defgroup{PB}{\group{Baltic Poland}}{1.~\seazoneBaltique,
    2--3.~\provinceDanzig, 4--5.~\province{West Preussen},
    6--7.~\provinceKurland, 8.~\provinceLivonija, 9.~\provinceMemel,
    10.~\provincePreussen}%
  \defgroup{LS}{\group{Smolensk}}{1--3.~\provinceSmolenska,
    4--5.~\provincePolacak, 6--7.~\provinceSeveria,
    8--10.~\provinceBaltarusija}%
  \defgroup{LL}{\group{Lithuania}}{1--5.~\provinceLietuva,
    6--8.~\provinceZemaitija, 9--10.~\provincePrypec}%
  % Use either GP+GK or GT+GL
  \defgroup{GP}{\group{Prussia}}{1--4.~\provinceMemel,
    5--7.~\provincePreussen, 8--10.~\province{Ost Pommern}}%
  \defgroup{GL}{\group{Livonia}}{1--3.~\provinceKurland,
    4--6.~\provinceEstland, 7--8.~\provinceLivonija, 9--10.~\provinceMemel}%
  \defgroup{GK}{\group{Kurland}}{1--5.~\provinceKurland,
    6--10.~\provinceLivonija}
  \defgroup{GT}{\group{Teutonics}}{1--2.~\provincePreussen,
    3--6.~\province{West Pommern}, 7--10.~\province{Ost Pommern}}%
  \defgroup{UU}{\group{Ukraine}}{1.~\provinceDon, 2.~\provinceDonets,
    3--4.~\provinceZaporozhye, 5-6.~\provincePoltava, 7-8.~\provincePodolie,
    9--10.~\provinceUkraine}%
  \defgroup{RB}{\group{Russia}}{1--2.~\provinceKaluga,
    3--4.~\provinceNovgorod, 5--6.~\provinceNeva, 7--8.~\provincePskov,
    9--10.~\province{Dikoe Pole}}%
  \defgroup{SB}{\group{Carpathians}}{1--5.~\provinceKarpatok,
    6--10.~\provinceBukovina} \defgroup{SC}{\provinceBrandenburg}{}%
  \defgroup{SP}{\group{Prussian Core}}{1--5.~\provinceAltmark,
    6--10.~\provinceNeumark}%
  \defgroup{SM}{\group{Moravia}}{1--5.~\provinceLausitz,
    6--9.~\provinceSilesie, 10.~\provinceMorava}%
  \defgroup{Sb}{\provinceBoheme}{}%
  \defgroup{SG}{\group{Great Prussia}}{1--3.~\provinceBerg,
    4.~\provinceNassau, 5--7.~\province{West Preussen}, 8--9.~\provinceDanzig,
    10.~\provinceWielkopolska}%
  \defgroup{SH}{\group{Hansa}}{1--4.~\provinceMecklenburg,
    5--6.~\provinceLubeck, 7--8.~\provinceHolstein, 9--10.~\provinceBremen}%
  \defgroup{Ss}{\group{Saxony}}{1--7.~\provinceAnhalt,
    8--10.~\provinceSachsen}%
  % Jym : rules allows for COL/TP of POL/PRU, these are safe of revolts.
  \centerline{%
    \graytabular\begin{tabular}{|c|cc|c|}\hline%
      Result & \POL I--IV & \POL V--VII & \PRU \\\hline\ghline%
      <0& \PM & \PM & \SC\\\ghline%
      0 & \PM & \PM & \SC\\\ghline%
      1 & \LL & \LL & \SM\\\ghline%
      2 & \PC & \PC & \SP\\\ghline%
      3 & \LS & \PC & \SP\\\ghline%
      4 & \UU & \UU & \GT\\\ghline%
      5 & \UU & \UU & \GL\\\ghline%
      6 & \PB & \PB & \SG\\\ghline%
      7 & \GT & \GP & \SM\\\ghline%
      8 & \GL & \GK & \Sb\\\ghline%
      9 & \PC & \LS & \Ss\\\ghline%
      10& \RB & \PM & \SP\\\ghline%
      11& \UU & \UU & \SM\\\ghline%
      12& \SB & \RB & \GL\\\ghline%
      13& \RB & \RB & \SH\\\hline\ghline%
    \end{tabular}}
  \showgrouplist{PB,PM,SB,PC,GK,LL,GP,RB,LS,GT,UU}{SG,SH,GL,SM,SP,Ss,GT}

  \caption{Revolt table for \POL and
    \PRU}\label{table:alt-revolt-poland-prussia}
\end{tablehere}

\clearpage



\subsection{Revolt tables for \RUS}

\aparag When a \REVOLT occurs in \RUS, roll on this table, in the column of
the current period.

\bparag If \RUS owns provinces of the \regionPerse, check for
\xnameref{chSpecific:Persia:uprising}.

\begin{tablehere}
  \defgroup{RC}{\group{Capitals}}{1--3.~\provinceMoskva,
    4--10.~\ville{Saint-Petersbourg} (or \provinceMoskva if not built)}%
  \defgroup{RR}{\provinceRyazan}{}%
  \defgroup{RO}{\group{ROTW}}{A random \TP/\COL (any nationality) in
    \continentSiberia.}%
  \defgroup{RN}{\group{Northern Russia}}{1--3.~\provinceLadoga,
    4--6.~\provinceKexholm, 7.~\provinceOnega, 8--10.~\provinceYaroslavl}%
  \defgroup{RW}{\group{Western Russia}}{1--2.~\provinceKaluga,
    3--4.~\provinceRyazan, 5.~\provinceNeva, 6--10.~\provinceNovgorod}%
  \defgroup{WW}{\group{Baltic lands}}{1--5.~\provincePskov,
    6--7.~\provinceKarelen, 8--9.~\provinceEstland, 10.~\provinceLivonija}%
  \defgroup{RU}{\group{Uralic Russia}}{1--2.~\provinceVyatka,
    3--4.~\provinceBolgars, 5--6.~\provinceStep, 7--8.~\provinceBashkiria,
    9--10.~\provinceUral}%
  \defgroup{KK}{\group{Kazan}}{1--2.~\provinceKazan, 3--4.~\provinceTatarstan,
    5--6.~\provinceCheboksary, 7--8.~\provinceMordoviya,
    9--10.~\provinceSamara}%
  \defgroup{KC}{\group{Crimea}}{1--2.~\provinceKhadzhibei,
    3--4.~\provinceZaporozhye, 5--6.~\provinceCrimee, 7--8.~\provinceCaffa,
    9--10.~\provinceAzov}%
  \defgroup{KS}{\group{Caucasus}}{1--2.~\provinceAstragan,
    3--4.~\provinceTerek, 5.~\provinceKuban, 6--7.~\provinceGeorgie,
    8--9.~\provinceDagestan, 10.~\provinceShirvan}%
  \defgroup{KD}{\group{Don}}{1--3.~\province{Dikoe Pole}, 4--7.~\provinceDon,
    8--10.~\provinceDonets}%
  \defgroup{UN}{\group{Northern Ukraine}}{1--5.~\provinceSeveria,
    6--10.~\provincePoltava}%
  \defgroup{UU}{\group{Cossacks}}{1.~\province{Dikoe Pole}, 2.~\provinceDon,
    3.~\provinceDonets, 4--5.~\provinceSeveria, 6.~\provincePoltava,
    7.~\provincePodolie, 8--10.~\provinceUkraine}%
  \defgroup{WS}{\group{Smolensk}}{1--5.~\provinceSmolenska,
    6--8.~\provincePolacak, 9--10.~\provinceBaltarusija}%
  \defgroup{WL}{\group{Lithuania}}{1--2.~\provinceLietuva,
    3--5.~\provinceZemaitija, 6--10.~\provincePrypec}%
  \centerline{%
    \graytabular\begin{tabular}{|c|ccccc|}\hline%
      Result & I--II &III--IV &V & VI & VII\\\hline\ghline%
      <0& \RC & \RO & \RO & \RC & \RO \\\ghline%
      0 & \RC & \RC & \RC & \RO & \RO \\\ghline%
      1 & \RN & \RN & \RN & \RN & \RN \\\ghline%
      2 & \RW & \RW & \RW & \RW & \RW \\\ghline%
      3 & \RU & \RU & \RU & \RU & \RU \\\ghline%
      4 & \WW & \KK & \KK & \KC & \WW \\\ghline%
      5 & \KK & \KK & \KK & \KK & \KK \\\ghline%
      6 & \KS & \KS & \KS & \KC & \KC \\\ghline%
      7 & \UN & \UN & \UU & \UU & \WL \\\ghline%
      8 & \KC & \KC & \UU & \UU & \UU \\\ghline%
      9 & \WS & \WS & \WS & \WS & \WS \\\ghline%
      10& \RR & \WW & \WW & \WW & \WW \\\ghline%
      11& \KD & \KD & \KD & \KD & \RC \\\ghline%
      12& \RW & \RU & \RO & \KS & \RO \\\ghline%
      13& \RU & \RU & \KC & \WL & \KS \\\hline\ghline%
      % pI: Western+Uralic
      % RM RM RN RW WW RU KK KS UU KC WS RR KD RW WW
      % KK KS UU WS RN RU RW
      % Kazan Caucasus Cossacks Smolensk North Uralic Western
      % KC WW KD RM RO
      % Crimea Baltic Don Moskva ROTW
    \end{tabular}}
  \showgrouplist{WW,RC,KS,UU,KC,KD,KK,WL,RN,RO,WS,UN,RU,RW}{}

  \caption{Revolt table for \RUS}\label{table:alt-revolt-russia}
\end{tablehere}

\clearpage



\subsection{Revolt table for \VEN and \TUR}

\aparag When a \REVOLT occurs in \VEN or \TUR, roll on this table, in the
column of the correct country and current period.

\bparag If \TUR owns provinces of the \region{Perse}, check for
\xnameref{chSpecific:Persia:uprising} if a revolt occurs in \TUR (not in
\VEN).

\begin{tablehere}
  \defgroup{VC}{\provinceVeneto}{}%
  \defgroup{VI}{\group{Italy}}{1--2.~\provinceMantova, 3--5.~\provinceRomagna,
    6.~\provinceLombardia, 7.~\provinceModena, 8.~\provinceParma,
    9.~\provinceLucca, 10.~\provinceTrentino}%
  \defgroup{VA}{\group{Adriatic}}{1--2.~\provinceFriuli,
    3--4.~\provinceIstria, 5--6.~\provinceKapela, 7--8.~\provinceDalmacija,
    9--10.~\provinceMontenegro}%
  \defgroup{VZ}{\seazoneAdriatique}{}%
  \defgroup{TI}{\group{Outposts}}{A random Venetian \TP; if none,
    \provinceIzmir}%
  \defgroup{MI}{\group{Islands}}{1--2.~\provinceCyclades,
    3--4.~\provinceCorfu, 5--6.~\provinceKreta, 7--8.~\provinceRhodos,
    9--10.~\provinceChypre}%
  \defgroup{BA}{\group{Balkans}}{1--2.~\provinceMoreas, 3--4.~\provinceHellas,
    5--6.~\provinceMontenegro, 7.~\provinceBosna, 8.~\provinceDalmacija,
    9.~\provinceSerbia, 10.~\provinceAlabania}%
  \defgroup{RO}{\group{ROTW}}{At random between \provinceIzmir, and \COL (any
    side)/\TP\faceplus of \TUR}%
  \defgroup{TX}{\provinceTrakya}{}%
  \defgroup{TA}{\group{Anatolia}}{1.~\provinceAntalya, 2.~\provinceBursa,
    3.~\provinceKocaeli, 4.~\provinceSinop, 5.~\provinceTrabzon,
    6.~\provinceAngora, 7.~\provinceIzmir, 8.~\provinceKonya,
    9.~\provinceAnadolu, 10.~\provinceKilikya}%
  \defgroup{TR}{\group{Romelia}}{1.~\provinceCanakkale, 2.~\provinceMakedonya,
    3.~\provinceRumeli, 4.~\provinceBulgaristan, 5--6.~\provinceValahia,
    7--8.~\provinceBasarabia, 9--10.~\provinceMoldova}%
  \defgroup{TC}{\group{Outer Empire}}{1--2.~\ctz{Turquie},
    3--4.~\provinceMalta, 5--6.~\provinceSzlovakia, 7--8.~\provinceCarniola,
    9--10.~\provinceBalaton}%
  \defgroup{TH}{\group{Hungary}}{1--2.~\provinceMagyarorszag,
    3.~\provinceCroatie, 4.~\provinceKapela, 5.~\provincePecs,
    6.~\provinceBanat, 7.~\provinceErdely, 8.~\provinceKarpatok,
    9.~\provinceBukovina, 10.~\provinceMures}
  \defgroup{TS}{\group{Sultanates}}{1.~\provinceAlep, 2.~\provinceSyrie,
    3.~\provinceLubnan, 4.~\province{Terra Sancta}, 5.~\provinceNil,
    6.~\provinceDelta, 7.~\provinceNubie, 8.~\provinceEgypte,
    9.~\provinceCataractes, 10.~\provinceTobrouk and \provinceSinai (Revolts
    strength at \bonus{-10}, possibly no revolt if Strength<2)}%
  \defgroup{TK}{\group{Caucasus}}{\provinceGeorgie, \provinceKuban,
    \provincePodolie, \provinceHacibey, \provinceUkrainya, \provinceShirvan,
    \provinceDagestan, \provinceCaffa}%
  \defgroup{TM}{\group{Arabs}}{1--2.~\provinceCyrenaique,
    3.~\provinceJordanie, 4--5.~\provinceIrak, 6--7.~\provinceBassorah,
    8.~\provinceNefud, 9--10.~\provinceTripoli}%
  \defgroup{Tp}{\group{Persia}}{1--2.~\provinceAzarbayadjan,
    3--4.~\provinceArmenie, 5--6.~\provinceKordistan, 7.~\provinceTigre,
    8.~\provincePars, 9.~\provinceVan, 10.~\provinceKermanshah}%
  \centerline{%
    \graytabular\begin{tabular}{|c|c|cccc|}\hline%
      Result & \VEN & \TUR I--II & \TUR III--IV & \TUR V--VI & \TUR VII
      \\\hline\ghline%
      <0& \VC & \TX & \TX & \TX & \TA\\\ghline%
      0 & \VC & \TA & \TA & \TA & \TR\\\ghline%
      1 & \VI & \TR & \TR & \TR & \TK\\\ghline%
      2 & \VZ & \TK & \TK & \TK & \TM\\\ghline%
      3 & \VA & \TS & \TS & \TS & \TS\\\ghline%
      4 & \VZ & \TP & \TP & \TH & \TS\\\ghline%
      5 & \VA & \TA & \TS & \TS & \TS\\\ghline%
      6 & \MI & \BA & \TH & \BA & \TH\\\ghline%
      7 & \BA & \BA & \BA & \BA & \BA\\\ghline%
      8 & \MI & \Tp & \Tp & \Tp & \Tp\\\ghline%
      9 & \MI & \TH & \TH & \TH & \TH\\\ghline%
      10& \BA & \TC & \TC & \TC & \TC\\\ghline%
      11& \VI & \MI & \MI & \MI & \MI\\\ghline%
      12& \VZ & \RO & \RO & \RO & \RO\\\ghline%
      13& \TI & \TM & \TM & \TM & \TM\\\hline\ghline%
    \end{tabular}}
  \showgrouplist{VA,BA,MI,VI,TI}{TA,TM,BA,TK,TH,Tp,TC,TR,RO,TS}

  \caption{Revolt table for \VEN and
    \TUR}\label{table:alt-revolt-venice-turkey}\end{tablehere}

\clearpage

%%%%%%
\definechapterbackground{Political Events}{politicalevents}
\chapter{Political Events}

pI:TD:Diplomatic pressures
% -*- mode: LaTeX; -*-

\section{Period I}\label{events:pI}



\subsection*{Event Table of Period I}

\begin{eventstable}[Period I events table]
  \tabcolsep=5pt\centering
  \begin{tabular}{|l|*{5}{c}|l|}
    \hline
    1\up{st}\textarrow & 1-4 & 5-6 & 7 & 8 & 9 & 10 \\\hline
    1 & 1  & R2 & 3   & R15 & R16 & \textbullet~1--2 \\
    2 & 1  & 3  & R11 & R14 & 3   &  +1 then \\
    3 & 1  & 10 & R12 & 4   & R11 & \nameref{events:pII}\\
    4 & 3  & 12 & R4  & 7   & R15 & \textbullet~3--10: \\
    5 & 5  & 13 & R8  & 11  & R4  & \nameref{events:pII} \\
    6 & R6 & 4  & R4  & R6  & R8  & \\
    7 & R2 & R6 & R5  & R8  & R3  & \\
    8 & 7  & 9  & 8   & 9   & R16 & \\
    9 & 11 & 13 & 3   & 10  & R7  & \\\hline
    10 & \multicolumn{5}{l}{\nameref{events:pII}} & \\\hline
  \end{tabular}
\end{eventstable}
\begin{eventstablespec}[General modifiers for the period]
  % \oldref{Rule 53.23.A, modified} \\
  After \continentAmerica has been discovered and until \ref{pI:Tordesillas}
  is rolled-for the first time, use the following modifiers for both dice each
  turn when rolling for events (a result less than 1 is 1):
  \begin{modlist}
  \item[\bonus{-1}] If \SPA is \CATHCR or \ref{pI:Reformation2} has not
    occurred;
  \item[\bonus{-1}] If new \COL/\TP counters were placed in \continentAmerica
    last turn;
  \item[\bonus{-1}] If \SPA or \POR control \payspapaute.
  \end{modlist}
\end{eventstablespec}
% grep '^.evnt' engEvnt1.tex|cut -f2 -d\[|cut -f1 -d\]|sed -e 's/$/|,%/g'#$

\eventssummary{%
  pI:Tordesillas|,%
  pI:Emperor Election|,%
  pI:War Italy Napoli|,%
  pI:War Italy Milano|,%
  pI:Hungarian Freedom|,%
  pI:Bohemian Alliance|,%
  pI:Hungarian Alliance|,%
  pI:Milanese Alliance|,%
  pI:Habsburg Dynasty|E/E,%
  pI:Revolt Comuneros|,%
  pI:Reformation|,%
  pI:Reformation2|,%
  pI:Reformation3|,%
  pI:Turkish Diplomatic Pressure|S{pI:TD:Barbaross
    brothers}/S{pI:TD:Vassalisation of Algeria}/S{pI:TD:Alignment
    Barbaresques}/E/E/E
,%
  pI:War Scotland|,%
  pI:End Golden Horde|,%
  pI:Pskov Ryazan|,%
} \eventssummary{%
  pI:War Russia Poland|,%
  pI:War Roads Spices|S{pI:WRS:War Indian Sea}/S{pI:WRS:Veneto-Turkish
    Commercial Dispute},%
  pI:Resistance Muslim Traders|,%
  pI:Chinese Expeditions|,%
  pI:Barbaros Brothers|,%
  O|,%
  pI:Habsburg Alliance|,%
  pI:Burgundy Inheritance|,%
  pI:Habsburg Bohemia|,%
  pI:Habsburg Hungary|,%
  pI:Fall Hungary|,%
  pI:Habsburg Milano|,%
  pI:Spanish Milano|,%
  pI:Fall Teutonic|,%
  pI:Spanish Naples|,%
}

\newpage\startevents



\event{pI:Tordesillas}{I-1}{Treaty of Tordesillas}{1}{RistoMod}

\history{1494}
\dure{end of Period III, or until \ref{pIII:Portuguese Annexation}, whichever
  comes first}

\condition{}
\aparag Re-roll and do not mark off if \continentAmerica has not been
discovered.
\aparag Both \SPA and \POR have to accept this event for it to take
effect. Otherwise it is marked off, but can occur again.

\phevnt
\aparag \FRA and \ENG receive a temporary \CB for this turn to declare war
against both \SPA and \POR.
\aparag \SPA and \POR receive each 50\ducats.

\effetlong
\aparag From now on \SPA and \POR have specific areas for overseas expansions:
\bparag The exclusive area of \POR contains \continent{Middle East},
\continentSiberia, \continentAsia (except \granderegionPhilippines,
\continent{Extreme Orient}), \continentAfrica, \continentBrazil.
\bparag The exclusive area of \SPA contains \continentAmerica except
\continentBrazil, \granderegionAmazonia, and \granderegion{Minas Gerais}.
\bparag The regions \granderegionAmazonia, \granderegion{Minas Gerais},
\granderegionPhilippines, and \continent{Extreme Orient} are shared.
\aparag[Effects on \SPA and \POR]
\bparag All markers of \SPA or \POR currently on map in the exclusive area of
the other \MAJ are immediately destroyed, or may be replaced by the other \MAJ
by equivalent markers of its own if there are some available, and the \MAJ
fulfils the conditions to place such a marker here (especially discoveries).
\bparag Their movements are limited to their areas, the sea zones bordering
them, and sea zones that borders only islands. \POR may also go in sea zones
\seazone{Horn} and \seazone{Chili}. \SPA may also go in
\seazone{Bonne-Esperance} and \seazone{Tempetes}.
\bparag For each stack of the country violating this restriction, this country
loses {\bf 1} \STAB per restricted province or sea zone trespassed in. All
units of \SPA or \POR in prohibited zones when the Treaty is signed must
immediately return home as per normal peace procedure.
\bparag Until the end of period II, \SPA and \POR have free overseas
\terme{CB} against any Minor country in his area, and against any European
country (Catholic or not) trespassing their area. The free overseas \terme{CB}
might be used in reaction at the end of a round where a trespassing occurs, or
at the beginning of the next turn.
\bparag Until the end of period \period{III}, \SPA and \POR may attack
\terme{Minor establishments} in their area at no cost in \STAB.
\bparag Until the end of period III, \SPA and \POR have the capability to burn
down \COL installed in their area (same condition as for burning \TP).
\bparag Spanish Missionaries and Missions can only go in the exclusive area of
\SPA until the end of the Treaty (not in the shared area).
\aparag[Shared Area]
\bparag The regions that can be disputed between \POR and \SPA can be explored
by both countries and they can settle \COL and \TP without penalty.
\bparag If \POR has a \TP or \COL in \granderegionPhilippines,
\granderegionAmazonia or \granderegion{Minas Gerais}, \SPA gains an Overseas
\CB against \POR.
\bparag If \SPA has a \TP or \COL in \granderegionPhilippines or
\continent{Extreme Orient}, \POR has an Overseas \CB against \SPA.
\aparag[Effects on other countries] Until the end of period II all \COL/\TP
placement attempts (successful or not), and any movement of units from the
European Map into \ROTW (not in the other way) by Catholic players other than
\SPA or \POR entail to this \MAJ a malus of \bonus{-2} to \STAB improvement
action and the loss of control of \payspapaute. This malus is applied at most
once per turn.
\aparag The \terme{Treaty of Tordesillas} can be declared void by \POR or \SPA
as a Diplomatic Announcement. The Treaty is at an end, the other power gains
an immediate free \terme{CB} against the announcer; the announcer loses
diplomatic control of \payspapaute.



\event{pI:Emperor Election}{I-2}{Election of the \HRE Emperor}{1}{RistoMod}

\history{1519}
\dure{until the Emperor is Habsburg.}

\phevnt
\aparag Election of the new Emperor has to be conducted. The pretenders are
the monarchs of \SPA, \FRA, \ENG, and \POL if they are Catholic.  Each
Pretender makes a secret bid of Ducats (a multiple of 10\ducats) for the
title, and the country with the highest bid will win. In case of draw, those
Pretenders bid again secretly an additional bid. All bids are lost.
\aparag A candidate from one (unspecified) minor country makes a bid of 2d10
times 10\ducats (rolled for after the initial bids of the players ; in case of
ties, this candidate bids in addition of 1d10 times 10\ducats). If the winner
of the bid, and so the Emperor is a from a Minor Power, he will live 1d10
turns before a new election takes place. The event is still marked.
\aparag Each Minor Country that has the Electoral Dignity: \paysCologne,
\paysPalatinat, \paysSaxe, \paysTreves, \paysMayence, \paysBrandebourg,
\paysBoheme gives a free bid of 10\ducats to the Pretender secretly decided at
the bidding time by the Major Country having Diplomatic alliance with the
Electorate. If they are Neutral, they give their bid to the pretender of the
minor country
\aparag At the first election, the House of Fugger provides an immediate and
mandatory national loan of 50\ducats for this election
% (interest rate of 5\% for a duration of 5 turns)
to \SPA, or 100\ducats if the monarch of \SPA is \monarque{Charles V}, that
are directly put in the RT.
\aparag The winner of the Crown gains 75\ducats and 10 VP.
\aparag The new Emperor has now the benefice pertaining to the \HRE,
see~\ruleref{chSpecific:HRE}.
\aparag If the Emperor is Spanish and \ref{pI:Habsburg Alliance} is in effect,
\SPA gains the possibility to involve \HAB in all wars in which \SPA is
currently involved (both as attacker and defender), but not conversely. This
is made with a free \CB.
\aparag If the Emperor is \FRA, \ENG or \POL, \SPA may declare a war (with no
\CB) against the new Emperor and \AUSaus will help in a offensive alliance. A
valid victory condition for \SPA is to cause immediately a new election where
the losing power can not be a candidate.

\effetlong
\aparag When the elected emperor dies or converts to another religion, a new
election occurs with the same system during the very next Event Phase.
\aparag However, if the dying Emperor is Spanish, the event terminates
permanently and no elections are held. The title of emperor reverts back to
\AUSaus and all effects of the event are cancelled. Furthermore, the
\dynasticaction{C}{2} is activated now if possible.
% C2=Spanish Milanese



\event{pI:War Italy Napoli}{I-3 (1)}{Wars in Italy (Napoli)}{1}{RistoMod}

\history{1494-1504, 1508}

\condition{Mandatory War.}
\aparag If \FRA is Protestant, mark off the event but apply \RD with the
\REVOLT in \FRA.
\aparag If \paysNaples exists no more, mark off the event, then apply and mark
off the second event.
\aparag The second event can not take place if the first one is not
finished. In that case re-roll and do not mark off. \phevnt
\aparag \FRA has a Mandatory \CB against \paysNaples. This \CB has to be used
this turn or the next, at the phase of Declaration of War. If the \CB is used,
the controller of \paysNaples may abandon the minor country with no cost, even
if it is own \VASSAL (because of valid Dynastic Claims of the French King).
\aparag If \FRA is already at war against this country, the war is linked to
this event at this turn and that fulfils the Mandatory \CB.

\phdipl
\aparag[Refusing the event]
\bparag At the very beginning of the Declarations Phase, \FRA may refuse the
event.
\bparag If \FRA refuses the event, it loses {\bf 2} \STAB and the rest of the
event is ignored.
\aparag[Entry in War of the Italian countries]
\bparag The following countries may be involved by themselves in the war:
\paysGenes, \paysMilan, \payspapaute, \paysSavoie, The following tests are
made each turn of the war (excepted if the \MIN was already forced out of the
war by a separate peace).
\bparag Those countries in the list that are allied to a \MAJ involved in the
war, make a mandatory test of Entry in the War as per the usual rules
\ruleref{chDiplo:Entry War Minor}, excepted that the \MAJ has no choice here
and this test is made even if the \MIN is not in \EG; if the \MIN is not in
\EG at least, use \bonus{-2} to the die roll in the test and a failure does
not change the diplomatic status of the \MIN.
\bparag Those countries in the list that are \Neutral, may join the following
\MAJ according to the roll of 1d10: 1 \FRA, 2-3 \HAB, 4 \VEN, 5-6 enters war
by itself, 7-10 stays \Neutral. A country joins a \MAJ only if it is involved
in the war ; it is then put in \EG of this \MAJ, and declares war of the
enemies of this \MAJ. If the \MAJ is not involved in the war, the \MIN stays
\Neutral.
\bparag If a \MIN enters war by itself, it declares war to all involved
countries then it asks help of the preferred country in its list that is not
one of its enemies.
\aparag[Diplomatic effects of the wars] \FRA has a bonus of \bonus{+2} for its
diplomacy on \paysToscane and \bonus{-1} for \payspapaute and \paysParme
during the event.
\aparag[The Serenissima in the Wars in Italy]
\bparag \VEN has a \CB against \FRA and/or \paysNaples, as long as the war is
not finished.
\bparag During this war also, \VEN may make limited intervention at the side
of any involved alliance each turn. Such limited intervention can begin at any
turn (not only the first) and \VEN can change side between turns. \VEN may
force any Italian \MIN in limited intervention for the enemy alliance, to be
fully involved in the war.
\bparag Conversely, \FRA and \HAB both have a free \CB against \VEN, to be
used at any turn of the war or on the turn following its conclusion.

\phmil
\aparag[First turn of the war] \FRA has the right of free access and supply in
all Italian minor countries not engaged in this war. Supply is not given by a
province if its city is besieged by country hostile to this city.
\aparag[Restricted War Field]
\bparag The war is restricted to \regionItalie if no side broadens the zone of
war.
\bparag The war is no more restricted if the side of \FRA invades a province
outside \regionItalie of the other side. \FRA loses immediately {\bf 1} \STAB
and 20 \VP, and if the invasion was not due to \FRA, the Major Power
responsible for it loses also {\bf 1} \STAB and 20 \VP. However, if dynastic
actions \shortdynasticaction{A}{1} and \shortdynasticaction{A}{2} have both
been played, the penalty in \VP is void.
\bparag The war is also no more restricted if the side enemy of \FRA invades a
province of the side of \FRA outside \regionItalie and that stack does not
draw its supply from \regionItalie.
\aparag At the time a stack of \FRA invades \provinceCampania, \FRA, \SPA and
\HAB gain free access in, but only supply across, Italian minor countries not
engaged in this war. Supply across a province is impossible if its city is
under siege by an enemy of this city.

\phinter
\aparag If \FRA does not manage the military conquest of \ville{Napoli} at any
time of this war, it loses 10 \VP at the end of the event.
\aparag If, on the contrary, \FRA annexes \provinceCampania, it gains 10\VP.
\aparag[Spanish reaction] \SPA has to choose to do \dynasticaction{A}{3} as
one of its diplomatic action on the turn following the beginning of the war
(this will use a Diplomatic action, with no cost and automatic success, but
\SPA is allowed another Dynastic Action this turn), thus activating event
\ref{pI:Spanish Naples} or renounces to its Inheritance: it then loses {\bf 3}
\STAB, and \dynasticaction{A}{3} is considered played for no effect.

\effetlong
\aparag If at any time of this war \FRA manages the military conquest of
\ville{Napoli}, it gains a \CB against \TUR for the rest of the period.
Moreover, \FRA may now annex \provinceTrakya until the end of the period.
\aparag Until the end of the current period, \FRA has a permanent \CB against
the owner of \provinceCampania.



\event{pI:War Italy Milano}{I-3 (2)}{Wars in Italy (Milano)}{1}{RistoMod}

\history{1510-1511 / 1513-1515}
\dure{Until the end of the war caused by this event.}

\condition{Mandatory War.}
\aparag If \FRA is Protestant, marked off the event but apply \RD with the
\REVOLT in \FRA.
\aparag The second event can not take place if the first one is not
finished. In that case re-roll and do not mark off.

\phevnt
\aparag \FRA has a Mandatory \CB against the owner of \provinceLombardia. This
\CB has to be used this turn or the next, at the phase of Declaration of
War. If \FRA is \CATHCR after \ref{pI:Reformation}, the \CB is free.
\aparag If \FRA is already at war against this country, the war has to become
the war linked to this event at this turn or the following (the choice is made
by \FRA during the Declarations of War) and that fulfils the Mandatory \CB.
\aparag If \FRA owns \provinceLombardia, any former owner of this province has
a free \CB against \FRA.

\phdipl
\aparag[Refusing the event]
\bparag At the very beginning of the Declarations Phase, \FRA or the owner of
\provinceLombardia may refuse the event.
\bparag If \FRA refuses the event, it loses {\bf 2} \STAB and the rest of the
event is ignored.
\bparag If the owner of \provinceLombardia refuses the event, it loses {\bf 3}
\STAB and gives \provinceLombardia to \FRA (or its former controller if it was
\FRA that refused the event). Then the rest of the event is ignored. If this
province is owned by the \HAB, \SPA may refuse the event (and lose the \STAB).
\aparag[Milan as a Minor country] If \provinceLombardia is owned by the Minor
country \paysMilan, \HAB have a free \CB in reaction to a Declaration of War
of \FRA against this country. \paysMilan is moved up to \EG on the diplomacy
track of \HAB if it was not already on a higher position.
\aparag[The Papacy and the war] If \payspapaute is allied to a \MAJ involved
in the war, each turn make a mandatory test of Entry in the War is made as per
the usual rules \ruleref{chDiplo:EW Effects}, excepted that the \MAJ has no
choice here and this test is made even if the \MIN is not in \EG; if the
\paysPapaute is not in \EG at least, use \bonus{-2} to the die roll in the
test and a failure does not change its diplomatic status. Exception: if
\payspapaute was forced out of this war, it does not enter back in it.
\aparag[Diplomatic effects of the wars] \FRA has a bonus of \bonus{+2} for its
diplomacy on \paysToscane and \bonus{-1} for \payspapaute and \paysParme
during the event.
\aparag[The Serenissima in the Wars in Italy]
\bparag \VEN has a \CB against \FRA and/or the owner of \provinceLombardia, as
long as the war is not finished.
\bparag During this war, \VEN may make limited intervention at the side of any
involved alliance each turn. Such limited intervention can begin at any turn
(not only the first) and \VEN can change side between turns.  \VEN may force
any Italian \MIN in limited intervention for the enemy alliance, to be fully
involved in the war.
\bparag Conversely, \FRA and \HAB both have a (normal) \CB against \VEN, to be
used at any turn of the war.
\aparag[Swiss Mercenaries] If \paysMilan is (or was) a vassal of \HAB
(according to \ref{pI:Habsburg Milano}), \HAB may gain \paysSuisse in \CE
automatically at the cost of one Diplomatic action.

\phmil
\aparag[Restricted War Field]
\bparag The war is restricted to \regionItalie if no side broadens the zone of
war.
\bparag The war is no more restricted if the side of \FRA invades a province
outside \regionItalie of the other side. \FRA loses immediately {\bf 1} \STAB
and 20 \VP and if the invasion was not due to \FRA, the Major Power
responsible for it loses also {\bf 1} \STAB and 20 \VP. However, if dynastic
actions \shortdynasticaction{A}{1} and \shortdynasticaction{A}{2} have both
been played, the penalty in \VP is void.
\bparag The war is also no more restricted if the side enemy of \FRA invades a
province of the side of \FRA outside \regionItalie and that stack does not
draw its supply from \regionItalie.
\aparag \paysSavoie gives free access and supply in its province to \FRA
during the first turn of the war, if it stays neutral in this war. Supply from
or across a province is impossible if its city is under siege by an enemy of
this city.
\aparag If \ref{pI:Habsburg Milano} was not played and \FRA besieges the city
of \provinceLombardia with at least one \ARMY\faceplus, it takes the city
without resolving the siege and annexes immediately the province; \FRA may
destroy the Minor country \paysMilan by this way.

\effetlong
\aparag[Passing through \paysSavoie]
\bparag At the instant \FRA annexes \provinceLombardia during the war, it
gains from \paysSavoie free access and supply through its provinces (but no
stopping in, or supply from) when at peace with \FRA. Supply across a province
is impossible if its city is under siege by an enemy of this city.
\bparag This right is void if/when \FRA is at war against \paysSavoie, and is
permanently lost if \FRA loses \provinceLombardia.
\bparag Enemies of \FRA gain the same right when at war with \FRA.
\aparag At the end of this event, if the Minor country \paysMilan still
exists, \dynasticaction{B}{2} is played then \HAB annexe all its provinces and
the minor country disappears.
\aparag Until the end of the current period, \FRA has a \CB against the owner
of \provinceLombardia.



\event{pI:Hungarian Freedom}{I-4 (1)}{Declaration of Hungarian
  Freedom}{1}{RistoMod}

\history{1505}

\condition{If \ref{pI:Habsburg Hungary} or \ref{pI:Fall Hungary} has already
  been activated, mark off but play \RD.}

\phevnt
\aparag The Hungarian Inheritance (\ref{pI:Habsburg Hungary}) that might be
pending is now impossible.
\aparag \POL has the immediate choice of supporting a Jagiellon dynasty in
\paysHongrie. If it does, it gains \paysHongrie in \MR at once, makes a white
peace with it if necessary, and gains a temporary \CB against any countries at
war against \paysHongrie.
\bparag Else, \paysHongrie becomes \Neutral.
\aparag \HAB has a temporary \CB against \paysHongrie. See also \ref{pI:Fall
  Hungary} that might happen.

\effetlong The \dynasticaction{C}{1}, the events \xref{pI:Hungarian Alliance}
and \xref{pI:Habsburg Hungary} are no more possible and will be ignored.



\event{pI:Bohemian Alliance}{I-4 (2)}{Dynastic Alliance with
  \paysBoheme}{1}{RistoMod}

\history{1526}

\condition{If \ref{pI:Habsburg Bohemia} has already been played, mark off and
  play \RD.}

\phevnt
\aparag The \dynasticaction{B}{1} is played, and it activates \ref{pI:Habsburg
  Bohemia}.



\event{pI:Hungarian Alliance}{I-5}{Dynastic Alliance with
  \payshongrie}{1}{RistoMod}

\history{1491, not activated}
\dure{until the activation of \ref{pI:Habsburg Hungary} or \ref{pI:Habsburg
    Bohemia}, or the \ref{pI:Hungarian Freedom}}

\condition{If \shortdynasticaction{C}{1} or \ref{pI:Hungarian Freedom} has
  already been played, mark off and play \RD.}

\phevnt
\aparag The \dynasticaction{C}{1} is played, and consequently \ref{pI:Habsburg
  Hungary} is pending.
\aparag \POL gains a temporary \CB against \paysHongrie.

\phdipl
\aparag At the beginning of each diplomatic phase, the diplomatic status of
\paysHongrie moves one level toward the track of \HAB, up to \EG. This ends if
\ref{pI:Fall Hungary}, \ref{pI:Hungarian Freedom} or \ref{pI:Habsburg Hungary}
happens.



\event{pI:Milanese Alliance}{I-6}{Dynastic Alliance with Milano}{1}{Risto}

\condition{If \shortdynasticaction{C}{1} or \ref{pI:Hungarian Freedom} has
  already been played, mark off and re-roll.}

\aparag The \dynasticaction{B}{2} is played, and it activates \ref{pI:Habsburg
  Milano}.



\event{pI:Habsburg Dynasty}{I-7 (1)}{Habsburg Dynastic Action}{2}{PBNew}

\phevnt
\aparag \SPA may immediately play one dynastic action of its choice, without
test nor cost.
\bparag This action may be an annexation of one of the Provinces of the
North-East, if applicable.



\event{pI:Revolt Comuneros}{I-7 (2)}{Revolt of the Comuneros}{1}{PBNew}

\history{1520-1522}

\phevnt
\aparag Place one \REVOLT\facemoins in \provinceToledo, one Rebel
\ARMY\facemoins, \LD with a minor \LeaderG. The rebels control the fortress
(reduced to level 2 max if need be).
\aparag Draw at random 2 other provinces where a \REVOLT\facemoins is placed,
by rolling 1d10: 1-2 \province{La Mancha}, 3-4 \province{Castilla La Nueva},
5-6 \provinceSalamanca, 7-8 \provinceLeon, 9-10 \province{Castilla La Vieja}
\aparag The Rebels are controlled by \RUS (the most remote player designers
could think of).  They will receive no reinforcement (excepted through \REVOLT
extension).



\event{pI:Reformation}{I-8 (1)}{Reformation}{1}{RistoMod}

\history{1517-1560}

\tour{Turn 1}

\phevnt
\aparag[Luther's 95 Thesis] \paysDanemark, \paysSuede, \paysBerg, \paysSuisse,
\paysHanse, \paysprovincesne, \paysHesse, \paysSaxe, \paysHanovre,
\paysOldenburg, \paysBrunswick, and \paysBoheme become Protestant.
\aparag \terme{Religious enmities} begin between Protestant and Catholic
countries. They will end when \ref{pIV:TYW} is terminated, or at the beginning
of period IV if this event ended before, or at the end of period IV if the
event is not yet finished.
\aparag[Orthodoxes in Poland] \POL has to decide of its attitude regarding
Orthodoxy: Conversion, Tolerance or Support.
\bparag The lasting effects are mainly described in
\ruleref{chSpecific:Poland:Orthodoxy}.
\bparag If \POL chooses Support of Orthodoxes, it loses {\bf 2} \STAB and
rolls for 2 \REVOLT on its table.
\bparag If \POL chooses Tolerance of Orthodoxes, it loses {\bf 1} \STAB and
rolls for 1 \REVOLT on its table.
\aparag[Russian Religious Attitude] \RUS has to decide its behaviour regarding
Religions: Championship of Orthodoxy or Religious Tolerance.
\bparag The lasting effects are mainly described in
\ruleref{chSpecific:Russia:Orthodoxy}.
\bparag If \RUS chooses Religious Tolerance, it loses {\bf 2} \STAB and rolls
for 1 \REVOLT on its table.

\tour{Turn 2}

\phevnt
\aparag \paysBrandebourg becomes Protestant. Play \ref{pI:Fall Teutonic} as a
supplementary event this turn.



\event{pI:Reformation2}{I-8 (2)}{Growth of the Reformation}{1}{RistoMod}

\history{1517-1560}

\phevnt
\aparag \FRA, \SPA, \ENG and \POL must choose between \CATHCR, \CATHCO or
Protestantism (forbidden to \SPA). The choice is made simultaneously and
secretly at the beginning of the Phase of Declarations. It cannot be
voluntarily changed later except by events. If \POL has chosen Support of
Orthodoxes, he is complied to choose \CATHCO now.

\phase{Consequence:}{Each country is affected by the following general
  consequences, added to specific effects for each country, described
  afterwards.}
\aparag[\CATHCR]
\bparag If only one of the eligible players chooses \CATHCR, he is permanent
Sole Defender of Catholic Faith and receives 20 \VP.
\bparag If more players choose \CATHCR, the Sole Defender of Catholic Faith is
determined according normal procedure but between them only.
\bparag If none of the eligible players chooses \CATHCR, all of them lose {\bf
  1} additional \STAB.
\bparag A bonus of \bonus{+1} is received for diplomacy on all Catholic
countries until the end of \terme{Religious Enmities}.
\aparag[\CATHCO]
\bparag {\bf 1} \STAB is lost.
\bparag One \REVOLT is rolled in the player country.
\bparag An additional Diplomatic Action is gained and a \bonus{+2} bonus is
received for diplomacy on all Protestant countries until the end of
\terme{Religious Enmities}.
\aparag[Protestantism]
\bparag No Diplomacy (support included) with \payspapaute until the end of the
current period. Control of \payspapaute is lost.
\bparag {\bf 2} \STAB are lost.
\bparag Two \REVOLT are rolled in the player country.

\begin{digressions}[Specific effects]


  \digression[pI:Reformation:France]{\sc France}
  \aparag[Independent \paysVhollande] If \paysVhollande is or comes into play
  before the \xnameref{pV:WoSS}, immediately apply \ref{pIII:Dutch Revolt}.
  \aparag[\CATHCR]
  \bparag Some events (especially \xnameref{pIII:FWR}, \xnameref{pV:Expulsion
    French Protestants}) are modified.
  \aparag[\CATHCO]
  \bparag \bonus{+1} bonus to \STAB improvement attempts this turn and the two
  following ones.
  \aparag[Protestantism]
  \bparag No Diplomacy (support included) with \payspapaute until the end of
  period III.
  \bparag Some events (\xnameref{pI:War Italy Napoli}, \xnameref{pI:War Italy
    Milano}, \xnameref{pII:War Italy}, \xnameref{pIV:La Rochelle},
  \xnameref{pIII:FWR} \xnameref{pV:Expulsion French Protestants},
  \xnameref{pV:Colbertian Mercantilism}) are modified.
  \bparag The turn and period limits of \FRA are changed. \FRA receives an
  explorer for one turn as per~\ref{eco:Explorer}.


  \digression[pI:Reformation:Spain]{\sc Spain}

  \aparag[\CATHCR]
  \bparag Permanent bonus \bonus{+2} for diplomacy on \payspapaute.
  \bparag \SPA gains the possibility of forcing Restoration of Catholicism in
  Protestant countries, with the relevant bonuses.
  \aparag[\CATHCO]
  \bparag A further {\bf -1} in \STAB is applied.
  \bparag A malus of \bonus{-2} to \STAB improvement attempts for the rest of
  the period and the following one
  \bparag Restoration of Catholicism in Protestant countries gives no bonuses.
  \bparag Dynastic actions are no more allowed, except when permitted or
  required by an event.
  \begin{designnote} Future option: modifications of some events: [temporary
    list II-9, III-1, III-7, III-8, III-11, IV-1 and V-8]. As this choice
    might largely change the course of the game, especially for the player of
    \VEN, it is good policy to have part of an agreement with this player
    before choosing this attitude.
  \end{designnote}


  \digression[pI:Reformation:England]{\sc England}

  \aparag[\CATHCR]
  \bparag The turn and period limits of \ENG are changed.
  \aparag[\CATHCO]
  \bparag \bonus{+1} bonus to \STAB improvement attempts this period and the
  following one.
  \aparag[Protestantism]
  \bparag The turn and period limits of \ENG are changed. \ENG receives an
  explorer for one turn as per~\ref{eco:Explorer}.
  \bparag Each time \ENG is rolled-for in the Revolt Country chart, the number
  of \REVOLT is doubled. This continues until the end of period III.
  \aparag The Religious and Civil Wars of \ENG (\xnameref{pII:Act Supremacy},
  \xnameref{pIV:English Civil War}, \xnameref{pV:Glorious Revolution} and
  \xnameref{pVI:Jacobite Rebellion}) depend on its Religious choice.


  \digression[pI:Reformation:Poland]{\sc Poland}

  \aparag[\CATHCR]
  \bparag Some events (\xnameref{pI:Fall Teutonic}, \xnameref{pIII:Union
    Poland Sweden}, \xnameref{pIV:TYW}, \xnameref{pV:Saxon King Poland}) are
  modified.
  \bparag \POL gain a \CB against all Protestant countries until the end of
  period III, and the right to convert them to Catholicism.
  \aparag[\CATHCO]
  \bparag \bonus{+1} bonus to \STAB improvement attempts this turn and the two
  following ones.
  \aparag[Protestantism]
  \bparag The Union of Lublin (see \xnameref{pII:Union Lublin} or
  \xnameref{pIII:Union Lublin}) is broken and will not be possible. Some other
  events (\xnameref{pIV:Bohemian Revolt}) are modified.
  \bparag The turn and period limits of \POL are changed.
\end{digressions}



\event{pI:Reformation3}{I-8 (3)}{Intensification of the
  Reformation}{1}{RistoMod}

\history{1517-1560}

\phevnt
\aparag[Calvin] \paysPalatinat, \paysThuringe and \paysecosse become
Protestant.



% \event{pI:Turkish Diplomatic Pressure}{I-9}{Turkish Diplomatic
%   Pressure}{2}{Risto}

% \history{no specific date}

% \condition{If \leaderBarbaros is in play, \TUR may choose to apply
%   \ref{pII:Algeria Vassalisation} instead.  If \leaderDragut is in play, \TUR
%   may choose to apply \ref{pII:Alignment of Barbaresques} instead.}

% \phdipl
% \aparag \TUR receives a bonus of \bonus{+3} for a Muslim minor of its
% choice. Choice has to be made secretly during the negotiations step.

\event{pI:Turkish Diplomatic Pressure}{I-9}{Turkish
  Dynamism}{*}{RistoMod/PBnew/Jym [BLP]}
% \history{Capture of Algiers: 1516, Recapture of Algiers: 1529,
%   Conquest of Tunis: 1534.}

\phevnt
\aparag \TUR chooses, when all events of this turn have been rolled, to apply
one of the following cases:
\bparag If \leader{Oruc Reis} is alive, \TUR may choose~\ref{pI:TD:Barbaross
  brothers}. This may only occur once per game.
\bparag If \leader{Barbaros2} is alive, \TUR may
choose~\ref{pI:TD:Vassalisation of Algeria}. This may only occur once per
game.
\bparag If This is period \period{II} or later, \TUR may
choose~\ref{pI:TD:Alignment Barbaresques}. This may only occur once per game.
\bparag \TUR may always choose~\ref{pI:TD:Diplomatic pressures}. This may
happens any number of time.

\subevent[pI:TD:Barbaross brothers]{Barbaross brothers}
\history{Capture of Algiers by Aruj and Hayreddin Barbarossa: 1516}
\phevnt
\aparag \TUR immediately chooses one \Presidio in \paysAlgerie which is
destroyed.

\aparag If not controlled by \TUR, \paysAlgerie becomes immediately Neutral.

\aparag On this turn, the Algerian \corsaire is raised \faceplus (even if it
was not in play).

\subevent[pI:TD:Vassalisation of Algeria]{Vassalisation of Algeria}
\history{Recapture of Algiers by Hayreddin Barbarossa, and formal sovereignty
  of Soliman: 1529}
\phevnt
\aparag \paysAlgerie is immediately placed on \VASSAL of \TUR.

\aparag \leaderBarbaros is now also a Turkish leader, and as long as he is
alive, \paysAlgerie is permanent Vassal of \TUR not subject to diplomacy.

\aparag At the death of \leaderBarbaros, the {\bf -3} malus for \TUR to all
diplomacy attempts against all \terme{Barbaresque} countries is cancelled.


\subevent[pI:TD:Alignment Barbaresques]{Alignment of the Barbaresques}
\history{Ottoman conquest of Tunis: 1534, alignment: around 1540}
\phevnt
\aparag From now on, the {\bf -3} malus for \TUR to all diplomacy attempts
against all \terme{Barbaresque} countries is cancelled.

\aparag \paysTunisie is immediately placed on \VASSAL of \TUR if \leaderDragut
is alive.

\subevent[pI:TD:Diplomatic pressures]{Turkish Diplomatic Pressures}
\history{No precise date}
\phdipl
\aparag \TUR receives a bonus of \bonus{+3} for a Muslim minor of its
choice. Choice has to be made secretly during the negotiations step.


\event{pI:War Scotland}{I-10}{War with Scotland}{1}{Risto}
\history{1513-1514}

\condition{}
\aparag Occurs only if \paysecosse is at present inactive. Otherwise re-roll.
\aparag \ENG can refuse this event (mark as played) by losing {\bf 2} \STAB
and 20 \VP. It also loses the control of \paysecosse and can then make no
diplomacy on it until the end of period.

\phevnt
\aparag \paysecosse declares war against \ENG, which loses the control of
\paysecosse.
\aparag Allies can be called for this war as per normal rules.
\aparag Control of \paysecosse is offered to the first country in the list:
\bparag Any current enemy of \ENG (follow the normal preferences to decide
which).
\bparag The current controller of \paysecosse or, failing that, another power,
according to the usual rules.

\phadm
\aparag For the duration of the event, \paysecosse receives reinforcements in
offensive attitude.



\event{pI:End Golden Horde}{I-11 (1)}{The End of the Golden Horde}{1}{PB}

\history{1502}

\condition{}
\aparag If \paysCrimee exist no more, mark off and play \RD instead.

\phevnt
\aparag \paysCrimee declares war to \paysSteppes. The war is not played.
\aparag Both countries make mandatory White Peaces in existing wars.

\phdipl
\aparag Diplomacy, Call to Allies or Limited intervention is forbidden for
these two countries for the duration of the turn, and neither exterior
involvement in this war is allowed.

\phpaix
\aparag The Khanate of the Golden Horde is defeated by \paysCrimee at the end
of turn. From now on, the minor country \paysSteppes has reduced military
forces and stop helping other Khanates when attacked.



\event{pI:Pskov Ryazan}{I-11 (2)}{Russian Annexation of Pskov and
  Ryazan}{1}{PB}

\history{1510 and 1517}

\phevnt
\aparag The provinces \provincePskov and \provinceRyazan become Russian
National provinces.
\aparag \RUS can annex immediately one the two countries \paysPskov or
\paysRyazan of its choice.
\aparag A \MAJ having the annexed country on its track has a \CB against \RUS
at this turn.
\aparag \POL has a \CB against \RUS at this turn.



\event{pI:War Russia Poland}{I-12}{War between Russia and Poland}{1}{PB}

\history{1507-1522 / 1534-1537}

\condition{If \RUS and \POL are already at war against each other, mark off
  the case and play \RD instead.}

\phevnt
\aparag \RUS has a temporary free \CB against \POL and \POL has a temporary
free \CB against \RUS. Those \CB may be used this turn or the following
turn. If no power uses it, both lose 1 \STAB on the second turn.



\event{pI:War Roads Spices}{I-13}{Wars on the Roads of Spices}{2}{PBMod}

\history{1508-09/non historic}

\condition{}
\aparag If there is a \TP/\COL producing a \POSPICE belonging to any European
country, apply the \ref{pI:WRS:War Indian Sea}. It can happen only once.
\aparag Otherwise, apply \ref{pI:WRS:Veneto-Turkish Commercial Dispute}.  This
event can also happen only once.
\aparag If the second event happened and the first is not possible, do not
mark off and re-roll.


\subevent[pI:WRS:War Indian Sea]{War in Indian Sea}

\phevnt
\aparag \paysEgypte and \paysGujerat allies themselves.  They declare an
overseas war to any European country having a \TP/\COL in \continentAfrica
north-east of \granderegionNatal (included) or \continentAsia west of
\granderegionMalaisie and \granderegionSumatra (both included). They naturally
break diplomatic relations with countries they declare war to.
\bparag If \paysEgypte exists no more, \TUR gains an \dipAT with \paysGujerat.
\aparag A Major country having \terme{Treaty} with the \paysGujerat or any
diplomatic status with \paysEgypte has an oversea \CB at this turn against all
the countries aimed by the event (all at once).

\phdipl
\aparag From now on, \VEN can make diplomacy to \paysAden, \paysOman and
\paysGujerat, even through it does not know adjacent sea zones or have
\TP/\COL adjacent. However if \VEN is at war against the owner of the
\CCs{Grand Orient} it can make no diplomacy on these countries, and any
\terme{Treaty} it might have is inactive during the war; still it can resist
diplomatic attempts from other Major powers.

\phadm
\aparag \paysEgypte gains the discoveries of all seas from
\seazone{Seychelles} to \seazone{Malaisie}, bordering coastal zones (and
\seazone{Indien} excepted). From now on, \paysEgypte has only one \ARMY
counter, but has also one \FLEET counter (but no navy in basic forces) and can
use all its detachments as \LD or \ND, and gains 2 counters \LDENDE.
\bparag If \paysEgypte exists no more, \TUR gains the discoveries of
\seazone{Mascate} and \seazone{Indus} (only).
\aparag In the first turn of war induced by the event \paysEgypte chooses
Naval reinforcement.
\aparag On the first turn of war caused by the event, \paysGujerat raises an
additional \FLEET\facemoins (even if it is beyond its basic forces; it keeps
these warships until the end of the war).

\phpaix
\aparag At the end of the first turn of the war (only the first), the two
minor countries do not automatically accept a White Peace as usual in Overseas
Wars.  A formal peace has to be obtained.


\subevent[pI:WRS:Veneto-Turkish Commercial Dispute]{Veneto-Turkish Commercial
  Dispute}

\phevnt
\aparag As long as the \CCs{Grand Orient} is in \paysEgypte, \TUR can not, by
any means, receive part of its income.

\phdipl
\aparag \TUR gains a temporary free \OCB against \VEN.
\aparag At any following turn, \VEN can nullify the event by announcing it at
the beginning of the Declaration phases.  \VEN loses {\bf 1 } \STAB and \TUR
regains rights to part of the income of \CCs{Grand Orient} if it controls
\paysdamas. \TUR loses the \CB given by the event whereas \VEN gains a \CB
against \TUR, valid once before the end of the current period.
\aparag If \TUR makes a winning peace of level 2 or more against \VEN in a war
(oversea or regular), it can ask for its right on the \CCs{Grand Orient}
instead of one peace condition.



\event{pI:Resistance Muslim Traders}{I-14}{Resistance of Muslim
  Traders}{1}{PBNew}

\history{Non historic}

\condition{}
\aparag If the country \paysGujerat is destroyed, all European \TP in
\granderegionGujarat, \granderegionMalacca, \granderegionSumatra,
\granderegionJava, and \granderegion{Iles aux epices} will be attacked by
Natives during the turn.
\aparag If the country \paysGujerat still exists, use the following events.

\phevnt
\aparag All undestroyed \TP of \paysGujerat regain their initial level.  All
European \TP in the same Region will suffer a \CONC attempt at this turn from
\paysGujerat (Medium Investment).
\aparag In all provinces were \TP of \paysGujerat have been destroyed before,
European \TP will be attacked by Natives during the turn.



\event{pI:Chinese Expeditions}{I-15}{Chinese Expeditions}{1}{PBNew}

\history{Abandoned before 1492}

\phevnt
\aparag \paysChine gains three \TP of level {\bf 3} in the following
provinces: \provinceCalicut, \province{Malacca S}, \provinceMadras, replacing
existing \TP from \paysGujerat, and 2 \DT on each \TP.
\aparag However, if an European country has already discovered a sea zone
adjacent to the postulated position of those \TP, the Chinese \TP is not
placed here but in one province (determined randomly among those free of
\TP/\COL) in the following Regions (in this order, 1 by region):
\granderegionJava, \granderegionCelebes, \granderegionSumatra,
\granderegion{Iles aux epices} (if there is not enough unoccupied provinces in
those, the remaining \TP are lost).  Those \TP only have one \DT and level 1.
\aparag The Chinese \TP take the exploitation of resources (\RES{Products of
  Orient} first then \RES{Spices}) without concurrence; a Major Power will
have to make proper \CONC to take them back.
\aparag From now on, \paysChine has increased basic forces.  Added to the 2
\ARMY\faceplus in mainland \paysChine, it has garrisons of 1 \LD per \TP (or 2
\LD if they remain from the event), one \FLEET\faceplus and one Admiral (use
one from the minor pool, with the added capacity to go in the \ROTW) that can
move freely in the \ROTW when at war. Its reinforcements are one
\ARMY\faceplus in mainland, and a \LD, a \ND for the garrisons.
\aparag \paysChine has a \FTI of 2. The Chinese \TradeFLEET in \stz{Formose}
is increased to level 4.
\aparag \paysChine is considered to have discovered all land regions of
\continentAsia (including islands but \granderegionOceania and
\granderegionPacifique excepted) and those of \continentAfrica north and east
of \granderegionNatal included. It also has discovered all sea zones bordering
those territories.



\event{pI:Barbaros Brothers}{I-16}{Barbaros Brothers}{1}{PBNew}
\condition{[BLP] Apply~\ref{pI:Turkish Diplomatic Pressure}}

% \history{1516}

% \condition{Takes place only if leader \leaderBarbaros is not yet in play or no
%   more alive.  If \leaderBarbaros is in play, apply \xnameref{pII:Algeria
%     Vassalisation}}

% \phevnt
% \aparag If not controlled by Turkish, \paysAlgerie becomes immediately in
% Neutral.
% \aparag On this turn, the Algerian \corsaire is raised \faceplus and is
% supposed to have an admiral of Manoeuvre 3 leading it (another Brother).



\event{pI:Habsburg Alliance}{I-A}{Dynastic Alliance of the Habsburg}{1}{PB}

\history[Philip the Handsome, Habsburg heir, marries Juana the Mad, heiress of
Spain.]{1496}

\activation{Activated by \dynasticaction{A}{1}}

\phevnt
\aparag \SPA and \HAB are now allied in a specific way as described
in~\ruleref{chSpecific:Habsburg Dynastic Alliance}.  The diplomatic counter of
\HAB is placed in \EG of \SPA.
\aparag \SPA has now the right to annex the \paysprovincesne through war (it
has a \CB for such a war) or diplomatic actions.
\aparag \SPA has a temporary \CB at this turn or the following against any
country possessing any province that was part of \paysBourgogne in 1492.

\effetlong
\aparag[The Habsburg] The special alliance is now enforced between \SPA and
\HAB as per \ruleref{chSpecific:Habsburg Dynastic Alliance}, until broken by
\ref{pV:WoSS}.



\event{pI:Burgundy Inheritance}{I-B}{Burgundy Inheritance}{1}{PB}

\history[Spain takes full political control of Burgundian heirdom.]{1506}

\activation{Activated by \dynasticaction{A}{2}}

\phevnt
\aparag \SPA annexes all provinces of \paysBourgogne and this country exists
no more.  \SPA has a \CB (this turn and the following one) against any country
possessing any province owned by \paysBourgogne in 1492.
\aparag A Spanish \MNU of \RES{Cloth} with 2 levels is set in
\provinceVlaandern.
\aparag At the instant where \paysHollande exists (by \ref{pI:Habsburg
  Alliance}), \provinceZeeland is transferred from \SPA to \paysHollande.
When \ref{pI:Habsburg Alliance} has been played as well as the current event,
\paysLiege can now be \VASSAL or in \ANNEXION of the owner of Spanish
Flanders, \SPA now (and possibly \FRA, \ENG or \AUS later).

\effetlong
\aparag[Holland before its revolt]
\bparag The minor country \paysHollande is created by this event.  It will
consist of all provinces of \paysprovincesne that \SPA has gained, and
\ref{pI:Burgundy Inheritance} gives additional provinces from \paysBourgogne,
that is all national provinces of \paysHollande. This minor country is
permanent \VASSAL of \SPA, not subject to diplomacy, until it revolts by
\ref{pIII:Dutch Revolt}. It has no military forces, and any war against it has
to be declared as a war against \SPA. \SPA can not raise forces in
\paysHollande.
\bparag The commercial system of \paysHollande contributes to \SPA: its
\TradeFLEET are added to those of \SPA in order to find who has the different
\CC.
\bparag \SPA does not receive income for the provinces of \paysHollande.
Instead, it can impose a \terme{Tax} on \paysHollande that amounts to
40\ducats plus 10 \ducats for each province in \paysHollande.
\bparag Event \xref{pIII:Dutch Revolt} will free \paysHollande and change the
previous rules. Each turn of \terme{Taxes} will liken the Revolt.



\event{pI:Habsburg Bohemia}{I-C}{Habsburg Bohemia}{1}{PB}

\history{1526}

\activation{Activated by \dynasticaction{B}{1}, or by events~\xref{pI:Bohemian
    Alliance} or \xref{pI:Habsburg Hungary}}

\phevnt
\aparag \HAB annexes all provinces of \paysBoheme and this country exists no
more.  The power that has \paysBoheme on its diplomatic track has a temporary
\CB against \HAB.
\aparag \HAB has a free \CB (this turn and the following one) against any
country possessing any province owned by \paysBoheme in 1492; \SPA decides if
\HAB uses it or not.
\aparag If \paysBoheme was at war, \HAB is substituted to this country for the
on-going war. \HAB offers its enemies the immediate possibility to sign a
White Peace.

\effetlong
\aparag \paysBoheme may reappear as a ``liege'' country of \HAB or \SPA (see
\ruleref{chSpecific:Spain:Autonomous States}) or by means of \ref{pIV:Bohemian
  Revolt}.



\event{pI:Habsburg Hungary}{I-D}{Habsburg Inheritance of Hungary}{1}{PB}

\history{Never activated}

\activation{The first \RD event beginning with turn 8 activates this Event
  instead of its normal effect if either \ref{pI:Hungarian Alliance} or
  \dynasticaction{C}{1} has been played, and \ref{pI:Hungarian Freedom} has
  not.}

\condition{}
\aparag Play the \ref{pI:Habsburg Bohemia} if was not already played.
\aparag If \paysHongrie exists no more, ignore the rest of the event.

\phevnt
\aparag All provinces of \paysHongrie are annexed by \HAB and the country is
destroyed.
\aparag If \paysHongrie was at war, \HAB is substituted to this country for
the on-going war. \HAB offers its enemies the immediate possibility to sign a
White Peace.
\aparag Event \xref{pI:Habsburg Bohemia} is activated at this turn.

\effetlong
\aparag The basic forces of \HAB are increased by an \ARMY\faceplus.
\aparag \paysHongrie may reappear as a ``liege'' country of \HAB or \SPA (see
\ruleref{chSpecific:Spain:Autonomous States}).
\aparag If \HAB controls at least 5 provinces of \paysHongrie, it may use the
counters of \paysHongrie.
\aparag All future Hungarian leaders are now given to \HAB.
\aparag If \TUR annexes \villeBuda before the end of period II, lasting
effects of \ref{pI:Fall Hungary} are applied instead, and this event is
supposed to have happened for the rest of the rules (Victory conditions and so
on).



\event{pI:Fall Hungary}{I-E}{Downfall of Hungary}{1}{PB/Jym [BLP]}

\history{1526}

\activation{}
\aparag Activated by~\ref{chSpecific:Hungary} on the turn following either
\begin{modlist}
\item a major victory of \TUR against a stack with a least one \ARMY counter
  of \paysHongrie, if \TUR chooses to activate it;
\item[OR] occupation of \villeBuda by \TUR;
\item[OR] Turkish control of at least 5 provinces owned by \paysHongrie.
\end{modlist}
\bparag The moment the condition is met, \POL can make a limited intervention
at the side of \paysHongrie and \HAB may make a limited intervention or enter
war at the side of \paysHongrie. These are not declarations of war, no \STAB
is lost and no reinforcements are rolled.
\bparag Once the condition is met, \TUR may not sign peace with \paysHongrie
this turn.
\bparag On the next turn, this event is considered to be the first event
rolled.

% \bparag After the collapse and for the remaining of the turn, \HAB and/or \POL
% can make a limited intervention as an ally of collapsing \paysHongrie, with no
% declaration of war and no cost in \STAB. This intervention can be made with
% all the available forces of the power at that time (no reinforcements),
% without the usual limit for limited intervention.
% \bparag If the collapse begins on the last round of a turn, or would begin on
% the next turn, the resolution of the event is postponed to the end of the next
% turn.
% [BLP] removed
% \aparag If \ref{pI:Hungarian Freedom} has been played, military control of
% \villeBuda by \HAB in a war against \paysHongrie activates the same event,
% during any Period.
% \bparag \TUR may then make a limited intervention per the same rules as above.

\phpaix
\aparag Note that this happens the turn the event is resolved, \emph{i.e.} one
turn after \TUR causes the Downfall. Thus, there is always at least one full
turn during witch \HAB and \POL may try and defend \paysHongrie.

\aparag \paysHongrie is destroyed. Its remaining provinces are given as
follows:
\bparag \provincePecs, \provinceCroatie, \provinceHongrie, \provinceKarpatok,
\provinceBukovina are annexed by whoever controls militarily the province
among \TUR, \HAB and \POL (the presence of stack with \ARMY\faceplus in a
province with fortresses of an allied collapsing \paysHongrie gives control to
the leader of this stack). Those controlled by \paysHongrie at the end are
annexed by \HAB. (These provinces have no extra shield)
\bparag \provinceSzlovakia, \provinceBalaton, \provinceKranj and
\provinceKapela are annexed by \HAB (and nobody gains the \VP). (These
provinces have a blurred Austrian shield reminder)
\bparag \provinceBanat, \provinceSerbia and \provinceBosnia (if owned by
\paysHongrie or Neutral) are annexed by \TUR. (These provinces have a blurred
Turkish shield reminder)
\bparag A minor country \paysTransylvanie is created, composed from the
remaining provinces of \paysHongrie: likely, \provinceErdely and
\provinceMures (These provinces have a blurred Transylvanian shield) plus any
province that \paysHongrie may have annexed since the beginning of the
game. This country is created as a special \VASSAL of whoever got
\provinceHongrie during the partition.
\bparag Excepted for some provinces annexed by \HAB, the usual \VP are given.

\aparag If a power controls provinces given to another power, it may declare
now a war with a \CB, or its troops withdraw (as per peace evacuation).

\aparag[\paysTransylvanie] [BLP] For the rest of the game, \paysTransylvanie
is a special \VASSAL of the owner of \provinceHongrie.
\bparag As soon as this province changes owner, the new owner immediately
becomes the Diplomatic patron of \paysTransylvanie.
\bparag No diplomacy is allowed on \paysTransylvanie. It is not subject to
Diplomatic events.

\aparag The limited interventions of \HAB and \POL (if any) end immediately.
\bparag However, if \HAB chose to enter war, a formal peace treaty must be
obtained at this turn or another one, as usual.

% \aparag The basic forces of \HAB are increased by an \ARMY\facemoins. All
% future Hungarian leaders are now given to \HAB.

% \aparag Whenever the \HAB annexes \provinceHongrie, the counters of
% \paysHongrie (2 \ARMY and 4 \DT) are permanently added to those available to
% \HAB.

% \aparag \paysHongrie may reappear as a ``liege'' country of \HAB (but not
% \SPA, see \ruleref{chSpecific:Spain:Autonomous States}). %or by means of
% event \ref{pIII:War HRE}.

\effetlong
\aparag The basic forces of \HAB are increased by an \ARMY\facemoins.
\aparag \paysHongrie may reappear as a ``liege'' country of \HAB or \SPA (see
\ruleref{chSpecific:Spain:Autonomous States}). %or by means of event
% \ref{pIII:War HRE}.
\aparag If \HAB controls at least 7 provinces of \paysHongrie, it may use the
counters of \paysHongrie.
\aparag All future Hungarian leaders are now given to \HAB.

\aparag[] [BLP] \ref{chSpecific:Little war} is now active.


\event{pI:Habsburg Milano}{I-F}{Habsburg Control of Milano}{1}{RistoMod}

\history{around 1520}

\activation{Activated by \ref{pI:Habsburg Dynasty} or \dynasticaction{B}{2}}
\aparag If \paysMilan is now at war, \HAB have a free \CB to join the war on
the side of \paysMilan. The rest of the event is activated when the \CB is
used.
\aparag If \HAB is not allied yet to \SPA, it uses the \CB of the event as
soon as it is not active elsewhere.

\phevnt
\aparag \paysMilan becomes a permanent \VASSAL of \HAB. \paysMilan and \HAB
are from now on one entity for wars and peaces.
\aparag If the province \provinceLombardia is french, a \REVOLT \facemoins is
placed herein and \HAB have a free \CB this or the following turn against
\FRA.

\effetlong
\aparag \dynasticaction{C}{2} is now possible.



\event{pI:Spanish Milano}{I-G}{Spanish Milano}{1}{RistoMod}

\history{around 1560}

\activation{Activated by \dynasticaction{C}{2}}

\phevnt
\aparag \SPA annexes \provinceLombardia if this province is in \paysMilan
(whether a permanent \VASSAL of \HAB or not) or owned by \HAB. The minor
country \paysMilan exists no more.
\aparag If \provinceLombardia is owned by another country, a \REVOLT
\facemoins is placed herein and \SPA and \HAB have free \CB this or the
following turn against this country.

\effetlong
\aparag \SPA can now raise troops in \provinceLombardia if it controls it,
with normal cost.
\aparag \SPA can recreate \paysMilan as a ``liege'' country (see
\ruleref{chSpecific:Spain:Autonomous States}).



\event{pI:Fall Teutonic}{I-H}{Secularisation of \pays{Teutoniques2}}{1}{PB}

\history{1525}

\activation{Activated by \ref{pI:Reformation2} or \ref{pIII:Northern
    Secularisation}, whichever occurs first}

\phevnt
\aparag \pays{Teutoniques2}, part of minor country \pays{Teutoniques1} become
Protestant.  All units from any country in \provincePreussen, \province{Ost
  Pommern} and \province{West Pommern} have to retreat when those provinces
are annexed by another country.
\aparag The province \provincePreussen become part of \region{Duche de
  Prusse}.
\bparag If \POL is \CATHCO, it annexes the province if owned by
\pays{Teutoniques1}, or has a \CB against its owner until the end of the
Period.
\bparag Else, \region{Duche de Prusse} is annexed by \paysBrandebourg, and
\provincePreussen become part of \paysBrandebourg. If this province is owned
by any other country than \pays{Teutoniques1}, this country has a \CB against
\paysBrandebourg.
\aparag The provinces \province{Ost Pommern} and \province{West Pommern} are
annexed by \paysHanse if owned by \pays{Teutoniques1}.
\bparag If one or the two provinces are owned by any country except \POL,
\paysHanse declares war to this country. \POL may (his choice) have \paysHanse
placed in \AM before usual calls for help is made, in which case \paysHanse
calls it to his help. Else, usual rules are used.
\bparag Else, if one or the two provinces are owned by \POL, \paysHanse
declares war to \POL and \paysBrandebourg too, allied with \paysHanse.
\bparag If a war results of this event, only \paysHanse can annex the 2
provinces.
\aparag The minor country \pays{Teutoniques1} (now Livonian Brothers of the
sword) loses one \ARMY counter, and its basic forces are diminished by one
\ARMY\faceplus.

\phpaix
\aparag If a war is prosecuted between minor countries only, it lasts one turn
and the side of \paysBrandebourg wins (gaining the provinces).



\event{pI:Spanish Naples}{I-I}{Spanish Naples}{1}{PB}

\history{1497 -- The Spanish rulers decide to take direct control of the
  kingdom of Naples}

\activation{Activated by \dynasticaction{A}{3}, or at the turn following
  \ref{pI:War Italy Napoli}, whichever occurs first.}

\phevnt
\aparag \SPA gains a permanent \CB against \paysNaples (even if on his own
diplomatic track), and also against any owner of a national province of
\paysNaples.

\phdipl
\aparag When \SPA declares a war against \paysNaples, \FRA has a \CB at this
turn only in a reaction to declare a war jointly to \SPA and \paysNaples.
\aparag \SPA may also annex the country by diplomatic means.

\phpaix
\aparag Any province of \paysNaples controlled by \SPA at the end of a turn is
immediately annexed without need for peace. If it was the last province of
\paysNaples, the country is destroyed. When \villeNaples is annexed by \SPA,
remaining provinces of \paysNaples surrender now, are annexed by \SPA and the
country is destroyed.
\aparag In period II, if \SPA has \paysNaples in diplomatic \ANNEXION, the
minor country is destroyed and permanently annexed by \SPA.
\aparag \SPA loses the \CB given by this event as soon as it owns every
national province of \paysNaples.

\effetlong
\aparag As long as \SPA owns \provinceCampania, it gains a free maintenance of
one \FLEET\facemoins, in period II, III and IV.
\aparag \SPA can recreate \paysNaples as a ``liege'' country (see
\ruleref{chSpecific:Spain:Autonomous States}).

% Local Variables:
% fill-column: 78
% coding: utf-8-unix
% mode-require-final-newline: t
% mode: flyspell
% ispell-local-dictionary: "british"
% End:

% LocalWords: pII pI Tordesillas Napoli Milano Habsburg Comuneros Pskov HRE
% LocalWords: Ryazan Barbaros RistoMod Minas Gerais Bonne Esperance Tempetes
% LocalWords: malus Serenissima EW Risto PBNew Mancha Castilla Nueva Vieja
% LocalWords: pV WoSS pIII FWR pIV Colbertian pVI TYW Lublin Khanate PBMod
% LocalWords: Khanates Malaisie Indien Mascate Veneto Iles epices Formose de
% LocalWords: undestroyed Vassalisation heirdom Teutoniques Ost Pommern WRS
% LocalWords: Duche Prusse Livonian


\clearpage

% -*- mode: LaTeX; -*-

\section{Period II}\label{events:pII}



\subsection*{Event Table of Period II}

\begin{eventstable}[Period II events table]
  \tabcolsep=5pt\centering%
  \begin{tabular}{|l|*{6}{c}|l|}
    \hline
    1\up{st}\textarrow& 1-3 & 4-5 & 6 & 7 & 8 & 9 & 10 \\ \hline
    1 & 2  & 1  & 10  & 13  & 1  & 1   & \textbullet~1--2:   \\
    2 & 3  & R2 & 11  & R14 & 2  & R8  & +1 then\\
    3 & R4 & 3  & 12  & R15 & R3 & R2  & \nameref{events:pI} \\
    4 & 5  & 4  & 15  & 16  & R4 & 11  & \textbullet~3--10:  \\
    5 & 6  & 8  & 16  & 17  & R5 & 12  & \nameref{events:pI} \\
    6 & 7  & 9  & 17  & R18 & 7  & 13  & \\
    7 & 8  & 11 & R18 & R21 & R8 & R19 & \\
    8 & 10 & 12 & R8  & 1   & R9 & R21 & \\
    9 & R9 & 21 & R21 & 19  & 14 & R20 & \\ \hline
    10 & \multicolumn{7}{l|}{\nameref{events:pIII}} \\ \hline
  \end{tabular}
\end{eventstable}
\begin{eventstablespec}[Habsburg Hungary]
  The first \RD event beginning with turn 8 activates \ref{pI:Habsburg
    Hungary} instead of its normal effect if either \ref{pI:Hungarian
    Alliance} or \dynasticaction{C}{1} has been played, and \ref{pI:Hungarian
    Freedom} has not.
\end{eventstablespec}

\eventssummary{%
  pII:Act Supremacy|,%
  pII:War Scotland|,%
  pII:Emperor Election|O{pI:Emperor Election},%
  pII:Habsburg Dynastic Commitments|E/E/E/E,%
  pII:War Italy|,%
  pII:End Kalmar|,%
  pII:War Persia Turkey|E/E,%
  pII:Algeria Vassalisation|,%
  pII:Alignment of Barbaresques|,%
  pII:War Poland Turkey|,%
  pII:Reformation|O{pI:Reformation}/O{pI:Reformation2}/O{pI:Reformation3},%
  pII:Schmalkaldic League|,%
  pII:War Indian Ocean|,%
} \eventssummary{%
  pII:Portuguese Colonial Dynamism|E/E/E,%
  pII:Spanish Colonial Dynamism|E/E/E,%
  pII:Union Lublin|,%
  pII:Conquest Khanates|,%
  pII:Superiority over Khanates|,%
  pII:War Russia Poland|,%
  pII:War Russia Turkey|,%
  pII:Forward Baltic Sea|,%
  pII:American Resistance|E/E,%
  pII:Chinese Expansion|,%
  pII:Apparition Mughal Empire|E/E,%
  O|,%
  pII:Mughal Expansions|T{many times},%
  pII:Crusade|T{many times},%
}

\newpage\startevents



\event{pII:Act Supremacy}{II-1 (1)}{Act of Supremacy}{1}{Risto}

\history{1534, 1539}
% \condition{Takes only place if the event \ref{pI:Reformation2} has already
% occurred. Otherwise re-roll and do not mark off.}

\condition{Takes place when rolled for, or when \monarque{Henry VIII} dies.}

\phevnt
\aparag \ENG has to choose its Heir, in accordance with its current religion.
\bparag[Catholic/No Reform] Mary I Tudor, or Edward VI
\bparag[\CATHCR] Mary I Tudor, or Edward VI
\bparag[\CATHCO] Edward VI, Jane Grey
\bparag[Protestantism] Jane Grey

\aparag[Marie I Tudor] \ENG is forced to be \CATHCR.  If it has changed, both
general and particular effects of \ref{pI:Reformation2} are applied
immediately.
\bparag It has a mandatory Dynastic (Defensive and Offensive) Alliance with
\SPA for 3 turns.  If at war, \SPA and \ENG make an immediate white peace.
\bparag Roll for 2 \REVOLT in \ENG in the table, using 1d10-2 for
localisation.

\aparag[Lady Jane Grey] \ENG is forced to be Protestant.  If it has changed,
both general and particular effects of \ref{pI:Reformation2} are applied
immediately.
\bparag Alliance between \ENG and \SPA are forbidden for 3 turns.
\bparag All \CATHCR \MAJ and also \POR and \VEN receive a temporary \CB
against \ENG.
\bparag Roll for 2 \REVOLT in \ENG in the table, using 1d10+3 for
localisation.

\aparag[Edward VI] \ENG must choose freely its Religious Attitude.  If it has
changed, both general and particular effects of \ref{pI:Reformation2} are
applied immediately.  If it is now \CATHCO, Edward VI (and truly also Mary I)
will reign at most 2 turns. (Note: determine values at random, Edward VI may
also die, but its successor will last only the second turn).
\bparag At the beginning of the second turn, roll for 2 \REVOLT in \ENG in the
table, using 1d10-2 for localisation.
\aparag[After Edward VI: Elizabeth or Mary] At the beginning of the third
turn, \ENG may opt immediately to choose between two possibilities:
\bparag[Mary Stuart] \ENG chooses to remain \CATHCO, in which case none of the
effects described underneath are applied. Instead, \ENG loses {\bf 1} in \STAB
(for having to face humiliation from the Pope).
\bparag["Elizabethan Settlement"] \ENG becomes \PROTANG, that is Protestant as
defined in \ref{pI:Reformation2}.  Both general and particular effects of the
event are applied immediately.  The only difference between Anglicanism and
Protestantism is relative to the Religious and Civil Wars of \ENG.
\bparag The Monarch of \ENG is now \monarque{Elisabeth I}.
\bparag \ENG receives 250\ducats in its Treasury.
\bparag All \CATHCR \MAJ and also \POR and \VEN receive a temporary \CB
against \ENG.

% \aparag[Protestantism:]
%% \bparag All catholic/Counter-Reformation major powers and also \POR receive
%% a temporary \CB against \ENG.
% \bparag Event \ref{pIV:English Civil War} is immediately activated.
% \aparag[\CATHCR]
% \bparag \ENG may become Anglican, that is Protestant as defined in the event
% \ref{pI:Reformation2}. Both general and particular effects of the event are
% applied immediately. The only difference between Anglicanism and
% Protestantism is relative to the Religious and Civil Wars of \ENG.
% \bparag \ENG receives 250\ducats in its Treasury.
% \bparag All \CATHCR major powers and also \POR and \VEN receive a temporary
% \CB against \ENG.
% \bparag \ENG may choose to remain Conciliatory, in which case none of those
% effects are applied. Instead, \ENG loses {\bf 1} in \STAB (for having to
% face humiliation from the Pope).
% \aparag[\CATHCR]
%% \bparag All protestant major powers receive a temporary \CB against \ENG.
% \bparag Event \ref{pIV:English Civil War} is immediately activated.
%% \bparag If \SPA is \CATHCR and the king wins \ref{pIV:English Civil War},
%% \ENG will have the same turn/period limits as if \CATHCO.



\event{pII:War Scotland}{II-1 (2)}{War with Scotland}{1}{PBNew}

\history{1542}

\condition{}
\aparag Occurs only if \paysecosse is at present inactive. Otherwise re-roll.
\aparag \ENG can refuse this event (mark as played) by losing {\bf 2} \STAB
and 20 \VP. It also loses the control of \paysecosse and can then make no
diplomacy on it until the end of period.
\bparag If \ANG has chosen the "Mary Stuart" option in \ref{pII:Act
  Supremacy}, the refusal of the war costs only {\bf 1} \STAB (no \VP, no
diplomatic consequences).

\phevnt
\aparag \paysecosse declares war against \ENG, which loses the control of
\paysecosse.
\aparag Allies can be called for this war as per normal rules.
\aparag Control of \paysecosse is offered to the first country in the list:
\bparag Any current enemy of \ENG (follow the normal preferences to decide
which).
\bparag The current controller of \paysecosse or, failing that, another power,
according to the usual rules.

\phadm
\aparag For the duration of the event, \paysecosse receives reinforcements in
offensive attitude.



\event{pII:Emperor Election}{II-2 (1)}{Election of the \HRE
  Emperor}{1}{RistoMod}

\condition{Same event as \ref{pI:Emperor Election}.}
\aparag If \shortref{pI:Emperor Election} has not occurred, play this event.
\aparag If \shortref{pI:Emperor Election} has already occurred, play the
following event.



\event{pII:Habsburg Dynastic Commitments}{II-2 (2)}{Habsburg Dynastic
  Commitments}{4}{PB}

\phevnt
\aparag \SPA \textbf{must} immediately play one dynastic action of its choice,
without test nor cost. Annexation of a province of the North-East is a valid
choice. If there is no such actions possible, treat as no event and mark off.



\event{pII:War Italy}{II-3}{War in Italy}{1}{Ristomod}

\history{1521-1526 / 1526/1530 / 1536-1539 / 1542-1544 / 1552-1559}

\condition{This event continues \ref{pI:War Italy Napoli} and \ref{pI:War
    Italy Milano}.}
\aparag If either \ref{pI:War Italy Napoli} or \ref{pI:War Italy Milano} is in
effect, re-roll without marking.
\aparag If \ref{pI:War Italy Napoli} was not played, play it, mark off and do
not apply the remaining of the present event.
\aparag If \FRA owns \provinceLombardia, mark off the event which is
considered played with only one effect: \HAB after \ref{pI:Habsburg Milano} or
\SPA after \ref{pI:Spanish Milano} has a free \CB against \FRA at this turn.
\aparag The event may happen more than once. If a \ref{pII:War Italy} is
happening when another event is rolled for, the second one is marked off and
treated as a \RD.

\phevnt
\aparag \FRA has a Mandatory \CB against the owner of \provinceLombardia. This
\CB has to be used this turn or the next, at the phase of Declaration of
War. If \FRA is Counter-Reformation after \ref{pI:Reformation2}, the \CB is
free.
\aparag If \FRA is already at war against this country, the war has to become
the war linked to this event at this turn or the following (the choice is made
by \FRA during the Declarations of War) and that fulfils the Mandatory \CB.

\phdipl
\aparag[Refusing the event]
\bparag At the very beginning of the Declarations Phase, \FRA or the owner of
\provinceLombardia may refuse the event.
\bparag If \FRA refuses the event, it loses {\bf 2} \STAB (or none if the
current period is III or after) and the rest of the event is ignored.
\bparag If the owner of \provinceLombardia refuses the event, it loses {\bf 3}
\STAB and gives \provinceLombardia to \FRA. Then the rest of the event is
ignored. If this province is owned by the \HAB, \SPA may refuse the event (and
lose the \STAB).
\aparag[Milan as a Minor country]
If \provinceLombardia is owned by the Minor country \paysMilan, \HAB have a
free \CB in reaction to a Declaration of War of \FRA against this
country. \paysMilan is moved up to \EG on the diplomacy track of \HAB if it
was not already on a higher position.
\aparag[Diplomatic effects of the wars]
\FRA has a bonus of {\bf +2} for its diplomacy on \paysToscane and {\bf -1}
for \paysPapaute and \paysParme during the event.
\aparag[The Serenissima in the Wars in Italy]
This rule is applied only if \VEN has announced a \terme{Policy of Italian
  dominance}.
\bparag \VEN has a \CB against \FRA and/or the owner of \provinceLombardia, as
long as the war is not finished.
\bparag During this war, \VEN may make limited intervention at the side of any
involved alliance each turn. Such limited intervention can begin at any turn
(not only the first) and \VEN can change side between turns.  \VEN may force
any Italian \MIN in limited intervention for the enemy alliance, to be fully
involved in the war.
\bparag Conversely, \FRA and \HAB both have a (normal) \CB against \VEN, to be
used at any turn of the war.
\aparag[Swiss Mercenaries]
If \paysMilan is a vassal or a possession of \HAB (according to
\ref{pI:Habsburg Milano}), \HAB gain \paysSuisse in \CE.

\phmvt
\aparag \paysSavoie gives free access and supply in its province to \FRA
during the first turn of the war, if it stays neutral in this war. Supply from
or across a province is impossible if its city is under siege by an enemy of
this city.

\effetlong
\aparag Until the end of the current period, \FRA has a \CB against the owner
of \provinceLombardia.



\event{pII:End Kalmar}{II-4}{End of the Union of Kalmar}{1}{Risto}

\history{1523}

\phevnt
\aparag The effect of specific \ruleref{chSpecific:Sweden:Union Kalmar} is
terminated.
\aparag Because of troubles between \paysDanemark and \paysSuede, both
countries make mandatory white peaces, lowers the European market by 75\ducats
this turn for everyone.
\aparag If \SUE is a \MAJ, roll for 2 \REVOLT in \SUE and \SUE loses {\bf 1}
\STAB.



\event{pII:War Persia Turkey}{II-5}{War between Persia and Turkey}{2}{Risto}

\history{1526-1555}

\condition{Takes place only if \paysperse is inactive. Otherwise re-roll.}

\phevnt
\aparag \paysperse declares war against \TUR.
\aparag \paysperse and \TUR can immediately call allies as per normal rules.
\aparag If \paysperse is neutral, it does not call any ally and is played by
\SPA.

\phadm
\aparag \paysperse receives reinforcements on offensive status for the
duration of this war.



\event{pII:Algeria Vassalisation}{II-6 (1)}{Turkish Vassalisation of
  Algeria}{1}{Risto}

\history{1519}

\condition{Takes place only if leader \leaderBarbaros is alive. Otherwise mark
  as played, but do not re-roll.}

\phevnt
\aparag \paysAlgerie is immediately placed on \VASSAL of \TUR.
\aparag \leaderBarbaros is now also a Turkish leader, and as long as he is
alive, \paysAlgerie is permanent Vassal of \TUR not subject to diplomacy.
\aparag At the death of \leaderBarbaros, the {\bf -3} malus for \TUR to all
diplomacy attempts against all \terme{Barbaresque} countries is cancelled.



\event{pII:Alignment of Barbaresques}{II-6 (2)}{Alignment of
  Barbaresques}{1}{Risto}

\history{1540}

\phevnt
\aparag From now on, the {\bf -3} malus for \TUR to all diplomacy attempts
against all \terme{Barbaresque} countries is cancelled.
\aparag \paysTunisie is immediately placed on \VASSAL of \TUR if \leaderDragut
is alive.



\event{pII:War Poland Turkey}{II-7}{War between Poland and Turkey}{1}{PB}

\history{1526-1535 -- it was not a formal war}

\condition{Turkey may refuse the event, in which case it is not marked and no
  event is re-rolled for. If the event is not refused, apply the following}

\phdipl
\aparag \TUR has a bonus of {\bf +2} on diplomatic actions on minor countries
\paysMoldavie, \paysValachie and \paysTransylvanie.
\aparag \TUR has a free \CB to be used at this turn of the following one
against \POL if it has a province adjacent to this country, or a minor country
in \AM at least, that is adjacent to \POL.
\aparag If \TUR is at war with \POL, any minor country adjacent to \POL that
is in \AM or higher of \TUR will join full war against \POL without test, and
so is placed in \EG.

\phadm
\aparag If there is a Polish \paysUkraine, \POL gains a free \ARMY\facemoins
to fill the Ukrainian army at each turn of the war.



\event{pII:Reformation}{II-8}{Reformation}{3}{Risto}

\history{1522-1560}

\condition{This event is the same as in period I and continues the effects,
  provoking either \ref{pI:Reformation}, \ref{pI:Reformation2} or
  \ref{pI:Reformation3}.}



\event{pII:Schmalkaldic League}{II-9}{War of the Schmalkaldic
  League}{1}{RistoMod}

\history{1546-1547}

\condition{}
\aparag If \ref{pI:Reformation} has not yet occurred once, do not mark off and
re-roll.
\aparag This event cease with the breaking of the League as described in the
event or in \ref{pIV:TYW}.

\phevnt
\aparag The following countries form a defensive league: \paysHesse,
\paysSaxe, \paysThuringe and \paysWurtemberg. They are considered as one
country for declaration of wars, and one alliance for peace terms.
\aparag The Emperor loses diplomatic control of all countries of the League
and can no longer make diplomatic actions on them. Those countries leaves \GE
if there is one.

\phdipl
\aparag The Emperor has a permanent \CB against the League. This \CB is free
if the Emperor is The Sole Defender of Catholic Faith (free \CB also if the
\HAB are Emperors for the Austrian branch and Defender of the Catholic Faith
for the Spanish branch). A war against any country of the League is called a
war against the League; it is a \terme{war of Religion} (so external
intervention is constrained).
\bparag \SPA may ask for limited or full intervention of the \HAB in this war.
\aparag The Emperor may grant the \terme{Truce of Augsburg} regarding the
liberty of belief in the \HRE. Such a decision costs {\bf 1} \STAB and 20 \VP.
\aparag When a war against the League occurs, the minor countries are allied
for any purposes and are played by the first major player in the list that is
not at war against any country of the League: \HOL, \ENG if Protestant, \FRA
if Protestant, \POL if Protestant, \SUE (if Protestant and period III+), \SPA,
\ENG, \FRA, \POL if not. This power is called for as an ally of the League,
but may refuse at no cost. The League plays at the same round of the player
who plays it (whether involved in the war or not).
\aparag Any Major Country having one of the minor countries in the League on
its diplomatic chart can make a limited intervention against the Emperor, as
an ally of the League.

\phpaix
\aparag If the Emperor is Spanish or Habsburg, a test to begin the
\ref{pIV:TYW} is made at the end of each turn of any war between the League
and the Emperor. This test is modified by {\bf +4}. See \ref{pIV:TYW} for the
result of the test and the possibility of this Religious War. If no such war
occurs, peace can be made on the following conditions.
\aparag Each minor country obeys to the usual rules for peace. As they are
allied, a peace against only one country is a separate peace.
\aparag A minor country forced to sign an unconditional surrender breaks from
the League for ever. This replace all the peace conditions.
\aparag The League may be dissolved under the following conditions:
\bparag the last country in the League is forced out, or
\bparag \paysHesse or \paysSaxe has been forced out of the League and the
Emperor has granted, or grants immediately the Truce of Augsburg (at the cost
of {\bf 1} \STAB and 20 \PV).
\aparag If the League is dissolved without the Truce of Augsburg, \SPA keeps
the title of Emperor for one more monarch.
\bparag If the Emperor is from \SPA or \HAB, and has made a war against the
League and suffered a Major Defeat against land forces of the League, it can
decide at the phase of peace to become \CATHCO as in
\ref{pI:Reformation2}. The war ends immediately in a white peace and the
application of the Truce of Augsburg in the \HRE. Both general and specific
events of \shortref{pI:Reformation2} will be applied to \SPA at the following
event phase.

\effetlong
\aparag The countries of the Schmalkaldic League will join some wars caused by
events: \xref{pIV:TYW}, \xref{pIV:Augsburg Revocation}, and \xref{pIV:Unity
  HRE}. The League may reinforce the intervention of \paysPalatinat in
\ref{pIII:FWR}. The League exists no more when involved in the \ref{pIV:TYW}.



\event{pII:War Indian Ocean}{II-10}{War in the Indian Ocean}{1}{PB}

\history{1536-1538 / 1546}

\condition{}
\aparag If a Treaty is militarily enforced between \POR and \paysOman or/and
\paysAden, apply \xnameref{pII:WIO:Revolt Oman Aden} for this (or these)
countries.
\aparag If no Treaty is enforced, apply \xnameref{pII:WIO:War Oman Aden}
against this (or these) \MIN. Both a Revolt and a War can occur (against
different countries).


\subevent[pII:WIO:Revolt Oman Aden]{Revolt of Oman/Aden}

\phdipl
\aparag \TUR has an overseas \CB against \POR at this turn. \TUR gains the
discoveries of \seazone{Mascate} and \seazone{Indus}

\phmil
\aparag The Natives of the region \granderegionOman or \granderegionAden are
activated and will attack units of \POR at this turn. They will not attack
Turkish forces this turn.

\phinter
\aparag If the attack of the colony by the Natives at the end of turn result
in at least 1 level of \COL that should be lost, those levels are not applied
to the COL of \paysOman or/and \paysAden but break the Treaty status of the
country with \POR (they now have No Relation and Portuguese forces are
redeployed immediately).
\aparag If \paysOman or/and \paysAden breaks free from a Treaty with \POR and
\TUR is at war with \POR, \TUR gains a Treaty with this (these) \MIN.


\subevent[pII:WIO:War Oman Aden]{War with Oman/Aden}

\phevnt
\aparag \paysOman or/and \paysAden declare(s) an oversea war to \POR. If both
are at war, they are allied.
\aparag \TUR has an oversea \CB against \POR at this turn, to enter the war as
an ally of \paysOman or/and \paysAden and it gains the discoveries of
\seazone{Mascate} and \seazone{Indus}.  If the \CB is used, \TUR gains a
Treaty with \paysOman or/and \paysAden.

\phadm
\aparag \paysOman or/and \paysAden at war receive(s) Naval Reinforcement at
the first turn of the war.

\phinter
\aparag If \paysOman or/and \paysAden occupy a \TP of \POR at the end of the
turn, they do not burn it if they have a \TP counter available and this \TP is
transformed in a \TP of the minor country. If there is no counter available,
the \TP is burnt down. The choice of the \TP converted is random.



\event{pII:Portuguese Colonial Dynamism}{II-11}{Portuguese Colonial
  Dynamism}{3}{Risto}

\phdipl
\aparag \POR gains a bonus of {\bf +3} for any diplomatic action on
non-European minor countries at this turn.

\phadm
\aparag \POR receives one additional and free strong investment \TP placement
action.
\aparag \POR receives a shift of one column to its favour in the actions
results table for all its \COL/\TP placement attempts this turn.
\aparag \POR may ignore restriction of~\ref{chExpenses:Pioneering} for this
turn.



\event{pII:Spanish Colonial Dynamism}{II-12}{Spanish Colonial
  Dynamism}{3}{Risto}

\phdipl
\aparag \SPA gains a bonus of {\bf +3} for any diplomatic action on
non-European minor countries at this turn.

\phadm
\aparag \SPA receives one additional and free strong investment \COL placement
action.
\aparag \SPA receives a shift of one column in its favour in the actions
results table for all its \COL/\TP placement attempts this turn.
\aparag \SPA may ignore restriction of~\ref{chExpenses:Pioneering} for this
turn.



\event{pII:Union Lublin}{II-13}{Union of Lublin}{1}{PB}

\history{1568}

\condition{If \POL is Protestant or has chosen Support of Orthodoxes, the
  union is impossible. Mark off the case and play \RD instead, with the
  \REVOLT in \POL.}

\activation{}
\aparag The rest of the event is activated when \POL decides to sign the
Union. That is to be announced at any current or following phase of
declaration.

\phdipl
\aparag Both countries in \POL are linked by an Union. All effects described
in \ruleref{chSpecific:Poland:Before Lublin} are applied no more and the new
conditions are described in \ruleref{chSpecific:Poland:Union Lublin}.
\aparag If \POL is not at war against any Major Power at the time of the
Union, play two \REVOLT in \POL. If it is at war against a Major Power, do not
draw any \REVOLT .
\aparag \RUS and \SUE has a \CB against \POL at the turn of declaration of the
Union.

\effetlong
\aparag The Union of Lublin can be broken if someone imposes a peace of level
at least 3 on \POL, and this counts as the gain of 2 provinces (or their
equivalent in War Reparation) for the terms of peace.



\event{pII:Conquest Khanates}{II-14}{Russian conquest of the Khanates}{1}{PB}

\history{Kazan 1547-1552}

\activation{\RUS may refuse this event, in which case it is not marked but no
  other event is rolled for.}

\phevnt
\aparag If \ref{pI:Pskov Ryazan} has not been played, it is played as an
additional event this turn.
\aparag Else, or on a second occurrence of the event, apply the following
effect.
% (Jym) Additional or instead of ?

\phdipl
\aparag \RUS has a free \CB against a Khanate of its choice at this turn only.

\phpaix
\aparag This Khanate will surrender unconditionally and will be entirely
annexed by \RUS if \RUS controls its capital and half of the provinces of the
Khanate.



\event{pII:Superiority over Khanates}{II-15}{Russian Superiority over the
  Khanates}{1}{PB}

\history{Astrakhan 1554-1556}

\activation{\RUS may refuse this event, in which case it is not marked but no
  other event is rolled for.}

\phevnt
\aparag \RUS advances its \terme{Land Technology} marker of {\bf 3}
boxes. This event might place the Land Technology of \RUS higher than
\terme{Orthodox} \terme{Land Technology}. This is allowed and the marker stays
in place until the \terme{Orthodox} \terme{Land Technology} becomes higher
than the one of \RUS, in which case \RUS can resume its progression.

\phdipl
\aparag \RUS has a free \CB against a Khanate of its choice at this turn only.

\phpaix
\aparag This Khanate will surrender unconditionally and will be entirely
annexed by \RUS if \RUS controls its capital and half of the provinces of the
Khanate.



\event{pII:War Russia Poland}{II-16}{War between Russia and Poland}{1}{PB}

\history{1507-1522 / 1534-1537}

\condition{If \RUS and \POL are already at war against each other, mark off
  the case and play \RD instead.}

\phevnt
\aparag \RUS has a temporary \CB against \POL and \POL has a temporary \CB
against \RUS. Those \CB may be used this turn or the following turn. If a
power does not use, it loses 1 \STAB on the second turn.



\event{pII:War Russia Turkey}{II-17}{War between Russia and Turkey}{1}{PB}

\history{Crimea 1521-1523, 1559, 1572}

\activation{\RUS has the control of this event.}

\phdipl
\aparag \RUS has a free \CB against a Khanate of its choice at this turn only.
\aparag If this \CB is used, the attacked country is placed at least in \AM of
\TUR that has now the opportunity to enter war to support the minor country or
not.

\phpaix
\aparag This Khanate will surrender unconditionally and will be entirely
annexed by \RUS if \RUS controls its capital and half of the provinces of the
Khanate.
\aparag If \TUR did not enter the war to support the Khanate and it is
destroyed as a result of this war, \TUR has a free \CB against \RUS the turn
following the conquest.



\event{pII:Forward Baltic Sea}{II-18}{Forward to the Baltic Sea}{1}{PB}

\history{1558-1561}

\condition{}
\aparag If the \pays{Teutoniques1} do not exist any more (either by conquest
or by event \ref{pIII:Northern Secularisation}), mark off and play \RD
instead.
\aparag If \RUS has no province adjacent to the \pays{Teutoniques1}, do not
mark off and roll for another event.

\phevnt
\aparag \RUS has a free \CB against the \pays{Teutoniques1}.

\phadm
\aparag The \pays{Teutoniques1} take their reinforcements in offensive
attitude during the first turn of the conflict.

\phpaix
\aparag Before testing for any peace, 1d10 is rolled, modified by the peace
differential of \RUS against the \pays{Teutoniques1}. If the result is 6 or
more, the \pays{Teutoniques1} collapse and no peace occurs now. At the
following event phase, the first event considered rolled for is automatically
\ref{pIII:Northern Secularisation}.



\event{pII:American Resistance}{II-19}{Resistance of the American
  Empires}{2}{PB}

\history{not historic}

\condition{}
\aparag If there is no \COL in \continent{America} (excepted the islands) do
not mark off and re-roll.

\aparag If both empire have already collapsed, play \RD instead of this event
and mark off.

\aparag Else, \paysInca or \paysAzteque (decide randomly, or take the one that
did not collapse), is affected by the following event.

\phevnt

\aparag The permanent Treaty of this empire with European countries is
nullified. From now on, it is dealt with as a normal non-European country.

\aparag The technology of both \paysInca and \paysAzteque raise to the
technology of \paysChine and other countries of \ROTW.

\aparag Both empires can still be destroyed by capturing their capital city if
the invading forces survive an attack by Natives at the end of turn. The
normal rules are then applied: creation of a \COL of level 3, destruction of
the minor country, reduction to 2 \DT of the force of Natives in every
province of the region; if the conqueror is \SPA, a Mission is installed in
the new \COL and the highest rank Conquistador present in the region is
nominated as Vice-Roy.



\event{pII:Chinese Expansion}{II-20}{Chinese Oversea Expansion}{1}{PBNew}

\history{abandoned before 1492}

\condition{}
\aparag If \ref{pI:Chinese Expeditions} was not played, play this event and
mark off the present one.
\aparag If \ref{pI:Chinese Expeditions} has been played, play the remaining of
this event.

\phevnt
\aparag \paysChine installs one new \TP of level 1 in \granderegionFormose and
one in \granderegionPhilippines if there is any province still empty, with 1
\DT on each one. It takes the exploitation of one \RES{Products of Orient}
(without concurrence; a Major Power will have to make proper \CONC to take
them back).
\aparag If \paysChine has lost some \TP since \ref{pI:Chinese Expeditions}, it
declares an overseas war to any European country having a \TP or \COL in the
same region as any lost \TP. If it has lost none, it declares an Overseas War
to any European power having a \TP in \granderegionFormose or
\granderegionPhilippines.
% (Jym) What? What happens in a case of \MIN/\MIN?

\phmil
\aparag If \paysChine is at war due to this event, it adds one \ARMY\faceplus
to its basic forces, as an invasion force with a general from the minor
pool. Its reinforcements are increased in this war by \LD and \ND. It can of
course use its usual basic forces and reinforcements, and the Natives in
\paysChine.

\phpaix
\aparag If \paysChine controls a foreign \TP at the end of the military turn,
they do not burn it if they have a \TP counter available and this \TP is
transformed in a Chinese \TP. If there is no counter available, the \TP is
burnt down. The choice of the \TP that are converted is random if there is not
enough counters.
\aparag On the first turn of this war (only), \paysChine does not accept
automatically a white peace. A formal peace should be obtained.



\event{pII:Apparition Mughal Empire}{II-21}{Apparition of the Mughal
  Empire}{2}{PBNew}

\history{1526-1555}

\phevnt
\aparag On the first event, the non-European minor country \paysMogol is
created. It has 2 \ARMY\faceplus and the leader \leader{Great Mughal} (until
replaced by a further event).
\aparag The \paysMogol will try to invade \textbf{2} regions during the turn,
following the procedure \ref{pII:Mughal Expansions} described underneath.
\aparag Even if the country does gain no region, it still exists (and can gain
provinces with new events).



\event{pII:Mughal Expansions}{II-A}{\paysMogol Expansions}{*}{PBNew}

\activation{When a \nameref{pII:Mughal Expansions} is called for by an event.}

\phevnt
\aparag The \paysMogol will try to invade the regions in (or near) India in
the following order: \granderegionDelhi, \granderegionAfghanistan,
\granderegionAoudh, \granderegionBengale, \granderegionGujarat,
\granderegionPendjab,\granderegionIndus, \granderegionBalouchistan,
\granderegionOrissa, \granderegionGondwana, \granderegionMumbai,
\granderegionHyderabad, \granderegionMalabar, \granderegionKarnatika. A
circled number on the map shows this order.
\aparag Each event will call for a varying number of invasions (between 1 and
4). The province invaded are determined and the invasion resolved in
parallel. The provinces are aimed in the following order.
\bparag The regions with the lowest number and no \paysMogol
\countermark{Area} counter in it (so it is not ``conquered'' or ``lost'' due
to failed invasion or a rebellion) are the first aimed, by an invasion. Note
that a failed invasion during one event will force the \paysMogol to invade
again the same region during the next expansion.
\bparag Then if needed, the regions having a \paysMogol \countermark{Lost
  Area} counter and with the lowest number are second to be aimed at, for a
new invasion that will have a malus of \bonus{-1}.
\bparag If there is not enough uncontrolled regions to make all the attempts
called for by an event, a test of Rebellion is made in replacement for the
remaining actions called for. The regions aimed are those that are conquered
and have the highest number. A Rebellion is resolved as a invasion but with
\bonus{-1}.
\aparag The list of regions invaded is defined globally during the event, and
the resolution will wait the end of the turn. The \paysMogol is not
(technically speaking) at war with countries having \TP/\COL or regions in the
aimed regions. The invasion attempt will be resolved at the end of the
military phases. Thus, the expansion does not interfere with other kinds of
war that can take place and involve the \paysMogol.

\phinter
\aparag[European resistance to invasion]
\bparag Each country having a \TP/\COL in a province of an invaded region can
choose to oppose or not the Mughal invasion at the end of the military
rounds. The Major Powers decide simultaneously. This decision is taken
province by province (one can resist somewhere and do nothing somewhere else)
and one needs a land stack to resist in a given province. An opposition does
not affect the diplomatic status of any power with the \paysMogol
\bparag Non-European minor countries do not oppose invasion. European minor
countries may oppose if their diplomatic patron decides it. They can use their
non-European basic forces for this.
% (Jym) What about neutrals?
\bparag In each province where invasion is resisted, a land battle is fought
between the forces of the European country and the 2 \ARMY\faceplus of the
\paysMogol This complete force is used in each battle (assuming that they have
plenty of time to muster reserves).
\bparag If the region is not invaded but in Rebellion, the \paysMogol use only
one \ARMY\faceplus.
\bparag The current leader of the \paysMogol is used in each battle.
\bparag Depending on the winner of the battle, the invasion test will be
modified to improve or lower the chance of conquest by the \paysMogol. Note
that no resistance is not as bad as a failed resistance.
\aparag[Invasion tests] For each invaded region, a test is made on the
following table, by rolling 1d10 added to modifiers.

\newcommand{\mughalinvasions}{ \GT{mughalinvasions}{Mughal Invasions}%
  \GTcontent{%
    \graytabular\begin{tabular}{lll} 1d10+mod. & Result & \TP/\COL
      Loss\\\hline\ghline%
      \textlessequal1 & 1 adjacent province is lost& 0 \\\ghline%
      2--4 & failed conquest & 1 \\\ghline%
      5 & failed conquest & 2 \\\ghline%
      6--7 & conquest & 3 \\\ghline%
      8--9 & conquest & 4 \\\ghline%
      10--11 & conquest & 4 \\\ghline%
      \textgreatequal12 & conquest & 6 \\\ghline%
    \end{tabular}
  }%
  \GTlegend[caption=captiona,east north,text width=67mm]{%
    \begin{modlist}
    \item[+3] if \shortleader{Akbar} leads the invasion
    \item[+2] per battle gained in resistance in the region
    \item[-2] per battle lost in resistance in the region
    \item[-1] if the region belongs to a minor country or has a \TP of a
      non-European minor country in it.
    \item[-1] if the region was lost once, or is in Rebellion
    \item[\textpm?] modifier called by some events.
    \end{modlist}
  }%
  \GTdecorate%
}

\GTtable{mughalinvasions}

\aparag[Invasion results]
\bparag \textbf{Conquest} means a successful invasion. Put a counter in the
region showing that is now belongs to the \paysMogol The first time region
\granderegionBengale is conquered, its resources raise to 2 for each type.
\bparag \textbf{Failure} is just what it means ; the regions is left to its
current owner (even in case of a Rebellion).
\bparag On a \textbf{result of 1 or less}, the conquest is failed (or the
Revolt successful). One region is lost to the \paysMogol; put a \paysMogol
\countermark{Lost Area} counter in the region (or flip over the counter
already therein). The region affected is the first one in the list that is not
already lost by \paysMogol (we give here only the numbers): 2, 11, 14, 13, 12,
10, 9, 8, 7, 6, 5, 4, 3.
\bparag The \textbf{Losses} for \TP/\COL are the level lost by each colonial
settlement in the conquered province. Each level of fortification in the
\TP/\COL forfeited counts for one of those loses (including permanent
fortresses given by cities if there is a \COL; the level may be lost, and
comes back automatically for the next turn).
\bparag If a minor country (\paysGujerat, \paysVijayanagar, or \paysAfghans,
\paysMysore, \paysHyderabad) loses its last region due to an invasion, it is
destroyed immediately. It may reappear later due to new events.



\event{pII:Crusade}{II-B}{Call to Crusade}{*}{JymMod}

\history{Did not happen}
\dure{Until the end of the war.}

\condition{May be triggered by \TUR conquest of christian provinces.}

\phevnt
\aparag[Call to crusade] Each Catholic country has a mandatory free \CB
against \TUR to be used immediately.
\bparag As an exception, the \terme{Sole Defender of the Catholic Faith} must
decide first to use it or not. Then, these \CB are resolved in initiative
order.
\bparag All countries that use this \CB are call crusaders and are
automatically allied against \TUR.

\aparag[Mediation of the Pope] Any Catholic country can immediately propose a
white peace to any or all of its current Christian enemies.
\bparag If one or more of these peaces is refused, the free crusade \CB is
consider to be fulfilled (for the country that asked for the mediation). The
would-be crusader is not forced to declare war on \TUR or loss \STAB.
\bparag Catholic minors always accept this peace. Other minors never accept it
(and thus give an ``excuse'' for not participating).
\bparag If a country does not ask the mediation of the Pope, the fact that it
is at war is not an excuse for avoiding the Crusade.

\aparag[Refusing to participate] Any Catholic country that either refused to
participate or rejected the mediation of the Pope suffers from the following
effects:
\bparag Loss of 1\STAB (2\STAB for the \terme{Sole Defender of the Catholic
  Faith}).
\bparag Loss of the diplomatic control of \paysPapaute.
\bparag All other Catholic majors have a normal \CB against this country this
turn.

\aparag If no major country participates in a Crusade, no minor participates
either and the rest of the event is ignored.

\phdipl
\aparag[Minor Countries and Crusades] The following minor countries only: \HAB
(if Emperor or \paysHongrie has been inherited), \paysHongrie, \paysPapaute,
\paysGenes, \paysChevaliers, \paysToscane and \paysParme always participate in
a Crusade.
\bparag If they are on the diplomatic track of a crusader, they are
immediately raised in \EW (if not already higher).
\bparag Otherwise, they are temporarily put in \EW of the first crusader (the
first country that declared war on \TUR, either the \terme{Sole Defender of
  the Catholic Faith} or the one with higher initiative). They will return
back to the \Neutral box at the end of the crusade.

\aparag Other Catholic minors may participate if controlled by the crusader,
using the normal rules.

\aparag Protestant, Orthodox and Muslims minors may not participate in a
Crusade (even if controlled by a crusader).

\aparag[\paysHongrie, \paysHabsbourg, the \HRE.]
% Jym: Crusade stop in pIV when VEN is minor.
% \bparag \pays{Venise} (as a minor country), even if controlled by a player,
% participates also to the Crusade if one modified die-roll of 8 or more is
% obtained.  This die-roll is modified by +1 for each Venetian province of
% 1492 already conquered by the Turkish player.
\bparag If the \hab is the Emperor of the \HRE, it participates automatically
in the Crusade if at least one provinces of either \paysHabsbourg, \HRE or
\paysHongrie is owned by \TUR.
\bparag \paysHongrie automatically participates in the Crusade on a die roll
of 8 or more. This roll is modified by \bonus{+1} for each province of
\paysHongrie owned by \TUR.

\aparag[Endorsement of \paysPapaute]
Crusaders receive at the end of each Diplomatic phase a global diplomatic
income of 150\ducats, shared equitably between them in divisions of 25\ducats
(the surplus going on the first participant).
\bparag This money is coming from the \paysPapaute, so the usual 50\ducats
gift (see \ruleref{chSpecific:Papacy:Gold}) that \paysPapaute gives for a \AM
status is not perceived anymore.
\bparag This is valid during all the length of the current Crusade. At the
same time, the modifier value for \SUB on \paysPapaute becomes -150.

\phadm

\aparag[Crusader army] The crusaders, whether major or minor, may used the
Crusader \ARMY counters to hold troops of any crusader country.
\bparag Whatever the actual content of these counters, they are considered to
be of class \CAIII and have all the features of this class.
\bparag Track the nationalities of the \LD in these \ARMY in order to give
them back to their owner.
% Jym, adding:
\bparag Crusader \ARMY may be lead by \LeaderG of any crusader country, even
if it has no \LD inside.
\bparag Note that he may well ``pick up'' troops from other crusaders without
their agreement.

\aparag[Military Leader of the Crusade] A \LeaderG or \LeaderA of the first
participant player is chosen as leader of the Crusade. For the duration of the
Crusade, he is considered to possess the highest hierarchical rank (even above
monarchs).
\bparag He is allowed to lead any troops of crusaders countries. He may thus
lead a stack with no troops of his own nationality.

\phmil
\aparag[The way to Crusade] crusaders countries automatically give free access
to their territory and supply to other crusaders.
% Jym, adding:
\bparag In the rare case where two crusaders are still at war elsewhere, they
must choose upon entering enemy territory whether the stack is crusading (and
allied) or not. The status of a stack may not change before it exits enemy
territory. Crusader stacks still in enemy territory at the end of the Crusade
are immediately moved into friendly territory per the peace redeployment
procedure.

% Jym, precision on the definition of "waging crusade".
\begin{designnote}
  The following points are meant to force crusaders to really ``wage crusade''
  and not sit and watch. There are undoubtedly loopholes in them that tricky
  players will find and use to circumvent the Crusade rules. Remember here
  what the spirit of the rule is: if you're part of the Crusade, you must
  really participate in the Crusade. Use good sense and fair play. Do not let
  a player that really participated in the Crusade be punished by this. Do not
  let a player that found a loophole to abuse it. Make an homerule if you
  don't think this correct.
\end{designnote}
\aparag[Participating to the crusade] At the end of the first military round
of each turn of the Crusade, each major crusader country must design one of
its stack with at least 3\LD or \ND (or 6\NGD) belonging to it as a ``main
crusading stack''.

\aparag At the end of each following round, each crusader major country loss
1\STAB unless at least one of the following conditions is true.
\bparag All his troops initially in his main crusading stack (they may split)
have been destroyed (reinforcing the crusading stack does not prevent the
destruction of the initial troops).
\bparag All his troops initially in the crusading stack moved this round and
end up closer to the territory of \TUR or its allies.
\bparag Troops of this country (any troops) have participated this round in at
least one battle (land or sea) or siege (besieger or besieged) against \TUR or
its allies.
\bparag Troops of this country (any troops) are in a province owned by \TUR
(not its allies).

\phpaix
\aparag[Crusades and Separate Peace]
A crusader major country that makes separate peace with \TUR undergoes a loss
of 3 \STAB (instead of the usual 2 for breaking an alliance). This separate
peace also gives, in addition, a temporary \CB to all the other crusading
players against him, valid until the end of the Crusade (instead of the usual
next turn only).
% Jym: useless as TUR can always discuss "informally" with FRA/VEN/... and let
% *him* propose the separate peace. We're not trying to reimplement Grand
% Siecle's peace procedure...
% \bparag If the Turkish player asks for peace from a Crusader, he must ask it
% from all of them. He must choose the provinces part of peace conditions
% among Christian provinces the most recently conquered by him.
\bparag No \MIN participating in a Crusade may be tested by the Turkish player
for separate peace attempts, except if a \MAJ has signed a separate peace with
Turkey (including the same turn).

\aparag[Peace conditions] If \TUR cedes territory to the crusaders, it must be
chosen among the following provinces, in order:
\bparag Any province that were christian in 1492, in reverse order of conquest
by \TUR (the most recent conquest first) ; \provinceMoreas , \provinceHellas,
\province{Terra Sancta}, \provinceLubnan, \provinceAlep.
\bparag Only provinces controlled by crusaders may be chosen.
\bparag These provinces are given back to their 1492 owner (if Christian),
even if he did not participate in the Crusade and recreating it if it was
destroyed. Provinces initially belonging to a non-Christian country are given
to \paysChevaliers.
% Jym: let's avoid creating a "Latin states of Orient" minor...
\bparag Provinces of the \regionBalkans that are automatically annexed by
Christians during a Crusade are also given to \paysChevaliers.
\bparag Each province that \TUR loses during a Crusade give 10\VPs to each
crusader still at war against \TUR.

\stopevents

% Local Variables:
% fill-column: 78
% coding: utf-8-unix
% mode-require-final-newline: t
% mode: flyspell
% ispell-local-dictionary: "british"
% End:

% LocalWords: pI pIII pII Habsburg Kalmar Vassalisation Barbaresques Lublin
% LocalWords: Schmalkaldic Khanates Mughal Risto PBNew RistoMod Ristomod pIV
% LocalWords: Serenissima malus Barbaresque TYW Mascate Khanate Teutoniques
% LocalWords: lll captiona mughalinvasions HRE WIO Jym


\clearpage

% -*- mode: LaTeX; -*-
\definechapterbackground{Political Events of Period III}{DuboisStBarthelemy}
\chapter{Political Events of Period III}
%\section{Period III}
\label{events:pIII}



\subsection*{Event Table of Period III}

\begin{eventstable}[Period III events table]
  \centering\tabcolsep=4.5pt%
  \begin{tabular}{|l|*{6}{c}|l|}
    \hline
    1\up{st}\textarrow& 1-3 & 4-5 & 6 & 7 & 8 & 9 & 10 \\ \hline
    1 & 1  & 1  & 22  & 5   & 22  & R15 & \\
    2 & 6  & 12 & 11  & 1   & R11 & R11 & \textbullet~1--2:\\
    3 & 8  & 11 & 18  & 11  & R6  & 12  & +1 then\\
    4 & 1  & 2  & 19  & R6  & R7  & R13 & \periodref{II}\\
    5 & 11 & 3  & R20 & 4   & 8   & 14  & \\
    6 & 14 & 4  & R21 & 10  & 9   & 20  & \\
    7 & 15 & 5  & 11  & R13 & 10  & 21  & \textbullet~3--10:\\
    8 & 17 & 9  & 7   & R15 & 17  & R23 & \periodref{II}\\
    9 & 20 & 13 & 3   & 16  & 18  & R2  & \\ \hline
    10& \multicolumn{7}{l|}{Roll in \periodref{IV}} \\ \hline
  \end{tabular}
\end{eventstable}
\begin{eventstablespec}[General modifiers for the period]
  For each 4 (complete) turns during which \SPA has taxed \paysHollande since
  the beginning of the game (as per \ref{chSpecific:Spain:Dutch Tax}), the
  second die-roll is modified by \textbf{-1} until \ref{pIII:Dutch Revolt}
  occurs.
\end{eventstablespec}

\eventssummary{%
  pIII:Dutch Revolt|,%
  pIII:VOC|,%
  pIII:League Nassau|,%
  pIII:Amsterdam Stock Exchange|,%
  pIII:East Indian Company|,%
  pIII:End Auld Alliance|,%
  pIII:War Sweden Denmark|,%
  pIII:Oxenstierna|,%
  pIII:War England Scotland|,%
  pIII:Portuguese Disaster|,%
  pIII:Portuguese Annexation|,%
  pIII:Northern Secularisation|,%
  pIII:War Persia Turkey|S{pIII:WPT:Persian Attack}/S{pIII:WPT:Annexation
    Iraq},%
  pIII:Revolt Grenade|,%
  pIII:FWR|L{pIII:FWR Beginning}/L{pIII:FWR Barthelemy}/L{pIII:FWR League}/%
  L{pIII:FWR Succession}/L{pIII:FWR Last Stand},%
  pIII:Revolt Corsica|,%
  pIII:Union Poland Sweden|,%
  pIII:Union Lublin|,%
  pIII:Oprichnina|,%
  pIII:Times of Troubles|O{pIV:Times of Troubles},%
  pIII:War Siberia|,%
} \eventssummary{%
  pIII:Creation Arkhangelsk|,%
  pIII:Persian Safavids|E/O{pIII:Sultanate of Aceh},%
  pIII:Revolt Ceylon|,%
  pIII:Mughal Akbar|E/E/O{pIII:Sultanate of Aceh},%
  pIII:Fall Vijayanagar|E/E,%
  pIII:China Colonial Attitude|E/O{pIII:Sultanate of Aceh},%
  pIII:Sultanate of Aceh|,%
  pIII:Japanese Expedition|,%
  O|,%
  pIII:Union Russia Poland|T{alt. hist.},%
  pIII:Religious War Sweden|T{alt. hist.},%
  pIII:Religious War Poland|T{alt. hist.},%
  pIII:FWR Detailed|,%
  pIII:FWR Beginning|,%
  pIII:FWR Barthelemy|,%
  pIII:FWR League|,%
  pIII:FWR Succession|,%
  pIII:FWR Last Stand|,%
  pIII:FWR Final|,%
}

\newpage\startevents



\event{pIII:Dutch Revolt}{III-1 (1)}{Revolt of the United
  Provinces}{1}{RistoMod}

\history{1568-1609}

\condition{For each occurrence of \ref{pIII:Dutch Revolt}, check the effect
  here.}
\aparag Can only occur after the beginning of period III, unless
\paysVhollande exists. Otherwise re-roll and do not mark off.
\bparag This event triggers the \xnameref{pI:Reformation2} if it has not yet
occurred or the \xnameref{pI:Reformation3} if the second Reformation event had
occurred and not the third.
\bparag If \paysVhollande exists, the Revolt is triggered immediately (either
\xnameref{pIII:DR:First Holland Revolt} or \xnameref{pIII:DR:Subsequent
  Revolts}).
\bparag If \HOL is a major country, apply \ref{pIII:VOC} the second time, and
\ref{pIII:League Nassau} the third time.
% (JCD) If \Holmin exists not owned by \SPA, no further revolt?
\bparag If \HOLmin exists and is not on Spanish diplomatic track, apply
\ref{pIII:League Nassau} instead, then \ref{pIII:VOC} the third time.
\bparag In all other cases, the Revolt of \HOL occurs (possibly again). Keep
reading.
\aparag[Revolt and Spanish religious choice]
\bparag If \SPA is \CATHCR, the Revolt is triggered immediately (either
\xnameref{pIII:DR:First Holland Revolt} or \xnameref{pIII:DR:Subsequent
  Revolts}).
\bparag If \SPA is \CATHCO, \SPA must refuse or grant \emph{Commercial
  Liberties} to \paysHollande. A refusal triggers the Revolt as above.
\bparag If \SPA gives \emph{Commercial Liberties} to \paysHollande, \SPA gains
{\bf 1} \STAB then 1d10 is rolled, added to the following modifiers:
\begin{modlist}
\item[\bonus{+1}] for each turn of taxes on \paysHollande
\item[\bonus{-2}] if the Truce of Augsburg is in effect
\item[\bonus{-1}] if \ENG is Catholic
\item[\bonus{-1}] if \xnameref{pIII:FWR} has occurred at least once and the
  \paysHuguenots never had a favourable truce.
\end{modlist}
The result is:
\begin{modlist}[2em]
\item[\textlessequal0] \paysHollande becomes a Special \VASSAL of
  \SPA\NGTnb{a}
\item[1--2] \paysHollande becomes a normal minor, initially vassal of
  \SPA\NGTnb{b}
\item[3--5] \paysHollande becomes a neutral minor\NGTnb{c}%
  \begin{tikzpicture}[remember picture,overlay]
    \NGTnotabene{}{See~\xref{pIII:DR:Independence without Revolt}}
  \end{tikzpicture}
\item[\textgreatequal6] Revolt (either \xnameref{pIII:DR:First Holland Revolt}
  or \xnameref{pIII:DR:Subsequent Revolts}). The \xnameref{pIII:DR:War Holland
    Portugal} may also be activated.
\end{modlist}


\subevent[pIII:DR:First Holland Revolt]{First Revolt against the Spanish
  Crown}

\phevnt
\aparag The Major country \paysmajeurHollande (or \HOL) is created and
\MAJHOLx changes to this new power according to the rules for the Grand
Campaign.
\aparag \HOL owns its national territory: \theminorprovincesshort{hollande},
regardless of their last owner. \SPA loses 10 \VP for each of the provinces
now of \HOL that were not his just before the event. \paysprovincesne is
dissolved and does not exist anymore.
\bparag Former (non-Spanish) owners of those provinces can declare a war
against \HOL but have no \CB.
\aparag\label{onlyfirstrevolt1} \provinceBrabant and \provinceLimburg are
militarily controlled by \HOL, regardless of their owner at the time of event.
\bparag If this owner is not \SPA, he has the choice to give them to \HOL or
has to declare a limited intervention immediately, as an ally of \SPA in the
Religious War; \HOL may then freely involve fully this power in the war. Both
provinces are valid ground for the war even if the intervention is limited.
\bparag If the owner was \SPA and \SPA conceded \emph{Commercial Liberties} at
the beginning of the event, both provinces are now owned by \HOL; else they
remain Spanish.
\aparag \SPA owns a \Presidio of level 3 in \provinceZeeland.
\aparag \HOL has a \STAB of {\bf +2}, a \DTI and a \FTI of 3; the
technological markers of \HOL are placed 1d6 boxes in front of the Latin
technology, and 7-1d6 for Naval technology (same roll!) Its initial Royal
Treasury is 400\ducats.
\aparag \HOL deploys the following counters: \MNU of \RES{Instruments} in
\provinceZeeland, of \RES{Cloth} in \provinceUtrecht, of \RES{Metal} in
\provinceGelderland (all of level 1, level 2 if \emph{Commercial Liberties}
were granted); 1 \ARMY\facemoins, 1 \FLEET\faceplus, 2 \DT, and 4 levels of
fortification anywhere in owned provinces.
\aparag\label{onlyfirstrevolt2} The current \HOL monarch is
\monarque{Willem I} with values 7/9/9. He lasts seven turns and does not check
for survival during the first three. He is also a general
\leaderwithdata{Willem I}. The government is \terme{Stadhouder}.
\aparag\label{onlyfirstrevolt3} \HOL knows
% Pierre notes, 2008
\seazoneAcores, \seazoneCanarias and 8 other zones of its choice. Sea zones
with malus count as 1+malus zones in this count.
% 1d10+4 sea zones,
% chosen between those of \SPA and \POR, forming a continuous path from Europa
% (add 4 if \emph{Commercial Liberties} were granted).
\aparag All non-Dutch units inside territories held by \HOL are removed as per
normal peace procedure.
\aparag \HOL is at war with \SPA and \SPA is considered to be victim of a
declaration of war at this turn. No calls for allies are made.  This is a
Religious Civil War between \HOL and \SPA (see \ref{chDiplo:Religious Civil
  War}).
\aparag\label{onlyfirstrevolt4} Place a Dutch controlled \REVOLT \faceplus in
\provinceVlaanderen and \REVOLT \facemoins in \provinceFlandre and
\provinceHainaut.

\phdipl
\aparag An Armistice will be possible, after the first turn of revolt (this is
an exception to the rules on Religious Wars).
\aparag Usual foreign interventions are permitted.  if \FRA is involved in
\ref{pIII:FWR}, its intervention is restricted as follows.
\bparag If \FRA is \CATHCR or Protestant, \FRA may only use its own forces
(and not those of the heretic minor) to help the side sharing its
religion. The French heretic minor country may make a foreign intervention by
its own to help the side sharing its religions; this is decided by the \MAJ
that controls this country when it rebels.
\bparag If \FRA is \CATHCO, it can make a foreign intervention with any side
(not both at the same time).

\phadm
\aparag During the first turn of war, \HOL can exceed the purchase limits for
naval units and buy land forces without any double or triple price multipliers
for exceeding the basic allowance.
\aparag\label{onlyfirstrevolt5} All units bought during the first turn of the
war and placed under \monarque{Willem I} become automatically \Veteran.

\phpaix
\aparag This event can terminate in two ways:
\bparag \SPA conquers all \HOL national provinces. In this case \SPA has won
the war and \HOL is no more. \MAJHOL player has to wait for another
opportunity to play a Major country (according to the rules of the Grand
Campaign).  All the rules for \paysHollande possessed by \SPA are applied
again. The \COL or \TP of \paysHollande remain and are part of \SPA for
military aspects, but they can not be improved. The commercial fleets are
managed as before the war. The Taxation of Holland is possible anew.
\bparag A peace of any kind is made between \SPA and \HOL. Exceptionally, a
peace of level 5 allows the transfer of any number of provinces (3 if the
powers do not agree). As an additional condition to normal peace conditions
\SPA must recognise the independence of \HOL after which all normal rules
apply and \HOL has become an ordinary player country.
\aparag A peace treaty between \SPA and \HOL cannot be made during the same
turn the revolt event occurred.  White peace is not allowed to end this war.
\aparag Any peace treaty between \SPA and \HOL entails an enforced peace of 3
consecutive turns between those two countries, that can only be broken by
using a \CB given by an event.  During this period, neither of them can
declare war to the other, nor to their respective vassals.
\aparag After peace has been made between \HOL and \SPA, \HOL can continue
harassing Spanish annexed Portugal (see \ref{pIII:DR:War Holland Portugal})
until the end of period IV.
\aparag During the war between \HOL and \SPA neither side loses \STAB due to
the number of turns engaged in war as per normal rules. Instead, for being at
war with each other, or with the allies of each other, they lose the following
fixed amounts:
\begin{modlist}
\item[Period III] \SPA {\bf 1} \STAB, \HOL {\bf 1} \STAB.
\item[Period IV] \SPA {\bf 2} \STAB, \HOL {\bf 1} \STAB.
\item[Period V+] \SPA {\bf 3} \STAB, \HOL {\bf 2} \STAB.
\end{modlist}
\aparag This applies only to the war between \HOL and \SPA due to this event
and only to \SPA and \HOL.  Other allies involved in this war lose \STAB in
the usual manner as well as \HOL and \SPA for non-connected wars.

\phinter
\aparag \SPA receives 5\VP each turn that the Independence of \HOL is not
recognised (the war of Revolt goes on or the Revolt has failed) in period III.
This bonus is reduced to 2\VP during period IV and terminates in period V. The
bonus is given even if the turn was spent in Armistice.


\subevent[pIII:DR:War Holland Portugal]{War between Holland and Portugal}

\condition{If \HOL is in Revolt against \SPA and \paysPortugal has been
  annexed by \SPA according to \ref{pIII:Portuguese Annexation}, add the
  following event to a Revolt (first and subsequent ones).}

\phevnt
\aparag \paysPortugal and \HOL are involved in an Overseas War, as long as the
War of Revolt continues between \SPA and \HOL.
\aparag \paysPortugal uses its forces as defined in \ref{pIII:POR
  Ann.:Portugal Annexed} and \SPA can help it as they are allied.

\phdipl
\aparag An Armistice in the war between \SPA and \HOL does not imply an
Armistice between \PORmin and \HOL.

\phadm
\aparag All \COL and \TP of \POR occupied by \HOL give all their revenue to
\HOL (and none to \SPA) as if owned.

\phinter
\aparag All \TP\facemoins of \POR occupied by \HOL can be replaced by \HOL \TP
with 1 level less.
\aparag Portuguese \TP may not be annexed in this way or burnt by \HOL at the
turn where \SPA recognises the Independence of \HOL (but see afterwards).

\phpaix
\aparag This war terminates at the end of period IV, or if \HOL is conquered
or recognised by \SPA or if \PORmin breaks free from \SPA. In the latter case,
\HOL has a free \OCB against \paysPortugal to be used immediately. Else, \HOL
has to leave Portuguese territory at the end of the turn.
\aparag When the Independence of \HOL is recognised, \HOL can immediately
annexe two \COL or \TP of \paysPortugal, or only one \COL or \TP if the peace
is unfavourable.  In both cases, the level of the \COL/\TP remains the
same. \HOL must have military control of these settlements, but the agreement
of \SPA about which \TP/\COL are gained is not needed.
\bparag Instead of one \TP/\COL, \HOL may obtain the right of implantation of
fleets in \STZ bordering Portuguese \COL/\TP.
\aparag Until the end of Period IV, \HOL having won the Revolt gains an \OCB
against \paysPortugal as long as this country is annexed by \SPA.


\subevent[pIII:DR:Subsequent Revolts]{Subsequent Revolts}

\phevnt
\aparag If the Revolt occurs again after a failed Revolt, the rules are the
same as in \xnameref{pIII:DR:First Holland Revolt} except for the following
points.
\aparag Points \XNRofsectionfalse\ref{onlyfirstrevolt1},
\ref{onlyfirstrevolt2}, \ref{onlyfirstrevolt3},
\ref{onlyfirstrevolt4}\XNRofsectiontrue \ and \ref{onlyfirstrevolt5} above are
ignored.
\aparag Technological markers are where they were left at the end of the
previous Revolt, or at the box of Latin technology (the better). The Treasury
of \HOL is 200\ducats.  The monarch is determined at random; \monarque{Willem
  I} is not available, neither as a Monarch nor as a General.


\subevent[pIII:DR:Independence without Revolt]{Independence without Revolt}

\phevnt
\aparag \future{This option is experimental...}%
\paysHollande becomes a minor country composed of all its national territory:
\theminorprovincesshort{hollande}, regardless of their last owner.  \SPA loses
5 \VP for each of the provinces now in \HOL that were not his own just before
the event. \paysprovincesne is dissolved and does not exist anymore.
\bparag Former (non-Spanish) owners of those provinces can declare a war
against \HOL but have no \CB.
\aparag The characteristics of \paysHollande are as defined in the Annexes.
It has one action of \TP, one action of \COL, one action of \CONC all with
medium investment. It places its \TradeFLEET as in period I or II until the
end of period V; afterwards it has one action for commercial fleet.
\aparag If \paysHollande is not a special \VASSAL of \SPA:
\bparag Any war engaged in period III between \SPA and this country becomes a
Revolt, as per \xnameref{pIII:DR:First Holland Revolt} (keeping existing \COL
or \TP and all discoveries of sea zones made by \SPA (and \POR if annexed by
\SPA) and all its own land discoveries);
\bparag The player formerly in charge of the \TradeFLEET of \paysHollande has
the mandatory task of resolving administrative actions of \HOL and will
resolve its discoveries;
\bparag This player earns \VP for any monopolies of \paysHollande.
\bparag \paysHollande is subject to normal diplomacy;
\aparag If \paysHollande is a special \VASSAL of \SPA, this country has the
task of resolving the administrative actions (which are mandatory).
\paysHollande is not subject to diplomacy.
\aparag \paysHollande may be involved in Overseas Wars, and may declare one
(controller's choice).

\phadm
\aparag If \paysHollande is a special \VASSAL of \SPA, \SPA gains 50\ducats
per turn plus 2\ducats for each face of \COL/\TP of \paysHollande (funds
raised from \paysHollande), instead of the usual income of the provinces for a
vassal.
\aparag Until the end of period V, if at peace or doing limited intervention
only, \paysHollande raises one \FLEET\faceplus and one \ARMY\faceplus to be
used overseas each turn, in discoveries and battles against Natives; it also
has one simple campaign at each round. The named \LeaderE and \LeaderC of \HOL
are used, with a minimum of one \LeaderE and one \LeaderC to be taken in
unnamed counters. The discoveries or wars are resolved by the player in charge
of the administrative actions.
\aparag If at war, it uses its full forces and reinforcements.

\effetlong
\aparag \paysHollande may Revolt against \SPA because of some war between
these two countries in period III.
\aparag Or \paysHollande may break free or/and become a Major Power because of
\ref{pIV:TYW}.
\aparag Finally, a peace of level 5 against \SPA breaks the special status of
\VASSAL and \paysHollande becomes a neutral minor country; the player waiting
to play \HOL according to the rules of the Grand Campaign has the choice to
immediately become \HOL.
\aparag In all those cases, the event and the rules described here terminate.



\event{pIII:VOC}{III-1 (2)}{Vereenigde Oostindische Compagnie}{1}{RistoMod}

\history{Vereenigde Oostindische Compagnie was created in 1602}

\condition{}
\aparag If this event already happened because of \ref{pIV:Dutch Colonial
  Dynamism}, reapply \numberref{pIV:Dutch Colonial Dynamism} instead.
\aparag If \HOL does not satisfy 2 conditions over 3 re-roll and do not mark
off: having at least 3 \TP in \continent{Asia}; this is turn 20 or after;
Dutch government is \terme{Parliament}.

\phevnt
\aparag \HOL may create the VOC at any event phase, as soon as it wants. It
costs 100\ducats and causes the rest of the event.
\aparag At the moment the VOC is created:
\bparag \HOL receives 3 levels of commercial fleets to be placed in any
eligible \STZ bordering \continent{Asia}.
\bparag \FTI for \HOL is immediately raised by one level.


\phadm
\aparag The turn the VOC is created, \HOL may ignore restriction
of~\ref{chAdministration:Pioneering}.

\effetlong
% Maximum \FTI in the \ROTW is now 5 in periods III and IV.
% \aparag \HOL gains one action of \TP and a minimum of one \LeaderC in period
% III.
\aparag \HOL gains an \OCB against any Catholic country having \TP or \COL in
\continent{Asia}, valid during periods III and IV.
\aparag Periods limits of \HOL change once the VOC is created.


\event{pIII:League Nassau}{III-1 (3)}{League of Nassau}{1}{PBNew}

\history{Alternative history}

\condition{}
\aparag If \HOL is a Major country and \SPA did not recognise it, apply \RD
with a \REVOLT in the following table instead of this event and mark off.
\bparag 1.~\provinceZeeland, 2.~\provinceHolland, 3.~\provinceUtrecht,
4.~\provinceLimburg, 5.~\provinceLiege, 6.~\provinceLuxemburg,
7.~\provinceHainaut, 8.~\provinceFlandre, 9.~\provinceVlaanderen,
10.~\provinceBrabant.
\aparag If the Independence of \HOL was recognised or if \paysHollande is
minor country, apply the rest of the event.

\phevnt
\aparag \paysHollande breaks any diplomatic status with \SPA, whether special
\VASSAL or regular diplomatic status and becomes neutral.
\aparag The countries \paysOldenburg, \paysHanovre, \paysHanse and \paysBerg
form an offensive alliance, called the League of Nassau. They leave an
existing \GE.  They are considered as one country for declarations of war and
for peace terms (excepted for separate peaces).
\aparag The League of Nassau declares a war to \paysTreves, \paysCologne and
\paysMayence. The Emperor of the \HRE has a free \CB to declare war to the
League of Nassau and be allied to the three Archdioceses; in this case it
controls these countries. If the Emperor does not involve himself in the war,
the \SDCF will have control of those Archdioceses during the war, or \SPA is
nobody has this responsibility.
\aparag Any country having diplomatic status with one of these minor countries
can do a limited intervention to support this side (and then has to break
diplomatic relations with minor countries of the enemy side), except the
Emperor who can only enter war with the Archdioceses (and can do this in a
limited way or full war).
\bparag Note that if the Emperor is \AUSmin, \SPA can make a limited
intervention on the side of \AUSmin as well.
\aparag If \HOL exists, it can do a limited intervention as an ally of the
League of Nassau.
\aparag The League of Nassau is controlled by the following Major existing
power: \HOL, the player responsible for the administrative actions of
\payshollande (if not \SPA), \ENG if Protestant, \FRA if Protestant, \SUE
(regardless of religion).

\phadm
\aparag The three Archdioceses can use the counter of the \HRE for their
troops even if the Emperor is not at war along them. They take their
reinforcements in defensive mode during the first turn of the war.
\aparag The countries in the League of Nassau take their first turn
reinforcements in offensive mode, except \paysHanse which has Offensive or
Naval reinforcements (controller's choice).

\phmil
\aparag The minor countries of the \HRE that are at war can pass through and
stop in every province of the \HRE. They can not siege or pillage provinces
belonging to minor countries not involved in this war.
\aparag The troops of the Emperor have the same right of passing through and
stopping in the \HRE, as well as the forces in limited intervention of other
Major countries.

\phpaix
\aparag A test to begin a Religious War in \HRE is made at the end of the
first turn of this war started by the League of Nassau.  This test is modified
by \bonus{+2} if \SPA if \CATHCR and \bonus{0} if it is \CATHCO. See
\ref{pIV:TYW} for the result of the test and the possibility of this Religious
War, and the renewal or not of the test on following turns.  If no such war
occurs, peace can be made on the following conditions.
\aparag Each minor country obeys to the usual rules for peace. Those in the
League are allied so a peace against only one country is a separate peace.
\aparag A minor country forced to sign an unconditional surrender breaks from
the League for ever. The League ceases to exist when only one country remains
in it, or at the time of \shortref{pIV:TYW}.
\aparag If the three Archdioceses are not supported by the Emperor, the League
tries to obtain peace using the system for minor countries as if it was one
major country (the controller of the League of Nassau decides of the terms of
peace).
\aparag The controlling player of both sides gain 5 \PV for each level of
favourable peace signed at the end of the war, and 10 \PV for each enemy minor
country that had to sign an unconditional surrender; they lose 10 \PV for each
minor country of their side that had to sign an unconditional surrender.
Those \PV are not awarded if the war triggers \shortref{pIV:TYW}.

\effetlong
\aparag If the League of Nassau exists when \shortref{pIV:TYW} occurs, it will
join the Protestant side. The League ceases existence when there is only one
minor country left in the League at the end of a war.



\event{pIII:Amsterdam Stock Exchange}{III-2}{Amsterdam Stock
  Exchange}{1}{Risto}

\history{1608}

\effetlong
\aparag \HOL can from now on lend 150\ducats in the Diplomacy phase, plus
100\ducats during the turn (instead of 100\ducats plus 50\ducats).

\aparag From now on, \HOL receives a bonus for its International Loan rolls
and Bankruptcy rolls.

\aparag From now on, \HOL is more resilient to exceeding limits in \MNU.

% \begin{oldcompta}
%   \aparag From now on \HOL receives a bonus equal to its \DTI to all die-rolls
%   on international loan amount and interest (not length) in the loans table
%   % (Jym) Table just said "international" and usually more up to date
%   \aparag \HOL is also more resistant to Bankrupt and more tolerant to
%   trespassing of commercial limits.
%   % (Jym) What? According to chapter 3 it should be less resilient
% \end{oldcompta}



\event{pIII:East Indian Company}{III-3 (1)}{East Indian Company}{1}{Risto}

\history{1600}

\condition{}
\aparag If both following conditions are not satisfied: this is turn 20+ and
\ENG has at least 2 \TP in \continent{Asia}, apply first \ref{pIII:End Auld
  Alliance}, or re-roll if already played.

\phevnt
\aparag \ENG may create the EIC at any event phase, as soon as it wants. It
costs 100\ducats and causes the rest of the event.
\aparag \ENG receives 2 levels of commercial fleets to be placed in any
eligible \STZ bordering \continent{Asia}.

\effetlong
\aparag \FTI for \ENG is immediately raised by one level and its maximum level
is permanently raised as written in the tables.
\aparag Turn limits for \ANG change.



\event{pIII:End Auld Alliance}{III-3 (2)}{End of the Auld Alliance}{1}{PBNew}

\history{1560 - Treaty of Edinburgh}
% \dure{until the end of the Anglo-Scottish war}

\condition{}
\aparag Occurs only if \paysecosse is at present inactive. Otherwise re-roll.
\aparag If \ANG has chosen the ``Mary Stuart'' option in \ref{pII:Act
  Supremacy}, this event is void of any effect.
\aparag \ENG can refuse this event (mark as played) by losing {\bf 2} \STAB
and 20 \VP. It also loses the control of \paysecosse and can then make no
diplomacy on it until the end of period.

\phevnt
\aparag Apply \ref{pI:Reformation3} (John Knox in Scotland!).
\aparag \paysecosse wants to declare itself liege of \FRA.  \ENG has the
choice to contest this declaration, by using a free \CB against \paysecosse.
In this case, \paysecosse stays Neutral and Allies can be called for this war
as per normal rules. Else, \paysecosse becomes \VASSAL of \FRA.

\phadm
\aparag For the duration of the event, \paysecosse receives reinforcements in
defensive attitude.



\event{pIII:War Sweden Denmark}{III-4 (1)}{Northern Seven Years War}{1}{PB}

\history{1563-1570}

\condition{This event can not occur if \SUE is not a Major Power; do not mark
  off and re-roll if it is not the case.}

\phevnt
\aparag \paysDanemark declares a war to \SUE. If \SUE was at peace,
\paysDanemark is controlled according to the normal rules. If it was not, the
controller is chosen in priority among the countries at war against \SUE.

\phadm
\aparag During the first turn \paysDanemark will take its reinforcements in
offensive status with an added bonus of \bonus{+2}. For the following turns,
the attitude is free but \paysDanemark keeps the \bonus{+2} to reinforcements
during all this war.



\event{pIII:Oxenstierna}{III-4 (2)}{Oxenstierna}{1}{PBNew}

\history{1612-1654}
\dure{as long as \strongministre{Oxenstierna} remains the excellent minister}

\phevnt
\aparag \SUE receives an excellent Minister, \ministre{Oxenstierna}, which has
values 6/8/8. He will last for 3 turns plus a random length for Minister, see
\ref{eco:Excellent Minister}.
\aparag \SUE gains immediately {\bf 1} in \STAB.

\phadm
\aparag \SUE may ignore restriction of~\ref{chAdministration:Pioneering} for
this turn.



\event{pIII:War England Scotland}{III-5}{War between England and
  Scotland}{1}{Risto}

\history{1542-1548}

\condition{}
\aparag Occurs only if \paysecosse is at present inactive. Otherwise re-roll.
\aparag Cannot take place if \ref{pIV:Union Scotland} has already occurred.
In that case mark-off and re-roll. May cancel \shortref{pIV:Union Scotland} if
the latter occurs while the present event is still active.
\aparag \ENG can refuse this event (mark as played) by losing {\bf 3} \STAB
and 20 \VP.  It also loses the control of \paysecosse and can then make no
diplomacy on it until the end of period.

\phevnt
\aparag \paysecosse declares war against \ENG, which loses the control of
Scotland.
\aparag \ENG can immediately call allies as per normal rules.
\aparag If this leads to declarations of war against \paysecosse, the
controller of \paysecosse may come to its help as per normal rules, and so on.
\aparag If \paysecosse is neutral, its control is decided randomly between
\SPA and \FRA unless one of them is already at war with \ENG (and the other
not), in which case that country takes precedence and receives \paysecosse in
\EG. Control cannot be refused.

\phadm
\aparag For the duration of the event \paysecosse receives reinforcements in
offensive attitude.



\event{pIII:Portuguese Disaster}{III-6}{Portuguese Disaster in
  Africa}{1}{Risto}

\history{1578}

\condition{}
\aparag Can occur only if \paysportugal exists as a minor country, otherwise
re-roll.
\aparag If \ref{pIII:Portuguese Annexation} is in effect, apply \RD with a
\REVOLT in \SPA.
\aparag Else if \dynasticaction{C}{3} was played, activate
\shortref{pIII:Portuguese Annexation} just after the effects of this event.

\phevnt
\aparag If \paysPortugal is currently activated in a war, it immediately
offers a mandatory white peace to all its enemies.
\aparag \paysPortugal loses all its non-national provinces (excepted
\provinceTanger and \provinceAcores); they are given back to their owner of
1492.
\aparag Whatever the current status of \paysPortugal, the reference level of
each Portuguese \TradeFLEET in the \ROTW map is reduced by one (even if being
thus eliminated).
\aparag All Portuguese fortifications in the \ROTW map outside
\continent{Asia} and \continent{Brazil} lose {\bf 1} level.  Remaining
fortifications are added to the basic forces maintained by \paysPortugal (but
will not be rebuilt once destroyed).
\aparag From now on, \paysPortugal has only one action of \TP/\COL each turn,
and no fleet action.



\event{pIII:Portuguese Annexation}{III-7}{Annexation of Portugal by
  Spain}{1}{RistoMod}

\history{1580-1640}

\condition{Can occur only if \paysPortugal is a minor power.}

\phevnt
\aparag \SPA receives a free \CB against \paysPortugal until the end of
current period.  If \SPA is \CATHCR, then during the first turn of a war
caused by this event, \paysPortugal receives no reinforcements.
\aparag In addition to the usual involvement of a \MAJ to help an attacked
minor country, \ENG and \FRA can make a limited intervention to help
\paysPortugal.
\aparag[Annexation]
\bparag If \xnameref{pIII:Portuguese Disaster} has not happened yet and \SPA
achieves an unconditional victory over \paysPortugal, this minor is considered
to have been annexed to \SPA in a special way and \xnameref{pIII:POR
  Ann.:Portugal Annexed} is applied. The political marker of \pays {Portugal}
is placed in \ANNEXION of \SPA.
\bparag If \nameref{pIII:Portuguese Disaster} has happened, \paysPortugal is
at war by its own (neither full nor limited intervention), \SPA can annex
\paysPortugal by winning a peace of level 2 against it.
\bparag If \nameref{pIII:Portuguese Disaster} has happened and \paysPortugal
has help from a \MAJ, \SPA will annex \paysPortugal by winning a peace of
level 4 against it.


\digression[pIII:POR Ann.:Portugal Annexed]{Portugal in Annexation}

\phdipl
\aparag \paysPortugal is permanently annexed to \SPA, and its political marker
is placed accordingly. The counters of \paysPortugal are not removed from
play.
% \aparag \provinceTanger is given by \paysPortugal to \SPA.
\aparag In game terms \paysPortugal is treated as a part of \SPA mainly for
\VP purposes.  In most other respects it becomes a special, permanent \VASSAL
of \SPA. This applies to separate wars and peace treaties, placement of units
and markers, etc. and covers all aspects not specially modified in this event
description.  If \paysPortugal was currently engaged in a separate war against
someone else than \SPA, its enemies must immediately sign a white peace with
it, or declare war to \SPA with a free \CB (unless they are already at war
with \SPA).
\aparag \SPA annexes all non national provinces of \paysPortugal except
\provinceAcores.
\aparag \SPA cannot voluntarily cede any part of \paysPortugal, including
\COL/\TP to other players. Neither can it sell Portuguese sea charts or grant
authorisation of trade in a sea bordering a Portuguese \COL/\TP.
\aparag A War declared against annexed \paysPortugal gives a free \CB (\OCB if
this is an Overseas war) to \SPA to intervene in the war.  A war against \SPA
does not imply necessarily \paysPortugal in the war.

\phadm
\aparag \SPA receives a part of the incomes of \paysPortugal: it receives all
income from \TP/\COL, Exotic Resources, \TradeFLEET (but no income from
European provinces, foreign or domestic commerce, manufactures -- these are
removed). This income can not be higher than 400\ducats, plus the East Indies
convoy.
\bparag \SPA gains the \PV for the monopolies detained by \paysPortugal. It
does not combine resources or fleets of \paysPortugal with it to determine
monopolies or the ownership of a Commercial Centre.
\aparag \SPA must pay for the maintenance and recruitment of Portuguese units
and fortresses as if they were Spanish units (except that their content
remains that of Portuguese units and they can only be placed within Portuguese
territory, including \COL/\TP).  \SPA has 3\GD of basic forces and an
additional limit of recruitment of 1\LD and 1\ND to maintain or raise
Portuguese units. One unnamed Portuguese \LeaderE leads the naval forces.
\aparag \SPA can make administrative actions for Portuguese \TradeFLEET and
\COL/\TP, but using Portuguese \FTI/\DTI (without the former Portuguese bonus
for \ROTW actions).  \SPA has 2 (in periods III and IV) or 1 (period V)
actions for Portuguese \COL and can use also its own actions for Portuguese
establishments. One of these actions can be used on a Portuguese \TP each
turn. \SPA has one action of \TradeFLEET in periods III and IV for Portuguese
fleets.

\phmil
\aparag \SPA must pay for all campaign activations of Portuguese units jointly
with Spanish units.

\phpaix
\aparag \SPA can renounce annexation at the end of any peace phase (except on
the same turn when \ref{pIV:Portuguese Revolt} occurs) losing control of
\paysPortugal and {\bf 3} \STAB.
\bparag If \SPA renounces the inheritance before \shortref{pIV:Portuguese
  Revolt} occurs, \paysPortugal is placed in forced \EG of \SPA until the
death of current Spanish monarch. After that, it is treated as normal minor
and subject to diplomacy.
\bparag If \SPA renounces the inheritance after \shortref{pIV:Portuguese
  Revolt} has occurred, \paysPortugal becomes neutral and it makes a white
peace with \SPA.  The rebels are considered to have won.



\event{pIII:Northern Secularisation}{III-8}{Secularisation of
  \pays{Teutoniques1}}{1}{PB}

\history{1561}

\condition{}
\aparag If \ref{pI:Fall Teutonic} was not played, it is played this turn as a
supplementary event.

\phevnt
\aparag Minor country \pays{Teutoniques1} is destroyed.  Its provinces are
shared as follows:
\bparag \provinceEstland is given to \SUEsue.
\bparag \provinceMemel joins the \region{Duche de Prusse} and is given to
whoever controls this Duchy (\POLpol or \paysBrandebourg).
\bparag \provinceLivonija and \provinceKurland are associated as the
\region{Duche de Kurland}. This Duchy is claimed by \SUEsue and \POLpol.
\bparag If one of these provinces was conquered by another country than the
one that should take it, this wronged country has a \CB against the country
possessing the province. A minor country will always use this \CB.
\bparag All other provinces are given to their legitimate owner in 1492 (as
indicated on the map).
\aparag[War for Kurland]
\bparag \POLpol has a \CB against \SUEsue; refusal to use it costs {\bf 1
  \STAB} and gives all the \region{Duche de Kurland} to \SUEsue. \POLmin
always uses the \CB.
\bparag \SUEsue has a \CB against \POL; refusal to use it costs {\bf 1 \STAB}
and gives all the \region{Duche de Kurland} to \POL. \SUEMin always uses the
\CB.
\bparag If both countries use their \CB against the other one, \POLpol owns
both provinces, but \SUE has initially the military control of
\provinceLivonija.  They can make no Armistice on the first turn of this war.
\bparag If neither \SUE nor \POL use this \CB, \payscourlande is created as a
normal minor country with the two provinces.

\phdipl
\aparag Any country which was at war against \pays{Teutoniques1} has an
immediate free \CB to be used jointly against \POLpol and \SUEsue (and
\payscourlande if it exists).  This might provoke a three-sided war (excepted
if one of \POL or \SUE at least has abandoned the \region{Duche de Kurland})
in which the invading country keeps its eventual initial military control of
any province in \pays{Teutoniques1}.
\aparag If such a country does not declare war, its forces are withdrawn from
\pays{Teutoniques1} and it gives up any conquered province that was owned by
\pays{Teutoniques1} in 1492 to their new owner (as defined above).
\aparag Any other country adjacent to \pays{Teutoniques1} when they disappear
has a \CB to be used jointly against \POLpol and \SUEsue (and \payscourlande
if it exists).



\event{pIII:War Persia Turkey}{III-9}{War between Persia and Turkey}{1}{Risto}

\history{1606-1639}

\condition{}
\aparag If main provinces of \paysperse are conquered, activate a
\xnameref{chSpecific:Persia:uprising}.
\aparag First time : if \paysperse is inactive, use \xnameref{pIII:WPT:Persian
  Attack}.
\aparag Second time, or first time and \paysperse is currently at war against
\TUR, use \xnameref{pIII:WPT:Annexation Iraq}.
\aparag Otherwise, re-roll and do not mark off.


\subevent[pIII:WPT:Persian Attack]{Persian Attack of Turkey}

\activation{}
\aparag If \TUR does not own provinces that were Persian at the beginning of
the game, it may refuse the event in two ways:
\bparag By losing {\bf 3} \STAB and 150\ducats.
\bparag Or, by surrendering immediately to \paysPerse conceding a peace of
level 2 and ceding a province bordering Persian territory (in priority a
province adjacent to \paysPerse).
\aparag In this case the box is marked off, but the event can happen later if
rolled for anew.

\phevnt
\aparag \paysPerse declares war against \TUR.
\aparag \TUR can immediately call for allies as per normal rules.
\aparag If this leads to declarations of war against \paysperse, the
controller of \paysperse may come to its help as per normal rules, and so on.
\aparag If \paysperse is neutral, it is played by \SPA (which cannot then come
to its aid).

\phadm
\aparag \paysperse receives reinforcements in offensive status for the
duration of the event.


\subevent[pIII:WPT:Annexation Iraq]{Annexation of Iraq}

\phevnt
\aparag \paysIrak is annexed to \paysperse and removed from game.
\aparag If \TUR owns any province initially in \paysIrak, place there a
\REVOLT \faceplus and one or 2 \REVOLT \facemoins controlled by \paysperse;
one \REVOLT in each province, the \REVOLT \faceplus is placed at random.

\phadm
\aparag If either of the conditions above are met with, Iraqi basic force is
added to the forces of \paysperse until the end of the war.



\event{pIII:Revolt Grenade}{III-10}{Revolt in Sierra Nevada}{1}{Risto}

\history{1568-1570}

\phevnt
\aparag Place a \REVOLT \facemoins in non-Muslim \provinceGranada,
\provinceCordoba and \province{La Mancha}. The \REVOLT are controlled by \TUR.

\phdipl
\aparag \TUR has a \CB against all the owners of revolted provinces.
\bparag Exceptionally, \TUR may make a limited intervention at the side of the
\REVOLT as if this was a civil war.
\aparag If \TUR declares war to the controller of \provinceGranada or is in
limited intervention against it, it receives 5 \VP at the moment its (or its
minor allies) troops arrive to any of the revolted provinces. This does not
have to be done during the current turn, but the bonus \VP are gained only
once.

\phmil
\aparag During the rebellion there exists an additional malus of \bonus{-3} to
all attempts to suppress \REVOLT in \provinceGranada if \SPA is \CATHCR. An
additional malus of \bonus{-1} is received for each Turkish or minor allied
\LD inside any province in \REVOLT (even if besieged).

\phinter
\aparag \REVOLT caused by this event can never extend beyond \provinceGranada,
\provinceCordoba, \provinceMurcia and \province{La Mancha} (with a maximum of
two \REVOLT counters per province).
\aparag If the \REVOLT survives the first turn, place a minor general on it.
\aparag For each interphase this event continues \TUR receives 2 \VP.  This
bonus is increased to 10 \VP per interphase whenever \TUR or its minor ally
units are within \provinceGranada (a war must have been declared to the
controller to do this).
\aparag If a \REVOLT \faceplus exists for a whole turn in \provinceGranada
without being suppressed at any point during this turn, a new minor
\paysGrenade is created and becomes a permanent \VASSAL of \TUR (but the war
is not necessarily ended). It owns any of the 4 mentioned provinces having a
\REVOLT in them, but has no capital (so can be destroyed by any country).
\aparag If \provinceGranada is later annexed by any other player than \TUR,
place a \REVOLT \facemoins in the province during the peace phase and consider
this event as having been activated again, but without the malus of \bonus{-3}
for suppress of \REVOLT . If \paysGrenade still exists (owning other provinces
than \provinceGranada), consider this \REVOLT as being controlled by it.

\effetlong
\aparag[Final expulsion of the Moriscos] Certain effects of the politics of
expulsion are removed.

\event{pIII:FWR}{III-11}{Wars of Religion in France}{5}{PBNew}

\history{1560-1598}

\condition{See at the end of this section the \ref{pIII:FWR Detailed} which is
  the detailed description of those wars.}



\event{pIII:Revolt Corsica}{III-12}{Revolt in Corsica}{1}{Risto}

\history{1564-1567}

\phevnt
\aparag A \REVOLT \faceplus is placed in \provinceCorsica. The preference list
for the control of this \REVOLT is the one for the (would-be) \paysCorse.
However, the \REVOLT cannot be controlled by the controller of \paysGenes, who
is omitted from this list.
\aparag \paysGenes immediately offers white peace to any enemy currently
engaged in war with it.  From now on, it cannot declare war on anyone as long
as the event lasts.
\aparag If no-one controls \paysGenes at present, the controller is chosen as
per normal rules when minor neutral is activated.
\aparag This event is treated as a civil war in \paysGenes (see
\ref{chDiplo:Religious Civil War}). Only the controllers of \paysGenes and of
the \REVOLT are allowed to do a \terme{Foreign Intervention} with their own
forces.

\phadm
\aparag This event must be played even if no player country is involved in
it. \paysGenes receives reinforcements and can use its troops as if activated
in a war.

\phinter
\aparag If the \REVOLT survives the first turn, place \leaderSampiero who is
now available for 5 turns.
\aparag If the \REVOLT survives four turns, a new minor country \paysCorse is
created and the rebellion is over. The controller of the \REVOLT gains 10
\VPs.
\aparag If the rebellion is crushed, controller of \paysGenes gains 10\VPs.


\event{pIII:Union Poland Sweden}{III-13}{Union between \paysmajeurPologne and
  \paysmajeurSuede}{1}{PB}

\history{1595-1599}

\condition{}
\aparag If there is no Major power \POL, re-roll and do not mark off.
\aparag If there is no Major power \SUE and \POL is not Supporter of
Orthodoxy, re-roll and do not mark off.
\aparag If the Polish Monarch is \monarque{Zygmunt I} during its first 5 turns
of reign, re-roll and do not mark off.
\aparag Apply one of the following events, according to the religious
attitudes:
\bparag If \SUE is Catholic, apply \ref{pIII:Religious War Sweden};
\bparag If \POL is Supporter of Orthodoxy, apply \ref{pIII:Union Russia
  Poland};
\bparag If \POL and \SUE are Protestant, apply \ref{pIII:Religious War
  Poland}.
\bparag If \SUE is Protestant and \POL is Catholic, use this present event.
\bparag If none of the preceding situations happened, mark off the box and
apply \RD.

\phevnt
\aparag The Polish Monarch dies and the Heir of the Swedish Crown is elected
in Poland. \POL has now the Monarch \monarque{Zygmunt III}, with values 5/5/6
and is also general \leaderwithdata{Zygmunt III}. Its reign will last 9 turns.
\aparag The Vasa Dynasty remains on the Polish throne until a Dynastic crisis
occurs in Poland or an event (or some elected specific general) changes the
Dynasty; \POL has to lose 2 \STAB to keep its Dynastic Claims or this
terminates the event.  From now on, \POL has Dynastic Claims over \SUE.

\effetlong
\aparag \POL can renounce its Claims at any diplomatic phase (that is a
declaration) and that terminates the event. \POL loses 1 \STAB.
\aparag Each time there is a new monarch in \SUE, \POL has a \CB against \SUE
at this turn to claim for its Inheritance. In case of Dynastic Crisis in \SUE,
\POL is a valid pretender as long as it has Dynastic Claims over Sweden.
\bparag The first time after the beginning of the event that this situation
happens, \POL must either use the \CB or lose 2 \STAB or renounce its Claims
(costs 2 \STAB).
\aparag The first new Swedish Monarch after this event will be
\monarque{Charles IX}, with values 8/6/6 (but not a general) and random
duration (ignore \terme{Fragile health} and \terme{Dynastic crisis}.
Exception: if \monarque{Gustave Adolphe} was to be the new monarch due to
another event, use \monarque{Gustave Adolphe}.

\phdipl
\aparag If a war is declared because of its \CB, \SUE is now in Civil
Religious War (see \ref{chDiplo:Religious Civil War}). Apart from \POL, only
foreign intervention in the war is allowed.
\aparag The first time a war is declared due to Dynastic Claims, \POL gains
the military control of one province owned by \SUE, chosen by \POL (the
capital is forbidden). This effect is not applied for subsequent wars.

\phadm
\aparag \POL can recruit troops in Swedish provinces that are under its
military control, at double price (because those are not normal recruitment
provinces).
\aparag \POL can use outside its own territory only land forces paid with
ducats and not paid with free maintenance (mercenaries only). There is no such
restriction for naval forces, nor if the kings of \SUE are \PROTRIG in which
case the war is not limited for \POL. Note that it is not mandatory to use the
free maintenance.

\phpaix
\aparag If \SUE wins the war, a valid peace term is to ask for renouncement to
Dynastic Claims (equivalent of one province).
\aparag If \POL wins the war with a peace of level 3 or more, or forces an
unconditional peace, the Monarch of \POL becomes ruler of \SUE as one of the
Victory conditions (instead of one province).
\bparag The Monarch of \SUE is executed; now \SUE uses the values of the
Monarch of \POL. \SUE is considered Catholic during the Union (in every
aspects).
\bparag \SUE has a mandatory offensive alliance with \POL in which she is
complied to answer any call.
\bparag \SUE can not declare war without a \CB or the agreement of \POL.  It
can not declare war against \POL.
\bparag The alliance is in question when the Monarch of \POL dies or if \POL
refuses to answer a call for defensive war (not offensive war), or if \POL
declares a war against \SUE.  A new monarch is rolled for \SUE. \POL having
still Dynastic Claims over Sweden, it can renew the war to impose its ruler
but it renews the Union if \POL wins a peace of any level against \SUE. As
long as the war continues, the union exists for Victory Conditions, if not in
its consequences.
\bparag Note that if \ref{pIII:Union Poland Sweden} is rolled for a new time
when the Union exists, \SUE is Catholic and \ref{pIII:Religious War Sweden} is
thus applied.



\event{pIII:Union Lublin}{III-14}{Union of Lublin}{1}{PB}

\history{1569}

\condition{}
\aparag This event is described in \ref{pII:Union Lublin}.
\aparag If it has already occurred, mark off and apply either
\ref{pIII:Oprichnina} or \ref{pIII:Times of Troubles}.



\event{pIII:Oprichnina}{III-15 (1)}{Oprichnina}{1}{PB}

\history{1565-1572}
\dure{as long as there is a \REVOLT in Russia.}

\condition{}
\aparag If \monarque{Ivan IV} has not been yet Monarch of \RUS, do not mark
off and re-roll.
\aparag If \monarque{Ivan IV} is already dead, mark off and apply \RD the
first time (with a \REVOLT in \RUS), the second event the next time.

\phevnt
\aparag \RUS is in Civil War for the duration of the event.
\aparag \REVOLT are placed in \provinceMoscou and \provinceNovgorod; their
force is randomly decided.
\aparag Another \REVOLT is rolled for in Russia.
\aparag The Russian leader \leaderKurbsky is withdrawn from game as long as
\monarque{Ivan IV} rules in \RUS and can not be used.

\phadm
\aparag \RUS is not restricted by limits of land building this turn only, and
has no penalty for doing so. %
% Jym, 05/2011 No more restricted area for RUS.  As \provinceMoscou is
% occupied by a \REVOLT through, it has to pay double cost to build troops
% somewhere else.
However, the cost for building new troops is doubled for the duration of this
event.

\phmil
\aparag \monarque{Ivan IV} must take the field and lead a land stack as long
as this event last, respecting the usual hierarchy rules.
\aparag The land force of \monarque{Ivan IV} pillages every province it is in
at the end of each round.
% Bertrand is tied, brought before the cine-club and must watch in extenso
% Ivan the Terrible of Eisenstein, and this once per turn as long as the event
% is active. Sylvain has the moral duty of commenting the movie.

\phpaix
\aparag If at the peace phase there is no \REVOLT left in \RUS, one Russian
\ARMY (one counter and the equivalent of 4 \DT) is destroyed by \RUS and \RUS
gains {\bf 1} in \STAB.



\event{pIII:Times of Troubles}{III-15 (2)}{The Time of Troubles in
  Russia}{1}{PB}

\history{1605-1613}

\condition{}
\aparag If \ref{pIII:Oprichnina} is still in effect, mark off and apply \RD.
\aparag If not, apply \ref{pIV:Times of Troubles}.



\event{pIII:War Siberia}{III-16}{War in Siberia}{1}{Risto}

\history{non-historical}

\condition{Can occur only after the elimination of \payssiberie.  Otherwise
  re-roll.}

\phevnt
\aparag Place a Turkish controlled \REVOLT \facemoins in each Russian \COL/\TP
in \continent{Siberia}.

\phadm
\aparag Native forces within the revolted provinces return to their full
strength and are activated.
\aparag Furthermore, during the first turn only, an unmodified die-roll is
made for rebel reinforcements in offensive attitude. Troops thus received fill
the former \payssiberie counters and can be placed in any of the revolted
provinces.

\phmil
\aparag Rebels using \payssiberie counters draw supplies from native
territories (the same way as natives do), but can only do so either if there
is no \RUS controlled forts/fortresses in the province, or from the \REVOLT
counters, which they can use as supply bases.
\aparag Rebels using \payssiberie counters can move also outside their
original provinces.
\aparag Rebel natives and \payssiberie units automatically try to destroy
Russian \COL/\TP in provinces they occupy at the end of a full round, if these
are not protected by Russian units or fortresses. Roll one die: on 7 or more,
the \COL/\TP is destroyed.

\phinter
\aparag \REVOLT caused by this event never extend during the redeployment
phase.
\aparag During the native attacks phase count each \REVOLT \facemoins counter
as 2 native \DT when counting the modifications to the attack die-roll, and
rebel forces using \payssiberie counters are also used.



\event{pIII:Creation Arkhangelsk}{III-17}{Arkhangelsk and the Muscovy Trade
  Company}{1}{Risto}

\history{1584}

\condition{Requires permission from \RUS and \ENG to take effect. Otherwise
  re-roll.}

\phevnt

\aparag The port of Arkhangelsk (to the north of the European map) is created.
It cannot be accessed by any units, but still meets the requirement of having
a port along the Atlantic Ocean for purposes of placing commercial fleets.

\aparag \CTZ Russia is created, but its monopoly bonus remains 5 until the
\xnameref{chSpecific:Russia:St-Petersburg}.

\aparag English commercial fleet of 4 levels is placed in \CTZ Russia.

\aparag Muscovy Trade Company provides \ENG automatically with 10 \VP and
50\ducats.

\phadm

\aparag Until the \nameref{chSpecific:Russia:St-Petersburg}, \ENG can use both
its \DTI and \FTI as modifiers to all commercial actions in \CTZ Russia.
\aparag \RUS may ignore restriction of~\ref{chAdministration:Pioneering} for
this turn.



\event{pIII:Persian Safavids}{III-18}{Persian Safavids}{1}{PB}

\history{1590-1722}

\phevnt

\condition{}
\aparag If main provinces of \paysperse are conquered, activate a
\xnameref{chSpecific:Persia:uprising}.
\aparag Else, apply only the following effects.

\phevnt
\aparag \paysperse obtains the general \leader{Abbas Shah} that will stay for
6 turns.

\effetlong
\aparag \paysperse has now the same technological level as \TUR.  Its armies
are of class \CAI and it has 3 \ARMY available.
\aparag \paysperse can now send armies through regions in \ROTW belonging to
no one during wars, without activation of Natives. They are constrained by the
supply rules. They can assail and burn \TP or \COL (as if \TP) military
occupied at the end of a turn.



\event{pIII:Revolt Ceylon}{III-19}{Revolts in \granderegionCeylan}{1}{Risto}

\phevnt
\aparag \ROTW area \granderegionCeylan declares war against the owner of a
\TP/\COL in it.
\aparag If this is a minor country, the \TP/\COL will be attacked by the
Natives at the end of the military turn, without any defence from Europe.
\aparag If this is a player, the war proceeds as a normal war against natives.



\event{pIII:Mughal Akbar}{III-20}{The Great Moghol Akbar}{2}{PB}

\history{1556-1605}

\phevnt
\aparag If the non-European minor country \paysMogol does not exist, it is
created now.  It can use 2 \ARMY\faceplus and leader \leaderwithdata{Akbar}.
\aparag If \paysMogol already existed, its ruler only is changed from the
\leader{Grand Moghol} to \leaderAkbar (until replaced by a further event).
\aparag The \paysMogol will try to invade \bonus{4} regions during the turn,
according to \ref{pII:Mughal Expansions}.
\aparag Even if the country has no region after the invasions, it still exists
(and can gain provinces with new events).
\aparag \granderegionBengale has from now on 2 \RES{Spices}, 2 \RES{Products
  of Orient} and 2 \RES{Cotton} available instead on 1 (representing the
change of commercial fluxes because of the Mughals).



\event{pIII:Fall Vijayanagar}{III-21}{Wars in India}{2}{PB}

\history{1565 / 1585-1594}

\phevnt
\aparag If it was still existing, minor country \paysVijayanagar is destroyed
(by internal fights).  Every \TP (not \COL) that are in the minor country
\paysVijayanagar at the time of its disappearance will face an attack by
Natives that are activated against every country this turn.
\aparag If \paysVijayanagar had already been destroyed, every \TP/\COL in
\continent{India} loses 1 level due to internal strife in India.
\aparag \granderegionKarnatika has from now on 2 \RES{Spices} and 2
\RES{Products of Orient} available instead on 1 (representing the change of
commercial fluxes from the north to the south because of the Mughals and the
destruction of the Indian Empire).
\aparag If the \paysMogol exist, they invade one province, the next in the
list according to \ref{pII:Mughal Expansions}.



\event{pIII:China Colonial Attitude}{III-22 (1)}{\paysChine colonial
  attitude}{1}{PB}

\history[Closure of China was the historical choice.]{1557}

\condition{}
\aparag If \paysChine has no \TP, apply \xnameref{pIII:CCA:Closure China}.
\aparag If \paysChine has any \TP left, roll 1d10 added to the number of \TP
it has. If the result if 6 or higher, commercial exclusivity policy in
\paysChine triggers the event \xnameref{pIII:CCA:Closure China}. If the result
is 5 or less, apply \xnameref{pIII:CCA:Commercial Dynamism China}.


\subevent[pIII:CCA:Closure China]{Closure of \paysChine}

\phevnt
\aparag Any country having a \TP in \paysChine may sign immediately a Treaty
with \paysChine, and so gains \dipAT. If accepted, only one \TP of the country
is kept in \paysChine; \TP in excesses are destroyed. If refused, \paysChine
declares an Overseas War against the power.
\aparag From now on, \dipAT allows each country to keep only one \TP in
\paysChine (and not one per region). The remaining \TP can be upgraded, and it
causes no reaction by \paysChine.
\aparag The basic forces and reinforcements of \paysChine are now its mainland
army only (no overseas garrisons of fleets).

\effetlong
\aparag From now on, no new \TP counter can be placed in any area belonging to
\paysChine by means of administrative actions.
\aparag No regular diplomacy is permitted on \paysChine.  The Activation level
of \paysChine becomes 11 (except for areas conquered that are not mainland
\paysChine, where the Activation is 6).

\aparag The only way to have a new \TP in \paysChine is to take control of the
\TP of another country (then the Treaty status is given to the new controller
of the \TP and lost by the previous one) or to force a Treaty on \paysChine by
means of a war against it.

\aparag From now on, the \terme{Manila Galleon} is
available. See~\ref{chSpecific:Manila Galleon}.


\subevent[pIII:CCA:Commercial Dynamism China]{Commercial dynamism of
  \paysChine}

\phevnt
\aparag \paysChine gains a \TP with level 6 in every coastal city of its
territories.  An automatic concurrence with any existing establishment is made
until only one \TP survives in each province. Its fleet in \stz{Chine} rises
to level 5 (and automatic concurrence might also be necessary).
\aparag Japanese \TP in \granderegionCorea and \granderegionFormose are
destroyed (by Chinese invasions).

\effetlong
\aparag \paysChine has a \FTI of 2 (raised to 3 from period V on) and a \DTI
of 3 and uses both \FTI and \DTI for concurrence in its own provinces. Form
now on, consider \stz{Chine} as its \CTZ.
\aparag \TP of \paysChine exploit the resources in their region and those are
counted as normal exploitation for monopolies and evolution of prices.
\aparag European countries having monopoly in \stz{Chine} may declare a
commercial embargo against \paysChine.  No \TP (not \COL) may exploit anything
in \paysChine as long as the embargo continues (both Chinese and European
\TP); so they are not counted in for monopolies and evolution of
prices. Moreover, no commercial fleet in \stz{Chine} gives any income. This
embargo gives an oversea \CB to every European country having a \TP in
\paysChine.
\aparag Each turn, all Chinese \TP in continental \paysChine gain one level
(with a maximum of 6), overseas \TP one level (with a maximum of 3) and
\paysChine gains one \TradeFLEET level in \stz{Chine} (with a maximum level
of 6). Destroyed \TP do not come back but the commercial fleet keeps coming
back even if destroyed.
\aparag Basic reinforcements are increased to one \ARMY\faceplus in mainland,
and 2 \LD, 2 \ND for the garrisons.



\event{pIII:Sultanate of Aceh}{III-22 (2)}{Sultanate of Aceh}{1}{PB/Jym [BLP]}

\history{1565}

\phevnt
\aparag Create the Sultanate of \paysaceh. Place its \TP\facemoins with 3
levels in
% (Jym) putting Centre first as is control the strait for the CC GO.
\granderegionSumatra (in the first empty province: Centre, North then South;
if none, place it in the Northern one and make automatic concurrence).
\bparag It proposes a \dipAT to \TUR that has the choice to accept it or not
immediately.
\bparag Forces are deployed as per the Annex.

\aparag[Malahayati] [BLP] \paysaceh receives the admiral
\leaderwithdata{Malahayati} for 9 turns.

\effetlong
% (JCD) This is not a reminder
\aparag Before 1700 \paysaceh has a \TP action every turn (strong investment)
to increase its \TP up to the original level 3, if ever its level is less (or
was destroyed).

\aparag The \TP of \paysaceh may never be annexed at peace.

\aparag[Malahayati] [BLP] As long as \leaderMalahayati is alive, increase the
basic forces of \paysaceh by \FLEET\faceplus.
\bparag As long as there is at least \FLEET\facemoins of \paysaceh in play
(including if it is at peace), the Malacca fortified strait is closed to every
country without a \dipAT on \paysaceh.

\aparag No other establishment (\COL or \TP) may be created in the province if
the \TP of \paysaceh is here.
\bparag Existing establishments, including those that would be created while
the \TP of \paysaceh is temporarily destroyed, stay without harm.


\event{pIII:Japanese Expedition}{III-23}{Japanese Expedition in
  \granderegionCorea}{1}{PB}

\history[Both invasions failed, historically.]{1592/1597}

\phevnt
\aparag Place a Japanese \TP in a province of \granderegionCorea,
\provinceSeoul if possible, \provincePyongyang if \provinceSeoul is occupied;
if both are occupied, this event is marked off but ignored.
\aparag The \TP has 3 levels and exploits all resources of \granderegionCorea
(other countries will have to take them by regular concurrence).
\aparag A Japanese colonial force of 1 \ARMY\faceplus defends the \TP; it may
gain \ARMY\facemoins in reinforcement each turn if needed.  This army does not
activate the Natives and an attack in this region may be aimed at the Japanese
only and so does not activate the Natives of \granderegionCorea. As soon as
the \TP is no more Japanese or destroyed, normal activation rules for Natives
apply and the colonial force is removed.



\subsection{Some Alternative History Events}



\event{pIII:Union Russia Poland}{III-A}{Union between \paysmajeurPologne and
  \paysmajeurRussie}{1}{PB}

\history{Alternative history}

\phevnt
\aparag The Polish Monarch dies and the Heir of Russia is elected in
Poland. \POL has now the Monarch \monarque{Dimitri}.  Its values and its reign
length are random, as if an heir from \RUS.
\aparag The Russian dynasty remains on the Polish throne until a Dynastic
crisis occurs in Poland or an event (or some elected specific general) changes
the Dynasty; this terminates the event. From now on, \RUS has Dynastic Claims
on \POL.

\activation{}
\aparag When the current Tsar of \RUS dies, \monarque{Dimitri} becomes the
Monarch of \RUS for its remaining reign length.
\aparag He can choose to abandon the Polish crown; that costs {\bf 1} \STAB to
\RUS, a new dynasty is elected in \POL (as if after a Dynastic Crisis, or a
general-monarch may be elected if one is available), and the event is ended.
\aparag It can choose to keep both crowns and \xnameref{pIII:URP:Effect of the
  Union} is now applied.

\effetlong
\aparag At each time there is a new Tsar in \RUS, beginning with
\monarque{Dimitri}, \POL can accept the Union or try to break it.
\bparag If the Union is accepted, the new Tsar becomes (or remains) the ruler
in \POL and \RUS gains 20 \PV each time.
\bparag If it is refused, a new Monarch is rolled for \POL, as if after a
Dynastic Crisis, or a general-monarch may be elected if one is available.  A
War for Dynastic Union might happen, see underneath.
\bparag Any other event calling for a change of Polish Monarch is impossible
when the Union holds; do not mark off this event and roll anew.


\subevent[pIII:URP:Effect of the Union]{Effect of the Union}

\effetlong
\aparag \RUS and \POL shares the same Monarch; \RUS has the control on the
Monarch (what he is doing, its values, and so on).
\aparag \POL has a mandatory offensive alliance with \RUS in which it is
complied to answer any call.
\aparag \POL may not declare war without a \CB or the agreement of \RUS. If it
has a \CB against \RUS, it can declare war to it and lose \STAB due to
breaking of alliance (but this one is renewed afterwards).
\aparag \RUS has no specific obligation regarding the alliance, and does not
lose \STAB if it doesn't answer the call. It can declare war to \POL but that
breaks the union and this war is now as described in \xnameref{pIII:URP:War
  Dynastic Union}.  Determine a new Polish Monarch.
\aparag \POL does not change of religious attitude because of the Union.


\subevent[pIII:URP:War Dynastic Union]{War for Dynastic Union}

\phdipl
\aparag If \POL has refused a continuation of the Union, \RUS has a free \CB
against \POL to be used immediately, and will lose {\bf 1} \STAB if it refuses
the \CB. In that case, \RUS renounces also to its Dynastic Claims on \POL.
\aparag If a war is declared, \POL is in Civil War against \RUS (see
\ref{chDiplo:Religious Civil War}). \RUS is permitted full intervention in
this war.
\aparag Roll for 2 \REVOLT in \POL when such a war erupts.

\phpaix
\aparag If \POL wins the war or signs a white peace, the Union and the
Dynastic Claims of \RUS are forfeited.
\aparag If \RUS wins the war with a peace of level 2 or more, the Monarch of
\RUS becomes ruler of \POL also as an victory condition (instead of 1
province).
\bparag The previous Monarch of \POL is executed; now \POL uses the values of
the Monarch of \RUS and the Union (see above) is renewed.



\event{pIII:Religious War Sweden}{III-B}{Religious War in Sweden}{1}{PB}

\history{Alternative history}

\condition{}
\aparag \SUE proposes an immediate white peace to every countries is at war
against.  Minor countries sign it, and Major Countries have the choice to sign
such a white peace or to sign an Armistice. If an Armistice is decided,
military occupation remains in provinces where the city is controlled (other
are evacuated), no combat is possible between the enemy sides, and Swedish
provinces that are occupied by enemies are out for the Religious War (see
\ref{chDiplo:Religious Civil War}). The Armistice will last until the end of
the Religious War and causes no loss of \STAB at the end of each turn.

\tour{Turn 1}

\phevnt
\aparag Roll for 4 \REVOLT in \SUE. Those \REVOLT has to be all in Swedish
provinces and in different provinces. The force of the \REVOLT is random but
they all control the city.  This forms the side of Rebels. They are opposed to
Loyalists.
% (JCD) We do not want loyalists keeping to Finland only and resisting forever
% only from there. Possibly do something about that.
\aparag The player of \SUE chooses its side:
\bparag If his initial choice was Catholic, he must play the Loyalists;
\bparag If \SUE is Catholic because of Union with \POL or because of Forced
Conversion, the player can choose Loyalists or Rebels.
\bparag If the player chooses to play Rebels, a new Monarch is rolled for on
the last column for values, with a random reign length (ignore Dynastic
Crisis). The characteristics of the previous Monarch has to be written down
(in case of victory of Loyalists) and this Monarch can be used as a general by
Loyalists.
\aparag A test is made for each military unit (per counter deployed), each
leader and each \COL or \TP with 1d10:
\begin{modlist}
\item[1--5] controlled by Loyalists;
\item[6--10] controlled by Rebels.
\end{modlist}
\aparag The side not played by \SUE is controlled by:
\bparag \POL if this is the Loyalists and \POL is Catholic;
\bparag \HAB if this is the Loyalists and \POL is not Catholic (Protestant or
Orthodox);
\bparag \ENG if this is the Rebels and \ENG is Protestant;
\bparag \MAJHOL if this is the Rebels and \ENG is Catholic.
\aparag During the Religious War, \SUE may not declare any war, nor make
diplomacy on minors (except in reaction). Events calling for an intervention
of \SUE are played as if \SUE makes an immediate Armistice or White Peace.
\aparag Foreign countries can be involved in this war only by foreign
intervention, excepted for what is listed below.

\phdipl
\aparag If \POLpol is Catholic, it has a \CB against the Rebels to join war
alongside Loyalists. \POLmin always uses this \CB.
\aparag If \DANdan is Protestant, it has a \CB against the Loyalists to join
war alongside Rebels. \DANMin uses this \CB only if \POL uses one.

\tour{as long as the war continues}

\phadm
\aparag The side played by \SUE uses the normal rules for Major Powers. It
controls the province where its owns the city and, if playing the Rebels,
disregards any \REVOLT (they don't affect its income because they are allied
to it).
% (JCD) Adapted to new accounting
\bparag Its initial treasury is at most two thirds of the treasury at the end
of the event phase. The loss is of at least 50\ducats.
% \begin{oldcompta}
%   \bparag The initial treasury is 2/3 of the treasury at the end of the event
%   phase.
% \end{oldcompta}
\aparag The other side has a basic maintenance equal to that of \SUE in the
current period and receives reinforcements as a minor country.  It uses the
fully controlled provinces (minus \REVOLT for the side of Loyalists) as their
basic income (for the modifier).
\aparag Each side has only a minimum of one general (and has any general
coming from the initial test).

\phmil
\aparag If \POL is at war, it can not have more than one stack in National
provinces of \paysmajeurSuede and provinces of \regionNorvege.

\phinter
\aparag The \REVOLT extend as usual.

\phpaix
\aparag Only unconditional surrender is permitted to Loyalists and Rebels. If
there are no \REVOLT left and no cities owned by Rebels, the Rebels surrender
(whether played by \SUE or as a minor).  If there are no national provinces of
Sweden not in \REVOLT or controlled by the Rebels, a minor Loyalists surrender
automatically.
\bparag If the Loyalists win, \SUE remains Catholic and has its Monarch ruling
before the event.
\bparag If the Rebels win, \SUE becomes \PROTTOL (with a new ruler if they
were not played by \SUE).
\aparag[Consequences for Poland]
\bparag If \POL was at war and the Loyalists win, \POL gains 40 \PV.
\bparag If \POL was at war and the Rebels win, the war continues as a normal
war between \POL and \SUE (a peace can be signed now at the same turn).
\aparag[Consequences for \paysDanemark]
\bparag If \DANdan was at war and the Rebels win, a province of \SUE is given
to \DANdan (choice of \SUE, if possible a province that was once owned by
\DANdan).
\bparag If \DANdan was at war and the Loyalists win, the war continues as a
normal war between \DANdan and \POL/\SUE.  A peace can be proposed at the same
turn.
\aparag The player of \SUE on the losing side loses 20 \PV.



\event{pIII:Religious War Poland}{III-C}{Religious War in Poland}{1}{PBNew}

\history{Alternative history}

\activation{Replaces \ref{pIII:Union Poland Sweden} if \POL is protestant.
  The Swedish heir is elected as king of Poland, but remains protestant. He
  must fight a religious war in its new kingdom. Will be a variation on
  \ref{pIV:Polish Civil War}.}

\clearpage

%% *-* latex-mode *-*



\event{pIII:FWR Detailed}{III-D}{Religious Wars in France}{5}{PBNew}

\history{1562-1598}{The wars are fragmented in 5 parts. \\
  (1) First, Second and Third wars (1562-1570) with many truces broken by one
  side or the other. \\
  (2) Fourth and Fifth wars (1570-1575), where the Massacre
  of the Saint-Barth\'el\'emy heightens the intensity of the war.\\
  (3) Sixth and Seventh wars (1575-1580) where the Catholic League and the
  Duke of Guise seem almighty, and a background announced Dynastic Crisis. \\
  (4) Eighth war (1585-1598) that is the war of Succession for the French
  Crown. \\
  (5) Alternative history: more troubles if France is not Conciliant (mainly
  with foreign support).  }

\dure{until the end of \ref{pIII:FWR Last Stand} or \ref{pIII:FWR Succession}
  (as specified in these events) or at the end of period III.}

\activation{This event is composed by many sections describing first the
  general conditions under which the wars are fought, then specifics of the
  evolution of the Wars: from a set of strictly Religious Wars that go harder
  and harder to a War of Succession. The passage from one event to another is
  described hereafter.}
\aparag This event can not happen before turn 11 (1540). If the turn if 10 or
before, re-roll and do not mark off.
\aparag Only one \ref{pIII:FWR} can be rolled and marked off each turn. If a
second one is obtained, do not mark off and re-roll.
\aparag After the end of this event, \ref{pIII:FWR} triggers an event \RD, and
the box is marked.
\bparag If \FRA is \CATHCO, its Monarch will have a malus of \bonus{+2} to his
Survival Test next turn.
\bparag If \FRA is \CATHCR or Protestant, the \REVOLT is rolled on the table
of \FRA.
\aparag From the first to the end of the last event, \FRA is in religious
Civil War and is limited in many aspects.

\phevnt
\aparag[The states within the State] Two minor countries, \paysHuguenots and
\paysLigue are created for this event. No diplomacy is authorised on them;
they have the same technology and military features as \FRA.
% \bparag When at peace with \FRA, all their provinces and counters (armies,
% leaders,...) belong to \FRA and are used as such.
% \bparag When at war against \FRA, \FRA still earns the income of their
% provinces (except where there are \REVOLT !)  and those minor countries
% never pillage the provinces in \FRA.
\aparag[Les Huguenots]
\bparag \hug has the following provinces (if in \FRA): \provinceCaux,
\provinceTouraine, \provincePoitou, \provinceQuercy, \provinceGuyenne,
\provinceLanguedoc, \provinceBearn, \provinceDauphine, \provinceCevennes
(those provinces have a white shield border).
\bparag \hug is protestant.
\bparag Its main controller is \ENG (if Protestant) or \HOL (if it exists) or
\SUE (if Protestant), else \MAJHOL. This major power will be noted \HUG (and
the minor \hug); it may change at each turn (depending on the changes of
religion).
\aparag[La Ligue]
\bparag \lig has the following provinces (if in \FRA):
% \provinceArmor, \provinceFinistere, \provinceMorbihan,
\provinceNormandie, \provinceMaine, \province{Ile-de-France},
\provinceOrleanais, \provincePicardie, \provinceChampagne, \provinceBerry,
\provinceBourgogne, \provinceLyonnais, \provinceProvence (those provinces have
a yellow shield border).
\bparag \lig is \CATHCR.
\bparag Its main controller is the Sole Defender of the Catholic Faith (if it
is not \FRA), \SPA (if \CATHCR), \ENG (if Catholic), or \SPA (\CATHCO) in the
last possibility.  This major power will be noted \LIG (and the minor \lig);
it may change at each turn (depending on the changes of religion).
\aparag The Loyalists are \FRA and its allies. The Rebels are the revolted
minor country (\lig or \hug) and its allies. \REB is the Major Power that
controls the Rebels (\LIG or \HUG).
\aparag The Catholic side is the one of \lig else of Catholic \FRA.
\aparag The Protestant side is the one of \hug else of Protestant \FRA.

\aparag[Military units]
\bparag Basic forces of \FRA drops to \ARMY \facemoins (or \ARMY \facemoins,
\LD if in period \period{II}). Counters limit for \FRA drops to 3 \ARMY (and 2
\ARMY for each minor).
\bparag Basic forces of the new minors is \ARMY \facemoins, \LD (or \ARMY
\faceplus if in period \period{II}) if it has not the same religion than \FRA
and \ARMY \facemoins (\ARMY \facemoins, \LD if in period \period{II}) if it
has the same religion than \FRA.
\bparag If the minor is at war against \FRA, then it is controlled by its main
controller (either \HUG or \LIG). Else, if \FRA is at war (even civil war
against the other minor) then \FRA may use its troops as if they were french
troops.
\bparag If \FRA is at peace, the main controller of each minor may declare a
limited intervention (following usual rules) of this minor in any existing war
during the diplomatic phase. If the minor has the same religion than \FRA,
this can only be done if \FRA agrees to.  The main controller plays the troops
of the minor and pay for its campaign or reinforcements.
\bparag If \FRA is at peace, and the main controller doesn't want to use the
troops of the minor (or can't), then \FRA may use them as if they were its own
troops.
\bparag If \FRA is at peace, it may build troops of any of the two minors at
regular cost. This counts toward purchase limit of the turn.
\bparag If the minor is not used by somebody else, \FRA has to pay the
maintenance of any troops in addition to the basic maintenance of the minor.
\bparag If \FRA is at peace and the minor has less than its basic forces and
is not used in another war by its main controller, then \FRA has to build
troops of the minor. It is not complied to buy more than the turn limit or to
go bankruptcy, but it must build troops for the minor prior to any other
administrative action. If both minors lack troops, \FRA must start building
troops of the minor having a different religion than its own.
\bparag If \FRA is at peace with the minor, it cannot voluntary dismiss
(i.e. by not paying upkeep) troops of the minor below what was left at the end
of the last civil war. Yet, if the loss is due to any other reason (such as
being used in another war or by its main controller in a foreign
intervention), \FRA is not complied to buy new troops up to this value (just
up to the basic maintenance of the minor).

\aparag[Incomes]
\bparag If \FRA is at war against the minor, then it get no land income from
the provinces of the minor (this also may change the industrial and commercial
incomes of \FRA). Manufactures in these provinces do not provide income
either.
\bparag If \FRA is at peace, the provinces of the minor having the same
religion as \FRA are treated exactly like french provinces: they provide full
land income, manufactures and gold mines provide also full income.
\bparag If \FRA is at peace, the provinces of the minor having different
religion than \FRA only provide half their regular income: land income is
halved (this also change industrial and commercial income), manufactures
provide only half their facial value and half their percentage, gold mines
provide only 10\ducats, \ldots
\bparag If \FRA is in civil war (but not against the minor), provinces of the
minor only provide half their regular income (as above).
\bparag The (land) income not perceived by \FRA does not increase its foreign
trade.
\bparag If \FRA is at peace, it only gets 75\% of its colonial income if its
catholic.

\aparag[Military control]
\bparag If \FRA is not at war against the minor, then both may use provinces
belonging to both of them as supply sources.
\bparag If \FRA is at war against the minor, then supply may go through any
province not containing an unbesieged hostile troop or \REVOLT .

\effetlong
\aparag[Fragile Health of the Valois]
\bparag From the beginning of the event, and as long as the French Monarch is
a Valois, it adds \bonus{+3} to its Survival Test.
\aparag[Lack of Heirs]
\bparag An additional test of Dynastic Crisis is made at the beginning of each
turn (at the Monarch Survival Phase). A malus of \bonus{-1} is applied for
each \ref{pIII:FWR} rolled since the beginning of the game.
\bparag If a Dynastic Crisis occurs (because of the previous test or of a
normal test after the death of the Monarch), apply directly \ref{pIII:FWR
  Succession}.  If a Dynastic Crisis occurs without the death of the Monarch,
the rules of the event use the historical name \anchormonarque{Henri III} to
designate the current Monarch of \FRA.

\aparag[Mandatory Change of Religious Attitude] \FRA can be complied to change
its Religious choice during the war because of a Coup (\ref{pIII:FWR
  Succession}), or an unconditional surrender caused by foreign powers. The
following points occur (but not if the change is voluntary when designating an
Heir of the Valois).
\bparag \FRA goes down to {\bf -3} in \STAB, loses \bonus{-1} in \FTI, and
loses 30 \PV.
\bparag The controller of the side imposing its Heir by a Coup, or the
countries that force a unconditional surrender gain 30 \PV each time a
mandatory change is made.

\begin{digressions}[General troubles in France each time an event happens]


  \digression[pIII:FWR:Politic Crisis]{Politic crisis}

  \phevnt
  \aparag \FRA loses {\bf 2} \STAB.
  \aparag The diplomacy of \FRA is lowered by \bonus{-2} (minimum of 3).
  \aparag \FRA and its adversaries make a mandatory white peace (exception:
  see \ref{pIII:FWR Last Stand}).
  \aparag \FRA is involved in religious civil war when at war against
  Rebels. No-one can declare a war to \FRA at those times, but \MAJ may do
  \terme{Foreign Intervention} in the war each time the war resumes (new event
  or broken Truce) excepted if explicitly forbidden.


  \digression[pIII:FWR:Economic Crisis]{Economic crisis}

  \phevnt
  \aparag On the first event, the Royal Treasury of \FRA is diminished by half
  and loses at least 50\ducats.  On subsequent events, the Royal Treasury of
  \FRA is diminished by 50\ducats if greater than 50\ducats, goes to 0\ducats
  if greater than 20\ducats, and be diminished by 20\ducats if less than
  20\ducats.

  \bparag If \FRA makes a bankruptcy while at war against the rebels, they
  will receive \ARMY\facemoins extra reinforcement (\LD each if there are two
  rebels).
  % On the first event, the Royal Treasury of \FRA is halved and loses at
  % least 50\ducats if \FRA has not enough in its royal treasury, then it lose
  % everything, lose and 1 \STAB and rebels will receive \ARMY \facemoins
  % extra reinforcement (or \DT each if there are two rebels). On the
  % following ones, the Royal treasury of \FRA is halved with a maximal lose
  % of 50\ducats.
  % \aparag It receives no commercial income, no industrial income, and
  % no colonial income during the wars; however if a war resumes at the
  % end of a Truce, \FRA receives half of those industrial, commercial
  % and colonial income during the first turn after the Truce, as the
  % Truce is broken after some months (or years) of relative peace.
  \aparag \FRA (and also \hug and \lig) makes a mandatory trade refusal
  against all other countries. This does not provide CB or entail loss of
  stability and only last while \FRA is in civil war.
  % \aparag Industrial income is halved (after a reduction due to \hug
  % or \lig being at war with \FRA).
  \bparag \FRA only gets 75\% of its colonial income if protestant, 50\% if
  \CATHCO and 25\% if \CATHCR.
  \aparag \FRA can make no economic action (\COL, \TP, \TFI, \CONC) during the
  wars (even if the Truce was broken this turn), except as a reaction to
  concurrence.
  \aparag A \PIRATE\faceplus is placed in \CTZ of \FRA; at most one \PIRATE
  can be here due to this event.
  \aparag \FRA has to pay separate campaigns for any troop going in the \ROTW
  or whose movement end on the \ROTW map (so, it can bring back troops from
  the \ROTW without penalty).


  \digression[pIII:FWR:Uprisings]{Uprisings in France}

  \phevnt
  \aparag If \FRA is \CATHCR or \CATHCO, the Rebels are \hug.  If \FRA is
  Protestant, the Rebels are \lig. \FRA is at war against the Rebels (it is
  not a declaration of war by the Rebels).
  \aparag If \FRA is \CATHCR or \CATHCO, roll 1d10 and place \REVOLT
  \facemoins in the following provinces, excepted in the first province where
  the \REVOLT is \faceplus:
  \bparag result odd: \provincePoitou, \provinceQuercy, \provinceGuyenne,
  \provinceLanguedoc, \provinceAuvergne;
  \bparag result even: \provinceCaux, \provincePoitou, \provinceGuyenne,
  \provinceTouraine, \provinceVendee.
  \aparag If \FRA is \CATHCR, add a \REVOLT \faceplus in \provinceDauphine and
  a \REVOLT \facemoins in \provinceArmor.
  \aparag If the die-roll was 9 or 10 (between 7 and 10 if \FRA is \CATHCR),
  place a \REVOLT \facemoins on a randomly chosen colony (or \TP if no colony
  is available).
  \aparag If \FRA is Protestant, place a \REVOLT \faceplus in
  \province{Ile-de-France}, a \REVOLT \facemoins in \provinceLyonnais and roll
  1d10 for the other ones (the \REVOLT is \faceplus in the first province of
  the list and \facemoins in the others):
  % \bparag result even: \provinceQuercy, \provincePoitou,
  % \provinceDauphine, \provinceTouraine, \provinceCaux;
  \bparag result even: \provinceProvence, \provinceNormandie, \provinceMaine,
  \provinceTroyes, \provinceVendee;
  \bparag result odd: \provinceOrleanais, \provinceChampagne,
  \provinceTouraine, \provinceCaux, \provincePicardie.
  \aparag If the die-roll was 10, place a \REVOLT \facemoins on a randomly
  chosen colony (or \TP if no colony is available).
  \aparag The Rebels receive 2 minor unnamed generals to be placed on \REVOLT
  (they can only lead \REVOLT , not forces of the Rebels, and are eliminated
  when the \REVOLT is finally suppressed).
  \aparag The Rebels own its provinces %and any province of \FRA where
  % there is a \REVOLT .


  \digression[pIII:FWR:Military Troubles]{Military Troubles}

  \phevnt
  \aparag On the first event, only the basic forces of \FRA are kept (\ARMY
  \faceplus, \ARMY \facemoins, \DT), in veteran status.  If \FRA has less than
  this, it will receive less troops than stated. The rebels takes their forces
  first, then the non-rebelled minors and lastly \FRA.
  \aparag Roll 1d10:
  \bparag result even: \FRA keeps \ARMY \facemoins and \DT; the Rebels have
  \ARMY \facemoins; the minor of the same religion as \FRA has \ARMY
  \facemoins;
  \bparag result odd; \FRA keeps \DT, the Rebels have \ARMY \faceplus; the
  minor of the same religion as \FRA has \ARMY \facemoins.
  % \aparag The Rebels may already have some forces (remaining after a
  % favourable Truce) that are added to this forces. All those forces are
  % veterans.
  \aparag If the current turn is in period II, \FRA adds \ARMY \facemoins to
  its forces and the minor sharing its religion add \DT.
  \aparag If \FRA is Emperor of the \HRE, it can use the \ARMY of \HRE as a
  help in this war.
  \aparag Minor country \paysLorraine is activated and allied of the Catholic
  side. It gives 1 \DT, both sides can pass or stop in its provinces but the
  \REVOLT never extend in those.
  \aparag The forces of the Rebels are deployed in their provinces that are in
  \REVOLT .  The forces of \FRA are placed in any province of \FRA that does
  not belong to the Rebels.
  \aparag The naval forces of \FRA may defect as follows. Roll 1d10.
  \bparag result 1-8: \FRA keeps all the naval forces.
  \bparag result of 9: 1 \DN is given to the Major Power controlling the
  Rebels and the rest are Rebel forces.
  \bparag result of 0: 1 \DN is given to the first Protestant country of the
  list: \HOL, \ENG, \SUE, \POL, or to the Major Power controlling the rebels
  if there is none, and the rest are Rebel forces.
  \bparag Naval forces of the Rebels have to go in a port of Rebels.  When, at
  the end of a round, there is no port left to Rebels, the navy comes back in
  the ownership of \FRA.

  \phadm
  \aparag \FRA can build reinforcements as usual and deploys them in provinces
  not owned by the Rebels.
  \aparag The Rebels gain reinforcements in offensive mode on the minor table,
  with a bonus of \bonus{+2} and some other modifiers (see the various steps
  of the events). It gains only the \LD written in the table, not the F, CM or
  leaders.
  \bparag If \FRA is not \CATHCO, add \bonus{+1} to the roll.
  \bparag The Rebels receive 1\fortress if the result is even, or 2\fortress
  if the result is equal to 11 or higher.
  \bparag The reinforcements of the Rebels are deployed in provinces in
  \REVOLT , and the fortresses can only be deployed in provinces with \REVOLT
  \faceplus.
  \aparag[Leaders] After the building of forces, the loyalty of the leaders is
  tested.
  \bparag \leaderMontmorency is always loyal to \FRA.
  \bparag \lig receives \leader{Henri de Guise}.
  \bparag \hug receives \leaderColigny, \leaderConde and, beginning with
  \ref{pIII:FWR Barthelemy}, \leaderNavarre.
  \bparag Every other named leader is checked by rolling 1d10: used by the
  Catholic side if result 1-7; used by the Protestant side if the result is
  8-10.
  \bparag Each side should have at least two leaders. If one has less, it
  receives an unnamed general from those of \FRA.
  \bparag Neither the Loyalists nor the Rebels can use mercenary generals.
  \bparag This repartition is made once for all the following wars; but \FRA
  can use all its leaders (whether from \lig or \hug) during Truces.


  \digression[pIII:FWR:Military Operations]{Military operations during the
    wars}

  \phmil
  \aparag The Rebels control all cities of provinces with \REVOLT at start.
  It draws supply from all provinces of the rebel minor country and from
  cities it controls.
  \aparag \FRA controls all cities of provinces not in \REVOLT .  It draws
  supply from any such provinces.
  \aparag French Leaders of both side are only killed in battles if the
  die-roll was a natural 1. Else, if they would be killed (due to modifiers),
  they are Captured instead and are freed when a Truce happens.
  \aparag The Rebels and the minor countries that are involved in the war have
  a simple campaign each turn. Their controller may pay for a more important
  campaign (by spending the cost of the campaign minus 20\ducats).
  \aparag A city owned by the rebel minor country makes an immediate voluntary
  surrender if besieged by a land stack that is commanded by a named rebel
  general and that sets a siege with at least one \ARMY \faceplus.
  % \aparag If \FRA proposes a Truce favourable to the Rebels at the end
  % of any military round, it is signed immediately and the military
  % operations for this war ceases for the rest of the turn.  This is
  % not possible during events \ref{pIII:FWR Succession} and
  % \ref{pIII:FWR Last Stand}.
  % (Jym) (4) and (5) (no Truce). TODO deal with (1)


  \digression[pIII:FWR:Truces]{Truces during the Wars of Religion}

  \phpaix
  % \aparag Each war from \ref{pIII:FWR Beginning} to
  % \ref{pIII:FWR League} last only one turn, then ends with a
  % Truce favouring one side of the other. The war can resume the next
  % turn or a following one. \ref{pIII:FWR Succession} and
  % \ref{pIII:FWR Last Stand} are full Civil War were no Truces are
  % admitted.
  % \aparag[Favoured side in the Truce]
  % \bparag Excepted if a Truce favourable to the Rebels was proposed by
  % \FRA, determine the side favoured by the Truce according to the
  % following calculus of their respective positions.
  % \bparag A side obtains 1 point per victory in land battle, 2 per
  % major victory, 1 per siege made (but none for automatic surrenders),
  % and the Rebels gain 1 per \REVOLT remaining in France at the end of
  % turn (before any extension).
  % \bparag Whomever has the highest total obtains a favourable Truce;
  % The Rebels are favoured if it is a draw.
  \aparag At the end of any turn, \FRA may propose peace to the rebelled
  minor. This is treated as a regular peace with minor. This can not be done
  during \ref{pIII:FWR Succession} and \ref{pIII:FWR Last Stand} who have
  specific ending conditions.
  % \bparag The initial situation is the one at the beginning of the
  % military phase. Any \REVOLT crushed adds a +1 bonus to the french
  % die roll. Automatic surrenders of cities do not provide any peace
  % differential. (Jym) rather do the opposite => no differential for
  % \REVOLT , but differential for sieges, even automatic ? would be more
  % in line with the rest of the game...
  \bparag The initial situation is the one at the beginning of the military
  phase. \REVOLT do not count toward the peace differential, but provinces
  taken (including automatic surrender) count.
  \bparag Money may not be asked/given as a peace condition.
  \bparag A valid peace condition is the establishment or demolition of a
  safety place. If a safety place is granted, the minor may put a level 3
  fortress in an owned province. If possible it must be put in a province
  initially in \REVOLT \faceplus.
  \bparag The first peace condition must be a safety place (if possible).
  \bparag Any colonial establishment still having a \REVOLT when peace is
  signed immediately lose one level (and may thus be destroyed).
  \aparag If no truce is granted, \REVOLT do not extend as normal but \FRA
  loses stability for both the \REVOLT and the duration of war.
  \aparag Two white peaces count as a losing truce toward french objectives
  (but a single white peace has no effect).
  \aparag If \FRA removes all the \REVOLT counters and retakes all rebel
  fortresses, it may ask for an (automatic) unconditional surrender. It
  immediately gains 20 \VP and 2 stability. The next \ref{pIII:FWR} rolled
  will be played as \ref{pIII:FWR Succession} (even if \FRA is \CATHCO) after
  which the civil wars will permanently stop.
  \aparag If \FRA is for two consecutive turns of the same war at -3 in
  stability and does not manage to sign a peace, it must surrender
  unconditionally and suffer a mandatory change of religion.
  \aparag If \FRA sign a favourable peace, it gain 1 stability.

  \aparag[Effect of a Truce] All \REVOLT are suppressed in \FRA; the naval
  forces are back in the ownership of \FRA (except the \DN that might have
  been seized by foreigners).
  % \bparag If the Truce favours \FRA, \FRA annexes one province of the
  % rebel minor country; all fortresses of the Rebels are withdrawn; all
  % remaining Rebel land forces are given to \FRA; \FRA gains {\bf 1}
  % \STAB.
  % \bparag If the Truce favours the Rebels, they annexe one province
  % (from \FRA or the opposed minor country; if possible a province they
  % once owned); one of its provinces becomes a Place of Safety and
  % gains a level 3 fortress (if possible, a province having a \REVOLT
  % \faceplus before the Truce); the Rebels keep half of its land forces
  % (round up), the rest is dismantled and \FRA will have to build
  % forces anew to its basic forces before the next war; all fortresses
  % in the provinces of the rebel minor country remain.
  % \aparag During Truces, \FRA earn half of its colonial, industrial
  % and commercial income. This includes turns were a Truce is broken by
  % the following mechanism.

  \phdipl
  \aparag During Truces, \FRA is not limited in its diplomatic and
  administrative actions, and can also be involved in external wars (using its
  forces as well as those of \lig and \hug). This does not include turns where
  a Truce breaks down. Remember that both \lig and \hug may be used by their
  controllers.
  \aparag The Truce can be questioned at the beginning of any phase of
  Diplomacy:
  \bparag Roll 1d10 + %the difference between the position value for
  % Rebels at the end of last war and the value for \FRA + \STAB of \FRA
  the level of the peace (in favour of the rebel) \bonus{-1} per turn since
  the beginning of the Truce. If the result is 4 or below, the Rebels will
  break the Truce.
  \bparag Else, if \FRA did not have a favourable Truce and wants to break it,
  it is automatic.
  \bparag If the Truce is broken, apply \ref{pIII:FWR:Politic Crisis},
  \ref{pIII:FWR:Economic Crisis}, \ref{pIII:FWR:Uprisings},
  \ref{pIII:FWR:Military Troubles} at the end of the Diplomatic phase.
\end{digressions}



\event{pIII:FWR Beginning}{III-D (1)}{The first 3 Wars of Religion}{1}{PB}

\tour{Turn 1}

\phevnt
\aparag The Wars of Religion begin; apply the general conditions and the
lasting effects on the Valois as found in \numberref{pIII:FWR Detailed}.
\aparag Apply the full effects of \ref{pIII:FWR:Politic Crisis},
\ref{pIII:FWR:Economic Crisis}, \ref{pIII:FWR:Uprisings} and
\ref{pIII:FWR:Military Troubles}.
\aparag[Michel de l'Hospital]\label{pIII:FWR:Hospital} If \FRA is \CATHCR, it
can now decide to play the rest of the event as \CATHCO.  Its religion changes
immediately, using only the lasting effects of the \ref{pI:Reformation}; the
initial \REVOLT are played as \CATHCR though.
\aparag For each \REVOLT that should be placed, roll a die: the \REVOLT
actually happens only if the result if 6 or higher. Add 1 to the die roll if
\FRA is not \CATHCO (do not add if \FRA just changed its attitude due to
Michel de l'Hospital, but still use the \CATHCR line for placing \REVOLT ).

\phdipl
\aparag No Foreign intervention allowed on the first turn.
\aparag \REB can make a very limited intervention in the war, only with naval
forces (in order to install or break a blockade; no naval movement of Rebel
land forces), that costs no \STAB.
\aparag If \FRA is \CATHCR, \LIG can make a limited intervention as an ally of
\FRA.

\begin{digressions}[Specific conditions of the first event]


  \digression[pIII:FWR:War1 Military]{Military operations during the first
    event}

  \phmil
  \aparag Use the general rules of \ref{pIII:FWR:Military Operations}.
  \aparag If all the leaders of on side are captured, wounded or killed, this
  side signs a level 1 peace in favour of its enemy at the end of the round.

  \aparag At the beginning of each military round (except the first), a new
  \REVOLT is rolled for in France.
  \bparag This revolt is always rolled on the table for \FRA in period III,
  even if this is not the current period. Moreover, if \FRA is catholic,
  \textbf{subtract} its \STAB from the localisation die roll rather than
  adding it.
  \bparag If this \REVOLT is in the rebel minor country and has no \REVOLT nor
  Loyalist land force in it, place a new \REVOLT \facemoins which takes the
  city.
  \aparag A city in \FRA that had not a \REVOLT \faceplus at the beginning of
  the current war nor is a safety place, makes an immediate voluntary
  surrender if besieged by a land stack of \FRA (or its allies) that sets a
  siege with at least one \ARMY \faceplus and there is no more \REVOLT in the
  province (including if the \REVOLT was just crushed this round).


  \digression[pIII:FWR:Peace1]{Peace during the first event}

  \phpaix
  % \aparag A Truce is necessarily signed, and the favoured side is determined
  % as explained in \ref{pIII:FWR:Truces}. All the effects
  % explained here are applied (so the \REVOLT are withdrawn before extension
  % or Stability loss).
  \aparag No peace of level higher than 2 can be signed during this first war,
  especially no unconditional surrender can happen.
  \aparag If \LIG was in limited intervention, allied to a \CATHCR\ \FRA, it
  wins 15 \PV if the Truce is in favour of \FRA and \LIG had forces in at
  least one battle or one siege (including voluntary surrender) against the
  Rebels.
  \aparag \FRA may choose to commit \xnameref{pIII:FWR Barthelemy} on any
  later turn. Consider that \numberref{pIII:FWR Barthelemy} is one of the four
  events rolled this turn and apply all the relevant effects.

  \tour{Turn 2 and following: Extension of the War}


  \digression[pIII:FWR:Continuation1 War]{Extension of the war}

  \phevnt
  \aparag[\REVOLT extension]
  \bparag For each two \REVOLT still existing in France (including colonial
  empire), roll die on the \REVOLT table for \FRA. If the province is neither
  occupied by loyalist troops or part of the non-rebelling minor, place a
  \REVOLT \facemoins which takes the city there.
  \bparag Roll a die. Add 2 if \FRA is \CATHCR, subtract 2 if \FRA is
  protestant. On a roll of 6 or more, place a \REVOLT \facemoins in a randomly
  chosen french colony (if there is no french colony or all have 2 \REVOLT
  \faceplus, in a randomly chosen \TP).

  \phadm
  \aparag Rebel will receive reinforcement as on turn 1.

  \phdipl
  \aparag Foreign interventions are now permitted.
  \aparag \REB can make a limited intervention as an ally on the Rebels (and
  it is not limited to naval forces only from now on).
  \aparag \HOL can make a limited intervention as an ally of a rebel \hug.
  \aparag \SPA can make a limited intervention as an ally of a rebel \lig.

  \phmil
  \aparag[Intervention of \paysPalatinat]\label{pIII:FWR:Palatinate} If
  inactive, \paysPalatinat makes a limited intervention as an ally of the
  Rebels (it is a mercenary army). It is played by \REB. The intervention
  force is \leader{Jean-Casimir}, one \ARMY \faceplus and 1 \DT.  If the
  \nameref{pII:Schmalkaldic League} or the \nameref{pIII:League Nassau}
  exists, and the Rebels are \hug, this intervention is made with 2 \ARMY
  \faceplus.  \leader{Jean-Casimir} is a general of \paysPalatinat (and serves
  this country if it is at war elsewhere) that will stay as long as
  \ref{pIII:FWR Barthelemy} is not finished.  After that, \paysPalatinat is
  without leader (for intervention) or has normal generals (for other wars).

  \tour{Turn 2 and following: Breaking of Truces}


  \digression[pIII:FWR:Continuation1 Truce]{Breaking of Truces}

  \phevnt
  \aparag If a Dynastic Crisis occurs, \ref{pIII:FWR Succession} will happen
  at this turn. If \numberref{pIII:FWR} is rolled for at this turn, mark off
  the box and consider that it triggers \numberref{pIII:FWR Succession}.
  \aparag As long as a new \numberref{pIII:FWR} is not rolled for, the Truce
  can be broken as explained in \ref{pIII:FWR:Truces}. A war begins anew, as
  explained there.
  \aparag If a new \shortref{pIII:FWR} is rolled for in the Political Event
  Phase, the next phase of \shortref{pIII:FWR} begins (\numberref{pIII:FWR
    Barthelemy}, \numberref{pIII:FWR League} or \numberref{pIII:FWR
    Succession}). Go to this event.
  \aparag If none of this happens, \FRA is in civil peace, and has its
  activity limited by \ref{pIII:FWR:Truces} only.

  \phadm
  \aparag If the Truce has been broken, apply the full effects of
  \ref{pIII:FWR:Politic Crisis}, \ref{pIII:FWR:Economic Crisis},
  \ref{pIII:FWR:Uprisings} and \ref{pIII:FWR:Military Troubles}, and the
  following points.

  \phdipl
  \aparag Foreign interventions are now permitted.
  \aparag \REB can make a limited intervention as an ally on the Rebels (and
  it is not limited to naval forces only from now on).
  \aparag \HOL can make a limited intervention as an ally of a rebel \hug.
  \aparag \SPA can make a limited intervention as an ally of a rebel \lig.

  \phmil
  \aparag The war is prosecuted according to \ref{pIII:FWR:Military
    Operations}, and \ref{pIII:FWR:War1 Military}.
  \aparag[Intervention of \paysPalatinat] If inactive, \paysPalatinat makes a
  limited intervention as an ally of the Rebels (it is a mercenary army). It
  is played by \REB. The intervention force is \leader{Jean-Casimir}, one
  \ARMY \faceplus and 1 \DT.  If the \nameref{pII:Schmalkaldic League} or the
  \nameref{pIII:League Nassau} exists, and the Rebels are \hug, this
  intervention is made with 2 \ARMY \faceplus.  \leader{Jean-Casimir} is a
  general of \paysPalatinat (and serves this country if it is at war
  elsewhere) that will stay as long as the \ref{pIII:FWR Barthelemy} is not
  finished. Beginning with next event, \paysPalatinat is back to normal (no
  leader for intervention or normal generals for other wars).

  \phpaix
  % \aparag A Truce is necessarily signed, and the favoured side is determined
  % as explained in \ref{pIII:FWR:Truces}. All the effects
  % explained here are applied (so the \REVOLT are withdrawn before extension
  % or Stability loss).
  \aparag If a Major Power makes a limited intervention and the side it helps
  obtains a Truce in its favour, the Major Power gains 10 \PV if it had land
  forces in at least one battle or one siege (including voluntary surrender)
  against the enemy side.

\end{digressions}



\event{pIII:FWR Barthelemy}{III-D (2)}{The Saint-Barthelemy}{1}{PB}

\tour{Turn 1}

\phevnt
\aparag A new war breaks out. Apply the full effects of \ref{pIII:FWR:Politic
  Crisis}, \ref{pIII:FWR:Economic Crisis}, \ref{pIII:FWR:Uprisings} and
\ref{pIII:FWR:Military Troubles}.
\aparag \leaderNavarre is available as a \paysHuguenots general.

\phdipl
\aparag No Foreign intervention is allowed.
\aparag \REB can make a somewhat limited intervention in the war, only with
naval forces (in order to make or break blockade; no naval movement of Rebel
land forces) or with land forces in coastal besieged provinces of the Rebels,
in order to stop the siege; afterwards it can withdraw or remain in this
province only.
\aparag The Rebels control all cities in the rebel minor country (and not only
those with a \REVOLT in there).
\aparag \FRA can then announce an attempt of
\xnameref{pIII:FWR:Saint-Barthelemy}, and resolves this odious deed. This is
of course mandatory if this event happen due to \FRA's choice during
\ref{pIII:FWR Beginning}.
\aparag If \FRA is \CATHCR, \LIG can make a limited intervention as an ally of
\FRA.

\begin{digressions}[Specific conditions of the second event]


  \digression[pIII:FWR:Saint-Barthelemy]{Massacre of the Saint-Barth\'el\'emy}

  \phdipl
  \aparag 1d10 is rolled for every rebel leader, excepted \leader{Henri de
    Guise} and \leaderNavarre. An even result means that the leader was killed
  in the Massacre.
  \aparag Each city in the rebel minor country is taken by \FRA by rolling
  1d10 higher than the level of the fortress; one die is rolled for each
  city. The cities taken this way are military controlled by \FRA but still
  owned by the rebel minor country.
  \aparag The Rebels will have a malus of \bonus{-1} to receive its
  reinforcements at this turn.
  \aparag The Rebels can no longer make a limited intervention in
  \ref{pIII:Dutch Revolt}.
  \aparag \FRA loses {\bf 1} \STAB.
  \aparag The Survival roll of the French Monarch is modified by an additional
  \bonus{+1} until the end of the Wars of Religion.


  \digression[pIII:FWR:War2 Military]{Military operations after the
    Saint-Barth\'el\'emy}

  \phmil
  \aparag Use the general rules of \ref{pIII:FWR:Military Operations}.
  \aparag If all the leaders of on side are captured, wounded or killed, this
  side signs a level 1 peace in favour of its enemy at the end of the round.
  \aparag At the beginning of each military round (except the first), a new
  \REVOLT is rolled for in France. If this \REVOLT is in the rebel minor
  country and has no \REVOLT nor Loyalist land force in it, place a new
  \REVOLT \facemoins which takes the city.
  \aparag \FRA (and its allies) have a bonus of \bonus{+1} to suppress \REVOLT
  in France and perform automatic surrenders of rebel fortresses as in the
  previous war.
\end{digressions}

\phpaix
% \aparag A Truce is necessarily signed at the end of the turn, and the
% favoured side is determined as explained in \ref{pIII:FWR:Truces}. All the
% effects explained here are applied (so the \REVOLT are withdrawn before
% extension or Stability loss).
\aparag If \LIG was in intervention, allied to a \CATHCR\ \FRA, it wins 15 \PV
if the Truce is in favour of \FRA and \LIG had forces in at least one battle
or one siege (including voluntary surrender) against the Rebels.

\tour{Turn 2 and following}

\phevnt
\aparag The event goes on as described in \ref{pIII:FWR:Continuation1 Truce},
except that the military operations follow the rules of \ref{pIII:FWR:War2
  Military}, or as in \ref{pIII:FWR:Continuation1 War} if no peace was signed.



\event{pIII:FWR League}{III-D (3)}{The Rise and Fall of the League}{1}{PB}

\tour{Turn 1}

\phevnt
\aparag A new war breaks out. Apply the full effects of \ref{pIII:FWR:Politic
  Crisis}, \ref{pIII:FWR:Economic Crisis}, \ref{pIII:FWR:Uprisings} and
\ref{pIII:FWR:Military Troubles}.
\aparag If \REB spends 50\ducats, the Rebels will have a bonus of \bonus{+1}
to their reinforcement roll.
\aparag If \FRA is \CATHCR or \CATHCO, \LIG may give finances to \lig.  It
spends 100\ducats and takes the control of the stack commanded by
\leader{Henri de Guise} (he can take new forces during the military rounds as
long as the hierarchy is respected). One purpose of this is to attempt a Coup
by the League (as explained in \ref{pIII:FWR:League Coup}).

\phdipl
\aparag Usual Foreign interventions are permitted (even during the first
turn).

\begin{digressions}[Specific conditions of the third event]


  \digression[pIII:FWR:War3 Military]{Military operations during the League}

  \phmil
  \aparag Use the general rules of \ref{pIII:FWR:Military Operations}.
  \aparag At the beginning of each military round (except the first), a new
  \REVOLT is rolled for in France. If this \REVOLT is in the rebel minor
  country and has no \REVOLT nor Loyalist land force in it, place a new
  \REVOLT \facemoins which takes the city.
  \aparag \FRA (and its allies) have a bonus of \bonus{+2} to suppress \REVOLT
  in France and perform automatic surrenders of rebel fortresses as in the
  previous wars.


  \digression[pIII:FWR:League Coup]{Guise Coup and assassination}

  \phpaix
  % \aparag A Truce is necessarily signed at the end of the turn, and the
  % favoured side is determined as explained in
  % \ref{pIII:FWR:Truces}. All the effects explained here are
  % applied (so the \REVOLT are withdrawn before extension or Stability loss).
  \aparag If \LIG has taken control of \leader{Henri de Guise} and this
  general is not Captured, it may attempt a Coup that will make \leader{Henri
    de Guise} the Heir of the kingdom, by spending 100\ducats more.
  \aparag If \FRA is \CATHCO, or if \LIG has taken control of \leader{Henri de
    Guise}, \FRA may attempt to murder this pretender, even if \LIG does not
  attempt a Coup.
  \aparag Both those operations are described in the following event,
  \ref{pIII:FWR Succession} and are resolved as described in
  \xnameref{pIII:FWR:Coup Murder Pretender}.
  \bparag If the Coup is successful, \ref{pIII:FWR Succession} begins the very
  next turn, with \monarque{Henri de Guise} as the mandatory Heir (see
  afterwards).
  \bparag If \leader{Henri de Guise} was murdered and no event
  \shortref{pIII:FWR} happens (by Dynastic Crisis or rolled event), the Truce
  is broken by the \lig who is the Rebel for one particular war. Apply the
  procedure for a Truce broken, with \lig as the Rebels.
\end{digressions}

\tour{Turn 2 and following}

\phevnt
\aparag The event goes on as described in \ref{pIII:FWR:Continuation1 Truce},
except that the military operations follow the rules of \ref{pIII:FWR:War3
  Military}, or as in \ref{pIII:FWR:Continuation1 War} if no peace was signed.
\bparag If \leader{Henri de Guise} was murdered the previous turn and no
\shortref{pIII:FWR} happens (either by Dynastic Crisis or rolled event), the
Truce is now broken by the \lig who is the Rebel for this particular war.
Apply the procedure for the breaking of a Truce, with \lig as the Rebels.
\lig receives the general \leaderwithdata{Mayenne}.
\bparag Else, the Rebels are those of the previous war if the Truce is broken.
\aparag[Foreign limited interventions] (added to those already allowed).
\bparag Some limited interventions are allowed here; a country can help only
the first at-war country listed, or none at all.
\bparag \HOL can help \hug else a non \CATHCR\ \FRA.
\bparag \ENG Protestant or \CATHCR can help \hug else a non \CATHCR\ \FRA.
\bparag \ENG \CATHCR can help \lig, else a non Protestant \FRA.
\bparag \SPA can help \lig, else a non Protestant \FRA.



\event{pIII:FWR Succession}{III-D (4)}{War of Succession}{1}{PB}

\activation{This events is activated by a Dynastic Crisis during the Wars of
  Religion, or as the fourth event of \numberref{pIII:FWR}, or after a
  successful Coup by \leader{Henri de Guise}.}

\tour{Turn 1}

\phevnt
\aparag \hug and \lig revolt and will fight to impose their pretender on the
French Crown. Every one is sure now that there is no direct Heir of the last
Valois Monarch, \monarque{Henri III}.
\aparag If the French Monarch \monarque{Henri III} died at the beginning of
this turn, \FRA has to choose its Heir. Apply now the effects of
\xnameref{pIII:FWR:Designation Heir}, followed by the effect of the new
Religious attitude.
\aparag If a Coup was successful at the previous turn, the designated Heir is
now the one of the side having made this Coup. Apply his choice of Religious
Attitude.
\aparag Otherwise, apply only the event corresponding to the current Religious
attitude of \FRA; \FRA will have the opportunity to modify the would-be Heir
at the time of the death of the last Valois Monarch.
\aparag Only a Coup or a mandatory change of religion can change the Heir once
he is appointed.
\aparag Apply the full effects of \ref{pIII:FWR:Politic Crisis},
\ref{pIII:FWR:Economic Crisis}, \ref{pIII:FWR:Uprisings} and
\ref{pIII:FWR:Military Troubles}. Also apply \ref{pIII:FWR:War4 Military} and
\ref{pIII:FWR:War4 Peace}.
\begin{digressions}[The choice of the Heir]


  \digression[pIII:FWR:Designation Heir]{Designation of the Heir}

  \phevnt
  \aparag There are three possible Heirs.  Each one is linked to the choice of
  a Religious attitude, and \FRA can not change completely its attitude on its
  own: \CATHCR can not choose Protestant and a Protestant \FRA can not choose
  \CATHCR.  Any other choice is permitted.  \FRA can be forced to change its
  attitude because of a Coup.
  \aparag[\CATHCR] The Heir is \monarque{Henri de Guise}.  If \leader{Henri de
    Guise} is alive, the general is also the Heir; if not it's a cousin with
  random military capacities. The Heir has values 6/9/7.  When the Monarch is
  \monarque{Henri de Guise}, \FRA gains a free maintenance for one \ARMY
  \faceplus, event if it is still in Civil War. \FRA immediately annexes
  \provinceLorraine.
  \aparag[\CATHCO] The Heir would be \monarque{Henri IV}, that is a converted
  \monarque{Henri de Navarre}. If \leaderNavarre is alive, the general is also
  the Heir; if not it's a cousin with random military capacities. The Heir has
  values 9/9/9. When the Monarch is \monarque{Henri IV}, \FRA gains a free
  maintenance for one \ARMY \faceplus, event if it is still in Civil War.
  \aparag[Protestant] The Heir is \monarque{Henri de Navarre} who remains
  Protestant. If \leaderNavarre is alive, the general is also the Heir; if not
  it's a cousin with random military capacities. The Heir has values 9/9/9.
  \aparag[A new religious attitude] The designation of an Heir changes
  immediately the Religious Stand of \FRA. The Heir is Crowned now if the king
  is dead, or assists the king and will be crowned at the time of its
  death. If the Heir dies, another of the same family (and same
  characteristics) will stand forward. An Heir does not make Survival Test
  before its crowning; it will last 5 turns beginning with the turn of its
  crowning.
  \aparag
  Apply one of \xnameref{pIII:FWR:France is Protestant},
  \xnameref{pIII:FWR:France is Counter-Reformation} or
  \xnameref{pIII:FWR:France is Conciliant}.


  \digression[pIII:FWR:France is Protestant]{France is Protestant}

  \phevnt
  \aparag \lig rebels, following the general rules.
  \aparag If \monarque{Henri III} is dead and the Heir is crowned, \LIG can
  make a limited intervention from the first turn of the war.  Moreover, \lig
  will have a bonus of \bonus{+2} to its reinforcement roll.
  \aparag If \leader{Henri de Guise} is dead, \lig receive the general
  \leaderMayenne (B.2.2.2).
  \aparag \LIG can always make a limited intervention from the second turn of
  the war onward.
  \aparag \hug is immediately annexed by \FRA: its provinces become french
  provinces (and provide income as such) and its units (armies, leaders)
  become french units. Both the counter limits and free maintenance of \FRA
  resumes their regular values.


  \digression[pIII:FWR:France is Counter-Reformation]{France is \CATHCR}

  \phevnt
  \aparag \hug rebels, following the general rules.
  \aparag If \monarque{Henri III} is dead and the Heir is crowned, \HUG and
  \HOL can make a limited intervention from the first turn of the war.
  \aparag \HUG and \HOL can always make a limited intervention from the second
  turn of the war onward.
  \aparag \lig is immediately annexed by \FRA: its provinces become french
  provinces and its units become french units. Both the counter limit and
  maintenance of \FRA resume their regular values.


  \digression[pIII:FWR:France is Conciliant]{France is \CATHCO}

  \phevnt
  \aparag[If the king is \monarque{Henri III}, a Valois]
  \bparag Both \lig and \hug rebel, and a three-sided war begins between \FRA
  and the two Rebels.
  % \bparag The initial repartition of French forces is: \FRA has \ARMY
  % \faceplus, \lig has \ARMY \facemoins and \hug has \DT.
  % \bparag If the naval forces desert, decide at random if it is to
  % join the \hug or the \lig.
  \bparag \leaderNavarre is a possible Heir but is hesitant.  He is used as a
  general by \FRA, excepted if \hug controls or besieges \ville{Paris}.  He
  will go the side of \FRA as soon as he is chosen as Heir at the death of
  \monarque{Henri III}, or could go back to the Protestant side if
  \monarque{Henri de Navarre} is the chosen Heir, or if a Protestant Coup is
  made.
  \aparag Notice that as soon as \monarque{Henri III} die, one of the minor
  (the one having the chosen heir) will sign peace with \FRA and be
  immediately annexed.
  % Leader \leaderNavarre is a possible Heir but is hesitant.  Neither
  % the \hug nor \FRA can use it. He will go the side of \FRA as soon as
  % he is chosen as Heir at the death of \monarque{Henri III}, or could
  % go back to the Protestant side if \monarque{Henri de Navarre} is the
  % chosen Heir, or if a Protestant Coup is made.
  \aparag[If the king is the Heir,] (brand-new catholic \monarque{Henri IV}).
  \bparag \lig rebels, following the general rules.
  \aparag If \leader{Henri de Guise} is dead, \lig receive the general
  \leaderMayenne (B.2.2.2).
  \aparag If \leader{Henri de Guise} is alive, \lig will have a bonus of {\bf
    +2} to its reinforcement roll.
  \aparag \hug is immediately annexed by \FRA.
\end{digressions}

\phdipl
\aparag Foreign intervention are allowed.

\phadm
\aparag \FRA gets full income of all non-revolted, controlled provinces,
including those belonging to a revolted rebel or in the \ROTW.
\aparag As soon as the last Valois dies, \FRA is no more restricted in
administrative actions.
\aparag[Reinforcements of Rebels]
\bparag If \LIG spends 50\ducats, the \lig will have a bonus of \bonus{+1} to
their reinforcement roll.
\bparag If \HUG spends 50\ducats, the \hug will have a bonus of \bonus{+1} to
their reinforcement roll.

\tour{Turn 2 and afterwards}

\phevnt
\aparag Except for what follows, use the same rules as turn 1.
\aparag If the French Monarch \monarque{Henri III} died at the beginning of
some turn, \FRA has to choose its Heir (if no Coup has imposed an Heir). Apply
the effect of \ref{pIII:FWR:Designation Heir}, and then the effect of the
(possibly new) Religious attitude that follows. The revolted side receives new
\REVOLT according to \ref{pIII:FWR:Uprisings}.
\aparag Else, if a Coup was successful, apply \ref{pIII:FWR:Uprisings} to roll
for new \REVOLT of the now rebel side. The war resumes with rebels depending
on the new religious attitude.
\aparag If a pretender was murdered on the previous turn, new \REVOLT are
rolled for according to \ref{pIII:FWR:Uprisings} for this side only.
\aparag \paysSavoie will make (or continue) a limited intervention as an ally
of \lig (or \FRA if \CATHCR), with an \ARMY \faceplus and one unnamed minor
general.

\phadm
\aparag[Reinforcements of Rebels]
\bparag Reinforcements will be received for the rebel side(s) according to
\ref{pIII:FWR:Military Troubles} but the initial repartition of forces is not
made anew (it has already been done).
\bparag If \LIG spends 50\ducats, the \lig will have a bonus of \bonus{+1} to
their reinforcement roll.
\bparag If \HUG spends 50\ducats, the \hug will have a bonus of \bonus{+1} to
their reinforcement roll.

\begin{digressions}[Specific conditions of the War of Succession]


  \digression[pIII:FWR:War4 Military]{Military operations during the War of
    Succession}

  \phmil
  \aparag Use the general rules of \ref{pIII:FWR:Military Operations}.
  \aparag \FRA and its allies have a bonus of \bonus{+3} to suppress \REVOLT
  in France.
  \aparag \paysPalatinat makes (or continues) a limited intervention as an
  ally of the side of \leaderNavarre or \monarque{Henri de Navarre} with
  \ARMY\faceplus, \LD and a random general. If the Monarch is \monarque{Henri
    III} with \monarque{Henri IV} as the chosen Heir, \paysPalatinat makes no
  intervention.
  \aparag \FRA draws supply from any province in France (including those of
  \lig and \hug), except those in \REVOLT
  \aparag \lig and \hug draw supply only from the provinces they control.
  \aparag[Voluntary surrender]
  \bparag A city besieged by \FRA with at least one \ARMY \faceplus,
  voluntarily surrenders if there was no \REVOLT \faceplus in it at the
  beginning of the turn, nor is it a Place of Safety and there is no more
  \REVOLT in the province (including if the \REVOLT was just crushed this
  round).
  \bparag A city besieged by \lig with at least one \ARMY \faceplus,
  voluntarily surrenders if it is in the territory owned by \lig.
  \bparag A city besieged by \hug with at least one \ARMY \faceplus,
  voluntarily surrenders if it is in the territory owned by \hug.


  \digression[pIII:FWR:War4 Peace]{How to end the War of Succession?}

  \phpaix
  \aparag If there are only 2 sides in this war, the War of Succession ends if
  \FRA control Paris and wins a Major Victory over Rebel forces (at least 3
  \DT of Rebels) or if all Rebel forces and \REVOLT have been eliminated.
  \bparag \FRA has to spend 100\ducats to stop the war; no Coup or
  Assassination can happen. Apply \ref{pIII:FWR:End of the War of Succession}.
  \aparag If there are only 2 sides in this war, the War of Succession ends if
  \FRA has no land forces left and the Rebel controls the city of Paris. A
  Coup in favour of the Rebels is automatically made with no possible murder
  attempt by \FRA. A mandatory change of Religious attitude is imposed on \FRA
  and the new Monarch is the Heir of the winning side. Apply \ref{pIII:FWR:End
    of the War of Succession}.
  \aparag \FRA ends as barely victorious if this is the end of the first turn
  of period IV (then no Coup is permitted). Apply now \ref{pIII:FWR:End of the
    War of Succession} and \ref{pIII:FWR Final}.
  \aparag If \lig is in rebellion, controls the city of Paris, and
  \leader{Henri de Guise} is alive, then \LIG can spend 100\ducats for an
  attempt of Counter-Reformation Coup.
  \aparag If \hug is in rebellion, controls the city of Paris, and
  \leaderNavarre is alive, then \HUG can spend 100\ducats for an attempt of
  Protestant Coup.
  \aparag If a Coup is attempted, \FRA can try to murder the pretender
  (\leader{Henri de Guise} or \leaderNavarre).
  \aparag If no Coup is attempted, \FRA can try to murder one pretender of
  revolted \lig or \hug (\leader{Henri de Guise} or \leaderNavarre).
  \aparag \FRA loses no \STAB because of the \REVOLT but loses \STAB as in an
  usual war (1 the first turn, 2 the second, \dots)
  \aparag The war keeps on until one side is victorious; there is no Truce.


  \digression[pIII:FWR:Coup Murder Pretender]{Coup and Murder of the
    Pretender}

  \phpaix
  \aparag The side attempting the Coup (\LIG or \HUG) has to spend 100\ducats
  then rolls 1d10 and adds \bonus{+2} if \FRA is \CATHCO; \bonus{+2} if the
  \ref{pIII:FWR:Saint-Barthelemy} was not perpetrated; \bonus{+3} if the
  \ref{pIII:FWR:Saint-Barthelemy} was made against the religious faction of
  the coup's side; \bonus{+2} if \FRA makes no Murder attempt; \bonus{+1} per
  victory of the pretender's minor country with at least one \ARMY \faceplus.
  \aparag[Failure of the Coup] If the result of the Rebels is 9 or lower, the
  Coup is failed. It may succeed if the result is 10 or higher.
  \aparag If \FRA attempts to murder the pretender, it rolls 1d10, and add
  \bonus{+2} for each point of \STAB that it spends (it has to have those
  points); and \bonus{+3} is no Coup attempt was made.
  \aparag[Result of Assassination] If the result of \FRA is 9 or lower, the
  murder is failed. It may succeed if the roll is 10 or higher.  \FRA loses
  {\bf 1} \STAB, and the Valois \monarque{Henri III} will have
  an additional permanent malus of \bonus{+3} to its Survival Test until his
  death. \\
  \centerline{\textit{"Il est encore plus grand mort que vivant."}}
  \aparag[If both a Coup and a Murder succeed]
  \bparag If the result of \FRA is higher of equals to the result of the Coup,
  the Coup actually fails; the Pretender is murdered.
  \bparag Else (if the result of Rebels is higher than the result of \FRA),
  the Coup succeeds. \FRA makes a mandatory change of Religious attitude and
  of designated Heir. The pretender is not killed (miraculously saved!) and
  becomes the new Heir.
  \aparag[Successful Coup]
  \bparag The new mandatory Heir is the one (\monarque{Henri de Guise} or
  \monarque{Henri de Navarre}) of the side doing the Coup and the Religious
  attitude of \FRA is changed according to this new Heir.
  \bparag When \monarque{Henri III} dies, the Heir is crowned as the French
  King.
  \bparag If this case, on the next turn, a Civil War with the new sides
  depending of the new Religious attitude continues, or begins if the Coup was
  during event (3).
  \aparag In addition, \FRA has a mandatory defensive alliance with the
  controller of the side having done the Coup, and this power can now make
  full intervention in the war until the end of \nameref{pIII:FWR}.


  \digression[pIII:FWR:End of the War of Succession]{End of the War of
    Succession}

  \phinter
  \aparag \STAB of \FRA is raised by {\bf 2}.
  \aparag The new Monarch is the last designated Heir (\monarque{Henri III} is
  pushed aside if he is still alive...)
  \aparag All \REVOLT and forces of minor countries \hug and \lig are
  removed. But they continue to exist (they can rebel one more time if \FRA is
  not \CATHCO).
  \aparag[Intervention of Foreign countries]
  \bparag Minor countries having forces left in \FRA propose an immediate
  white peace to \FRA. If it is accepted, they withdraw and are at peace with
  \FRA. Else, they are now in a regular war with \FRA (but no one is victim of
  a declaration of war).
  \bparag Any Major power having forces left in \FRA has to sign a white
  peace, or are from now on in regular war with \FRA.  Their military activity
  is no more limited; nobody is victim of a declaration of war (but \FRA and
  its enemies are at war), and regular call to allies will be possible on the
  next turn.  This war causes normal loss of \STAB, beginning with a loss of
  {\bf 1} \STAB this turn.
  \bparag The only specificity of this war is that, if a unconditional peace
  is forced on \FRA, the winning power must change the Monarch of \FRA to the
  Heir of its Religious Attitude.  In this case this is the only condition of
  the peace, and \FRA has a mandatory defensive alliance with the winners
  during the reign of the new Monarch.
  \aparag As soon as \FRA is at peace at an end-of-turn and \CATHCO,
  \ref{pIII:FWR Final} is applied.
\end{digressions}



\event{pIII:FWR Last Stand}{III-D (5)}{Last Stand of the Heretics}{1}{PB}

\history{Alternate history}

\condition{}
\aparag If \ref{pIII:FWR Succession} is not finished, do not mark off and
reroll.
\aparag If \FRA is \CATHCO and no unconditional surrender was obtained by \FRA
against \hug in a previous war, mark off the event, play \RD instead and the
French king will have a malus of \bonus{+2} to his Survival test for the next
turn.
\aparag If \FRA is \CATHCO but did force an unconditional surrender of \hug in
a previous war, \hug rebels itself.
\aparag If \FRA is Protestant or \CATHCR at the end of \ref{pIII:FWR
  Succession} and \ref{pIII:FWR Final} was not applied, the rest of the event
happens.
\aparag If \FRA is Protestant or \CATHCR but \ref{pIII:FWR Final} already
occurred, play \RD instead with the \REVOLT on the table of \FRA.

\phevnt
\aparag One of \lig or \hug rebels itself depending on the religion of
\FRA. Apply the full effects of \ref{pIII:FWR:Politic Crisis},
\ref{pIII:FWR:Economic Crisis}, \ref{pIII:FWR:Uprisings} and
\ref{pIII:FWR:Military Troubles}.  Also apply \ref{pIII:FWR:War5 Military} and
\ref{pIII:FWR:War5 Peace}.
\aparag If the revolting minor was already annexed by \FRA (this may happen if
a mandatory religious change is then forced on \FRA), recreate it
immediately. It will get no troops at beginning.
\aparag If the non-rebelling minor still exists, it is immediately annexed by
\FRA: its provinces become regular French provinces and its units become
french units.
\aparag \REB is not obliged to do a white peace with \FRA.
\bparag If it chooses to continue a war, it can make a full military
intervention in the Civil War. But it will continue to suffer a normal loss of
\STAB at the end of turns, whereas \FRA will lose at most {\bf 2} \STAB each
turn during the Civil War.
\bparag If it chooses to sign a white peace, or if it was at peace, \REB can
make a limited intervention in the war.
\aparag \LIG can make a limited intervention as an ally of a \CATHCR\ \FRA.
\aparag \HOL can make a limited intervention as an ally of Protestant
\FRA. Else it can make a limited intervention as an ally of \hug.

\phdipl
\aparag Usual foreign interventions are allowed.

\begin{digressions}[Specific conditions of the last event]


  \digression[pIII:FWR:War5 Military]{Military operations during the fifth
    event}

  \phmil
  \aparag Use the rules of \ref{pIII:FWR:Military Operations}.
  \aparag \FRA and its allies have a bonus of \bonus{+4} to suppress \REVOLT
  in France.
  \aparag A city in \FRA that had not a \REVOLT \faceplus at the beginning of
  the turn, makes an immediate voluntary surrender if besieged by a land stack
  of \FRA (or its allies) that sets a siege with at least one \ARMY \faceplus
  and there is no more \REVOLT in the province (including if the \REVOLT was
  just crushed this round).


  \digression[pIII:FWR:War5 Peace]{How to end the Last Stand?}

  \phpaix
  \aparag \FRA loses no \STAB because of the \REVOLT . It loses at most {\bf
    2} \STAB per turn because of the war.
  \aparag No Truce happens ever in this civil war. It keeps going until one
  side wins.
  \aparag The War ends if \FRA controls Paris and wins a Major Victory over
  Rebel forces (at least 3 \DT of Rebels) or if all Rebel forces and \REVOLT
  have been eliminated.
  \aparag The War ends if \FRA has no land forces left and the Rebel controls
  the city of Paris. An change of Heir in favour of the Rebels is
  automatically made (with no possible murder attempt by \FRA) that causes a
  mandatory change of Religious attitude. The new Monarch of \FRA is the Heir
  of the winning side.
  \aparag \FRA ends as barely victorious if the last turn of period III has
  ended (now or previously).


  \digression[pIII:FWR:End of the Last Stand]{End of the Last Stand}

  \phpaix
  \aparag \STAB of \FRA is raised by {\bf 2}.
  \bparag The new Monarch is the last designated Heir (if it did change; the
  former one is pushed aside)
  \bparag All \REVOLT %and forces of minor countries \hug and \lig
  are removed.
  \bparag Minor countries having forces left in \FRA propose an immediate
  white peace to \FRA. If it is accepted, they withdraw and are at peace with
  \FRA. Else, they are now in a regular war with \FRA (but no one is victim of
  a declaration of war).
  % \bparag Minor countries having forces left in \FRA withdraw and are at
  % peace with \FRA.
  \bparag Any Major power having forces left in \FRA has to sign a white
  peace, or are from now on in regular war with \FRA.  Their military activity
  is no more limited; nobody is victim of a declaration of war (but \FRA and
  its enemies are at war), and regular call to allies will be possible on the
  next turn.  This war causes normal loss of \STAB, beginning with a loss of
  {\bf 2} \STAB this turn for everyone.  The Sole Defender of the Catholic
  Faith could impose a change of Religion, but by normal rules and not by
  specific rules of this event.
  \bparag When this War ends, apply \ref{pIII:FWR Final}.
\end{digressions}



\event{pIII:FWR Final}{III-D (Final)}{End of the Wars of Religion}{1}{PB}

\activation{}
\aparag This event is applied when the fifth event \xref{pIII:FWR} is at last
resolved.
\aparag This event is applied also as soon as \FRA is at peace and \CATHCO
after the end of the fourth event.
\aparag At the end of the last turn of the period III (or the first turn of
period IV if \ref{pIII:FWR Succession} is happening), this event is applied
regardless of other conditions.

\phinter
\aparag The Wars of Religion are ended. Further events \numberref{pIII:FWR}
cause \RD.
\aparag The Monarch should be the designated Heir, or the Heir is crowned
right now.
\aparag Minor countries \hug and \lig are immediately annexed by \FRA. All
their provinces are now regular provinces of \FRA. All their land forces
become french land forces. \FRA gets back its regular counter limit and
maintenance.  The navy is given back to \FRA. If alive, \leaderConde,
\leaderColigny, \leaderMayenne, \leaderNavarre and \leader{Henri de Guise}
retire (excepted the now Monarch); all other french leaders are now regular
french leaders.
\aparag If the king is \monarque{Henri de Guise} or \monarque{Henri IV}, \FRA
gains a free maintenance of one \ARMY \faceplus until the end of his
reign. This is not the case if the Monarch is \monarque{Henri de Navarre}.
\aparag[Victory Points]
\ENG, \HOL and \SPA win each 25 \PV if they have been allied at least once to
the side of the Heir that won finally the wars. They lose 25 \PV if they have
fought against this winning side.
\aparag[Economic consequences] Roll 1d10 and add \bonus{+1} for each
favourable truce conceded to the rebels, \bonus{+1} if \FRA has been complied
to change its Religious attitude, and \bonus{+1} is \FRA is \CATHCR.
\bparag Result 1-3: 1 level of French \TradeFLEET is lost to \HOL;
\bparag Result 4-5: 1 level of French \TradeFLEET is lost to \HOL, and 1 to
\ENG;
\bparag Result 6-10: 2 levels of French \TradeFLEET are lost to \HOL, and 1 to
\ENG; the \FTI of \FRA is diminished by \bonus{-1};
\bparag Result 11+: 2 levels level of French \TradeFLEET are lost to \HOL, and
2 to \ENG; both \FTI and \DTI of \FRA are diminished by \bonus{-1};
\bparag \HOL chooses first the \TradeFLEET it takes, then \ENG chooses.
\bparag If \FRA is \CATHCR, the level chosen by \HOL are lost but not received
by \HOL; \ENG gains the levels if it is Catholic, if not those levels are lost
for everyone.
\bparag if \ENG is \CATHCR and \FRA is not, \SUE chooses and gains the levels
instead of \ENG.
% \aparag Roll 1d10. Add \bonus{+2} if \CATHCR, \bonus{-2} if Protestant. On a
% result of 6+, one french \COL or \TP (decide at random between all those
% that exist) is lost.
\aparag[Undesired policy]
\bparag If the chosen Heir was Protestant but \FRA is no more Protestant at
the end of the Wars of Religion, \FRA has a malus of \bonus{-2} to all its
colonial actions during the period IV and its \FTI and \DTI are diminished by
a further 1.
\bparag If the chosen Heir was \CATHCO but \FRA is \CATHCR at the end of the
Wars of Religion, \FRA has a malus of \bonus{-1} to all its colonial actions
during the period IV.  Each event \RD obtained in period IV has a chance to
make appear a second \REVOLT \faceplus in \FRA. Roll 1d10: 1-3
\provincePoitou, 4-6: \provinceGuyenne, 7-10: none.
\bparag If the chosen Heir was \CATHCR but \FRA is no more \CATHCR, \FRA has a
malus of \bonus{-2} to all its Technological actions during the period IV.
Each event \RD obtained in period IV has a chance to make appear a second
\REVOLT \faceplus in \FRA. Roll 1d10: 1-3 \provinceArmor, 4-6:
\provinceOrleanais, 7-10: none.

% Local Variables:
% fill-column: 78
% coding: utf-8-unix
% mode-require-final-newline: t
% mode: flyspell
% ispell-local-dictionary: "british"
% End:

% LocalWords: pIII FWR PBNew malus Ligue Ile de unbesieged l'Hospital pII
% LocalWords: Barthelemy Casimir Schmalkaldic Mayenne Conciliant reroll Jym
% LocalWords: TODO


\stopevents

% Local Variables:
% fill-column: 78
% coding: utf-8-unix
% mode-require-final-newline: t
% mode: flyspell
% ispell-local-dictionary: "british"
% End:

% LocalWords: pII pIV pIII VOC Oxenstierna FWR Barthelemy Lublin Oprichnina
% LocalWords: Safavids Mughal RistoMod pI hollande Auld Stadhouder Compagnie
% LocalWords: Vereenigde Oostindische PBNew TYW Risto EIC POR Teutoniques de
% LocalWords: Kurland Duche Mancha malus interphase WPT Jym Prusse Zygmunt
% LocalWords: Vasa Moghol Mughals Petersbourg Petersburg CCA Formose URP JCD
%  LocalWords:  Moriscos


\clearpage

% -*- mode: LaTeX; -*-
\chapter{Political Events of Period IV}
%\section{Period IV}
\label{events:pIV}



\subsection*{Event Table of Period IV}

\begin{eventstable}[Period IV events table]
  \tabcolsep=4.5pt\centering%
  \begin{tabular}{|l|*{6}{c}|l|}
    \hline
    1\up{st}\textarrow& 1-3 & 4-5 & 6 & 7 & 8 & 9 & 10 \\ \hline
    1 & 1  &  1 & 13 & R4 & R19 & 7   & \textbullet~1--2: \\
    2 & 12 & 14 & 15 & R5 & 18  & 8   & +1 then\\
    3 & 17 & 15 & 9  & 6  & 17  & R9  & \nameref{events:pIII}\\
    4 & 18 & 16 & 10 & R7 & 16  & R17 & \textbullet~3--10:\\
    5 & 10 &  4 & R11& 8  & 14  & R18 & \nameref{events:pIII}\\
    6 & 3  &  2 & 12 & 9  & 1   & 19  & \\
    7 & 7  &  6 & 1  & 11 & R5  & R20 & \\
    8 & 22 & R4 & 2  & 12 & 21  & R4  & \\
    9 & 5  & R7 & 3  & 13 & R22 & 8   & \\ \hline
    10 & \multicolumn{7}{l|}{\nameref{events:pV}} \\ \hline
  \end{tabular}
\end{eventstable}

\eventssummary{%
  pIV:Bohemian Revolt|,%
  pIV:Augsburg Revocation|,%
  pIV:Olivares|,%
  pIV:Unity HRE|,%
  pIV:War Persia Turkey|,%
  pIV:Persian Safavids|O{pIII:Persian Safavids},%
  % Change to Morocco/Portuguese Revolt ?
  pIV:Portuguese Revolt|,%
  pIV:Morocco|,%
  pIV:Act Navigation|,%
  pIV:Union Scotland|,%
  pIV:English Civil War|,%
  pIV:English Restoration|,%
  pIV:London Stock Exchange|,%
  pIV:Amsterdam Stock Exchange|O{pIII:Amsterdam Stock Exchange},%
  pIV:Dutch Colonial Dynamism|E/E/E,%
  pIV:Liberum Veto|,%
  pIV:Great Elector|,%
} \eventssummary{%
  pIV:Oxenstierna|O{pIII:Oxenstierna},%
  pIV:Union Poland Sweden|O{pIII:Union Poland Sweden},%
  pIV:Torstensson|,%
  pIV:Swedish Nobles Unrest|,%
  pIV:La Rochelle|E/O{pIV:Morocco},%
  pIV:Richelieu|,%
  pIV:Fronde|,%
  pIV:Times of Troubles|,%
  pIV:Revolt Cossacks|,%
  pIV:Extension Moghol|E/E,%
  pIV:Wars India|E/E,%
  pIV:Revolt Singala|E/E,%
  pIV:China Colonial Attitude|O{pIII:China Colonial
    Attitude}/S{pIV:CCA:Vassalisation Korea},%
  pIV:Japan Colonial Attitude|,%
  pIV:Deluge|,%
  pIV:Koprulu|,%
  O|,%
  pIV:TYW|,%
  pIV:Polish Civil War|,%
}

\newpage\startevents



\event{pIV:Bohemian Revolt}{IV-1 (1)}{Bohemian Revolt}{1}{PBNew}

\history{1618-1621}[This event describes the War for \paysBoheme, whereas the
break out of a general German conflict (that historically followed this event)
is dealt with in \ref{pIV:TYW}.]

\phevnt
\aparag[The Winter King] The minor country \paysBoheme is created / separated
/ breaks alliance (depending on its previous status) from its current
allegiance (even a \GE), and allies itself with \paysPalatinat (which would
also breaks from an existing \GE). The first Major power in the list: \FRA
(except if \CATHCR), \POL (if Protestant), else \SUE (even if Catholic)
controls both those countries and have them placed in \EG on its diplomatic
track.
\aparag \paysBaviere and \hab declares war to these two countries. This is a
Religious Civil War (see \ref{chDiplo:Religious Civil War}) in the \HRE.
\bparag \AUS has a free \CB against \paysBoheme and must use it or lose {\bf
  2} \STAB.
% (JCD) probably, \paysBaviere should be controlled by \MAJHAB, not
% \SPA. Changing.
\bparag If \HAB declares war, \paysBaviere is placed in \EG of \HAB and is
controlled by \MAJHAB.
\bparag \SPA is allowed a limited intervention in the war as an ally of
\HAB. Other countries are constrained by usual rules.

\aparag[The Revolt of Bethlén]
\bparag A \REVOLT \faceplus is placed in a randomly chosen province belonging
to \HAB in \paysHongrie. It controls the city. The \REVOLT is controlled by
\RUS.
\bparag The military forces of the Revolt of Bethlén can use up to 1 \ARMY and
2 \LD of the Hungarian counters (and the \HAB can use at most one \ARMY and 2
\LD from Hungarian counters).
\aparag \TUR cannot declare war against \HAB at this turn.

\phadm
\aparag \AUSMin receives its usual forces and reinforcements.
\aparag \paysBaviere has 1 \ARMY \faceplus, 3 \LD (all Veterans), 1 \fortress
and is commanded by \leaderwithdata{Tilly} (lasting 4 turns).  It has 2
Multiple Campaigns. \paysBaviere has 2 \ARMY counters at its disposition
during the whole length of this event.
\bparag[Tilly's training] [BLP] Troops of \paysBaviere (not its allies)
stacked with \leaderTilly are always \TTER.
\aparag \paysBoheme has 1 \ARMY \faceplus, 1 \LD (Conscripts) and 1 \fortress.
\aparag \paysPalatinat has 1 \ARMY \faceplus (Veterans) leaded by
\leaderwithdata{Mansfeld} (lasting 3 turns). It has 1 Multiple Campaign.
\aparag The Revolt of Bethlén consists of one Hungarian \ARMY \faceplus
(Conscripts) and \leaderwithdata{Bethlen} (lasting 4 turns) placed in the
province of the \REVOLT .
\aparag None of \paysBaviere, \paysBoheme, \paysPalatinat and the Revolt of
Bethlén receive reinforcements on the first turn. They receive normal
reinforcements beginning with the second turn of the war.
\aparag The reinforcements of the Revolt of Bethlén are based only on the
provinces in \payshongrie that he controls or that are in \REVOLT . If there
are none, or if \leaderBethlen is not in play (dead or wounded), it receives
no reinforcements.

\phmil
\aparag \leaderTilly may lead any stack of \paysBaviere or its allies.
\aparag \leaderMansfeld may lead any stack of \paysPalatinat or its allies. It
can retreat with 1 \LD (only) in any neutral Protestant or mixed \HRE country
and remain there (after a battle or a retreat before battle).

\aparag[Destruction of Bohemia]\label{pIV:BR:Destruction}
If \villePrague is captured, \paysBoheme is destroyed at the end of the
current round. All its provinces are now owned by \HAB.  Its military forces
are disbanded and its provinces not yet military controlled by \HAB are
considered controlled by rebels (use Control markers of \paysBoheme); they
surrender as soon as an \ARMY \faceplus besieges them, or by regular siege
with smaller forces.

\aparag[Bethlén]
\bparag The forces of \leaderBethlen are always in restricted supply in the
national provinces of \paysHongrie (provinces with Hungarian shield). They use
the \REVOLT in \paysHongrie and the cities they control as regular supply
sources.
\bparag A force lead by \leaderBethlen can withdraw in \paysTransylvanie or
national provinces of \payshongrie owned by \TUR (by retreat or movement). If
he retreats there, he must stay there until the end of turn but may go out on
any following turn.
\bparag If \leaderBethlen and/or its forces are in \paysTransylvanie
or national provinces of \payshongrie owned by \TUR, \TUR may make a foreign
intervention against both the revolt of \leaderBethlen and \HAB. Declaring the
intervention cost 1\STAB to \TUR.
% \bparag As long as \leaderBethlen and/or its forces are in \paysTransylvanie
% or national provinces of \payshongrie owned by \TUR, \TUR has a free \CB
% against \leaderBethlen. If \TUR uses it, the forces of \leaderBethlen are
% also at war against \TUR and \TUR may enter, besiege and attack any force in
% national provinces of \payshongrie.

\phpaix
\aparag Before any peace is made, roll a test for the possible breakout of a
Religious war, according to \ref{pIV:TYW}.  A \bonus{-2} is applied to this
roll.
\aparag Add a Revolt \facemoins in a national province of \payshongrie if
\leaderBethlen is therein.
\aparag[Survival of \paysBoheme]\label{pIV:BR:Survival} If no such war occurs
and \villeVienne is controlled by the enemies of \HAB, the war end as a
victory of \paysBoheme.  The minor country is fully recreated ; \HAB has a
mandatory peace for 3 turns with \paysBoheme. \HAB loses {\bf 1} \STAB and 30
\PV ; The controller of \paysBoheme gains 30 \PV. \HAB gains the permanent
right to make the complete conquest of \paysBoheme.  \paysBoheme and
\paysPalatinat are placed in \AM of their controlling \MAJ.
\aparag \HAB and \paysBoheme stop war only when \paysBoheme is destroyed or if
\villeVienne is occupied by enemies.  Other countries use normal peace rules
(but are allied to \HAB and \paysBoheme and subjects to Separate Peace
modifiers if any).
\aparag[Victory conditions if the war becomes the TYW]
\bparag if the \nameref{pIV:TYW:Peace Prague} favours the \alliance and they
control \villeVienne, apply \ref{pIV:BR:Survival} if \paysBoheme still exists.
\bparag If the \nameref{pIV:TYW:Peace Prague} favours the \ligue, \paysBoheme
is destroyed as in \ref{pIV:BR:Destruction}.
\bparag Else, \paysBoheme remains at war after the \nameref{pIV:TYW:Peace
  Prague} and will survive the \nameref{pIV:TYW:Peace Westphalie} if not
destroyed before that during the war.

\tour{Turn 2 and following}

\phdipl
\aparag If this event does not evolve in \xnameref{pIV:TYW} (because there has
already been one, or an Appeasement of the religious fight was obtained), the
controller of \paysBoheme and \paysPalatinat may make a full intervention in
the war.



\event{pIV:Augsburg Revocation}{IV-1 (2)}{Revocation of the Truce of
  Augsburg}{1}{PBNew}

\history{Alternative history}

\condition{Check the conditions in the given order until one is found true.}
\aparag If events \xref{pIV:Bohemian Revolt}, \xref{pIV:Unity HRE} or
\xref{pIV:TYW} are happening now, do not mark off and re-roll.
\aparag If there is a \GE, apply the \xnameref{pIV:Augsburg:GE vs Northern
  Alliance}.
\aparag If \eventref{pIV:TYW} has not yet happened, apply the
\xnameref{pIV:Augsburg:HRE vs Augsburg}.
\aparag Else, apply \xnameref{pIV:Augsburg:Troubles HRE}.


\subevent[pIV:Augsburg:GE vs Northern Alliance]{Revolt of a Northern Alliance}

\phevnt
\aparag A Northern Alliance of countries of the \HRE is created.  The
countries \paysOldenburg, \paysHanovre, \paysHesse, \paysHanse, and \paysBerg
breaks free from the \GE and are allied.
\aparag \GE and \hab declare war to all those countries (and are controlled by
\SPA). The \GE is in Civil War (see \ref{chDiplo:Religious Civil War}).
\bparag \AUS has a free \CB against the whole Northern Alliance (to be used
immediately, or forfeited at the cost of {\bf 2} \STAB).
\aparag The Northern Alliance is controlled by the first Protestant \MAJ in
the list that accepts the alliance: \HOL, \ENG, \SUE, \FRA, \POL.  It has a
\CB to enter the war. If it declines, the next country in the list has the
opportunity to do the same (and so on).  The \MAJ gains all the \MIN of the
Alliance in \EG.  If nobody enters the war, \SUE controls the Northern
Alliance (but is not involved and does not gain diplomatic control).
\aparag If \HOL controls the Northern Alliance, it gains the advantages of
\ref{pIV:TYW:Northern HRE Alliance}, as long as the Alliance exists.

\phdipl
\aparag \SPA can make a full intervention as an ally of \HAB.
\aparag If they have not declined control of the Alliance, \FRA and \ENG (if
they are not Counter-Reformation) and \SUE can make a limited intervention in
the war alongside the Northern alliance.

\phpaix
\aparag If no \MAJ entered the war to control the Northern Alliance, it is
dealt with as one country for the peace in this war (except attempts of
Separate Peace), with a malus of \bonus{-4} to make peace.
\aparag A peace of level 3 or higher against the \MAJ in control (or the
Northern Alliance itself if there is none) would dissolve the Alliance in
addition to the peace.
\aparag If the war ends and the Alliance is not dissolved:
\bparag The \MIN are now normal independent countries that are no more part of
the \GE.
\bparag If the \MAJ was \HOL, it gains the benefits of \ref{pIV:TYW:Northern
  HRE Alliance}. Otherwise, the Alliance is dissolved for game purpose.
\aparag Remember that, according to \ref{pIV:TYW:German Empire}, a peace of
level 3 or higher against the \MAJHAB may dissolve the \GE.  Conversely, any
Unconditional Peace against a country once part of the \GE forces is back in
the \GE.


\subevent[pIV:Augsburg:HRE vs Augsburg]{War of Revocation of the Truce of
  Augsburg}

\phevnt
\aparag The Emperor of the HRE has the possibility to revoke this Truce (even
if it was not given in game terms). If he declines to do so, his country loses
{\bf 2 \STAB} and the event terminates.  If the Truce of Augsburg is revoked,
alliances are created in the \HRE and the \HRE is in Civil War.
\aparag[Northern Alliance] If a Northern Alliance already exists, skip this
paragraph.
\bparag A Northern Alliance of countries of the \HRE is created.  The
countries are \paysOldenburg, \paysHanovre, \paysHesse, \paysHanse, and
\paysBerg (if they exist). If there was no \terme{Truce of Augsburg} at the
beginning of the event, \pays{Hesse} and \pays{Berg} are not in the Alliance.
\bparag The Northern Alliance is controlled by the first Protestant \MAJ in
the list that accepts the alliance: \HOL, \ENG, \SUE, \FRA, \POL.  It has a
\CB to enter the war. If it declines, the next country in the list has the
opportunity to do the same (and so on).  The \MAJ gains all the \MIN of the
Alliance at \EW.  If nobody enters the war, \SUE controls the Northern
Alliance (but is not involved and does not gain diplomatic control).
\bparag If \HOL controls the Northern Alliance, it gains the advantages of
\ref{pIV:TYW:Northern HRE Alliance}, as long as the Alliance exists.
\aparag[Southern Alliance] If a Southern alliance already exists, skip this
paragraph.
\bparag A Southern HRE Alliance is created by association of \paysBaviere,
\paysMayence, \paysAlsace, \paysBade and \paysWurtemberg (if they exist).
\bparag The Southern Alliance is controlled by the first Catholic \MAJ in the
list that accepts the alliance: \AUS, \SPA, \POL, \FRA.  It has a \CB to enter
the war. If it declines, the next country in the list has the opportunity to
do the same (and so on).  The \MAJ gains all the \MIN of the Alliance at \EW.
If nobody enters the war, \MAJHAB controls the Southern Alliance (but is not
involved and does not gain diplomatic control).
\bparag \AUSMin gains the \MIN on its track, not \SPA, if \SPA accepts the
alliance.
\bparag If \MAJHAB controls the Southern Alliance, it gains the advantages of
\ref{pIV:TYW:Southern HRE Alliance}, as long as the Alliance exists.
\aparag Both Alliances are at war against each other.  The \HRE is in Civil
War.

\phdipl
\aparag \SPA can make a limited or full intervention alongside the Southern
Alliance (excepted if it declined the control and involvement).
\aparag \SUE can make a limited intervention alongside the Northern Alliance
(excepted if it declined the control and involvement).

\phpaix
\aparag A test to begin a Religious War in HRE is made at the end of the first
turn of this war, with a \bonus{-4} modifier.  See \ref{pIV:TYW} for the
result of the test and the possibility of this Religious War, and the renewal
or not of the test on following turns.  If no such war occurs, peace can be
made as usual.
\aparag{The alliances after the war}
\bparag If \HOL was controlling the Northern Alliance, the Alliance may last
after the war at the condition of \ref{pIV:TYW:Northern HRE Alliance}.
\bparag If \MAJHAB was controlling the Southern Alliance, the Alliance may
last after the war at the condition of \ref{pIV:TYW:Southern HRE Alliance}.
\bparag In other cases, the Alliances would not last after the end of the war.


\subevent[pIV:Augsburg:Troubles HRE]{Troubles in the Holy Roman Empire}

\condition{This event may happen twice, once because of \xnameref{pIV:Augsburg
    Revocation} and another time because of \xnameref{pIV:Unity HRE}}

\phevnt
\aparag \AUS, \HOL and \HIS rolls for one \REVOLT .
\aparag The effect of a diplomatic event is made on every minor country that
is part of the \HRE (fidelity/religion):
\FEforeachlist{listofminorsHRE}{\pays{\loopitem}
  (\theminorfid{\loopitem}/\theminorreligionshort{\loopitem})}{, }.



\event{pIV:Olivares}{IV-2 (1)}{Olivares}{1}{Risto}

\history{1621-1643}
\dure{as long as \strongministre{Olivares} remains the excellent minister}

\condition{}
\aparag \SPA can refuse this event if it so wishes. In that case mark-off as
played.
\aparag \SPA can freely remove \ministre{Olivares} from office at the end of
any following monarch survival phase and the event terminates.

\phevnt
\aparag \SPA receives the excellent minister \ministre{Olivares}, with values
8/9/7.  These minister values supersede the current values of the Monarch (if
they are inferior). This Minister will last for a random length of Excellent
Minister, see \ref{eco:Excellent Minister}.
\aparag If \SPA monarch dies while the this event is still in effect, use the
minister values as a basis for rolling for the values of the new
monarch. Otherwise the monarch returns with its original values when the
minister dies and play continues normally.
\aparag \SPA may receive an additional Art manufacture of level {\bf 2} (if
available, and if \SPA wants so) placed according to normal rules, and also 2
additional \TradeFLEET levels placed in any eligible trade zone (even if it
had no \SPA commercial fleet counter before, and may be in different zones).
\bparag \SPA may now move the \RES{Cloth} \MNU without any drawback
(see~\ref{chSpecific:Spain:Cloth}).
\aparag The \ctz{Espagne} can no more be attacked by Pirates and Privateers in
the Mediterranean Sea. Attacks are to be made from the Atlantic.
\aparag The malus for foreign occupation for Stability improvement is changed
from \bonus{-3} to \bonus{-5} in national provinces only, and none for other
provinces (normal rule).

\aparag The reference level of \paysGenes in \CTZ \SPA is reduced to 0 if the
Spanish player chooses so.
\aparag From now on, \SPA can raise a second privateer that can go in any \STZ
of the \CCs{Atlantic} (in Europe or in the \ROTW).



\event{pIV:Unity HRE}{IV-2 (2)}{War for the Unity of the HRE}{1}{PBNew}

\history{alternative history}

\condition{Check the conditions in the given order until one is found true.}
\aparag If events \xref{pIV:Bohemian Revolt}, \xref{pIV:Augsburg Revocation}
or \xref{pIV:TYW} are happening now, do not mark off and re-roll.
\aparag If \ref{pIV:TYW} finished during the current period, mark off and roll
for one \REVOLT in each of the following countries: \AUT, and \FRA.
\aparag If there is a \GE, apply the \xnameref{pIV:HRE:GE vs Prussian
  Alliance}.
\aparag If \ref{pIV:TYW} never happened, apply the
\xnameref{pIV:HRE:Brandenburg vs Bavaria}.
\aparag Else, use \xnameref{pIV:Augsburg:Troubles HRE}.


\subevent[pIV:HRE:GE vs Prussian Alliance]{Revolt of Brandenburg and allies}

\phevnt
\aparag A Northern Alliance of countries of the \HRE is created. \PRUMin,
\paysSaxe and \payspalatinat are created anew and break free from the
\GE. They are allied.
\aparag \GE and \hab declare war to all those countries (and are controlled by
\SPA). The \GE is in Civil War.
\bparag \AUS has a free \CB against the whole Northern Alliance (to be used
immediately, or forfeited at the cost of {\bf 2} \STAB).
\aparag This countries are controlled by the first Protestant \MAJ in the list
that accepts the alliance: \SUE, \ENG, \HOL, \FRA, \POL.  It has a \CB to
enter the war. If it declines, the next country in the list has the
opportunity to do the same (and so on).  The \MAJ gains all the \MIN of the
Alliance in \EG.  If nobody enters the war, \SUE controls the Northern
Alliance (but is not involved).

\phdipl
\aparag \SPA can make a full intervention as an ally of \HAB.
% (JCD) What about HOL?
\aparag If they have not declined control of the Alliance, \FRA and \ENG (if
they are not Counter-Reformation) and \SUE can make a limited intervention in
the war alongside the Northern alliance.

\phpaix
\aparag If no \MAJ entered the war to control of the \MIN involved, they are
dealt with as one country for the peace in this war (except attempts of
Separate Peace), with a malus of \bonus{-4} to make peace.
\aparag A peace of level 3 or higher against the \MAJ in control (or the
Alliance itself if there is none) would dissolve the Alliance.
\aparag If the war ends and the Alliance is not dissolved, the \MIN are now
normal separate countries that are no more part of the \GE. The Alliance is
then dissolved.
\aparag Remember that, according to \ref{pIV:TYW:German Empire}, a peace of
level 3 or higher against the \MAJHAB may dissolve the \GE.  Conversely, any
Unconditional Peace against a country once part of the \GE forces is back in
the \GE.


\subevent[pIV:HRE:Brandenburg vs Bavaria]{War between \paysBrandebourg and
  \paysBaviere}

\phevnt
\aparag \paysBrandebourg declares a war to \paysBaviere.  \paysSaxe and
\payspalatinat are allied to \paysBrandebourg and declares also a war to
\paysBaviere.
\aparag \AUSMin declares a full war against the enemies of \paysBaviere.
\bparag \AUS has instead a free \CB to enter war as an ally of \paysBaviere,
and will lose {\bf 2} \STAB if it does not use it.

\phdipl
\aparag Each \MAJ that controls one of the involved countries may react as per
the usual rules to enter in limited intervention (only).
\aparag \SPA may make a limited intervention as an ally of the side of
\paysBaviere.

\phpaix
\aparag A test to begin a Religious War in HRE is made at the end of the first
turn of this war, with a \bonus{-2} modifier.  See \ref{pIV:TYW} for the
result of the test and the possibility of this Religious War, and the renewal
or not of the test on following turns.  If no such war occurs, peace can be
made as usual.



\event{pIV:War Persia Turkey}{IV-3 (1)}{War between Turkey and
  Persia}{1}{Risto}

\history{1606-1639}

\condition{This event has the same conditions and effects as \ref{pIII:War
    Persia Turkey}. It is nonetheless a different event (thus both can happen
  separately).}



\event{pIV:Persian Safavids}{IV-3 (2)}{Persian Safavids}{1}{PB}

\history{1590-1722 -- The high tide of Shah Abbas.}

\condition{}
\aparag This event is the same as \ref{pIII:Persian Safavids}.  If it did not
happen, apply immediately its effects. Apply additionally \ref{pIV:Persian
  Safavids:Fall Ormus}.
\aparag If it happened and main provinces of \paysperse are conquered,
activate a Persian Uprising as per the rules.
\aparag Else, apply the following events.

\phevnt
\aparag[Fall of Ormus]\label{pIV:Persian Safavids:Fall Ormus} \dipAT and
\dipFR status with \paysOrmus are immediately broken to \dipNR. This might
cause an Activation of \paysperse against a \TP in \paysOrmus.
\aparag[Conquest of Oman] \paysperse attacks \paysOman and that results in
breaking all diplomatic status of \paysOman. This applies also to military
\dipAT imposed by \PORmin (troops are redeployed).
\aparag[Submission of \granderegion{Afghanistan}]
\bparag \granderegion{Afghanistan} is no more part of the \paysMogol or
\paysAfghans (which is destroyed at this point), but submitted to
\paysperse. The Natives are used by \paysperse in this region.
\bparag As long as \paysPerse masters \granderegion{Afghanistan}, the Silk
resources of this region may be exploited by \provinceOrmus (usual concurrence
with \TP or \COL in \granderegion{Afghanistan}).
\bparag Persian units can go in \granderegion{Afghanistan} and have supply in
every provinces. But only \villeHerat and the European provinces of \paysperse
are supply sources.
\bparag \RUS and \TUR have \OCB against \paysperse as long as it owns
\granderegion{Afghanistan}.
\bparag \granderegion{Afghanistan} can be conquered later by \paysMogol, or
can become \paysAfghans again by \ref{pVI:India:Afghan Empire}.
\bparag \paysperse also loses the area in a losing Peace of level 2 or higher
(in regular or Overseas war) that has no other condition of peace. In Overseas
Wars, the occupation of a province without city counts as a province
occupied. In every war, the control of \villeHerat and its province counts as
for a Persian province.
\bparag When \paysperse loses the area, all the effects described here are
nullified.



\event{pIV:Portuguese Revolt}{IV-4 (1)}{National Revolt of the
  Portugal}{1}{Risto}

\begin{todo}
  Province Tanger should go to Morocco. Helper of POR should gain a predisio
  on Tanger + a \TP of POR in case of victory (no reannexion). HIS should not
  be able to attack if at war otherwise. Helper should be first in order at
  Methuen. Helper should be Catholic?

  Maybe swap Portuguese revolt with Alaouite dynasty and re-add Portuguese
  revolt as secondary event in pV (typically of WoSS which is four times in
  the table). The real war only started in 1660, the turning point between pIV
  and pV.
\end{todo}

\history{1640-1668}

\condition{Occurs only if \xnameref{pIII:POR Ann.:Portugal Annexed} is
  currently in effect.}
\aparag Else, if \ref{pIII:Portuguese Annexation} was never rolled for, do not
mark off and re-roll.
\aparag Otherwise treat as a \RD instead, with a \REVOLT in \SPA.

\phevnt
\aparag All effects of the \xnameref{pIII:POR Ann.:Portugal Annexed} are
cancelled and \paysPortugal returns to play as a minor country. \paysPortugal
receives the same provinces it had at the time of its annexation to \SPA
notwithstanding who currently owns such provinces. It also receives all
Portuguese \COL/\TP, missions, forts/fortresses, commercial fleets etc. that
are currently in Spanish hands.
% (Jym) Included \provinceTanger and other provinces given to SPA during
% annexation (JCD) Yes, why all of the provinces?
\aparag All non-Portuguese \COL in \continent{Brazil} receive a \REVOLT
\faceplus controlled by \paysPortugal. They can't extend outside the regions
of \continent{Brazil}.
\aparag All non-Portuguese troops inside its territories are removed as per
normal peace phase.
\aparag All Portuguese troops are removed from the map, as \paysPortugal is
initially at peace (keep the basic forces in the \ROTW where needed).
\aparag \ENG may accept \paysportugal in \EG; if \ENG declines, same to \FRA,
then to \SUE; if no country accepts, it remains neutral.

\phdipl
\aparag All players who are forced to cede provinces to Portugal by this event
receive a temporary free \CB to be used this turn.
\aparag Players who want to fight against Portuguese \REVOLT in their own \COL
have to declare an Overseas war against \paysPortugal and have a free \OCB to
do so.  Else, their \COL is freely given to \paysportugal (no loss of \STAB or
\VPs).
\aparag \SPA receives a free \CB that lasts until the end of the next period
and can be used multiple times.

\phpaix
\aparag The Portuguese Revolt in a \COL causes the loss of at most {\bf 1}
\STAB to each \MAJ.
\aparag Any \COL having 2 Portuguese \REVOLT \faceplus in it is immediately
annexed by \paysPortugal.
\aparag If \SPA uses its free \CB against \paysPortugal and wins an enforced
unconditional surrender over Portugal, it can reapply \xnameref{pIII:POR
  Ann.:Portugal Annexed}. All Portuguese possessions as they are now are
reannexed to \SPA as described there. Reannexation of Portugal as by
\numberref{pIII:POR Ann.:Portugal Annexed} is only possible in wars \SPA
started using its free \CB. In addition, \SPA gains a \Presidio in
\provinceTanger if the province was Portuguese.
\aparag If \SPA uses its free \CB against \paysPortugal, but does not win an
enforced unconditional surrender over it, the controller of \paysPortugal
receives 30 \VP when peace is made. This can occur several times.
refshort{pIII:POR Ann.:Portugal Annexed} is only possible in wars \SPA started
using its free \CB.  + \fphase In addition, \SPA gains a \Presidio in
\province{Tanger} if + the province was Portuguese.  \ephase If \SPA does use
its free \CB against \pays{Portugal}, but does not win an enforced
unconditional surrender over it, the controller of \pays{Portugal} receives 30
\VP when peace is made. This can occur several times.
\aparag Whatever the result of the war, if \ENG was intervening in the war, it
gains \provinceTanger if the province was Portuguese.



\event{pIV:Morocco}{IV-4 (2)}{Alaouite dynasty in
  \paysmaroc}{1}{PB+JymNew}

\begin{todo}
  Maybe here for giving back Tangier (except presidio?) to Morocco?
\end{todo}

\history{1631}
\dure{Until the end of the game}

\effetlong
\aparag \TUR has a malus of \bonus{-3} to diplomacy with \paysmaroc.
\aparag \paysmaroc loses its \corsaire counter.
\aparag Fidelity of \paysmaroc is now 10.



\event{pIV:Act Navigation}{IV-5}{Act of Navigation}{1}{RistoMod}

\history{1651}
\dure{until English defeat in a war caused by this event, or by event
  \ref{pV:Glorious Revolution}}

\condition{}
\aparag Can occur only if \ENG is currently \PROTANG. Otherwise re-roll.
\aparag Can occur only if \ref{pIV:English Civil War} has already occurred
(not necessarily ended). Otherwise re-roll.
\aparag \ENG can refuse the event, in which case it is marked off and \RD is
applied instead.

\phevnt
\aparag All non-English commercial fleet counters in \ctz{Angleterre} are
eliminated and \ENG receives 2 \TradeFLEET levels in \ctz{Angleterre} (up to 6
levels). All powers that lose their counters as a result of this, receive a
\CB against England until the end of current period.
\aparag From now on, only \ENG can place \TradeFLEET levels in
\ctz{Angleterre}.
\aparag From now on, all \MAJ have an \OCB against \ENG, usable once each
period.

\phadm
\aparag \ENG may ignore restriction of~\ref{chAdministration:Pioneering} for
this turn.

\phpaix
\aparag If a \CB against \ENG received through this event has been used by a
player, and if such a player wins at least a level 4 victory against \ENG, he
may renounce the effects of this event instead of any other peace conditions
(all the allies must agree with this as per normal peace procedure).
\aparag In such a case all non-English \TradeFLEET levels in \ctz{Angleterre}
lost due to this event are returned and \ENG \TradeFLEET in \ctz{Angleterre}
is reduced to 1 whatever its current level.



\event{pIV:Union Scotland}{IV-6}{Personal Union between England and
  Scotland}{1}{Risto}

\history{1603}
\dure{until \ref{pVI:Act Union} occurs or the union is dismantled by
  \ref{pIV:English Civil War}}

\condition{Can occur only if \paysecosse is at peace with \ENG.  Otherwise
  re-roll and do not mark off as played.}

\phevnt
\aparag \paysecosse becomes a permanent \VASSAL of \ENG whatever its current
situation.
\aparag If \paysecosse is currently at war, its opponent must immediately
either accept a white peace with it, or declare war to \ENG with a free \CB.
Normal call for allies can be made for such a war at this point.



\event{pIV:English Civil War}{IV-7 (1)}{English Civil War}{1}{RistoMod}

\history{1642-1648}

\condition{}
\aparag If \monarque{Elisabeth I} rules in \ENG, do not mark off and reroll.
\aparag If \ENG is \CATHCR or \CATHCO and period is III, do not mark off and
reroll.
\aparag If \ENG is currently at war, it offers an immediate white Peace or
Armistice to all its enemies, and will renew the offer at the end of each
turn.
\bparag This event is activated as soon as \ENG is at peace or in armistice
with every other \MAJ (\MIN automatically accept an Armistice).

\phevnt
\aparag A Religious Civil War (\ref{chDiplo:Religious Civil War}) erupts in
\ENG between the \parl and the \paysRoyalists.

\aparag[Which side is played ?]
\bparag If \ENG is \CATHCR by choice in \ref{pI:Reformation2}, the player
keeps playing \royal.
\bparag If \ENG is \CATHCO, or \CATHCR by forced conversion, the player
chooses the side he will play.
\bparag If \ENG is Anglican or Protestant, the player plays \parl.
\bparag The other side will be the Rebels; the \MAJ controlling the Rebels
will be called \REB.
\bparag\label{pIV:ECW:Monarchs} The \royal are governed by the English Monarch
before the event (and he can be used as a general). The \parl are ruled by a
Monarch \monarque{Parliament} of values 5/8/8 that makes no test of Survival.
It gives a bonus of \bonus{+2} to the rolls for all administrative actions
(except exceptional taxes). It may not be used as a general.

\aparag If not played by \ENG, \parl is played by the first Protestant country
in the list: \HOL, \FRA, \SUE or else by \POL.

\aparag Three \REVOLT are rolled for in \ENG. These \REVOLT are hostile to
both sides and controlled by \TUR.

\aparag If not played by \ENG, \royal are played by the first \CATHCR \MAJ in
the list: \SPA, \FRA, \HOL, \VEN else by the first Catholic \MAJ in the list:
\SPA, \FRA, \VEN, \SUE, \POR, \POL.  Failing that it is played by \RUS.

\aparag[Initial position]
\bparag \royal control \provinceMidlands, \provinceCornwall, \provinceDurham
and 1d10/3 (round down) provinces adjacent to \provinceMidlands (to be chosen
by their controller). Add \bonus{+2} to the roll if \ENG was
Counter-Reformation or Protestant. \royal controls all (non-revolted)
provinces in \regionIrlande.
\bparag The \parl control all other (non-revolted) provinces in \ENG.
\bparag \royal and the \parl receive up to the equivalent of basic land forces
of \ENG; the Rebels take the forces before (so they can take everything is
there is not enough). Additional forces are removed.
\bparag The Rebels add 1\LD (Veteran) in any controlled province, and 1\LD
(Conscript) in \provinceDurham (if \royal) or \provinceWessex (if \parl).
\bparag \ENG loses 1 \ND, and the rest is controlled by the \parl.
\bparag All named leaders in play are controlled by the \parl.

\aparag[Economic consequences]
\bparag \ENG loses one-third of its treasury, and at least 50\ducats (this
might cause a Bankruptcy).
\bparag Two \PIRATE\faceplus are placed in \CTZ England.
\bparag All \TP, \COL, \TradeFLEET, etc., remain under the control of \ENG.

\aparag If \ref{pIV:Union Scotland} is in effect, apply \xnameref{pIV:ECW:War
  Scotland} in addition.

\phdipl
\aparag If \ENG was \CATHCR, \SPA if also \CATHCR may make a full intervention
on the side of the \royal.
% (JCD) I added /Anglican there.
\aparag If \ENG was \PROTANG, \SPA if \CATHCR may make a limited intervention
on the side of the \royal.
\aparag If \ENG was not \CATHCR, \HOL if Protestant may make a limited
intervention on the side of the \parl.

\phadm
\aparag[Reinforcements]
\bparag The Rebels roll for reinforcement with offensive status, or naval
status at \bonus{-3}, during all the war. On the first turn, they roll for
offensive with a modifier of \bonus{+4} if \ENG was Protestant, \bonus{+2} if
it was \CATHCR or \PROTANG, of \bonus{0} otherwise (\CATHCO).
\bparag On following turns, they receive a modifier of \bonus{+1} for every 2
provinces they control, with a maximum of \bonus{+4}.
\bparag If the Rebels are the \parl, they can take up to 2 \LD as \ND instead.
\bparag The Rebels have as many counters as \ENG available.
\aparag \ENG uses normal purchase rules, except that its purchase limits are
doubled during the Civil War.
\aparag The \royal receive the general \leaderwithdata{Rupertroy} on the first
turn of the war; he will last 7 turns.% He has a double usage as an admiral.
\aparag The \parl receive the general \leaderwithdata{Cromwell} at the end of
the first turn of the war (before the Peace Segment).  He will last for the 5
following turns. The \parl benefits from a Military Revolution at that point
(\textit{The New Model Army}, see rules \ruleref{chAdministration:Military
  Revolutions}, that is to take immediately any Land Technology available at
most in 4 turns, and in the mean time, is raised to the highest Technology
available at that time).
\aparag The \royal have the Land Technology of \ENG at the beginning of the
event. If played by \ENG, they may raise their technology as per usual rules;
else their Land technology is raised by {\bf 1} each turn of the war beginning
with the second.

\phpaix
\aparag The Civil War ends only if either party achieves both following
conditions:
\bparag Military control of \province{East Anglia} and five other English
National provinces with at least 3 ports.
\bparag Elimination of all enemy army counters, or at least two major
victories against them.
\aparag If the \royal win, \ENG is ruled by its previous Monarch and becomes
\CATHCR (exception: if \ENG was \CATHCO, it remains so). \leaderRupertroy is
kept as a general; land technology of \ENG is at the level reached by the
\royal.
% (JCD): I added (or remains)
\aparag If the \parl wins and \ENG was Catholic or \PROTANG, \ENG becomes (or
remains) \PROTANG.  It is ruled by the \monarque{Parliament}
(see~\ref{pIV:ECW:Monarchs}).
\bparag If \leaderCromwell is in play at the end of the war, it becomes Lord
Protector of the Kingdom, and is an English Monarch that raises the values of
the \monarque{Parliament} to 8/8/9. His Reign is to last the number of turns
remaining for the general (of the initial 5 turns).  A test of survival has to
be done for him. As long as his reign continues, \ENG gains a free maintenance
of one \ARMY\faceplus.
\bparag When \leaderCromwell dies, or at the beginning of the sixth turn after
the end of the Civil War, apply \ref{pIV:English Restoration} as one of the
event of the turn.
\bparag \leaderRupertang becomes an admiral only, kept by \ENG as one of its
own.

\aparag If \ENG was Protestant (not \PROTANG) and the \parl wins, \ENG remains
so. It is rules by the \monarque{Parliament} (see~\ref{pIV:ECW:Monarchs}).
\bparag If \leaderCromwell is in play at the end of the war, he stays as a
general only. \leaderRupertang is not used by \ENG.
\bparag At the beginning of the sixth turn after the end of the Civil War,
roll for a new Monarch on columns 9 for the three values. \ENG is ruled by a
Protestant Republic lead by some strong Lord Protector of the Commonwealth
(represented by the Monarch).
\aparag Regardless of the winner, \leaderMonck and \leaderBlake are admirals
from now on.


\subevent[pIV:ECW:War Scotland]{War with Scotland}

\phevnt
\aparag \paysecosse declares war against the \royal and becomes neutral.
\paysecosse is controlled by \FRA, but no allies can ever take part in this
war. This declaration of war does not trigger a truce in the civil war as per
normal rules.

\phadm
\aparag \paysecosse rolls for reinforcements in offensive status.  It has a
minor general added to its base forces.

\phmil
\aparag Scottish units may not enter England during the first 2 rounds of
their war.
% (Jym): What? (Pierre) I do insist
\aparag On the turn following their entrance in England, \royal gain as added
reinforcements \leaderwithdata{Montrose}, 2 \LD and control of one mountainous
province in \paysecosse of their choice.

\phpaix
\aparag When the Civil War ends, \ENG may decide to continue an on-going war
against \paysecosse (it will be counted as the second turn of the war).
\aparag If \ENG (\royal or, after the end of the Civil War, the \parl) scores
an enforced unconditional victory over \paysecosse during this war, Scotland
is restored to permanent \VASSAL of \ENG as per \ref{pIV:Union Scotland}.  In
all other cases, it reverts to a normal minor.



\event{pIV:English Restoration}{IV-7 (2)}{The Parliament and the English
  Kings}{1}{PBNew}

\history{1660}

\condition{May not happen if the \xnameref{pIV:English Civil War} is not
  finished yet. Re-roll and do not mark off.}

\phevnt
\aparag If \ENG is \PROTANG or \CATHCO, apply \xnameref{pIV:ENG
  Restoration:Restoration}.
\aparag If \ENG is \CATHCR, apply \xnameref{pIV:ENG Restoration:Asking
  Reforms}.
\aparag If \ENG is Protestant, apply \xnameref{pIV:ENG Restoration:War Against
  Puritanism}.


\subevent[pIV:ENG Restoration:Restoration]{The Restoration of the English
  Kings}

\phevnt
\aparag \ENG has the choice of crowning now the Pretender (an Heir of the
Monarch overthrown by \ref{pIV:English Civil War}); if not, \ref{pV:Glorious
  Revolution} is applied now (with worsened consequences).
\aparag If the Pretender is crowned, roll for his values using those of the
Monarch overthrown by the \nameref{pIV:English Civil War}. The effects of
\monarque{Cromwell} or the \monarque{Parliament} are ended (and
\leaderCromwell is put out of play).
\aparag \ENG receives the general (also usable as admiral) \leader{Duke of
  York} that will stay for 5 turns (note: he actually became king in 1685 but
we choose to ignore this and separate the general from the king).


\subevent[pIV:ENG Restoration:Asking Reforms]{The Parliament asks for more
  reforms}

\phevnt
\aparag \ENG has to choose one of the 2 following attitude.
\aparag[Reforms granted]
\bparag \ENG becomes Anglican. It loses {\bf 2} \STAB and rolls for 2 \REVOLT
.
\bparag \SPA, if \CATHCR, has a free \CB against \ENG.
\aparag[Refusal]
\begin{todo}
  ``CHANGE'' (Pierre's notes).
\end{todo}
\bparag Apply \ref{pV:Glorious Revolution} now (with worsened consequences).


\subevent[pIV:ENG Restoration:War Against Puritanism]{Civil War between
  Protestants and Puritans}

\phevnt
\aparag \ENG is now in Civil war (\ref{chDiplo:Religious Civil War}) between
two sides: the (Puritans and Calvinist) Rebels (possibly with Orange
Partisans) and the (Protestant) Royalists. Catholics rebel against both sides.
\bparag The Rebels are controlled by a Protestant \FRA, or \HOL (or \SUE if
there is no \HOL). They use the \paysroyalistes counters.
\bparag The (Protestant) Royalists are controlled by \ENG and use \ENG
counters; all \ENG leaders are with them.
\aparag Four Rebel \REVOLT are rolled for in England (re-roll until in English
owned territory). They control all the fortresses.
\bparag A Rebel \ARMY \facemoins and a \LeaderG are placed in one of these
provinces.
\aparag Catholic \REVOLT \faceplus are placed both in \provinceConnacht and
\provinceMumhan and the \REVOLT control both fortresses.

\phdipl
\aparag The controller of the Rebels have a \CB against \ENG to make a limited
intervention against \ENG this turn, that can become a full intervention on
the second turn.

\phadm
\aparag The Rebels roll for reinforcements in offensive or naval status (but
with \bonus{-2} for naval).
\aparag All reinforcements must be placed in a province with existing rebel
units, allied units, or controlled cities (\REVOLT are not enough). If none,
no reinforcements are received.

\phpaix
\aparag Peace is determined with usual rules except that:
\bparag The Rebels surrender unconditionally if they have no forces nor
\REVOLT left (fortresses do not count).
\bparag If the English King is overthrowned by \REVOLT, it also surrenders
unconditionally to the Rebels and their controller.
\aparag If the Rebels win, \ENG will have a \terme{Dynastic Crisis} next turn,
and loses 50 \PV.
\bparag \eventref{pVI:Act Union} is broken. If it did not happen yet, may
occur later.
\aparag If the Rebels win unconditionally and their controller was involved in
full intervention, additional consequences are:
\bparag \ENG makes a mandatory Dynastic Alliance with the controller of the
Rebels and must give a \COL or \TP as dowry.
\bparag \ENG makes a mandatory offensive alliance with the controller of the
Rebels for 2 turns. It cannot declare war against it (except with \CB from
events; then, the alliance has to be broken with the usual cost in \STAB).
\bparag \eventref{pVI:Act Union} is broken. If it did not happen yet, it may
not occur later.



\event{pIV:London Stock Exchange}{IV-8 (1)}{Creation of the London Stock
  Exchange}{1}{Risto}

\history{1554}

\condition{\ENG chooses to apply this event or \ref{pIII:East Indian
    Company}. Mark the one that is chosen.}

\phadm
\aparag \ENG may ignore restriction of~\ref{chAdministration:Pioneering} for
this turn.

\effetlong
\aparag \ENG can from now on lend 250\ducats per turn to other countries.
% in the Diplomacy phase, plus 100\ducats during the turn (instead of
% 100\ducats plus 50\ducats).

\aparag From now on, \ANG receives a bonus for its International Loan rolls
and Bankruptcy rolls.

\aparag From now on, \ANG is more resilient to exceeding limits in \MNU.

% \begin{oldcompta}
%   \aparag From now on \ENG receives a bonus equal to its \DTI to all die-rolls
%   on international loan amount and interest (not length) in the loans table
%   % (Jym) Table just said "international" and usually more up to date
%   \aparag \ENG is also more resistant to Bankrupt and more tolerant to
%   trespassing of commercial limits.
%   % (Jym) According to the rule of chapter 3, it is actually less resilient
% \end{oldcompta}



\event{pIV:Amsterdam Stock Exchange}{IV-8 (2)}{Creation of the Amsterdam Stock
  Exchange}{1}{Risto}

\history{1608}

\condition{This event is the same as \ref{pIII:Amsterdam Stock Exchange}.}



\event{pIV:Dutch Colonial Dynamism}{IV-9}{Dutch Colonial Dynamism}{3}{Risto}

\condition{\HOL chooses to apply this event or \ref{pIII:VOC}. Mark the one
  that is chosen.}

\phevnt
\aparag \HOL receives an additional commercial fleet level to any eligible
\STZ zone in \ROTW map (even if none existed before).

\phdipl
\aparag For this turn \HOL receives a bonus of \bonus{+2} to all diplomatic
actions made on countries from the \ROTW map.

\phadm
\aparag \HOL receives an additional and free strong \TP placement attempt.
\aparag For this turn \HOL receives a bonus of \bonus{+1} to all
administrative actions made in \ROTW map.
\aparag \HOL may ignore restriction of~\ref{chAdministration:Pioneering} for
this turn.



\event{pIV:Liberum Veto}{IV-10 (1)}{Liberum Veto}{2}{PB}

\history{1652}

\phevnt
\aparag The conditions of the \textit{Liberum Veto} (see
\ref{chSpecific:Poland:Liberum Veto}) are now enforced.

\phadm
\aparag If \POL is at peace after the diplomacy phase of this turn, it loses 2
in \STAB.

\effetlong
\aparag Each time a new dynasty is elected in \POL, it can decide to impose
Absolutism in the Republic. This decision is made at the phase of the monarch
survival (before the events) ; it causes an additional event, \ref{pIV:Polish
  Civil War}. There can be no further additional event at this turn.



\event{pIV:Great Elector}{IV-11}{The Great Elector Friedrich-Wilhelm of
  Prussia}{1}{PB}

\history{1640-1688}

\phevnt
\aparag \leader{Friedrich-Wilhelm} is now the ruler of \paysBrandebourg and a
general [A.2.3.3]. He will last 6 turns.  The basic force of this country is
now one \ARMY\faceplus, 1 \LD, 1 \fortress and 1 general. Its counter limits
are 2 \ARMY and 5 \LD.  The fidelity of the country is 9 from now on.
\aparag \paysBrandebourg claims the \region{Duche de Prusse}:
\provincePreussen, \province{Ost Pommern} and \provinceMemel.
\bparag Minor countries cede those provinces immediately to \paysBrandebourg.
\bparag Major countries have the possibility to cede them or not (and lose \PV
normally).

\phdipl
\aparag If a country declares a war against a \MAJ that owns one of those
territories, he can ask for a full intervention of \paysBrandebourg as an ally
(which is put in \EW immediately).
\aparag If \POL owns provinces of the \region{Duche de Prusse}, it can cede
all of them to \paysBrandebourg in exchange for an alliance with
\paysBrandebourg. \POL does not lose the \PV.  \paysBrandebourg signs a white
peace, is put in \EW of \POL and may be called as ally by \POL in any war it
is currently involved in.
\bparag \POL is now the first power in the list of preference of
\paysBrandebourg.

\phpaix
\aparag In any war involving \paysBrandebourg, only this country may annex a
province of the Duchy of Prussia if its alliance wins; if its alliance wins,
it asks for one province or refuses the peace (so that the other powers must
break their alliance and make a separate peace).



\event{pIV:Oxenstierna}{IV-12 (1)}{Oxenstierna}{1}{PBNew}

\condition{Same event as \ref{pIII:Oxenstierna}.}



\event{pIV:Union Poland Sweden}{IV-12 (2)}{Union between \paysmajeurPologne
  and \paysmajeurSuede}{1}{PB}

\condition{Same event as \ref{pIII:Union Poland Sweden}.}



\event{pIV:Torstensson}{IV-13 (1)}{Torstensson's War}{1}{PB}

\history{1643-1645}

\phevnt
% (JCD) TODO: adapt to DANdan
\aparag \SUE has a mandatory free \CB against \paysDanemark at this turn (even
if their are allied in another war).
\aparag If \SUE refuses the \CB, it loses 2 \STAB.



\event{pIV:Swedish Nobles Unrest}{IV-13 (2)}{Agitation of the Swedish
  Nobles}{1}{PBNew}

\history{1650's}

\phevnt
\aparag If \SUE is Catholic and \ref{pIII:Religious War Sweden} did not happen
yet, it occurs now.
\aparag If \SUE is \PROTRIG, roll for two \REVOLT in \SUE.
\aparag If \SUE is \PROTTOL and at war, rolls for one \REVOLT in \SUE, \SUE
loses 2x \STAB and its monarch changes (abdication of the Queen Kristin).
\aparag If \SUE if \PROTTOL but not at war, roll for 4 \REVOLT in \SUE (do not
place the \REVOLT if not inside \SUE, but do not reroll either) and a Revolted
\ARMY appears in one of those provinces with a general.
\aparag The resulting \REVOLT are controlled by \SPA.


\event{pIV:La Rochelle}{IV-14}{Revolt of La Rochelle}{1}{RistoMod}

\history{1626}
\dure{Until the suppression of the \REVOLT in \provincePoitou and the conquest
  of \ville{La Rochelle}.}

\condition{}
\aparag If \ref{pIII:FWR} is not finished yet, do not mark off and re-roll.
\aparag If the owner of \provincePoitou is Protestant, roll on its Revolt
table and place a \REVOLT\Faceplus if this is a Catholic province, and a
\REVOLT\Facemoins otherwise. The event is marked off and considered as played.

\phevnt
\aparag Place 2 \REVOLT \faceplus and a \LD in \provincePoitou.  Retreat all
other units from the province.
\bparag Roll for two Revolts in \FRA. Place a \REVOLT\Facemoins if the
province is Protestant (or mixed if \FRA is \CATHCR) and nothing otherwise.
% \aparag Place 2 Revolts \facemoins in other protestant provinces (randomly),
% or in Protestant or Mixed provinces if \FRA is \CATHCR.
\aparag The fortress of La Rochelle is controlled by the Rebels and upgraded
to the highest level available to the owner of the province.
\aparag Place a \PIRATE\faceplus in \CTZ of \FRA.
\aparag The Rebels/\REVOLT are controlled by \ENG, or \FRA if \ENG owns the
province. This war is a Religious Civil War (see \ref{chDiplo:Religious Civil
  War}) between Protestants and Catholics and normal Foreign interventions are
allowed.

\phadm
\aparag As long as the event lasts, the owner of \provincePoitou has a malus
of \bonus{-1} to all its administrative actions in the \ROTW.

\phmil
\aparag If a Foreign power enters a land province in the power at war that is
not \provincePoitou during its intervention, it loses 1 \STAB.
\aparag If the owner of \provincePoitou is \FRA and \ministre{Richelieu} is in
the game, consider that the port of the fortress is under blockade if a french
army besieges it.

\phpaix
\aparag If the fortress is controlled by the Rebels, it counts has a \REVOLT
\facemoins for the loss of \STAB by the owner of \provincePoitou due to
\REVOLT .
\aparag The owner of \provincePoitou may cede the province to the controller
of the \REVOLT , losing 30 \PV for doing this.
\aparag The controller of the \REVOLT earns 5 \PV at the end of each turn that
the Rebels exist (\REVOLT or fortress in \provincePoitou).



\event{pIV:Richelieu}{IV-15}{Richelieu}{1}{RistoMod}

\history{1624-1642}
\dure{as long as \strongministre{Richelieu} remains the excellent minister}

\condition{}
\aparag If \ref{pIII:FWR} is not finished yet, do not mark off and re-roll.
\aparag \FRA can refuse this event if it so wishes. In that case mark-off a
played.
\aparag \FRA can freely remove \ministre{Richelieu} from office at the end of
any following monarch survival phase and the event terminates.

\phevnt
\aparag \FRA receives automatically the excellent minister
\ministre{Richelieu}, with values 9/8/7.  These minister values supersede the
current values of the Monarch (if they are inferior). This Minister will last
for a random length of Excellent Minister, see \ref{eco:Excellent Minister}.
\aparag \FRA gains one level of \TradeFLEET in any \CTZ or \STZ of its choice.

\phadm
\aparag As long as \ministre{Richelieu} lives, \FRA has a bonus of \bonus{+2}
to any die-roll for External Administrative Actions and to improve its \FTI.
\aparag \FRA may ignore restriction of~\ref{chAdministration:Pioneering} for
this turn (only).

\phinter
\aparag When \FRA monarch dies, his successor is \monarque{Louis XIV}.



\event{pIV:Fronde}{IV-16}{The Fronde}{1}{PB}

\history{1648-1653}
\dure{3 turns or as long as \monarque{Louis XIV} is not adult (whichever is
  the latest). In any case, it ends after the turn of revolts.}

\phevnt
\aparag If \monarque{Louis XIV} has not already been king of \FRA, the current
king of \FRA dies and the new king is \monarque{Louis XIV}.
\bparag \monarque{Louis XIV} has values 7/6/9, scheduled to last 12 turns and
starts as a baby.
\bparag He'll become adult at the beginning of the fourth turn of reign, thus
ending the event.
\aparag If due to \xnameref{pIV:Richelieu}, \ministre{Richelieu} was still in
charge, then during the first two turns of reign of \monarque{Louis XIV},
\ministre{Mazarin} will be minister with values 7/8/7.
\aparag If \monarque{Louis XIV} is already king, or if his reign is already
finished, then the event lasts for 2 turns.

\phdipl
\aparag Until the end of the event, \FRA may only offer a white or losing
peace to all minors, and peace based on the peace differential to each major
countries, with a maximum level of 1 in the favour of \FRA.
\bparag At each turn, \FRA offer and cannot refuse Armistices with opponents.
\bparag Neutral minor countries always accept that peace.
% \bparag Other countries (either majors or minors) choose their attitude
% freely toward that proposal: accept, refuse or armistice.
\bparag At the third turn of the event, if \ministre{Mazarin} is minister,
then major countries cannot refuse an armistice.
\aparag \FRA may not declare war as long as the event lasts (except
\xnameref{pIV:TYW} and \xnameref{pV:WoSS}).
\aparag If, at the end of a diplomacy phase, \FRA is not at war (don't count
armistices), the Fronde happens.

\tour{Turn of the revolts}

\phdipl
\aparag Half of French troops in Europe become rebel. \FRA choose a stack of
troops staying loyal, thus taking up to half the total number of \LD (rounded
down). The rest becomes the troops of the Fronde.
\bparag If in play, \leaderConde becomes leader of the Fronde. Otherwise, a
randomly chosen general (a named one if there is one in play) becomes leader
of the Fronde.
\aparag The Fronde is controlled by a country currently at armistice with
\FRA. If none exists, the order of priority to control the Fronde is: \HIS,
\ANG, \HOL, \AUS, \POL, \RUS, \SUE, \TUR.
\aparag Naval forces, admirals, everything in the \ROTW as well as
administrative counters (\MNU, \ldots) stay in the control of \FRA.

\phadm
\aparag \FRA collects neither land nor vassals income this turn. \FRA does get
other incomes.
\aparag The Fronde rolls for reinforcements with offensive attitude and no
modifier.
\aparag No side may get reinforcements such that its total force (in Europe)
is above the basic force of \FRA for the current period.

\phmil
\aparag Countries in armistice with \FRA can enter the civil war on the side
they want.
\aparag Fleet of \FRA may stay in ports controlled by the rebels without
penalties.
\aparag Except for the capital of \FRA, fortresses in France are friendly to
both sides.
\bparag A province is controlled by one side if it has an army in the province
and there is no enemy troop besieged in the fortress.
\bparag Other provinces are considered friendly to both sides for supply or
movement cost.
\aparag The capital of \FRA is always controlled by the loyalists until the
Fronde takes the fortress.

\phpaix
\aparag The side controlling the capital of \FRA at the end of turn wins.
\bparag If the Rebels win, \monarque{Louis XIV} (and \ministre{Mazarin}) is
overthrown. During the next turn, there will be a dynastic crisis in \FRA. The
player controlling the Fronde gains 10 \VP.
\bparag In any case, both the loyalist and Fronde's units become units of \FRA
as soon as the event is finished.



\event{pIV:Times of Troubles}{IV-17 (1)}{The Times of Troubles in
  Russia}{1}{PB}

\history{1605-1613}

\condition{}
\aparag If \monarque{Ivan IV} is not dead yet, do not mark-off and re-roll.
\aparag If \RUS chose \terme{Religious Tolerance}, mark off and use \RD
instead.
\aparag If \RUS is at war, the event is pending. It will activate at the
beginning of the first turn where \RUS is at peace and a roll of 6 or higher
is obtained on 1d10.

\tour{Turn 1}

\phevnt

\aparag The Russian monarch dies and is replaced by \monarque{Boris Godunov}.
His values are 5/8/4 and he will reign 5 turns; he is a general
\leaderwithdata{Godunov}.

\aparag \RUS is now in Religious Civil War (see \ref{chDiplo:Religious Civil
  War}). Rebels are Catholic; loyalists (\RUS) are Orthodox.

\aparag Roll for 6 \REVOLT in Russia. Only provinces actually in \RUS revolt,
other rolled-for are ignored. The \REVOLT are controlled by \POL.

\aparag Rebels gain one \ARMY\faceplus in one province in \REVOLT, and the
control of the city.

\aparag Rebels own any revolted province with no Russian armies in there
(except \province{Moscou}) and provinces they control. These provinces are
their supply sources.

\aparag \RUS owns all non-revolted provinces they control. They are their
supply sources.

\aparag All other provinces are disputed. Supply of both sides may cross those
provinces if there is no enemy force within.

\phdipl

\aparag During the event, \RUS may ask for help of \SUE. The condition is the
cession of one Russian province to \SUE; if this province is revolted, it
becomes Swedish only when it is no more revolted and its is controlled by \RUS
or \SUE. During the rest of the event, this province (even Swedish) can be
entered and attacked by all belligerents.

\bparag If an intervention of \SUE is agreed upon, \SUE has to commit at least
4 \LD in Russia, following the conditions of limited intervention.  \SUE can
not withdraw any force sent in Russia.

\aparag Major countries may make \terme{Foreign Intervention} in this war.

\phadm

\aparag Rebels receive offensive reinforcements at each turn, using the
provinces they own.

\aparag Rebels have the general \leaderDmitry (until he dies) for 5 turns.

\phpaix

\aparag See the explanations hereafter, valid for all turns.

\tour{Turn 2 and afterwards}

\phevnt

\aparag \monarque{Boris Godunov} has a malus of \bonus{+3} to his survival
roll. If he dies, a period of anarchy follows and \RUS has values 3/3/3 as a
monarch. On the next turn, \monarqueRomanov (in fact Fyodor and Michael) is
the new monarch 6/5/6; as this monarch represents the whole family, do not
roll for his survival (it is automatic).

\aparag if \monarque{Boris Godunov} is dead (on this turn or a previous one),
\leaderDmitry also rolls for survival with a \bonus{+3} as sole modifier
during the event.

\aparag As long as the event continues, roll for 3 \REVOLT in \RUS (that occur
only is in a Russian owned province).

\phdipl

\aparag \POL may make a full or limited intervention as ally of the Rebels. It
has a \CB to do so, or a free \CB is \leaderDmitry is alive. This
intervention is not affected by excessive foreign intervention.

\aparag If \POL was involved in this war on the previous turn and \SUE is
making an intervention allied to the loyalists, \POL may generalise the war
with a free \CB in a full war between \SUE and \POL. This does not change the
terms of their respective interventions in the Civil War.

\phpaix

\aparag \REVOLT in provinces that are controlled or occupied by \POL do not
extend and do not count for the conditions of victory of this event.

\aparag If half (round-up) of the Russian national provinces are in \REVOLT ,
\monarque{Boris Godunov} is overthrown and killed with no further
consequences.

% TODO exception to ??

\aparag A side fulfils the military condition of victory if it won a major
victory against the enemy or if it controls all cities in national provinces,
or if the enemy (not their foreign allies) has no \ARMY left.

\aparag The event ends as a victory for the Rebels or the Loyalists under the
following conditions.

\bparag Rebels win if \monarque{Boris Godunov} is dead and they control
\villeMoscou and they fulfil the military condition of victory; or they win if
\monarque{Boris Godunov} is dead and Loyalists surrender willingfully.

\bparag The Loyalists win if all the \REVOLT are eliminated in owned national
provinces and they fulfil the military condition of victory.

\bparag When the victory is obtained, all the \REVOLT and the Rebel armies are
destroyed.

\bparag The intervention of \SUE ends; \RUS has now a free \CB (one use)
against \SUE until the end of the period.

\aparag If the Loyalists win, \leaderDmitry is eliminated.

\bparag If \monarque{Boris Godunov} is alive, he is now legitimate ruler of
Russia. He has now values 6/8/5.  \RUS gains 10\PV.

\bparag If he is not, the new ruler is \monarqueRomanov for 5 turns, with
values 6/5/6. Russian \STAB is increased by 1.

\aparag If the Rebels win, \monarque{Boris Godunov} is eliminated.

\bparag If \leaderDmitry is alive, he becomes tsar \monarqueDimitri with
values 4/7/5 (and the turns left). \RUS loses 3 in \STAB.  If \POL is still
intervening in the war, \RUS is now in mandatory defensive alliance with \POL
during \monarqueDimitri's reign. In addition, \POL gains one province in \RUS
that it currently controls or occupies (its choice).

\bparag If \leaderDmitry is dead, the new ruler is \monarqueRomanov for 5
turns, with values 6/5/6. Russian \STAB is decreased by 2.  If \POL is still
intervening in the war, it gains one province in \RUS that it currently
controls or occupies (its choice).

\bparag In both cases, \POL gains 10 \PV and signs a white peace with \RUS.



\event{pIV:Revolt Cossacks}{IV-17 (2)}{Revolt of the Cossacks}{1}{PB}

\history{1654-1667}
\dure{until the end of the wars caused by the event.}

\condition{If the religious attitude of \POL is Tolerance of Orthodoxy, the
  event does not occur. Mark off and play \RD instead.}

\tour{Turn 1}

\phevnt
\aparag One province of \paysUkraine belonging to \POL (if none, belonging to
\HAB) secedes and create the new minor \paysukraine. The province is
\provincePoltava if available, else, the closest to this one (chosen by the
new protector or controller of \paysUkraine). Units of other countries inside
are immediately redeployed.
\aparag The new minor is a special \VASSAL of its protector. No diplomacy is
allowed on it.
\bparag The protector stops being protector if it declares war to
\paysukraine. The next possible protector in the list becomes the new
protector.
% (Jym) Originally also if does not come to help, but vassal means war to the
% minor is war to the major, so no way not to help
\aparag \paysUkraine never makes a separate peace without its protector and
must be included in the same peace treaty.
\aparag Possible protectors are (in order): \POL (if Orthodox), \RUS, \TUR,
\POL (if not Orthodox). If there are no (more) protectors, \paysUkraine
becomes a normal minor country.

\phdipl
\aparag \POL has a free \CB against \paysUkraine if it loses at least one
province during the formation of that country.
% (JCD): the event does not happen if \POL is orthodox! removing
% \aparag If \POL is Orthodox and does not declare the war, roll for two
% \REVOLT in \POL and ignore the rest of the event.
\aparag If \paysUkraine (as a special Polish \VASSAL) owns a province of
\paysCrimee (a province with a Crimean shield, even blurred), then \POL may
ask for a limited intervention of \paysCrimee in this war.
\bparag This does not change the diplomatic status of \paysCrimee nor its
controller. \paysCrimee is played by its usual controller decided by the usual
rules.
\bparag If \POL wins after an intervention of \paysCrimee, it must give one
province back to it.

\phadm
\aparag If \POL is at war against another \MAJ during the event, \HAB can make
a limited intervention as an ally of \POL.

\effetlong
\aparag \ref{chSpecific:Poland:Polish Ukraine} is no more valid.

\tour{Turn 2 and after}

\phevnt
\aparag If \POL is at war against \paysukraine, \SUE has a free \CB against
\POL.
\bparag If \SUE is at war against \POL, \RUS has a free \CB against \SUE (can
be used in reaction).
\bparag If \RUS uses this \CB and \paysdanemark is either inactive or already
at war with \SUE, then \paysdanemark is put in \EG of \RUS and enters war
against \SUE (if not already at war).

\phpaix
\aparag Normal rules for peace apply, except that allies of \POL cannot annex
provinces of \paysukraine that they didn't own before the event.



\event{pIV:Extension Moghol}{IV-18}{Extension of the Moghol Empire}{2}{PB}

\history{1635-1638 / 1653-1657}

\phevnt
\aparag If the non-European minor country \paysMogol does not exist, it is
created now. Its ruler is now \leader{Grand Moghol} (if period is IV or later,
it replaces \leaderAkbar if he was in play).
\aparag The \paysMogol will try to invade \textbf{3} regions during the turn,
according to \ref{pII:Mughal Expansions}.
\aparag Even if the country has no region after the invasions, it still exists
(and can gain provinces with new events).
\aparag \granderegionBengale and \granderegionKarnatika become rich region,
with 2 resources of each kind shown on the map (instead of 1).



\event{pIV:Wars India}{IV-19}{Wars in India}{2}{PB}

\history{1631-1635 / 1656-1659}

\phevnt
\aparag If it was still existing, minor country \paysVijayanagar is destroyed
(by internal fights).  Every \TP (not \COL) that is in the minor country
\paysVijayanagar at the time of its disappearance will face an attack by
Natives that are activated against every country this turn.
\aparag If \paysVijayanagar had already been destroyed, choose randomly 2 \TP
and/or \COL in \continent{India} that will be attacked by the Natives in the
region, due to internal strife in India.
\aparag \granderegionKarnatika has from now on 2 \RES{Spices} and 2
\RES{Products of Orient} available instead on 1 (if not yet done).
\aparag If the \paysMogol exist, they invade one province with a modifier of
\bonus{-2}, the next in the list according to \ref{pII:Mughal Expansions}.



\event{pIV:Revolt Singala}{IV-20}{Revolts in
  \granderegionSingala/\granderegionFormose}{2}{PB}

\history{1630}

\condition{If there is no \TP/\COL in \granderegionSingala nor
  \granderegionFormose, do not mark off and re-roll.}

\phevnt
\aparag Choose randomly the province of the revolt between
\granderegionSingala or \granderegionFormose if both contain a \TP/\COL. If
not, the chosen province is the one containing the \TP/\COL.
\aparag Place a \REVOLT \facemoins in the chosen region. This \REVOLT is not
connected to the Natives but military forces sent there to suppress it may
have to confront the Natives if they react.



\event{pIV:China Colonial Attitude}{IV-21}{\paysChine Colonial
  Attitude}{1}{PB}

\history{1557 / 1637}

\condition{This event is the same as \ref{pIII:China Colonial
    Attitude}. Exception: if \xnameref{pIII:CCA:Closure China} is already
  effective, apply \xnameref{pIV:CCA:Vassalisation Korea} instead.}


\subevent[pIV:CCA:Vassalisation Korea]{Vassalisation of Korea}

\phevnt
\aparag Two Chinese armies and the natives of \granderegionCorea attack any
\TP/\COL that are in the area (even Japanese \TP).
\aparag \granderegionCorea is now part of \paysChina.

\phpaix
\aparag If a \TP has survived, \pays{Chine} concedes a new \dipAT to the owner
of the \TP, if it didn't have any. The owner still has to pay as for usual
\dipAT with \paysChine.



\event{pIV:Japan Colonial Attitude}{IV-22}{\paysJapon Colonial
  Attitude}{1}{PB}

\history[Tokugawa's Commercial Restrictions in history]{1638}

\condition{}
\aparag If \paysJapon has no \TP, use \xnameref{pIV:JCA:Closure Japan}.
\aparag If \paysJapon has a \TP on the map (in \granderegionCorea), use
\xnameref{pIV:JCA:Japan Commercial Dynamism}.


\subevent[pIV:JCA:Closure Japan]{Tokugawa's Commercial Restrictions}

\phevnt
\aparag One country having a \TP in \paysJapon may sign immediately a Treaty
with \paysJapon, and so gains \dipAT. If more than one country have a \TP in
\paysJapon, all owners (except minor powers) make a secret bidding (minimum
bid is 50 \ducats).  The largest bidder wins and gains the \dipAT; all the
bids are lost and all other \TP are removed from \paysJapon.
\aparag When the \dipAT is accepted, only one \TP of the country is kept in
\paysJapon; excess \TP are destroyed. If refused, \paysJapon declares an
Overseas War against the power.
\aparag From now on, \dipAT allow one country to keep only one \TP in
\paysJapon (and not one per region). The remaining \TP can be upgraded, and it
causes no reaction by \paysJapon.
\aparag The basic forces and reinforcements of \paysJapon are now its mainland
army only (no overseas garrisons or fleets).

\effetlong
\aparag From now on, no new \TP counter can be placed in any area belonging to
\paysJapon by means of administrative actions.
\aparag No regular diplomacy is permitted on \paysJapon.  The Activation level
of \paysJapon becomes 11.


\subevent[pIV:JCA:Japan Commercial Dynamism]{Commercial dynamism of
  \paysJapon}

\phevnt
\aparag \paysJapon gains a \TP with level 6 in \provinceSeoul,
\provincePyongyang and with level 3 in \granderegionFormose, if those
provinces do not contain foreign \TP. \paysJapon has a \FTI of 2, raised to 3
from period V on.
\aparag If there are \TP in any of those provinces, \paysJapon declares an
Overseas War against all the country having those. This war may not be ended
by an automatic white peace.

\phadm
\aparag Basic forces of \paysJapon are increased to 2 \ARMY\faceplus in
\paysJapon, plus 1 \ARMY\faceplus (in \granderegionCorea at start), 2 \LD and
1 \FLEET\facemoins overseas.
\aparag Basic reinforcements are increased to one \ARMY\faceplus in mainland,
and 1 \ARMY\facemoins, 1 \ND for the garrisons.
\aparag If \paysJapon has a \TP counter, it gains 1 level, up to level 6 in
\provinceSeoul and \provincePyongyang, and level 3 in
\granderegionFormose. Choose one randomly for this increase if there are
several \TP.  These \TP exploit the resources in the region and are counted as
normal exploitation for monopolies and evolution of prices.

\phmil
\aparag Japanese forces outside \granderegionJapan do not activate the Natives
and an attack in regions with Japanese \TP may be aimed at the Japanese only
and so does not activate the Natives of the region. As soon as the \TP is no
more Japanese or destroyed, normal activation rules for Natives apply.

\phpaix
\aparag If \paysJapon does not lose the war and there is no \TP in any of the
3 provinces claimed, it places a \TP in there of level 1.

% Jym, 05/2013, Pierre's notes (2007).
\event{pIV:Deluge}{IV-y}{Swedish Deluge}{1}{PBNotEvenWritten}

\history{1648 (Khmelnytsky Uprising)-1667 (Truce  of
  Andrusovo)}[Russo-Swedo-Polish wars, Second Northern war]
\dure{2 turns.}

If \POL is at war, fortresses in \paysLithuanie let enemy supply go through
their province.

Should appear either during IV-17(2), or as IV-10(2).


% Jym, 05/2013, Pierre's notes (2007).
\event{pIV:Koprulu}{IV-z}{\ministreKoprulu}{1}{RistoMod}

Same as~\ref{pV:Koprulu}. Should appear late in the table only. (Jym):
Possibly as IV-17(3) or IV-11(2).

\vfill \pagebreak

%% *-* latex-mode *-*



\event{pIV:TYW}{IV-A}{Thirty Years' War}{1}{PB}

\history{1618-1648}

\activation{This war is a consequence of some religious fighting in the
  \HRE. If \ref{pV:WoSS} has already begun, this event is not possible
  anymore. Ignore it.}
\aparag It might be triggered by \xnameref{pII:Schmalkaldic League},
\xnameref{pIII:League Nassau}, \xnameref{pIV:Bohemian Revolt},
\xnameref{pIV:Augsburg Revocation} or \xnameref{pIV:Unity HRE}.  This event
may happen only once; before that, at the end of the first turn of a war
caused by one of the previous event, make the following test.
\aparag Roll 1d10 and add the modifiers:\par
\begin{modlist}
\item[\bonus{+4}] in period \period{II}
\item[\bonus{+2}] in period \period{III}
\item[\bonus{-2}]for each turn of the current war before this turn
\item[\bonus{-1}] if the peace modifier of the \HAB is >0
\item[\bonus{+2}] if \monarque{Charles V} rules \SPA
\item[\bonus{+2}] if \SPA has chosen \CATHCO
\item[\bonus{+2}] if \villeVienne is not owned and controlled by \HAB
\item[\bonus{+2}] if Augsburg confession was granted
\item[\bonus{+4}] is test during \xnameref{pII:Schmalkaldic League}
\item[\bonus{+2}] if test during \xnameref{pIII:League Nassau} and \SPA is
  \CATHCR
\item[\bonus{-2}] if test during \xnameref{pIV:Bohemian Revolt}
\item[\bonus{-2}] if test during \xnameref{pIV:Unity HRE}
\item[\bonus{-4}] if during \xnameref{pIV:Augsburg Revocation}
\item[\bonus{\textplusminus1}] if \ministre{Richelieu} or \ministre{Mazarin}
  are still present (choice of \FRA)
\item[\bonus{+1}] If \nameref{pIII:FWR} have yet to happen
\item[\bonus{+3}] If \nameref{pIII:FWR} are happening now
\item[\bonus{-1}] If Protestant won in \nameref{pIII:FWR}
\item[\bonus{+1}] If Counter-Reformation won in \nameref{pIII:FWR}
\end{modlist}

\aparag Result:\par
\begin{modlist}
\item[\geq 11] Appeasement of the religious fight
\item[7--10] Agitations in the \HRE
\item[\leq 6] Eruption of the Religious War
\end{modlist}
\aparag[Appeasement of the religious fight] The current war does not
degenerate in a general Religious War. No further test will be made for this
war.
\aparag[Agitations in the \HRE]
\bparag One \MIN enemy of \HAB will have a bonus of \bonus{+2} to its
reinforcement roll next turn (Alliance's choice).
\bparag \paysSaxe joins the enemy side of the \HAB in full intervention (or
\paysBrandebourg if \paysSaxe is already at war).
\bparag At the end of the next turn, roll this test anew to see if a Religious
War breaks.
\aparag[Eruption of the Religious War] The rest of the event will be applied
as one of the 4 regular events of the next turn.  No peace is made for the war
of this turn in the \HRE (except for specific rules of this war about
conquered minor countries).  The Thirty Years' War is now about to begin.

\phevnt
\aparag For the duration of the war, all countries have an additional trade
refusal of 150\ducats.
\begin{digressions}[War setup]


  \digression[pIV:TYW:Creation of the Germanic Alliances]{Creation of the
    Germanic Alliances}
  \aparag Two German sides are made up for this war: the (German) Catholic
  \ligue and the Protestant \alliance (more properly called: \emph{Protestant
    Union} or \emph{League of Evangelical Union}).  All minor countries of the
  \HRE at war will be part of one or another. When a minor country joins one
  alliance, it is placed in Neutral diplomatic position and will change of
  status before the end of the war only if specified by this event or another
  political event. The \HRE is now in Civil and Religious War
  (see~\ref{chDiplo:Religious Civil War}), with all the usual restrictions.
  \bparag The \alliance is formed by all the German minor countries that were
  enemies of the \HAB during the previous turn.
  \bparag \HAB and its German allies (minor countries at war with it) form the
  \ligue. \AUSMin is part of the \ligue as any other minor. \paysBaviere
  automatically joins this alliance.
  \bparag The stability of both sides is placed on \bonus{+2}, modified by any
  Major Victory of the preceding turn of their side (battles with troops of
  German minor countries or \HAB).  This stability will evolve during the turn
  because of the major victory/defeat of any forces in their alliance that is
  in any province of the \HRE (even if there are only forces of non Germanic
  major powers).
  \aparag[Attitude of the Netherlands] If \HOL is not a Major Power, the
  following conditions apply:
  \bparag If \payshollande is either owned by \SPA or is \paysprovincesne or
  \paysVhollande, apply \ref{pIII:Dutch Revolt}. This gives a new status to
  \payshollande (it may trigger the following points if still a \MIN).
  \bparag If \payshollande is a \VASSAL of \SPA (special or regular),
  \payshollande breaks its special status with \SPA. \SPA has an immediate
  free \CB against \payshollande ; if used, \payshollande revolts against the
  Spanish Crown, (re)apply \numberref{pIII:Dutch Revolt} and \HOL is now a
  Major Power. If it does not use it, apply \ref{pIV:TYW:VEN transfer}. For
  the rest of the event \HOLhol is neutral, and may not be involved in any
  manner in the incoming war. Ignore any reference to \HOLhol hereafter for
  this event.
  \bparag If \payshollande is a normal minor country, apply \ref{pIV:TYW:VEN
    transfer}. \HOLhol is involved in the war.
  \aparag[Transfer to \HOL]\label{pIV:TYW:VEN transfer} If
  \payshollande is liberated by the preceding paragraph, \VEN may be allowed
  to choose between incarnating \AUS or \HOL according to the rules of the
  Grand Campaign.
  \bparag If \VEN chooses \AUSMin (which becomes \AUS), \payshollande is now a
  normal minor country.
  \bparag If \VEN chooses \HOLmin (which becomes \HOL), \HOL is created with
  no Revolt (using the current position of \HOLmin).
  \bparag TODO: establish full starting position of non-revolted \HOL.
  \aparag The \alliance is controlled according to the order of preference (a
  player may not refuse control): \HOL, \ENG (Protestant), \FRA (Protestant),
  \SUE (Protestant), \RUS.
  \aparag The \ligue is controlled according to the order of preference (a
  player may not refuse control): \SPA (Counter-Reformation), \AUS (if it
  exists), \SPA (Conciliatory).
  \aparag If the \nameref{pII:Schmalkaldic League} or the \nameref{pIII:League
    Nassau} still do exist, the countries part of the League immediately join
  the Protestant \alliance and the Leagues are dissolved.
  \aparag If the period IV has not begun yet, the Major Powers: \SPA, \HOL,
  \SUE, \FRA and \AUT have to choose immediately if they take or not the
  Objectives relevant to this war. The Objective are conditions to be true at
  the end of period IV (and not especially this war).


  \digression[pIV:TYW:Extension Alliances]{Extension of the alliances}
  \aparag Every minor country of the \HRE that is not part of the war is
  checked for war entry at the beginning of each turn. One rolls 1d10, added
  to the \STAB of the side it could join, the current turn of the war
  (\bonus{+1} this first turn), and specific modifier for some countries.  On
  a result of {\bf 6 or higher}, this country enters the war.
  \aparag The list of the countries of the \HRE is given in
  \ref{table:TYW:Extension table}, with the side they will join and their
  starting force.  All those forces are conscripts, except where indicated.
  It is possible that, given the peculiar conditions of the war triggering the
  Religious War, a country ends up in a different side of the one which should
  be expected.
  \begin{table}\centering
    \begin{tabular}{l|l|c|p{.5\textwidth}}
      Country & Side & Mod. & Forces \\\hline
      \paysBaviere & \ligue & Auto. & \ARMY\faceplus, \LD, \fortress and at
      least 1 General (see below);
      % if none available, use \leader{Mercy} that will stay
      % until his death or when this war ends
      may use 2 \ARMY counters for all
      the duration of the war; starting forces are Veterans.\\
      \paysCologne & \ligue & & \LD, 1 \fortress\\
      \paysLiege & \ligue & & \fortress\\
      \paysMayence & \ligue & & \fortress\\
      \paysTreves & \ligue & & \fortress\\
      \paysAlsace & \ligue & --2 & \LD, \fortress\\
      \paysLorraine & \ligue & --4 & \LD\\
      \paysWurtemberg & \ligue & --2 & 2 \LD\\
      \paysThuringe  & \ligue & --2& none\\
      \paysBade &\alliance& & 2 \LD and \LeaderG (Georg Friedrich of
      Baden)\\
      \paysPalatinat &\alliance&& \ARMY\facemoins and \fortress\\
      \paysBerg &\alliance& --2& \LD\\
      \paysBrandebourg &\alliance& --2&\ARMY\facemoins and \LeaderG\\
      \paysBrunswick &\alliance& &\ARMY\facemoins and \LeaderG (Christian
      of Brunswick)\\
      \paysHanovre & \alliance & & \LD and \fortress\\
      \paysOldenburg & \alliance & --2 & \fortress\\
      \paysHanse&\alliance && \LD, \DN\\
      \paysHesse& \alliance& --2& \ARMY\facemoins and \fortress\\
      \paysSaxe&\alliance &--4& \ARMY\facemoins, \LD  and \fortress\\
      \paysBoheme &\alliance && \ARMY\facemoins and \LD\\
    \end{tabular}
    \caption{Extension of the Alliances during the Thirty Years' War}%
    \label{table:TYW:Extension table}
  \end{table}

  \aparag[Mercy] If there is no named \LeaderG of \paysBaviere in play, it
  receives \leaderMercy.
  \bparag If there is one, as soon as he dies (wound is not enough),
  \paysBaviere immediately receives \leaderMercy.
  \bparag \leaderMercy stays in play for 4 turns. If he arrives in the middle
  of a turn (due to death of his predecessor), this turn fully counts as his
  first turn of activity.

  \aparag The forces written may be inferior to the basic forces of the
  country (representing the confused situation).  They are only used when the
  country join the alliance. If already at war a previous turn, a country
  keeps all that is deployed and gains nothing new.
  \aparag If \AUSmin joins war at this time, they receive their basic force
  plus 1 \ARMY \faceplus (but no supplementary random reinforcement ; that
  will be part of those of the \ligue) as Veterans.
  \aparag No intervention (full or limited) of foreign countries are allowed
  if it is not explicitly written in this event.
  \aparag \paysSaxe may be used as mercenaries during this event once it
  surrendered all its home territory to the enemy. Its army is available to
  the side that controls its home territories; if this side loses subsequently
  part of the provinces, it still uses the army but can no more recruit
  Saxons; if it loses all the provinces, the Saxon forces are removed (and
  available now to the enemy).
\end{digressions}

\tour{Turn 1 (1624--1629)}

\phevnt
\aparag From now, and until the war is ended by the \xnameref{pIV:TYW:Peace
  Westphalie}, no Diplomacy is possible on minor countries of the \HRE, no
attempt to have them enter in a war also, and no declaration of war against
them is possible outside the rules of this event.
\aparag After the creation and the extension of both German sides in the war,
some foreign countries can be involved in it also.
\aparag The controller of each alliance can declare war to German minor
countries that refused to be in war this turn, precipitating them in the enemy
alliance (regardless of their religion).
\aparag \SPA enters the war as an ally of \ligue. This is not a formal
declaration of war and costs no \STAB.
\aparag \HOLhol enters war as an ally of \alliance.  This is not a formal
declaration of war and costs no \STAB. \HOLMin receives its full basic forces,
has a separate die-roll for reinforcements, is allied to the \alliance but not
part of it (for the conditions specifying that the \alliance sues for peace).
\aparag \ENG can do a limited intervention. Its side is the \alliance if \ENG
is Protestant, the \ligue if it is \CATHCR, or the one of its choice if it has
chosen \CATHCO.
\aparag \SUE, if \PROTRIG, can do a limited intervention as an ally of the
\alliance.
\aparag The Emperor of the \HRE, if he is not \HAB, can begin a limited
intervention in the War as an ally of the \ligue.
\aparag Any Major Power that was doing a limited intervention during the
previous turn (as defined in the original war) can continue this limited
intervention to help the same side.
% (JCD): TODO adapt to DAN
\aparag[The Danish Crusade]\label{pIV:TYW:Danish Crusade}
\DANMin makes a mandatory white peace with all its adversaries.  It then
enters the war as an ally of \alliance (but not part of it). It has 2 \ARMY
\faceplus (Veteran), 1 \FLEET \facemoins, 1 \fortress, 2 Multiple Campaigns
and is led by its general-king \leader{Christian IV} present for 4 turns.  It
does not receive reinforcements on this turn. \DANMin is played by \ENG.
\aparag All those alliances and interventions during the whole war are made
with the German alliances; the foreign countries are not allied with each
other except if they decide to sign a specific alliance. Else, they are not
obliged to continue the fight together (no penalty to sign peace) and only
separate peace from the German alliance is required.

\begin{digressions}[Specific rules for the war]


  \digression[pIV:TYW:Turkish Frontier]{The Turkish frontier}
  \aparag As long as there are 2 \ARMY\faceplus of \HAB in \villeVienne or any
  province once owned by \paysHongrie and a \LeaderG, \TUR may not declare a
  war to \HAB (but may continue one). For the first turn, this restriction is
  enforced if \HAB has this force available anywhere in the \HRE instead.
  \aparag If \villeVienne is conquered by the \alliance, or the previous
  condition is not respected at the Diplomatic Phase, \TUR has no such
  restriction.
  \aparag If \TUR takes \villeVienne, the \ligue will concede a winning peace
  to the \alliance at the end of the turn. A Crusade might then happen.
  % \aparag \TUR is entitled to make a Foreign Intervention against the \ligue
  % in this war if otherwise at peace and \paysTransylvanie is \VASSAL or
  % annexed.

  \aparag[] [BLP] \ref{chSpecific:Little war} is reactivated for \TUR only,
  and only with a small stack (up to 5\LD plus one \Pasha).
  \bparag That is, \TUR (not \paysCrimee) may send one (small) stack in non
  controlled former provinces of \paysHongrie and loses \STAB accordingly.
  \bparag Additionally, \TUR may also send this stack in national provinces of
  \AUS.

  \digression[pIV:TYW:German Reinforcements]{German reinforcements}

  \phadm
  \aparag Reinforcements for both \alliance and \ligue are determined globally
  for all German minor countries involved in an alliance.
  \aparag The \alliance is due to receive 4 \LD and the result of random
  reinforcements with a global modifier of \bonus{+2}.
  \bparag The attitude chosen must be offensive during the first two turns of
  the war and may be either offensive or defensive afterwards.
  \aparag The controller of the \alliance can pay 50\ducats to give a further
  \bonus{+1} to the reinforcement roll, or 100\ducats for a \bonus{+2}. If it
  does not pay, \SUE has the opportunity to do so and in this case will
  control \alliance for this turn only.
  \aparag The reinforcements of the \alliance are lowered by 1 \LD for each
  one of the following cities that have been conquered by the enemies (even if
  liberated later on): \villeMagdeburg and:
  \begin{itemize}
  \item \villeStuttgart, \villeErfurt if the war follows
    \xnameref{pII:Schmalkaldic League},
  \item \villeMunster, \villeRostock if the war follows \xnameref{pIII:League
      Nassau},
  \item \villeSpeyer, \villePrague if the war follows \xnameref{pIV:Bohemian
      Revolt}
  \item \villeBrunswick, \villeWeimar if the war follows
    \xnameref{pIV:Augsburg Revocation} or \xnameref{pIV:Unity HRE}.
  \end{itemize}
  \aparag The reinforcements of the \alliance are also lowered by 1 \LD for
  each two cities in the following list that have been conquered by the
  enemies (even if liberated later on): \villeHannover, \villeCassel,
  \villeDresden, \villeBerlin, \villeLubeck, \villeHamburg.
  \aparag If \AUSmin is part of the \ligue, the \ligue is due to receive 3 \LD
  and the result of random reinforcements with a global modifier of
  \bonus{+2}. Else (\AUS is a \MAJ), the \ligue receives only a random
  reinforcements with a global modifier of \bonus{+2}.  The \ligue uses the
  \ARMY counter of the \HRE regardless of who the Emperor is.
  \bparag The attitude chosen must be offensive during the first two turns of
  the war and may be either offensive or defensive afterwards.
  \aparag The controller of the \ligue can pay 50\ducats to give a further
  \bonus{+1} to the reinforcement roll, or 100\ducats for a \bonus{+2}.
  \aparag The reinforcements of the \ligue are lowered by 1 \LD for each one
  of the following cities that have been conquered by the enemies (even if
  liberated later on): \villeVienne, \villeSalzburg and \villeMunich.
  \aparag{Placement: \alliance then \ligue}
  \bparag The reinforcements obtained are freely distributed among the
  countries part of the alliance. \AUS as a Major power buys its own
  reinforcements but may take up to 2 \LD from the \ligue as its own
  reinforcements.
  \bparag They can only be placed in provinces not pillaged, not controlled by
  the enemy and free of enemy forces.
  \bparag They have to be placed in a province of their nationality, or with
  at least one \LD of the same nationality if their country is not completely
  occupied by the enemy.
  \aparag[Wallenstein] \HAB may hire mercenary general
  \leaderwithdata{Wallenstein}. He costs 40\ducats (payed by the controller of
  \ligue) to recruit him for one turn.
  \bparag If \leaderWallenstein is not hired at turn 1 or 2 of this war, he
  will not be available later. He can not be hired anew after the
  \xnameref{pIV:TYW:Peace Prague}. The first time \leaderWallenstein is hired,
  he appears anywhere in a friendly province of \payshabsbourg or \paysBoheme
  with one Veteran \ARMY\faceplus (use an \AUS or \HRE counter).
  \bparag \leaderWallenstein can command any stack of the \ligue (including
  \HAB) but no Bavarian counter.
  \bparag If at the end of a turn the \STAB of the \ligue is positive or its
  situation favourable, \leaderWallenstein is automatically dismissed. He can
  be hired again on the round and/or turn after \ligue suffered a Major
  Defeat.
  \bparag \MAJHAB can assassinate \leaderWallenstein at any time (even if he
  is currently dismissed). He is eliminated and \ligue (and \AUT) gain
  immediately {\bf 1} in \STAB.
  \bparag After the \nameref{pIV:TYW:Peace Prague}, \leaderWallenstein is no
  more available (and cannot be murdered anymore).
  \aparag Three mercenary generals are available each turn of this war.  They
  can be recruited by the \ligue or the \alliance. A general is recruited for
  one turn only. He can lead any stack of the alliance (including allied
  \MAJ); by paying 10\ducats more, he can lead a stack even if there is a
  general with higher rank.


  \digression[pIV:TYW:Condition War]{General conditions of the war}

  \phmil
  \aparag Each alliance has a Simple Campaign available each round.  Major or
  Multiple Campaign could be paid for by the controller of the alliance (cost
  lowered by 20\ducats).
  \aparag Each alliance and their allies draw supply in the \HRE from any
  province controlled by their side that is not pillaged or that has an
  unblockaded port.
  \aparag Supply can be traced through any neutral province, or controlled
  province (pillaged or not).
  % (JCD): Neither \HOL nor \FRA nor \POL?
  \aparag Both alliances can freely cross any neutral \HRE minor countries ;
  this is also permitted to \DANdan, \SUE, \ENG in limited intervention, \HAB
  of course and \SPA but not to other allies.
  \aparag Alternatively, a side may, before its movement, declare war against
  any neutral country of the \HRE. Its forces are immediately deployed.
  \aparag All pillages of the \ligue and of the \alliance are decided by their
  controller and goes in their Treasury.
  \aparag A Major Victory involving forces of one or both alliances adjust the
  \STAB of this side accordingly of the usual rules.


  \digression[pIV:TYW:Winning War]{Who is winning the war ?}

  \phpaix
  \aparag No minor country of an alliance ever makes a regular peace (even
  unconditional) outside of the peace of its alliance.
  \aparag One side may be in favoured position depending on the military
  control of the following cities.
  \bparag The \alliance is awarded 2 points for the control of \villeVienne.
  \bparag One point is awarded for each of those: \villeSpeyer, \villePrague,
  \villeMunich, \villeFreiburg, \villeStrasbourg, \villeHannover, \villeKleve,
  \villeCassel, \villeMagdeburg, \villeBerlin, \villeDresden, \villeFrankfurt
  and \villeBrunswick
  \bparag \textonehalf point is awarded for each of these: \villeKoln,
  \villeStuttgart, \villeUlm, \villeMainz, \villeTrier, \villeHamburg,
  \villeMunster and \villeErfurt
  % (Pierre): may be add \villeWiesbaden
  \bparag A side has a favoured position of it has at least 3 points more than
  the other alliance.
  \aparag Both the \alliance and the \ligue lose each {\bf 2} \STAB.
  \aparag Then if a side is favoured, it gains {\bf 1} \STAB.
  \aparag \SPA, \HOL and \AUS lose {\bf 1} \STAB if they were not in the
  original war (in full intervention, not just a limited one) on the previous
  turn.
  % (even if it was a war that lasted since more than one turn
  % ; this war counts as one turn of the current one): their second turn
  % of war just ended.
  \aparag \SPA, \HOL, \AUS lose {\bf 2} \STAB if they were at war (full
  intervention) on the previous turn (even if it was a war that lasted since
  more than one turn ; this war counts as one turn of the current one): their
  second turn of war just ended.
\end{digressions}
\aparag[Result of the Danish Crusade]
\bparag If \DANdan wins a battle against at least 1 \ARMY\faceplus of the
\ligue (or its allies) in the \HRE, is never routed in battle and has forces
left in \HRE at the end of the turn, then its Crusade is successful.
\bparag Thus the \alliance gains {\bf 1} \STAB ; \DANmin is placed in \EG of
\ENG, annexes immediately \provinceLubeck and \provinceHolstein (or
\provinceMecklenburg if it owns already both) and will continue its
intervention until the end of the war, or when it signs any separate peace (in
this war or another). It will not receive reinforcements \emph{per se}, but
some can be given from those of the \alliance.
\bparag If the Danish Crusade failed, \DANmin makes a white peace and
withdraws from the war. \leader{Christian IV} remains as a Danish general for
the full 4 turns.

\tour{Turn 2 -- The Lion of the North (1629--1632)}

\phevnt
\aparag Check for a possible extension of each alliance, see
\ref{pIV:TYW:Extension Alliances}.
\aparag \SUE has to enter the war as an ally of the Protestant \alliance.  If
it is Catholic, roll for 2 \REVOLT in \SUE and it loses {\bf 1} \STAB ;
nothing happens if it is Protestant -- no \CB is necessary and this is not a
declaration of war.
\aparag[Military revolution] \SUE receives \monarque{Gustave Adolphe}. He is
due to last 7 turns.
\bparag If the current Monarch has 1 or 2 turns of life left,
\monarque{Gustave Adolphe} would be his heir. If \monarque{Gustave Adolphe}
dies (in battle) before the current Monarch, \SUE will use the columns 7 to
roll its next Monarch.
\bparag If the current Monarch has more than 2 turns left, \monarque{Gustave
  Adolphe} replaces him entirely and will last for the remaining of the 7
turns as a Monarch (but a death in battle).
\bparag \monarque{Gustave Adolphe} is a military genius, a general
\leaderwithdata{Gustav-Adolf}. As long as the war goes on for \SUE, it
benefits from a Military Revolution (see \ref{chExpenses:Military
  Revolutions})
\bparag[] [BLP] The moment \leader{Gustav-Adolf} dies (even in the middle
of a round), \SUE receives \leaderBaner for 3 turns. \leaderBaner replaced the
deceased king (replace one counter by the other).
\bparag[\leader{Sachsen-Weimar}]\label{pIV:TYW:Saxe-Weimar}
\leaderwithdata{Sachsen-Weimar} joins \SUE for 7 turns also.
\bparag If \monarque{Gustave Adolphe} dies, \FRA (if allied to \SUE) may hire
\leader{Saxe-Weimar} as a mercenary general to fight in the present war.  It
costs 30\ducats the first turn, then 20\ducats to keep \leader{Saxe-Weimar};
when \leader{Saxe-Weimar} is not paid one turn, he is eliminated (he does not
go back to \SUE). \leader{Saxe-Weimar} takes command of one German stack of
the \alliance when he goes to \FRA; at each following turns, \FRA can take
half (round down) of the reinforcements of the \alliance (up to 4\LD) to be
placed with \leader{Saxe-Weimar}.  If he dies the forces go back to normal
status in the \alliance.

\aparag \FRA, if Protestant, can begin a limited intervention in the war on
the side of the \alliance.

\aparag Any \MAJ that was doing a limited intervention during the previous
turn (as defined in the original war) can continue this limited intervention
to help the same side.

\aparag \xnameref{pIV:TYW:Turkish Frontier} is in effect this turn.

\phadm
\aparag Roll for reinforcements as in the first turn, see
\xnameref{pIV:TYW:German Reinforcements}.

\phmil
\aparag The war is conducted according to \xnameref{pIV:TYW:Condition War}.
\aparag \SUE takes the control of the forces of one minor country of the
\alliance (its choice). This country can change from one turn to the other and
is chosen at the beginning of any military round of the turn.
\aparag \SUE may force a minor country to enter the war in the \alliance if it
is one of the countries that could join the \alliance and \SUE has at least 1
\ARMY\faceplus and \monarque{Gustave Adolphe} in a province of the country.
\aparag If \SUE makes a siege of allied or neutral \provinceMecklenburg,
\province{Ost Pommern} or \province{West Pommern} with at least one \ARMY
\faceplus, then the city surrenders without fighting at the end of the round.
\aparag All cities taken (by siege, assault or automatic surrender) with at
least one Swedish \ARMY, or only Swedish troops, have now Swedish garrisons
(and the town counts as Swedish presence in the \HRE).  Other Major powers put
their garrison if the city is taken with only their own forces (else, German
garrisons are in charge).

\phpaix
\aparag The balance of the war is checked as in \xnameref{pIV:TYW:Winning
  War}.  The losses of \STAB are applied except that now there is one turn
more:
\bparag Both the \alliance and the \ligue lose each {\bf 3} \STAB.
\bparag Then if a side is favoured, it gains {\bf 2} \STAB.
\bparag Any Major Power in its second turn of war lose {\bf 2} \STAB.
\bparag \SPA, \HOL, \AUS lose {\bf 3} \STAB if they are in their third turn of
war.
\bparag \SUE and \ENG if continuing their intervention lose {\bf 1} \STAB.
\aparag[Suing for peace]\label{pIV:TYW:Suing turn 2}
\bparag A German alliance sues for the \xnameref{pIV:TYW:Peace Prague} when it
is at \bonus{-3} in \STAB at the end of two consecutive turns, and the
position in the \HRE is not in its favour. The enemy side grants necessarily
this peace.
\bparag If both alliances are at \bonus{-3} in \STAB at the end of any turn,
their controllers can agree to a Status Quo and sign the
\nameref{pIV:TYW:Peace Prague}.

\bparag When the \nameref{pIV:TYW:Peace Prague} is signed, the German
alliances are partly dissolved; their stability will not be recorded further
and most of the minor countries in these alliances make a peace.  The
alliances want to stop the war and sign a peace so, from now on, all foreign
countries have no constraint to sign peaces also. It would not be a separate
peace from the German alliance point of view (but could be from another
country...)

\bparag However, if some Major Powers want to keep fighting in the \HRE and
refuse to sign the \nameref{pIV:TYW:Peace Prague}, see \ref{pIV:TYW:War after
  Prague}. Keeping fighting means that the Major power does not sign treaty of
peace with every enemy (that are \MAJ, the enemy German alliance, and possibly
\HOLmin and \DANmin); moreover this country is not allowed to sign a Truce
next turn. \AUSMin signs or not the \nameref{pIV:TYW:Peace Prague} alongside
\SPA.
\bparag If no Major Power contests the \nameref{pIV:TYW:Peace Prague} by
continuing the fight, apply now the \xnameref{pIV:TYW:Peace Westphalie}.

\tour{Turn 3 (1632--1636) and after: a Foreign War}
\history{Turn 4: 1637--1641 (first turn after the Peace of Prague); Turn 5:
  1642--1648 (from Rocroi and Jankov to Lens); Turn 6: 1648-1654 (La Fronde);
  Turn 7: 1654--1660.}

\phevnt
\aparag Check for a possible extension of each alliance, see
\xnameref{pIV:TYW:Extension Alliances}.
\aparag No limited intervention of the previous turn can be carried on.
\aparag At any turn, \FRA and \ENG can enter the war as an ally of the side
they chose. They have a \CB against a side which has not their Religious
Stand, and none against an alliance having the same Religious Attitude; the
\alliance is Protestant and the \ligue is \CATHCR.
\aparag At any turn, \POL (unless it is Orthodox) can make a full or limited
intervention in the war as an ally of any side. \POL can do such an
intervention only once during the war. It has a \CB only against an alliance
that has not the exact same Religious Attitude (relative to Catholicism) as
itself.

\phadm
\aparag Roll for reinforcements as in the first turn, see
\xnameref{pIV:TYW:German Reinforcements}.
\aparag Two turns after a Military Revolution caused by \SUE, the Land
Technology of the Latin minor countries reaches this new Technology.

\phmil
\aparag The war is conducted according to \xnameref{pIV:TYW:Condition War}.
\aparag \SUE takes the control of the forces of one minor country of the
\alliance (its choice). This country can change from one turn to the other and
is chosen at the beginning of any military round of the turn.
\aparag On the third turn only (not after), if \SUE makes a siege of allied or
neutral \provinceMecklenburg, \province{Ost Pommern} or \province{West
  Pommern} with at least one \ARMY \faceplus, then the city surrenders without
fighting at the end of the round.
\aparag All cities taken (by siege, assault or automatic surrender) with at
least one Swedish \ARMY, or only Swedish troops, have now Swedish garrisons
(and the town counts as Swedish presence in the \HRE).  Other Major powers put
their garrison if the city is taken with only their own forces (else, German
garrisons are in charge).

\phpaix
\aparag The balance of the war is checked as in \xnameref{pIV:TYW:Winning
  War}.  The losses of \STAB are applied with one turn more. This war can not
cause a loss more than {\bf 4} \STAB at the end of turn.  On turn 3 of the
Religious War, the losses should be:
\bparag the \alliance and the \ligue lose {\bf 4} \STAB;
\bparag the favoured side then gains 2 \STAB;
\bparag any Major Power in its third turn of war lose {\bf 3} \STAB.
\bparag \SPA, \HOL, \AUS lose {\bf 4} \STAB if they were at war before the
Religious War in the \HRE.
\bparag \SUE loses {\bf 2} \STAB.
\bparag Any other Major Power intervening in the war at this turn lose {\bf 1}
\STAB.
\aparag[Suing for peace] As described in \ref{pIV:TYW:Suing turn 2}.
\aparag If \SUE, \ENG or \POL (in full intervention) do not hold any city nor
have any \ARMY left in the \HRE, they make a mandatory white peace
against all its enemies in this war. This will count as a losing position in
\xnameref{pIV:TYW:Peace Westphalie}.
\aparag If \POL is doing a limited intervention and wins a battle against at
least one \ARMY\faceplus of the enemy side (any nationality) in the \HRE, then
loses no battle in the \HRE, the alliance it helps gains {\bf 1} in \STAB
(\AUS also). \POL may then annex \province{Ost Pommern} or any province in the
\HRE that once was Polish. Its limited intervention lasts only one turn.

\begin{digressions}[Between Prague and Westphalie]


  \digression[pIV:TYW:Peace Prague]{Peace of Prague}
  \aparag If the \ligue is favoured by the Peace:
  \bparag The \xnameref{pIV:TYW:Southern HRE Alliance} is created
  \bparag \paysBaviere gains permanently its second \ARMY and \paysPalatinat
  loses its own; \paysBaviere is now an Electorate.  It also gains a permanent
  \bonus{+1} to its reinforcement rolls.
  \bparag \paysBaviere annexes \provinceOberPfalz, except if this war follows
  \xnameref{pII:Schmalkaldic League}, in which case it annexes
  \provinceSchwaben.
  \bparag \paysBaviere is now in \AM with \HAB (move its diplomatic marker
  accordingly).
  \bparag A Total Victory of the \ligue in the \xnameref{pIV:TYW:Peace
    Westphalie} is possible.
  \bparag Any specific consequence given by the victory of the side of the
  \ligue in the war having caused \ref{pIV:TYW} is applied.
  \bparag The Truce of Augsburg is revoked.
  \bparag \SPA and \AUS gain 30 \PV, \SUE loses 10 \PV.
  \bparag \DANdan loses its second \ARMY counter, unless its crusade was
  successful.
  \aparag If the Peace is a Status Quo:
  \bparag \paysBaviere keeps its second army for the continuation of this war
  (but not permanently).
  \bparag The Truce of Augsburg is in effect.
  \bparag No side can achieve Total Victory in the \xnameref{pIV:TYW:Peace
    Westphalie}.
  \aparag If the \alliance is favoured by the Peace:
  \bparag The Truce of Augsburg is in effect.
  \bparag A \xnameref{pIV:TYW:Northern HRE Alliance} is created and allied to
  \HOL.
  \bparag \paysOldenburg, \paysHanovre, \paysHesse, \paysHanse and \paysBerg
  are placed in \EG of \HOL.
  \bparag A Total Victory of the \alliance is now possible.
  \bparag \HOL and \SUE gain 30 \PV.


  \digression[pIV:TYW:War after Prague]{The War after Prague}
  \aparag Only some minor countries continue the war. All other minor
  countries of the \HRE surrender: their forces are withdrawn and their cities
  are considered as taken for the reinforcements.
  \bparag On the side of the \ligue: \HAB and, if the Peace is not in favour
  of the \alliance, \paysBaviere.
  \bparag On the side of the \alliance: the controller is now \SUE and it
  chooses 2 countries, (only 1 if \ligue won the Peace of Prague), that will
  continue the fight from the following list: \paysHesse, \paysHanovre,
  \paysPalatinat, \paysSaxe.
  \bparag If the Peace is favourable to the \ligue, \paysSaxe reverses its
  alliance and enters war with the Catholics.  All its forces are withdrawn
  from the map, and the cities of \paysSaxe surrender immediately to the
  Catholics; Protestant forces in the provinces are withdrawn.
  \bparag \paysBrandebourg will continue (or enter) the war as an ally of the
  Protestant if \SUE gives up its claims on \province{West Pommern} to
  \paysBrandebourg in \xnameref{pIV:TYW:Peace Westphalie}.
  \bparag If \FRA hires \leader{Saxe-Weimar} at this turn (continuing from a
  previous turn or not), he keeps one stack of any one protestant
  country. This country remains at war (until it surrendered unconditionally
  or \leader{Saxe-Weimar} is no more at the service of \FRA). It will receive
  reinforcements for this stack (using the mechanism for the stack of
  \leader{Saxe-Weimar}).
  \bparag Provinces of \paysHanse that are controlled by a country still at
  war stay at war even if \paysHanse itself is not at war anymore.

  \aparag The minor countries that continue the war are allied in their
  alliance, and with the Major countries in the war. But they want peace so
  they will stop fighting as soon as all foreign minor/major countries do
  likewise.
  \bparag A minor country of the \HRE can now be ejected from its alliance and
  from the war, but only by imposing an unconditional surrender on it; other
  regular peaces are not possible.

  \aparag All other minor countries that were in both alliances are now at
  peace; they all have now a Neutral diplomatic status.  All the cities in
  those countries are considered conquered in order to check for
  reinforcements.
  \aparag Foreign minor country \DANmin stops the war whereas \HOLmin
  continues. A regular peace has to be obtained against it.
  \aparag Do not forget that this war causes at most a loss of {\bf 4} \STAB
  for each country at the end of turn.  If the War caused by the Revolt of the
  United Provinces continue, it resumes its normal loss in \STAB only if an
  Armistice is made (at least 1 turn) between \SPA and \HOL at the end of the
  present war; else the present war has to continue and so does the loss of
  {\bf 4} \STAB each turn.


  \digression[pIV:TYW:Peace Westphalie]{Peace of Westphalie}
  \aparag This Peace is signed at the end of a turn, beginning with the turn
  of the \nameref{pIV:TYW:Peace Prague}, if all Major countries in the war
  agree to end the war, that is to sign Peaces or Armistices between them. The
  following effects are implemented as further consequences of the regular
  Peace Treaties.
  \aparag The Emperor of the \HRE is now \HAB if this was not, for the rest of
  the game.
  \aparag The Major Countries that can be involved in the war are \SPA (and
  \AUSmin), \AUS, \FRA, \HOL, \SUE, \ENG and \POL.
  \bparag A Major Power that stops the war (it has signed Peaces or Armistices
  with all other Major Powers at the end of some turn) before the end has a
  losing position for this Peace; it has also this position if it signs a
  mandatory white peace (for any reason).
  \bparag A Major Power has a dominant position if it signs only winning
  Treaties of Peace with countries of the other side (no Armistices or White
  Peaces either) on the last turn of this war.
  \bparag A Major Power has a losing position if it signs only losing Treaties
  of Peace with countries of the other side (no Armistices or White Peaces
  either) on the last turn of this war.
  \bparag In other cases, the position is neutral.

  \aparag[Spain or Austria]
  \bparag These specific conditions are for \MAJHAB.
  \bparag A \AUSmin will continue to fight with \SPA until the end of the war
  (except by unconditional surrender, following the rules for all minor
  countries from the \HRE still at war after the \nameref{pIV:TYW:Peace
    Prague}).
  \bparag If both \HIS and \AUS are in dominant position and a Catholic Total
  Victory was possible, the \pays{German Empire} is created (see
  \shortref{pIV:TYW:German Empire}).
  \bparag If \HAB is in dominant position but no Catholic Total Victory was
  possible, a \xnameref{pIV:TYW:Southern HRE Alliance} is associated to \HAB.
  The countries in this alliance are put in \EG of \HAB: \paysBaviere,
  \paysTreves, \paysAlsace, \paysBade and \paysWurtemberg.
  \bparag \HAB in neutral position: nothing more.
  \bparag \HAB in loosing position: destruction of the
  \xnameref{pIV:TYW:Southern HRE Alliance}.

  \aparag[Spain] If \SPA is in dominant position, it gains a permanent {\bf
    +1} bonus in Diplomacy on Catholic countries of the \HRE.

  % and there is a Major \AUS,
  % the \xnameref{pV:WoSS} will only concern provinces in the Low Countries and
  % northern Italy (those in southern Italy are not part of the Inheritance and
  % remain Spanish).

  \aparag[Austria] If \AUS is in neutral position, it gains a permanent {\bf
    +1} bonus in Diplomacy on Catholic countries of the \HRE.

  \aparag[The Netherlands]
  \bparag If \HOLhol has a dominant position and a Protestant Total Victory is
  possible, \paysHanse annexes \provinceOldenburg and \HOL gains \paysHanse as
  a permanent \VASSAL.  Eliminating the \xnameref{pIV:TYW:Northern HRE
    Alliance} will now need a Peace of level 5 against \HOL.
  \bparag If \HOLhol has a dominant position (but without possible Protestant
  Total Victory), it gains \paysHanse as a normal \VASSAL and \paysHanse
  annexes \provinceOldenburg. The \xnameref{pIV:TYW:Northern HRE Alliance} is
  created and allied to \HOLhol with the corresponding effects.
  \bparag If \HOL has a neutral position, it has the choice to allow or not to
  the destruction of \paysHanse (its controller in the case of a \HOLmin).
  \bparag Else, if \HOL (or minor \payshollande) is in losing position, the
  \paysHanse is destroyed and the \xnameref{pIV:TYW:Northern HRE Alliance} is
  dissolved.

  \aparag[Sweden]
  \bparag If \SUE has a dominant position, it annexes \provinceMecklenburg,
  then \province{West Pommern} if it has not renounced its claims on this
  province (else it gains \paysBrandebourg in \EG) and \provinceBremen or
  \provinceLubeck (its choice). It then chooses one Protestant minor country
  (or 3 minor countries if a Protestant Total Victory was possible) of the
  \HRE that is (are) placed in \EG on its Diplomatic chart.
  \bparag If \SUE is in neutral position, it annexes \provinceMecklenburg,
  then \province{West Pommern} if it has not renounced its claims on this
  province; else it gains \paysBrandebourg in \EG. It then chooses one
  Protestant minor country of the \HRE that is placed in \EG on its Diplomatic
  chart.
  \bparag If \SUE is in losing position, it gains nothing.

  \aparag[France]
  \bparag If \FRA is in dominant position, it gains a \bonus{+1} bonus for
  Diplomacy on countries of the \HRE until the end of the period and a free
  \CB against \HIS, to be used during this period.
  \bparag If \FRA is in dominant or neutral position, it gains \paysAlsace as
  a \VASSAL and \paysCologne in \EC.

  \aparag[England] If \ENG is in dominant position, it gains a \bonus{+1}
  bonus for Diplomacy on countries of the \HRE until the end of period V.  It
  also gains a minor country of its choice, having the same religion as \ENG,
  that is placed in \EG on its chart.
  \aparag[Poland] If \POL is in dominant position after a full intervention,
  it gains a \bonus{+1} bonus for Diplomacy on countries of the \HRE until the
  end of period V.  It also gains a minor country of its choice, having the
  same religion as \POL, that is placed in \EG on its chart.
  \aparag When a major country can take a the diplomatic control of a minor
  country, the order of choice is the order written here, and a power can only
  choose neutral minor country of the \HRE (not those already allied to
  someone else).
  \aparag \paysBrandebourg annexes \province{Ost Pommern} if it is in
  \paysHanse.
  \aparag Then, if \paysHanse has to be destroyed, its remaining provinces are
  now given as follows: \SUE takes \provinceBremen, \paysBrandebourg takes
  \province{West Pommern} and \provinceMecklenburg, then \DANmin all the
  remaining ones.
  \bparag Otherwise, \paysHanse is considered to have no capital (its
  provinces may thus be annexed by anybody).
  \aparag From now on, any major power that owns a province in \HRE or
  adjacent to a province of the \HRE may, when at war, enter and remain in any
  neutral province of the \HRE. The cost in \MP is the same as enemy
  territory. The neutral provinces can not be pillaged, besieged nor give
  supply (but supply lines can cross those if there are no enemy force
  within).
  \aparag In any cases, \paysHanse has no more capital (all its provinces can
  be annexed regularly).
  \aparag[Victory Points]
  \bparag A Major Power in dominant position at the end of the war wins 30 \PV
  (added to those of the treaties of Peace).
  \bparag A Major Power in losing position at the end of the war loses 30 \PV.
\end{digressions}

\begin{digressions}[German alliances emerging from the war]


  \digression[pIV:TYW:Northern HRE Alliance]{Northern \HRE Alliance}

  \effetlong
  \aparag When this alliance exists, it is allied to \HOLhol.  It represents
  treaties between \paysOldenburg, \paysHanovre, \paysHesse, \paysHanse and
  \paysBerg.
  \bparag These countries are put in \AM of \HOL.

  \aparag \HOL has a permanent bonus of \bonus{+2} in Diplomacy on these
  countries.
  \aparag \HOL gains also a income of 10\ducats for each coastal city in
  \paysHanse if it is on his diplomatic track.
  \bparag This Northern alliance is dissolved when \HOL signs a losing Peace
  of level 3 or higher, or when it controls no country of the alliance. The
  bonuses are permanently lost.


  \digression[pIV:TYW:Southern HRE Alliance]{Southern \HRE Alliance}

  \effetlong
  \aparag A Southern \HRE alliance is associated to \HAB, composed by the
  following countries: \paysBaviere, \paysMayence, \paysAlsace, \paysBade and
  \paysWurtemberg.
  \bparag These countries are put in \AM of \HAB.

  \aparag Each of these countries on the \HAB or \MAJHAB diplomatic chart will
  give an income of 10\ducats to \MAJHAB.
  \aparag \MAJHAB gains a \bonus{+1} bonus in Diplomacy on every Catholic
  countries in the \HRE.
  \aparag This Southern alliance is dissolved when \MAJHAB signs a losing
  Peace of level 3 or more, or when neither \MAJHAB nor \HAB controls any
  country of the Alliance.  The bonuses are permanently lost.
  \aparag When a \GE is created, the Southern alliance is also dissolved (and
  becomes part of the \GE).


  \digression[pIV:TYW:German Empire]{German Empire}

  \effetlong
  \aparag All minor countries of the \HRE (except \HAB which remains
  independent) are associated in one minor country, called the \pays{German
    Empire}. This country is a permanent \VASSAL of \MAJHAB. It can use 4
  \ARMY counters, and 12 \LD (for practical ease, use the counter of the \HRE
  and any counter of some part of the empire, with no notion of nationality --
  there are all from the \GE).  Its basic forces are one \ARMY\faceplus and
  one \ARMY\facemoins. It has a modifier of \bonus{+2} for reinforcements and
  always makes peace with \MAJHAB.
  \aparag \MAJHAB receives an income of 100\ducats from the \HRE (and not the
  exact value of the country) and can use its port on the Baltic Sea.
  \aparag When the \pays{German Empire} exists, the Dynastic Alliance between
  \AUSmin and \SPA is both defensive and offensive.
  \aparag Some events may dissolve part of the \pays{German Empire} by
  creating a League (\xnameref{pII:Schmalkaldic League}, \xnameref{pIII:League
    Nassau}, \xnameref{pIV:Bohemian Revolt}, \xnameref{pIV:Augsburg
    Revocation}, \xnameref{pIV:Unity HRE}) which ceases to be in the Empire,
  and is (depending on the event) at war with the Emperor. An unconditional
  peace of the Emperor on any of those countries bring it back in \pays{German
    Empire}.
  \aparag Event \xref{pV:Kingdom Prussia} liberates \paysBrandebourg from
  \pays{German Empire} (and it can't be forced back in).
  \aparag When any province with a capital of \pays{German Empire} is lost as
  the result of a Peace, the minor country having this capital is renewed as a
  free country, having status \EG or \VASSAL (if possible) with the \MAJ that
  liberated it (player's choice). \HAB can force the \MIN back in the
  \pays{German Empire} by means of an unconditional peace on it.
  % (Jym) For the Hansa, is it enough to free one capital?
  \aparag Some events (\xnameref{pIV:Augsburg Revocation}, \xnameref{pIV:Unity
    HRE} and \xnameref{pV:Devolution War}) can cause Civil War in \pays{German
    Empire} that foreign countries can help in order to dissolve \pays{German
    Empire}.
  \aparag The \xnameref{pV:WoSS} may separate the Spanish dynasty from the
  Austrian dynasty because of a Crisis of Succession.
  \bparag If \SPA chooses a \AUSmin Heir, the \pays{German Empire} fights
  along their side with no Dynastic Separation.
  \bparag If \SPA chooses another Heir than a \AUSmin, the \pays{German
    Empire} is dissolved but \paysBaviere, \paysMayence, \paysLorraine,
  \paysBade and \paysWurtemberg are placed in \AM of \HAB and enters war at
  its side; and \HAB gains the benefits of \xnameref{pIV:TYW:Southern HRE
    Alliance}. All other countries that are recreated at this time are
  Neutral.
  \bparag \AUS (if major) keeps the \pays{German Empire}.
  \bparag See the other conditions in this event.
  \aparag The \pays{German Empire} ceases to exist as soon as its controller
  is forced to sign any peace of level 3 or more.  In addition to the normal
  peace conditions, \pays{German Empire} is dissolved: all minor countries of
  the HRE are back to previous frontiers, and are neutral.
\end{digressions}

% Local Variables:
% fill-column: 78
% coding: utf-8-unix
% mode-require-final-newline: t
% mode: flyspell
% ispell-local-dictionary: "british"
% End:

% LocalWords: defensive se Hansa offensive pIII FWR Schmalkaldic pIV TYW pII
% LocalWords: Ost Pommern HRE Westphalie unblockaded JCD Sachsen Weimar Quo
% LocalWords: pV WoSS Jym Rocroi Jankov


\vfill \pagebreak



\event{pIV:Polish Civil War}{IV-B}{Civil War in Poland}{1}{PB}

\history{\textit{alternative history}}
\dure{Until the end of the war}

\phevnt
\aparag Can only happen once, either as explained in \ref{pIV:Liberum Veto} or
in \ref{pV:Saxon King Poland}.
\aparag \POL is now in civil war. One side, called ``Absolutists'' remain
loyal to the King and try to impose Absolutism in \POL while the other side,
called ``Rebels'' is lead by the great nobles of the kingdom trying to keep
the Republic and the elective monarchy.
\bparag The player plays the Absolutists.
\aparag If they have a province bordering \POL, the following countries can
enter a full war against any of the side: \RUS, \SUE, \HAB, \PRU.
\bparag They have a free \CB this turn against both sides of the civil war.
\bparag Other countries can only make a foreign intervention as per normal
rules.
\aparag \textbf{Economic and Political crisis}: The \RT of \POL is diminished
by half and loses at least 50\ducats. \POL loses 2 \STAB.
% \begin{oldcompta}
%   \bparag Do not take into account the minimal loss of 50\ducats.
% \end{oldcompta}
\aparag The Rebels control the following provinces:
\bparag \provinceMalopolska, \provinceLietuva ;
\bparag one other province randomly chosen in \paysmajeurPologne;
\bparag two other provinces randomly chosen in \paysmajeurLithuanie.
\bparag The 5 provinces must be different and all possessed by \POL at the
beginning of the event.
% (JCD): Controlling cities is the same as controlling provinces! removing
% \bparag The Rebels control the cities in these 5 provinces.
\aparag Roll for two \REVOLT in \POL. There are \facemoins and do not control
the cities.
\aparag If \ref{pIV:Times of Troubles} already happened but not
\ref{pIV:Revolt Cossacks} and the religious attitude of \POL is not Tolerance
of the Orthodoxy, \nameref{pIV:Revolt Cossacks} happens immediately.
\aparag The Rebels side is played by the first country at war against the
Absolutists in the following list: \RUS, \SUE, \HAB, \TUR, \HOL, \ANG, \FRA,
\PRU.
\bparag If none is at war against the Absolutists, then the Rebels are played
by the first country in the same list which is not at war as an ally of the
Absolutists.

\phadm
\aparag Lands forces of \POL equal to the basic forces for the period
(excluding Ukraine) become Rebels.
\bparag If \POL does not have enough troops raised, an immediate levy happens,
paid for by the treasure of \POL (even if this causes a bankruptcy).
\aparag The basic upkeep for the Absolutists is the one of \paysPologne only
(\ARMY\faceplus).
\bparag The player may use the counters of \paysPologne and two \ARMY (they
can be taken from any unused country, and are similar to any other Polish
Army)
\bparag Absolutists receive normal income from the provinces they control.
\bparag Absolutists troops in rebel provinces (at the beginning of the war)
are retreated normally.
\bparag Fleet stay loyal to the Absolutists.
\bparag The king of \POL must be used as a general of the Absolutists, except
if he is \monarque{August II}.
\aparag The Rebels side uses the counters of \paysLithuanie as well as two
revolts \ARMY.
\bparag He does not get reinforcement at the first turn of the war.
\bparag At the first turn of the war, the Rebels forces can be freely
redeployed in the controlled provinces.
\bparag If a named general (other than \leaderPatkul when \monarque{August II}
is king) is in play, he takes side for the Rebels. Otherwise, the Rebels are
lead by a random mercenary general and get an extra random general.
\aparag \REVOLT in \POL are friendly to the Rebels.
\bparag A rebel general can lead a \REVOLT . A \REVOLT \facemoins count as
2\LD for hierarchy rules.
\aparag Starting with the second turn of the war, Rebels get reinforcement
either in offensive or defensive attitude based on the income of the province
they control (control the city with no absolutist army in the province).
\aparag If \paysukraine is not in revolt or independent due to \ref{pIV:Revolt
  Cossacks}, the Ukrainian army can be used by the Absolutists (but without
the basic upkeep for it).
\aparag If the king is member of the dynasty of \paysSaxe, he can use the
forces of the minor as per the rules of \ref{pV:Saxon King Poland}.
\bparag In that case, \paysSaxe is at war against the Rebels and their allies
can freely cross the \HRE and wage war in \paysSaxe.

\phmil
\aparag Absolutists and Rebels get supply from the cities they control.
\bparag They can cross enemy provinces without besieging the city.
\bparag This is only true for polish forces. Not for the foreign allies.
\bparag The Absolutists cannot cross freely the provinces with a \REVOLT .

\phpaix
\aparag Victory in the civil war occurs as soon as one side gets two out of
the following three conditions:
\bparag controlling the capital (controlling \provinceMalopolska and, if
\villeVarsovie has been made capital, \provinceMazowia) ;
\bparag controlling the country (military control of at least 60\% of the
provinces, that is controlling the city without enemy presence ; provinces
with a \REVOLT and the city still controlled by the Absolutists count for
nobody) ;
\bparag military victory (having one more major victory than the other side
this turn, or the other side as no more \ARMY in play).
\aparag The war lasts as long as no side wins.
\aparag Wars with foreign countries can be ended by separate peaces.
\bparag If the Absolutists are not fully at war against another major country,
\POL does not lose \STAB due to the war (but does so due to \REVOLT ).
\bparag A (foreign) peace in the civil war is also a peace with \POL (if
another war was occurring), or a separate peace with loss of 2 \STAB for
breaking the alliance with the side of the civil war the foreign country was
allied to.
\aparag[Absolutists victory]
\bparag The effect of \ref{chSpecific:Poland:Liberum Veto} are cancelled.
\bparag Events \xref{pVI:Great Northern War}, \xref{pVII:Bar Confederation},
\xref{pVII:First Partition Poland}, \xref{pVII:Second Partition Poland} and
\xref{pVII:National Revival of Poland} are modified.
\bparag Any country fully allied with the Absolutists who accept the peace
annexes a province of \POL (\POL choose which).
\bparag The Rebels armies are eliminated.
\bparag The \REVOLT stay in place.
\aparag[Rebels victory]
% (Jym): added condition on election
\bparag A dynastic crisis occurs and a new king is elected (this is a change
of polish dynasty), a general cannot be elected king unless he took the side
of the Rebels.
\bparag A Polish provinces is given to each \MAJ who was fully at war against
the Absolutists (choice is made by the \MAJ receiving the province, in order
of initiative).
\bparag The \STAB of \POL immediately becomes -1.
\bparag The \REVOLT and the Absolutists armies are removed.

\stopevents

% Local Variables:
% fill-column: 78
% coding: utf-8-unix
% mode-require-final-newline: t
% mode: flyspell
% ispell-local-dictionary: "british"
% End:

% LocalWords: pIV TYW HRE offensive reroll minister Olivares Safavids pIII pV
% LocalWords: Liberum Oxenstierna Torstensson Moghol Singala CCA PBNew JCD de
% LocalWords: Vassalisation Bethlén Mansfeld Westphalie Ausgsburg Schmalkaldic
% LocalWords: malus Risto Espagne HOL Ormus POR Jym reannexed Reannexation
% LocalWords: RistoMod Angleterre ECW Montrose Duche Prusse Pommern Ost JCA
% LocalWords: Torstensson's Fyodor defensive Formose pII Alaouite pVI pVII
% LocalWords: PBNotEvenWritten willingfully


\clearpage

% -*- mode: LaTeX; -*-

\section{Period V}\label{events:pV}



\subsection*{Event Table of Period V}

\begin{eventstable}[Period V events table]
  \tabcolsep=5pt\centering%
  \begin{tabular}{|l|*{5}{c}|l|}
    \hline
    1\up{st}\textarrow& 1-4 & 5-6 & 7 & 8 & 9 & 10 \\ \hline
    1 & 1  & 7  & 1  & 21  & R3   & \textbullet~1--2:\\
    2 & 2  & 8  & R2 & R22 & R4   & +1 then\\
    3 & 3  & 9  & R3 & 2   & 5    & \nameref{events:pIV}\\
    4 & 4  & 10 & 4  & 3   & 6    & \textbullet~3--10:\\
    5 & 5  & 11 & 6  & 9   & R16  & \nameref{events:pIV}\\
    6 & 6  & 12 & 7  & R10 & 17   & \\
    7 & 14 & 13 & 15 & 12  & 18   & \\
    8 & 17 & 15 & 23 & 13  & R19  & \\
    9 & 18 & 16 & R4 & 14  & R20  & \\ \hline
    10 & \multicolumn{6}{l|}{1--6 \nameref{events:pVI}, 7--10 \nameref{events:pIV}} \\ \hline
  \end{tabular}
\end{eventstable}

\eventssummary{%
  pV:Devolution War|,%
  pV:Chamber of Reunion|,%
  pV:League Augsburg|,%
  pV:Glorious Revolution|,%
  pV:WoSS|,%
  pV:Colbertian Mercantilism|,%
  pV:Expulsion French Protestants|,%
  pV:Grand Siecle|,%
  pV:English Dynamism|,%
  pV:Montecuccoli to Eugen|,%
  pV:de Witt|,%
  pV:Peter the Great|,%
  pV:Saxon King Poland|,%
} \eventssummary{%
  pV:Kingdom Prussia|,%
  pV:War Sweden Denmark|,%
  pV:Koprulu|,%
  pV:Fights Iroquois|,%
  pV:Slave Revolts WI|E/E,%
  pV:Wars India|E/E,%
  pV:Treaty Nerchinsk|,%
  pV:Invasion Formosa|,%
  pV:Japan Trade|,%
  pV:Revolt Cossacks|O{pIV:Revolt Cossacks},%
  pV:Revolt Catalunya|,%
  pV:Hungary|,%
  pV:Transylvania|,%
  pV:Cretan war|,%
  pV:Morean war|,%
  pV:Revolt Pueblos|,%
  pV:Tangiers|,%
  pV:Khoikhoi|E/E,%
  pV:Bill Test|,%
  pV:Kuruc|,%
}

\newpage\startevents



\event{pV:Devolution War}{V-1}{War of Devolution}{1}{Risto}

\history{1667-1668}

\condition{Can occur only if \FRA is not currently in a war (including Civil
  Wars).  Otherwise, re-roll.}

\phdipl
\aparag \FRA receives a free \CB for this turn against one owner of either
\provincePicardie, \provinceArtois, \provinceFlandre or \provinceHainaut.
This event is triggered off by \FRA using this \CB to declare a war, and if it
declines to do this the rest of the event does not occur.
\aparag \HOL and \ENG may each sign a Defensive Alliance with the victim of
\FRA declaration of war per above, provided both sides agree, immediately at
this turn or on the following turn.  The alliance provides an immediate \CB as
reaction against the declaration of war of \FRA.

\phadm
\aparag \FRA can collect incomes in the above mentioned provinces whenever
they are militarily conquered by \FRA.

\phpaix
%\aparag If victorious, \FRA receives 30 \PV at the end of this war.
\aparag If \FRA is not victorious, \HOL receives, if it was at war, 30 \PV at
the end of a war against \FRA triggered by this event.



\event{pV:Chamber of Reunion}{V-2 (1)}{Chamber of Reunion}{1}{Risto}

\history{1681-1684}

\condition{}
\aparag Cannot occur if there is a German Empire. In that case mark off, but
do not consider as played for the first time.
\aparag Cannot occur if \provinceAlsace is not part of \paysAlsace.  In that
case mark off and considered as played for the first time.

\phevnt
\aparag \FRA annexes \provinceAlsace. This provides \SPA, \HOL, \ENG and \AUS
a temporary \CB against \FRA for this turn.
\aparag If \FRA currently militarily occupies \provincePicardie,
\provinceRosselo, \province{Franche-Comte} and/or \provinceArtois, it can
immediately annex any such province without any peace treaty.

\phdipl
\aparag The current Emperor (or \SPA if \AUSmin is Emperor) receives a bonus
of \bonus{+3} for its diplomacy on all \HRE minors this turn.



\event{pV:League Augsburg}{V-2 (2)}{War of the League of Augsburg}{1}{Risto}

\history{1688-1697}

\condition{}
\aparag Can occur only if \FRA is not involved in a war (including civil
war). Otherwise re-roll.
\aparag Cannot occur if \ref{pV:Devolution War} has not already been
finished. Otherwise re-roll.

\phevnt
\aparag \FRA may immediately annex one of the following provinces:
\provincePicardie, \provinceRosselo, \province{Franche-Comte},
\provinceLuxemburg, \provinceAlsace or \provinceLorraine.  Such annexation is
regarded as a free declaration of war against the owner of the province chosen
(unless eliminated in the process).
\aparag If \FRA uses this opportunity to annex a province, \HOL and \ENG
receive a temporary free \CB against \FRA for this turn. They do not
necessarily have to be in alliance with the victim of French aggression or
with each other (but they may decide so if both sides agree).



\event{pV:Glorious Revolution}{V-3}{The Glorious Revolution in
  England}{1}{PBMod}

\history{1688-1690}

\condition{}
\begin{todo}
  If \ANG is \CATHCR?
\end{todo}
\aparag If \ENG is \PROTRIG:
\bparag Put a \REVOLT \facemoins in each Irish province except \provinceUladh,
one \LD and one general in one of the revolted provinces. \ENG is not in Civil
war, the \REVOLT are controlled by \HOL.
\bparag \paysEcosse declares war on \ANG (breaking any alliance it may have
with \ANG) and call for allies as per normal rules.
\aparag Otherwise (\CATHCR, \CATHCO or \PROTANG), use the rest of the event.
% (Jym) What if \ANG is \CATHCR? (JCD) Rest of the event seems to support that
% the event is used anyway.

\phevnt
\aparag \ENG is considered to have overthrown its current monarch. \ENG is now
in Civil war between two sides: the Rebels, called ``Royalists'' (followers of
the old king) are \CATHCR , and the Loyalists, called ``Orange'' are \PROTANG
(see \ref{chDiplo:Religious Civil War}).
\bparag The Royalists are controlled by a Catholic \FRA, or \SPA
otherwise. They use the counters of \paysroyalistes.
% \bparag Roll for the statistics of the new English monarch from the House of
% Orange. If the same House is ruling in \HOL, both countries share an offensive
% alliance and a mandatory defensive alliance. They make an immediate white
% peace.
\begin{todo}
  \ANG choose be able to choose the order in which he propose the crown to
  other protestant countries.

  Clarify the rules for the union in case ``Orange'' is not \HOL.
\end{todo}
\bparag The loyalists are controlled by the English player and use the
counters of \ANG. They are automatically allied with the first country in the
following list who accept: \HOL, Protestant \FRA, \SUE. These countries are
allied as per (REF NEEDED, See Special Rule for \ANG) and immediately makes a
white peace.
\aparag In support of the overthrown monarch, two \REVOLT are rolled for in
England. Furthermore a \REVOLT \faceplus is placed both in \provinceConnacht
and \provinceMumhan and the rebels control both fortresses. A \LD and a
general are placed in one Irish province.
\bparag If this event is caused by \ref{pIV:English Restoration}, a royalist
\ARMY \faceplus is raised in \provinceCymru (or in any province of Scotland
if~\ref{pIV:Union Scotland} is effective). The Royalists control the fortress
in this province and one other (or two other provinces if in
Scotland). Otherwise, Royalists get an \ARMY\facemoins and control of the
fortress in this province and one other.
\aparag If~\ref{pIV:Union Scotland} is effective, \paysEcosse allies itself to
\paysroyalistes and is at war with the Loyalists (with no declaration of
war).

\phdipl
\aparag The controller of the rebels has a \CB against \ENG to make a limited
intervention against \ENG this turn, that can become a full intervention on
the second turn. If \ENG was \CATHCR or the event was caused by
\ref{pIV:English Restoration}, the controller may make a full intervention
from the first turn on.

\phadm
\aparag The Royalists roll for reinforcements in offensive or naval status
(but with \bonus{-2} for naval).
\aparag All reinforcements must be placed in a province with existing rebel or
allied units, not just \REVOLT or cities. If none, no reinforcements are
received.

\phpaix
\aparag Peace is determined with usual rules except that:
\bparag The Royalists surrender unconditionally if they have no forces nor
\REVOLT left (fortresses do not count).
\aparag If the the new English king is overthrown by \REVOLT , it also
surrenders unconditionally to the Royalists and their controller.
\aparag[Victory of Royalists] If the Royalists win (alone or with their
controller), the king is restored (with his values as a monarch) and the House
of Orange is expelled.
\bparag \ENG becomes \CATHCR (except if it was \CATHCO, in which case it
remains so). It loses 50 \PV.
\bparag \xnameref{pVI:Act Union} is broken. If it did not happen yet, it may
occur later.
\aparag[Total Victory of Royalists] If the Royalists and their controller
(making a full intervention) impose an unconditional surrender to \ENG,
additional consequences are:
\bparag \ENG makes a mandatory Dynastic Alliance with the controller of the
Rebels and must give a \COL or \TP as dowry.
\bparag \xnameref{pVI:Act Union} is broken. If it did not happen yet, it may
not occur later (with some modifications).%  TODO This paragraph is subject to
% debate
% (Jym) Total defeat allows the Act of Union but not partial defeat?  (JCD)
% Yes, WTF ? The Act of Union is more straightforward about that
%
% (Jym, 2013) Pierre's notes from 2008 seem to be in the opposite
% direction...
\bparag \ENG makes a mandatory offensive alliance with the controller of the
rebels for 2 turns. It cannot declare war against it (except with \CB from
events; in this case the alliance has to be broken with the usual cost in
\STAB).



\event{pV:WoSS}{V-4}{The War of Spanish Succession}{1}{PBMod}

\history{1700-1713}

\activation{}
\aparag This event cannot occur before period V. Re-roll and do not mark off
if this is not the case.
% ELSE: before the end of the Religious struggles.
\aparag When the event occurs, its effects are not actually applied. They will
be triggered at the death of the current Spanish Monarch.
\aparag If there is a \GE, see the specific modifications in
\ref{pIV:TYW:German Empire}.

\tour{Death of the Monarch}

\phevnt
\aparag \SPA may concede immediately white or losing peace to all its current
enemies. Unaligned \MIN always accept a white peace.

\phdipl
\aparag \SPA designates an heir to the Spanish throne. The choice must be made
among the following countries:
\bparag One of the following \MAJ that is Catholic: \FRA, \AUS, \ENG;
\bparag \AUSMin;
\bparag Another Catholic minor country.

\aparag A \MAJ may decline the offer, but cannot then take part in any war
ensuing from this event, nor can it be positively affected by the event (for
objectives or any possible gain in the event).
\bparag In that case, \SPA proposes a different Heir, and so on, until one
accepts (minor powers always accept).
\bparag The power that accepts will be designated as the \emph{Heir} in the
rest of the event.
\bparag If the Heir is a minor power, all its decisions are made on its behalf
by \SPA.

\aparag If \AUSaus is not the chosen Heir, the dynastic alliance between the
Habsburg powers is now cancelled.
\bparag \AUSMin becomes also the major \AUS.

\aparag The Heir has to propose a settlement for the Spanish
possessions. Three attitudes are possible:
\begin{itemize}
\item \xnameref{pV:WoSS:Integrity Inheritance}
\item \xnameref{pV:WoSS:Seizing Inheritance}
\item \xnameref{pV:WoSS:Dividing Inheritance}
\end{itemize}

\aparag Several parts of the Inheritance are desired by some Major Powers.
Here is the list of the different parts at stake, especially the regional
groups for all province owned by \SPA that are not in its National territory
and the \MAJ that can be nominated for receiving these parts:
\bparag[Spanish Low Countries] In national territory of \paysmajeurHollande or
in former country \paysBourgogne except for \province{Franche-Comte}.
Interested: \FRA, \ENG, \AUS, \HOL, \SPA.
\bparag[South Italy] Provinces of Kingdom of Naples and Sicily (\paysNaples).
Interested: \FRA (if Catholic), \ENG (if Catholic), \AUS, \SPA.
\bparag[North Italy] All the remaining provinces in \regionItalie and
\payssuisse (except \provinceNice) plus \provinceMalta.  Interested: \FRA (if
Catholic), \AUS, \SPA.
\bparag[French Borders] All the provinces adjacent to or in French National
Territory that are not in one of the previous groups (that includes
\provinceNice, \province{Franche-Comte} and \provinceRoussillon).  Interested:
\FRA, \AUS, \SPA.
\bparag[North Africa] All provinces and \Presidios in North Africa.
Interested: \FRA (if Catholic), \ENG, \SPA.
\bparag[The Remaining] All other European provinces owned by \SPA that are not
its National Territory.
\bparag[Mediterranean Concessions] \provinceGibraltar, \provinceBaleares and 1
\COL (of \HIS or a major heir).  Interested: \ENG, \HOL, \SPA.
\bparag[Dynastic link and alliance with Portugal] This can only be chosen if
\paysportugal is either annexed by \HIS as per \ref{pIII:POR Ann.:Portugal
  Annexed} or if \ref{pVI:Methuen:Normal} did not happen yet and \paysportugal
is on the diplomatic track of \HIS. Apply immediately \ref{pVI:Methuen:WoSS}
with the \MAJ taking this spoils has the beneficiary of the Treaty and
consider that event played. Interested: \FRA, \ENG, \HOL, \SPA.
% (Jym) réécriture pour tenter la compatibilité avec le traité de Methuen...
% (Jym) Formulation alambiquée...  Si Methuen a eu lieu, a provoqué une 2ème
% révolte de POR que SPA a re-gagné, cette formulation laisse la possibilité
% de le voler maintenant...  (Pierre) : Applicable only if \ref{pVI:Treaty
% Methuen} is not active yet, and \SPA has still dynastic ties with \POR. For
% \SPA, this is the reaffirmation of Dynastic Ties and Pretense over \POR. For
% other \MAJ, this condition voids the effect of the Spanish Dynastic Ties
% with \POR and it activates an Alliance analog to the Treaty of Methuen for
% the \MAJ that takes it. Interested: \FRA, \ENG, \HOL, \SPA.
\bparag[Asiento] See \ref{chSpecific:Spain:Asiento}. Interested: \FRA, \ENG,
\HOL, \SPA.
% (Jym, 2013) If I understand Pierre's notes correctly:
\bparag[Colonial Empire] Two \COL of \HIS or the heir (if \MAJ). Interested:
\ANG, \FRA, \HOL, \HIS.

\aparag The attitude chosen gives the Heir some constraints on the Inheritance
project (which groups are attributed to which power).
\bparag Note that for \AUS, some groups count only as half: \emph{North
  Italy}, \emph{South Italy}, \emph{French Borders}.

\begin{digressions}[The Inheritance Project]
  \bgroup\def\EUEVtypeofdigression{choice}


  \digression[pV:WoSS:Integrity Inheritance]{Integrity of the Inheritance.}
  \phase[(Diplomatic before the war, Peace after the war)]{Diplomatic or Peace
    phase}{}
  \aparag The Heir decides to keep all provinces Spanish.
  \aparag The Heir obtains a compulsory offensive alliance lasting 5 turns
  with \SPA. \SPA must always honour this alliance, if called to do so. It
  cannot make a separate peace from the Heir, unless compelled to do so by
  enforced surrender. It is also considered as a Dynastic Alliance.
  \aparag The Heir may take one of the following advantage: \terme{Dynastic
    link and alliance with Portugal}, \terme{Asiento}, \terme{Mediterranean
    Concessions} or \terme{North Italy} (\AUS only for this last one) if
  interested.
  \aparag Then \SPA cedes two provinces of its choice to the Heir.


  \digression[pV:WoSS:Seizing Inheritance]{Seizing the Inheritance.}
  \aparag The Heir takes any or all the groups at stake defined above as
  interesting him.
  \aparag The Heir obtains a compulsory offensive alliance lasting 3 turns
  with \SPA. \SPA must always honour this alliance, if called to do so. It
  cannot make a separate peace from the Heir, unless compelled to do so by
  enforced surrender. It is also considered as a Dynastic Alliance.


  \digression[pV:WoSS:Dividing Inheritance]{Dividing the Inheritance}
  \aparag The Heir decides to share the spoils of the Spanish possessions with
  other Powers. It may propose any/all of the groups above to Powers that have
  interest in the share, and can take some of them for its own sake.
  \bparag Choosing this option costs 1 \STAB to the Heir plus 1 \STAB per part
  of the inheritance given to someone else than \HIS or the heir, as well as
  15\VPs per part given to someone else than \HIS or the Heir (due to its
  bargaining of the Heirdom) (or to \SPA of the Heir is a \MIN). The \STAB has
  to be paid, if the heir (or \HIS) has not enough \STAB, it may not give more
  parts.
  \bparag Each power may obtain at most two groups.
  \aparag The Heir obtains a compulsory defensive alliance lasting 3 turns
  with \SPA.  \SPA must always honour this alliance, if called to do so, yet
  it can make separate peace if it wants. It is also considered as a Dynastic
  Alliance.
  %
  \egroup
\end{digressions}

\begin{digressions}[Conditions of the War of Spanish Succession]


  \digression[pV:WoSS:War Spanish Succession]{War of Spanish Succession}

  \phdipl
  \aparag Some powers (if not chosen as Heir) may want to contest the
  Inheritance and declares a War to both \SPA and its Heir, jointly: \HOL,
  \FRA, \AUS, \ENG.
  \bparag They have a free \CB to do so.
  \bparag All the powers contesting the Inheritance are automatically in the
  same Alliance, called the Opposing Alliance.
  \bparag As par usual rules, other \MAJ may be called to participate in one
  or the other Alliance.
  \bparag If the Heir is a minor power, \SPA leads the Heir alliance and a
  Separate Peace against this minor does not affect the war.
  \bparag If the Heir is a major power, it decides for the Alliance (excepted
  if out of the war before \SPA).

  \aparag If none contest the Inheritance, this ends the event and the Heir
  and \SPA are deemed to have won the War, and all the other powers to have
  lost it.

  \aparag If \ref{pV:WoSS:Dividing Inheritance} has been taken, a power to
  whom at least one group has been proposed has the choice, in case there is a
  war, to contest the Inheritance (as per above), or to support the Division
  and join the Heir Alliance. In that case, it has to declare war and has a
  \CB to do so.

  \aparag If there is a war, any country that is not in one of the Alliances
  forfeits all possible benefits due to the war.

  \aparag The Heir, \SPA and the \MAJ in their Alliance take all the groups
  they are entitled by the chosen Inheritance attitude immediately. Those
  gains are temporary in the sense that they may revert to other powers
  depending on the result of the war. The Opposing Alliance powers will
  receive nothing before the end of the war.

  \aparag[Maximilian's change of side] [BLP] If the Heir is not the emperor
  and there is a war, the Heir may choose one electorate. For the duration of
  the war, he has a bonus of \bonus{+5} for diplomacy on this
  minor. Exceptionally, diplomacy may be made on this minor even if it is at
  war.

  \phadm
  \aparag For the duration of the event, \ENG receives the use of the leader
  \leader{Royal Marines}. This is in addition to the normal limits.


  \digression[pV:WoSS:Peace Spanish Succession]{Peace following Spanish
    Succession}

  \phpaix
  \aparag The result of the war depends of the level of the peace signed
  between the Alliances. The War ends when \SPA or its Heir is making Peace
  and the other is doing the same or is already out of the war.
  \aparag In this Peace, the victory condition is first the application (or
  not) of the proposed Inheritance project, second the giving of some of the
  groups presented before as compensations. To them, one adds the following
  groups (that are spoils for war only):
  % (Jym) Adding:
  \bparag[Dynastic link and alliance with Portugal] At the peace, this can
  also be chosen if \paysportugal was given to a country in the opposing
  alliance at the beginning of the war. It is not possible to choose this
  compensation at peace if \ref{pVI:Methuen:Normal} was triggered as a regular
  event and gave the Portuguese alliance to a country other than \HIS.
  % (Jym) Remet complètement l'alliance portuguaise en jeu à la fin de la
  % guerre, comme les autres compensations. Précautions à prendre pour que ça
  % ne soit remis en cause que si ça avait été donné pour "acheter" un MAJ au
  % début de la guerre. Si l'alliance est antérieure à la guerre, elle ne peut
  % plus bouger.
  \bparag[Territorial Concessions] Give any two provinces to any power (only
  province not in a group given to anyone, except \SPA). In priority:
  provinces adjacent to provinces already owned by the \MAJ.  Interested:
  \FRA, \ENG, \AUS.
  \bparag[Independence of Catalunya] Only if a \REVOLT or the Opposing
  Alliance controls \provinceCatalogne: it becomes an independent minor
  country.  Counts as half an objective only.  Interested: \FRA, \ENG.
  \bparag[Olivares politics cancelled] This nullifies the effects of
  \ref{pIV:Olivares}.  Counts as half a group objective only.  Interested:
  \FRA, \ENG, \HOL.
  \aparag If the Heir Alliance is victorious, with a PD of 3 or more: the
  proposed Inheritance project is applied completely.
  \aparag If the Heir Alliance is victorious, with a PD of 1 or 2: the
  proposed Inheritance project is applied but the Heir has to give a group as
  a compensation to one of the \MAJ in the enemy alliance (chosen by the
  Heir).
  \aparag If a white Peace is signed: the proposed Inheritance project is
  applied but the Heir has to give two groups as a compensation to \MAJ in the
  enemy alliance (proposed by the Heir).
  \aparag If the Opposing Alliance is victorious, with a PD of 1: the proposed
  Inheritance project is applied but the Heir has to give two groups as a
  compensation to \MAJ in the enemy alliance (chosen by the Opposing
  Alliance).
  \aparag If the Opposing Alliance is victorious, with a PD of 2: the proposed
  Inheritance project is not applied. The Opposing Alliance decides of a new
  Inheritance project based on the rules of \xnameref{pV:WoSS:Dividing
    Inheritance} that is applied and cannot be contested.
  \aparag If the Opposing Alliance is victorious, with a PD of 3 or more: the
  proposed Inheritance project is not applied. The Opposing Alliance decides
  of a new Inheritance project based on the rules of
  \xnameref{pV:WoSS:Dividing Inheritance} that is applied and cannot be
  contested. The restriction that at most 2 groups may be given to a power is
  lifted.
  \aparag If \HAB was the Heir and the Inheritance project is overruled, the
  Dynastic Alliance between the Habsburg ends and \AUSmin becomes \AUS.
  \aparag If \SPA is victim of an Unconditional Peace, the new dynasty is
  overthrown.
  \bparag The Heir loses 30 \PV and the Dynastic Alliance is cancelled.
  \bparag \SPA lose all the groups at stake in the Inheritance.
  \bparag If the war still goes on, they are temporarily given to the Heir
  until the end of the war.  If the Heir wins the war anyway, any group that
  should have been attributed to \SPA is considered to be his before applying
  the Peace conditions.  If there are groups he is not interested into that
  are still his afterwards, he has to freely give them to any power (including
  \SPA, as an exception to this rule and the following).
  \aparag If a power makes a Separate Peace, it forfeits all the possible
  benefits to be gained in the war (all the groups mentioned before).
  \bparag If it already had any (thanks to a Division of Inheritance), the
  objective are given back to \SPA (or the Heir if \SPA is out of the war).
  \aparag If, at the end of the war, \provinceNaples is owned by someone else
  than \HIS (or an autonomous \VASSAL of \HIS), then \paysSavoie annex
  \provinceSaldigna
  \begin{todo}
    It should be both provinces of Sicily, exchanged for Sardinia
    after~\ref{pVI:WoQA}.
  \end{todo}

  \effetlong
  \aparag \provinceGibraltar becomes an \terme{arsenal} if attributed and
  owned by this event to a player that is not \SPA.
\end{digressions}



\event{pV:Colbertian Mercantilism}{V-5}{Colbertian Mercantilism in
  France}{1}{RistoMod}

\history{1661-1683}

\condition{}
\aparag \FRA may decline the event if he wants so. Mark off the event as
played and ignore the rest.

\phevnt
\aparag \FRA receives an Excellent Minister \strongministre{Colbert} with
values 8/9/8. He will last a random length for Minister, see
\ref{eco:Excellent Minister}.
\aparag All major powers with commercial fleets in \ctz{France} must pay 10
\ducats per level they want to keep. The money goes to French treasury. All
minor commercial fleets in \ctz{France} are permanently removed (their
reference level is 0).
\aparag Moreover, if either \CATHCR or \CATHCO, \FRA receives 5 levels of
\TradeFLEET in \ctz{France}.  Mandatory competition is solved immediately if
need be.
\aparag All major powers who lost at least two levels of \TradeFLEET in the
process have an \OCB against \FRA until the end of the next period.
\aparag \FRA receives an additional \RES{Art} \MNU level, if available.

\phdipl
\aparag For the rest of the game, \FRA has an \OCB against everyone with
\TradeFLEET in \ctz{France}, and a \CB against a power having a
\TradeFLEET\faceplus in this \CTZ.

\phadm
\aparag\label{pV:Colbert:limits} Some French turn and period limits and basic
forces are raised during some periods.
\aparag As long as \ministreColbert is Minister, \FRA increases by half its
basic naval construction limit.
\aparag From now on, all new non-French \TradeFLEET levels placed in
\ctz{France} cost 10\ducats tax to be payed directly to French Treasury at the
moment such fleet levels are placed on the map.
\aparag\label{pV:Colbert:CB} \FRA receives a permanent additional bonus of
\bonus{+5} for all competition attempts it makes in \CTZ France. However there
is no malus for making competition attempts against \FRA.
% \begin{oldcompta}
%   \aparag From now on, \FRA may begin the construction of Versailles.
% \end{oldcompta}
\aparag \FRA may ignore restriction of~\ref{chExpenses:Pioneering} for this
turn (only).

\phpaix
\aparag The permanent tax implied in the event and the \CB can be later
annulled and \ministreColbert dismissed by scoring an unconditional victory
against \FRA and claiming their annulment in place of the taking of one
province. \FRA retains the other benefits (\numberref{pV:Colbert:limits},
\numberref{pV:Colbert:CB}).



\event{pV:Expulsion French Protestants}{V-6}{Expulsion of the French
  Protestants}{1}{PBNew}

\history{1685}

\condition{}
\aparag If \FRA is Protestant, roll for one (Catholic) \REVOLT in France and
consider the event as played (mark off, do not reroll).
\aparag If \FRA is \CATHCO, it can refuse the event and loses 3 \STAB and 10
\PV.
\aparag If \FRA is \CATHCR, it can refuse the event and loses 4 \STAB and 30
\PV.
\aparag If \FRA refuses the event, it can no more use \CB given by events
\shortref{pV:Glorious Revolution} and \shortref{pVI:Jacobite Rebellion}.

\phevnt
\aparag \FRA loses 1 level from both its current \FTI and \DTI.
\aparag The first protestant in the following list of precedence:
\HOL/\ENG/\SUE, gains one \TradeFLEET level of its choice taken from a
\TradeFLEET fleet in a \STZ where both countries are present (does not apply
if none available) and two free \COL attempt with strong investments.
\bparag The country receiving these actions may ignore restriction
of~\ref{chExpenses:Pioneering} for this turn.
\bparag If there is no Protestant power, \FRA loses one \TradeFLEET of his
choice.



\event{pV:Grand Siecle}{V-7}{``Le Grand Si\`ecle''}{1}{PBNew}

\history{1661-1702}

\phevnt
\aparag \FRA chooses, when all events of this turn have been drawn, to apply
one of the following events (that did not happen yet): \ref{pV:Devolution
  War}, \ref{pV:Chamber of Reunion} (or \ref{pV:League Augsburg} if it already
happened), \ref{pV:Colbertian Mercantilism} or \ref{pV:Expulsion French
  Protestants}.

\phadm
\aparag \FRA may ignore restriction of~\ref{chExpenses:Pioneering} for this
turn (only).



\event{pV:English Dynamism}{V-8}{English Dynamism}{1}{PBNew}

\history{}

\phevnt
\aparag \ENG chooses, when all events of this turn have been drawn, to apply
any one of the following events (if it did not happen yet): \ref{pIII:East
  Indian Company}, \ref{pIV:London Stock Exchange}, \ref{pIV:Act Navigation},
\ref{pVI:Treaty Methuen}.

\bparag The chosen event must be playable (no more than 1 period before or
after the current one).

\aparag In addition, \ENG has one free \OCB against \HOL, to be used before
the end of the period.

\phadm
\aparag \ENG may ignore restriction of~\ref{chExpenses:Pioneering} for this
turn.



\event{pV:Montecuccoli to Eugen}{V-9}{From \sectionleader{Montecuccoli}
  to \sectionleader{Prinz Eugen}}{1}{PBNew}

\history{1645-1700}

\phevnt
\aparag Depending on the current turn, check if the following general is still
in play ; if he is not, recall him immediately (even if he is dead: he was
only severely wounded and retired but the military situation require his
presence):
\bparag \leaderPappenheim between turns 28 and 32 (inclusive) ;
\bparag \leaderMontecuccoli between 33 and 38 (inclusive) ;
\bparag \leader{ER Starhemberg} between 39 and 41 (inclusive) ;
\bparag \leader{Prinz Eugen} between 42 and 49 (inclusive).
\aparag Armies of \AUSaus are now of class \CAIV.
\aparag \AUSMin now has a Land Technological marker that increases of two
levels each turn, beginning on the Latin level.
% (Jym) 05/2013: commented out in Pierre's notes from 2007...
% \bparag On turn 41, \HAB automatically take the \TMAN technology.



\event{pV:de Witt}{V-10}{\ministre{de Witt}}{1}{Risto}

\history{1653-1672}

\condition{\HOL can refuse this event if it wishes so. In that case mark off
  as played.}
\aparag \HOL can freely dismiss \strongministre{de Witt} (if Minister) at the
end of any following monarch survival phase and the event terminates.

\phevnt
\aparag \HOL receives a personality \ministre{de Witt} who may be used as
Monarch of a \terme{Parliament} government, or an excellent minister of a
\terme{Stadhouder} government, with values 9/7/9.  He will last for a random
length for Minister, see \ref{eco:Excellent Minister}.
\aparag During the last two turns of \ministre{de Witt}'s term in office (be
it Monarch or Minister), add \bonus{+1} to the monarch survival test.  If the
monarch dies during these two turns, \ministre{de Witt} is also removed and
this terminates the event before the new monarch is chosen.

\phadm
\aparag \HOL basic forces are increased by \FLEET\facemoins and \ARMY\faceplus
during every turn if is engaged in a war (Overseas, limited or full-fledged)
as long as \ministre{de Witt} is minister or monarch.



\event{pV:Peter the Great}{V-11}{Peter the Great}{1}{Risto}

\history{1689-1725}

\condition{}
\aparag If this is period \period{IV} and~\ref{pIV:Times of Troubles} is not
finished, do not mark off and reroll.
\aparag If \monarque{Peter the Great} was already received, nothing happens
  with this event (do not apply \RD instead).

\phevnt
\aparag The heir of the current monarch of \RUS is automatically
\monarque{Peter the Great} with values 9/9/9. See \ref{chSpecific:Russia:Peter
  the Great}.

\phadm
\aparag \RUS may ignore restriction of~\ref{chExpenses:Pioneering} for this
turn.



\event{pV:Saxon King Poland}{V-12}{Augustus II, a Saxon king in
  Poland}{1}{Risto}

\history{1697-1733}
\dure{Until there is a change of dynasty in \POL.}

\condition{}
\aparag If \POL is Orthodox or \CATHCR, the event is ignored. Mark off and
play \RD with the \REVOLT in \POL instead.
\aparag If \POL is at war against \paysSaxe, the event is ignored. Do not mark
off and re-roll.

\phevnt
\aparag The king of \POL is replaced, if it is a named general, he stay to
serve \POL as a general, otherwise, he is removed from the game. The new king
is \monarque{August II}, elector of Saxony.
\bparag He is scheduled to last for 7 turns.
\bparag His value are randomly chosen like after a \terme{Dynastic Crisis}.
\bparag \monarque{August II} may not be used as a general.
\bparag This is a change of dynasty in \POL.
\aparag \paysSaxe becomes a permanent \VASSAL of \POL as long as the event
lasts.
\bparag No diplomacy is allowed on \paysSaxe while the dynasty rules in \POL.
\bparag \paysSaxe is considered to be part of \POL for declaring wars of
signing peace (no separate peace is allowed, \ldots)
\aparag Any war against either \paysSaxe or \POL when the event occurs
immediately becomes a war against both (without formal declaration of war).

\phadm
\aparag \paysSaxe still get reinforcements as a minor country when at war. Its
troops can freely cross the \HRE and \POL. \POL can raise extra troops from
\paysSaxe (German mercenaries).
\aparag Troops of \POL do not get extra rights to enter countries of the \HRE
(however, \paysSaxe is always allied).
% (Jym) Not enter Saxony ? or \HRE ? After TYW ?  (Pierre) Polish forces are
% not allowed to enter Saxony or a country of the HRE that is not allied or at
% war with Poland. They are managed normally according to the status of Poland
% (minor or major). (JCD) I rewrote from the French version I leave former
% English version in comments
%
% \aparag Troops of \POL may not enter \paysSaxe or any other country of the
% \HRE which is not allied or at war against \POL.

\phpaix
\aparag Only an unconditional surrender can force either \POL or \paysSaxe to
a separate peace.
\bparag In this case, the losing country cannot enter the same war again but
the alliance between \POL and \paysSaxe is still in effect.

\effetlong
\aparag As long as the dynasty of \paysSaxe rules in \POL, the king can try to
impose Absolutism at the conditions of \ref{pIV:Polish Civil War}.
\bparag This can be done at the beginning of the second turn of reign of
\monarque{August II} and then whenever a new king (of the dynasty of
\paysSaxe) rules \POL.
\bparag This must be announced at the beginning of the event phase,
\numberref{pIV:Polish Civil War} is considered to be the first event rolled
for this turn.



\event{pV:Kingdom Prussia}{V-13}{Creation of the Kingdom of
  Prussia}{1}{RistoMod}

\history{1701}

\condition{If \ref{pIV:Great Elector} as not been played yet, mark off and
  play \numberref{pIV:Great Elector} instead.}
\begin{todo}
  Should be \PRUpru instead of \paysBrandebourg.
\end{todo}
\phevnt
\aparag If \POLpol still owns provinces of \region{Duche de Prusse}, they are
immediately annexed by \paysBrandebourg. \POL gets an immediate free \CB
against \paysBrandebourg.
\aparag \paysBerg is annexed by \paysBrandebourg.
\bparag Another country owning \paysBerg, either renounces it (and gives it to
\paysBrandebourg), or is declared war upon by \paysBrandebourg.

\effetlong
\aparag Basic forces of \paysBrandebourg are now 2 \ARMY\faceplus, one general
and 3 levels of fortification.
\bparag Its counters limit becomes 3 \ARMY and 5 \LD and its basic
reinforcement becomes 2 \LD.
\aparag Troops of \paysBrandebourg can freely cross the \HRE even if not at
war, in the same way the Emperor can.
\aparag The Elector of \paysBrandebourg wants to become king. This happens as
soon as one of the following condition is true:
\bparag The emperor grants the royal crown. \paysBrandebourg is put in \EC of
the Emperor (usually \HAB).
\bparag The country of the Emperor gives a unfavourable peace to
\paysBrandebourg. Instead of one peace conditions, \paysBrandebourg gets the
royal crown.
\bparag The Emperor signs an unfavourable peace of level 3 or more against
anyone. \paysBrandebourg takes the royal crown and the emperor has a free \CB
against it at the following turn.
% (Jym) free CB unique (for next \phdiplo) or permanent but usable only once?
\aparag Whatever the condition, the emperor loses {\bf 1} \STAB when
\paysBrandebourg becomes the kingdom or Prussia (the minor country is still
called \paysBrandebourg).
% (Jym) Maybe possibility for Emperor to take back the crown, as with an
% unconditional or 3+ peace?



\event{pV:War Sweden Denmark}{V-14}{War between \paysmajeurSuede and
  \paysDanemark}{1}{PB}

\history{1675-1679}

\phevnt
\aparag \DANMin and \PRUmin, if inactive, declare war to \SUE.
\aparag \PRU as a major country has a \CB against \SUE. If it doesn't use this
\CB, it loses 1 \STAB and the control of \paysDanemark. If it uses this \CB,
it gains \paysDanemark in \EG.
\aparag \DAN as a major country has a \CB against \SUE. If it doesn't use this
\CB, it loses 1 \STAB and the control of \PRUmin. If it uses this \CB, it
gains \PRUmin in \EG.
\aparag Normal call for allies occur. Especially, a major country with
diplomatic control (\AM or better) of either \DANmin or \PRUmin is called by
the minor.
\aparag \SUE does lose diplomatic control of both \paysDanemark and
\paysBrandebourg.



\event{pV:Koprulu}{V-15}{\ministreKoprulu}{1}{RistoMod}

\history{1656-1683}

\condition{\TUR can refuse this event if it wishes so. In that case mark off
  as played.}
\aparag If \TUR has performed any reform of level 2, mark off and play \RD
instead, with the \REVOLT in \TUR.
\aparag \TUR can freely dismiss \strongministre{Koprulu} at the end of any
following monarch survival phase and the event terminates.

\phevnt
\aparag \TUR receives an Excellent Minister \ministreKoprulu with values
8/9/7. He will last for 8 turns.
% 2 turns plus a random length for Minister, see \ref{eco:Excellent Minister}.
The Minister is not dismissed if the \TUR monarch dies ; \TUR rolls for the
values of the new monarch using the values of the Monarch only with no malus
nor bonus.
\aparag \TUR receives an additional level of \MNU of \RES{Metal}.
\aparag Four corrupted pashas may be removed immediately with no penalty.
\aparag \leader{Sadri Azam} is replaced by \leaderKoprulu while the event is
in effect. If this general is killed, captured or defeated in a Major Victory,
\TUR loses two additional \STAB or may choose to end immediately the event. If
the event is not ended, the general comes back in play (another one in the
same dynasty) on the following turn.

\phadm
\aparag Turkish Reforms cannot be attempted while the event is in effect.



\event{pV:Fights Iroquois}{V-16}{Fights against the Iroquois}{1}{Risto}

\phevnt
\aparag Roll 1d10:
\bparag If the result is even, \paysIroquois declares an Overseas war to one
power that has a \COL/\TP adjacent to them (this \COL/\TP is chosen randomly
to decide which power is the target). It will first try to invade this
settlement, and will go against the other ones of the same country only if
this one is captured/destroyed.
\bparag If the result is odd, the natives of a randomly chosen \COL of a major
power (including annexed Portugal) in an unsubdued area in \continent{North
  America} are activated and will attack this \COL at the end of the turn.



\event{pV:Slave Revolts WI}{V-17}{Slave Revolts in the West Indies}{2}{Risto}

\phevnt
\aparag Roll 1d10 for each power having \COL in areas \granderegionCuba,
\granderegionHaiti and/or \granderegionAntilles. On a result of 7 or more, a
\REVOLT \facemoins is placed in one randomly chosen \COL of the power.



\event{pV:Wars India}{V-18}{Wars in India}{2}{PBNew}

\history{Aurangzeb (1658-1707) / Revolts of the Marathi}

\phevnt
\aparag If the non-European minor country \paysMogol does not exist, it is
created now. Its ruler is now \leader{Grand Moghol} (replacing \leaderAkbar if
he was in play).
\aparag If it was still existing, minor country \paysVijayanagar is destroyed
(by internal fights).
\aparag \granderegionBengale and \granderegionKarnatika becomes rich region,
with 2 resources of each kind shown on the map (instead of 1).
\aparag If the \paysMogol exist, they invade one province with a modifier of
\bonus{-2}, the next in the list according to the event \ref{pII:Mughal
  Expansions}.
\aparag From now on, \paysMysore and \paysHyderabad are created as soon as no
other country owns their region.
\aparag Every \TP/\COL in \continentIndia that is in a region owned by a minor
country will face an attack by the natives of the area (disregarding the
existence or not of a Treaty). Attacks caused by this event will be resolved
at the end of turn with a modifier of \bonus{+4}.



\event{pV:Treaty Nerchinsk}{V-19}{The Treaty of Nerchinsk}{1}{RistoMod}

\history{1689}

\phevnt
\aparag \paysChina annexes all provinces in \granderegionAmour, and all
provinces adjacent to Mongolia (the white zone) in \granderegionBaikal. Its
Activation level is 6 in these provinces.
\aparag \RUS and any power having \COL/\TP in any of these provinces may now
make diplomacy on \paysChina in order to obtain \dipAT with it. This Treaty
allows the power to have at most 2 \COL/\TP that will draw no reaction from
\paysChina.
\aparag It is not possible for one power to have a \dipAT status for this
effect, and another one for a \TP in \paysChina. It is one or the other.

\phadm
\aparag \RUS may ignore restriction of~\ref{chExpenses:Pioneering} for this
turn.



\event{pV:Invasion Formosa}{V-20}{Invasion of Formosa by China}{1}{RistoMod}

\history{1683}

\begin{todo}
  Add test depending on situation and possibility of failure?
\end{todo}

\phevnt
\aparag \paysChina invades \granderegionFormose.  This province is now owned
by \paysChina and subjected to all the relevant rules. Activation level for
this province is 6.
\aparag Any foreign \TP/\COL in the region will be attacked by the Natives of
the province this turn.
% \aparag If \ref{pIII:CCA:Closure China} is in effect, a power still having a
% \TP in the province may sign a \dipAT with \paysChina. If it succeeds (before
% its \TP is destroyed), it may keep the \TP as if it were there at the moment
% of the closure.
\aparag If a \TP has survived, \pays{Chine} concedes a new \dipAT to the owner
of the \TP, if it didn't have any. The owner still has to pay as for usual
\dipAT with \paysChine.



\event{pV:Japan Trade}{V-21}{Trade Regulations in Japan}{1}{PB}

\history{1638 and afterwards}

\phevnt{}
\aparag If \ref{pIV:JCA:Closure Japan} happened, reduce any \TP in Japan by 2
levels.
\aparag If \ref{pIV:JCA:Japan Commercial Dynamism} happened or none of
\ref{pIV:Japan Colonial Attitude}, apply now \ref{pIV:JCA:Closure Japan}.



\event{pV:Revolt Cossacks}{V-22}{Revolt of the Cossacks}{1}{PB}

\history{1654-1667}

\condition{This event is the same as \ref{pIV:Revolt Cossacks} which happens
  now if it did not occur yet. Else, treat as \RD, with \REVOLT in \POL.}



\event{pV:Revolt Catalunya}{V-23}{Revolt in \provinceCatalunya}{1}{PBNew}

\history{1640-1652 / 1705-1707}

\phevnt
\aparag Place a \REVOLT \facemoins in \provinceCatalunya ; the \REVOLT
controls also the fortress. Any military force in the province must retreat.
\bparag If this event happens during \ref{pV:WoSS}, the \REVOLT is \faceplus
instead.
\aparag If \SPA is at war against \FRA, \ENG or \AUT, the \REVOLT is friendly
to the first of those countries that is an enemy of \SPA.

% Placeholder

\event{pV:Hungary}{V-s}{Revolt in \paysHongrie}{1}{PBNotEvenWritten}
\begin{todo}
  Probably a duplicate of \ref{pV:Kuruc}.

  Remove the army class change from \ref{pV:Montecuccoli to Eugen}.
\end{todo}

\phevnt
\aparag 4 (or 5 ?) random \HAB provinces in former territory of \paysHongrie
revolt: roll for strength at random.
\bparag The rebels are controlled by \TUR and friendly to \TUR.
\aparag \TUR has a free \CB against \HAB this turn.

\phadm
\aparag Armies of \AUSaus are now of class \CAIV.

\event{pV:Transylvania}{V-t}{Christian prince in
  \paysTransylvanie}{1}{PBNotEvenWritten}
\history{1648 (George II R\'{a}k\'{o}czi + Turkish Invasion)? / 1687
  (Transylvania recognise sovereignty of \HAB)? / 1699 (Treaty of Karlowitz)?}

\begin{todo}
  Maybe in early pVI.

  Maybe handled differently (Transylvania goes to owner of Buda).

  Maybe part of \ref{pV:Kuruc}.
\end{todo}

\phdipl If \paysTransylvanie is on the Diplomatic track of \TUR, it becomes
\Neutral.

\event{pV:Cretan war}{V-u (1)}{Cretan war}{1}{PBNotEvenWritten}
\history{1645-1669}

\begin{todo}
  hist. : 3 expeditions to the Dardanelles, \TUR annexes Creta, \VEN make
  small gains in Dalmatia.
\end{todo}

\event{pV:Morean war}{V-u (2)}{Morean war}{1}{PBNotEvenWritten}
\history{1684-1699}

\begin{todo}
  Morosini + conquest of Morea.

  Could be in early pVI also.
\end{todo}

\phpaix If \paysVenise sign a white or favourable peace, it annex and
additional province in \regionBalkans or Mediterranean island.

\event{pV:Revolt Pueblos}{V-v}{Revolt of the Pueblos}{1}{PBNotEvenWritten}
\history{1680}


\event{pV:Tangiers}{V-w}{Reconquest of Tangiers}{1}{PBNotEvenWritten}
\history{16??}

\begin{todo}
  Probably to remove. Should be handled by the diplo event where minor retake
  a presidio. Can be added to~\ref{pVI:Barbaresques} if needed.
\end{todo}

\event{pV:Khoikhoi}{V-x}{Khoikhoi-Dutch wars}{*}{RistoMoved}
\history{1659/1673/1674-1677}
\begin{todo}
  May replace \ref{pV:Slave revolt WI} in the table since this one has
  been moved in the \REVOLT table.
\end{todo}

\phevnt
\aparag Natives in \granderegion{Cap} W. province are activated with 2\LD and
a leader, whatever the printed value.

\event{pV:Bill Test}{V-y}{Bill of Test}{1}{RistoMoved}
\history{1673}

Same as~\ref{pVI:Bill Test}. Should be moved in pV.

\event{pV:Kuruc}{V-z}{The Great Kuruc Uprising}{1}{PBNotEvenWritten}

\history{1678-1684}[Part of the Great Turkish War (series of wars fought from
1662 to 1699) that lead to the famous (second) siege and battle of Vienna of
1684]
\dure{Until the end of the war caused by the event.}

\phevnt
\aparag[] [BLP] If \paysHongrie still exists, it is immediately destroyed and
all its provinces are annexed by \AUS. No \VPs are gained for these
annexations.

\aparag[] [BLP] \ref{chSpecific:Little war} is no more active.

\stopevents

% Local Variables:
% fill-column: 78
% coding: utf-8-unix
% mode-require-final-newline: t
% mode: flyspell
% ispell-local-dictionary: "british"
% End:

% LocalWords: Heirdom Minister malus reroll PBNew pIV pVI pV WoSS Colbertian
% LocalWords: Siecle Montecuccoli de Koprulu Nerchinsk Catalunya Kuruc Risto
% LocalWords: Franche PBMod TODO Habsburg Asiento Olivares RistoMod ecle Azam
% LocalWords: Starhemberg Stadhouder Duche Prusse Sadri Moghol LocalWords:
% PBNotEvenWritten
%  LocalWords:  Morean


\clearpage

% -*- mode: LaTeX; -*-
\definechapterbackground{Political Events of Period VI}{MorierBattleCulloden}
\chapter{Political Events of Period VI}
%\section{Period VI}
\label{events:pVI}



\subsection*{Event Table of Period VI}

\begin{eventstable}[Period VI events table]
  \tabcolsep=5pt\centering%
  \begin{tabular}{|l|*{5}{c}|l|}
    \hline
    1\up{st}\textarrow& 1-4 & 5-6 & 7 & 8 & 9 & 10 \\ \hline
    1 & 1  & 4  & 4  & R16 & 3   & \\
    2 & 2  & 5  & 18 & 17  & 4   & \textbullet~1--2 \\
    3 & 3  & 9  & R1 & 18  & 5   &  +1 then \\
    4 & 6  & 10 & 2  & 19  & 6   & \periodref{V}\\
    5 & 7  & 15 & 11 & 8   & R7  & \\
    6 & 8  & 16 & R12& R11 & 15  & \\
    7 & 11 & 17 & 13 & 12  & 9   & \textbullet~3--10: \\
    8 & 12 & 1  & 14 & R13 & R10 & \periodref{V}\\
    9 & 13 & R2 & 7  & 1   & R18 & \\ \hline
    10 & \multicolumn{6}{l|}{1--6 \periodref{VII}, 7--10 \periodref{V}} \\ \hline
  \end{tabular}
\end{eventstable}

\eventssummary{%
  pVI:Great Northern War|,%
  pVI:WoSS|O{pV:WoSS},%
  pVI:Kingdom Prussia|O{pV:Kingdom Prussia},%
  pVI:Jacobite Rebellion|S{pVI:Jacobite:First Revolt}/%
  S{pVI:Jacobite:Bonny Prince Charlie},%
  pVI:Act Establishment|,%
  pVI:Vassalisation Hanover|,%
  pVI:Treaty Methuen|,%
  pVI:Act Union|,%
  pVI:Bill Test|,%
  pVI:Heinsius|,%
  pVI:WoPS|,%
} \eventssummary{%
  pVI:War Turkey|E/E,%
  pVI:WoAS|,%
  pVI:Kurland|,%
  pVI:Slave Revolts WI|E/E,%
  pVI:Bantu Raids|E/E,%
  pVI:Last Great Mughals|,%
  pVI:Wars India|T{A}/T{B}/T{C},%
  pVI:Mazepa|,%
  pVI:WoJE|,%
  pVI:Comuneros|,%
  pVI:WoQA|,%
  pVI:Alberoni|,%
  pVI:Bulavin Rebellion|,%
  pVI:Africa|E/E,%
  pVI:Camisards|,%
  pVI:Barbaresques|,%
}

\newpage\startevents



\event{pVI:Great Northern War}{VI-1}{The Great Northern War}{1}{PBNew}

\history{1700-1721}
\dure{until the end of the war caused by the event.}
% (Jym) All \CB switched from \phevnt -> \phdipl. It looks cleaner to manage
% reactions at diplomacy time due to all \CB... the only difference is
% preventing the signing of alliances, but this does not really matter here
% since there are two turns to use the \CB... (JCD) I am not sure to follow
% this line of reasoning, but in fact, I don't think this matters much.

\phdipl
\aparag[Russian aggression of \SUE] \RUS has a free \CB against \SUE if they
have a common frontier.
\bparag This \CB can be used at this turn or the next one.
\bparag If \RUS does not use this \CB, it loses 2 \STAB at the end of the
diplomacy phase of the next turn. This becomes a loss of 3 \STAB during and
after the reign of \monarque{Peter the Great}.
\aparag[Polish aggression of \SUE] \POLpol has a normal \CB against \SUE if
they have a common frontier.
\bparag This \CB can be used at this turn or the next one.
\bparag \POLpol is affected by \xnameref{pVI:GNW:Polish Civil War}).
\bparag If \POL does not use this \CB, it loses 2 \STAB at the end of the
diplomacy phase of the next turn. This becomes a loss of 3 \STAB if either
\ref{pIV:Liberum Veto} never happened, or Absolutism has been established
(\ref{chSpecific:Poland:Absolutism}) or the dynasty of \payssaxe currently
rules \POL per \ref{pV:Saxon King Poland}.
\bparag This \CB can be used as a reaction to the \CB of \RUS above, or as a
regular \CB.
\bparag If there is a \POLmin (special or normal), apply \ref{pVI:GNW:Minor
  Poland}.
\aparag[Forfeit] If neither \RUS nor \POL use their \CB by the end of next
turn, consider the event played and \SUE is considered to have won the war for
all purposes (especially for the lasting effects).
\bparag If either \RUS or \POL are already at war against \SUE, either can
declare that they transform the war into this event. This is considered as
using the \CB provided by the event (with no \STAB cost in the case of \POL)
and triggers everything triggered by the use of the \CB.
\aparag[Swedish generalisation of the war] If one of \RUS or \POL uses its \CB
to declare war on \SUE, then \SUE has a free \CB against the other one.
\bparag This \CB is used as a reaction to the \CB of \RUS or \POL.
\bparag[Surprise aggression] As an exception, this \CB can be used at the
beginning of any military round of any turn of the war. In this case, the
country enters war without a call for allies.
\aparag[Prussian involvement] If \PRU is a major country, it has a \CB against
either \POL or \SUE (its choice).
\bparag This \CB can be used at the turn of the event or at the next
one. There is no penalty for not using it.
\aparag[Danish aggression] \paysDanemark may enter the war against \SUE (see
\xnameref{pVI:GNW:War Denmark}).
\aparag[Alliances] \RUS, \POL, \PRU or \SUE are not necessarily allied in the
war. They have to sign a formal alliance if they want to be allied.

\tour{After the war begins}

\phadm
\aparag At the first turn of the war (only), \SUE receive reinforcements as a
minor country. It makes one roll in offensive attitude \textbf{and} one in
defensive attitude.
\bparag These reinforcements are \Veteran. They do not count toward this turn
purchase limit.

\phmil
\aparag If the dynasty of \payssaxe rules in \POL, troops of \POL and \SUE can
cross the \HRE in order to wage war in \payssaxe.
\bparag No side may besiege or pillage provinces of the \HRE belonging to
countries not at war.
\aparag Troops of \SUE may enter provinces of \regionUkraine even if they
belong to a country not at war (they may thus trigger \ref{pVI:Mazepa}).
\bparag This gives a free \CB against \SUE to both the owner of the province
and the protector of \paysukraine to be used during the next turn.
% (Jym) In \PPI, the \CB was against enemies of \SUE if \TUR used it. Here,
% \TUR already has a \CB because of the revolt of \leaderMazepa and it gives a
% double \CB to \TUR (against either \RUS or \SUE). Easier to write like that

% (Jym), notes Pierre, small civil war even if absolutism:
\aparag Fortress owned by \POL and controlled by \SUE gives full supply to
\SUE.

\phpaix
\aparag[Starting the Revolt of Mazepa] If there is any \ARMY counter of \SUE
in any province of \regionUkraine at the beginning of the peace phase then
\ref{pVI:Mazepa} will occur next turn. Consider it as the first event rolled
for during the next event phase.
\bparag This revolt will occur even if the peace is signed at this turn. In
this case, the revolt is considered to have occurred at the very end of the
turn, before signing the peace.
\aparag If \SUE signs no unfavourable peace for this war (including if the war
does not occur), it immediately wins 50 \VP.

\effetlong
% (Jym) Err, how about the overseas choice ?  (JCD) Does not matter, the event
% is more specific. Let the third army become a totally normal army.
\aparag If \SUE signs no unfavourable peace for this war (including if the war
does not occur), then \SUE may use up to 3 \ARMY counters in Europe with no
condition on the number of provinces and even if the politics of \ROTW
expansion was chosen earlier.

\begin{digressions}[\payspologne and \paysDanemark in the Great Northern War]


  \digression[pVI:GNW:War Denmark]{War in \paysdanemark}
  % (Jym) : was: \DAN in \EC of \RUS + entry in war.  Reformulation to avoid
  % \minmaxing...  We might need diagrams to know which cases should be
  % needed...
  \aparag If \paysdanemark is inactive:
  \bparag If \RUS declares war on \SUE, then \paysdanemark is put in \CE of
  \RUS and fully enters war against \SUE.
  \bparag If \RUS does not declares war on \SUE, but \POL does, then
  \paysdanemark is put in \CE of \POL and fully enters war against \SUE.
  \aparag If \paysdanemark is already at war against \SUE:
  \bparag If its controller is \RUS or \POL and uses its \CB, then it is
  raised in \EG of its controller.
  \bparag If it is not allied to any \MAJ, it is put in \EC of the first \MAJ
  to use its \CB against \SUE (\RUS first, then \POL).
  \bparag If its controller is \RUS or \POL and does not use its \CB, the war
  goes on but \SUE can now obtain the truce (see \ref{pVI:GNW:Danish Truce}).
  \bparag If \paysdanemark is at war against a \MAJ declaring war to \SUE, it
  immediately proposes a white peace. If another \MAJ declares war to \SUE,
  \paysdanemark is then put in \CE of this \MAJ and enters war against \SUE.
  \aparag Otherwise (\paysdanemark at war against someone not part of the
  Great Northern War), \paysdanemark does not partake to the Great Northern
  War.
  % (Jym) Now included in event \leaderMazepa.
  % \aparag If \paysukraine is at war against \RUS or if
  % \ref{pVI:Mazepa} is occurring, \TUR has a free \CB against \RUS at
  % the next diplomacy phase.
  % \bparag This \CB exists at each turn were one of the conditions is true.

  \phpaix
  \aparag\label{pVI:GNW:Danish Truce} If the capital of \paysdanemark is
  controlled by \SUE at the beginning of a peace phase, or if \paysdanemark
  loses a major defeat (on land or on sea) against \SUE (not its allies), it
  proposes a truce to \SUE.
  \bparag If \SUE accepts the truce, \SUE evacuates the capital of
  \paysdanemark but keeps other controlled provinces.
  \bparag If the peace is signed during this truce, provinces of \paysdanemark
  controlled by \SUE must be taken into account when computing peace
  differential.
  \bparag The truce lasts for 3 turns after which \paysdanemark automatically
  enters back in the war.
  % (Jym) added:
  \bparag During the truce, \paysdanemark stays on the diplomatic track of its
  patron and is still considered at war for all purposes.
  % (Jym) hum, not too sure about that. Especially because of the
  % \seazoneOresund levies


  \digression[pVI:GNW:Polish Civil War]{Polish Civil War}
  \begin{histoire}[Tumult in Poland]
    Multiples candidates losing the Polish crown when Augustus II of Saxony
    was elected in 1697 were still trying to influence the Polish
    politics. They all played a complex political game for the crown during
    this war. Even if he was military forced to abdicate at the treaty of
    Altranst\"{a}dt, Augustus was soon back in the war and got his throne
    back. Sweden did not manage to impose a lasting king, even if Stanislas
    Leszinski was elected for a short and contested reign in 1706. Stanislas
    tried to come back at the death of Augustus, this time with the help of
    France, yielding to the War of Polish Succession.
  \end{histoire}
  % (Jym) More references to IV-A for Absolutism that may also come with the
  % the National Revival in pVII (associated to event of Kosciusko).

  \condition{}
  \aparag If Absolutism has been established in \POL, ignore this sub-event.
  % (Jym) WoPS:Polish Victory establishes Absolutism... What?
  \aparag If \payspologne is a special \EW of either \FRA or \SUE per
  \ref{pVI:WoPS:Polish Victory} or a regular \MIN (without Absolutism), see
  the modifications of the Civil War in \ref{pVI:GNW:Minor Poland}.

  \phmil
  \aparag If a Swedish \ARMY first enters a province owned by \POL and no
  battle (except overrun) occurs, the fortress may surrender to \SUE.
  \bparag Roll 1d10, add the current \STAB of \POL (0 if it is a \MIN), add
  the level of the fortress. If the result is 5 or less, the fortress
  immediately surrenders to \SUE.
  % (Jym) added:
  \bparag \SUE has to stop movement in the province in order to try this
  surrender, but it occurs during its movement segment and not during the
  siege segment.
  % (Jym) added:
  \bparag Troops inside the fortress are redeployed as if \emph{Honor of war}
  had been granted. The fortress does not lose one level for being taken.
  % (Jym) I do hesitate about next point. Possibly too tough to manage with a
  % war that lasts 4 turns. But without that, I am afraid that it would be
  % too powerful to reattempt everywhere until this succeeds. In doubt, I
  % comment, since this was not there before.
  %
  % \bparag This can only occurs once per province during the war.
  %
  % (Jym) Saxony is reliable:
  \bparag Provinces of \payssaxe are not subject to automatic surrender to
  \SUE.

  \phpaix
  \aparag If, at the beginning of a peace phase, \SUE controls \villeVarsovie
  or the \STAB of \POL is 0 or lower, \SUE manages to impose its pretender as
  a king for (part of) \POL.
  \bparag If \payspologne is a \MIN, this can only occurs if \SUE controls
  \villeVarsovie.
  \bparag \SUE receives \leaderwithdata{Poniatowski2}. Remove Polish
  \leaderPoniatowski if in play. If he was not in play (even if already dead),
  he will stay with \SUE for 2 turns.
  \bparag Starting with next turn, \SUE can raise up to one \ARMY\faceplus in
  any controlled or owned national province of \POL. This \ARMY has the class,
  technology and cost of Polish troops. It does not decrease the number of
  Polish (or regular Swedish) counters available. It does not count toward
  purchase limits for \SUE nor for \POL.
  % (Jym) to settle the matter of change if the \ARMY is broken
  \bparag \SUE may not have more than 4\LD worth of ``Polish'' troops and may
  not split them. It may, however, use one \LD counter if needed.
  \bparag This is a Swedish \ARMY and can thus trigger \ref{pVI:Mazepa}.
  \aparag If at the beginning of a peace phase, \SUE controls both
  \villeVarsovie and either \villeDresden (if the \payssaxe dynasty rules
  \POL) or \villeCracovie (otherwise), \POL propose a mandatory truce to \SUE.
  \bparag If \SUE accepts the truce, it may immediately annex one province of
  \POL (\SUE chooses which).
  \bparag This truce can only be imposed once during the war.
  \bparag During the truce, \SUE keeps control of the fortresses it controls
  at the beginning of the truce.
  % (Jym) :
  \bparag However, \POL gives back to \SUE the provinces of \SUE it controls
  at the beginning of the truce.
  \bparag As long as the truce lasts, \SUE can freely cross provinces
  controlled by \POL. They count as enemy provinces for movement purpose and
  \SUE cannot stop in them or pillage them. Supply may cross these provinces.
  \bparag During the truce, \POL do not lose \STAB because of the war (as if
  in armistice).
  \bparag The truce can be broken by \POL either after 3 turns of truce or
  during a turn following a major defeat of \SUE
  \aparag\label{pVI:GNW:Stanislas Victory} If \POL signs an unfavourable
  peace after a truce was imposed (even if broken), then \SUE manage to impose
  its pretender on the throne.
  \bparag The new king of \POL is \monarqueStanislas with values 6/5/6. He
  will last as long as a random length for Minister, see \ref{eco:Excellent
    Minister}. This is a new dynasty.
  \bparag \label{pVI:GNW:Stanislas} As long as \monarqueStanislas rules, \POL
  and \SUE are in defensive alliance and \POL must answer any call for ally
  made by \SUE.


  \digression[pVI:GNW:Minor Poland]{Minor Poland}
  % (Jym) Only WoPS may impose durably the \EG of \POLmin. GNW can do that
  % only for the duration of Stanislas reign (by breaking the \EG of WoPS if
  % needed)

  \activation{These effects modify and overrules the effects of
    \ref{pVI:GNW:Polish Civil War} if \payspologne is already a minor
    country.}

  \phdipl
  \aparag If \payspologne is a regular minor country, it makes a mandatory
  white peace with all its enemies (except \SUE and allies) and uses its \CB
  to declare war on \SUE. It will call for allies as per regular rules.
  \aparag If \payspologne is a regular minor country, apply all the effects of
  \ref{pVI:GNW:Polish Civil War} except \ref{pVI:GNW:Stanislas}. Use the
  following instead: For the reign of \monarqueStanislas, \payspologne is put
  in \EG of \SUE and no diplomacy is allowed on it, after which \payspologne
  becomes a normal minor country.
  \bparag For all purposes except incomes (declarations of war, victory
  conditions, \ldots) consider that special \EG as if \payspologne were a
  \VASSAL of \SUE.
  % (Jym) taken from \PPI. Not sure this is useful...
  \bparag As an exception to the normal rules, the order of preference for
  controlling \payspologne during this war is: \PRU, \FRA, \AUS, \HOL, \ENG,
  \RUS.
  \bparag If \payspologne signs no unfavourable peace during this war, it is
  put in \EG of the country that controlled it. Otherwise, it becomes neutral.
  \aparag[\payspologne special minor of \SUE] Due to \ref{pVI:WoPS:Polish
    Victory}, any declaration of war against \SUE also includes
  \payspologne. Apply \ref{pVI:GNW:Polish Civil War} substituting \RUS for
  \SUE (including the benefits of \leaderPoniatowski and his \ARMY). \RUS can
  impose its pretenders on the throne.
  \bparag If \RUS imposes its pretender on the Polish throne, \payspologne it
  put in \EG of \RUS, with no diplomacy possible, for the reign of
  \monarqueStanislas after which \payspologne becomes a normal minor country.
  \aparag[\payspologne special minor of \FRA] Due to \ref{pVI:WoPS:Polish
    Victory}, \FRA decides whether \payspologne uses its \CB against \SUE or
  not.
  \bparag If \payspologne is at war, it is played by \FRA
  % (Jym) added:
  but \FRA does not have to enter war against \SUE (it \emph{may} choose to do
  so, using the normal \CB of \POL).
  \bparag If \SUE manages to impose its pretender, this breaks the special
  status of \payspologne. It becomes a special \EG of \SUE (as above) for the
  reign of \monarqueStanislas and after that a regular minor country.
  \bparag If \SUE does not manage to impose its pretender, \payspologne stays
  a special \EG of \FRA.

\end{digressions}



\event{pVI:WoSS}{VI-2}{The War of Spanish Succession}{1}{PBMod}

\begin{todo}
  Add possibility to gives ``compensations'' to some minors to ``buy'' them in
  the war and make them change side. Historically: Sicily for \paysSavoie and
  bid on the imperial throne for \paysBaviere.
\end{todo}

\condition{This event is the same as \ref{pV:WoSS} which happens now if it did
  not occur yet. Else, treat as \RD and mark off.}



\event{pVI:Kingdom Prussia}{VI-3}{Creation of the Kingdom of
  Prussia}{1}{RistoMod}

\condition{This event is the same as \ref{pV:Kingdom Prussia} which happens
  now if it did not occur yet. Else, treat as \RD and mark off.}



\event{pVI:Jacobite Rebellion}{VI-4}{Jacobite Rebellion}{2}{RistoMod}

\history{1715/1745-46}

\condition{}
% (JCD) Added the mark off part. Obviously, no Jacobite rebellion for Catholic
% \ANG.
\aparag If \ANG is \CATHCR or \CATHCO, roll for two \REVOLT in \ANG, then mark
off and consider as played.
\aparag This event can only happen if \paysecosse is on the diplomatic track
of \ANG or if \ANG owns at least four provinces of \paysecosse. Otherwise, do
not mark off and re-roll.
\aparag There are two rebellions with slightly different initial
conditions. Apply the rules hereafter, but read initial placement in
\xnameref{pVI:Jacobite:First Revolt} or \xnameref{pVI:Jacobite:Bonny Prince
  Charlie}.

\phdipl
\aparag The rebellion is controlled by \FRA if Catholic, otherwise by \HIS.
\aparag If \FRA is \CATHCR, it has a \CB to make a full intervention at the
side of \paysecosse.
\bparag If \FRA is Catholic, it can make a limited intervention at the side of
\paysecosse.
\bparag If \FRA is Protestant, it can make a limited intervention at the side
of \ANG.
\aparag \HOL can make a limited intervention at the side of \ANG.
\aparag \HIS can make a limited intervention at the side of \paysecosse.
\begin{todo}
  Intervention only if Alberoni is or was minister. Need to write Alberoni
  before enforcing this condition.
\end{todo}
\aparag Other countries can make foreign intervention as per normal religious
wars rules (see \ref{chDiplo:Religious Civil War}). \paysecosse is considered
to be Catholic during this war.

\phadm
\aparag Rebels roll for reinforcements in offensive attitude for the duration
of the war.
\bparag Rebels can use the counters of both \paysecosse and \paysroyalistes.
\bparag reinforcements must be put in provinces where there are already rebels
or allied troops (not just \REVOLT ). If none exist, the rebels receive no
reinforcements.

\phmil
\aparag The \REVOLT are supply sources for the rebels and limited supply
sources for their allies.

\phpaix
\aparag \ANG wins if there are no more \REVOLT and either there is no more
rebel \ARMY or the rebels and their allies have suffered one more major defeat
that \ANG this turn.
\bparag In this case, remove all rebel counters from the map.
\bparag \paysecosse get back to the diplomatic position it had before the war
on the English track.
\bparag If \FRA was fully at war, a normal peace has still to be signed.
\aparag The rebels win if the king is overthrown by the \REVOLT or if they
control \villeLondon and there is at least one \REVOLT still in play or if a
fully allied \FRA manages to impose an unconditional surrender to \ANG.
\bparag If the rebels win and were not allied to any \CATHCR country, \ANG
becomes \CATHCO.
\bparag If the rebels win and were allied to a \CATHCR country, \ANG becomes
\CATHCR.
\bparag At the beginning of the next turn, the king of \ANG dies and an
automatic \terme{Dynastic Crisis} occurs in \ANG. This overrules \ref{pVI:Act
  Establishment}.
\aparag Apply the following additional effects if \FRA was fully at war and
manages to impose an unconditional surrender to \ANG:
\bparag \ANG loses 50\VP.
\bparag Events \shortref{pIV:Act Navigation} and \shortref{pVI:Act Union} are
cancelled.
\bparag \ANG makes an enforced dynastic alliance with \FRA and must give a
\COL or \TP of its choice as a dowry.
\bparag \ANG makes an enforced offensive alliance with \FRA for two turns and
must respect it when \FRA calls it as ally.
\bparag \ANG cannot declare war to \FRA for the duration of the new king and
his successor.


\subevent[pVI:Jacobite:First Revolt]{First Jacobite Rebellion}
\history{1715}

\phevnt
\aparag If \paysecosse was allied to \ANG, remove all its troops from the map.
\bparag \paysecosse is not considered to be \VASSAL or \ANNEXION by \ANG as
long as the war lasts (for incomes or victory condition purpose).
\aparag Place a \REVOLT \facemoins in each of the following provinces:
\provinceHighlands, \provinceMoray and \provinceAlba.
%\bparag Place a \ARMY\facemoins of \paysecosse in one of the revolted
%provinces.


\subevent[pVI:Jacobite:Bonny Prince Charlie]{Bonny Prince Charlie}
\history{1745-1746}

\phevnt
\aparag If \paysecosse was allied to \ANG, remove all its troops from the map.
\bparag \paysecosse is not considered to be \VASSAL or \ANNEXION by \ANG as
long as the war lasts (for incomes or victory condition purpose).
\aparag Place a \REVOLT \facemoins in each of the following provinces:
\provinceHighlands, \provinceMoray and \provinceAlba.
\bparag Place a \ARMY\faceplus of \paysecosse and general \leader{Prince
  Charles} in one of the revolted provinces.



\event{pVI:Act Establishment}{VI-5}{Act of Establishment}{1}{Risto}

\history{1701}

\effetlong
\aparag From now on \ENG can no longer suffer dynastic crisis due to a roll on
the Monarch Reign table.
\aparag However, it can still suffer dynastic crisis due to events.



\event{pVI:Vassalisation Hanover}{VI-6}{Vassalisation of
  \payshanovre}{1}{Risto}

\history{1714}

\condition{}
\aparag Cannot occur if \ENG is not Protestant. In that case mark as played.
\aparag Cannot occur if \ref{pVI:Act Union} and \ref{pVI:Act Establishment}
have not already occurred both. In that case re-roll and do not mark off.

\phevnt
\aparag If \payshanovre is currently in a war against \ENG, it offers
immediately a white peace.
\aparag \payshanovre becomes a permanent \VASSAL of \ENG for the rest of the
game. No diplomacy is allowed on \payshanovre.

\effetlong
\ephase Revolts in \pays{hanovre} are no more automatically suppressed if
inactive.  \fphase \ANG may now use the troops of \pays{hanovre} to fight
revolts inside its territory and use its troops to fight revolts inside
\pays{hanovre}.



\event{pVI:Treaty Methuen}{VI-7}{Treaty of Methuen}{1}{RistoMod}

\history{1703}

\condition{}
% (Jym) let's be explicit about the rare cases that would lead to a double
% application of this event
\aparag This event can normally only happen once, either triggered by
\ref{pV:WoSS} (at the beginning of the war or at peace time) or by rolling for
it in the table.
\bparag If the event has already been rolled for when \ref{pV:WoSS} occurs,
then \emph{Dynastic link and alliance with Portugal} is not at stake in the
war, except if \HIS managed to re-annex \paysportugal after the event.
\bparag If \emph{Dynastic link and alliance with Portugal} was chosen by a
\MAJ during \ref{pV:WoSS}, then consider the event as already played, mark off
and play \RD instead as per normal rules.
\aparag If this event was triggered by \ref{pV:WoSS}, apply
\xnameref{pVI:Methuen:WoSS}, else apply \xnameref{pVI:Methuen:Normal}


\subevent[pVI:Methuen:Normal]{Treaty of Methuen}
\history{1703}

\condition{}
\aparag If \paysportugal is annexed by \HIS as per \ref{pIII:POR Ann.:Portugal
  Annexed}, play \ref{pIV:Portuguese Revolt} in addition to this event (even
if \numberref{pIV:Portuguese Revolt} already occurred and was won by \HIS).
% (Jym) Choice of \PORmin if at war breaking its alliance vs event waiting the
% peace, I chose the former because it's funnier.
\aparag If \ENG is at war against \paysportugal allied to a \MAJ, \paysportugal
breaks its alliance, sign a white peace with \ENG, becomes neutral and the
event occurs.
\bparag Allies of \ENG have the choice to either sign a white peace with
\paysportugal or break their alliance with \ENG and stay at war with
\paysportugal.
% (Jym) event several times in the table, so giving a penalty.
\bparag If \ENG is at war against \paysportugal (not allied to a \MAJ), then
the event cannot occur. Mark-off and play \RD instead.

\phdipl
\aparag \ENG receives a bonus of \bonus{+5} for its diplomacy on \paysportugal
for this turn only.

\effetlong
\aparag From now on \paysportugal always gives rights to trade to \ENG as per
\ref{chAdministration:Limited Access:Giving Rights}, even if it is not on the
English diplomatic track.


\subevent[pVI:Methuen:WoSS]{Dynastic link and alliance with Portugal}
\history{not historic}

\condition{}
\aparag If this event is triggered by \HIS, \paysportugal is annexed by
\HIS. Apply all the effects of \ref{pIII:POR Ann.:Portugal Annexed}.
\bparag Otherwise, apply this event.

\phevnt
\aparag If \paysportugal was annexed by \HIS as per \ref{pIII:POR
  Ann.:Portugal Annexed}, it breaks its annexation and becomes a regular minor
country.
\aparag \paysportugal signs a white peace with the \MAJ triggering the event.
\bparag Allies of the \MAJ triggering the event have the choice to either sign
a white peace with \paysportugal or break their alliance with the \MAJ and
stay at war with \paysportugal.
\aparag If it was not on the diplomatic track of the \MAJ triggering the
event, \paysportugal becomes neutral.

\phdipl
\aparag The \MAJ triggering the event receives a bonus of \bonus{+5} for its
diplomacy on \paysportugal for this turn only.

\effetlong
\aparag From now on \paysportugal always gives rights to trade to the \MAJ
triggering the event as per \ref{chAdministration:Limited Access:Giving
  Rights}, even if it is not on its diplomatic track.



\event{pVI:Act Union}{VI-8}{Act of Union}{1}{RistoMod}

\history{1704}

\condition{}
\aparag Cannot occur if \ENG has been defeated in a Jacobite rebellion
(\ref{pV:Glorious Revolution} or \ref{pVI:Jacobite Rebellion}). In that case
mark off as played.
% (Jym) and play \RD with the R in \ENG instead ?
\aparag Cannot occur if a Jacobite rebellion is still active.
\bparag In that case, mark off but re-roll another event.
\bparag During the first event phase after the end of the rebellion,
\ref{pVI:Act Union} will automatically be the first event rolled this turn.
\aparag Cannot occur if \paysecosse is not \VASSAL of \ENG.
\bparag In that case, mark off but re-roll another event.
\bparag During the first event phase where \paysecosse is \VASSAL of \ENG,
\ref{pVI:Act Union} will automatically be the first event rolled this turn.
% (Jym) Above, my interpretation of conditions of Risto + remark of Pierre
% below: (Risto) 1. Can occur only if Scotland is currently in \VASSAL of \ENG
% (whether by event 6:IV or not). Otherwise mark off as played.  3. Cannot
% occur if Jacobite rebellion (12:V or 10:VI) is still active.  In that case
% re-roll, but do not mark off as played.  (Pierre) Changes: if conditions 1
% or 3 : triggered as soon as Scotland is \VASSAL

\phevnt
\aparag \paysecosse is annexed by \ENG.
\bparag All \TradeFLEET levels of \paysecosse are immediately added to the
\TradeFLEET levels of \ENG in the same zone. This may cause automatic
concurrence to be solved immediately. If after that \ENG has more than 6
levels of \TradeFLEET in any zone, reduce to 6 levels.
% \bparag Units of \paysecosse currently in play are disbanded.  (Jym) Or not?
% Giving forces for free during a war that might have a big reinforcement roll
% may look very juicy but opposite to that, removing troops helping \ANG to
% sustain a siege looks absurd. Deleting.

\effetlong
\aparag All provinces belonging to \paysecosse in 1492 are now considered as
national provinces of \ENG.
\aparag From now on, \ENG can raise, upkeep and use military counters of
\paysecosse (not \TradeFLEET) as if it were its own counters.



\event{pVI:Bill Test}{VI-9}{Bill of Test}{1}{Risto}
\begin{todo}
  Change!
\end{todo}
\history{1673}

\effetlong
\aparag From now on \ENG can no longer be forced to change religion by foreign
conquest.
\bparag However, it can still be forced to change religion as a result of
\ref{pV:Glorious Revolution} or \ref{pVI:Jacobite Rebellion}.



\event{pVI:Heinsius}{VI-10}{Heinsius}{1}{Risto}

\history{1689-1720}
\dure{as long as \strongministre{Heinsius} remains the excellent minister}

\condition{\HOL can refuse this event if it so wishes. In that case mark off
  as played.}
\aparag \HOL can freely dismiss \ministreHeinsius at the end of any following
monarch survival phase and the event terminates.

\phevnt
\aparag \HOL receives an excellent minister \ministreHeinsius, with values
9/8/7.  He will last for a random length for Minister, see \ref{eco:Excellent
  Minister}.

\phdipl
\aparag \HOL can once ignore a call for help by an ally without the loss of
stability for such a treachery.



\event{pVI:WoPS}{VI-11}{War of Polish Succession}{1}{PB}

\history{1733-1735}
\dure{Until the end of the war caused by the event.}

\condition{}
\aparag The event is pending. It will be activated as soon as the year is 1700
or more and the king of \POL dies.
\bparag If the event is pending while \POL becomes a minor, continue to roll
for survival of the king every turn until his death (either scheduled or
premature) activate the event.

\phevnt
\aparag If this was not already the case, \POL becomes the \POLmin. \PRUMin
becomes the major \PRU. See \ref{chSpecific:Campaign:Becoming Prussia} for
details.
\aparag The crown of \payspologne is proposed to the step-father of a foreign
king and \payspologne looks for the protection of this foreign king.
\bparag If \ref{pVI:Great Northern War} happened and \SUE managed to impose
its candidate on the throne of \payspologne, then the potential protectors
are, in order, \SUE then \FRA.
\bparag In all other cases (\numberref{pVI:Great Northern War} did not happen
or wasn't won by \SUE), the potential protectors are, in order, \FRA then
\SUE.
\bparag The first potential protector must immediately accept or refuse the
crown. If it refuses, then the second one must either accept or refuse.
\aparag \payspologne immediately signs a white peace with its protector and is
put in \EG of its protector.
\bparag If both protectors refuse, \payspologne will fight alone in the
upcoming war. Apply only the first point of the diplomatic phase (\CB for
\RUS) as well as the effects of the peace phase on the future of \payspologne
% (Jym) Added the peace conditions in the case where there are no
% protectors. Most likely peace -3 for \POLmin and in this case beginning of
% the open bar.

\phdipl
\aparag \RUS has a free \CB against \payspologne this turn.
\aparag \AUS has a free \CB against \payspologne this turn.
\aparag \SUE (if not protector) and \PRU both have a (normal) \CB against
\payspologne this turn.
\aparag If \payssaxe was ruling \payspologne due to \ref{pV:Saxon King Poland}
and a war against \payspologne is declared due to this event, \payssaxe also
declares war on \payspologne and is put in \EG of the first country at war
against \payspologne in the following list: \RUS, \AUS, \SUE, \PRU.
% (Jym) Added \PRU at the end of the list since it is technically possible
% that it may be the sole country at war...
\aparag All countries entering war against \payspologne due to this event are
considered allied for the duration of the war without need to sign a formal
alliance.
\aparag If nobody declares war on \payspologne, it becomes a permanent \EG of
its protector as if there has been a Polish victory. Apply all the effects of
\xnameref{pVI:WoPS:Polish Victory}.

\phpaix
\aparag An extra malus of \bonus{-4} is applied for all separate peace against
\payspologne or \payssaxe (if it entered war due to being allied with
\payspologne by \ref{pV:Saxon King Poland}) for this war.
% (Jym) Formulation a bit complex for Saxony because the malus has no reason
% to be if the Polish Dynasty is no more in place or never was but Saxony was
% just a normal ally of say \AUS.
\begin{digressions}[Conditions of Victory]


  \digression[pVI:WoPS:Polish Victory]{Polish Victory}
  \aparag If \payspologne (and its side) signs a favourable peace of level 3
  or more, \payspologne becomes a permanent \EG of its protector.
  \bparag For all purposes except incomes (declarations of war, victory
  conditions, \ldots) consider that \payspologne is a \VASSAL of its
  protector.
  \bparag No diplomacy is allowed on \payspologne anymore.
  \bparag The protector immediately wins 50 \VP.
  % (Jym) Added: (mostly if \POL wins without protector)
  \aparag Absolutism is established in \payspologne.
  % (Jym) See below the trip about Lorraine annexation
  \aparag At the peace, the protector can annex any province, even the
  capital, of one minor country.
  \bparag This province must be adjacent to the territory of the protector.
  \bparag This can destroy the country.
  \bparag The minor must be either on the diplomatic track of the protector or
  on the diplomatic track of one of its enemies (even if not at war).
  \bparag This count as one peace condition if the province is occupied by the
  protector (or its allies) or as all peace conditions (for the protector and
  its allies) otherwise (minor not at war, or even allied with the protector).
  \bparag If \SUE is the protector, it can annexe this way the whole
  \regionNorvege whatever the current diplomatic status of \paysdanemark (or
  \paysVnorvege). This always count as all the peace conditions for the
  alliance of \SUE.


  \digression[pVI:WoPS:Polish Defeat]{Polish Defeat}
  \aparag If \payspologne (and its side) signs an unfavourable peace of level
  3 or more, the protector loses 15\VP (even if it was not at war).
  \bparag \payspologne becomes neutral. From now on, it will never be able to
  go above \SUB on the diplomatic track.
  \bparag Absolutism is abolished in \payspologne.
  \aparag From now on, \RUS, \AUS, \PRU and all countries of the \HRE can
  freely cross provinces of \payspologne. The provinces are considered enemy
  and don't give supply, it is not allowed to stop in \payspologne or pillage
  its provinces because of attrition.
  \aparag If they were still at war against \payspologne when the peace is
  signed, both \RUS and \AUS win 50\VP.


  \digression[pVI:WoPS:Status Quo]{Status Quo}\label{pVI:WoPS:Status quo}
  \aparag If neither side gets a full victory as per the previous cases, apply
  these effects.
  \aparag \payspologne is put in \EG of its protector. It is a normal minor.
  % (Jym) Technically, the protector may have let \POL down, so \POL may no
  % more be in \EG. But the loss may be not that high.
  \aparag The protector loses 15\VP (even if not at war).
  \aparag If they were still at war against \payspologne when the peace is
  signed, both \RUS and \AUS win 30\VP.
  \aparag Absolutism is abolished in \payspologne.
  \aparag From now on, \RUS, \AUS, \PRU and all countries or the \HRE can
  freely cross provinces of \payspologne. The provinces are considered enemy
  and don't give supply, it is not allowed to stop in \payspologne or pillage
  its provinces because of attrition.
  % (Jym) What ? Give a \CB to a \MIN ?
  \bparag Crossing polish provinces gives a \CB to \payspologne for the next
  diplomacy phase.
  % (Jym) See below the trip about Lorraine annexation
  \aparag At the peace, the protector can annex the last province of one minor
  country who only has one province left, even if this is a capital.
  \bparag This province must be adjacent to the territory of the protector.
  \bparag This destroys the country.
  \bparag The minor must be either on the diplomatic track of the protector or
  on the diplomatic track of one of its enemies (even if not at war).
  \bparag This does not count as a peace condition and is done in addition to
  the normal peace.
  \bparag If the protector chooses to annexe a province of a minor country not
  on its track (but on the track of one of its enemies), it must gives to its
  diplomatic patron the diplomatic control of a minor from its own track which
  is at least at the same level of diplomatic control. The enemy of the
  protector chose which diplomatic compensation he takes.
  \bparag If \SUE is the protector, it can annex this way the whole
  \regionNorvege as if it was only one province. \paysdanemark (or
  \paysVnorvege) must be on its track, or on the track of one enemy (in which
  case diplomatic compensation apply as above).
  % (Jym) Trip Lorraine (hist.) \FRA supports Stanislas and obtains a losing
  % Status Quo. Some escape for Stanislas must be found for the international
  % standing of Louis XV. Francois I of Lorraine is the Emperor (and husband
  % to Maria-Theresa). He exchanges Lorraine against \paysParme and Lorraine
  % goes to Stanislas. (\PPI) Lorraine could be annexed during the war, but
  % it usually did not work, either France won or \SUE looked stupid. (game
  % problem) 1. Give \SUE the occasion of a good deal. 2. Allow a bit more
  % choice to the protector especially if Lorraine was already annexed.
  % 3. Avoid the search for Status Quo for the protector. Answers: 1. Norway!
  % 3. Thing a bit strange with an annexation in case of win. More interesting
  % because no compensation to give. Choice about the minor. Case study: SUE
  % may annex Norway, Kurland. Or some other things in the \HRE, such as bits
  % of \paysHanse, Munster, Oldenburg, Bremen. The bad thing would be to annex
  % bits of Hanover, but let's see \SUE try to anger \ENG.
  % For \FRA, Lorraine, Alsace, Liege, Savoy, Switzerland, Palatinate,
  % Cologne, Trier, Baden, \paysMonferrato and \paysgenes.
  % It looks fine, seen like that.

\end{digressions}



\event{pVI:War Turkey}{VI-12}{War against Turkey}{2}{RistoMod}

% (Jym) I wonder about the 2 dates and occurrences. But in EU9, it is a
% problem because \AUS is major all the time.
\history{1716-18/1737-39}

\condition{The first eligible in the following list occurs, each case can only
  happen once per game}
\aparag \AUS receives a free \CB against \TUR for this turn. It can choose to
decline this offer, in which case proceed with the list.
% (Jym) Not possible in EU9. Removing: b. Inactive \HAB declares war against
% \TUR.  (JCD) We should keep it. In EU9, c. will be used; in EU8, possibly
% too.
\aparag If inactive, \AUSmin declares war against \TUR. It calls for allies as
usual.
\aparag If inactive, \paysVenise declares war against \TUR. It calls for
allies as usual and will have \bonus{+2} to all the reinforcements check made
during this war.
\aparag If none of the conditions apply, nothing happens.



\event{pVI:WoAS}{VI-13}{War of Austrian Succession}{1}{PB}

\history{1740-1748}
\dure{Until the end of the war}

\condition{}
\aparag Cannot happen if there is a \GE. In this case, mark off and play \RD
instead.
\aparag Cannot happen before period VI (thus, \AUSmin has become \AUS
anyway). In this case, do not mark off and re-roll.
\aparag Cannot happen before the start of the war caused by \ref{pV:WoSS}. In
this case, do not mark off and re-roll.

\phevnt
\aparag[The Pragmatic Sanction]
\bparag The king of \AUS dies. The new queen is \monarque{Maria Theresia}
(values 8/8/7, lasts 8 turns, does not roll for survival during 5 turns, adds
\ARMY\faceplus as basic forces).
% (Jym) I am not sure about how the dissociation is managed when \AUS becomes
% a \MAJ in pIV. By default:
\bparag Mandatory dynastic dies between \HIS and \AUS are voided (if still
existent).
\bparag If \paysbaviere won the electorate during \ref{pIV:TYW}, it opposes
the Sanction and pretends to the throne of \AUS. Otherwise, \payspalatinat
does.
\aparag \AUS loses control of the pretending country.
\aparag The pretending country proposes a white peace to its current enemies
and then declares war to \AUS.
\aparag If this is not already the case, \POL becomes the \POLmin. \PRUMin
becomes the major country \PRU.
\bparag See \ruleref{chSpecific:Campaign:Becoming Prussia} for details on how
to handle this.
% (Jym) What? this was attached to the point above (PPI)

% \PRU offers an immediate white peace to its enemies. \PRU then has a free
% \CB against \POL.

% (Jym) I imagine the white peace would rather be proposed by \POL rather than
% \PRU. And why this \CB?

\phdipl
\aparag \PRU has a free \CB against \AUS at this turn (only).
\bparag If it uses it, \PRU and the pretending country are allied for the
duration of the war.
\aparag \FRA has a \CB against \AUS during every turn of the war caused by the
event.
\bparag If it uses it, place the pretending country in \EG of \FRA.
\bparag If \PRU and \FRA use it, they are allied for the war without need for
signing a formal alliance.
\bparag If \FRA does not use this \CB at the first turn of war, the pretending
country will call for allies as per normal rules.
\aparag \ANG has a free \CB against \FRA as a reaction of the previous \CB
(only).
\bparag This \CB can only be used in reaction to \FRA declaring war to \AUS.
\bparag If it uses it, \ANG and \AUS are allied for the war, without need for
signing a formal alliance.

\phadm
\aparag At the first turn of the war (only), \PRU rolls for reinforcements as
a minor country (in offensive attitude).
% (Jym)
\bparag These reinforcements are \Veteran. They do not count toward this turn
purchase limit.
\aparag At the first turn of the war (only), \AUS rolls for reinforcements as
a minor country (in defensive attitude).
% (Jym)
\bparag These reinforcements are \Conscripts. They do not count toward this
turn purchase limit.

\phpaix
\aparag If \AUS signs an unconditional surrender, it loses the imperial
throne. The pretending country becomes Emperor for the rest of the game.
\bparag In that case, \PRU automatically gets the royal dignity as per
\ref{pV:Kingdom Prussia}. If that event didn't happen yet, consider it to be
the first event rolled next turn with any mention to \paysbrandebourg
referring to \PRU instead (in that case, \AUS \textbf{must} give the royal
crown to \PRU in the following diplomacy phase). (JCD) TODO there is probably
a problem with that, since \AUS will no more be Emperor...
\aparag Extra \VP are granted for the control of certain provinces at the end
of the war.
\bparag \PRU gains 25\VP per province annexed from \AUS. It loses 20\VP if it
annexes none.
\bparag \AUS gains 20\VP per province annexed from \PRU. It loses 25\VP if it
annexes none.
\bparag The player controlling the pretending country gains 30\VP per province
annexed from \AUS and loses 15\VP if the pretending country annexes no
province. These \VP are also lost (or won) by \AUS.



\event{pVI:Kurland}{VI-14}{War of Succession in Kurland}{1}{PBnew}

\history{1730-1731}
\dure{As long as \payscourlande exists.}

\phevnt
\aparag The provinces \provinceCourlande and \provinceLivonija declare
independence from their current owner and form the minor country
\payscourlande.
\aparag \leader{von Sachsen}, or, if he's not alive, a random mercenary
general lasting 4 turns, takes command in the new duchy and look for a
protector.
\bparag The following countries must immediately accept or refuse to become
protector of the duchy (in order): \FRA, \AUS, \PRU, \HOL.
% (Jym) Suppression of references to the country that Sachsen serves
\bparag If all of them refuse, then the general wisely chooses to stand
back. \payscourlande doesn't get a general and won't get reinforcements in any
war.
\bparag If there is a protector, then \payscourlande becomes a permanent
\VASSAL of its protector and no diplomacy is allowed on it.

\phdipl
\aparag Any country owning one province or more of the minor when the event
happens gets a free \CB against \payscourlande.
% (Jym):
\bparag A minor country uses this \CB only if there is already a major country
using this \CB (for the other province).

\phadm
\aparag The general of \payscourlande can lead troops of its protector.

\phpaix
\aparag \payscourlande has no capital and can thus be annexed by anybody.

\effetlong
\aparag The protector loses 30 \VP at the end of the game if \payscourlande
does not exist.



\event{pVI:Slave Revolts WI}{VI-15}{Slave Revolts in the West
  Indies}{*}{Risto}

\history{No precise date}

\phevnt
\aparag Roll 1d10 for each power having \COL in areas \granderegionCuba,
\granderegionHaiti and/or \granderegionAntilles. On a result of 7 or more, a
\REVOLT\faceplus is placed in one randomly chosen \COL of the power.



\event{pVI:Bantu Raids}{VI-16}{Bantu Raids}{*}{Risto}

\history{No precise date}

\begin{todo}
  May represent the early Xhosa wars starting in 1779 but should then be
  pushed in VII. Otherwise, could be removed.
\end{todo}

\phevnt
\aparag Natives of area \granderegionNatal and the two coastal provinces
bordering it are activated for this turn and shall attack all \COL/\TP in
these provinces.

\phadm
\aparag The strength of the natives activated by this event is always 6\LD
(whatever the printed value) and they automatically receive a native leader.



\event{pVI:Last Great Mughals}{VI-17}{The Last of the Great Mughals}{1}{PBnew}

\history{1707 (Death of Aurangzeb)}

\phevnt
\aparag The general \leader{Great Mughal} is removed from the game.
\aparag \xnameref{pII:Mughal Expansions} cannot happen anymore.
\aparag The basic forces of \paysmogol becomes \ARMY\faceplus.
\aparag Reaction of country \paysmogol becomes 3.
\aparag \paysmogol loses 1d10/3 (round to closest) areas (the ones with the
largest numbers).

\effetlong
\aparag \paysmysore and \payshyderabad are created as soon as their respective
province does not belongs to \paysmogol anymore.
\bparag This can happen either at the start of this event, due to the
provinces lost by this event or at some other point in the game if \paysmogol
loses provinces.
\bparag Both countries are not necessarily created at the same time.
\aparag Colonial powers may now raise Indian troops (``Sepoy'') as per their
respective specific rules.



\event{pVI:Wars India}{VI-18}{Wars in India}{3}{PBnew}

\condition{}
\aparag Roll 1d10 and apply the correct subevents.
\bparag 1-4 = A) War between \paysmogol and \paysperse. Apply
\xnameref{pVI:India:Mughal Persian War}.
\bparag 5-8 = B) War between \paysafghans and \paysperse. Apply both
\xnameref{pVI:India:Afghan Empire} and \xnameref{pVI:India:Fall Persian
  Safavids}.
\bparag 9-10 = C) War between \paysafghans and \paysmogol. Apply both
\xnameref{pVI:India:Afghan Empire} and \xnameref{pVI:India:Rise Marathi}. This
case may not happen before either~\ref{pVI:Last Great Mughals}, re-roll
another case if needed.

\aparag Each of the three previous cases can only happen once. If it already
happen, re-roll another case.

\aparag Each of the following sub-event can only happen
once. \xnameref{pVI:India:Afghan Empire} may occur due to two different cases
(B and C). The second time, ignore it and only plays the other sub-event.

\aparag In each of the three case, natives in one random province in
\continentIndia are activated.

\subevent[pVI:India:Mughal Persian War]{\paysmogol-\paysperse War}
\history{1739}

\phevnt
\aparag \paysmogol loses all provinces except the areas \granderegionDelhi,
\granderegionAoudh, \granderegionBengale, \granderegionGondwana and
\granderegionOrissa.
\aparag Lower the difficulty and tolerance (for \COL and \TP implantation) by
2 in every province controlled by \paysmogol
\aparag \paysperse gets the general \leaderwithdata{Nadir Shah} for 5 turns.

\phdipl
\aparag Test fidelity of \paysPerse and \paysOrmus.


\subevent[pVI:India:Afghan Empire]{Afghan Empire}
\history{1747}

\phevnt
\aparag The minor country \paysafghans is created and owns area
\granderegionAfghanistan except \provinceHerat if owned by \paysPerse.


\subevent[pVI:India:Fall Persian Safavids]{Fall of the Persian Safavids}
\history{1749}

\phevnt
\aparag The lasting effect of \ref{pIII:Persian Safavids} are cancelled.
\aparag \provinceHerat is annexed by \paysafghans

\phdipl
\aparag Test fidelity of \paysPerse and \paysOrmus.

\subevent[pVI:India:Rise Marathi]{Rise of the Marathi}
\history{1746-1761}

\phevnt
\aparag \paysmogol only loses all provinces except the areas
\granderegionDelhi, \granderegionAoudh, \granderegionBengale and
\granderegionGondwana.
\aparag Lower the difficulty and tolerance (for \COL and \TP implantation) by
2 in every province in \continentIndia.
\bparag This is not cumulative with the decrease caused inside \paysmogol by
\xnameref{pVI:India:Mughal Persian War}.



\event{pVI:Mazepa}{VI-19}{Revolt of Mazepa}{1}{PBnew}

\history{1708-1709}

\condition{}
\aparag \paysukraine is looking for a new protector.
\bparag If this event is triggered during \ref{pVI:Great Northern War}, either
by troops of \SUE entering \regionUkraine or by rolling for it on the table,
then the new protector is \SUE.
% (Jym) added:
\bparag If the current protector of \paysukraine is at war against another
\MAJ, then the new protector is chosen among the countries at war against the
current protector in the following list: \RUS, \POL, \TUR, \AUS, \SUE, \PRU.
\bparag If the current protector of \paysukraine is not at war against any
other \MAJ, then the new protector is chosen in the following list: \POL (if
Orthodox), \RUS, \TUR, \POL, \AUS, \SUE, \PRU.
% (Jym) First neighbours as protector then further away

\phevnt
\aparag The potentials protectors are asked in order if they accept or not to
protect \paysukraine.
\bparag If all refuse, \paysukraine will not have a protector for the duration
of the war.
\aparag \paysukraine declares war on its former protector and the new
protector must immediately join this war with no cost in \STAB.
\aparag Counters of \paysukraine are immediately removed from play.
\aparag Place a \REVOLT \faceplus in a province of \regionUkraine.
\bparag If the event is triggered by Swedish presence, then the \REVOLT is put
in the province where the Swedish \ARMY is. Otherwise, a random province is
chosen in \paysukraine.
% (Jym) general Mazepa with extravagant values !
\aparag Place general \leaderwithdata{Mazepa} with the \REVOLT, scheduled to
last 4 turns.
\aparag Place a \LD of \paysukraine in the revolted province.
\bparag If the new protector either has a common border with \paysukraine or a
``king ranked'' general in a province adjacent to \paysukraine, place an
\ARMY\facemoins instead.
\bparag ``King ranked'' generals are those bearing the king symbol, namely
monarchs, \leader{Carl XII} as an heir to the throne
(\ref{chSpecific:Sweden:Charles XII}) or any Turkish Vizier
(\ref{chSpecific:Turkey:Vizier}).
\aparag The revolt is considered active as long as \leaderMazepa is alive and
at least one \REVOLT exists in one of the provinces of \regionUkraine.

\phdipl
\aparag Any country possessing a province of \regionUkraine with a \REVOLT in
it has a free \CB against either the former or the current protector (its
choice).
\bparag Minor countries use this \CB against the new protector.
\aparag As long as the revolt is active, \TUR as a free \CB against either the
former or the new protector (its choice).

\phadm
\aparag If the revolt is active \paysukraine roll for reinforcements in
offensive attitude, base on the income of the provinces with a \REVOLT in
them.
\bparag The reinforcement roll has a malus of \bonus{-2} unless a ``king
ranked'' leader of the new protector is in or adjacent to \paysukraine.

\phmil
% (Jym) Add a limited supply for \paysukraine else it may botch if it falls
% out of war and Mazepa has zero supply
\aparag \REVOLT are limited supply sources for the troops of \paysukraine but
are not supply source for the protector.
% (Jym) added, so that \REVOLT may spawn as long as the \ARMY is there
\aparag If a stack containing troops of \paysukraine takes a fortress, place a
\REVOLT \facemoins in the province.

\phpaix
\aparag The \REVOLT can extend in any province of \regionUkraine.
\bparag \REVOLT in \paysUkraine cause loss of \STAB to the \textbf{former}
protector. Other \REVOLT in \regionUkraine cause loss of \STAB to the owner of
the province as per normal rules.
\aparag If the new protector signs a white or favourable peace while the
revolt is still active, all the provinces of \regionUkraine belonging to
countries that were at war against the new protector during this war are
annexed by the \MIN \paysukraine. The new protector gain all the benefits of
\ref{pIV:Revolt Cossacks}.
\aparag Otherwise, the former protector stays protector of \paysukraine (with
the provinces still belonging to the minor after the peace is signed).

% (Jym), Placeholders

\event{pVI:WoJE}{VI-s}{War of Jenkins' ear}{1}{PBNotEvenWritten}
\history{1739-1748}
\begin{todo}
  \ANG vs \HIS in America. Later part of~\ref{pVI:WoAS}.
\end{todo}


\event{pVI:Comuneros}{VI-t}{Revolt of the Comuneros}{1}{PBNotEvenWritten}
\history{1721-1735}
\begin{todo}
  Revolt in Paraguay. Maybe doable via revolt tables only.
\end{todo}

\event{pVI:WoQA}{VI-u}{War of the Quadruple Alliance}{1}{PBNotEvenWritten}
\history{1718-1720}
\begin{todo}
  \SPA vs \paysNaples.
\end{todo}

\event{pVI:Alberoni}{VI-v}{Alberoni}{1}{PBNotEvenWritten}
\history{1711-1719}
\begin{todo}
  Excellent (?) minister for \HIS. Should be VI-2(2). Could be related
  to~\ref{pVI:WoQA}.
\end{todo}

\event{pVI:Bulavin Rebellion}{VI-w}{Bulavin's Rebellion}{1}{PBNotEvenWritten}
\history{1707-1708}
\begin{todo}
  Revolt in \paysAstrakhan.
\end{todo}


\event{pVI:Africa}{VI-x}{Troubles in Africa}{*}{JymNew}
\history{No precise date. Hypothetical clashes with inland African empires.}

\begin{todo}
  Should replace~\ref{pVI:Bantu Raids}.
\end{todo}

\phevnt
\aparag Roll one die on the following table: 1. \granderegionSenegal ;
2. \granderegionCotedivoire ; 3. \granderegionCotedor ;
4. \granderegionCameroun (except \province{Fernando Po}; 5. \granderegionGabon
; 6. \granderegionCongo ; 7. \granderegionAngola ; 8. \granderegionNyasa (two
Southern provinces) ; 9. \granderegionNyasa (two Northern provinces) ;
10. \granderegionKenya.
\bparag The natives in the two provinces designed are activated. They have a
strength of 4\LD and one \LeaderG, whatever the printed value.

\event{pVI:Camisards}{VI-y}{Revolt of the Camisards}{1}{JymNew}
\begin{todo}
  Maybe should be V-6 (2).
\end{todo}
\history{1702-1711}

\condition{If \ref{pV:Expulsion French Protestants} %
  % Edit de Fontainebleau, revocation de l'Edit de Nantes
  did not occur yet, apply it now in addition to this event.}
% Gros des combats de 1702 a 1704.  Je me souviens plus de ce que j'avais
% dit...  De mémoire, R- en Languedoc, R+ (+general (Roland/Cavalier)) + A-
% huguenot (+general (l'autre)) en Cevennes.  l'A a des renforts mais a un
% moment FRA peut payer pour virer les généraux.  J'avais du envoyer un mail
% avec tout ca mais je trouve plus.



\event{pVI:Barbaresques}{VI-z}{End of the Ottoman rule in North
  Africa}{1}{PBNotEvenWritten}
\history{17??}
\dure{Until the end of the game}

\effetlong
\aparag If \ref{pIV:Morocco} did not happen yet, apply it immediately in
addition to this event.
\aparag \TUR has a malus of \bonus{-3} to diplomacy with all \Barbaresques
(\paysCyrenaique, \paysTripoli, \paysTunisie, \paysAlgerie and \paysMaroc).
\bparag This malus supersedes the malus on \paysmaroc given by
\numberref{pIV:Morocco} and is not cumulative with it.

\stopevents

% Local Variables:
% fill-column: 78
% coding: utf-8-unix
% mode-require-final-newline: t
% mode: flyspell
% ispell-local-dictionary: "british"
% End:

% LocalWords: pV pVII pVI WoSS Vassalisation Methuen Heinsius WoPS WoAS dt
% LocalWords: Kurland Mughals Mazepa Camisards Barbaresques PBNew Altranst
% LocalWords: Leszinski Poniatowski PBMod RistoMod offensive Jym malus Quo
% LocalWords: pIV Risto Theresia PBnew JCD TODO von Sachsen Mughal pII Sadri
% LocalWords: subevents Safavids Azam JymNew PBNotEvenWritten GNW PPI
%  LocalWords:  Comuneros


\clearpage

% -*- mode: LaTeX; -*-
\chapter{Political Events of Period VII}
%\section{Period VII}
\label{events:pVII}



\subsection*{Event Table of Period VII}

\begin{eventstable}[Period VII events table]
  \tabcolsep=5pt\centering%
  \begin{tabular}{|l|*{5}{c}|l|}
    \hline
    1\up{st}\textarrow& 1-4 & 5-6 & 7 & 8 & 9 & 10 \\ \hline
    1 & 1  & 8  & 12 & 1  & R3  & \textbullet~1--2: \\
    2 & 2  & 9  & 19 & R18& 4	& +1 then \\
    3 & 3  & 10 & 2  & 10 & 5   & \nameref{events:pVI}\\
    4 & 4  & 11 & 18 & 11 & R12 & \textbullet~3--10: \\
    5 & 6  & 14 & 20 & 5  & R13 & \nameref{events:pVI}\\
    6 & 7  & 15& R5  & R6 & 4 	& \\
    7 & 13 & 16& 1   &  7 & 15  & \\
    8 & 19 & 21& 17  &  8 & 16  & \\
    9 & 1  & 4  & R8 &  9 & 7  	& \\ \hline
    10 & \multicolumn{6}{l|}{\nameref{events:pVI}} \\ \hline
  \end{tabular}
\end{eventstable}

\eventssummary{%
  pVII:Seven Years War|,%
  pVII:Bavarian Succession|,%
  pVII:Batavian Revolution|,%
  pVII:Independence War|E/E/E/E,%
  pVII:French Revolution|S{pVII:Revolution:Bastille}/%
  S{pVII:Revolution:Terror},%
  pVII:Bar Confederation|,%
  pVII:First Partition Poland|,%
  pVII:Second Partition Poland|E/E/E,%
  pVII:National Revival of Poland|S{pVII:NRP:Kosciusko}/%
  S{pVII:NRP:Commonwealth Revival},%
  pVII:Mameluks Revolt|,%
  pVII:Revolt Indonesia|,%
} \eventssummary{%
  pVII:Sale Corsica|,%
  pVII:Pugatchev Revolt|,%
  pVII:Potemkin|,%
  pVII:War Crimea|,%
  pVII:War Finland|,%
  pVII:Forward Balkans|,%
  pVII:Wars India|O{pVI:Wars India},%
  pVII:Vassalisation Hanover|O{pVI:Vassalisation Hanover},%
  pVII:William Pitt|,%
  pVII:Kaunitz|,%
  pVII:Comuneros|,%
  pVII:Xhosa|E/E,%
  pVII:USA-Morocco|,%
} \newpage\startevents



\event{pVII:Seven Years War}{VII-1}{The Seven Years War}{1}{PBnew}

\history{1756-1763}
\dure{until the end of the war caused by the event.}

\condition{}
\aparag Cannot happen before period VII if \PRU is not a major country and at
peace.
\bparag In this case, do not mark of an re-roll.

\phevnt
\aparag \PRU has a free \CB against \paysSaxe to be used at this turn or the
next one.
% (Jym) added:
\bparag Refusal to use this \CB cost \PRU 3 \STAB and \PRU is considered to
have lost the war for all the effects described below.
\bparag When \PRU uses this \CB, \paysSaxe propose an immediate white peace to
all its other enemies.
\bparag If \PRU does not use this \CB this turn, apply (in addition)
\xnameref{pVII:SYW:French Indian War}.
\aparag As a reaction to \PRU declaring war to \paysSaxe, \AUS has an
immediate free \CB against \PRU.
\aparag As a reaction to \PRU declaring war to \paysSaxe, \FRA has an
immediate normal \CB against \PRU.
\bparag If both \FRA and \AUS use these \CB, they are considered allied in the
war without need to sign a formal alliance.
% (Jym) Is this \CB in the right direction. I got the impression that the
% alliance reversal took place before the war really burst out. Then Frederic
% II took the initiative by attacking Saxony, but it's really \FRA attacking
% \ANG in Europe (taking Minorque with the naval victory of La Galissonniere
% over Byng then attack on Hanover by d'Estree (the "Forward" of Clash of
% Monarchs)).
\aparag As a reaction to \FRA declaring war to \PRU, \ENG has a free \CB
against \FRA.
\bparag If \ENG uses this \CB, \ENG and \PRU are considered allied in the war
without needing to sign a formal alliance.
\aparag \RUS has a \CB against \PRU and a \CB against \AUS (if at war) for the
duration of the war.
\bparag If \RUS uses one of these \CB, it is considered allied with the other
side in the war without need to sign a formal alliance.
\aparag Normal calls for allies may occur as a reaction to any of these
declarations of war.

\aparag As long as the event last, \RUS as a malus of \bonus{+3} to the
survival rolls of its monarch before \monarque{Pierre II}.

\phdipl
\aparag If \paysSaxe is at war against \PRU but its controller is not,
\paysSaxe is put in \EG of the first country at war against \PRU in the
following list: \AUS, \ENG, \SUE, \FRA, \RUS.
\bparag During the war, the controller of \paysSaxe has a bonus of \bonus{+5}
for diplomacy on any minor of the \HRE except \paysbaviere.

\phadm
\aparag Purchase limits for \PRU are doubled for the duration of the war.
% (JCD) Controlling the minor or the province ?
\bparag During the war, \PRU may raise troops in any province belonging to a
minor it controls.
\aparag At the first turn of the war, all minors at war must choose their
reinforcements in offensive attitude.
\bparag At the following turns, it must be either offensive of naval attitude.

\phpaix
\aparag As long as \monarque{Friedrich II} is alive, \PRU cannot be forced to
peace if at {\bf -3} in \STAB for two consecutive turns.
\bparag It can, however, be forced to peace if all its provinces are occupied.
\aparag If \PRU signs an unfavourable peace, \FRA and \AUS win 50 \VP (each)
if they were at war against \PRU.
\bparag In this case, if \ENG was at war it loses 25 \VP or 50 \VP is this was
an unconditional surrender.
\aparag If its side imposes an unconditional surrender (to either \PRU or
\AUS), \RUS can annex all provinces of \payspologne adjacent to \RUS
territory.
\bparag This counts as one peace condition for the alliance of \RUS.
\bparag The allies of \RUS will have a \CB against \RUS at the following turn
to contest this annexation.
\bparag If \payspologne is a special \EG of \FRA or \SUE per either
\ref{pVI:WoPS:Polish Victory} or \ref{pVI:GNW:Stanislas}, this annexation can
only be done if \RUS is not allied with the protector of \payspologne.
\bparag This annexation is impossible if Absolutism is established in
\payspologne.
\aparag If \PRU forces \paysSaxe to an unconditional surrender, it wins 25
\VP.
\aparag If \AUS is forced to unconditional surrender it loses 50 \VP and {\bf
  1} \STAB.


\subevent[pVII:SYW:French Indian War]{The French and Indian War}
\history{Colonial tensions erupted into a state of war in 1754 in America}

\phevnt
\aparag \FRA and \ENG are now in a state of overseas war.
\bparag This is not a declaration of war, hence there is no cost of \STAB for
any of them.
\aparag Reactions are allowed as if the war was continuing from a previous
turn except:
\bparag They may not generalise the war at this turn, unless using another
\CB.
\bparag They may not sign an armistice this turn.

% (Jym) Treaty of Paris I get the impression that getting the whole Canada
% area is a bit more than what is doable in a normal peace treaty in Europa. I
% propose a clause of grouped annexation similar to \HIS/\AUS in Italy: all
% \TP/\COL of a great area in the American enlarged view is equal to 1
% province. This should not be tied to SYW but let more hazy and for many more
% people (typically \ENG,\FRA,\HOL,\SUE)...  Overall, I am more keen about
% letting some fuzzy tricks than can lead to historical facts rather than
% linking to unique people and events to do these things.



\event{pVII:Bavarian Succession}{VII-2}{The War of Bavarian
  Succession}{1}{RistoMod}

\history{1778-1779}

\condition{}
\aparag Cannot occur if \paysbaviere is currently at war against \AUS. In that
case mark off and play \RD instead.

\phevnt
\aparag \paysbaviere offers to become a permanent \AM of \AUS.
\bparag If this offer is accepted, \paysbaviere cannot anymore fall below \AM
of \AUS, but diplomacy is still possible on it.
\bparag If the offer is refused, ignore the rest of the event.
\aparag If the offer is accepted, \PRU as free \CB against \AUS to be used
immediately.
\aparag If \PRU does not use the \CB, \AUS is considered to have won the war
for all relevant effects and \VP.
\bparag Normal calls for allies follow if a war is declared.
\bparag It is possible that \paysbaviere stays out of the war\ldots

\phpaix
\aparag If \AUS signs a white or unfavourable peace, \paysbaviere becomes a
normal minor again and \PRU win 20 \VP.
\aparag If \AUS signs a favourable peace, \AUS win 25 \VP and the previous
controller of \paysbaviere (if any) has a temporary \CB against \AUS at the
next turn.



\event{pVII:Batavian Revolution}{VII-3}{Batavian Revolution}{1}{RistoMod}

\history{1785-1787}
\dure{Until the \REVOLT are crushed or the government is overthrown.}

\condition{}
\aparag If \payshollande is a minor country, apply \ref{pVII:BR:Minor
  Holland}. The second time, apply \RD and mark off.
\aparag Else apply \ref{pVII:BR:Major Holland} (twice if needed).


\subevent[pVII:BR:Minor Holland]{Minor Holland in Revolution}

\phevnt
\aparag Place a \REVOLT \facemoins in each province owned by \HOLmin.
\aparag \HOLMin immediately proposes a peace based on the current peace
differential (or a white peace if the situation favours \HOLmin) to all its
enemies.
% (JCD) this event has to be rewritten for \HOL major. So this is meaningless,
% so removing.
% \bparag If this means separate peace, there is no extra penalty for \HOLmin
% in doing so.
\bparag Minor countries accept this peace.
% (JCD) same thing
% \aparag Then \HOL loses 3 \STAB and 50\ducats.

\phdipl
\aparag If \HOLmin and \PRU are allied, \PRU may intervene to help.
% (Jym) added
\bparag No other major may intervene.

\phadm
% (JCD) Changing this for a minor. Anyway, where would troops be risen?
% \aparag \HOL may not raise any new troop during the first turn of this
% event.
\aparag \HOLMin does not get any reinforcement roll for the first turn of this
event.

\phmil
\aparag \PRU may not send more than two stacks in provinces owned by \HOLmin.

\phpaix
\aparag \HOLMin keeps proposing peace to its enemies as long as \REVOLT still
exist during the peace phase.
\aparag If at the end of a turn, there are \REVOLT in more than half of
provinces belonging to \HOLmin, the minor has gone through a
revolution. Return the diplomatic marker of \HOLmin to neutral status (unless
activated in a war; in this case place it in \MA of the controller).
% (Jym) Deleting because \HOL major -> government overthrown normally.  2. If,
% at the end of a turn there are \REVOLT in more than half the provinces
% belonging to Holland, this minor is regarded to have gone through a
% revolution. Return the Dutch marker to neutral status (unless activated in
% which case place it in MA of the controller) and remove all \REVOLT markers.
% (JCD) Reinstating. This event HAS TO be rewritten for a \MAJ \HOL.

% (Jym) Deleting (JCD) Not reinstating this time, because diplomacy on minors
% during wars is really problematic.
% \aparag \PRU may do diplomacy on \HOLmin even if \HOLmin is at war somewhere
% else.


\subevent[pVII:BR:Major Holland]{War between Orangists and Patriots}

\phevnt
\aparag If at war, \HOL makes a mandatory white peace with all its enemies.
\bparag \MAJ allied to or at war with \HOL will be able to make a foreign
intervention in the Civil War (on any side).
\bparag Other countries will be able to intervene as mentioned below. No other
countries may intervene in the Civil War.
\aparag \HOL is in Civil War (see \ref{chDiplo:Religious Civil War}) between
the Patriots, the Republicans and the Orangists.
\bparag The Orangists use up to one \ARMY, 5\LDND and one \FLEET counter (for
example, from \paysroyalistes).
\bparag The Patriots use up to two \ARMY and 6\LD from \paysrebelles.
\bparag Republicans use up to one \ARMY, 5 \LDND and all possible naval forces
of \paysHollande.
\bparag \HOL can choose to take either the side of Patriots or Orangists. The
choice is made after the revolts have been rolled for.
\aparag[Rise of the Patriots] For each province of \HOL in Europe, roll
2d10. Add \bonus{-3} if \ref{pVII:French Revolution} already started. If the
roll is lower or equal to the income of the province, the province is in
\REVOLT. Roll for the strength of the revolt in \ref{table:alt-revolt-global}.
\bparag The Patriots control \provinceHolland.
\aparag For all \COL, roll 1d10. On 1, put a \REVOLT\faceplus and a Patriot
\LD; on 10, put an Orangists \LD and a control for the Orangists in the
province. All other \ROTW counters of \HOL are owned by the Republicans.
\bparag \HOL has to announce its support of one side at this point. It will
play this side.
\aparag[Orangists resistance] The Orangists call for help in that order: \PRU,
\SUE, non-revolutionary \FRA, the owner of the \emph{Spanish Low Countries}
after \ref{pV:WoSS}.
\bparag The first to answer the call will play the Orangists (if \HOL supports
the Patriots) and will be allowed an intervention of at most two stacks (not
one as per usual rules) of at most one \ARMY\faceplus.
\bparag Other countries will not be able to intervene.
\bparag If no country wishes to intervene in this list and \HOL chose the
Patriots, \PRU will play the Orangists.
\bparag The Orangists decide of one safe place (historically
\provinceGelderland) that they own (even if in \REVOLT). This must not be
\provinceHolland nor \provinceUtrecht. The \REVOLT is removed if there was
one, but a \LD or a \LeaderG is moved in another \REVOLT.
\aparag[Republicans and the VOC] Naval forces and most \ROTW Dutch settlements
will mostly stay out of the war. The moves of these forces will be played by
\ENG.
\bparag \ENG will be able to intervene with a normal foreign intervention.
\bparag Administrative actions will be very limited and played by \HOL, using
3/3/3 as Monarch values during all the war.
\aparag[Call for the Revolution] Revolutionary \FRA (after
\nameref{pVII:French Revolution} started) may be able to send one stack of
conventional troops and two stacks of Revolutionary troops to help
Patriots. It can declare its intervention during the military rounds if
sending Revolutionary troops; but it may not gain as much by doing so.
\bparag If \HOL supports the Orangists, and nobody supports the Patriots, then
\TUR will play the Patriots.
\aparag[The Dutch Fleet] Orangists pick one \FLEET counter, moved to one port
they control (if none, one port of their supporter or simply at sea). All
other naval forces go to Republicans.

\phdipl
\aparag \HOL can react to attacks on its minor countries. It can not do any
other diplomatic actions.

\phadm
\aparag[Incomes] Orangists and Republicans get land income from the provinces
they (or their allies) control in European provinces of \HOL.
\bparag Patriots get the land income from the provinces they control or that
are in \REVOLT.
\bparag Half (rounded down) of Vassal income goes to Orangists. The rest goes
to Republicans.
\bparag There is no commercial income. \ROTW income goes to whoever controls
the place (usually Republicans).
\bparag \MNU give their basic income (the fixed part) to the side getting
revenues from the province.
\aparag[Administrative actions] The only actions that can be done are paid on
\HOL \RT directly. They are: reactions to concurrence, improving already
existing \COL or \TP, improving already existing \TradeFLEET.
\aparag[Raising armies] Republicans have to pay for the maintenance of naval
and land forces in their keep first of all (what can not be paid for is
dismantled). With the rest, they may purchase troops only to be bought in
territories they control or in \ANG. \ANG may give money to
Republicans. Republican land forces are always \Conscripts.
\bparag On the first turn, land forces in Europe of \HOL are disbanded.
\bparag Patriots and Orangists have their own budget and a purchase limit of
2\LD. The first \ARMY\facemoins they buy on the first turn is \Veteran, the
rest is \Conscripts. Their supporter may give money.

\phmil
\aparag Province flooding (\ref{chSpecific:Holland:Flooding}) can not be used
during this event.
\aparag For movement, supply and attrition, provinces with \REVOLT are
friendly to Patriots unless an enemy force is within.
\bparag Patriots consider all cities with \REVOLT in the province as
blockaded.
\bparag \REVOLT are weak supply points for Patriots.

\phpaix
\aparag \HOL loses {\bf 1} \STAB. No \STAB increase is possible during the
event.
\aparag No armistice may be signed by the various sides.
\aparag[Victory of Orangists] It there are no more revolts and no more troops
of Patriots in national territory, Orangists get an automatic victory.
\bparag The Monarch is reinstated. \STAB of \HOL becomes {\bf +3} minus one
per turn of revolution.
\bparag If this event happens again, \HOL will have a \bonus{-2} to the
strength of \REVOLT.
\bparag The supporters of Orangists get 20 \VP (possibly including \HOL).
Supporters of Patriots lose 20 \VP.
\aparag[Victory of Patriots] If there are \REVOLT in all national provinces or
no more Orangists (or allies) troops in national territory or it is the third
turn of the revolution and there is still at least one \REVOLT or this is the
last turn of the game or \STAB is at {\bf -3} for two consecutive turns,
Patriots get an automatic victory.
\bparag All revolts are removed. The government is overthrown. Read below for
the lasting consequences.
\bparag A Monarch will be rolled anew at next turn, as if there were a
\terme{Dynastic Crisis}.
\bparag \STAB of \HOL becomes {\bf 0}.
\bparag The supporters of Patriots get 20 \VP (possibly including
\HOL). Supporters of Orangists lose 20 \VP.
\aparag[Victory of Republicans] Republicans are considered victors if any
other side wins in one turn or two turns. They lose if the revolution ends
after three turns.
\bparag \ENG is entitled to 1 or 2 compensations (given by the Orangists or
the Patriots or taken to the VOC during the troubles) of \HOL's choice: 1
level in a \CTZ, 2 levels in a \STZ, one \COL, one \TP. Automatic concurrence
may follow from this. There are two compensations if the victory was in one
turn. \ENG gets two compensations for a victory of either side in 1 turn.
\bparag If \ENG lost military forces (either naval or land) during the
Revolution, it is entitled to 20 \VP in addition. If the Republicans lost,
\ENG loses 20 \VP.
\aparag There are no other peace outcomes.
\aparag In case of victory, supporters (including \HOL) of the winning side
gain 20\VP and the forces of the winning side are converted to \HOL
counters. Supporters of the losing side lose 20\VP and the forces of the
losing side are disbanded.

\effetlong
\aparag In case the Patriots win, apply the following points:
\bparag The \terme{Stadhouder} government is no more possible.
\bparag All monarchs have a \bonus{+2} to their survival roll. \terme{Dynastic
  Crisis} will cost 1 \STAB with no other consequences.
\bparag The maximum ADM value of the Monarch (or Minister) is now 7. However,
the real rolled-for value is used for rolling the next Monarch.
\bparag The maximum DIP value of the Monarch is now 5. However, the real
rolled-for value is used for rolling the next Monarch.
\bparag The minimum MIL value of the Monarch is now 7.
\bparag If the event happens again, the \REVOLT strength will have a
\bonus{+2} modifier.
\bparag The VOC is dissolved. The basic \LeaderGov is available each turn only
if 1d10 (rolled during the Monarch Survival phase) is even. This also removes
some constraints on \TFI and turns the \TPaction available each turn into a
\TPaction or \COLaction, at the choice of \HOL.
\bparag National provinces of \HOL will count in favour of \FRA for
the``natural frontier'' objectives (not for the rest).
\bparag \HOL loses 1 diplomatic action.
\bparag \HOL has a mandatory defensive alliance with Revolutionary \FRA for at
least three turns (as soon as possible)



\event{pVII:Independence War}{VII-4}{War of Independence in the
  Colonies}{*}{RistoMod}

\date{1775-1783}
\dure{Until the end of the rebellion.}

\condition{}
\aparag If none of the following already occurred, do not mark off and
re-roll:
\bparag \ref{pVII:SYW:French Indian War} (only if the war is already
finished).
\bparag \ref{pVII:William Pitt}.
\bparag \ref{pVII:French Revolution}
% (Jym) I split in three separate events, it is easier this way. Just getting
% the initial group is sufficiently difficult to deserve its own
% subevent. This makes the rest easier to read and write.
\aparag The first time, apply \xnameref{pVII:IW:First Revolt}, the second and
subsequent times, apply \xnameref{pVII:IW:Further Revolts}. Each time,
\xnameref{pVII:IW:Where} is used to determine which colonies try to get their
independence.


\digression[pVII:IW:Where]{Where does the revolt occurs ?}
A revolutionary war erupts in a group of colonies. The target group is chosen
by first selecting a subcontinent and then a major country. The major country
must have a certain number of colonies in the target subcontinent in order to
start the revolt. The first major country meeting the criteria is subject to
the revolution.
% (Jym) Not possible in the Caribbeans? Brazil?  (JCD) Indonesia ?
\aparag The possible target subcontinents are, in order:
\bparag \continent{North America}
\bparag \continent{South America}
\bparag \continentBrazil
\bparag \continentIndia
\bparag \continentAsia (except \continentIndia)
% (Jym) Not possible for \POR, even minor (Brazil).
\aparag \label{pVII:IW:Protestant} The possible target players are the
protestant ones in the following list:
\bparag \ENG, \FRA, \HIS, \HOL
\aparag The target group of colonies is elected by first looking for players
meeting the criteria in the first subcontinent, then the second and so on.
% (Jym) If I remember correctly, what follows was changed (condition of
% revolts). I copy Risto anyway.
\aparag The target group of colonies must contain at least 10 levels of \COL
in four adjacent provinces (with land access between them).
% (Jym) This makes a revolt in the Caribbeans or in the DEI impossible.
\bparag It is possible that some of these provinces have no \COL in them as
long as there are 10 levels of \COL or more in four provinces.
% (Jym) No malus for \ANG if it is not eligible ?  (JCD/Jym) If the target
% exists, the controller of the revolt should choose three large areas that
% uprise in the continent (maybe not connex, this is a tactical consideration
% best handled by humans)
\aparag If no target exists, nothing happens but the event is nonetheless
considered played (mark off, do not re-roll, do not play \RD).
\aparag Once the target group of colonies is found, roll 1d10 with the
following modifiers:
% (Jym) Revolt if result is large.

% (Jym) Enemy present -> foreign agitation?  Wiki "French and Indian wars"
% (plural) :

% The overwhelming victory of the British played a role in eventual loss of
% their thirteen American colonies. Without the threat of French invasion, the
% American colonies saw little need for British military protection. In
% addition, the people resented British efforts to limit their colonization of
% the new French territories to the west of the Appalachian Mountains, as
% stated in the Proclamation of 1763, in an effort to relieve encroachment on
% Native American territory. These pressures contributed to the American
% Revolutionary War.

\begin{modlist}
\item[\bonus{-5}] If no other player has a \COL inside the four target
  provinces.
\item[\bonus{+1}] For each other player that has \COL or \TP within two
  provinces of the group or
\item[\bonus{+2}] For each other player that has \COL or \TP adjacent to the
  group.
  % (Jym) Metropolitan troops -> quick squashing of the revolt (?).
\item[\bonus{-1}] If the player has any \LD in the group or
\item[\bonus{-2}] If the player has any \ARMY in the group.
  % (Jym) Indian allies -> squashed revolt?
  % Or more probably peace with Indian => no need to be protected by the
  % metropolis.
\item[\bonus{-2}] If the player has \dipFR or \dipAT with a minor adjacent to
  the group.
\item[\bonus{+3}] If another player has \dipFR or \dipAT with a minor adjacent
  to the group and the player has neither \dipFR nor \dipAT with this minor
  country.
\end{modlist}
\aparag \label{pVII:IW:Test} If the result is 5 or more, the rebellion
occurs. A non-modified 10 is an automatic rebellion while an non-modified 1
always means that no rebellion occurs.
\bparag If no rebellion occurs, nothing happens but the event is nonetheless
considered played (mark off, do not re-roll, do not play \RD).

% (Jym) \wikipedia separates "American Revolution" -> political and social.
% "American Revolutionary War" -> military.


\subevent[pVII:IW:First Revolt]{American Revolutionary War}

\condition{Choose a target \MAJ and group of colonies as indicated in
  \ref{pVII:IW:Where}.}

\phevnt
\aparag The \MAJ choose one \COL within the revolted group. Place a \REVOLT
\facemoins in each other \COL of the group.
\bparag Place 3\LD (of \paysusa) on one of the \REVOLT .
\bparag Rebels control all the fortresses in the revolted colonies.

\phdipl
% (Jym) Removing, from memory this is replaced by the special power of \ANG
% The target may use a bonus of +5 to certain minors that can exceptionally be
% used in \ROTW map against the rebels (only). The bonus can only be used to
% raise these minors to CE and not higher. These minors can only be used when
% inactive and only by following the restrictions of CE (even if they are
% actually higher up in the diplomatic track). The minors in question are the
% following: \ANG: \payshanovre and \payshesse b. \FRA: \payssavoie and
% \payslorraine c. \SPA: Portugal d. \HOL: \payspalatinat (Jym) adding this;
% Rationale: the minor is just created it is on nobody track, so only a
% limited intervention of the \MAJ ally is possible, this matches the French
% intervention (JCD) I think \paysusa should rather call neighbouring
% countries not only \FRA/\ENG.
\aparag The rebels calls for allies as indicated in the preferences of
\paysusa.

\phadm
\aparag The \MAJ does not get income from the \COL that initially revolted,
even if the \REVOLT are suppressed.
\bparag It cannot either raise troops there or use the colonial militia.
\bparag It can, however, build fortresses in these \COL.
\aparag The \MAJ receives no income from \TradeFLEET in \STZ adjacent to a
\COL that initially revolt, even if the \REVOLT are suppressed.
\bparag All other player get double income (but not double bonus) from
\TradeFLEET in these \STZ.
% (Jym) Maybe allow naval if it takes place in Caribbeans or Indonesia?
\aparag Rebels can choose reinforcements in either offensive or defensive
attitude. They use the counters of \paysusa.
\aparag If the \MAJ has a general that can be used by \paysusa (either
\leaderWashington or \leader{La Fayette}), this general goes to the side of
the rebels.
\bparag If \leaderArnold is alive, he joins the rebels.
\bparag The rebels must have at least two generals for the duration of the
event. Use the unnamed generals of \paysusa if needed.
\aparag The \MAJ receives at no cost a mercenary that can be used in the \ROTW
and is considered to have rank Z.
\aparag \leaderWashington and \leader{La Fayette}, if not already rebels, can
be sent by their owner (\ENG or \FRA) to help them.
\bparag The owner chooses each turn whether it keeps the general or send him
to help the rebels.
\bparag This general is in addition to the minimum two generals of the rebels.
\bparag Once the event is finished, this leader goes back to his major
country.

\phmil
\aparag \REVOLT are supply sources for the rebel troops.
\aparag Remember that \paysusa (hence, the rebels) roll for reinforcements
after each winter round and not only once per turn.

\phpaix
\aparag The event stops at the end of the second turn of revolt.
\bparag If all \REVOLT have been suppressed by the end of the second turn,
\MAJ wins the war.
\bparag Otherwise, the rebels win.
\aparag If the rebels are crushed, remove all the units of the rebels, remove
the named leaders of \paysusa from the game (not the one sent by a major).
\aparag If the rebels win, the minor country \paysusa is created.
\bparag All the \COL in the initial group of revolt are part of \paysusa, even
those where the \REVOLT were suppressed.
\bparag All the provinces of \paysusa are considered as European provinces for
all game purposes.


\subevent[pVII:IW:Further Revolts]{Bolivarian Revolutions}
\history{Spanish American Wars of Independence (Bolivar):
  1808-1829/Independence of Brazil: 1823-1825.}

\condition{}
\aparag If another \xnameref{pVII:Independence War} is currently occurring, do
not mark off and re-roll.
\aparag If another \xnameref{pVII:Independence War} is already finished and
was won by the rebels, if \xnameref{pVII:Revolution:Bastille} did not occur
yet, apply it instead.
\aparag Otherwise (revolt crushed or \nameref{pVII:Revolution:Bastille}
already occurred or a previous occurrence resulted in ``no revolt'' after the
test of \ref{pVII:IW:Test}), choose a target country as indicated in
\ref{pVII:IW:Where}, ignoring the religion condition of
\ref{pVII:IW:Protestant}.
\aparag Once a target is found, if the die roll of \ref{pVII:IW:Test}
indicated a revolt, roll another die and apply the corresponding result:
\begin{modlist}[1.5em]
\item[10] Another revolt occurs
\item[9] Extension to a near continent
\item[6--8] Small revolt
\item[1--5] Nothing happens. The event is nonetheless considered played (mark
  off, do not re-roll, do not play \RD).
\end{modlist}
\aparag[Another revolt occurs] Another revolt occurs as described in
\ref{pVII:IW:First Revolt}. Both revolts are separate one from another and, if
created, both countries are different. Use whatever name and counters you wish
to refer to the second and subsequent ones (Canada, Bolivia, Brazil,
Indonesia, \ldots)
\aparag[Extensions to a near continent] If the target subcontinent is adjacent
to the original one (either \continent{North America} and \continent{South
  America} %
% (Jym) \continentCaraibes, \continentBresil
or \continentIndia and \continentAsia), a new revolt occurs as above,
otherwise treat as a \emph{Small revolt} below.
\aparag[Small revolt] Place three \REVOLT \facemoins in the target group of
colonies. Don't use any minor forces. No independence may result from these
\REVOLT . Another \ref{pVII:Independence War} may occur before all the \REVOLT
are crushed.



\event{pVII:French Revolution}{VII-5}{The French Revolution}{2}{PBMod}

\history{1789-1799}
\begin{histoire}
  The first event corresponds to the bankruptcy of the French monarchy as well
  as the peasant crisis leading to the Storming of the Bastille and a change
  of government. Several possible new forms of government can exists depending
  on the choices of the player and the other majors. The second event
  corresponds to the internal dynamics of the Revolution yielding to
  uncontrolled effects.
\end{histoire}
\dure{until the end of the game.}

\condition{}
\aparag If none of the following happened, do not mark off and re-roll:
\bparag End of \xnameref{pVII:Seven Years War}.
\bparag Beginning of \xnameref{pVII:Independence War} (the revolt must have
started).
\bparag \xnameref{pVII:Batavian Revolution} is finished and was successful.
\aparag The first time, apply \xnameref{pVII:Revolution:Bastille}. The second
time, apply \xnameref{pVII:Revolution:Terror}.
\begin{designnote}
  \textit{``\`{A} partir de la R\'{e}volution, les r\`{e}gles de bon sens
    cessent de s'appliquer.''}\\
  ~\hfill (Pierre, August 2007).
\end{designnote}


\subevent[pVII:Revolution:Bastille]{Storming the Bastille}

\phevnt
\aparag \textbf{Political and social crisis}
\bparag If \FRA is at war against another \MAJ, it loses 1 \STAB. Otherwise,
it loses 3 \STAB.
% (Jym) What about minors on French track? I suppose they should be kept for
% period/game \VP, but that they do, in fact, go away (or at least that \FRA
% cannot call for allies). \FRA should not be able to do any diplomatic
% actions at all, too. In doubt, I will leave that for the \MAJ (JCD) \HOLmin
% after Batavian revolution should be able to stay
\bparag \FRA is considered to have broken its alliances with all countries
(major or minor). This does not cause any extra loss of \STAB.
% (Jym) It is not too difficult to count immediately the \VP of diplomacy then
% switch the minors to Neutral. It is more complicated to count the \VP of
% vassal provinces for the end game.  (JCD) This is not reasonable to count on
% a vassal for French territory, that's what it means.
\bparag Roll for two \REVOLT in \paysmajeurFrance.
\bparag Future survival rolls for the French monarch get a malus of
\bonus{+2}. The malus will be \bonus{+5} if \FRA goes to the
\monarqueConvention government.
\bparag The following countries have a free \CB against \FRA until the end of
the game: \ENG, \AUS, \PRU, \HIS, \HOL (unless if \ref{pVII:Batavian
  Revolution} was won by the rebels).
\bparag \FRA has a normal \CB until the end of the game against each major
country and against each minor country adjacent to its territory.
\bparag These \CB can be used as diplomatic reaction to any other diplomatic
announcement.
\aparag \textbf{Economical crisis}
\bparag \FRA loses 100\ducats. Then its Royal Treasure is halved %
% (Jym) RT<0 possible
with a minimum loss of 50\ducats.
\bparag From now on, \FRA loses 10\% of its gross income
(\lignebudget{Events}).
\bparag From now on, \FRA pays inflation as if it were bringing gold from
\continentAmerica.

\phdipl
\aparag If \payspologne is a special \EG of \FRA (per \ref{pVI:WoPS:Polish
  Victory}), as soon as another \MAJ declares war on \FRA, so does
\payspologne. Troops of \payspologne are allowed to cross the \HRE.
\aparag At the end of each diplomatic phase, test for a change of government
in \paysmajeurFrance. Roll 1d10 modified as follows:
\begin{modlist}
\item[\bonus{-4}] if \xnameref{pVII:Independence War} never occurred;
  % (Jym) The second event IW triggers Bastille. So there is at maximum one IW
  % before Bastille, no doubt on the winner of IW.
\item[\bonus{-2}] if \nameref{pVII:Independence War} is finished and the
  rebellion was crushed;
\item[\bonus{+2}] if \nameref{pVII:Independence War} is finished and \paysusa
  has been created;
\item[\bonus{+2}] if \FRA used this turn a \CB provided by this event;
\item[\bonus{+4}] if \FRA is at war without declaring any war this turn;
  % (Jym) Next bonus just for the destitution turn? Not very good. Once a King
  % died, the end is not far.
\item[\bonus{+6}] if the king of \FRA died during this event.
\end{modlist}
\aparag The result of the die roll tells which is the new government of \FRA:
\begin{modlist}[2em]
\item[1--6] The government is unchanged.
\item[7--13] The government switches to (or remains)
  \monarqueConvention. Apply \xnameref{pVII:Revolution:Convention}.
\item[14+] The government switches to \monarqueTerror. It won't be able to
  change back to anything else: stop doing this test each turn. Apply
  \xnameref{pVII:Revolution:TerrorGov}.
\end{modlist}

\phmil
\aparag During all wars caused by this event, enemies of \FRA are considered
allied inside the territory of \FRA or when fighting French troops. They may
be at war elsewhere and nonetheless be allied (and stack together or intercept
French troops attacking the other country,\ldots) fighting \FRA.
% (Jym) from memory
\aparag Countries at war against \FRA are limited to 1 stack inside the
national territory of \FRA.
\bparag They are not limited if fighting out of the national territory of
\FRA.
\bparag The \ARMY provided by \xnameref{pVII:Revolution:Nobles} does not count
toward this limit. It is always allowed inside \FRA.

\phpaix
\aparag If \villeParis is controlled by the enemies of \FRA and there are no
more \ARMY of \FRA in play, the Revolution is crushed and a new king is put on
the throne of \FRA.
\bparag The game ends at the end of this turn.
\bparag Each country at war against \FRA wins 30 \VPs.
\bparag \FRA wins 15 \VPs at the end of the game if the revolution has not
been crushed.

\begin{digressions}[Effects of the Revolution]


  \subevent[pVII:Revolution:Convention]{Convention (and constitutional
    monarchy)}
  \history{1789-1792}
  % (Jym) Immediate effects in \phdipl because the change test is done
  % in \phdipl...

  \phdipl
  \aparag When the government changes to \monarqueConvention:
  \bparag Apply \xnameref{pVII:Revolution:Nobles},
  \xnameref{pVII:Revolution:Chouans}, \xnameref{pVII:Revolution:Armies} and
  \xnameref{pVII:Revolution:Natural Frontiers}.
  \bparag Roll for one \REVOLT in \FRA.

  \effetlong
  \aparag If still alive, the king of \FRA has a \bonus{+5} malus to all his
  survival rolls (instead of the \bonus{+2} for the Revolution).
  \aparag If the king dies, he is replaced by \monarqueConvention with values
  3/6/7. This government never rolls for survival.
  \aparag During each event phase of \monarqueConvention, roll for one \REVOLT
  in \FRA.


  \subevent[pVII:Revolution:TerrorGov]{Reign of Terror and Directoire}
  \history{1792-1799}

  \phdipl
  \aparag When the government switch to \monarqueTerror:
  \bparag The French king (or \monarqueConvention) is immediately killed, he
  is replaced by \monarqueTerror with values 5/6/9. This government never
  rolls for survival.
  \bparag Roll for 3 \REVOLT in \FRA. % Revolts of the Federated
  \bparag If they were not already activated, apply
  \xnameref{pVII:Revolution:Nobles}, \xnameref{pVII:Revolution:Chouans} and
  \xnameref{pVII:Revolution:Natural Frontiers}.
  \bparag Apply \xnameref{pVII:Revolution:Levee Masse}
  \bparag Increase the \DTI and \FTI of \FRA by 1 each (max. 5).
  \bparag Each \MAJ has a free \CB against \FRA to be used immediately.

  \phadm
  \aparag At the turn the government switch to \monarqueTerror, the gross
  income of \FRA is halved (round down, \lignebudget{Events}). This is not
  cumulative with the permanent \bonus{-10\%} caused by the event.
  % \bparag This replaces the loss caused by \STAB.

  \effetlong
  \aparag During each event phase of \monarqueTerror, roll for two \REVOLT in
  \FRA.

  \phpaix
  \aparag[End of Modern History] The game ends at the end of the second turn
  of \monarqueTerror.

  % (Jym) \wikipedia English uses "\'Emigr\'e" (in VF).


  \digression[pVII:Revolution:Nobles]{\'Emigr\'es}

  \phadm
  \aparag The first country at war against \FRA in the following list gets the
  benefits of the \'Emigr\'es: \AUS, \PRU, \HIS, \ENG, \payspologne (and its
  controller).
  \aparag The \MAJ gets a French Royal \ARMY\facemoins with a \LeaderG\anonyme
  of \FRA.
  % (Jym) adding
  \bparag This \ARMY can appear in any province owned by \FRA or by the \MAJ
  receiving it.
  \bparag It is considered class \CAIII\ with 4 artilleries per
  \ARMY\faceplus.
  % (Jym) In pVII, \FRA has 6 art/A, class III has 5. 4 art/A is either \RUS
  % or \TUR.
  \aparag This \ARMY can be reinforced (or recreated if destroyed) at the cost
  of the French royal troops.
  % (Jym) adding
  \bparag This \ARMY can be raised again or receive reinforcements in any
  province owned by the \MAJ receiving it or any French province either in
  \REBELLION or \REVOLT or controlled by another country.
  % (Jym) adding
  \aparag This \ARMY is freely maintained in veteran (new troops are
  conscripts as per normal rules).
  % (Jym) adding
  \aparag This \ARMY must fight against \FRA. If in \FRA it cannot leave the
  provinces in or adjacent to \FRA national territory and if created out of
  \FRA it must goes to \FRA by the shortest path. It is considered allied with
  all countries except \FRA. It can co-exist with troops all countries but
  \FRA and will never take part in any battle except against \FRA.


  \digression[pVII:Revolution:Chouans]{Chouans and Royalist Uprisings}

  \phdipl
  \aparag \terme{Chouans} are played by \ENG (even if not at war against
  \FRA).
  \aparag Place a \REBELLION \facemoins in each \provincePoitou and
  \provinceVendee.
  \bparag French troops in these provinces must retreat.
  \aparag Place a rebel \ARMY\faceplus and a general in one of these
  provinces.

  \phadm
  \aparag As long as a \REBELLION exists in either \provincePoitou,
  \provinceVendee, \provinceMorbihan, \provinceArmor or \provinceFinistere,
  the \terme{Chouans} get 1\LD in reinforcement (except the first turn).

  \phmil
  \aparag Instead of moving, 1\LD may ``hide'' in \provinceVendee (only). It
  does not count as military presence any more but gives a malus of \bonus{-2}
  to suppress the \REBELLION .
  \bparag If the \REBELLION is suppressed, this \LD is destroyed.
  \aparag These \REBELLION are friendly to any enemy of
  \FRA. \REBELLION\Faceplus are also supply sources for any enemy of \FRA.


  \digression[pVII:Revolution:Natural Frontiers]{Natural Frontiers}

  \condition{}
  % (Jym) I redefined the natural frontiers along the Rhine
  \aparag The ``Natural Frontiers'' of \FRA consist in:
  \bparag All national provinces of \FRA.
  \bparag All provinces adjacent to national provinces of \FRA except those in
  \HIS or \paysSuisse.
  \bparag All provinces on the left-hand side of river Rhine, that is all the
  provinces between \FRA and (included) \provinceAlsace, \provincePfalz,
  \provinceTrier, \provinceKoln, \provinceLimburg, \provinceUtrecht and
  \provinceZeeland.

  \phpaix
  \aparag \FRA automatically annexes any province within its Natural Frontier
  that it militarily controls during peace phase, unless they belong to
  Patriotic \HOL (see~\ref{pVII:Batavian Revolution}).


  \digression[pVII:Revolution:Armies]{Revolutionary Armies}

  \phadm
  \aparag \FRA can now use the Revolutionary \ARMY counters.
  \bparag Each new \ARMY raised from now on is Revolutionary.
  \bparag Already existent (royal) \ARMY are not affected and stay until
  destroyed or disbanded.
  \bparag \FRA may not have more than 6 \ARMY counters in play at the same
  time.
  \bparag Both the royal counters (of \FRA) and the new revolutionary counters
  (labelled ``Révolutionnaires'') belong to the same country for all purpose
  of leadership.

  \aparag Recruitment and upkeep cost of Revolutionary \ARMY is halved (upkeep
  of royal \ARMY is unchanged).

  \aparag Land recruitment limit is doubled.

  \aparag Naval recruitment cost is doubled.

  \aparag \FRA may not used Licensed privateers as described in
  \ruleref{chSpecific:France:Privateers}.

  \aparag[Revolutionary leaders] [BLP]
  \bparag All leaders of \FRA are dismissed. \FRA now uses the revolutionary
  leaders (excluding \leaderBonaparte).
  \bparag The leaders limits for \FRA is now 3\LeaderG/1\LeaderA.
  % En pVII : 3G, 4A, 1C@
  \bparag The revolutionary leaders are treated as \anonyme leaders rather
  than named ones. That is, \FRA draws them at random in order to reach
  its limits and they change every turn.

  \digression[pVII:Revolution:Levee Masse]{``La Patrie en danger''}
  All the effects of \xnameref{pVII:Revolution:Armies} are applied. In
  addition:

  \phdipl
  \aparag All French \ARMY are immediately replaced by Revolutionary \ARMY.

  \aparag \FRA may have up to 8 \ARMY counters in play.

  % \aparag Remove all named leaders of \FRA. \FRA gets 5 unnamed generals and 1
  % unnamed admiral (who can go in the \ROTW) (these are the new limits for
  % \FRA).
  \aparag[Revolutionary leaders] [BLP]
  \bparag All leaders of \FRA are dismissed. \FRA now uses the revolutionary
  leaders.
  \bparag The leaders limits for \FRA is now 5\LeaderG/1\LeaderA.
  % En pVII : 3G, 4A, 1C@
  \bparag The revolutionary leaders are treated as \anonyme leaders rather
  than named ones. That is, \FRA draws them at random in order to reach
  its limits and they change every turn.

  \aparag General \leaderwithdata{Bonaparte} is available for \FRA during the
  first turn of \monarqueTerror, starting with the first round after
  W2. % (Jym) : S3 or later
\end{digressions}


\subevent[pVII:Revolution:Terror]{Reign of Terror (Robespierre)}
\history{1792}
\dure{until the end of the game.}

\condition{}
% (Jym) A second IW may be running after the first one was crushed and allows
% Robespierre to come. If there were several IW, one victory is enough.
\aparag Can happen only if \xnameref{pVII:Independence War} is ongoing or if
\paysusa has already been created.

\phevnt
\aparag \FRA loses 1 \STAB.
\aparag \FRA goes to \monarqueTerror. Apply
\xnameref{pVII:Revolution:TerrorGov}.

\phmil
\aparag The military phase starts in W0.

\phpaix
\aparag The game ends at the end of this turn.



\event{pVII:Bar Confederation}{VII-6}{The Confederation of the Bar}{1}{PBnew}

\history{1768}

\condition{}
\aparag Cannot occur if there is no more \payspologne. in that case, mark off
and play \RD.
\aparag Cannot happen before the start of the war caused by \ref{pVI:WoPS}. In
that case do not mark off and re-roll.
\aparag Cannot happen if \ref{pVII:Second Partition Poland} already occurred
and the partition was accepted (with or without war) at least once. In that
case, mark off and play \RD.
\bparag Can, however, occur if \ref{pVII:First Partition Poland} occurred and
the partition was accepted.

\phevnt
\aparag Absolutism is established in \payspologne.



\event{pVII:First Partition Poland}{VII-7}{First Partition of
  Poland}{1}{PBnew}

\history{1772}

\condition{}
\aparag If \POL is still a major country, do not mark off and re-roll.
\aparag If there is a war between at least two of the following countries:
\RUS, \AUS, \PRU, do not mark off and re-roll.
\aparag If \payspologne doesn't exist any more, mark off and play \RD instead.
\aparag Depending on the current status of \payspologne, apply the correct
subevent (apply the first matching case). Only one such subevent may occur in
the game. In each case, the partition may be accepted and is described in
\xnameref{pVII:1PP:Partition}.
\bparag If Absolutism is established in \payspologne, apply
\xnameref{pVII:1PP:Absolutism or Protector}.
\aparag If \payspologne is a special \EG of either \FRA or \SUE as per
\ref{pVI:WoPS:Polish Victory} or \ref{pVI:GNW:Stanislas}, apply
\xnameref{pVII:1PP:Absolutism or Protector}.
\bparag If \payspologne is neutral or on the diplomatic track of either \RUS,
\AUS or \PRU, apply \xnameref{pVII:1PP:Poland Neutral}.
\bparag If \payspologne is on the diplomatic track of another major who
accepts the partition, apply \xnameref{pVII:1PP:Poland Neutral}
\bparag Otherwise, apply \xnameref{pVII:1PP:Poland Minor}.


\digression[pVII:1PP:Partition]{First Partition Plan of \payspologne}
\aparag The proposed partition of \payspologne gives the following provinces
to each major country:
\bparag \RUS gets all the Polish provinces in \regionUkraine,
\provinceSeveria, \provinceSmolenska, \provinceBaltarusija and
\provincePolacak.
\bparag \PRU gets all the provinces of \region{Duche de Prusse} and
\province{West Preussen}.
\bparag \AUS gets all the Polish provinces formerly part of \payshongrie,
\provinceMorava and \provinceMalopolska
\bparag \SUE gets a province of its choice, not part of the share of any other
country, adjacent to its territory.
\aparag If some of the provinces explicitly mentioned (not those part of a
group) no more belongs to \payspologne, the major instead gets a free \CB
against the owner of the province for the next diplomacy phase.
\aparag The acceptance of the partition plan depends on the status of
\payspologne and the result of the ensuing war.


\subevent[pVII:1PP:Absolutism or Protector]{\payspologne is absolutist or has
  a protector}

\phevnt
\aparag \RUS, \AUS, \PRU and \SUE all have a normal \CB against \payspologne
and its protector.
\bparag If \payspologne has no protector (but is absolutist), it call for
allies as per normal rules, the major accepting to help it has a free \CB
against all countries that declared war to \payspologne and is called
protector in the rest of the event.
\aparag If several countries declare war on \payspologne using this \CB, they
can choose to be allied for the duration of the war without need to sign a
formal alliance.
\bparag However, they can also choose to wage separate wars in which case they
can fight among them inside the territory of \payspologne and the national
territory of \POL. In this case, each alliance is considered separately for
the peace conditions.
\bparag There may be several different alliances fighting against \payspologne
(and among themselves).

\phpaix
\aparag \payspologne won't sign a separate peace in this war.
\aparag If the protector signs an unfavourable peace of level 3 or more, or if
\payspologne without protector signs an unconditional surrender, the following
effects are added to the peace:
\bparag \payspologne becomes a normal minor (and no more a special \EG).
\bparag \payspologne becomes neutral.
\bparag Absolutism is abolished in \payspologne
\bparag From now on, any country can annex the capital of \payspologne.
\bparag Instead of all peace conditions, the enemies of \payspologne can
choose to apply the partition proposed in \xnameref{pVII:1PP:Partition}, in
which case only the countries that were at war against \payspologne get their
share.


\subevent[pVII:1PP:Poland Minor]{\payspologne is a regular ally}

\phevnt
\aparag \RUS, \AUS, \PRU and \SUE all have a free \CB to be used conjointly
against \payspologne and its diplomatic patron.
\aparag If several countries declare war on \payspologne using this \CB, they
can choose to be allied for the duration of the war without need to sign a
formal alliance.
\bparag However, they can also choose to wage separate wars in which case they
can fight among them inside the territory of \payspologne and the national
territory of \POL. In this case, each alliance is considered separately for
the peace conditions.
\bparag There may be several different alliances fighting against \payspologne
(and among themselves).

\phadm
\aparag \payspologne must take reinforcements in defensive attitude for the
duration of the war.

\phpaix
\aparag \payspologne may sign a separate peace as per normal rules.
\aparag If \payspologne or the major helping it signs an unfavourable peace of
level 3 or more, the following effects are added to the peace:
\bparag \payspologne becomes neutral.
\bparag From now on, any country can annex the capital of \payspologne.
\bparag Instead of all peace conditions, the enemies of \payspologne can
choose to apply the partition proposed in \xnameref{pVII:1PP:Partition}, in
which case only the countries that were at war against \payspologne get their
share.


\subevent[pVII:1PP:Poland Neutral]{\payspologne is not defended}

\phevnt
\aparag \payspologne becomes neutral.
\aparag The partition described in \xnameref{pVII:1PP:Partition} is accepted
and every country take is share.



\event{pVII:Second Partition Poland}{VII-8}{Second Partition of
  Poland}{*}{PBnew}

\history{1791, 1793}

\condition{}
\aparag If \ref{pVII:First Partition Poland} did not occur yet, do not mark
off and re-roll.
\aparag If there is a war between at least two of the following countries:
\RUS, \AUS, \PRU, do not mark off and re-roll.
\aparag If \payspologne doesn't exist any more, mark off, play and \RD with
the \REVOLT in \POL.
\bparag In addition, if \ref{pVII:Independence War} already occurred at least
once, play that event again.
\aparag The event is resolved in the same way as \ref{pVII:First Partition
  Poland} (depending on the status of \payspologne) but with the partition
plan described here.
\aparag This event may occur several times.


\digression[pVII:2PP:Partition]{Second and following Partition Plans}
\aparag The proposed partition of \payspologne gives the following provinces
to each major country:
% (Jym) Smolensk disappeared of the \RUS package between 1PP and 2PP.
\bparag \RUS gets all the Polish provinces in \regionUkraine,
\provinceSeveria, \provinceBaltarusija and \provincePolacak. If none of the
belong to \payspologne, \RUS gets instead \provinceLietuva, \provinceZemaitija
and \provincePrypec.
\bparag \PRU gets all the provinces of \region{Duche de Prusse} and
\province{West Preussen}. If none of the belong to \payspologne, \PRU gets
instead \provinceDanzig, \provinceWielkopolska and \provinceMazowia.
\bparag \AUS gets all the Polish provinces formerly part of \payshongrie,
\provinceMorava and \provinceMalopolska. If none of them belong to
\payspologne, \AUS gets instead \provinceWolyn and \provinceLublin.
\bparag \SUE gets a province of its choice, adjacent to its territory, even
one part of the share of another country.
\aparag If some of the provinces explicitly mentioned (not those part of a
group) no more belongs to \payspologne, the major instead gets a free \CB
against the owner of the province for the next diplomacy phase.
\aparag If some provinces are claimed by several countries, the one occupying
it at the time of the partition annexes the province. \SUE does if nobody
occupy it.



\event{pVII:National Revival of Poland}{VII-9}{National Revival of
  Poland}{2}{PBnew}

\history{1795}

\condition{Cannot occur before \ref{pVII:First Partition Poland}. In that
  case, do not mark off and re-roll.}
\aparag Each of these events can happen only once.
\bparag If there no more \payspologne, apply \xnameref{pVII:NRP:Kosciusko}.
\bparag If \payspologne still exists, apply \xnameref{pVII:NRP:Commonwealth
  Revival}.


\subevent[pVII:NRP:Kosciusko]{Kosciusko's revolt}
\history{1795}

\phevnt
\aparag Place \REVOLT in the following provinces: \provinceLietuva,
\provinceMazowia, \provinceLublin and \provinceWielkopolska.
\bparag The \REVOLT are \faceplus if \ref{pVII:French Revolution} already
occurred at least once and \facemoins otherwise.
\bparag Military troops in these provinces must retreat.
\bparag Only the fortress of \villeVarsovie is taken by the rebels.
\bparag Put an \ARMY\facemoins of \payspologne with general
\leaderwithdata{Kosciuszko} (lasting until the end of the game) in a revolted
province.
\aparag The minor country \payspologne is created anew with these troops and
provinces.
% (Jym) Putting this in \phevnt so that the choice has to be without any
% private discussions.
\aparag \payspologne is looking for a foreign help. The following countries
must immediately accept or refuse, in order:
\bparag \FRA, if \ref{pVII:French Revolution} already occurred at least once.
\bparag \FRA or \SUE, whichever last got \payspologne as a special \EG due
either to \ref{pVI:WoPS:Polish Victory} or \ref{pVI:GNW:Stanislas Victory}
\bparag \FRA, \SUE, \AUS, \PRU.
\aparag The country who accepts to help \payspologne immediately declares war
(with a \CB) against all the countries owning a national province of \POL.
% (Jym) adding
\bparag This is just one declaration of war, not one per enemy country. Hence
the \STAB loss is only 1.
\aparag \payspologne is put in \EG of its helper.

\phdipl
% (Jym) switched to \phdipl to avoid any question about who gets the income of
% these provinces.
\aparag The \MAJ who accepted to help \payspologne must immediately give to
\payspologne all the national provinces of \POL it currently owns.
\bparag There is no loss of \VP for these provinces.

\phadm
\aparag \payspologne get reinforcements as a regular minor based on the income
of provinces it owns and control (as per normal rules).

\phmil
\aparag Troops of \payspologne stacked with \leaderKosciuszko are always
veterans.

\phpaix
\aparag \REVOLT may spread only in national provinces of \POL but may do so
even through frontiers of major countries.
\aparag \payspologne will not sign a white or unfavourable peace in this war.
\aparag If there are no more \REVOLT and no more troops of \payspologne, the
minor is destroyed again.
% (Jym) adding: the protector has no real risk since he will get back the
% ceded provinces
\bparag Ownership of provinces goes back to whoever owned them at the
beginning of the war.
% (Jym) adding: so that the protector has a risk
\bparag Other countries involved in the war may either sign a white peace or
continue fighting.
\aparag If \payspologne and its allies sign a favourable peace, all provinces
annexed at the peace must be national provinces of \POL and are given to
\payspologne.
% (Jym) Is it possible to take something else than provinces in this peace?
\bparag \payspologne becomes a permanent \EG of its protector as described in
\ref{pVI:WoPS:Polish Victory}.
\bparag \payspologne should now own: the four initially revolted provinces,
the provinces given by the protector and the provinces annexed at the peace.
\bparag This may happen also if the \REVOLT and troops were crushed but the
protector kept on fighting and won the war.


\subevent[pVII:NRP:Commonwealth Revival]{Commonwealth's Revival}
\history{not historic}

\phadm
\aparag \payspologne receives the general \leaderwithdata{Kosciuszko} for the
rest of the game.
\aparag Until the end of the game, each turn where there is a declaration of
war against \payspologne, roll for two \REVOLT in \POL.
\bparag The \REVOLT may happen in any country (not only \payspologne) and
their force is rolled at random.
\bparag The \REVOLT must occur in national territory of \POL. If they fall out
of it, re-roll another \REVOLT . However, both \REVOLT may occur in the same
province.
\bparag If \ref{pVII:French Revolution} occurred at least once in a previous
turn, roll four \REVOLT instead of two.

\phmil
\aparag Troops of \payspologne stacked with \leaderKosciuszko are always
veterans.
\aparag \REVOLT created by this event (and their fortresses or troops) are
allied with \payspologne.
% (Jym) adding
\bparag \REVOLT counters are limited supply sources for the troops of
\payspologne (only, not its allies).

\phpaix
\aparag The \REVOLT may only spread in national provinces of \POL but can do
so through national borders of major countries.
\aparag Revolted provinces count as if controlled by \payspologne for the
peace procedure.



\event{pVII:Mameluks Revolt}{VII-10}{Independence of the Mameluks in
  Egypt}{1}{RistoMod}

\history{1795 (Bonaparte in Egypt)}

\condition{}
\aparag If the current monarch of \TUR has an \ADM of at least 8, he can
choose to cancel the event.
\bparag In this case, place a \REVOLT (with random strength) in all the former
provinces of \paysmamelouks.

\phevnt
\aparag \paysmamelouks is recreated. It owns all the provinces it had at the
start of the game that now belong to \TUR.
\bparag Its basic forces are \ARMY\facemoins, \LD and it can use all its
counters.
\bparag \TUR loses \VP for the provinces lost.

\phdipl
\aparag \TUR has a temporary free \CB against \paysmamelouks for this turn
only.

\phpaix
\aparag If \TUR achieves an enforced unconditional victory over \paysmamelouks
during a war caused by this event, it can annex it again, gaining \VP for the
provinces annexed.

\effetlong
\aparag \FRA, \ENG, \HOL and \HIS have a permanent \CB against \paysmamelouks.
% (Jym) adding
\bparag If several of them use this \CB without being formally allied, they
can fight inside the territory of \paysmamelouks and \seazone{Levantin}
even if not at war elsewhere.

\aparag If, at the beginning of a peace phase, one of them controls the
capital and half the other provinces of \paysmamelouks, \paysmamelouks becomes
a permanent \VASSAL of the major occupying it and no diplomacy is possible on
it.
\bparag If the major later signs an unfavourable peace, one peace condition
can be to turn back \paysmamelouks into a regular normal country who then
becomes neutral.
\bparag It is also always possible to wage war against \paysmamelouks and
``steal'' the special \VASSAL status by occupying it.
% (Jym) adding
\aparag From now on, \FRA, \ENG and \HOL can declare war on \payschevaliers at
normal cost (instead of the one mentioned in \ruleref{chSpecific:Knights}) and
they can annex the capital province of \payschevaliers thus destroying the
country.



\event{pVII:Revolt Indonesia}{VII-11}{Revolt in Indonesia}{*}{Risto}

\history{No precise date}

\phevnt
\aparag Place one \REVOLT \facemoins and one \REVOLT \faceplus in two randomly
chosen \COL/\TP in areas \granderegionJava, \granderegionSumatra,
\granderegionBorneo and \granderegionCelebes. Both \REVOLT can occur in the
same place. Roll on \ref{table:alt-revolt-global} for the control of these
\REVOLT .



\event{pVII:Sale Corsica}{VII-12}{Sale of Corsica}{1}{Risto}

\history{1759}

\condition{If \provinceCorsica does not belong to either \payscorse or
  \paysgenes, treat this as a \REVOLT in \provinceCorsica (roll for strength
  as usual) and mark off.}

\phevnt
\aparag \provinceCorsica is for sale. Each player must immediately make a
secret bid for it and the highest bid annexes \provinceCorsica. Only the
winning bid is actually paid. If it bids at least 1\ducats, \FRA receives a
bonus of 50\ducats for its bid.

\phdipl
\aparag If \provinceCorsica is currently occupied by foreign troops, the owner
of those troops must either declare a war to the new controller of this
province profiting from a \CB, or withdraw its forces as per peace process.



\event{pVII:Pugatchev Revolt}{VII-13}{Revolt of Pugatchev}{1}{RistoMod}

\history{1773-1774}
\dure{Until the end of the civil war.}

\tour{The initial revolt}

\phevnt
\aparag A civil war erupts in \RUS. The rebels are controlled by \SPA, or by
\SUE if \SPA is allied to \RUS.
\aparag Place a \REVOLT \facemoins in the former provinces of the following
minor countries currently belonging to \RUS: \payskazan, \paysastrakhan,
\payssteppes, \payscrimee%
% (Jym) Before, \paysKazan and \paysAstrakhan were only one province. The
% event is a bit harder now, but remains largely manageable from experience.
and all \ROTW provinces adjacent to \RUS European territory that have \RUS
\COL/\TP in them. Roll for two additional \REVOLT in \RUS. If the result is
outside \RUS territory, ignore and do not re-roll.
\aparag Place a revolt \ARMY\facemoins and general \leaderPugachev in any
revolted province (he can either lead the \ARMY or a \REVOLT ).
\bparag The class of rebels armies is the same as \RUS.

\phdipl
\aparag Countries adjacent to \RUS can make a foreign intervention in any side
of the war.

\phadm
\aparag The rebels roll for reinforcements in offensive status during each
turn of the civil war.
\bparag The modifier for reinforcement is computed based on the income of the
provinces in \REVOLT , even if the rebel does not control the fortress.

\phmil
\aparag All rebel units can use \REVOLT counters as supply bases in the same
way as fortresses as long as there are no non-defeated enemy units present at
the moment supply is needed.

\phpaix
\aparag The war end either by suppressing all the \REVOLT or if the \REVOLT
cause the government to be overthrown.
\aparag There is no extension of \REVOLT if the rebels suffer a major defeat
or if there is no more \ARMY counter of the rebels.

\tour{Siberian revival}

\phadm
\aparag Starting from the third turn of the revolt, if a rebel army is located
during this phase in any former province of \payskazan, \paysastrakhan, or
\payssteppes, the rebels receive the \paysSiberie \ARMY\faceplus as extra
reinforcement this turn.
\bparag This extra reinforcement can only happen once in the war.
\bparag This army can freely stack with the rebels or exchange \LD in order to
replenish one or another.



\event{pVII:Potemkin}{VII-14}{Potemkin}{1}{Risto}

\history{1783-1791}
\dure{as long as \strongministre{Potemkine} remains the excellent minister}

\condition{\RUS can refuse this event if it so wishes. In that case mark off
  as played.

If \monarque{Pierre II} rules Russia, \RUS may choose to postpone the event
for one turn.}
\aparag \RUS can freely dismiss \ministrePotemkine at the end of any following
monarch survival phase and the event terminates.

\phevnt
\aparag \RUS receives an excellent minister \ministrePotemkine, with values
9/8/8.  He will last for a random length for Minister, see \ref{eco:Excellent
  Minister}.

\phadm
\aparag \RUS basic force is increased by \FLEET\facemoins during every turn
\RUS is engaged in a war and \ministrePotemkine is in charge.

\phmil
\aparag As long as this event is in effect \RUS receives an additional bonus
of \bonus{+1} to all attempts to suppress \REVOLT .



\event{pVII:War Crimea}{VII-15}{War in Crimea}{2}{PBnew}
\begin{todo}
  Add something about Orlov's revolt in the first occurrence of the
  event. Plus probably something to allow \RUS to go out of Black sea and help
  the Greek revolt.
\end{todo}
\history{1768-1774, 1787-1792}
% (Jym) one occurrence only? Present twice in the table.

\phevnt
\aparag \RUS has a Free \CB against \TUR at this turn or the next one.



\event{pVII:War Finland}{VII-16}{War in Finland}{1}{PBnew}

\history{1788-1790}

\phevnt
\aparag \SUE has a free \CB against \RUS if \RUS owns at least one province in
\regionFinlande.
\aparag \RUS has a free \CB against \SUE if \SUE owns at least one province in
\regionFinlande or on the \regionBaltique (between \provinceNeva and
\provinceCourlande included).



\event{pVII:Forward Balkans}{VII-17}{Forward to the Balkans}{1}{PBnew}

% (Jym) Present only once in the table.
\history{No precise date}

\phevnt
\aparag \AUT has a Free \CB against \TUR at this turn or the next one.



\event{pVII:Wars India}{VII-18}{Wars in India}{3}{PBnew}

\condition{}
\aparag If \ref{pVI:Last Great Mughals} did not happen yet, apply it instead.
\aparag Otherwise, apply \xnameref{pVI:Wars India} but with the following die
roll:
\bparag 1-4 = A) War between \paysmogol and \paysperse. Apply
\xnameref{pVI:India:Mughal Persian War}.
\bparag 5-6 = B) War between \paysafghans and \paysperse. Apply both
\xnameref{pVI:India:Afghan Empire} and \xnameref{pVI:India:Fall Persian
  Safavids}.
\bparag 7-10 = C) War between \paysafghans and \paysmogol. Apply both
\xnameref{pVI:India:Afghan Empire} and \xnameref{pVI:India:Rise Marathi}. This
case may not happen before either case A above, re-roll another case if
needed.



\event{pVII:Vassalisation Hanover}{VII-19 (1)}{Vassalisation of
  \payshanovre}{1}{Risto}

\phevnt
\aparag Same event as \ref{pVI:Vassalisation Hanover}.
\aparag If already occurred, apply \ref{pVII:William Pitt}.



\event{pVII:William Pitt}{VII-20}{William Pitt}{1}{Risto}

\history{1757-1761}
\dure{as long as \strongministre{Pitt} remains the excellent minister}

\condition{\ANG can refuse this event if it so wishes. In that case mark off
  as played.}
\aparag \ANG can freely dismiss \ministrePitt at the end of any following
monarch survival phase and the event terminates.

\phevnt
\aparag \ANG receives an excellent Minister \ministrePitt, with values 9/8/8.
He will last for a random length for Minister, see \ref{eco:Excellent
  Minister}.

\phdipl
\aparag \ANG may send \VASSAL troops in the \ROTW without paying the \STAB
indicated in~\ref{chSpecific:England:Minors at war}.

\phadm
\aparag \ANG basic forces are increased by \FLEET\facemoins and \ARMY\faceplus
during every turn where \ANG is engaged in a war (including oversea war) and
\ministrePitt is in charge.



\event{pVII:Kaunitz}{VII-21}{Kaunitz}{1}{Risto}

\history{1753-1793}
\dure{as long as \strongministre{Kaunitz} remains the excellent minister}

\condition{\AUS can refuse this event if it so wishes. In that case mark off
  as played.}
\aparag \AUS can freely dismiss \ministreKaunitz at the end of any following
monarch survival phase and the event terminates.

\phevnt
\aparag \AUS receives an excellent Minister \ministreKaunitz, with values
9/8/7.  He will last for a random length for Minister, see \ref{eco:Excellent
  Minister}.

% (Jym), Placeholder

\event{pVII:Comuneros}{VII-x}{Revolt of the Comuneros}{1}{JymNotEvenWritten}
\history{1779-1781}
\begin{todo}
  Revolt in New Granada. Probably useless (handle by revolt tables).
\end{todo}

\event{pVII:Xhosa}{VII-y}{Xhosa wars}{1}{JymNotReallyWritten}
\history{1779-1781/1789-1793/1799-1803}

\begin{todo}
  These may be the true intention of the ``Bantu raids'' of pVI. May
  replace~\ref{pVII:Revolt Indonesia} since it moved in \REVOLT tables.

  Same effect as~\ref{pVI:Bantu Raids}.
\end{todo}

\event{pVII:USA-Morocco}{VII-z}{Moroccan-American Treaty of
  Friendship}{1}{JymVetoPending}

\history{1777}

\condition{If \paysusa does not exists, do not mark off and reroll.}
\dure{Until the end of the game}

\effetlong
\aparag Place one level of \TradeFLEET of \paysusa in \stz{Lion}.
\bparag The reference level for \paysusa in \stz{Lion} is now 1.
% (Jym) http://en.wikipedia.org/wiki/Moroccan-American_Treaty_of_Friendship +
% Barbaresques piracy degenerated in war with the \paysusa in 1801 and 1815:
% http://en.wikipedia.org/wiki/Barbary_Wars

\stopevents

% Local Variables:
% fill-column: 78
% coding: utf-8-unix
% mode-require-final-newline: t
% mode: flyspell
% ispell-local-dictionary: "british"
% End:

% LocalWords: pVI pVII Batavian Mameluks Pugatchev Vassalisation Kaunitz de
% LocalWords: PBnew RistoMod Bolivarian PBMod malus Directoire Masse migr JCD
% LocalWords: artilleries Chouans Patrie subevent Duche Prusse Preussen NRP
% LocalWords: Kosciuszko Mediterranee Risto Safavids JymVetoPending reroll IW
% LocalWords: Jym SYW Orangists VOC Stadhouder Emigr Mughal Barbaresques http
% LocalWords: Potemkine Comuneros


\clearpage

% Local Variables:
% fill-column: 78
% coding: utf-8-unix
% mode-require-final-newline: t
% mode: flyspell
% ispell-local-dictionary: "british"
% End:

% LocalWords: se monarchduration malus monarchvalue reroll montferrat modene
% LocalWords: lucca milan hongrie moldavie valachie mazovie transylvanie pV
% LocalWords: papaute toscane parme venise corse arabie irak georgie damas
% LocalWords: mamelouks suisse wurtemberg savoie treves lorraine mayence Ile
% LocalWords: hollande hanovre hesse palatinat oldenburg iroquois inca saxe
% LocalWords: azteque baviere alsace thuringe habsbourg boheme brandebourg
% LocalWords: brunswick pologne Vlithuanie Vpommeranie japon teutoniques de
% LocalWords: hanse danemark Vnorvege Vfinlande Vliflandie courlande ecosse
% LocalWords: portugal Virlande Vbelgique Khanates ryazan pskov cosaquesdon
% LocalWords: kazan crimee ukraine gujarat vijayanagar mysore hyderabad aden
% LocalWords: maroc algerie tunisie tripoli cyrenaique perse ormus oman Tras
% LocalWords: soudan mogol cccccc ccccc os Montes Ost Pommern Illes Balears
% LocalWords: persia Dikoe Sancta ccc politicalevents
