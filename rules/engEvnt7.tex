% -*- mode: LaTeX; -*-
\chapter{Political Events of Period VII}
%\section{Period VII}
\label{events:pVII}



\subsection*{Event Table of Period VII}

\begin{eventstable}[Period VII events table]
  \tabcolsep=5pt\centering%
  \begin{tabular}{|l|*{5}{c}|l|}
    \hline
    1\up{st}\textarrow& 1-4 & 5-6 & 7 & 8 & 9 & 10 \\ \hline
    1 & 1  & 8  & 12 & 1  & R3  & \textbullet~1--2: \\
    2 & 2  & 9  & 19 & R18& 4	& +1 then \\
    3 & 3  & 10 & 2  & 10 & 5   & \nameref{events:pVI}\\
    4 & 4  & 11 & 18 & 11 & R12 & \textbullet~3--10: \\
    5 & 6  & 14 & 20 & 5  & R13 & \nameref{events:pVI}\\
    6 & 7  & 15& R5  & R6 & 4 	& \\
    7 & 13 & 16& 1   &  7 & 15  & \\
    8 & 19 & 21& 17  &  8 & 16  & \\
    9 & 1  & 4  & R8 &  9 & 7  	& \\ \hline
    10 & \multicolumn{6}{l|}{\nameref{events:pVI}} \\ \hline
  \end{tabular}
\end{eventstable}

\eventssummary{%
  pVII:Seven Years War|,%
  pVII:Bavarian Succession|,%
  pVII:Batavian Revolution|,%
  pVII:Independence War|E/E/E/E,%
  pVII:French Revolution|S{pVII:Revolution:Bastille}/%
  S{pVII:Revolution:Terror},%
  pVII:Bar Confederation|,%
  pVII:First Partition Poland|,%
  pVII:Second Partition Poland|E/E/E,%
  pVII:National Revival of Poland|S{pVII:NRP:Kosciusko}/%
  S{pVII:NRP:Commonwealth Revival},%
  pVII:Mameluks Revolt|,%
  pVII:Revolt Indonesia|,%
} \eventssummary{%
  pVII:Sale Corsica|,%
  pVII:Pugatchev Revolt|,%
  pVII:Potemkin|,%
  pVII:War Crimea|,%
  pVII:War Finland|,%
  pVII:Forward Balkans|,%
  pVII:Wars India|O{pVI:Wars India},%
  pVII:Vassalisation Hanover|O{pVI:Vassalisation Hanover},%
  pVII:William Pitt|,%
  pVII:Kaunitz|,%
  pVII:Comuneros|,%
  pVII:Xhosa|E/E,%
  pVII:USA-Morocco|,%
} \newpage\startevents



\event{pVII:Seven Years War}{VII-1}{The Seven Years War}{1}{PBnew}

\history{1756-1763}
\dure{until the end of the war caused by the event.}

\condition{}
\aparag Cannot happen before period VII if \PRU is not a major country and at
peace.
\bparag In this case, do not mark of an re-roll.

\phevnt
\aparag \PRU has a free \CB against \paysSaxe to be used at this turn or the
next one.
% (Jym) added:
\bparag Refusal to use this \CB cost \PRU 3 \STAB and \PRU is considered to
have lost the war for all the effects described below.
\bparag When \PRU uses this \CB, \paysSaxe propose an immediate white peace to
all its other enemies.
\bparag If \PRU does not use this \CB this turn, apply (in addition)
\xnameref{pVII:SYW:French Indian War}.
\aparag As a reaction to \PRU declaring war to \paysSaxe, \AUS has an
immediate free \CB against \PRU.
\aparag As a reaction to \PRU declaring war to \paysSaxe, \FRA has an
immediate normal \CB against \PRU.
\bparag If both \FRA and \AUS use these \CB, they are considered allied in the
war without need to sign a formal alliance.
% (Jym) Is this \CB in the right direction. I got the impression that the
% alliance reversal took place before the war really burst out. Then Frederic
% II took the initiative by attacking Saxony, but it's really \FRA attacking
% \ANG in Europe (taking Minorque with the naval victory of La Galissonniere
% over Byng then attack on Hanover by d'Estree (the "Forward" of Clash of
% Monarchs)).
\aparag As a reaction to \FRA declaring war to \PRU, \ENG has a free \CB
against \FRA.
\bparag If \ENG uses this \CB, \ENG and \PRU are considered allied in the war
without needing to sign a formal alliance.
\aparag \RUS has a \CB against \PRU and a \CB against \AUS (if at war) for the
duration of the war.
\bparag If \RUS uses one of these \CB, it is considered allied with the other
side in the war without need to sign a formal alliance.
\aparag Normal calls for allies may occur as a reaction to any of these
declarations of war.

\aparag As long as the event last, \RUS as a malus of \bonus{+3} to the
survival rolls of its monarch before \monarque{Pierre II}.

\phdipl
\aparag If \paysSaxe is at war against \PRU but its controller is not,
\paysSaxe is put in \EG of the first country at war against \PRU in the
following list: \AUS, \ENG, \SUE, \FRA, \RUS.
\bparag During the war, the controller of \paysSaxe has a bonus of \bonus{+5}
for diplomacy on any minor of the \HRE except \paysbaviere.

\phadm
\aparag Purchase limits for \PRU are doubled for the duration of the war.
% (JCD) Controlling the minor or the province ?
\bparag During the war, \PRU may raise troops in any province belonging to a
minor it controls.
\aparag At the first turn of the war, all minors at war must choose their
reinforcements in offensive attitude.
\bparag At the following turns, it must be either offensive of naval attitude.

\phpaix
\aparag As long as \monarque{Friedrich II} is alive, \PRU cannot be forced to
peace if at {\bf -3} in \STAB for two consecutive turns.
\bparag It can, however, be forced to peace if all its provinces are occupied.
\aparag If \PRU signs an unfavourable peace, \FRA and \AUS win 50 \VP (each)
if they were at war against \PRU.
\bparag In this case, if \ENG was at war it loses 25 \VP or 50 \VP is this was
an unconditional surrender.
\aparag If its side imposes an unconditional surrender (to either \PRU or
\AUS), \RUS can annex all provinces of \payspologne adjacent to \RUS
territory.
\bparag This counts as one peace condition for the alliance of \RUS.
\bparag The allies of \RUS will have a \CB against \RUS at the following turn
to contest this annexation.
\bparag If \payspologne is a special \EG of \FRA or \SUE per either
\ref{pVI:WoPS:Polish Victory} or \ref{pVI:GNW:Stanislas}, this annexation can
only be done if \RUS is not allied with the protector of \payspologne.
\bparag This annexation is impossible if Absolutism is established in
\payspologne.
\aparag If \PRU forces \paysSaxe to an unconditional surrender, it wins 25
\VP.
\aparag If \AUS is forced to unconditional surrender it loses 50 \VP and {\bf
  1} \STAB.


\subevent[pVII:SYW:French Indian War]{The French and Indian War}
\history{Colonial tensions erupted into a state of war in 1754 in America}

\phevnt
\aparag \FRA and \ENG are now in a state of overseas war.
\bparag This is not a declaration of war, hence there is no cost of \STAB for
any of them.
\aparag Reactions are allowed as if the war was continuing from a previous
turn except:
\bparag They may not generalise the war at this turn, unless using another
\CB.
\bparag They may not sign an armistice this turn.

% (Jym) Treaty of Paris I get the impression that getting the whole Canada
% area is a bit more than what is doable in a normal peace treaty in Europa. I
% propose a clause of grouped annexation similar to \HIS/\AUS in Italy: all
% \TP/\COL of a great area in the American enlarged view is equal to 1
% province. This should not be tied to SYW but let more hazy and for many more
% people (typically \ENG,\FRA,\HOL,\SUE)...  Overall, I am more keen about
% letting some fuzzy tricks than can lead to historical facts rather than
% linking to unique people and events to do these things.



\event{pVII:Bavarian Succession}{VII-2}{The War of Bavarian
  Succession}{1}{RistoMod}

\history{1778-1779}

\condition{}
\aparag Cannot occur if \paysbaviere is currently at war against \AUS. In that
case mark off and play \RD instead.

\phevnt
\aparag \paysbaviere offers to become a permanent \AM of \AUS.
\bparag If this offer is accepted, \paysbaviere cannot anymore fall below \AM
of \AUS, but diplomacy is still possible on it.
\bparag If the offer is refused, ignore the rest of the event.
\aparag If the offer is accepted, \PRU as free \CB against \AUS to be used
immediately.
\aparag If \PRU does not use the \CB, \AUS is considered to have won the war
for all relevant effects and \VP.
\bparag Normal calls for allies follow if a war is declared.
\bparag It is possible that \paysbaviere stays out of the war\ldots

\phpaix
\aparag If \AUS signs a white or unfavourable peace, \paysbaviere becomes a
normal minor again and \PRU win 20 \VP.
\aparag If \AUS signs a favourable peace, \AUS win 25 \VP and the previous
controller of \paysbaviere (if any) has a temporary \CB against \AUS at the
next turn.



\event{pVII:Batavian Revolution}{VII-3}{Batavian Revolution}{1}{RistoMod}

\history{1785-1787}
\dure{Until the \REVOLT are crushed or the government is overthrown.}

\condition{}
\aparag If \payshollande is a minor country, apply \ref{pVII:BR:Minor
  Holland}. The second time, apply \RD and mark off.
\aparag Else apply \ref{pVII:BR:Major Holland} (twice if needed).


\subevent[pVII:BR:Minor Holland]{Minor Holland in Revolution}

\phevnt
\aparag Place a \REVOLT \facemoins in each province owned by \HOLmin.
\aparag \HOLMin immediately proposes a peace based on the current peace
differential (or a white peace if the situation favours \HOLmin) to all its
enemies.
% (JCD) this event has to be rewritten for \HOL major. So this is meaningless,
% so removing.
% \bparag If this means separate peace, there is no extra penalty for \HOLmin
% in doing so.
\bparag Minor countries accept this peace.
% (JCD) same thing
% \aparag Then \HOL loses 3 \STAB and 50\ducats.

\phdipl
\aparag If \HOLmin and \PRU are allied, \PRU may intervene to help.
% (Jym) added
\bparag No other major may intervene.

\phadm
% (JCD) Changing this for a minor. Anyway, where would troops be risen?
% \aparag \HOL may not raise any new troop during the first turn of this
% event.
\aparag \HOLMin does not get any reinforcement roll for the first turn of this
event.

\phmil
\aparag \PRU may not send more than two stacks in provinces owned by \HOLmin.

\phpaix
\aparag \HOLMin keeps proposing peace to its enemies as long as \REVOLT still
exist during the peace phase.
\aparag If at the end of a turn, there are \REVOLT in more than half of
provinces belonging to \HOLmin, the minor has gone through a
revolution. Return the diplomatic marker of \HOLmin to neutral status (unless
activated in a war; in this case place it in \MA of the controller).
% (Jym) Deleting because \HOL major -> government overthrown normally.  2. If,
% at the end of a turn there are \REVOLT in more than half the provinces
% belonging to Holland, this minor is regarded to have gone through a
% revolution. Return the Dutch marker to neutral status (unless activated in
% which case place it in MA of the controller) and remove all \REVOLT markers.
% (JCD) Reinstating. This event HAS TO be rewritten for a \MAJ \HOL.

% (Jym) Deleting (JCD) Not reinstating this time, because diplomacy on minors
% during wars is really problematic.
% \aparag \PRU may do diplomacy on \HOLmin even if \HOLmin is at war somewhere
% else.


\subevent[pVII:BR:Major Holland]{War between Orangists and Patriots}

\phevnt
\aparag If at war, \HOL makes a mandatory white peace with all its enemies.
\bparag \MAJ allied to or at war with \HOL will be able to make a foreign
intervention in the Civil War (on any side).
\bparag Other countries will be able to intervene as mentioned below. No other
countries may intervene in the Civil War.
\aparag \HOL is in Civil War (see \ref{chDiplo:Religious Civil War}) between
the Patriots, the Republicans and the Orangists.
\bparag The Orangists use up to one \ARMY, 5\LDND and one \FLEET counter (for
example, from \paysroyalistes).
\bparag The Patriots use up to two \ARMY and 6\LD from \paysrebelles.
\bparag Republicans use up to one \ARMY, 5 \LDND and all possible naval forces
of \paysHollande.
\bparag \HOL can choose to take either the side of Patriots or Orangists. The
choice is made after the revolts have been rolled for.
\aparag[Rise of the Patriots] For each province of \HOL in Europe, roll
2d10. Add \bonus{-3} if \ref{pVII:French Revolution} already started. If the
roll is lower or equal to the income of the province, the province is in
\REVOLT. Roll for the strength of the revolt in \ref{table:alt-revolt-global}.
\bparag The Patriots control \provinceHolland.
\aparag For all \COL, roll 1d10. On 1, put a \REVOLT\faceplus and a Patriot
\LD; on 10, put an Orangists \LD and a control for the Orangists in the
province. All other \ROTW counters of \HOL are owned by the Republicans.
\bparag \HOL has to announce its support of one side at this point. It will
play this side.
\aparag[Orangists resistance] The Orangists call for help in that order: \PRU,
\SUE, non-revolutionary \FRA, the owner of the \emph{Spanish Low Countries}
after \ref{pV:WoSS}.
\bparag The first to answer the call will play the Orangists (if \HOL supports
the Patriots) and will be allowed an intervention of at most two stacks (not
one as per usual rules) of at most one \ARMY\faceplus.
\bparag Other countries will not be able to intervene.
\bparag If no country wishes to intervene in this list and \HOL chose the
Patriots, \PRU will play the Orangists.
\bparag The Orangists decide of one safe place (historically
\provinceGelderland) that they own (even if in \REVOLT). This must not be
\provinceHolland nor \provinceUtrecht. The \REVOLT is removed if there was
one, but a \LD or a \LeaderG is moved in another \REVOLT.
\aparag[Republicans and the VOC] Naval forces and most \ROTW Dutch settlements
will mostly stay out of the war. The moves of these forces will be played by
\ENG.
\bparag \ENG will be able to intervene with a normal foreign intervention.
\bparag Administrative actions will be very limited and played by \HOL, using
3/3/3 as Monarch values during all the war.
\aparag[Call for the Revolution] Revolutionary \FRA (after
\nameref{pVII:French Revolution} started) may be able to send one stack of
conventional troops and two stacks of Revolutionary troops to help
Patriots. It can declare its intervention during the military rounds if
sending Revolutionary troops; but it may not gain as much by doing so.
\bparag If \HOL supports the Orangists, and nobody supports the Patriots, then
\TUR will play the Patriots.
\aparag[The Dutch Fleet] Orangists pick one \FLEET counter, moved to one port
they control (if none, one port of their supporter or simply at sea). All
other naval forces go to Republicans.

\phdipl
\aparag \HOL can react to attacks on its minor countries. It can not do any
other diplomatic actions.

\phadm
\aparag[Incomes] Orangists and Republicans get land income from the provinces
they (or their allies) control in European provinces of \HOL.
\bparag Patriots get the land income from the provinces they control or that
are in \REVOLT.
\bparag Half (rounded down) of Vassal income goes to Orangists. The rest goes
to Republicans.
\bparag There is no commercial income. \ROTW income goes to whoever controls
the place (usually Republicans).
\bparag \MNU give their basic income (the fixed part) to the side getting
revenues from the province.
\aparag[Administrative actions] The only actions that can be done are paid on
\HOL \RT directly. They are: reactions to concurrence, improving already
existing \COL or \TP, improving already existing \TradeFLEET.
\aparag[Raising armies] Republicans have to pay for the maintenance of naval
and land forces in their keep first of all (what can not be paid for is
dismantled). With the rest, they may purchase troops only to be bought in
territories they control or in \ANG. \ANG may give money to
Republicans. Republican land forces are always \Conscripts.
\bparag On the first turn, land forces in Europe of \HOL are disbanded.
\bparag Patriots and Orangists have their own budget and a purchase limit of
2\LD. The first \ARMY\facemoins they buy on the first turn is \Veteran, the
rest is \Conscripts. Their supporter may give money.

\phmil
\aparag Province flooding (\ref{chSpecific:Holland:Flooding}) can not be used
during this event.
\aparag For movement, supply and attrition, provinces with \REVOLT are
friendly to Patriots unless an enemy force is within.
\bparag Patriots consider all cities with \REVOLT in the province as
blockaded.
\bparag \REVOLT are weak supply points for Patriots.

\phpaix
\aparag \HOL loses {\bf 1} \STAB. No \STAB increase is possible during the
event.
\aparag No armistice may be signed by the various sides.
\aparag[Victory of Orangists] It there are no more revolts and no more troops
of Patriots in national territory, Orangists get an automatic victory.
\bparag The Monarch is reinstated. \STAB of \HOL becomes {\bf +3} minus one
per turn of revolution.
\bparag If this event happens again, \HOL will have a \bonus{-2} to the
strength of \REVOLT.
\bparag The supporters of Orangists get 20 \VP (possibly including \HOL).
Supporters of Patriots lose 20 \VP.
\aparag[Victory of Patriots] If there are \REVOLT in all national provinces or
no more Orangists (or allies) troops in national territory or it is the third
turn of the revolution and there is still at least one \REVOLT or this is the
last turn of the game or \STAB is at {\bf -3} for two consecutive turns,
Patriots get an automatic victory.
\bparag All revolts are removed. The government is overthrown. Read below for
the lasting consequences.
\bparag A Monarch will be rolled anew at next turn, as if there were a
\terme{Dynastic Crisis}.
\bparag \STAB of \HOL becomes {\bf 0}.
\bparag The supporters of Patriots get 20 \VP (possibly including
\HOL). Supporters of Orangists lose 20 \VP.
\aparag[Victory of Republicans] Republicans are considered victors if any
other side wins in one turn or two turns. They lose if the revolution ends
after three turns.
\bparag \ENG is entitled to 1 or 2 compensations (given by the Orangists or
the Patriots or taken to the VOC during the troubles) of \HOL's choice: 1
level in a \CTZ, 2 levels in a \STZ, one \COL, one \TP. Automatic concurrence
may follow from this. There are two compensations if the victory was in one
turn. \ENG gets two compensations for a victory of either side in 1 turn.
\bparag If \ENG lost military forces (either naval or land) during the
Revolution, it is entitled to 20 \VP in addition. If the Republicans lost,
\ENG loses 20 \VP.
\aparag There are no other peace outcomes.
\aparag In case of victory, supporters (including \HOL) of the winning side
gain 20\VP and the forces of the winning side are converted to \HOL
counters. Supporters of the losing side lose 20\VP and the forces of the
losing side are disbanded.

\effetlong
\aparag In case the Patriots win, apply the following points:
\bparag The \terme{Stadhouder} government is no more possible.
\bparag All monarchs have a \bonus{+2} to their survival roll. \terme{Dynastic
  Crisis} will cost 1 \STAB with no other consequences.
\bparag The maximum ADM value of the Monarch (or Minister) is now 7. However,
the real rolled-for value is used for rolling the next Monarch.
\bparag The maximum DIP value of the Monarch is now 5. However, the real
rolled-for value is used for rolling the next Monarch.
\bparag The minimum MIL value of the Monarch is now 7.
\bparag If the event happens again, the \REVOLT strength will have a
\bonus{+2} modifier.
\bparag The VOC is dissolved. The basic \LeaderGov is available each turn only
if 1d10 (rolled during the Monarch Survival phase) is even. This also removes
some constraints on \TFI and turns the \TPaction available each turn into a
\TPaction or \COLaction, at the choice of \HOL.
\bparag National provinces of \HOL will count in favour of \FRA for
the``natural frontier'' objectives (not for the rest).
\bparag \HOL loses 1 diplomatic action.
\bparag \HOL has a mandatory defensive alliance with Revolutionary \FRA for at
least three turns (as soon as possible)



\event{pVII:Independence War}{VII-4}{War of Independence in the
  Colonies}{*}{RistoMod}

\date{1775-1783}
\dure{Until the end of the rebellion.}

\condition{}
\aparag If none of the following already occurred, do not mark off and
re-roll:
\bparag \ref{pVII:SYW:French Indian War} (only if the war is already
finished).
\bparag \ref{pVII:William Pitt}.
\bparag \ref{pVII:French Revolution}
% (Jym) I split in three separate events, it is easier this way. Just getting
% the initial group is sufficiently difficult to deserve its own
% subevent. This makes the rest easier to read and write.
\aparag The first time, apply \xnameref{pVII:IW:First Revolt}, the second and
subsequent times, apply \xnameref{pVII:IW:Further Revolts}. Each time,
\xnameref{pVII:IW:Where} is used to determine which colonies try to get their
independence.


\digression[pVII:IW:Where]{Where does the revolt occurs ?}
A revolutionary war erupts in a group of colonies. The target group is chosen
by first selecting a subcontinent and then a major country. The major country
must have a certain number of colonies in the target subcontinent in order to
start the revolt. The first major country meeting the criteria is subject to
the revolution.
% (Jym) Not possible in the Caribbeans? Brazil?  (JCD) Indonesia ?
\aparag The possible target subcontinents are, in order:
\bparag \continent{North America}
\bparag \continent{South America}
\bparag \continentBrazil
\bparag \continentIndia
\bparag \continentAsia (except \continentIndia)
% (Jym) Not possible for \POR, even minor (Brazil).
\aparag \label{pVII:IW:Protestant} The possible target players are the
protestant ones in the following list:
\bparag \ENG, \FRA, \HIS, \HOL
\aparag The target group of colonies is elected by first looking for players
meeting the criteria in the first subcontinent, then the second and so on.
% (Jym) If I remember correctly, what follows was changed (condition of
% revolts). I copy Risto anyway.
\aparag The target group of colonies must contain at least 10 levels of \COL
in four adjacent provinces (with land access between them).
% (Jym) This makes a revolt in the Caribbeans or in the DEI impossible.
\bparag It is possible that some of these provinces have no \COL in them as
long as there are 10 levels of \COL or more in four provinces.
% (Jym) No malus for \ANG if it is not eligible ?  (JCD/Jym) If the target
% exists, the controller of the revolt should choose three large areas that
% uprise in the continent (maybe not connex, this is a tactical consideration
% best handled by humans)
\aparag If no target exists, nothing happens but the event is nonetheless
considered played (mark off, do not re-roll, do not play \RD).
\aparag Once the target group of colonies is found, roll 1d10 with the
following modifiers:
% (Jym) Revolt if result is large.

% (Jym) Enemy present -> foreign agitation?  Wiki "French and Indian wars"
% (plural) :

% The overwhelming victory of the British played a role in eventual loss of
% their thirteen American colonies. Without the threat of French invasion, the
% American colonies saw little need for British military protection. In
% addition, the people resented British efforts to limit their colonization of
% the new French territories to the west of the Appalachian Mountains, as
% stated in the Proclamation of 1763, in an effort to relieve encroachment on
% Native American territory. These pressures contributed to the American
% Revolutionary War.

\begin{modlist}
\item[\bonus{-5}] If no other player has a \COL inside the four target
  provinces.
\item[\bonus{+1}] For each other player that has \COL or \TP within two
  provinces of the group or
\item[\bonus{+2}] For each other player that has \COL or \TP adjacent to the
  group.
  % (Jym) Metropolitan troops -> quick squashing of the revolt (?).
\item[\bonus{-1}] If the player has any \LD in the group or
\item[\bonus{-2}] If the player has any \ARMY in the group.
  % (Jym) Indian allies -> squashed revolt?
  % Or more probably peace with Indian => no need to be protected by the
  % metropolis.
\item[\bonus{-2}] If the player has \dipFR or \dipAT with a minor adjacent to
  the group.
\item[\bonus{+3}] If another player has \dipFR or \dipAT with a minor adjacent
  to the group and the player has neither \dipFR nor \dipAT with this minor
  country.
\end{modlist}
\aparag \label{pVII:IW:Test} If the result is 5 or more, the rebellion
occurs. A non-modified 10 is an automatic rebellion while an non-modified 1
always means that no rebellion occurs.
\bparag If no rebellion occurs, nothing happens but the event is nonetheless
considered played (mark off, do not re-roll, do not play \RD).

% (Jym) \wikipedia separates "American Revolution" -> political and social.
% "American Revolutionary War" -> military.


\subevent[pVII:IW:First Revolt]{American Revolutionary War}

\condition{Choose a target \MAJ and group of colonies as indicated in
  \ref{pVII:IW:Where}.}

\phevnt
\aparag The \MAJ choose one \COL within the revolted group. Place a \REVOLT
\facemoins in each other \COL of the group.
\bparag Place 3\LD (of \paysusa) on one of the \REVOLT .
\bparag Rebels control all the fortresses in the revolted colonies.

\phdipl
% (Jym) Removing, from memory this is replaced by the special power of \ANG
% The target may use a bonus of +5 to certain minors that can exceptionally be
% used in \ROTW map against the rebels (only). The bonus can only be used to
% raise these minors to CE and not higher. These minors can only be used when
% inactive and only by following the restrictions of CE (even if they are
% actually higher up in the diplomatic track). The minors in question are the
% following: \ANG: \payshanovre and \payshesse b. \FRA: \payssavoie and
% \payslorraine c. \SPA: Portugal d. \HOL: \payspalatinat (Jym) adding this;
% Rationale: the minor is just created it is on nobody track, so only a
% limited intervention of the \MAJ ally is possible, this matches the French
% intervention (JCD) I think \paysusa should rather call neighbouring
% countries not only \FRA/\ENG.
\aparag The rebels calls for allies as indicated in the preferences of
\paysusa.

\phadm
\aparag The \MAJ does not get income from the \COL that initially revolted,
even if the \REVOLT are suppressed.
\bparag It cannot either raise troops there or use the colonial militia.
\bparag It can, however, build fortresses in these \COL.
\aparag The \MAJ receives no income from \TradeFLEET in \STZ adjacent to a
\COL that initially revolt, even if the \REVOLT are suppressed.
\bparag All other player get double income (but not double bonus) from
\TradeFLEET in these \STZ.
% (Jym) Maybe allow naval if it takes place in Caribbeans or Indonesia?
\aparag Rebels can choose reinforcements in either offensive or defensive
attitude. They use the counters of \paysusa.
\aparag If the \MAJ has a general that can be used by \paysusa (either
\leaderWashington or \leader{La Fayette}), this general goes to the side of
the rebels.
\bparag If \leaderArnold is alive, he joins the rebels.
\bparag The rebels must have at least two generals for the duration of the
event. Use the unnamed generals of \paysusa if needed.
\aparag The \MAJ receives at no cost a mercenary that can be used in the \ROTW
and is considered to have rank Z.
\aparag \leaderWashington and \leader{La Fayette}, if not already rebels, can
be sent by their owner (\ENG or \FRA) to help them.
\bparag The owner chooses each turn whether it keeps the general or send him
to help the rebels.
\bparag This general is in addition to the minimum two generals of the rebels.
\bparag Once the event is finished, this leader goes back to his major
country.

\phmil
\aparag \REVOLT are supply sources for the rebel troops.
\aparag Remember that \paysusa (hence, the rebels) roll for reinforcements
after each winter round and not only once per turn.

\phpaix
\aparag The event stops at the end of the second turn of revolt.
\bparag If all \REVOLT have been suppressed by the end of the second turn,
\MAJ wins the war.
\bparag Otherwise, the rebels win.
\aparag If the rebels are crushed, remove all the units of the rebels, remove
the named leaders of \paysusa from the game (not the one sent by a major).
\aparag If the rebels win, the minor country \paysusa is created.
\bparag All the \COL in the initial group of revolt are part of \paysusa, even
those where the \REVOLT were suppressed.
\bparag All the provinces of \paysusa are considered as European provinces for
all game purposes.


\subevent[pVII:IW:Further Revolts]{Bolivarian Revolutions}
\history{Spanish American Wars of Independence (Bolivar):
  1808-1829/Independence of Brazil: 1823-1825.}

\condition{}
\aparag If another \xnameref{pVII:Independence War} is currently occurring, do
not mark off and re-roll.
\aparag If another \xnameref{pVII:Independence War} is already finished and
was won by the rebels, if \xnameref{pVII:Revolution:Bastille} did not occur
yet, apply it instead.
\aparag Otherwise (revolt crushed or \nameref{pVII:Revolution:Bastille}
already occurred or a previous occurrence resulted in ``no revolt'' after the
test of \ref{pVII:IW:Test}), choose a target country as indicated in
\ref{pVII:IW:Where}, ignoring the religion condition of
\ref{pVII:IW:Protestant}.
\aparag Once a target is found, if the die roll of \ref{pVII:IW:Test}
indicated a revolt, roll another die and apply the corresponding result:
\begin{modlist}[1.5em]
\item[10] Another revolt occurs
\item[9] Extension to a near continent
\item[6--8] Small revolt
\item[1--5] Nothing happens. The event is nonetheless considered played (mark
  off, do not re-roll, do not play \RD).
\end{modlist}
\aparag[Another revolt occurs] Another revolt occurs as described in
\ref{pVII:IW:First Revolt}. Both revolts are separate one from another and, if
created, both countries are different. Use whatever name and counters you wish
to refer to the second and subsequent ones (Canada, Bolivia, Brazil,
Indonesia, \ldots)
\aparag[Extensions to a near continent] If the target subcontinent is adjacent
to the original one (either \continent{North America} and \continent{South
  America} %
% (Jym) \continentCaraibes, \continentBresil
or \continentIndia and \continentAsia), a new revolt occurs as above,
otherwise treat as a \emph{Small revolt} below.
\aparag[Small revolt] Place three \REVOLT \facemoins in the target group of
colonies. Don't use any minor forces. No independence may result from these
\REVOLT . Another \ref{pVII:Independence War} may occur before all the \REVOLT
are crushed.



\event{pVII:French Revolution}{VII-5}{The French Revolution}{2}{PBMod}

\history{1789-1799}
\begin{histoire}
  The first event corresponds to the bankruptcy of the French monarchy as well
  as the peasant crisis leading to the Storming of the Bastille and a change
  of government. Several possible new forms of government can exists depending
  on the choices of the player and the other majors. The second event
  corresponds to the internal dynamics of the Revolution yielding to
  uncontrolled effects.
\end{histoire}
\dure{until the end of the game.}

\condition{}
\aparag If none of the following happened, do not mark off and re-roll:
\bparag End of \xnameref{pVII:Seven Years War}.
\bparag Beginning of \xnameref{pVII:Independence War} (the revolt must have
started).
\bparag \xnameref{pVII:Batavian Revolution} is finished and was successful.
\aparag The first time, apply \xnameref{pVII:Revolution:Bastille}. The second
time, apply \xnameref{pVII:Revolution:Terror}.
\begin{designnote}
  \textit{``\`{A} partir de la R\'{e}volution, les r\`{e}gles de bon sens
    cessent de s'appliquer.''}\\
  ~\hfill (Pierre, August 2007).
\end{designnote}


\subevent[pVII:Revolution:Bastille]{Storming the Bastille}

\phevnt
\aparag \textbf{Political and social crisis}
\bparag If \FRA is at war against another \MAJ, it loses 1 \STAB. Otherwise,
it loses 3 \STAB.
% (Jym) What about minors on French track? I suppose they should be kept for
% period/game \VP, but that they do, in fact, go away (or at least that \FRA
% cannot call for allies). \FRA should not be able to do any diplomatic
% actions at all, too. In doubt, I will leave that for the \MAJ (JCD) \HOLmin
% after Batavian revolution should be able to stay
\bparag \FRA is considered to have broken its alliances with all countries
(major or minor). This does not cause any extra loss of \STAB.
% (Jym) It is not too difficult to count immediately the \VP of diplomacy then
% switch the minors to Neutral. It is more complicated to count the \VP of
% vassal provinces for the end game.  (JCD) This is not reasonable to count on
% a vassal for French territory, that's what it means.
\bparag Roll for two \REVOLT in \paysmajeurFrance.
\bparag Future survival rolls for the French monarch get a malus of
\bonus{+2}. The malus will be \bonus{+5} if \FRA goes to the
\monarqueConvention government.
\bparag The following countries have a free \CB against \FRA until the end of
the game: \ENG, \AUS, \PRU, \HIS, \HOL (unless if \ref{pVII:Batavian
  Revolution} was won by the rebels).
\bparag \FRA has a normal \CB until the end of the game against each major
country and against each minor country adjacent to its territory.
\bparag These \CB can be used as diplomatic reaction to any other diplomatic
announcement.
\aparag \textbf{Economical crisis}
\bparag \FRA loses 100\ducats. Then its Royal Treasure is halved %
% (Jym) RT<0 possible
with a minimum loss of 50\ducats.
\bparag From now on, \FRA loses 10\% of its gross income
(\lignebudget{Events}).
\bparag From now on, \FRA pays inflation as if it were bringing gold from
\continentAmerica.

\phdipl
\aparag If \payspologne is a special \EG of \FRA (per \ref{pVI:WoPS:Polish
  Victory}), as soon as another \MAJ declares war on \FRA, so does
\payspologne. Troops of \payspologne are allowed to cross the \HRE.
\aparag At the end of each diplomatic phase, test for a change of government
in \paysmajeurFrance. Roll 1d10 modified as follows:
\begin{modlist}
\item[\bonus{-4}] if \xnameref{pVII:Independence War} never occurred;
  % (Jym) The second event IW triggers Bastille. So there is at maximum one IW
  % before Bastille, no doubt on the winner of IW.
\item[\bonus{-2}] if \nameref{pVII:Independence War} is finished and the
  rebellion was crushed;
\item[\bonus{+2}] if \nameref{pVII:Independence War} is finished and \paysusa
  has been created;
\item[\bonus{+2}] if \FRA used this turn a \CB provided by this event;
\item[\bonus{+4}] if \FRA is at war without declaring any war this turn;
  % (Jym) Next bonus just for the destitution turn? Not very good. Once a King
  % died, the end is not far.
\item[\bonus{+6}] if the king of \FRA died during this event.
\end{modlist}
\aparag The result of the die roll tells which is the new government of \FRA:
\begin{modlist}[2em]
\item[1--6] The government is unchanged.
\item[7--13] The government switches to (or remains)
  \monarqueConvention. Apply \xnameref{pVII:Revolution:Convention}.
\item[14+] The government switches to \monarqueTerror. It won't be able to
  change back to anything else: stop doing this test each turn. Apply
  \xnameref{pVII:Revolution:TerrorGov}.
\end{modlist}

\phmil
\aparag During all wars caused by this event, enemies of \FRA are considered
allied inside the territory of \FRA or when fighting French troops. They may
be at war elsewhere and nonetheless be allied (and stack together or intercept
French troops attacking the other country,\ldots) fighting \FRA.
% (Jym) from memory
\aparag Countries at war against \FRA are limited to 1 stack inside the
national territory of \FRA.
\bparag They are not limited if fighting out of the national territory of
\FRA.
\bparag The \ARMY provided by \xnameref{pVII:Revolution:Nobles} does not count
toward this limit. It is always allowed inside \FRA.

\phpaix
\aparag If \villeParis is controlled by the enemies of \FRA and there are no
more \ARMY of \FRA in play, the Revolution is crushed and a new king is put on
the throne of \FRA.
\bparag The game ends at the end of this turn.
\bparag Each country at war against \FRA wins 30 \VPs.
\bparag \FRA wins 15 \VPs at the end of the game if the revolution has not
been crushed.

\begin{digressions}[Effects of the Revolution]


  \subevent[pVII:Revolution:Convention]{Convention (and constitutional
    monarchy)}
  \history{1789-1792}
  % (Jym) Immediate effects in \phdipl because the change test is done
  % in \phdipl...

  \phdipl
  \aparag When the government changes to \monarqueConvention:
  \bparag Apply \xnameref{pVII:Revolution:Nobles},
  \xnameref{pVII:Revolution:Chouans}, \xnameref{pVII:Revolution:Armies} and
  \xnameref{pVII:Revolution:Natural Frontiers}.
  \bparag Roll for one \REVOLT in \FRA.

  \effetlong
  \aparag If still alive, the king of \FRA has a \bonus{+5} malus to all his
  survival rolls (instead of the \bonus{+2} for the Revolution).
  \aparag If the king dies, he is replaced by \monarqueConvention with values
  3/6/7. This government never rolls for survival.
  \aparag During each event phase of \monarqueConvention, roll for one \REVOLT
  in \FRA.


  \subevent[pVII:Revolution:TerrorGov]{Reign of Terror and Directoire}
  \history{1792-1799}

  \phdipl
  \aparag When the government switch to \monarqueTerror:
  \bparag The French king (or \monarqueConvention) is immediately killed, he
  is replaced by \monarqueTerror with values 5/6/9. This government never
  rolls for survival.
  \bparag Roll for 3 \REVOLT in \FRA. % Revolts of the Federated
  \bparag If they were not already activated, apply
  \xnameref{pVII:Revolution:Nobles}, \xnameref{pVII:Revolution:Chouans} and
  \xnameref{pVII:Revolution:Natural Frontiers}.
  \bparag Apply \xnameref{pVII:Revolution:Levee Masse}
  \bparag Increase the \DTI and \FTI of \FRA by 1 each (max. 5).
  \bparag Each \MAJ has a free \CB against \FRA to be used immediately.

  \phadm
  \aparag At the turn the government switch to \monarqueTerror, the gross
  income of \FRA is halved (round down, \lignebudget{Events}). This is not
  cumulative with the permanent \bonus{-10\%} caused by the event.
  % \bparag This replaces the loss caused by \STAB.

  \effetlong
  \aparag During each event phase of \monarqueTerror, roll for two \REVOLT in
  \FRA.

  \phpaix
  \aparag[End of Modern History] The game ends at the end of the second turn
  of \monarqueTerror.

  % (Jym) \wikipedia English uses "\'Emigr\'e" (in VF).


  \digression[pVII:Revolution:Nobles]{\'Emigr\'es}

  \phadm
  \aparag The first country at war against \FRA in the following list gets the
  benefits of the \'Emigr\'es: \AUS, \PRU, \HIS, \ENG, \payspologne (and its
  controller).
  \aparag The \MAJ gets a French Royal \ARMY\facemoins with a \LeaderG\anonyme
  of \FRA.
  % (Jym) adding
  \bparag This \ARMY can appear in any province owned by \FRA or by the \MAJ
  receiving it.
  \bparag It is considered class \CAIII\ with 4 artilleries per
  \ARMY\faceplus.
  % (Jym) In pVII, \FRA has 6 art/A, class III has 5. 4 art/A is either \RUS
  % or \TUR.
  \aparag This \ARMY can be reinforced (or recreated if destroyed) at the cost
  of the French royal troops.
  % (Jym) adding
  \bparag This \ARMY can be raised again or receive reinforcements in any
  province owned by the \MAJ receiving it or any French province either in
  \REBELLION or \REVOLT or controlled by another country.
  % (Jym) adding
  \aparag This \ARMY is freely maintained in veteran (new troops are
  conscripts as per normal rules).
  % (Jym) adding
  \aparag This \ARMY must fight against \FRA. If in \FRA it cannot leave the
  provinces in or adjacent to \FRA national territory and if created out of
  \FRA it must goes to \FRA by the shortest path. It is considered allied with
  all countries except \FRA. It can co-exist with troops all countries but
  \FRA and will never take part in any battle except against \FRA.


  \digression[pVII:Revolution:Chouans]{Chouans and Royalist Uprisings}

  \phdipl
  \aparag \terme{Chouans} are played by \ENG (even if not at war against
  \FRA).
  \aparag Place a \REBELLION \facemoins in each \provincePoitou and
  \provinceVendee.
  \bparag French troops in these provinces must retreat.
  \aparag Place a rebel \ARMY\faceplus and a general in one of these
  provinces.

  \phadm
  \aparag As long as a \REBELLION exists in either \provincePoitou,
  \provinceVendee, \provinceMorbihan, \provinceArmor or \provinceFinistere,
  the \terme{Chouans} get 1\LD in reinforcement (except the first turn).

  \phmil
  \aparag Instead of moving, 1\LD may ``hide'' in \provinceVendee (only). It
  does not count as military presence any more but gives a malus of \bonus{-2}
  to suppress the \REBELLION .
  \bparag If the \REBELLION is suppressed, this \LD is destroyed.
  \aparag These \REBELLION are friendly to any enemy of
  \FRA. \REBELLION\Faceplus are also supply sources for any enemy of \FRA.


  \digression[pVII:Revolution:Natural Frontiers]{Natural Frontiers}

  \condition{}
  % (Jym) I redefined the natural frontiers along the Rhine
  \aparag The ``Natural Frontiers'' of \FRA consist in:
  \bparag All national provinces of \FRA.
  \bparag All provinces adjacent to national provinces of \FRA except those in
  \HIS or \paysSuisse.
  \bparag All provinces on the left-hand side of river Rhine, that is all the
  provinces between \FRA and (included) \provinceAlsace, \provincePfalz,
  \provinceTrier, \provinceKoln, \provinceLimburg, \provinceUtrecht and
  \provinceZeeland.

  \phpaix
  \aparag \FRA automatically annexes any province within its Natural Frontier
  that it militarily controls during peace phase, unless they belong to
  Patriotic \HOL (see~\ref{pVII:Batavian Revolution}).


  \digression[pVII:Revolution:Armies]{Revolutionary Armies}

  \phadm
  \aparag \FRA can now use the Revolutionary \ARMY counters.
  \bparag Each new \ARMY raised from now on is Revolutionary.
  \bparag Already existent (royal) \ARMY are not affected and stay until
  destroyed or disbanded.
  \bparag \FRA may not have more than 6 \ARMY counters in play at the same
  time.
  \aparag Recruitment and upkeep cost of Revolutionary \ARMY is halved (upkeep
  of royal \ARMY is unchanged).
  \aparag Land recruitment limit is doubled.
  \aparag Naval recruitment cost is doubled.
  \aparag \FRA may not used Licensed privateers as described in
  \ruleref{chSpecific:France:Privateers}.


  \digression[pVII:Revolution:Levee Masse]{``La Patrie en danger''}
  All the effects of \xnameref{pVII:Revolution:Armies} are applied. In
  addition:

  \phdipl
  \aparag All French \ARMY are immediately replaced by Revolutionary \ARMY.
  \aparag \FRA may have up to 8 \ARMY counters in play.
  \aparag Remove all named leaders of \FRA. \FRA gets 5 unnamed generals and 1
  unnamed admiral (who can go in the \ROTW) (these are the new limits for
  \FRA).
  \aparag General \leaderwithdata{Bonaparte} is available for \FRA during the
  first turn of \monarqueTerror, starting with the first round after
  W2. % (Jym) : S3 or later
\end{digressions}


\subevent[pVII:Revolution:Terror]{Reign of Terror (Robespierre)}
\history{1792}
\dure{until the end of the game.}

\condition{}
% (Jym) A second IW may be running after the first one was crushed and allows
% Robespierre to come. If there were several IW, one victory is enough.
\aparag Can happen only if \xnameref{pVII:Independence War} is ongoing or if
\paysusa has already been created.

\phevnt
\aparag \FRA loses 1 \STAB.
\aparag \FRA goes to \monarqueTerror. Apply
\xnameref{pVII:Revolution:TerrorGov}.

\phmil
\aparag The military phase starts in W0.

\phpaix
\aparag The game ends at the end of this turn.



\event{pVII:Bar Confederation}{VII-6}{The Confederation of the Bar}{1}{PBnew}

\history{1768}

\condition{}
\aparag Cannot occur if there is no more \payspologne. in that case, mark off
and play \RD.
\aparag Cannot happen before the start of the war caused by \ref{pVI:WoPS}. In
that case do not mark off and re-roll.
\aparag Cannot happen if \ref{pVII:Second Partition Poland} already occurred
and the partition was accepted (with or without war) at least once. In that
case, mark off and play \RD.
\bparag Can, however, occur if \ref{pVII:First Partition Poland} occurred and
the partition was accepted.

\phevnt
\aparag Absolutism is established in \payspologne.



\event{pVII:First Partition Poland}{VII-7}{First Partition of
  Poland}{1}{PBnew}

\history{1772}

\condition{}
\aparag If \POL is still a major country, do not mark off and re-roll.
\aparag If there is a war between at least two of the following countries:
\RUS, \AUS, \PRU, do not mark off and re-roll.
\aparag If \payspologne doesn't exist any more, mark off and play \RD instead.
\aparag Depending on the current status of \payspologne, apply the correct
subevent (apply the first matching case). Only one such subevent may occur in
the game. In each case, the partition may be accepted and is described in
\xnameref{pVII:1PP:Partition}.
\bparag If Absolutism is established in \payspologne, apply
\xnameref{pVII:1PP:Absolutism or Protector}.
\aparag If \payspologne is a special \EG of either \FRA or \SUE as per
\ref{pVI:WoPS:Polish Victory} or \ref{pVI:GNW:Stanislas}, apply
\xnameref{pVII:1PP:Absolutism or Protector}.
\bparag If \payspologne is neutral or on the diplomatic track of either \RUS,
\AUS or \PRU, apply \xnameref{pVII:1PP:Poland Neutral}.
\bparag If \payspologne is on the diplomatic track of another major who
accepts the partition, apply \xnameref{pVII:1PP:Poland Neutral}
\bparag Otherwise, apply \xnameref{pVII:1PP:Poland Minor}.


\digression[pVII:1PP:Partition]{First Partition Plan of \payspologne}
\aparag The proposed partition of \payspologne gives the following provinces
to each major country:
\bparag \RUS gets all the Polish provinces in \regionUkraine,
\provinceSeveria, \provinceSmolenska, \provinceBaltarusija and
\provincePolacak.
\bparag \PRU gets all the provinces of \region{Duche de Prusse} and
\province{West Preussen}.
\bparag \AUS gets all the Polish provinces formerly part of \payshongrie,
\provinceMorava and \provinceMalopolska
\bparag \SUE gets a province of its choice, not part of the share of any other
country, adjacent to its territory.
\aparag If some of the provinces explicitly mentioned (not those part of a
group) no more belongs to \payspologne, the major instead gets a free \CB
against the owner of the province for the next diplomacy phase.
\aparag The acceptance of the partition plan depends on the status of
\payspologne and the result of the ensuing war.


\subevent[pVII:1PP:Absolutism or Protector]{\payspologne is absolutist or has
  a protector}

\phevnt
\aparag \RUS, \AUS, \PRU and \SUE all have a normal \CB against \payspologne
and its protector.
\bparag If \payspologne has no protector (but is absolutist), it call for
allies as per normal rules, the major accepting to help it has a free \CB
against all countries that declared war to \payspologne and is called
protector in the rest of the event.
\aparag If several countries declare war on \payspologne using this \CB, they
can choose to be allied for the duration of the war without need to sign a
formal alliance.
\bparag However, they can also choose to wage separate wars in which case they
can fight among them inside the territory of \payspologne and the national
territory of \POL. In this case, each alliance is considered separately for
the peace conditions.
\bparag There may be several different alliances fighting against \payspologne
(and among themselves).

\phpaix
\aparag \payspologne won't sign a separate peace in this war.
\aparag If the protector signs an unfavourable peace of level 3 or more, or if
\payspologne without protector signs an unconditional surrender, the following
effects are added to the peace:
\bparag \payspologne becomes a normal minor (and no more a special \EG).
\bparag \payspologne becomes neutral.
\bparag Absolutism is abolished in \payspologne
\bparag From now on, any country can annex the capital of \payspologne.
\bparag Instead of all peace conditions, the enemies of \payspologne can
choose to apply the partition proposed in \xnameref{pVII:1PP:Partition}, in
which case only the countries that were at war against \payspologne get their
share.


\subevent[pVII:1PP:Poland Minor]{\payspologne is a regular ally}

\phevnt
\aparag \RUS, \AUS, \PRU and \SUE all have a free \CB to be used conjointly
against \payspologne and its diplomatic patron.
\aparag If several countries declare war on \payspologne using this \CB, they
can choose to be allied for the duration of the war without need to sign a
formal alliance.
\bparag However, they can also choose to wage separate wars in which case they
can fight among them inside the territory of \payspologne and the national
territory of \POL. In this case, each alliance is considered separately for
the peace conditions.
\bparag There may be several different alliances fighting against \payspologne
(and among themselves).

\phadm
\aparag \payspologne must take reinforcements in defensive attitude for the
duration of the war.

\phpaix
\aparag \payspologne may sign a separate peace as per normal rules.
\aparag If \payspologne or the major helping it signs an unfavourable peace of
level 3 or more, the following effects are added to the peace:
\bparag \payspologne becomes neutral.
\bparag From now on, any country can annex the capital of \payspologne.
\bparag Instead of all peace conditions, the enemies of \payspologne can
choose to apply the partition proposed in \xnameref{pVII:1PP:Partition}, in
which case only the countries that were at war against \payspologne get their
share.


\subevent[pVII:1PP:Poland Neutral]{\payspologne is not defended}

\phevnt
\aparag \payspologne becomes neutral.
\aparag The partition described in \xnameref{pVII:1PP:Partition} is accepted
and every country take is share.



\event{pVII:Second Partition Poland}{VII-8}{Second Partition of
  Poland}{*}{PBnew}

\history{1791, 1793}

\condition{}
\aparag If \ref{pVII:First Partition Poland} did not occur yet, do not mark
off and re-roll.
\aparag If there is a war between at least two of the following countries:
\RUS, \AUS, \PRU, do not mark off and re-roll.
\aparag If \payspologne doesn't exist any more, mark off, play and \RD with
the \REVOLT in \POL.
\bparag In addition, if \ref{pVII:Independence War} already occurred at least
once, play that event again.
\aparag The event is resolved in the same way as \ref{pVII:First Partition
  Poland} (depending on the status of \payspologne) but with the partition
plan described here.
\aparag This event may occur several times.


\digression[pVII:2PP:Partition]{Second and following Partition Plans}
\aparag The proposed partition of \payspologne gives the following provinces
to each major country:
% (Jym) Smolensk disappeared of the \RUS package between 1PP and 2PP.
\bparag \RUS gets all the Polish provinces in \regionUkraine,
\provinceSeveria, \provinceBaltarusija and \provincePolacak. If none of the
belong to \payspologne, \RUS gets instead \provinceLietuva, \provinceZemaitija
and \provincePrypec.
\bparag \PRU gets all the provinces of \region{Duche de Prusse} and
\province{West Preussen}. If none of the belong to \payspologne, \PRU gets
instead \provinceDanzig, \provinceWielkopolska and \provinceMazowia.
\bparag \AUS gets all the Polish provinces formerly part of \payshongrie,
\provinceMorava and \provinceMalopolska. If none of them belong to
\payspologne, \AUS gets instead \provinceWolyn and \provinceLublin.
\bparag \SUE gets a province of its choice, adjacent to its territory, even
one part of the share of another country.
\aparag If some of the provinces explicitly mentioned (not those part of a
group) no more belongs to \payspologne, the major instead gets a free \CB
against the owner of the province for the next diplomacy phase.
\aparag If some provinces are claimed by several countries, the one occupying
it at the time of the partition annexes the province. \SUE does if nobody
occupy it.



\event{pVII:National Revival of Poland}{VII-9}{National Revival of
  Poland}{2}{PBnew}

\history{1795}

\condition{Cannot occur before \ref{pVII:First Partition Poland}. In that
  case, do not mark off and re-roll.}
\aparag Each of these events can happen only once.
\bparag If there no more \payspologne, apply \xnameref{pVII:NRP:Kosciusko}.
\bparag If \payspologne still exists, apply \xnameref{pVII:NRP:Commonwealth
  Revival}.


\subevent[pVII:NRP:Kosciusko]{Kosciusko's revolt}
\history{1795}

\phevnt
\aparag Place \REVOLT in the following provinces: \provinceLietuva,
\provinceMazowia, \provinceLublin and \provinceWielkopolska.
\bparag The \REVOLT are \faceplus if \ref{pVII:French Revolution} already
occurred at least once and \facemoins otherwise.
\bparag Military troops in these provinces must retreat.
\bparag Only the fortress of \villeVarsovie is taken by the rebels.
\bparag Put an \ARMY\facemoins of \payspologne with general
\leaderwithdata{Kosciuszko} (lasting until the end of the game) in a revolted
province.
\aparag The minor country \payspologne is created anew with these troops and
provinces.
% (Jym) Putting this in \phevnt so that the choice has to be without any
% private discussions.
\aparag \payspologne is looking for a foreign help. The following countries
must immediately accept or refuse, in order:
\bparag \FRA, if \ref{pVII:French Revolution} already occurred at least once.
\bparag \FRA or \SUE, whichever last got \payspologne as a special \EG due
either to \ref{pVI:WoPS:Polish Victory} or \ref{pVI:GNW:Stanislas Victory}
\bparag \FRA, \SUE, \AUS, \PRU.
\aparag The country who accepts to help \payspologne immediately declares war
(with a \CB) against all the countries owning a national province of \POL.
% (Jym) adding
\bparag This is just one declaration of war, not one per enemy country. Hence
the \STAB loss is only 1.
\aparag \payspologne is put in \EG of its helper.

\phdipl
% (Jym) switched to \phdipl to avoid any question about who gets the income of
% these provinces.
\aparag The \MAJ who accepted to help \payspologne must immediately give to
\payspologne all the national provinces of \POL it currently owns.
\bparag There is no loss of \VP for these provinces.

\phadm
\aparag \payspologne get reinforcements as a regular minor based on the income
of provinces it owns and control (as per normal rules).

\phmil
\aparag Troops of \payspologne stacked with \leaderKosciuszko are always
veterans.

\phpaix
\aparag \REVOLT may spread only in national provinces of \POL but may do so
even through frontiers of major countries.
\aparag \payspologne will not sign a white or unfavourable peace in this war.
\aparag If there are no more \REVOLT and no more troops of \payspologne, the
minor is destroyed again.
% (Jym) adding: the protector has no real risk since he will get back the
% ceded provinces
\bparag Ownership of provinces goes back to whoever owned them at the
beginning of the war.
% (Jym) adding: so that the protector has a risk
\bparag Other countries involved in the war may either sign a white peace or
continue fighting.
\aparag If \payspologne and its allies sign a favourable peace, all provinces
annexed at the peace must be national provinces of \POL and are given to
\payspologne.
% (Jym) Is it possible to take something else than provinces in this peace?
\bparag \payspologne becomes a permanent \EG of its protector as described in
\ref{pVI:WoPS:Polish Victory}.
\bparag \payspologne should now own: the four initially revolted provinces,
the provinces given by the protector and the provinces annexed at the peace.
\bparag This may happen also if the \REVOLT and troops were crushed but the
protector kept on fighting and won the war.


\subevent[pVII:NRP:Commonwealth Revival]{Commonwealth's Revival}
\history{not historic}

\phadm
\aparag \payspologne receives the general \leaderwithdata{Kosciuszko} for the
rest of the game.
\aparag Until the end of the game, each turn where there is a declaration of
war against \payspologne, roll for two \REVOLT in \POL.
\bparag The \REVOLT may happen in any country (not only \payspologne) and
their force is rolled at random.
\bparag The \REVOLT must occur in national territory of \POL. If they fall out
of it, re-roll another \REVOLT . However, both \REVOLT may occur in the same
province.
\bparag If \ref{pVII:French Revolution} occurred at least once in a previous
turn, roll four \REVOLT instead of two.

\phmil
\aparag Troops of \payspologne stacked with \leaderKosciuszko are always
veterans.
\aparag \REVOLT created by this event (and their fortresses or troops) are
allied with \payspologne.
% (Jym) adding
\bparag \REVOLT counters are limited supply sources for the troops of
\payspologne (only, not its allies).

\phpaix
\aparag The \REVOLT may only spread in national provinces of \POL but can do
so through national borders of major countries.
\aparag Revolted provinces count as if controlled by \payspologne for the
peace procedure.



\event{pVII:Mameluks Revolt}{VII-10}{Independence of the Mameluks in
  Egypt}{1}{RistoMod}

\history{1795 (Bonaparte in Egypt)}

\condition{}
\aparag If the current monarch of \TUR has an \ADM of at least 8, he can
choose to cancel the event.
\bparag In this case, place a \REVOLT (with random strength) in all the former
provinces of \paysmamelouks.

\phevnt
\aparag \paysmamelouks is recreated. It owns all the provinces it had at the
start of the game that now belong to \TUR.
\bparag Its basic forces are \ARMY\facemoins, \LD and it can use all its
counters.
\bparag \TUR loses \VP for the provinces lost.

\phdipl
\aparag \TUR has a temporary free \CB against \paysmamelouks for this turn
only.

\phpaix
\aparag If \TUR achieves an enforced unconditional victory over \paysmamelouks
during a war caused by this event, it can annex it again, gaining \VP for the
provinces annexed.

\effetlong
\aparag \FRA, \ENG, \HOL and \HIS have a permanent \CB against \paysmamelouks.
% (Jym) adding
\bparag If several of them use this \CB without being formally allied, they
can fight inside the territory of \paysmamelouks and \seazone{Mediterranee E}
even if not at war elsewhere.

\aparag If, at the beginning of a peace phase, one of them controls the
capital and half the other provinces of \paysmamelouks, \paysmamelouks becomes
a permanent \VASSAL of the major occupying it and no diplomacy is possible on
it.
\bparag If the major later signs an unfavourable peace, one peace condition
can be to turn back \paysmamelouks into a regular normal country who then
becomes neutral.
\bparag It is also always possible to wage war against \paysmamelouks and
``steal'' the special \VASSAL status by occupying it.
% (Jym) adding
\aparag From now on, \FRA, \ENG and \HOL can declare war on \payschevaliers at
normal cost (instead of the one mentioned in \ruleref{chSpecific:Knights}) and
they can annex the capital province of \payschevaliers thus destroying the
country.



\event{pVII:Revolt Indonesia}{VII-11}{Revolt in Indonesia}{*}{Risto}

\history{No precise date}

\phevnt
\aparag Place one \REVOLT \facemoins and one \REVOLT \faceplus in two randomly
chosen \COL/\TP in areas \granderegionJava, \granderegionSumatra,
\granderegionBorneo and \granderegionCelebes. Both \REVOLT can occur in the
same place. Roll on \ref{table:alt-revolt-global} for the control of these
\REVOLT .



\event{pVII:Sale Corsica}{VII-12}{Sale of Corsica}{1}{Risto}

\history{1759}

\condition{If \provinceCorsica does not belong to either \payscorse or
  \paysgenes, treat this as a \REVOLT in \provinceCorsica (roll for strength
  as usual) and mark off.}

\phevnt
\aparag \provinceCorsica is for sale. Each player must immediately make a
secret bid for it and the highest bid annexes \provinceCorsica. Only the
winning bid is actually paid. If it bids at least 1\ducats, \FRA receives a
bonus of 50\ducats for its bid.

\phdipl
\aparag If \provinceCorsica is currently occupied by foreign troops, the owner
of those troops must either declare a war to the new controller of this
province profiting from a \CB, or withdraw its forces as per peace process.



\event{pVII:Pugatchev Revolt}{VII-13}{Revolt of Pugatchev}{1}{RistoMod}

\history{1773-1774}
\dure{Until the end of the civil war.}

\tour{The initial revolt}

\phevnt
\aparag A civil war erupts in \RUS. The rebels are controlled by \SPA, or by
\SUE if \SPA is allied to \RUS.
\aparag Place a \REVOLT \facemoins in the former provinces of the following
minor countries currently belonging to \RUS: \payskazan, \paysastrakhan,
\payssteppes, \payscrimee%
% (Jym) Before, \paysKazan and \paysAstrakhan were only one province. The
% event is a bit harder now, but remains largely manageable from experience.
and all \ROTW provinces adjacent to \RUS European territory that have \RUS
\COL/\TP in them. Roll for two additional \REVOLT in \RUS. If the result is
outside \RUS territory, ignore and do not re-roll.
\aparag Place a revolt \ARMY\facemoins and general \leaderPugachev in any
revolted province (he can either lead the \ARMY or a \REVOLT ).
\bparag The class of rebels armies is the same as \RUS.

\phdipl
\aparag Countries adjacent to \RUS can make a foreign intervention in any side
of the war.

\phadm
\aparag The rebels roll for reinforcements in offensive status during each
turn of the civil war.
\bparag The modifier for reinforcement is computed based on the income of the
provinces in \REVOLT , even if the rebel does not control the fortress.

\phmil
\aparag All rebel units can use \REVOLT counters as supply bases in the same
way as fortresses as long as there are no non-defeated enemy units present at
the moment supply is needed.

\phpaix
\aparag The war end either by suppressing all the \REVOLT or if the \REVOLT
cause the government to be overthrown.
\aparag There is no extension of \REVOLT if the rebels suffer a major defeat
or if there is no more \ARMY counter of the rebels.

\tour{Siberian revival}

\phadm
\aparag Starting from the third turn of the revolt, if a rebel army is located
during this phase in any former province of \payskazan, \paysastrakhan, or
\payssteppes, the rebels receive the \paysSiberie \ARMY\faceplus as extra
reinforcement this turn.
\bparag This extra reinforcement can only happen once in the war.
\bparag This army can freely stack with the rebels or exchange \LD in order to
replenish one or another.



\event{pVII:Potemkin}{VII-14}{Potemkin}{1}{Risto}

\history{1783-1791}
\dure{as long as \strongministre{Potemkine} remains the excellent minister}

\condition{\RUS can refuse this event if it so wishes. In that case mark off
  as played.

If \monarque{Pierre II} rules Russia, \RUS may choose to postpone the event
for one turn.}
\aparag \RUS can freely dismiss \ministrePotemkine at the end of any following
monarch survival phase and the event terminates.

\phevnt
\aparag \RUS receives an excellent minister \ministrePotemkine, with values
9/8/8.  He will last for a random length for Minister, see \ref{eco:Excellent
  Minister}.

\phadm
\aparag \RUS basic force is increased by \FLEET\facemoins during every turn
\RUS is engaged in a war and \ministrePotemkine is in charge.

\phmil
\aparag As long as this event is in effect \RUS receives an additional bonus
of \bonus{+1} to all attempts to suppress \REVOLT .



\event{pVII:War Crimea}{VII-15}{War in Crimea}{2}{PBnew}

\history{1768-1774, 1787-1792}
% (Jym) one occurrence only? Present twice in the table.

\phevnt
\aparag \RUS has a Free \CB against \TUR at this turn or the next one.



\event{pVII:War Finland}{VII-16}{War in Finland}{1}{PBnew}

\history{1788-1790}

\phevnt
\aparag \SUE has a free \CB against \RUS if \RUS owns at least one province in
\regionFinlande.
\aparag \RUS has a free \CB against \SUE if \SUE owns at least one province in
\regionFinlande or on the \regionBaltique (between \provinceNeva and
\provinceCourlande included).



\event{pVII:Forward Balkans}{VII-17}{Forward to the Balkans}{1}{PBnew}

% (Jym) Present only once in the table.
\history{No precise date}

\phevnt
\aparag \AUT has a Free \CB against \TUR at this turn or the next one.



\event{pVII:Wars India}{VII-18}{Wars in India}{3}{PBnew}

\condition{}
\aparag If \ref{pVI:Last Great Mughals} did not happen yet, apply it instead.
\aparag Otherwise, apply \xnameref{pVI:Wars India} but with the following die
roll:
\bparag 1-4 = A) War between \paysmogol and \paysperse. Apply
\xnameref{pVI:India:Mughal Persian War}.
\bparag 5-6 = B) War between \paysafghans and \paysperse. Apply both
\xnameref{pVI:India:Afghan Empire} and \xnameref{pVI:India:Fall Persian
  Safavids}.
\bparag 7-10 = C) War between \paysafghans and \paysmogol. Apply both
\xnameref{pVI:India:Afghan Empire} and \xnameref{pVI:India:Rise Marathi}. This
case may not happen before either case A above, re-roll another case if
needed.



\event{pVII:Vassalisation Hanover}{VII-19 (1)}{Vassalisation of
  \payshanovre}{1}{Risto}

\phevnt
\aparag Same event as \ref{pVI:Vassalisation Hanover}.
\aparag If already occurred, apply \ref{pVII:William Pitt}.



\event{pVII:William Pitt}{VII-20}{William Pitt}{1}{Risto}

\history{1757-1761}
\dure{as long as \strongministre{Pitt} remains the excellent minister}

\condition{\ANG can refuse this event if it so wishes. In that case mark off
  as played.}
\aparag \ANG can freely dismiss \ministrePitt at the end of any following
monarch survival phase and the event terminates.

\phevnt
\aparag \ANG receives an excellent Minister \ministrePitt, with values 9/8/8.
He will last for a random length for Minister, see \ref{eco:Excellent
  Minister}.

\phdipl
\aparag \ANG may send \VASSAL troops in the \ROTW without paying the \STAB
indicated in~\ref{chSpecific:England:Minors at war}.

\phadm
\aparag \ANG basic forces are increased by \FLEET\facemoins and \ARMY\faceplus
during every turn where \ANG is engaged in a war (including oversea war) and
\ministrePitt is in charge.



\event{pVII:Kaunitz}{VII-21}{Kaunitz}{1}{Risto}

\history{1753-1793}
\dure{as long as \strongministre{Kaunitz} remains the excellent minister}

\condition{\AUS can refuse this event if it so wishes. In that case mark off
  as played.}
\aparag \AUS can freely dismiss \ministreKaunitz at the end of any following
monarch survival phase and the event terminates.

\phevnt
\aparag \AUS receives an excellent Minister \ministreKaunitz, with values
9/8/7.  He will last for a random length for Minister, see \ref{eco:Excellent
  Minister}.

% (Jym), Placeholder

\event{pVII:Comuneros}{VII-x}{Revolt of the Comuneros}{1}{JymNotEvenWritten}
\history{1779-1781}
\begin{todo}
  Revolt in New Granada. Probably useless (handle by revolt tables).
\end{todo}

\event{pVII:Xhosa}{VII-y}{Xhosa wars}{1}{JymNotReallyWritten}
\history{1779-1781/1789-1793/1799-1803}

\begin{todo}
  These may be the true intention of the ``Bantu raids'' of pVI. May
  replace~\ref{pVII:Revolt Indonesia} since it moved in \REVOLT tables.

  Same effect as~\ref{pVI:Bantu Raids}.
\end{todo}

\event{pVII:USA-Morocco}{VII-z}{Moroccan-American Treaty of
  Friendship}{1}{JymVetoPending}

\history{1777}

\condition{If \paysusa does not exists, do not mark off and reroll.}
\dure{Until the end of the game}

\effetlong
\aparag Place one level of \TradeFLEET of \paysusa in \stz{lion}.
\bparag The reference level for \paysusa in \stz{lion} is now 1.
% (Jym) http://en.wikipedia.org/wiki/Moroccan-American_Treaty_of_Friendship +
% Barbaresques piracy degenerated in war with the \paysusa in 1801 and 1815:
% http://en.wikipedia.org/wiki/Barbary_Wars

\stopevents

% Local Variables:
% fill-column: 78
% coding: utf-8-unix
% mode-require-final-newline: t
% mode: flyspell
% ispell-local-dictionary: "british"
% End:

% LocalWords: pVI pVII Batavian Mameluks Pugatchev Vassalisation Kaunitz de
% LocalWords: PBnew RistoMod Bolivarian PBMod malus Directoire Masse migr JCD
% LocalWords: artilleries Chouans Patrie subevent Duche Prusse Preussen NRP
% LocalWords: Kosciuszko Mediterranee Risto Safavids JymVetoPending reroll IW
% LocalWords: Jym SYW Orangists VOC Stadhouder Emigr Mughal Barbaresques http
% LocalWords: Potemkine Comuneros
