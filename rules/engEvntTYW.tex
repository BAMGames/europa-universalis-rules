%% *-* latex-mode *-*



\event{pIV:TYW}{IV-A}{Thirty Years' War}{1}{PB}

\history{1618-1648}

\activation{This war is a consequence of some religious fighting in the
  \HRE. If \ref{pV:WoSS} has already begun, this event is not possible
  anymore. Ignore it.}
\aparag It might be triggered by \xnameref{pII:Schmalkaldic League},
\xnameref{pIII:League Nassau}, \xnameref{pIV:Bohemian Revolt},
\xnameref{pIV:Augsburg Revocation} or \xnameref{pIV:Unity HRE}.  This event
may happen only once; before that, at the end of the first turn of a war
caused by one of the previous event, make the following test.
\aparag Roll 1d10 and add the modifiers:\par
\begin{modlist}
\item[\bonus{+2}] in period II and III
\item[\bonus{-2}]for each turn of the current war before this turn
\item[\bonus{-1}] if the peace modifier of the \HAB is >0
\item[\bonus{+2}] if \monarque{Charles V} rules \SPA
\item[\bonus{+2}] if \SPA has chosen \CATHCO
\item[\bonus{+2}] if \villeVienne is not owned and controlled by \HAB
\item[\bonus{+4}] is test during \xnameref{pII:Schmalkaldic League}
\item[\bonus{+2}] if test during \xnameref{pIII:League Nassau} and \SPA is
  \CATHCR
\item[\bonus{-2}] if test during \xnameref{pIV:Bohemian Revolt}
\item[\bonus{-2}] if test during \xnameref{pIV:Unity HRE}
\item[\bonus{-4}] if during \xnameref{pIV:Augsburg Revocation}
\item[\bonus{\textplusminus1}] if \ministre{Richelieu} or \ministre{Mazarin}
  are still present (choice of \FRA)
\item[\bonus{+1}] If \nameref{pIII:FWR} have yet to happen
\item[\bonus{+3}] If \nameref{pIII:FWR} are happening now
\item[\bonus{-1}] If Protestant won in \nameref{pIII:FWR}
\item[\bonus{+1}] If Counter-Reformation won in \nameref{pIII:FWR}
\end{modlist}

\aparag Result:\par
\begin{modlist}
\item[\geq 11] Appeasement of the religious fight
\item[7--10] Agitations in the \HRE
\item[\leq 6] Eruption of the Religious War
\end{modlist}
\aparag[Appeasement of the religious fight] The current war does not
degenerate in a general Religious War. No further test will be made for this
war.
\aparag[Agitations in the \HRE]
\bparag One \MIN enemy of \HAB will have a bonus of \bonus{+2} to its
reinforcement roll next turn (Alliance's choice).
\bparag \paysSaxe joins the enemy side of the \HAB in full intervention (or
\paysBrandebourg if \paysSaxe is already at war).
\bparag At the end of the next turn, roll this test anew to see if a Religious
War breaks.
\aparag[Eruption of the Religious War] The rest of the event will be applied
as one of the 4 regular events of the next turn.  No peace is made for the war
of this turn in the \HRE (except for specific rules of this war about
conquered minor countries).  The Thirty Years' War is now about to begin.

\phevnt
\begin{digressions}[War setup]


  \digression[pIV:TYW:Creation of the Germanic Alliances]{Creation of the
    Germanic Alliances}
  \aparag Two German sides are made up for this war: the (German) Catholic
  \ligue and the Protestant \alliance (more properly called: \emph{Protestant
    Union} or \emph{League of Evangelical Union}).  All minor countries of the
  \HRE at war will be part of one or another. When a minor country joins one
  alliance, it is placed in Neutral diplomatic position and will change of
  status before the end of the war only if specified by this event or another
  political event. The \HRE is now in Civil and Religious War
  (see~\ref{chDiplo:Religious Civil War}), with all the usual restrictions.
  \bparag The \alliance is formed by all the German minor countries that were
  enemies of the \HAB during the previous turn.
  \bparag \HAB and its German allies (minor countries at war with it) form the
  \ligue. \AUSMin is part of the \ligue as any other minor. \paysBaviere
  automatically joins this alliance.
  \bparag The stability of both sides is placed on \bonus{+2}, modified by any
  Major Victory of the preceding turn of their side (battles with troops of
  German minor countries or \HAB).  This stability will evolve during the turn
  because of the major victory/defeat of any forces in their alliance that is
  in any province of the \HRE (even if there are only forces of non Germanic
  major powers).
  \aparag[Attitude of the Netherlands] If \HOL is not a Major Power, the
  following conditions apply:
  \bparag If \payshollande is either owned by \SPA or is \paysprovincesne or
  \paysVhollande, apply \ref{pIII:Dutch Revolt}. This gives a new status to
  \payshollande (it may trigger the following points if still a \MIN).
  \bparag If \payshollande is a \VASSAL of \SPA (special or regular),
  \payshollande breaks its special status with \SPA. \SPA has an immediate
  free \CB against \payshollande ; if used, \payshollande revolts against the
  Spanish Crown, (re)apply \numberref{pIII:Dutch Revolt} and \HOL is now a
  Major Power. If it does not use it, apply \ref{pIV:TYW:VEN transfer}. For
  the rest of the event \HOLhol is neutral, and may not be involved in any
  manner in the incoming war. Ignore any reference to \HOLhol hereafter for
  this event.
  \bparag If \payshollande is a normal minor country, apply \ref{pIV:TYW:VEN
    transfer}. \HOLhol is involved in the war.
  \aparag[Transfer to \HOL]\label{pIV:TYW:VEN transfer} If
  \payshollande is liberated by the preceding paragraph, \VEN may be allowed
  to choose between incarnating \AUS or \HOL according to the rules of the
  Grand Campaign.
  \bparag If \VEN chooses \AUSMin (which becomes \AUS), \payshollande is now a
  normal minor country.
  \bparag If \VEN chooses \HOLmin (which becomes \HOL), \HOL is created with
  no Revolt (using the current position of \HOLmin).
  \bparag TODO: establish full starting position of non-revolted \HOL.
  \aparag The \alliance is controlled according to the order of preference (a
  player may not refuse control): \HOL, \ENG (Protestant), \FRA (Protestant),
  \SUE (Protestant), \RUS.
  \aparag The \ligue is controlled according to the order of preference (a
  player may not refuse control): \SPA (Counter-Reformation), \AUS (if it
  exists), \SPA (Conciliatory).
  \aparag If the \nameref{pII:Schmalkaldic League} or the \nameref{pIII:League
    Nassau} still do exist, the countries part of the League immediately join
  the Protestant \alliance and the Leagues are dissolved.
  \aparag If the period IV has not begun yet, the Major Powers: \SPA, \HOL,
  \SUE, \FRA and \AUT have to choose immediately if they take or not the
  Objectives relevant to this war. The Objective are conditions to be true at
  the end of period IV (and not especially this war).


  \digression[pIV:TYW:Extension Alliances]{Extension of the alliances}
  \aparag Every minor country of the \HRE that is not part of the war is
  checked for war entry at the beginning of each turn. One rolls 1d10, added
  to the \STAB of the side it could join, the current turn of the war
  (\bonus{+1} this first turn), and specific modifier for some countries.  On
  a result of {\bf 6 or higher}, this country enters the war.
  \aparag The list of the countries of the \HRE is given in
  \ref{table:TYW:Extension table}, with the side they will join and their
  starting force.  All those forces are conscripts, except where indicated.
  It is possible that, given the peculiar conditions of the war triggering the
  Religious War, a country ends up in a different side of the one which should
  be expected.
  \begin{table}\centering
    \begin{tabular}{l|l|c|p{.5\textwidth}}
      Country & Side & Mod. & Forces \\\hline
      \paysBaviere & \ligue & Auto. & \ARMY\faceplus, \LD, \fortress and at
      least 1 General (see below);
      % if none available, use \leader{Mercy} that will stay
      % until his death or when this war ends
      may use 2 \ARMY counters for all
      the duration of the war; starting forces are Veterans.\\
      \paysCologne & \ligue & & \LD, 1 \fortress\\
      \paysLiege & \ligue & & \fortress\\
      \paysMayence & \ligue & & \fortress\\
      \paysTreves & \ligue & & \fortress\\
      \paysAlsace & \ligue & --2 & \LD, \fortress\\
      \paysLorraine & \ligue & --4 & \LD\\
      \paysWurtemberg & \ligue & --2 & 2 \LD\\
      \paysThuringe  & \ligue & --2& none\\
      \paysBade &\alliance& & 2 \LD and \LeaderG (Georg Friedrich of
      Baden)\\
      \paysPalatinat &\alliance&& \ARMY\facemoins and \fortress\\
      \paysBerg &\alliance& --2& \LD\\
      \paysBrandebourg &\alliance& --2&\ARMY\facemoins and \LeaderG\\
      \paysBrunswick &\alliance& &\ARMY\facemoins and \LeaderG (Christian
      of Brunswick)\\
      \paysHanovre & \alliance & & \LD and \fortress\\
      \paysOldenburg & \alliance & --2 & \fortress\\
      \paysHanse&\alliance && \LD, \DN\\
      \paysHesse& \alliance& --2& \ARMY\facemoins and \fortress\\
      \paysSaxe&\alliance &--4& \ARMY\facemoins, \LD  and \fortress\\
      \paysBoheme &\alliance && \ARMY\facemoins and \LD\\
    \end{tabular}
    \caption{Extension of the Alliances during the Thirty Years' War}%
    \label{table:TYW:Extension table}
  \end{table}

  \aparag[Mercy] If there is no named \LeaderG of \paysBaviere in play, it
  receives \leaderMercy.
  \bparag If there is one, as soon as he dies (wound is not enough),
  \paysBaviere immediately receives \leaderMercy.
  \bparag \leaderMercy stays in play for 4 turns. If he arrives in the middle
  of a turn (due to death of his predecessor), this turn fully counts as his
  first turn of activity.

  \aparag The forces written may be inferior to the basic forces of the
  country (representing the confused situation).  They are only used when the
  country join the alliance. If already at war a previous turn, a country
  keeps all that is deployed and gains nothing new.
  \aparag If \AUSmin joins war at this time, they receive their basic force
  plus 1 \ARMY \faceplus (but no supplementary random reinforcement ; that
  will be part of those of the \ligue) as Veterans.
  \aparag No intervention (full or limited) of foreign countries are allowed
  if it is not explicitly written in this event.
  \aparag \paysSaxe may be used as mercenaries during this event once it
  surrendered all its home territory to the enemy. Its army is available to
  the side that controls its home territories; if this side loses subsequently
  part of the provinces, it still uses the army but can no more recruit
  Saxons; if it loses all the provinces, the Saxon forces are removed (and
  available now to the enemy).
\end{digressions}

\tour{Turn 1 (1624--1629)}

\phevnt
\aparag From now, and until the war is ended by the \xnameref{pIV:TYW:Peace
  Westphalie}, no Diplomacy is possible on minor countries of the \HRE, no
attempt to have them enter in a war also, and no declaration of war against
them is possible outside the rules of this event.
\aparag After the creation and the extension of both German sides in the war,
some foreign countries can be involved in it also.
\aparag The controller of each alliance can declare war to German minor
countries that refused to be in war this turn, precipitating them in the enemy
alliance (regardless of their religion).
\aparag \SPA enters the war as an ally of \ligue. This is not a formal
declaration of war and costs no \STAB.
\aparag \HOLhol enters war as an ally of \alliance.  This is not a formal
declaration of war and costs no \STAB. \HOLMin receives its full basic forces,
has a separate die-roll for reinforcements, is allied to the \alliance but not
part of it (for the conditions specifying that the \alliance sues for peace).
\aparag \ENG can do a limited intervention. Its side is the \alliance if \ENG
is Protestant, the \ligue if it is \CATHCR, or the one of its choice if it has
chosen \CATHCO.
\aparag \SUE, if \PROTRIG, can do a limited intervention as an ally of the
\alliance.
\aparag The Emperor of the \HRE, if he is not \HAB, can begin a limited
intervention in the War as an ally of the \ligue.
\aparag Any Major Power that was doing a limited intervention during the
previous turn (as defined in the original war) can continue this limited
intervention to help the same side.
% (JCD): TODO adapt to DAN
\aparag[The Danish Crusade]\label{pIV:TYW:Danish Crusade}
\DANMin makes a mandatory white peace with all its adversaries.  It then
enters the war as an ally of \alliance (but not part of it). It has 2 \ARMY
\faceplus (Veteran), 1 \FLEET \facemoins, 1 \fortress, 2 Multiple Campaigns
and is led by its general-king \leader{Christian IV} present for 4 turns.  It
does not receive reinforcements on this turn. \DANMin is played by \ENG.
\aparag All those alliances and interventions during the whole war are made
with the German alliances; the foreign countries are not allied with each
other except if they decide to sign a specific alliance. Else, they are not
obliged to continue the fight together (no penalty to sign peace) and only
separate peace from the German alliance is required.

\begin{digressions}[Specific rules for the war]


  \digression[pIV:TYW:Turkish Frontier]{The Turkish frontier}
  \aparag As long as there are 2 \ARMY\faceplus of \HAB in \villeVienne or any
  province once owned by \paysHongrie and a \LeaderG, \TUR may not declare a
  war to \HAB (but may continue one). For the first turn, this restriction is
  enforced if \HAB has this force available anywhere in the \HRE instead.
  \aparag If \villeVienne is conquered by the \alliance, or the previous
  condition is not respected at the Diplomatic Phase, \TUR has no such
  restriction.
  \aparag If \TUR takes \villeVienne, the \ligue will concede a winning peace
  to the \alliance at the end of the turn. A Crusade might then happen.
  % \aparag \TUR is entitled to make a Foreign Intervention against the \ligue
  % in this war if otherwise at peace and \paysTransylvanie is \VASSAL or
  % annexed.

  \aparag[] [BLP] \ref{chSpecific:Little war} is reactivated for \TUR only,
  and only with a small stack (up to 5\LD plus one \Pasha).
  \bparag That is, \TUR (not \paysCrimee) may send one (small) stack in non
  controlled former provinces of \paysHongrie and loses \STAB accordingly.
  \bparag Additionally, \TUR may also send this stack in national provinces of
  \AUS.

  \digression[pIV:TYW:German Reinforcements]{German reinforcements}

  \phadm
  \aparag Reinforcements for both \alliance and \ligue are determined globally
  for all German minor countries involved in an alliance.
  \aparag The \alliance is due to receive 4 \LD and the result of random
  reinforcements in defensive attitude with a global modifier of \bonus{+2}.
  \aparag The controller of the \alliance can pay 50\ducats to give a further
  \bonus{+1} to the reinforcement roll, or 100\ducats for a \bonus{+2}. If it
  does not pay, \SUE has the opportunity to do so and in this case will
  control \alliance for this turn only.
  \aparag The reinforcements of the \alliance are lowered by 1 \LD for each
  one of the following cities that have been conquered by the enemies (even if
  liberated later on): \villeMagdeburg and:
  \begin{itemize}
  \item \villeStuttgart, \villeErfurt if the war follows
    \xnameref{pII:Schmalkaldic League},
  \item \villeMunster, \villeRostock if the war follows \xnameref{pIII:League
      Nassau},
  \item \villeSpeyer, \villePrague if the war follows \xnameref{pIV:Bohemian
      Revolt}
  \item \villeFrankfurt, \villeErfurt if the war follows
    \xnameref{pIV:Augsburg Revocation} or \xnameref{pIV:Unity HRE}.
  \end{itemize}
  \aparag The reinforcements of the \alliance are also lowered by 1 \LD for
  each two cities in the following list that have been conquered by the
  enemies (even if liberated later on): \villeHannover, \villeCassel,
  \villeDresden, \villeBerlin, \villeLubeck, \villeHamburg.
  \aparag If \AUSmin is part of the \ligue, the \ligue is due to receive 3 \LD
  and the result of random reinforcements in defensive attitude with a global
  modifier of \bonus{+2}. Else (\AUS is a \MAJ), the \ligue receives only a
  random reinforcements with a global modifier of \bonus{+2}.  The \ligue uses
  the \ARMY counter of the \HRE regardless of who the Emperor is.
  \aparag The controller of the \ligue can pay 50\ducats to give a further
  \bonus{+1} to the reinforcement roll, or 100\ducats for a \bonus{+2}.
  \aparag The reinforcements of the \ligue are lowered by 1 \LD for each one
  of the following cities that have been conquered by the enemies (even if
  liberated later on): \villeVienne, \villeSalzburg and \villeMunich.
  \aparag{Placement: \alliance then \ligue}
  \bparag The reinforcements obtained are freely distributed among the
  countries part of the alliance. \AUS as a Major power buys its own
  reinforcements but may take up to 2 \LD from the \ligue as its own
  reinforcements.
  \bparag They can only be placed in provinces not pillaged, not controlled by
  the enemy and free of enemy forces.
  \bparag They have to be placed in a province of their nationality, or with
  at least one \LD of the same nationality if their country is nor completely
  occupied by the enemy.
  \aparag[Wallenstein] \HAB may hire mercenary general
  \leaderwithdata{Wallenstein}. He costs 40\ducats (payed by the controller of
  \ligue) to recruit him for one turn.
  \bparag If \leader{Wallenstein} is not hired at turn 1 or 2 of this war, he
  will not be available later. He can not be hired anew after the
  \xnameref{pIV:TYW:Peace Prague}.  The first time \leader{Wallenstein} is
  hired, he appears anywhere in a friendly province of \payshabsbourg or
  \paysBoheme with one Veteran \ARMY\faceplus (use an \AUS or \HRE counter).
  \bparag \leader{Wallenstein} can command any stack of the \ligue (including
  \HAB) but no Bavarian counter.
  \bparag If at the end of a turn the \STAB of the \ligue is positive or its
  situation favourable, \leader{Wallenstein} is dismissed.  He can be hired
  again on the round and/or turn after \ligue suffered a Major Defeat.
  \bparag \MAJHAB can assassinate \leader{Wallenstein} at any time.  He is
  eliminated and \ligue (and \AUT) gain immediately {\bf 1} in \STAB.
  \bparag After the \nameref{pIV:TYW:Peace Prague}, \leader{Wallenstein} is no
  more available.
  \aparag Three mercenary generals are available each turn of this war.  They
  can be recruited by the \ligue or the \alliance. A general is recruited for
  one turn only. He can lead any stack of the alliance (including allied
  \MAJ); by paying 10\ducats more, he can lead a stack even if there is a
  general with higher rank.


  \digression[pIV:TYW:Condition War]{General conditions of the war}

  \phmil
  \aparag Each alliance has a Simple Campaign available each round.  Major or
  Multiple Campaign could be paid for by the controller of the alliance (cost
  lowered by 20\ducats).
  \aparag Each alliance and their allies draw supply in the \HRE from any
  province controlled by their side that is not pillaged or that has an
  unblockaded port.
  \aparag Supply can be traced through any neutral province, or controlled
  province (pillaged or not).
  % (JCD): Neither \HOL nor \FRA nor \POL?
  \aparag Both alliances can freely cross any neutral \HRE minor countries ;
  this is also permitted to \DANdan, \SUE, \ENG in limited intervention, \HAB
  of course and \SPA but not to other allies.
  \aparag Alternatively, a side may, before its movement, declare war against
  any neutral country of the \HRE. Its forces are immediately deployed.
  \aparag All pillages of the \ligue and of the \alliance are decided by their
  controller and goes in their Treasury.
  \aparag A Major Victory involving forces of one or both alliances adjust the
  \STAB of this side accordingly of the usual rules.


  \digression[pIV:TYW:Winning War]{Who is winning the war ?}

  \phpaix
  \aparag No minor country of an alliance ever makes a regular peace (even
  unconditional) outside of the peace of its alliance.
  \aparag One side may be in favoured position depending on the military
  control of the following cities.
  \bparag The \alliance is awarded 2 points for the control of \villeVienne.
  \bparag One point is awarded for each of those: \villeSpeyer, \villePrague,
  \villeMunich, \villeFreiburg, \villeStrasbourg, \villeHannover, \villeKleve,
  \villeCassel, \villeMagdeburg, \villeBerlin, \villeDresden, \villeFrankfurt
  and \villeBrunswick
  \bparag \textonehalf point is awarded for each of these: \villeKoln,
  \villeStuttgart, \villeUlm, \villeMainz, \villeTrier, \villeHamburg,
  \villeMunster and \villeErfurt
  % (Pierre): may be add \villeWiesbaden
  \bparag A side has a favoured position of it has at least 3 points more than
  the other alliance.
  \aparag Both the \alliance and the \ligue lose each {\bf 2} \STAB.
  \aparag Then if a side is favoured, it gains {\bf 1} \STAB.
  \aparag \SPA, \HOL and \AUS lose {\bf 1} \STAB if they were not in the
  original war (in full intervention, not just a limited one) on the previous
  turn.
  % (even if it was a war that lasted since more than one turn
  % ; this war counts as one turn of the current one): their second turn
  % of war just ended.
  \aparag \SPA, \HOL, \AUS lose {\bf 2} \STAB if they were at war (full
  intervention) on the previous turn (even if it was a war that lasted since
  more than one turn ; this war counts as one turn of the current one): their
  second turn of war just ended.
\end{digressions}
\aparag[Result of the Danish Crusade]
\bparag If \DANdan wins a battle against at least 1 \ARMY\faceplus of the
\ligue (or its allies) in the \HRE, is never routed in battle and has forces
left in \HRE at the end of the turn, then its Crusade is successful.
\bparag Thus the \alliance gains {\bf 1} \STAB ; \DANmin is placed in \EG of
\ENG, annexes immediately \provinceLubeck and \provinceHolstein (or
\provinceMecklenburg if it owns already both) and will continue its
intervention until the end of the war, or when it signs any separate peace (in
this war or another). It will not receive reinforcements \emph{per se}, but
some can be given from those of the \alliance.
\bparag If the Danish Crusade failed, \DANmin makes a white peace and
withdraws from the war. \leader{Christian IV} remains as a Danish general for
the full 4 turns.

\tour{Turn 2 -- The Lion of the North (1629--1632)}

\phevnt
\aparag Check for a possible extension of each alliance, see
\ref{pIV:TYW:Extension Alliances}.
\aparag \SUE has to enter the war as an ally of the Protestant \alliance.  If
it is Catholic, roll for 2 \REVOLT in \SUE and it loses {\bf 1} \STAB ;
nothing happens if it is Protestant -- no \CB is necessary and this is not a
declaration of war.
\aparag[Military revolution] \SUE receives \monarque{Gustave Adolphe}. He is
due to last 7 turns.
\bparag If the current Monarch has 1 or 2 turns of life left,
\monarque{Gustave Adolphe} would be his heir. If \monarque{Gustave Adolphe}
dies (in battle) before the current Monarch, \SUE will use the columns 7 to
roll its next Monarch.
\bparag If the current Monarch has more than 2 turns left, \monarque{Gustave
  Adolphe} replaces him entirely and will last for the remaining of the 7
turns as a Monarch (but a death in battle).
\bparag \monarque{Gustave Adolphe} is a military genius, a general
\leaderwithdata{Gustav-Adolf}. As long as the war goes on for \SUE, it
benefits from a Military Revolution (see \ref{chExpenses:Military
  Revolutions})
\bparag[] [BLP] The moment \leader{Gustav-Adolf} dies (even in the middle
of a round), \SUE receives \leaderBaner for 3 turns. \leaderBaner replaced the
deceased king (replace one counter by the other).
\bparag[\leader{Sachsen-Weimar}]\label{pIV:TYW:Saxe-Weimar}
\leaderwithdata{Sachsen-Weimar} joins \SUE for 7 turns also.
\bparag If \monarque{Gustave Adolphe} dies, \FRA (if allied to \SUE) may hire
\leader{Saxe-Weimar} as a mercenary general to fight in the present war.  It
costs 30\ducats the first turn, then 20\ducats to keep \leader{Saxe-Weimar};
when \leader{Saxe-Weimar} is not paid one turn, he is eliminated (he does not
go back to \SUE). \leader{Saxe-Weimar} takes command of one German stack of
the \alliance when he goes to \FRA; at each following turns, \FRA can take
half (round down) of the reinforcements of the \alliance (up to 4\LD) to be
placed with \leader{Saxe-Weimar}.  If he dies the forces go back to normal
status in the \alliance.

\aparag \FRA, if Protestant, can begin a limited intervention in the war on
the side of the \alliance.

\aparag Any \MAJ that was doing a limited intervention during the previous
turn (as defined in the original war) can continue this limited intervention
to help the same side.

\aparag \xnameref{pIV:TYW:Turkish Frontier} is in effect this turn.

\phadm
\aparag Roll for reinforcements as in the first turn, see
\xnameref{pIV:TYW:German Reinforcements}.

\phmil
\aparag The war is conducted according to \xnameref{pIV:TYW:Condition War}.
\aparag \SUE takes the control of the forces of one minor country of the
\alliance (its choice). This country can change from one turn to the other and
is chosen at the beginning of any military round of the turn.
\aparag \SUE may force a minor country to enter the war in the \alliance if it
is one of the countries that could join the \alliance and \SUE has at least 1
\ARMY\faceplus and \monarque{Gustave Adolphe} in a province of the country.
\aparag If \SUE makes a siege of allied or neutral \provinceMecklenburg,
\province{Ost Pommern} or \province{West Pommern} with at least one \ARMY
\faceplus, then the city surrenders without fighting at the end of the round.
\aparag All cities taken (by siege, assault or automatic surrender) with at
least one Swedish \ARMY, or only Swedish troops, have now Swedish garrisons
(and the town counts as Swedish presence in the \HRE).  Other Major powers put
their garrison if the city is taken with only their own forces (else, German
garrisons are in charge).

\phpaix
\aparag The balance of the war is checked as in \xnameref{pIV:TYW:Winning
  War}.  The losses of \STAB are applied except that now there is one turn
more:
\bparag Both the \alliance and the \ligue lose each {\bf 3} \STAB.
\bparag Then if a side is favoured, it gains {\bf 2} \STAB.
\bparag Any Major Power in its second turn of war lose {\bf 2} \STAB.
\bparag \SPA, \HOL, \AUS lose {\bf 3} \STAB if they are in their third turn of
war.
\bparag \SUE and \ENG if continuing their intervention lose {\bf 1} \STAB.
\aparag[Suing for peace]\label{pIV:TYW:Suing turn 2}
\bparag A German alliance sues for the \xnameref{pIV:TYW:Peace Prague} when it
is at \bonus{-3} in \STAB at the end of two consecutive turns, and the
position in the \HRE is not in its favour. The enemy side grants necessarily
this peace.
\bparag If both alliances are at \bonus{-3} in \STAB at the end of any turn,
their controllers can agree to a Status Quo and sign the
\nameref{pIV:TYW:Peace Prague}.

\bparag When the \nameref{pIV:TYW:Peace Prague} is signed, the German
alliances are partly dissolved; their stability will not be recorded further
and most of the minor countries in these alliances make a peace.  The
alliances want to stop the war and sign a peace so, from now on, all foreign
countries have no constraint to sign peaces also. It would not be a separate
peace from the German alliance point of view (but could be from another
country...)

\bparag However, if some Major Powers want to keep fighting in the \HRE and
refuse to sign the \nameref{pIV:TYW:Peace Prague}, see \ref{pIV:TYW:War after
  Prague}. Keeping fighting means that the Major power does not sign treaty of
peace with every enemy (that are \MAJ, the enemy German alliance, and possibly
\HOLmin and \DANmin); moreover this country is not allowed to sign a Truce
next turn. \AUSMin signs or not the \nameref{pIV:TYW:Peace Prague} alongside
\SPA.
\bparag If no Major Power contests the \nameref{pIV:TYW:Peace Prague} by
continuing the fight, apply now the \xnameref{pIV:TYW:Peace Westphalie}.

\tour{Turn 3 (1632--1636) and after: a Foreign War}
\history{Turn 4: 1637--1641 (first turn after the Peace of Prague); Turn 5:
  1642--1648 (from Rocroi and Jankov to Lens); Turn 6: 1648-1654 (La Fronde);
  Turn 7: 1654--1660.}

\phevnt
\aparag Check for a possible extension of each alliance, see
\xnameref{pIV:TYW:Extension Alliances}.
\aparag No limited intervention of the previous turn can be carried on.
\aparag At any turn, \FRA and \ENG can enter the war as an ally of the side
they chose. They have a \CB against a side which has not their Religious
Stand, and none against an alliance having the same Religious Attitude; the
\alliance is Protestant and the \ligue is \CATHCR.
\aparag At any turn, \POL (unless it is Orthodox) can make a full or limited
intervention in the war as an ally of any side. \POL can do such an
intervention only once during the war. It has a \CB only against an alliance
that has not the exact same Religious Attitude (relative to Catholicism) as
itself.

\phadm
\aparag Roll for reinforcements as in the first turn, see
\xnameref{pIV:TYW:German Reinforcements}.
\aparag Two turns after a Military Revolution caused by \SUE, the Land
Technology of the Latin minor countries reaches this new Technology.

\phmil
\aparag The war is conducted according to \xnameref{pIV:TYW:Condition War}.
\aparag \SUE takes the control of the forces of one minor country of the
\alliance (its choice). This country can change from one turn to the other and
is chosen at the beginning of any military round of the turn.
\aparag On the third turn only (not after), if \SUE makes a siege of allied or
neutral \provinceMecklenburg, \province{Ost Pommern} or \province{West
  Pommern} with at least one \ARMY \faceplus, then the city surrenders without
fighting at the end of the round.
\aparag All cities taken (by siege, assault or automatic surrender) with at
least one Swedish \ARMY, or only Swedish troops, have now Swedish garrisons
(and the town counts as Swedish presence in the \HRE).  Other Major powers put
their garrison if the city is taken with only their own forces (else, German
garrisons are in charge).

\phpaix
\aparag The balance of the war is checked as in \xnameref{pIV:TYW:Winning
  War}.  The losses of \STAB are applied with one turn more. This war can not
cause a loss more than {\bf 4} \STAB at the end of turn.  On turn 3 of the
Religious War, the losses should be:
\bparag the \alliance and the \ligue lose {\bf 4} \STAB;
\bparag any Major Power in its third turn of war lose {\bf 3} \STAB.
\bparag \SPA, \HOL, \AUS lose {\bf 4} \STAB if they were at war before the
Religious War in the \HRE.
\bparag \SUE loses {\bf 2} \STAB.
\bparag Any other Major Power intervening in the war at this turn lose {\bf 1}
\STAB.
\aparag[Suing for peace] As described in \ref{pIV:TYW:Suing turn 2}.
\aparag If \SUE, \ENG or \POL (in full intervention) do not hold any city nor
have any land forces left in the \HRE, they make a mandatory white peace
against all its enemies in this war. This will count as a losing position in
\xnameref{pIV:TYW:Peace Westphalie}.
\aparag If \POL is doing a limited intervention and wins a battle against at
least one \ARMY\faceplus of the enemy side (any nationality) in the \HRE, then
loses no battle in the \HRE, the alliance it helps gains {\bf 1} in \STAB
(\AUS also). \POL may then annex \province{Ost Pommern} or any province in the
\HRE that once was Polish. Its limited intervention lasts only one turn.

\begin{digressions}[Between Prague and Westphalie]


  \digression[pIV:TYW:Peace Prague]{Peace of Prague}
  \aparag If the \ligue is favoured by the Peace:
  \bparag \paysBaviere gains permanently its second \ARMY and \paysPalatinat
  loses its own; \paysBaviere is now an Electorate.  It also gains a permanent
  \bonus{+1} to its reinforcement rolls.
  \bparag \paysBaviere annexes \provinceOberPfalz, except if this war follows
  \xnameref{pII:Schmalkaldic League}, in which case it annexes
  \provinceSchwaben.
  \bparag \paysBaviere is now in \AM with \HAB (move its diplomatic marker
  accordingly).
  \bparag A Total Victory of the \ligue in the \xnameref{pIV:TYW:Peace
    Westphalie} is possible.
  \bparag Any specific consequence given by the victory of the side of the
  \ligue in the war having caused \ref{pIV:TYW} is applied.
  \bparag The Truce of Augsburg is revoked.
  \bparag \SPA and \AUS gain 30 \PV, \SUE loses 10 \PV.
  \aparag If the Peace is a Status Quo:
  \bparag \paysBaviere keeps its second army for the continuation of this war
  (but not permanently).
  \bparag The Truce of Augsburg is in effect.
  \bparag No side can achieve Total Victory in the \xnameref{pIV:TYW:Peace
    Westphalie}.
  \aparag If the \alliance is favoured by the Peace:
  \bparag The Truce of Augsburg is in effect.
  \bparag A \xnameref{pIV:TYW:Northern HRE Alliance} is created and allied to
  \HOL.
  \bparag \paysOldenburg, \paysHanovre, \paysHesse, \paysHanse and \paysBerg
  are placed in \EG of \HOL.
  \bparag A Total Victory of the \alliance is now possible.
  \bparag \HOL and \SUE gain 30 \PV.


  \digression[pIV:TYW:War after Prague]{The War after Prague}
  \aparag Only some minor countries continue the war. All other minor
  countries of the \HRE surrender: their forces are withdrawn and their cities
  are considered as taken for the reinforcements.
  \bparag On the side of the \ligue: \HAB and, if the Peace is not in favour
  of the \alliance, \paysBaviere.
  \bparag On the side of the \alliance: the controller is now \SUE and it
  chooses 2 countries, or only 1 in case of unfavourable Peace, that will
  continue the fight from the following list: \paysHesse, \paysHanovre,
  \paysPalatinat, \paysSaxe.
  \bparag If the Peace is favourable to the \ligue, \paysSaxe reverses its
  alliance and enters war with the Catholics.  All its forces are withdrawn
  from the map, and the cities of \paysSaxe surrender immediately to the
  Catholics; Protestant forces in the provinces are withdrawn.
  \bparag \paysBrandebourg will continue (or enter) the war as an ally of the
  Protestant if \SUE gives up its claims on \province{West Pommern} to
  \paysBrandebourg in \xnameref{pIV:TYW:Peace Westphalie}.
  \bparag If \FRA hires \leader{Saxe-Weimar} at this turn (continuing from a
  previous turn or not), he keeps one stack of any one protestant
  country. This country remains at war (until it surrendered unconditionally
  or \leader{Saxe-Weimar} is no more at the service of \FRA). It will receive
  reinforcements for this stack (using the mechanism for the stack of
  \leader{Saxe-Weimar}).
  \aparag The minor countries that continue the war are allied in their
  alliance, and with the Major countries in the war. But they want peace so
  they will stop fighting as soon as all foreign minor/major countries do
  likewise.
  \bparag A minor country of the \HRE can now be ejected from its alliance and
  from the war, but only by imposing an unconditional surrender on it; other
  regular peaces are not possible.
  % (Jym): theta-N.1.b systematically gives control to SUE. Commenting out.
  % \bparag If the \nameref{pIV:TYW:Peace Prague} was not favourable to the
  % \alliance, the controller of the \alliance is now \SUE in priority
  % (whatever its Religion is).%, then \HOL (and so on).
  \aparag All other minor countries that were in both alliances are now at
  peace; they all have now a Neutral diplomatic status.  All the cities in
  those countries are considered conquered in order to check for
  reinforcements.
  \aparag Foreign minor country \DANmin stops the war whereas \HOLmin
  continues. A regular peace has to be obtained against it.
  \aparag Do not forget that this war causes at most a loss of {\bf 4} \STAB
  for each country at the end of turn.  If the War caused by the Revolt of the
  United Provinces continue, it resumes its normal loss in \STAB only if an
  Armistice is made (at least 1 turn) between \SPA and \HOL at the end of the
  present war; else the present war has to continue and so does the loss of
  {\bf 4} \STAB each turn.


  \digression[pIV:TYW:Peace Westphalie]{Peace of Westphalie}
  \aparag This Peace is signed at the end of a turn, beginning with the turn
  of the \nameref{pIV:TYW:Peace Prague}, if all Major countries in the war
  agree to end the war, that is to sign Peaces or Armistices between them. The
  following effects are implemented as further consequences of the regular
  Peace Treaties.
  \aparag The Emperor of the \HRE is now \HAB if this was not, for the rest of
  the game.
  \aparag The Major Countries that can be involved in the war are \SPA (and
  \AUSmin), \AUS, \FRA, \HOL, \SUE, \ENG and \POL.
  \bparag A Major Power that stops the war (it has signed Peaces or Armistices
  with all other Major Powers at the end of some turn) before the end has a
  losing position for this Peace; it has also this position if it signs a
  mandatory white peace (for any reason).
  \bparag A Major Power has a dominant position if it signs only winning
  Treaties of Peace with countries of the other side (no Armistices or White
  Peaces either) on the last turn of this war.
  \bparag A Major Power has a losing position if it signs only losing Treaties
  of Peace with countries of the other side (no Armistices or White Peaces
  either) on the last turn of this war.
  \bparag In other cases, the position is neutral.
  \aparag[Spain or Austria]
  \bparag These specific conditions are for \MAJHAB.
  \bparag A \AUSmin will continue to fight with \SPA until the end of the war
  (except by unconditional surrender, following the rules for all minor
  countries from the \HRE still at war after the \nameref{pIV:TYW:Peace
    Prague}).
  \bparag If \HAB is in dominant position and a Catholic Total Victory was
  possible, the \pays{German Empire} is created (see \shortref{pIV:TYW:German
    Empire}).
  \bparag If \HAB is in dominant position but no Catholic Total Victory was
  possible, a \xnameref{pIV:TYW:Southern HRE Alliance} is associated to \HAB.
  The countries in this alliance are put in \EG of \HAB: \paysBaviere,
  \paysTreves, \paysAlsace, \paysBade and \paysWurtemberg.
  \bparag \HAB in neutral or losing position: nothing more.
  \aparag[Spain] If \SPA is in dominant position and there is a Major \AUS,
  the \xnameref{pV:WoSS} will only concern provinces in the Low Countries and
  northern Italy (those in southern Italy are not part of the Inheritance and
  remain Spanish).
  \aparag[Austria] If \AUS is in neutral position, it gains a permanent {\bf
    +1} bonus in Diplomacy on Catholic countries of the \HRE.
  \aparag[The Netherlands]
  \bparag If \HOLhol has a dominant position and a Protestant Total Victory is
  possible, \paysHanse annexes \provinceOldenburg and \HOL gains \paysHanse as
  a permanent \VASSAL.  Eliminating the \xnameref{pIV:TYW:Northern HRE
    Alliance} will now need a Peace of level 5 against \HOL.
  \bparag If \HOLhol has a dominant position (but without possible Protestant
  Total Victory), it gains \paysHanse as a normal \VASSAL and \paysHanse
  annexes \provinceOldenburg. The \xnameref{pIV:TYW:Northern HRE Alliance} is
  created and allied to \HOLhol with the corresponding effects.
  \bparag If \HOL has a neutral position, it has the choice to allow or not to
  the destruction of \paysHanse (its controller in the case of a \HOLmin).
  \bparag Else, if \HOL (or minor \payshollande) is in losing position, the
  \paysHanse is destroyed and the \xnameref{pIV:TYW:Northern HRE Alliance} is
  dissolved.
  \aparag[Sweden]
  \bparag If \SUE has a dominant position, it annexes \provinceMecklenburg,
  then \province{West Pommern} if it has not renounced its claims on this
  province (else it gains \paysBrandebourg in \EG) and \provinceBremen or
  \provinceLubeck (its choice). It then chooses one Protestant minor country
  (or 3 minor countries if a Protestant Total Victory was possible) of the
  \HRE that is (are) placed in \EG on its Diplomatic chart.
  \bparag If \SUE is in neutral position, it annexes \provinceMecklenburg,
  then \province{West Pommern} if it has not renounced its claims on this
  province; else it gains \paysBrandebourg in \EG. It then chooses one
  Protestant minor country of the \HRE that is placed in \EG on its Diplomatic
  chart.
  \bparag If \SUE is in losing position, it gains nothing.
  \aparag[France]
  \bparag If \FRA is in dominant position, it gains a \bonus{+1} bonus for
  Diplomacy on countries of the \HRE until the end of the period.
  \bparag If \FRA is in dominant or neutral position, it gains \paysAlsace as
  a \VASSAL and \paysCologne in \EC.
  \aparag[England] If \ENG is in dominant position, it gains a \bonus{+1}
  bonus for Diplomacy on countries of the \HRE until the end of period V.  It
  also gains a minor country of its choice, having the same religion as \ENG,
  that is placed in \EG on its chart.
  \aparag[Poland] If \POL is in dominant position after a full intervention,
  it gains a \bonus{+1} bonus for Diplomacy on countries of the \HRE until the
  end of period V.  It also gains a minor country of its choice, having the
  same religion as \POL, that is placed in \EG on its chart.
  \aparag When a major country can take a the diplomatic control of a minor
  country, the order of choice is the order written here, and a power can only
  choose neutral minor country of the \HRE (not those already allied to
  someone else).
  \aparag \paysBrandebourg annexes \province{Ost Pommern} if it is in
  \paysHanse.
  \aparag Then, if \paysHanse has to be destroyed, its remaining provinces are
  now given as follows: \SUE takes \provinceBremen, \paysBrandebourg takes
  \province{West Pommern} and \provinceMecklenburg, then \DANmin all the
  remaining ones.
  \aparag From now on, any major power that owns a province in \HRE or
  adjacent to a province of the \HRE may, when at war, enter and remain in any
  neutral province of the \HRE. The cost in \MP is the same as enemy
  territory. The neutral provinces can not be pillaged, besieged nor give
  supply (but supply lines can cross those if there are no enemy force
  within).
  \bparag
  \aparag[Victory Points]
  \bparag A Major Power in dominant position at the end of the war wins 30 \PV
  (added to those of the treaties of Peace).
  \bparag A Major Power in losing position at the end of the war loses 30 \PV.
\end{digressions}

\begin{digressions}[German alliances emerging from the war]


  \digression[pIV:TYW:Northern HRE Alliance]{Northern \HRE Alliance}

  \effetlong
  \aparag When this alliance exists, it is allied to \HOLhol.  It represents
  treaties between \paysOldenburg, \paysHanovre, \paysHesse, \paysHanse and
  \paysBerg.
  \aparag \HOL has a permanent bonus of \bonus{+2} in Diplomacy on these
  countries.
  \aparag \HOL gains also a income of 10\ducats for each coastal city in
  \paysHanse if it is on his diplomatic track.
  \bparag This Northern alliance is dissolved when \HOL signs a losing Peace
  of level 3 or higher, or when it controls no country of the alliance. The
  bonuses are permanently lost.


  \digression[pIV:TYW:Southern HRE Alliance]{Southern \HRE Alliance}

  \effetlong
  \aparag A Southern \HRE alliance is associated to \HAB, composed by the
  following countries: \paysBaviere, \paysMayence, \paysAlsace, \paysBade and
  \paysWurtemberg.
  \aparag Each of these countries on the \HAB or \MAJHAB diplomatic chart will
  give an income of 10\ducats to \MAJHAB.
  \aparag \MAJHAB gains a \bonus{+1} bonus in Diplomacy on every Catholic
  countries in the \HRE.
  \aparag This Southern alliance is dissolved when \MAJHAB signs a losing
  Peace of level 3 or more, or when neither \MAJHAB nor \HAB controls any
  country of the Alliance.  The bonuses are permanently lost.
  \aparag When a \GE is created, the Southern alliance is also dissolved (and
  becomes part of the \GE).


  \digression[pIV:TYW:German Empire]{German Empire}

  \effetlong
  \aparag All minor countries of the \HRE (except \HAB which remains
  independent) are associated in one minor country, called the \pays{German
    Empire}. This country is a permanent \VASSAL of \MAJHAB. It can use 4
  \ARMY counters, and 12 \LD (for practical ease, use the counter of the \HRE
  and any counter of some part of the empire, with no notion of nationality --
  there are all from the \GE).  Its basic forces are one \ARMY\faceplus and
  one \ARMY\facemoins. It has a modifier of \bonus{+2} for reinforcements and
  always makes peace with \MAJHAB.
  \aparag \MAJHAB receives an income of 100\ducats from the \HRE (and not the
  exact value of the country) and can use its port on the Baltic Sea.
  \aparag When the \pays{German Empire} exists, the Dynastic Alliance between
  \AUSmin and \SPA is both defensive and offensive.
  \aparag Some events may dissolve part of the \pays{German Empire} by
  creating a League (\xnameref{pII:Schmalkaldic League}, \xnameref{pIII:League
    Nassau}, \xnameref{pIV:Bohemian Revolt}, \xnameref{pIV:Augsburg
    Revocation}, \xnameref{pIV:Unity HRE}) which ceases to be in the Empire,
  and is (depending on the event) at war with the Emperor. An unconditional
  peace of the Emperor on any of those countries bring it back in \pays{German
    Empire}.
  \aparag Event \xref{pV:Kingdom Prussia} liberates \paysBrandebourg from
  \pays{German Empire} (and it can't be forced back in).
  \aparag When any province with a capital of \pays{German Empire} is lost as
  the result of a Peace, the minor country having this capital is renewed as a
  free country, having status \EG or \VASSAL (if possible) with the \MAJ that
  liberated it (player's choice). \HAB can force the \MIN back in the
  \pays{German Empire} by means of an unconditional peace on it.
  % (Jym) For the Hansa, is it enough to free one capital?
  \aparag Some events (\xnameref{pIV:Augsburg Revocation}, \xnameref{pIV:Unity
    HRE} and \xnameref{pV:Devolution War}) can cause Civil War in \pays{German
    Empire} that foreign countries can help in order to dissolve \pays{German
    Empire}.
  \aparag The \xnameref{pV:WoSS} may separate the Spanish dynasty from the
  Austrian dynasty because of a Crisis of Succession.
  \bparag If \SPA chooses a \AUSmin Heir, the \pays{German Empire} fights
  along their side with no Dynastic Separation.
  \bparag If \SPA chooses another Heir than a \AUSmin, the \pays{German
    Empire} is dissolved but \paysBaviere, \paysMayence, \paysLorraine,
  \paysBade and \paysWurtemberg are placed in \AM of \HAB and enters war at
  its side; and \HAB gains the benefits of \xnameref{pIV:TYW:Southern HRE
    Alliance}. All other countries that are recreated at this time are
  Neutral.
  \bparag \AUS (if major) keeps the \pays{German Empire}.
  \bparag See the other conditions in this event.
  \aparag The \pays{German Empire} ceases to exist as soon as its controller
  is forced to sign any peace of level 3 or more.  In addition to the normal
  peace conditions, \pays{German Empire} is dissolved: all minor countries of
  the HRE are back to previous frontiers, and are neutral.
\end{digressions}

% Local Variables:
% fill-column: 78
% coding: utf-8-unix
% mode-require-final-newline: t
% mode: flyspell
% ispell-local-dictionary: "british"
% End:

% LocalWords: defensive se Hansa offensive pIII FWR Schmalkaldic pIV TYW pII
% LocalWords: Ost Pommern HRE Westphalie unblockaded JCD Sachsen Weimar Quo
% LocalWords: pV WoSS Jym Rocroi Jankov
