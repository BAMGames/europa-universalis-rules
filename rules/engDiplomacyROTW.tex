\section{Diplomacy with non-European countries}

\subsection{Diplomacy status in \ROTW} 
\aparag The following minor countries are on \ROTW map.  \paysInca,
\paysAzteque, \paysGujarat, \paysVijayanagar, \paysMogol, \paysChine,
\paysJapon, \paysSiberie, \paysOman, \paysAden, \paysSoudan,
\paysMysore, \paysHyderabad, \paysIroquois, \paysAfghans, \paysOrmus (a
special part of \paysPerse).  The relations between European Major
Powers and those countries are governed by different diplomatic rules.
\aparag[Generalities]
A Major Power has a specific status regarding  each one of those countries:
\begin{deflist}
\listingabbrev{dipNR}{No relation}
\listingabbrev{dipFR}{Formal relation}
\listingabbrev{dipAT}{Alliance Treaty}
\end{deflist}
\bparag \dipNR is not recorded; 
\bparag \dipFR and \dipAT are recorded by placing a \ROTW diplomatic
counter of the Major Power in the diplomatic status box of the relevant
minor country, that is found on the \ROTW map, on the side showing
\dipFR or \dipAT as needed.
\bparag Note that the number of \ROTW Diplomatic counters provided to
each \MAJ is limited by design. A Major Power may always decide to lose
a relation in order to free a needed counter. Each counter allow for one
\dipFR (front) or one \dipAT (back).%  Limits by Major Power: \POR: 6
%counters ; \FRA, \HOL and \ENG : 4 counters ; \SPA, \TUR, \RUS, \SUE,
%\POL and \VEN: 2 counters.
\aparag Diplomatic status is achieved by doing diplomatic actions, as
described in the \ruleref{chDiplo:Diplomatic Actions}. A diplomatic action
on a country in the \ROTW counts as one of the allowed actions, but it
is resolved differently.



\subsection{Diplomatic actions in the ROTW}
\aparag[Conditions to attempt actions.]
In order to attempt a diplomatic action on a \ROTW minor country,
a Major Power needs to have discovered at least one province of the
minor country, and needs to
\bparag either have a \TP/\COL in an area owned by, or adjacent to the
country, or adjacent to the same seazone,
\bparag or have a Commercial fleet in a seazone bordering that country,
\bparag or have an emissary in the minor country at the diplomatic
phase,
\bparag or be \TUR attempting action on \pays{Oman}, \pays{Aden} or
\pays{Soudan},
\bparag or be \VEN after \subeventref{pI:WRS:War Indian Sea}, attempting
action on \pays{Aden}, \pays{Oman} and \pays{Gujerat}
\bparag No diplomatic action is allowed if the power is fully at war
against the minor country of the \ROTW.

\aparag[Emissaries]
An emissary is a Conquistador (or an Explorator used as a C, with values
divided by 2), a Governor, a Missionary, or a Mission. To be helpful, an
emissary has to be in the target minor country, or in an adjacent
region, or in a province bordering the same (discovered) sea as the
minor country.

\aparag[Resolution of diplomatic actions in \ROTW]
The result of the action is always given by the difference between 1d10
rolled by the \MAJ (plus bonuses below) and the resistance given by the
sum of 2d10.
\bparag as for actions on European minors, the actions (and final bonus)
has to be written on the monarch sheet and the cost is recorded on
\lignebudget{Diplomatic actions}.
\diplorotw

\bparag An adjusted roll strictly higher than the resistance (2d10) plus
one raises the diplomatic status of one level (from \dipNR to \dipFR, or
from \dipFR to \dipAT), or of two if the difference is 5 or higher (all
the way to \dipAT).
\bparag Going to a higher level of relations is always voluntary and can
be declined.
\bparag More than one power can make a diplomatic action on a country in
the \ROTW at the same time. The attempts are not in opposition. Several
major countries may have \dipFR or \dipAT with the same minor at the
same time.
\bparag An adjusted roll less or equal to the resistance causes nothing,
except that a Missionary that served as an Emissary is killed (and may
come back afterwards).

\aparag[Reaction]
Any \MAJ sharing an \dipAT with the \MIN has the opportunity to react.
It uses the same condition and modifiers as diplomatic action in \ROTW.
As a reaction, the \MAJ pays the action (according to the investment),
this is recorded in \lignebudget{Diplomatic reactions}, but the action
is not counted as one of its own at this turn. If the roll of the
reacting player is higher than the resistance (sum of 2d10), the result
of the action is given by the comparison with his roll.
% Jym: useless because one needs an AT to react already...
%The
%reacting player may not augment his own status as a result of the
%reaction.

\aparag[Opposing to other countries' relations.]
A Diplomatic action may be aimed at diminishing the diplomatic relations
of some or all Major Powers with the minor. This counts as one
diplomatic attempt and is allowed provided the power satisfies the
conditions to make diplomacy on this minor country.  The opposed \MAJ(s)
is/are announced before the action and they defend their status as
usual, by paying the cost of a Diplomatic action (that is not counted as
one of their permitted actions for the turn), if they have no action
planned.
\bparag Both opposing \MAJ make a roll of 1d10, modified as above.  If
the acting \MAJ obtains a higher roll than an other \MAJ opposing the
action, the result is that this \MAJ lowers its diplomatic status of one
level (from \dipAT to \dipFR, from \dipFR to \dipNR).

\begin{exemple}[Diplomatic action]
  During turn 2, \leader{Da Gama} lands in India and stays inside the
  territory of \paysVijayanagar at the end of turn. Thus, he may act as
  an emissary during the diplomatic phase of turn 3. The special \FTI
  for \ROTW of \POR is 5 and the player chooses to make a small
  investment only. Thus, the final bonus is \bonus{+11} (\bonus{+6} for
  the Manoeuvre and \bonus{+5} for the \FTI) which is already rather
  good\ldots

  \POR rolls 6, for a total of 17 while the minor rolls 4 and 8 for a
  total of 12. The difference between the two is 5 which is enough to go
  directly to \dipAT. \POR now has to pay 2d10\ducats as presents to the
  local Rajahs (see below).
\end{exemple}

\begin{exemple}[Diplomatic reaction]
  At turn 8, \TUR manages to get an \dipAT with \paysAden, allowing it
  to get part of the spice trade. Since \ref{pI:WRS:War Indian Sea}
  occurred earlier, \VEN, always eager to get more hold on the spice
  trade, attempts some diplomacy on \paysAden and \TUR decides to
  react. None of them has emissary in the country. The \FTI are 4 for
  \VEN and 3 for \TUR. \VEN chooses to make a medium investment for a
  final bonus of \bonus{+4} (\FTI, \bonus{+2} for the investment but
  \bonus{-2} for the religious difference) while \TUR only reacts with a
  small investment for a final bonus of \bonus{+5} (\FTI, \bonus{+2} for
  being both Muslims).

  \VEN rolls 8, for a total of 12. \paysAden rolls 3 and 2 for a total
  on 5. If the action was not opposed, this would be enough to get an
  \dipAT! However, \TUR rolls 5, for a total of 10. Thus, the Turkish
  roll is taken into account rather than the minor one and \VEN only
  gets a difference of 2. Still enough to go to \dipFR.
\end{exemple}

\begin{exemple}[Hampering another status]
  It is turn 53 (1750). Both \FRA and \ANG have an \dipAT with
  \paysMysore. Sensing that colonial tensions may arise in a state of
  war sooner or later, the East Indian Company decides to play on the
  intra-indian struggles and sends \leaderClive in a attempt to convince
  \paysMysore to break its alliance with \FRA. The \emph{Compagnie des
    Indes Orientales} learns about it and quickly sends \leaderDupleix
  to try and counter the English deeds.

  \ANG makes an action on \paysMysore, specifically to lower the
  relation with \FRA, with a \FTI of 5, a manoeuvre of 4 for
  \leaderClive and a medium investment, thus getting a final modifier of
  \bonus{+11}. \FRA also has a \FTI of 5 and a manoeuvre of 4 for
  \leaderDupleix but only reacts with a small investment (after all,
  India can't be more important than the sugar Islands of the Caribbean,
  says the King) for a final modifier of \bonus{+11} (\bonus{+2} for
  defending its \dipAT).

  \ANG rolls 8, for a final result of 19 while \FRA rolls 7, for a final
  result of 18. Since the English result is higher, the diplomatic
  status of \FRA is lowered by one level and goes to \dipFR.
\end{exemple}

\subsection{Consequences of "Formal Relations"}
\aparag In the provinces of the minor country, neither Native Activation
(during each round), nor reaction of the \MIN due to the presence of
military forces, will be made if only stacks of one \LD would be
responsible of the test.
\bparag The presence of more than one \LD in any one province, or of an
\ARMY may still cause such activation.
\aparag Any war (normal or overseas) between the powers and the minor
country will break the status to \dipNR. Native reaction in a province
of the minor country is not a war and changes nothing.

\subsection{Consequences of an "Alliance Treaty"}
\subsubsection{Generalities}
\aparag For \paysInca, \paysAzteque, \paysGujarat, \paysMogol,
\paysChine, \paysJapon and \paysAfghans, the \MAJ has to pay 1d100
\ducats immediately, or the status remains \dipFR only.
\aparag For \paysVijayanagar, \paysSiberie, \paysOman, \paysAden,
\paysSoudan, \paysMysore, \paysHyderabad, \paysIroquois, and \paysOrmus,
the \MAJ has to pay 2d10\ducats immediately, or the status remains
\dipFR only.
\aparag The effect of \dipFR on lone \LD is still applied.
\aparag
Supplementary effects vary according to each \MIN.
\aparag Having an \dipAT is analogue to a \VASSAL status for Victory
Conditions.

\subsubsection{\paysJapon and \paysChina}\label{chDiplo:Diplo-japon}\label{chDiplo:Diplo-chine}
\aparag The \MAJ can have a \TP in each area of the minor country that
will not cause a test of reaction of the Native country at the beginning
of the turn.

\aparag[Closure of China or Japon.] Events \ref{pIII:CCA:Closure China}
and \ref{pIV:JCA:Closure Japan} close respectively \paysChine and
\paysJapon for the following effects:
\bparag The reaction level of the country is raised to 11 (so a reaction
is automatic if the conditions are met); the fidelity is raised to 16.
\bparag The country refuses any diplomacy, except as detailed
afterwards; existing diplomatic status remain so (and other powers are
forbidden to try opposing existing relations);
\bparag \dipAT allow each country to keep only one \TP in \paysChine or
\paysJapon, and not one per area (that \TP causes no reaction of the
minor country);
\bparag No new \TP counter can be placed in any area belonging to the
country, by means of administrative actions, except in
\granderegionFormose or \granderegionCoree;
\bparag The only way to have a new \TP is to take control of the \TP of
another country (by military means, and a peace, or by placing a new \TP
in the same province and using automatic concurrence to try to replace
the existing \TP) in which case the Treaty status is given to the new
controller of the \TP and lost by the previous one.%  Another way is to
% force a Treaty on the minor by means of a victorious war against it.
\bparag New areas that would be conquered later by \paysChine or
\paysJapon would suffer from the same restrictions, but existing \TP or
\COL are not destroyed immediately (unless the event says so).
Moreover, for the new areas controlled, the Activation level is 6 only
(and not automatic). In these area, it is possible to create new \TP by
administrative action, but the rest of the restrictions apply.
\aparag[Treaty of Nerchinsk.] Event \ref{pV:Treaty Nerchinsk} results in
the annexion by \paysChine of area \granderegionAmour, and some
provinces in \granderegionBaikal.
\bparag The Activation level of \paysChine is 6 herein.
\bparag Powers having a \COL/\TP in this area are allowed to attempt
diplomatic actions on \paysChine. If they manage an \dipAT status, they
can have and keep up to 2 \COL/\TP in \granderegionAmour, or (exclusive)
keep one existing in the rest of \paysChine (as per the previous rule ;
note that such a \TP can not be created) that will not cause reaction of
the minor.

\subsubsection{\sectionpays{Vijayanagar}}\label{chDiplo:Diplo-vijayanagar}
\aparag \pays{Vijayanagar} will not react to the presence of \TP in its
provinces. It will react to the presence of \COL.
\bparag Exception: with an \dipAT of \POR, \pays{Vijayanagar} will never
to the presence of a Portuguese \COL in its territory.
\aparag Neither \pays{Vijayanagar} nor natives in its territory will
react to the presence (movements or remaining) of stacks of at most one
\ARMY\faceplus in its territories.

\subsubsection{\sectionpays{Mogol}, \sectionpays{Siberie}, \sectionpays{Soudan}, \sectionpays{Afghans}}
\label{chDiplo:Diplo-mogol}
\label{chDiplo:Diplo-siberie}
\label{chDiplo:Diplo-soudan}
\label{chDiplo:Diplo-afghans}
\aparag The concerned minor country will not react to the presence of
\TP\facemoins in its provinces. It will react to the presence of \COL or
of \TP\faceplus.
\bparag Exception 1: with a Treaty, \paysAfghans will not react to the
presence of a \COL in \provinceHerat.
\bparag Exception 2: with a Treaty, \paysSoudan will not react to the
presence of \COL of \TUR.
\bparag Exception 3: with a Treaty, \paysMogol will never react to the
presence of a Portuguese \COL in its territory.
\aparag Neither the minor country nor natives in its territory will
react to the presence (movements or staying there) of stacks of at most
one \ARMY\faceplus in its territories.

\subsubsection{\paysGujerat, \paysOman, \paysAden,
  \paysaceh}\label{chDiplo:Diplo-aden}\label{chDiplo:Diplo-oman}\label{chDiplo:Diplo-gujarat}\label{chDiplo:Diplo-aceh}
\aparag Neither the minor country nor natives in its territory will
react to the presence (movements or staying there) of stacks of at most
one \ARMY\faceplus in its territories.
\aparag \label{chDiplo:AdenOmanExoticResources} If there is only one
power having \dipAT with the country, the resources produced by the
\TP/\COL of the minor country are given to this power (it gains the
income and count those resources as its own to obtain a monopoly).
\aparag The minor country can be used as an ally in wars.
\aparag They do not react to \COL of \TUR, except \paysaceh.
\aparag \paysOman controls \provinceSocotra if no power has an
establishment (fort, \TP or \COL) in the province.

\subsubsection{\sectionpays{Mysore}, \sectionpays{Hyderabad}}\label{chDiplo:Diplo-mysore}\label{chDiplo:Diplo-hyderabad}
\aparag The minor country will not react to the presence of \TP in its
provinces. It will react to the presence of \COL.
\aparag Neither the minor country nor natives in its territory will
react to the presence (movements or staying there) of stacks of at most
one \ARMY\faceplus in its territories.
\aparag The minor country can be used as an ally in wars.

\subsubsection{\sectionpays{Iroquois}}\label{chDiplo:Diplo-iroquois}
\aparag \pays{Iroquois} will not react to the presence of \TP\facemoins
in its provinces. It will react to the presence of \COL or of
\TP\faceplus.
\aparag Neither \pays{Iroquois} nor natives in its territory will react
to the presence (movements or staying there) of stacks of at most one
\ARMY\faceplus in its territories.
\aparag The minor country can be used as an ally in wars.
 

\subsubsection{\sectionpays{Ormus}, part of \sectionpays{perse}}\label{chDiplo:Diplo-ormus}
\aparag[Specifics of Ormus.] 
\province{Ormus} is a \ROTW province in \seazone{Persique} belonging to
\pays{perse}.
In general, \province{Ormus} is dealt with as a normal \ROTW province
(allowing forces to enter in it without war declaration, placement of \TP, etc.),
with usual Native reaction, or country \pays{Ormus} reaction.
\bparag No \COL can ever be placed in the province (but a \TP may be).
\bparag A reaction of the minor \pays{Ormus}
is actually a declaration of Overseas war by \pays{perse},
as is a war declaration against \pays{Ormus}.
\bparag A country at war against \pays{perse} and the owner of  forces or \TP in
\province{Ormuz} is allowed to attack it from the European map also.
\bparag The fortress in \province{Ormus} acts as a \Presidio against \province{Bam}.
\bparag See also~\ruleref{chBasics:Provinces:Ormus}.
\aparag[Effects of a Treaty.]
\dipAT with \pays{Ormus} allows a player to have a \TP in
\province{Ormus} that attracts no reaction from \pays{Ormus}, as long as
the \dipAT holds.
\bparag The power can also enter this province with military forces, or
fortify the \TP. This draw no reaction from \pays{Ormus}.
\aparag[Afghanistan.] \pays{perse} may also own
\granderegion{Afghanistan} because of some event. It will not react to
the presence in this area of \TP\facemoins of a power having a \dipAT
with \pays{Ormus}. It will react to the presence of \COL or of
\TP\faceplus.
\bparag \pays{perse} will also not react to the presence (movements or
staying there) of stacks of at most one \ARMY\faceplus in
\granderegion{Afghanistan}, if those are owned by a power having a
\dipAT with \pays{Ormus}.  Neither would natives react under this
condition.

\subsubsection{\sectionpays{Inca} and \sectionpays{Azteque}}\label{chDiplo:Diplo-inca}\label{chDiplo:Diplo-azteque}
\aparag[Permanent \dipAT of Incas and Aztecs.]
In 1492, \pays{Inca} and \pays{Azteque} are always in \dipAT with every
power.  This can change because of event \eventref{pII:American
  Resistance}, or when a power besieges their capital.
\aparag[Effect of \dipAT.]
\bparag The concerned minor country will never react, neither to
military forces, nor presence of \TP/\COL.
\bparag Natives in the area of the country can be attacked with no
declaration of war. The capitals of the empires can also be attacked
without war against the country (but Natives has to be attacked first
for assault or siege).
\aparag[Fall of the American empires.]
\bparag If its capital is controlled by a power at the end of a turn, an
American empire is destroyed. The number of Natives in each province is
now 2 \LD (instead of 20 \LD).
\bparag Place immediately a \COL of level 3 on the city, owned by the
power controlling the city. If this power is \SPA, it must immediately
place a mission there, either by drawing an available mission in the
pool, or by moving a deployed mission that is in the same area; then the
highest rank Conquistador present in the region is nominated as Vice-Roy
of the area.
\aparag[Attack on capital]
Whenever the capital of \pays{Inca} and \pays{Azteque} is attacked, a
test of reaction is made at the end of the round (after the result of
siege or assault). If there is a reaction, the concerned minor country
declares an immediate Overseas war against the aggressor.
\bparag Its troops are deployed (even in occupied provinces) and Natives
in all its provinces are activated for the war.
\bparag If this is the last round of the turn, the Fall of the Empire is
suspended for this turn (but may happen on the future turn).

\subsection{Countries from the \ROTW as ally}
\aparag Some countries from the \ROTW in \dipAT can be used as ally in
wars: \pays{Aden}, \pays{Oman}, \pays{Gujerat}, \pays{Mysore},
\pays{Hyderabad} and \pays{Iroquois}.  The power having the \dipAT can
ask for a limited intervention.  This is a declaration in reaction, and
is shown by placing the forces of the \MIN on the map.
\bparag If more than one power have \dipAT, all that want can ask for
limited intervention. Then they all roll 1d10, modified by the modifiers
for diplomatic actions in the \ROTW. The power that rolls highest gains
the intervention for this turn (in case of ties, no intervention). This
test should be renewed at each turn, and the side of intervention thus
may change.
\bparag[Reciprocal alliance.] As it is an alliance, if the \MIN is declared war
upon or if it declares war, it will call for its patron, that is also an
ally. If the power does not respond the alliance (at the least a
limited intervention), the status is broken to \dipFR. 
\aparag[Conditions of the Limited intervention in \ROTW.]
\bparag A limited intervention of a minor country is made only with its
basic forces. It draws supply only from its own provinces (and so can
not go further than 12 MP from its country). Its units can not go on the
European map.
\bparag The intervention is at most of one land stack and one
naval stack outside the provinces of the minor country. 
\bparag The \MIN receive reinforcements each turn in the administrative
phase. The base reinforcement is given in the annexes. These
reinforcements are only used to recreate de basic force of the \MIN,
should they be diminished.
\bparag All campaign costs for the \MIN are paid by its ally.
\bparag In the provinces that it controls, the \MIN is allowed to attack
forces of enemies, but the Natives are not activated (only the basic
forces may attack). During the end of turn, the forces can do "Native
attack" on \TP/\COL of an enemy power that is in an area the \MIN
controls, but this does not use also the Natives (unless specified in
the description of the country).
\bparag The \MIN is in fact out of the war. The \MIN is not part of
Peace Treaty.  But its territories could be crossed as it is usually
permitted.

\subsection{Military Diplomacy and  Treaty}
\aparag A power at war (normal or overseas) against a country in the
\ROTW signing a
victorious peace treaty of level 2 or higher, and forfeiting all other
conditions of peace, may do the following:
\bparag reducing any or all \dipAT and \dipFR of other powers,
to respectively \dipFR and \dipNR.
A power that has its diplomatic status broken this way gains a temporary
free Overseas \CB against the responsible power;
\bparag and, sign a \dipFR with the \ROTW country, or
upgrade a \dipFR in \dipAT.
 
\aparag A power at war (normal or overseas) against a country in the
\ROTW achieving a peace of level 4 or higher, and forfeiting all other
conditions of peace, may do the following:
\bparag break any or all \dipAT and \dipFR of other powers to \dipNR.
A power that has its diplomatic status  broken this way gains a temporary
free Overseas \CB against the responsible power;
\bparag and upgrade its position by imposing a \dipAT to the \ROTW country.

\aparag Note that Allies in this victorious war can each apply the
previous effects (excepted to break or reduce \dipFR or \dipAT of
Allies in the same war).

\subsection{Activation of \ROTW minors}
\aparag At the end of the diplomatic phase, a test is made for each
\ROTW minor to see whether it declares war against countries inside its
territory.
\bparag a \ROTW minor may react against any or all countries having
either troops (including \LDE, forts of fortresses) or colonial
establishment (\COL or \TP) inside its territory (the areas it owns and
the provinces with its own colonial establishments).
\bparag \dipFR and \dipAT may allow some troops and/or establishment
inside the territory of a minor without triggering activation, as
explained above.
\bparag Leaders alone (with no troops) never cause minor activation.

\aparag For each minor and each country that can cause activation of the
minor, roll one die.
\bparag If the roll is strictly smaller than the Activation level of the
minor, it declares an oversea war against the offending country (and
breaks an eventual \dipAT or \dipFR to \dipNR).
\bparag Otherwise, nothing happens.
\bparag Activation levels are given in the Appendix (in the description
of the minor) and recalled on the \ROTW diplomatic track (on the \ROTW
map).
\bparag Remember that for some countries (eg \paysChina), the activation
level may depend on the province where the troops or establishment are
located.
\bparag It is completely possible for a minor country to declare war
this way against one offending country  but nor against another, even in
the same turn. The test is made for each offending country separately
(in decreasing order of initiative in case the order is relevant).

\aparag Reactions after these declarations of war happen as usual.

\aparag Activation of \ROTW minor should not be confounded with
activation of the natives.
\bparag The former is the whole country declaring war, it is done in the
diplomatic phase and result in diplomatic announcements.
\bparag The later is local population reacting, it is done during each
military round and does not causes a new war or change the diplomatic
status. Moreover only one province is concerned each time.
\bparag Colonial establishments usually to not cause native activation
(the local population is rather happy to trade) while it may cause minor
activation (the government is not happy to see its trade regulation
broken by European).
\bparag The same troop, however may both cause minor activation and
native activation (and thus must roll both in the diplomatic phase and
each military round as long as the condition for activation exists).

\begin{designnote}
  Since the activation happens at the end of the diplomatic phase, you
  have one attempt to get a good diplomatic status after landing
  troops. This typically occurs in two cases:
  \begin{itemize}
  \item At the end of a military phase, an emissary lands in a
    country. During the upcoming diplomacy phase, the emissary has one
    attempt to establish diplomatic status with the country before the
    troops he might have with him cause minor activation.
  \item During the event phase, a \RD causes the diplomatic status of a
    \ROTW minor to decrease. You have one attempt to re-establish it
    before seeing your trade burnt to the ground (or more if by chance
    the minor is not activated this turn\ldots)
  \end{itemize}
\end{designnote}

%\aparag[Ormus.] The only possibility to impose an \dipAT on \pays{Ormus}
%is by conquest of the province \province{Bam} from \pays{perse}. As long
%as the province is owned by another power, an \dipAT is enforced between
%this power and \pays{Ormus} when they are not at war; no other power can
%achieve diplomatic status with \pays{Ormus} when \province{Bam} is not
%Persian.
