% -*- mode: LaTeX; -*-

\definechapterbackground{The powers: at home and abroad}{powers}
\chapter{The powers: at home and abroad}\label{chapter:ThePowers}

\begin{designnote}
  This Chapter describes the main concepts used in the game: structural limits
  of a country, stability, colonial settlements. It also includes the detailed
  turn sequence.

  Several concepts are common with other diplomacy and wargames while some of
  them are specific to \emph{Europa Universalis}. This Chapter only gives an
  overview of them so that the rest of the rules is readable. The rest of the
  rules is ordered in game turn order and each concept will be fully described
  (with all the rules governing it) in due time.
\end{designnote}




\section{Generalities}



\subsection{Precedence}

\aparag In case of apparent contradiction within the rules, resolve the
conflict with the following precedence:
\begin{itemize}
\item Event descriptions supersede any other rule (and often create abnormal
  situations).
\item Specific rules take precedence over regular rules. They are ``ways to
  cheat'' allowed (or mandatory) for each country.
\item Common rules only apply if not contradicted elsewhere.
\end{itemize}

\aparag If there is a contradiction between two events, then the one that
occurred the latest takes precedence. But this is usually not intended and
probably is a bug in the rules.

\aparag If there is a contradiction between two specific rules, or between two
regular rules, this is a bug. Please contact us so we can answer it.



\subsection{Rounding}

\aparag When rounding is required, it is always done in the disfavour of the
player performing the action.
\bparag Especially, any gain (in money, victory points, \ldots) is rounded
down while any loss is rounded up.
\bparag In case of doubt, use the rule of thumb ``who can the more can the
least''. If a country should gain 1.9\ducats, it has not gain 2\ducats, thus
the sum must be rounded down; conversely, if a country has to pay 1.1\ducats,
it has to pay more than 1\ducats and the debt must be rounded up.

\subsection{Order of resolution}
\aparag Often, several similar actions must be resolved simultaneously but may
require decisions of players (eg sieges, attacks of natives, automatic
competition \ldots)
\bparag In case of disagreement in the order of resolution, each player, in
decreasing order of initiative, chooses one of the action to resolve.
\bparag All players get their ``turn'' to choose, even those not implied in
the actions. And each player may choose an action in which he is not implied.

\section{Countries}

\aparag[Majors and minors] Countries are separated into \terme{Majors
  countries} and \terme{Minors countries}.

\aparag Majors countries are the ones who, during the historical framework of
the game, played a role of great influence in Europe or even in the whole
World, thus shaping History as we know it.
\bparag Some majors countries had a more local (geographically or timely)
influence.
\bparag Each player plays one Major country at a time. Some players play the
same Major during all the game while some switch mid-game.

\aparag Minors countries are countries who played only a small role in
History.
\bparag This can be either because they were too small (e.g. \paysCologne) or
because they were quickly destroyed by their powerful neighbour
(e.g. \paysDamas) , or because their influence was very local and only
influenced a couple of other nations (e.g. \paysPerse or \paysEcosse).
\bparag This does not mean that minors countries did not shape History, but
merely that they lack the World-wide or Europe-wide influence that England or
Austria had and that playing them would be less interesting.

\aparag See~\ref{chapter:Basics:Majors} for a list of majors countries
and~\ref{chapter:Minors:List} for a list of minors countries.

\aparag[Europeano-centrism] The game is, voluntarily, centred on Europe and
European powers. This is because we want to focus on the Age of Discoveries
and the way the colonial powers managed to take control of almost all the
World.
\bparag Thus, non-European powers are always minors countries, even those who
did had a large influence and territorial base such as \paysChine or
\paysMogol %"Mogolis Imp." ends with a dot. No need to add one.
\bparag This choice allows the game to focus on intra-European relationships.




\section{Religions and cultural groups}

\label{chThePowers:Religions}



\subsection{Religions and standings}

\aparag Each country, major or minor, has a \terme{religion}.
\bparag Several actions or events in the game depend on the religion of a
given country.
\bparag Several countries (both major and minor) may (or must) change religion
during the course of the game.
\bparag The religion of minor countries is indicated in the description of the
country, see~\ref{chapter:Minors:List}. The religion of major countries is
indicated in scenario description.
\bparag Religions are also indicated on the map. The colour of the border of
the main (non-blurred) shield in each province depends on the religion of the
province (which is usually the religion of the country).

\aparag Several religions are further subdivided into \terme{standings}. Some
actions depend not only on the religion but also on the precise standing of
the country.
\bparag Minors countries usually have no standings, unless explicitly stated.
\bparag The precise standing inside a religion is noted as
``Religion/Standing'' such as ``\CATHCR''. Sometimes, only the standing is
specified (eg ``Counter-Reform'' means ``\CATHCR'').
\bparag If no standing is precised, then the effect apply to all countries of
the given religion, whether they have a standing or not.
\bparag Majors countries have no standing at the beginning of the game and
have to choose one when~\ref{pI:Reformation} happens or when they change
religion as well in a few other circumstances.



\subsection{Cultural groups}

\aparag Each country (major or minor) belongs to one cultural groups (except
\POL and \RUS who belong to two groups).
\bparag These groups are used to determine the technological level of minor
countries and the way they progress. Check~\ref{chExpenses:Technology} for
details on technology.

\aparag Cultural groups usually contains all countries of one or more
religion. Thus, we use the same names (and symbols) to depict them. But these
should not be confused.

\aparag The cultural groups in which a minor country belongs is indicated as
its ``Military doctrine'' in the Appendix.



\subsection{List of religions and cultural groups}

\label{chThePowers:List of religions}
\aparag We give here a list of all religions, standings and cultural groups.

% \begin{todo}
%   Add religious symbols. The \verb+\Xcatholique+ and similar commands do not
%   appear to work within section titles.
% \end{todo}

\subsubsectionJ{Catholic}{\Xcatholique}
\aparag Before~\ref{pI:Reformation}, this religion has no standings
\bparag After, there are two Catholic standings: \terme{Conciliatory} and
\terme{Counter-Reform} (also called \terme{Counter-Reformation}).

\bparag If needed and not specified, consider Catholic minors as begin
\CATHCR.

\aparag Catholic provinces have a golden shield border.

\subsubsectionJ{Protestant}{\Xprotestant}
\aparag This religion is created by \ref{pI:Reformation}.
\bparag Before this event, treat all Protestant countries as Catholic.

% Jym: Puritan to uniformise appelation for ANG and other majors and to
% clarify stuff elsewhere.
%
% old -> new 
% "Anglican" -> "Protestant/Anglican" (or "Anglican") 
% "Protestant (or Anglican)" (sometimes "Protestant") -> Protestant 
% "Protestant (not Anglican)" -> "Protestant/Puritan" (or "Puritan").
\aparag There are four Protestant standings: \terme{Anglican} and
\terme{Puritan} (available only for \ANG) ; \terme{Strictly protestant} (or
\terme{Rigorous} and \terme{Tolerant} (available only for \SUE).
\bparag Protestant minors, as well as other Protestant majors, have no
standing.

\aparag Protestant provinces have a white shield border.

\subsubsectionJ{Latin}{\techlatin}
\aparag The \terme{Latin} cultural groups contains all Catholic and Protestant
countries, plus \POL, plus \RUS after its army reform.

\subsubsectionJ{Orthodox}{\Xorthodoxe}
\aparag There are two Orthodox standings, available only for \RUS:
\terme{Religious tolerance} and \terme{Champion of Orthodoxy}.
\bparag Orthodox minors, as well as \POL if it choose to become Orthodox, have
no standing

\aparag Orthodox provinces have a orange/brown shield border.

\aparag The Orthodox cultural group contains all Orthodox countries, plus
\POL.
\bparag Notice that \POL belongs to both the Latin and Orthodox groups,
whatever its religion but that \paysPologne (once it becomes minor) belongs
only to the Latin group.
% Jym : \RUS also stay Orthodox after reform. That means it will still raise
% the technology of Orthodox minors. There shouldn't be too many of them at
% this point, so it is OK. Moreover, I am too lazy to find a correct
% expression to get \RUS out of the Orthodox group...


\subsubsection{Christian}
\aparag Christian countries are either Catholic, Protestant or Orthodox.
\bparag If an event of effect affects Christian countries, then it affects all
countries of these three religions.

\subsubsectionJ{Sunni}{\Xsunnite}
\aparag This religion has no standings.

\aparag Sunni provinces have a green shield border.

\subsubsectionJ{Shi'ite}{\Xchiite}
\aparag This religion has no standings.

\aparag Shi'ite provinces have a blue shield border.


\subsubsection{Muslim}
\aparag Muslim countries are either Sunni or Shi'ite as well as some \ROTW
countries.
\bparag If an event of effect affects Muslim countries, then it affects all
countries of these two religions and \ROTW minor Muslim countries.
\bparag European Muslim countries are either Sunni or Shi'ite. \ROTW Muslim
countries do not have this distinction.
\bparag \ROTW Muslim countries are the one with a \Xsunnite symbol on the
\ROTW diplomacy track: \paysAden, \paysOman, \paysSoudan, \paysGujerat and
\pays{aceh}. Other \ROTW countries are considered as having no religion for
game purposes.

\subsubsectionJ{Islam}{\techislam}
\aparag The Islam cultural group contains all European Muslim countries and
some \ROTW countries.
\bparag beware that in the \ROTW the Islam group and the Muslim minors are not
the same things. eg: \paysGujerat is Muslim but not in the Islam group while
\paysMogol is within the Islam group but not Muslim (for game purposes).

\subsubsectionJ{Other religion}{\Xautrereligion}
\aparag The \terme{Other} ``religion'' groups all religions that are not
already specified.
\bparag It mostly includes Hinduism, Buddhism, Shinto and various Paganism.
\bparag We do not mean that these religions are all the same. But they played
no role in European conflicts and were treated more or less the same way by
Christian missionaries in India, Africa, America or Asia. Thus, they have the
same effect in game.
\bparag Similarly, we do not mean that religions or standings not listed here
(eg Judaism) did not exist. But they had no large scale effect and do not
require special rules within the game.


\subsubsection{Medieval}
\aparag The Medieval cultural group contains \paysInca and \paysAzteque.
\bparag Natives in % \continentAmerica,
% Not sure about America... During 18th century, the indians were armed by
% FRA/ANG and thus clearly not Medieval... But maybe Ohio is enough to
% represent this ???
\continentAfrica, \continentSiberia, \granderegionOceania and
\granderegionPacifique are considered to be part of this group.

\subsubsectionJ{\ROTW}{\techrotw}
\aparag The \ROTW cultural group contains all \ROTW countries that are neither
in the Islam group nor in the Medieval one.
\bparag Natives in continents and areas not listed as Medieval are considered
to be part of this group.
\bparag Beware that some \ROTW countries are of Muslim religion but belong to
this cultural group.


\subsubsection{Special cases}
\aparag \paysSuisse is both Catholic and Protestant.
\bparag Whenever a major attempts an action on it, considered its religion to
be the worst possible case between them.
\bparag Typically, \paysSuisse is always considered to have another religion
for Diplomatic actions ; Catholic/Counter-Reform countries have no religious
\CB against \paysSuisse and may not convert it (as it is also Catholic) ; and
so on.

\aparag \paysUSA is either Protestant or Catholic, but not both.
\bparag Its religion depends on the religion of the major against who it
declared its Independence. See~\ref{pVII:Independence War} for details.
\bparag \paysUSA may be created several times (representing Bolivarian
revolutions as well as hypothetical revolutions in Canada, India or
Indonesia). In this case, each of the different \paysUSA may have a different
religion.



\subsection{Religious enmities}

\aparag When the game start (in 1492), religious enmities are actives. They
last until the end of \ref{pIV:TYW}.
\bparag Religious enmities mostly make relations between Catholics and
Protestants harder, but they also hamper a bit relations between Christians
and Muslims.

% \section{Modification sur les revenus, juillet 2008}

% \subsection{Attendus}

% \begin{itemize}
% \item empêcher constitution d'un confortable matelas d'argent pour les pays
%   riches
% \item faire une partie des PVs par prélèvement sur richesse du pays
% \item rendre plus historque le recours incessant à l'emprunt
% \end{itemize}

% \subsection{Survol du principe}

% \begin{itemize}
% \item Le 'Gross Income' est la richesse du pays sur laquelle on prévoit les
%   dépenses usuelles.  Selon le résultat d'un test dit « États au vrai », une
%   part de cette richesse devient un revenu qui couvre les dépenses, une part
%   en dépenses de prestige (sauf si cet argent est utilisé pour les dépenses
%   du tour) et donne des PVs, et une part est empruntable en emprunt
%   national. On peut aussi emprunter internationalement.
% \item le Trésor Royal est différent : ce qui vient dans le trésor est ce qui
%   reste dans l'opération précédente. Il reçoit aussi l'or, les revenus des
%   pillages, des corsaires,...

%   Cet argent sert à la diplomatie (majeurs, mineurs, paix,...), à la mise
%   pour augmenter la stabilité, et à équilibrer si les dépenses sont plus
%   grandes que les recettes.
% \end{itemize}

% \subsection{Détails}

% Chaque pays mène une double comptabilité :
% \begin{itemize}
% \item Trésor Royal qui se continue d'un tour à l'autre,
% \item Richesse du pays sur laquelle les dépenses des phase administratives,
%   logistiques et militaires sont prises, qui est équilibré (par le test dit
%   d'«États au vrai «) en fin de tour (phase VII ) et qui peut alimenter le T
%   résor Royal si il reste alors des ducats.
% \item A cela s'ajoute les emprunts (nationaux ou internationaux) qui se
%   poursuivent d'un tour à l'autre tant qu'ils ne sont pas remboursés ou
%   annulés (par des Banqueroutes).
% \end{itemize}

% \subsubsection{Le Trésor Royal (TR)}
% \aparag Ce qui entre et sort du TR :
% \begin{itemize}
% \item Phase I (Event) et II (Diplo) : événements, diplomatie (dont
%   subventions), dons,...
% \item Phase VI : pillages, or, corsaires
% \item Phase VII : on résout dans l'ordre
%   \begin{enumerate}
%   \item Impôts exceptionnels (ajouter le résultat directement dans le TR)
%   \item « États au vrai » : test d10 pour Revenu/Prestige/Emprunt National
%   \item Emprunt international éventuel
%   \item Mise sur la stabilité
%   \item Paix (et indemnités)
%   \end{enumerate}
% \item Phase VIII : Inflation : le \% perdu est pris sur la valeur absolue du
%   TR, avec minimum = le \% en ducats ou la moitié du \% en ducats pour les
%   pays majeurs suivants : POL, SUE en pIII et IV, RUS en pI à pIV, AUS en
%   pIII à pIV.
% \end{itemize}
% \aparag le TR peut passer en négatif (mais attention à la Faillite en Phase
% III)

% \subsubsection{Richesse et dépenses du pays pendant un tour}
% \aparag En Phase III, tout le revenu = Richesse du pays.
% \bparag IMPORTANT : on ne compte pas de malus/bonus de stabilité ici.
% \aparag Faillite = 2 possibilités
% \begin{enumerate}
% \item après le calcul du revenu (Phase II), faillite obligatoire si le TR +
%   Richesse - (Intérêts+Remboursements obligés à ce tour) est négatif
%   \textrightarrow oblige à opérer une Banqueroute immédiate pour corriger
%   cela
% \item à tout instant, si la TR+Richesse est négatif \textrightarrow oblige à
%   opérer une Banqueroute immédiate de remise à zéro
% \end{enumerate}
% \aparag En Phases IV, V et VI, on ajoute toutes les dépenses ensemble
% (Actions administratives, Logistique, Intérêts et Remboursement des
% emprunts, Campagnes, Recrutements exceptionnels,...) = Dépenses du tour
% \aparag Phase VII, Impôts exceptionnels (ajouter le résultat directement
% dans le TR)
% \bparag NOTE : pour les impôts exceptionnels, ils sont décidés en phase IV
% (action Adm), le point de stabilité est perdu tout de suite et le
% modificateur est noté. Cependant, le d10 n'est lancé qu'en Phase VII et les
% ducats résultats vont directement dans le TR
% \aparag Phase VII, « États au vrai »
% \bparag En fin de tour, avant la mise pour augmenter la stabilité, on teste
% l'État au vrai des dépenses et du Trésor Royal
% (see~\tableref{table:Exchequer test}).  \GTtable{etatsauvrai}
% \bparag Résultat à lecture directe : \% en revenu normal, \% en revenu
% Prestige, \% capacité d'emprunt national (la colonne "emprunt international"
% n'est pas utilisée pour ce test)
% \begin{designnote}
%   À noter : colonne +1 à +4 : 3 1/2* ; colonne 0 et -1 : 2 1/2* ; colonne -2
%   à -4 : un seul 1/2*

%   Je vous donne les statistiques moyennes d'auto-finacement sans emprunt
%   (sachant que l'emprunt national vient redresser les faibles valeurs : on
%   peut tjs compter sur 70\% au minimum de disponible, voire 80\% si on peut
%   pas faire de F*, par exemple STAB à +2 ou +3, ou 0 et +1 et en paix sans
%   emprunts en cours) :
%   \begin{quote}\ttfamily\selectfont\obeylines \obeyspaces
%     STAB -3 -2 -1 0 +1 +2 +3 à +2: 64 72 78 81 86 87 91 à +1: 58 65 71 75 80
%     82 86 à 0: 52 58 64 68 73 76 80 à -1: 46 51 57 61 67 70 74 à -2: 40 45
%     51 56 61 64 68 [ comparaison effet STAB avant : 50 66 75 90 100 100 110
%     ]
%   \end{quote}
% \end{designnote}

% \bparag[SOLUTION ALTERNATIVE] paix complète donne +1, on augmente emprunt
% national de base de 10\% mais on met le bonus en emprunt national à 10\% de
% + si pas en paix

% \bparag[Equilibre des dépenses]
% Les dépenses du tount d'abord prises sur Revenu régulier
% \begin{itemize}
% \item revenu régulier ; ce qui reste => TR
% \item revenu de Prestige ; ce qui reste => PVs
% \item l'Emprunt National : revenu optionnel ce qui reste => TR
% \item si manque pour couvrir les dépenses : prendre dans TR
% \end{itemize}

% \bparag[Emprunts Nationaux]
% \begin{itemize}
% \item Une portion de la Richesse d'un pays n'est récupérable pour les
%   dépenses que via des emprunts nationaux, décidés à cette phase.
% \item On tient le compte des emprunts nationaux en ajoutant les sommes
%   empurntées, cumulées d'un tour sur l'autre jusqu'à leur rembousement.
% \item Chaque tranche complète de 100d compte comme un emprunt pour le test
%   d'États au vrai.
% \item Lors de la phase administrative (dépenses, Phase IV), le pays doit
%   payer 10\% des emprunts nationaux en cours (intérêts des emprunts). Les
%   remboursements d'emprunts ne se font qu'après avoir payé les intérêts, à
%   cette même Phase des dépenses.
% \end{itemize}

% \aparag Phase VII, Emprunts Internationaux
% \begin{itemize}
% \item Si l'argent disponible ne lui apparaît pas suffisant, un pays peut
%   tenter de recourir aux emprunts internationaux après avoir résolu les «
%   États au vrai ».
% \item Pour cela, un test est fait sur la table des « États au vrai », en
%   lisant la colonne "Emprunt International", avec les mêmes modificateurs,
%   et +1 au dé si contrôle d'un Stock Exchange
% \item Résultat: \% de la capacité financière internationale que l'on peut
%   emprunter (c'est un max., on peut prendre moins) ; ce qui est emprunté ira
%   dans le TR (ou dans les dépenses du tour)
% \item Capacité financière internationale = 50d + 50d par place financière
%   existante : HRE, Gènes, Amsterdam Stock Exchange, London Stock
%   Exchange. Doubler si la place financière est contrôlée par le pays (HRE:
%   Empereur, Gènes: patron diplo.) (see~\tableref{table:Exchequer test}).
% \item Chaque emprunt international est un emprunt individuel (qq soit la
%   somme).
% \item Intérêts : 10\% de l'emprunt à chaque tour (dans les dépenses)
% \item Remboursement : 3 tours pour le faire ; à mettre dans les dépenses (en
%   plus des intérêts du tour).
% \item On ne peut pas faire défaut au remboursement d'un emprunt
%   international, sauf à déclarer une Grande Banqueroute.
% \end{itemize}
% \aparag[Phase IV, Déclaration de Banqueroute] Les Banqueroutes se déclarent
% au moment des dépenses, avant d'avoir payé les intérêts des emprunts en
% cours. Elles sont optionnelles, sauf à voire le TR menacé de Faillite (en
% quel cas une Banqueroute d'un type permettant d'échapper à la Faillite est
% obligatoire) (see~\tableref{table:Bankruptcy Roll}).  \GTtable{bankruptcy}
% \bparag Note: destruction de TF/MNU : il faut détruire qq chose, sauf si
% plus aucun niveau de l'un et de l'autre
% \bparag[Petite Banqueroute]
% \begin{itemize}
% \item permet d'annuler jusqu'à 200d d'emprunt national
% \item effets négatifs : [perte de 5 PVs - TBD ?] et test de Banqueroute
% \end{itemize}
% \bparag[Grande Banqueroute]
% \begin{itemize}
% \item permet d'annuler jusqu'à 200d d'emprunt international, ou tous les
%   emprunts nationaux
% \item effets négatifs : perte de 15 PVs ; appliquer le résultat 1- de
%   Banqueroute
% \end{itemize}
% \bparag[Banqueroute complète (RAZ)]
% \begin{itemize}
% \item annule tous les emprunts en cours, remet le TR à 0 ducats
% \item effets négatifs : perte de 30 PVs ; appliquer le résultat 1- de
%   Banqueroute et réduire le DTI de 1
% \item compte comme double banqueroute (donc -2 au test d'États au vrai)
% \end{itemize}
% \bparag Effet sur les paix : si un pays fait Banqueroute et se retrouve à -3
% en stabilité, il doit immédiatement proposer une armistice pour le tour, et
% devra accepter toute paix faite sur la base du Différentiel de Paix à la fin
% du tour.
% \bparag TUR : Petite Banqueroute corrompt un pasha, Grande Banqueroute ou
% Remise à zéro corrompt 2 pashas.




\section{The passing of time}

% RaW: [24]



\subsection{Periods}

A period represents a number of game turns, historically homogeneous, with a
duration of approximately 30 to 50 years (more or less). The 62 game turns are
distributed in seven periods to simulate the different epochs of the era
covered by the game.

These periods give the rhythm of the campaign game, especially the 1492-1792
Grand Campaign. Each player possesses for his country, a series of strucutral
limits to his purchases and actions that is determined for each one of the
periods covered in the game. These limits are a maximum that cannot be
exceeded, except for a very few specific cases.



\subsection{List of periods}

The seven periods (with corresponding length in game turns and main historical
features) covered by the game are the following:
% Jym: the colour of the period on the turn track corresponds to the main
% power of the period and the first listed here.
\begin{itemize}
\item Period \period{I}, 1492-1519: 6 turns (discovery of the New World, Wars
  in Italy and consolidation of the powers)% light green -> POR
\item Period \period{II}, 1520-1559: 8 turns (Turkish expansion, exploration
  and colonisation by \SPA and \POR, Reformation and first religious
  struggles) % dark green -> TUR
\item Period \period{III}, 1560-1614: 11 turns (Spanish domination, Dutch war
  of independence, French wars of religion, Swedish rise to
  power) % yellow -> HIS
\item Period \period{IV}, 1615-1660: 9 turns (Dutch commercial domination,
  Thirty Years War)% blueish -> HOL
\item Period \period{V}, 1660-1699: 8 turns (French ``Grand Si\`{e}cle'', wars
  of \monarque{Louis XIV})% blue -> FRA
\item Period \period{VI}, 1700-1749: 10 turns (Russian and Prussian rise to
  power)% brown -> RUS
\item Period \period{VII}, 1750-1800: 10 turns (English domination, from the
  Seven Years War to the American revolution)% red -> ANG
\end{itemize}



\subsection{Limits by period}

\label{chThePowers:Limits}
\aparag Within each of the above periods, countries represented by the
different players have a certain number of structural limits for their
different actions and purchases as well as for the number of available
counters (whether military or commercial) and the content of these counters.
% There are limits as well as for the number of units, strength points
% contents and number of counters (especially Manufactures, Colonies and
% Trading Posts).

\aparag[Limits] These limits are valid each turn during the period for which
they apply.
\bparag The period limits cannot be exceeded, except for some particular cases
specified in the rules.

\aparag[Limits Tables] The limits, for each player, are regrouped in two
different tables on the specific player's aid. The first presents the limits
per period applicable globally for the whole duration of a period, such as the
maximum number of counters of a given type usable in the period. The second
table presents the limits per turn within each period, such as the maximum
purchase available each turn.
\bparag Taking into account the variable length, in number of turns, of the
different periods, it is usually necessary to read the numerous information
printed on these tables at the beginning of a given period, their usage
proving then very repetitive within a same period, thereby making them self
learning.
\bparag The monarch sheet holds space to write down the limits of the current
period to allow an easy access in game.

\begin{exemple}
  If you are unfamiliar with the game, take a player's aid with you before
  reading the following. We advice to use the Portuguese one as it contains
  few exceptions and is thus easier to understand.
\end{exemple}


\subsubsection{Period limits table}
\label{chThePowers:Period Limits}
\aparag The limits fixed in this table cannot be exceeded in principle. This
is valid in any and all turns of the period.
\bparag Some events or other particular circumstances may change the
limits. These special cases are all recalled in the table.

\aparag[Trade] The DTI (Domestic Trade Index) represents the global dynamic of
the internal trade of the country. The \FTI (Foreign Trade Index) represents
the global dynamic of the foreign trade of the country.
\bparag Some countries also have a special \FTI usable only for a precise set
of actions. See the Specific rules of the country for details.
\bparag \DTI, \FTI and special \FTI may vary between 1 and 5.
\bparag The actual value may never exceed the limit for the current period
printed in the table.
\bparag The actual value of the \DTI, \FTI and special \FTI is written by the
player on his monarch sheet.
\bparag If, for any reason, the actual \DTI, \FTI or special \FTI of a country
is above its period limit, %
% eg SPA after expulsion of Jews
immediately decrease it to its maximum value.
\bparag If the actual value is 1 and an event require it to be decreased, %
% eg, bankruptcy
don't change the value. The actual value may never be smaller than 1.
\bparag The actual value of the special \FTI may never be smaller than the
actual value of the \FTI. If the case arise, increase the value of the special
\FTI so that it is equal to the \FTI.

\aparag[Manufactures] The country has a limit of \MNU (Manufacture) counters
(triangle shaped) that it may have in play, on the map of Europe, during a
given period.
\bparag This is a limit in terms of counters. Every counter has two sides
representing an increasing capability of the \MNU.
\bparag This limit may be exceeded by 2 counters, at the risk of economical
losses. See~\ruleref{chThePowers:Exceeding Limits}.

\aparag[Colonial establishments] The country has a limit of \COL (Colonies)
and \TP (Trading Post) counters that may be placed on the map, outside of
Europe, for the whole length of a period.
\bparag This is a limit in terms of counters. Every counter has two sides with
up to 6 levels representing an increasing development of the establishment.
\bparag This limit may never be exceeded. If at any moment a country has more
\COL or \TP on map than counters available for the period, %
% eg, annexation.
immediately remove (at player's choice) exceeding counters.

\aparag[Fleet] Each country has a maximum number of \ND that may be in play at
the same time.
\bparag This counts both the \ND counters and the \ND inside the \FLEET
counters.
\bparag \NGD only count as half.
\bparag This limit may not be exceeded. If a country ever owns more \ND than
this limit, %
% eg captured after a battle, eco event
immediately destroy exceeding \ND (at player's choice).

\aparag[Troop size] Land and sea unit have some specific size. This is a
structural description of the military doctrine of the country rather than a
real limit.
\bparag A \FLEET counters may contain up to a certain number of \ND (first
value) and \NTD (second value), depending on their side. \NGD count as half a
\ND. \FLEET counters are only containers and may be created for free at any
time if the need occurs.
\bparag An \ARMY\faceplus counters automatically contains the indicated number
of Artillery. An \ARMY\facemoins only contains half as much (round
down). \ARMY counters do contain artillery even with \terme{Medieval}
technology (this is siege artillery only).
\bparag Land troops belong to a certain class of army. This is the military
doctrine of the country and may not be changed voluntarily. Armies of the same
class hold roughly the same number of infantry and cavalry.

\begin{exemple}
  During period \period{I}, Portugal has a maximum \DTI of 3, as per scenario
  description, it is also its actual value in 1492. Thus, \POR will not be
  able to increase its \DTI during period \period{I}. However, the maximum
  \DTI switch to 5 in period \period{II}. This does not automatically increase
  the actual \DTI, but simply allows \POR to attempt administrative actions to
  do so.

  During period \period{I}, \POR may never have in play more than 3 \MNU
  counters. Since two of them are placed in 1492, \POR may only create one
  more before 1520, when a fourth counter will be available. \POR may also
  increase the level of its \MNU (and flip them to their \Faceplus side) since
  this does not create new counters. A total of 6 Portuguese \MNU are provided
  in order to give the player choice on which industry to develop.

  During the first three periods of the game, \FLEET\facemoins of \POR may
  contain at most 2\ND and 1\NTD. Its \FLEET\faceplus may contain only 4\ND
  and 1\NTD. It is possible to have a non-full counter (such as a
  \FLEET\facemoins with only 1\ND and no \NTD or a \FLEET\faceplus with 3\ND
  and 1\NTD). The exact content of the counter being written on the colonial
  sheet. Note that due to maintenance cost, it is usually most unwise to have
  \FLEET counters with very few \ND in them. \POR may not, in period
  \period{I}, have more than 12 total \ND on the maps (including those in
  \FLEET).

  In period \period{I}, \ARMY\faceplus of \POR automatically contain 2
  artilleries (an abstract measure of the guns, howitzers and such). In 1520,
  this switch to 3 and all existing \ARMY are automatically upgraded to this
  value (as the typical content of field forces evolves with
  time). \ARMY\facemoins contains half that many artilleries, round down,
  hence only 1 in this case. Troops of \POR are of class \CAIII, an abstract
  measure of the military doctrine of the country (notably with respect to
  typical size of field forces as well as cavalry number and doctrine). This
  never changes (only a handful of countries change their army class). Class
  \CAIII\ regroups most occidental powers.
\end{exemple}


\subsubsection{Exceeding Limits in \MNU}
\label{chThePowers:Exceeding Limits}
\aparag A player can decide to exceed the limits of a period by up to 2 \MNU
counters, with the restriction that the absolute limit is the number of such
counters provided in the game.  However, this puts the economical stability at
risk.

\aparag If a revolt occurs in such a power, and if the result of the modified
die-roll serving to determine the strength of the revolt is even, then the
power immediately suffers economical losses.
\bparag \textit{Exception}: If it has created its Stock Exchange \HOL
(event \eventref{pIII:Amsterdam Stock Exchange}) and \ENG
(\eventref{pIV:London Stock Exchange}), suffer from losses only if the die
is %
%%% with revolt strength on 1d10, use:
% 6, 8 or 10.
%%% with revolt strength on 2d10, use:
10, 12, 14, 16, 18 or 20.
\bparag If there was no die-roll (eg some revolts caused by events), roll to
check for economical losses.

\aparag If the power is at \STAB {\bf -1}, {\bf -2} or {\bf -3} at the
beginning of a turn, it has 1 chance over 2 to suffer economical losses in
addition.
\bparag \textit{Exception}: If it has created its Stock Exchange \HOL and \ENG
may suffer losses only if \STAB is {\bf -2} or {\bf -3} at the beginning of a
turn.

\aparag[Economical Losses]
The power loses 1 in \STAB and 2 \MNU counters (not levels) are eliminated
(chosen at random).


\subsubsection{Actions and investments}
\aparag At each turn, each country is allowed to do a certain number of
actions (administrative or diplomatic) to increase its economical capacity or
diplomatic influence.

\aparag Most actions are performed by spending a certain amount of
money. There are usually three possible costs, called investments, for each
kind of action.
\bparag Paying an higher investment increase the chances of success of the
action.
\bparag Specific value of the monarch also have a lot of influence on the
chances of success.

\begin{playtip}
  It is usually better (in term of probability of success relative to the
  amount spent) to do several time the same action at the basic investment
  than to do it once at high investment. However, it will also take more time
  as some attempts will fail.

  Thus, for the long term development of the country (almost all
  administrative actions and often for Diplomacy), it is usually a good idea
  to use only basic investments. In some cases, however, the result has to be
  achieved as fast as possible (typically for raising \STAB or technology, and
  in some case for administrative actions to meet certain objectives). Then, a
  higher investment is the way to go but the result might cost a lot of
  \ducats\ldots
\end{playtip}

\aparag Each turn, all actions are first written down, and payed for, before
being resolved simultaneously. Thus, it is impossible to wait for the result
of a given action before deciding to do another one.

\aparag Most administrative actions are resolved on \ref{table:Administrative
  Actions} by cross-referencing a column (depending on the estate of the power
and the investment for the action) and a die-roll.


\subsubsection{Turn limits Table}
\label{chThePowers:Turn Limits}
\aparag The turn limits per period concerns essentially diplomatic and
administrative actions, and also the logistical elements (recruitment, etc.)
for each country and each period of the game.

\aparag[Diplomacy] Each country has a maximum number of diplomatic actions
relative to its diplomacy on minor countries allowed during each turn of the
period.

\aparag[Administration] Each country has a maximum number for every
administrative operation that it can attempt each turn within the period, in
particular:
\begin{itemize}
\item Commercial Development
\item Colonisation
\item Establishment of Trading Posts
\item Competition Action
\end{itemize}

\aparag A country may always use fewer actions (or even none) than allowed by
his maximum period/turn limit.

\aparag[Technology] In addition to these specific limits, each country may
attempt each turn to increase both its land and its naval technology.
\bparag However, only one of the two technology increases may be attempted
with more than a minimal investment.

\aparag[Domestic action] Each country may also each turn attempt one (and only
one) of the following actions:
\begin{itemize}
\item Increase its actual \DTI.
\item Increase its actual \FTI. This does not increase the special \FTI unless
  the \FTI becomes larger than the special \FTI (in which case the special
  \FTI is raised at the same value as the \FTI).
\item Increase its actual special \FTI.
\item Create a new \MNU. This can either switch an existing counter from its\
  \facemoins side to its \Faceplus or create a new \Facemoins counter.
\item Raise exceptional taxes.
\end{itemize}

\aparag[Free maintenance] Basic forces indicate the number of units considered
to have a free maintenance (i.e. those in play without having to pay their
maintenance each turn). Land units of this basic force are veterans.

\aparag[Military force purchase] The military force purchase indicates the
maximum number of forces, (either \LD or \ND), that the country may buy in
each Purchase phase.
\bparag \NGD count only for half a \ND in the purchase limit.
\bparag Exploitation of wood and fisheries increase the limit of \ND,
see~\ruleref{chExpenses:Naval Purchase}.
\bparag The \ND limit may not be exceeded.
\bparag the \LD limit may be exceeded. Every \LD recruited below the limit is
payed at normal cost, then every \LD below twice the limit is payed at double
cost and finally every \LD below three times the limit is payed at triple
cost.
\bparag It is not possible to recruit in a given turn more \LD than three time
the turn limit.

\aparag[Minimum Leaders] The player must also have in permanence a certain
number of leaders, specified by type (\LeaderA, \LeaderG, \LeaderC, \LeaderE,
\LeaderGov). If the historical (named) leaders that the player receives do not
suffice to reach this minimum leader quantity per type, he can take some
additional leaders at random, among his available unnamed \anonyme\ leaders of
the required type.
\bparag The \anonyme\ leaders of majors countries are removed each turn and
new ones are drawn if needed.
\bparag The monarch leader as well as the Turkish Vizier and Swedish heirs are
never counted in this limit (they all bore a ``monarch'' symbol (crown)).
\bparag If a country as more named leader than its limit, then no unnamed one
is drawn.
\bparag Should a country fall below its limit during the turn (due to death in
battle), a new \anonyme\ leader is drawn at the beginning of the next round in
order to reach the limit again.

\begin{exemple}
  In period \period{I}, \POR may attempt each turn up to 2 diplomatic action,
  1 Trade Fleet Implementation, 1 \COL, 2 \TP and 2 concurrence actions. None
  of them is mandatory. Any action performed must be paid for and is not
  guaranteed to succeed.

  In period \period{I}, \POR may not buy more than 4\ND each turn. It may buy
  up to 2\LD at normal cost, plus 2 more at double cost and again 2 more at
  triple cost. It cannot buy more than 6\LD in a given turn.

  In period \period{I}, \POR must have each turn at least 1\LeaderG,
  1\LeaderA, 1\LeaderC and 1\LeaderE. At turn 1, its only historical leader is
  \leaderDias, an \LeaderE. Thus \POR must draw at random amongst its leader
  one \anonyme\LeaderG, one \anonyme\LeaderA and one \anonyme\LeaderC. Should
  \leaderDias dies during the turn (eg, speared by natives in the Cape
  peninsula), he is replaced by a \anonyme\LeaderE at the beginning of the
  next round. Similarly, if the \anonyme\LeaderG attempts a war in \paysMaroc
  and dies in the desert, he is replaced by another \anonyme\LeaderG (possibly
  the same) at the beginning of next round.

  At turn 2, the leader limits have not changed but \POR received new
  historical leaders and now has: \leaderDias (provided he did not die on turn
  1), an \LeaderE ; \leaderCabral, another \LeaderE ; and \leader{Da Gama},
  who counts as a \LeaderC. Thus, it must draw a \anonyme\LeaderG and a
  \anonyme\LeaderA. Having 2 \LeaderE (more than the limit) is not a problem
  since both of them are named (historical leaders). However, if one of them
  happen to die during the turn (eg, lost at sea near \continentBresil), the
  other one is enough to fulfil the limit of 1 an no \anonyme\LeaderE is
  drawn. If, by a stroke of bad luck, both \leaderDias and \leaderCabral die
  during the turn, then a \anonyme\LeaderE (and only one) is drawn as
  replacement in order to reach the limit (of 1).

  Note that \leader{Da Gama} may be used as a \LeaderE but always counts
  towards the limit of \LeaderC (as per \ref{chMilitary:Double Sided
    Leaders}). Thus, it is possible for \POR to have 3 \LeaderE active at the
  same time (\leaderDias, \leaderCabral and \leader{Da Gama}), but \leader{Da
    Gama} still counts as a \LeaderC and no \anonyme\LeaderC is
  drawn. Similarly, if both \leaderDias and \leaderCabral die, a
  \anonyme\LeaderE is drawn whichever side \leader{Da Gama} is used.
\end{exemple}




\section{Estates of a Power}

% RaW: [25, 26, 27, 28, 30, 31]

% \subsection{Provinces}
% \aparag[National Provinces]\label{chThePowers:National Provinces}
% Jym: Is in chapter 1.

% \subsection{Monarch survival phase}
% RaW: 28 This phase is played simultaneously, each player testing the
% survival of his monarch, and determining if needed the characteristics of
% its successor.

% The player represents the monarch of his country. The monarch possesses 3
% values: Administrative (\ADM), Diplomatic (\DIP) and Military (\MIL). These
% values are used for administrative and diplomatic operations of the country
% and in a lesser measures in the military domain.

% The different characteristics of a monarch are used for administrative
% operations, diplomatic operations on minor countries and also to improve the
% technological level of the player's country. In a lesser measure, the \MIL
% characteristic may directly affect combat (if the monarch leaves in person
% to conduct the war).

% Every monarch has also a maximum life duration determined upon his arrival
% on the throne (i.e. in play). When he dies, characteristics of the successor
% are determined randomly.

% \aparag The Monarch survival sub-phase consists in the following operations
% (conducted simultaneously by each player):
% \bparag Monarchs Survival test
% \bparag Successor values determination (when required)
% \bparag Initiative calculation



\subsection{The monarch}

\aparag The player represents the monarch of the country he is in charge
of. He may execute different actions thanks to values of his monarch, whose
reign has a limited duration.

\aparag[Reign Length] A monarch has a reign length evaluated in number of
turns. A monarch must undergo a survival test each turn, and if he succeeds
them all successively, he dies at the beginning of the turn following its last
turn of reign. The reign length of a monarch is determined at the moment of
his advent.%, on the reign table (note that
% reign length of the first monarch in play is given for each campaign
% scenario).
\bparag Monarchs may die earlier than scheduled due to failure of the survival
test.

\aparag[Characteristics of a monarch] Each monarch possesses 3 values:
\bparag \ADM: Administration
\bparag \DIP: Diplomacy
\bparag \MIL: Military

\aparag These value usually varies between 3 and 9. Very few exceptions can
drop these values below 3 in which cases the new limit will be explicitly
stated.

\aparag Each one of these characteristics is determined only once, at the time
a new monarch ascends the throne (after the death of the precedent).
% , thanks to the Monarch Values Table~\tableref{table:Reign}

\aparag[Administrative Value] \ADM is the main modifier for most
administrative operations, either for the choice of the column, or for
die-roll modifications (see~\ruleref{chExpenses:Administrative Actions}).

\aparag[Diplomatic Value] \DIP is the main die-roll modifier for diplomatic
actions, such as the attempt to get control of a minor country
(see~\ruleref{chDiplo:Diplomatic Actions}).

\aparag[Military Value] \MIL is the main modifier of the technology
improvement operation (see~\ruleref{chExpenses:Technology Improvement}). It
also serves to determine combat values (maneuver, fire, shock) of the monarch
when serving as a General.


\subsubsection{The survival test}
\aparag Each monarch has to make a survival test at the beginning of each
turn.
\bparag if a monarch dies due to failure at this test or arriving at his
scheduled reign length, the player has to roll for a new monarch.

\aparag The procedure is explained in details in~\ruleref{chEvents:Survival}.


\subsubsection{Initiative}
\aparag The initiative of each player is the sum of their respective monarchs
values.
\bparag Ties must be solved by unmodified competitive die-rolls.

\aparag The initiative is used during the military phase, to determine the
order of play for the turn between the different players.
\bparag Players (or alliances) play in turn, according to the descending order
of initiative


\subsubsection{Ministers}
\aparag Countries may get an excellent minister to administrate them, either
by~\ref{eco:Excellent Minister} or by some political events.
\bparag For each of the three characteristic (\ADM/\DIP/\MIL), always use the
best between the one of the monarch and the one of a minister.
\bparag Exception: For determining the values of monarchs as general, use the
\MIL of the monarch, not the \MIL of a minister.

\aparag List of named excellent ministers (arriving through political events
or special rules):\\
\ministrePitt, \ministreRichelieu, \ministreMazarin, \ministreColbert,
\ministrePatkul, \ministrePotemkine, \ministreOxenstierna, \ministreKoprulu,
\ministreOlivares, %\ministreAlberoni,
\ministre{de Witt}, \ministreHeinsius, \ministreKaunitz.

% \subsection{The Royal Treasury}

% % RaW: 25+26
% \begin{todo}
%   Rewrite for compta v2.
% \end{todo}

% \subsubsection{Bankruptcy and Treasury
% Collapse}\label{chThePowers:Bankruptcy}

% --- SUITE A ENLEVER si COMPTABILITY v2 ---

% \aparag A power suffers from Bankruptcy at the instant when his royal
% treasury reaches 0 \ducats (or less), with consequences on its \STAB and
% economy.  Moreover, if its Royal Treasury goes underneath -200\ducats at any
% time or the Royal Treasury is negative after adding incomes (\terme{Net
% Income}, \lignebudget{29}, see~\ruleref{chapter:Incomes}), the power suffers
% Treasury Collapse.  The consequences are more serious for the Stability,
% income and investments of the concerned player.

% \aparag[Bankruptcy] A power suffers from bankruptcy at the instant when his
% royal treasury reaches 0 \ducats (or less).  The following effects are
% immediately applied:
% \bparag 1 \MNU (counter, not level) is eliminated (chosen at random)
% \bparag 1 level of \TradeFLEET are lost, chosen by the bankrupted player;
% \bparag the \STAB decreases by 1 level;
% \bparag[Exception] \FRA, \SPA loses 2 \STAB.  If it has created its Stock
% Exchange \HOL (event \eventref{pIII:Amsterdam Stock Exchange}) and \ENG
% (\eventref{pIV:London Stock Exchange}), loses also 2 levels.
% \bparag For \TUR, one \Pasha becomes corrupted.
% \bparag A Bankruptcy has also an unfavorable and permanent effect for the
% player when he rolls on the Loans table during each remaining turns of the
% current period.

% \aparag[Treasury Collapse]
% If the Royal Treasury of a power goes underneath -200\ducats at any time or
% the Royal Treasury is negative after adding incomes (\terme{Net Income},
% \lignebudget{29}, see ~\ruleref{chapter:Incomes}), the power suffers
% Treasury Collapse.  The following effects are immediately applied:
% \bparag 2 \MNU are eliminated (chosen at random)
% \bparag 4 levels of \TradeFLEET are lost, chosen by the bankrupted player;
% \bparag the \STAB decreases by 4 levels;
% \bparag the \DTI is decreased by 1;
% \bparag 15 \PV are lost.
% \bparag A Treasury Collapse has the same effect as a Bankruptcy when rolling
% on the Loans table during each remaining turns of the current period.

% \aparag[Bankruptcy and Peace]
% \bparag Bankruptcy or Treasury Collapse forces the bankrupted player to sign
% a peace with all of his enemies if bankruptcy brings his Stability to the -3
% level.  This peace will be determined and resolved during the immediately
% following Peace step.
% \\

% --- FIN: ENLEVER dans COMPTABILITY v2 --



\subsection{Stability}

% RaW: 27

\STAB is the most important indicator for a country in a campaign game. It is
Stability that allows players to best manage their country because it has an
impact on most administrative operations, as well as monarch survival, income,
war duration and on peace levels achieved.

\aparag The \STAB of each country fluctuates between -3 and +3. A positive
\STAB is a good thing while a negative one hampers the country.
\bparag The \STAB of each country is recorded on the \STAB track on the \ROTW
map.
\bparag Each major country (as well as some other entities) has a \STAB
counter used to record its \STAB.

\aparag[Variation of the Stability] The \STAB varies according to the actions
of the players or situations affecting the player (e.g., state of war), or
else by events.
\bparag These variations are indicated here and there in the concerned rules
or events. It is not necessary to learn each and every one variation, as they
will be reminded to the players when the need arises.
\bparag The main reason to loses \STAB is by wars (either declaring one or
going on in a existing war). \STAB may also be lost because of revolts,
bankruptcy and some other events.
\bparag The main way to gain \STAB is by paying the \STAB improvement
operation at the end of turn (see~\ruleref{chPeace:Stability
  Improvement}). \STAB may also be gained by ending war and a handful of
events.

% Removed compta v2.  27.2 STABILITY AND INCOME
%
% The Stability affects the income as follows:
%
% +10% to Income with a +3 Stability.
%
% From -10% up to -50% for a negative or null Stability.
%
% These values are indicated at the bottom of each box on the Stability track

% 27.3 STABILITY IMPROVEMENT The player may improve his Stability, by
% attempting a Stability Improvement Operation (see 50.2).

% Stability may also improve during the Redeployment phase when prosperity
% increases (see rule 49.1).

% \subsection{Economy}



\subsection{Technology}\label{chThepowers:Technology}

\aparag Technology is an abstract representation of the weapons, army
discipline, military doctrine and such.


\subsubsection{Generalities about technology}
\aparag[Levels and goals.] Each major power and cultural group, as well as
some other entities, has a technology level between 1 and 70.
\bparag Technological goals represent major breakthrough in the art of
war. They also have a level between 1 and 70.
\bparag Whenever the level of a country is higher than the level of a goal,
the country possess the corresponding technology.
\bparag These level are recorded with counters on the technology track (on the
\ROTW map).
\bparag The initial level of each technological goal is written on its counter
(or specified in the scenario). The initial level of majors and groups is
given in the scenario.

\aparag[Land and Sea.] Technology levels and goals are split in \terme{Land
  technology} (affecting armies and sieges) and \terme{Naval technology}
(affecting navies and exploration).
\bparag Always ignore anything dealing with Land technology when handling
Naval technology and always ignore anything dealing with Naval technology when
handling Land technology. Typically, Land technological goals do not affect
Naval technological levels and so on.
\bparag Neither \AUT nor \PRU have a Naval technology counter. Their naval
forces have the same technology than the Latin group.

\aparag[Cultural groups.] The technology of a minor country is the technology
of its cultural group.
\bparag Major countries also belong to cultural groups. Whenever the
technology of a major increase, the technology of its groups may also
increase.
% \bparag See~\ref{chThePowers:List of religions} for a list of cultural
% groups.


\subsubsection{Technological goals}
\aparag[Mobile markers] There are several counters on the technology track
that may move and represent the spread of the various technologies
(technological goal) that a country can achieve. There is no counter for the
technologies known to everybody in 1492 (\TMED for Land technology, \TCAR and
\TGA for Naval technology).
\bparag There is also a marker for the \TTER technology (that can be taken
only by \SPA) and for the \TVGA technology (that can be taken only by \VEN).
\bparag Each of the mobile markers has a turn (and the corresponding year)
written on it. This indicates the first possibility of access to the
corresponding technology.
\bparag If a country has a too rapid progression, its technology marker will
be blocked at the level just below the goal until the turn written on it is
reached.
% \bparag Mobile markers are placed at the beginning of the game on the
% corresponding boxes as written in the scenarios.

\begin{exemple}
  Technology \TARQ may not be discovered before turn 11, its initial level is
  21 (as indicated on the counter).

  At turn 9, \FRA has a Land technology of 19 and succeed in gaining 2
  levels. That should bring it to level 21 and give it \TARQ. However, \TARQ
  is not accessible yet. So the Land technology of \FRA is instead blocked at
  level 20 and the extra progression level is lost.

  At turn 10, \FRA may still not get \TARQ. Thus, trying to improve its Land
  technology is useless as no level may be gained.
\end{exemple}

\aparag[Stacking of counters.] Two technological goals counters (of the same
kind: Land or Sea) may never be stacked on the same box and must always have
at least one free box between them. If a goal should move to the same box as
another, or to the box immediately above, it stops two boxes ahead.
\bparag The technology marker of a country (or group) may never be exactly in
the same box as a technological goal (of the same kind). If it stop on it, it
gains a one box bonus. Conversely, if a goal drops, it stops one level before
any marker of the same kind.
\bparag Exception: Ignore the \TTER and \TVGA for everybody (including \SPA
and \VEN).

\begin{exemple}
  At turn 11, the Land technology of \FRA is at level 20, just before \TARQ at
  level 21. \TARQ becomes available and \FRA tries to raise its Land
  technology and manage to gain 1 level. This should put its marker in box 21
  on top of the \TARQ marker. Since stacking of a marker and a goal is
  forbidden, \FRA gets a bonus level of Land technology and goes directly to
  22.

  At turn 21, after technological improvement, no one managed to get \TMUS and
  the higher level of Land technology is 29. As per goals adjustment
  (see~\ref{chExpenses:Technology}), \TMUS should loss 1 level from 30 to
  29. However, that would put it on top of a marker, which is forbidden, thus
  it stays at 30 instead.
\end{exemple}


\subsubsection{List of technologies}
\aparag We give here a list of all Land and Naval technologies, in
chronological order, together with the year and turn of availability. For
historical reference, we also give the first country (or countries) that
acquired this technology, as well as a short commentary on what it
represents. Only the first three columns are of interest in game.
\bparag Note that some countries are sometime allowed to gain technologies
earlier than allowed, thus the dates here may be later than the historical
occurrence of the technology (typically, \SUE gets \TBAR in the 1630's and
\ENG gets it in the 1640's).

\bparag List of Land technologies:\par
\begin{tabular}{|l|cc||c|l|}
  \hline
  Name & Year & Turn & First & Remark\\
  \hline
  \TMED & & 1 && At start.\\
  \TREN & 1492 & 1 & \FRA, \TUR & Generalisation of field artillery.\\
  \TTER & 1520 & 7 & \SPA & \SPA only.\\
  \TARQ & 1540 & 11 & \FRA & ``\emph{Trace italienne}''.\\
  \TMUS & 1590 & 21 & \HOL &\\
  \TBAR & 1650 & 33 & \SUE, \ENG & \emph{L\"{a}derkanonen}, New Model Army.\\
  \TMAN & 1685 & 40 & \FRA, \AUS & Vauban fortification system.\\
  \TL & 1770 & 57 & \PRU & Oblique order.\\
  \hline
\end{tabular}

% \aparag[Land Technology] Current state
% \begin{verbatim}
%             Year  P-T    \#box Moral First(s)       Fortress
%             %             Medieval 0 1
%             %             Renaissance 1492 I-1 10 1/2 FRA, TUR f3*2
%             %             Tercio ~1520 II-7 16 3 SPA
%             %             Arquebus 1540 II-11 21 2 FRA (Trace italienne) -1
%             %             Ass.,f3*1
%             %             Musket 1590 III-21 30 3 HOL/21
%             %             Baroque 1650 IV-33 40 3 SWE/29,ENG/31 f4*2,t40:
%             %             f4*1
%             %             Manoeuvre 1685 V-40 51 3 FRA,AUT t40 Vauban: f5
%             %             Lace War 1770 VII-57 67 3 PRU/54
% \end{verbatim}

\bparag List of Naval technologies:\par
\begin{tabular}{|l|cc||c|l|}
  \hline
  Name & Year & Turn & First & Remark\\
  \hline
  \TGA & & 1 & & At start.\\
  \TCAR & & 1 & & At start.\\
  \TGLN & 1492 & 1 & \POR, \SPA & Caravels, Nefs, Great ships, \ldots\\
  \TGF & 1560 & 15 & \HOL, \ANG & Dutch Fluyt, British race-built
  galleons. Smaller and faster.\\
  \TVGA & 1560 & 15 & \VEN & \VEN only.\\
  \TBAT & 1590 & 21 & \HOL, \ANG &\\
  \TVE & 1645 & 33 & \ANG, \HOL & Ship-of-the-line, primitive battle line.\\
  \TTD & 1690 & 41 & \ANG & British standardisation, early professionalisation
  of the navies.\\
  \TSF & 1735 & 50 & \FRA, \ANG & Better blockade, faster 2nd line ships.\\
  \hline
\end{tabular}

% \aparag[Naval Technology] Current state
% \begin{verbatim}
%               Year  P-T   \#box Moral   First(s)
%               %               Carrack 0 1 1
%               %               Nao-Galeon 1492 I-1 11 2 POR SPA
%               %               Galleon-Fluyt 1560 II-15 24 2 HOL ENG
%               %               Galleass 1560 III-15 26 VEN only
%               %               Battery 1590 III-21 34 3 HOL, ENG
%               %               Vessels 1645 IV-33 44 3 ENG, HOL
%               %               Three-Decker 1690 V-41 58 3/4 ENG, FRA
%               %               74s 1735 VI-50 68 3/4 FRA/50,ENG
% \end{verbatim}

% \begin{designnote}
%   Carrack: also includes Caravels, Nefs, Great Ships,...\\
%   Nao-Galeon: Nau portugais, Galleon espagnol\\
%   Galleon-Fluyt: British Galleon, Dutch Fluyt,... [ex Lateen Sail] {Gallion
%   anglais affiné et petits
%   bateaux hollandais}\\
%   Battery: Galleass available for all\\
%   Vessels: Galleys worthless against this, Ship-of-the-line et ligne de bataille primitive\\
%   Three-Deckers: HOL: can not built anymore from TF action, (standardisation anglaise effective, route vers un peu de professionnalisation)\\
%   74s: Ships 74's guns (navires de 2nde ligne plus agiles, blocus renforcé)
% \end{designnote}




\section{Troubles at land and sea}



\subsection{Revolts}

% RaW: 30
A revolt is determined randomly, concerning where and when it occurs. A revolt
usually brings a drop in the \STAB level of the victim country. This country
has to crush the revolt as quickly as possible otherwise taking the risk to
witness an extension of that revolt (that can go as far as overthrowing the
country's monarch).

Revolts are generated by events. The revolt tables indicate the strength and
location of the revolt, according to the period in play. The revolt tables are
located at the start of the Events handbook.

One table is used to determine the victim country. Another group of tables
(one per country) helps determine the revolting province. Finally a table
gives the virulence (i.e. the strength) of the revolt.

\aparag[Resolving Revolts] Revolts are rolled for when required per the
political events. See~\ruleref{chEvents:Revolts} for full details on the
procedure.

\aparag[Revolt and Income] Provinces in revolt bring no income during the
country's income phase. Rather than recomputing the income of unrevolted
provinces each turn, this is recorded as a loss on the Economic Record Sheet
(line \ERS{Pillages, Revolts, Pashas}).

\aparag[City control] Unless specified by the event causing the revolt or by
the strength of the revolt, the city in a revolted province is still
controlled by whoever controlled it last for all aspects of the game (eg for
supply).

\aparag[Technology] Revolted troops (if any) have the same characteristics
(class, number of artillery, \ldots) as troops of the country in which the
revolt occurs.
\bparag The technology of the revolted troops is the technology of the country
in which it occurs at the beginning of the turn. Hence, it may be lower than
its actual technology if the country managed to increase it during the turn.

\aparag[Crushing revolts] During the military phase, a country may attempt to
crush a revolt by sending troops in the revolted province.
\bparag Revolts occurring inside minor countries that are not active are
automatically removed at the end of turn.

\aparag Only national units from the country in which it occurs may be used
against revolts. No units belonging to minors (even vassals) or major allies
may be used.
\bparag Exception: the Emperor may use Holy Roman Empire units to repress
revolts in the \HRE.
\bparag Exception: \ANG may use troops of \pays{ecosse}
(after~\eventref{pVI:Act Union}) to fight all its revolts and troops of
\pays{hanovre} (after~\eventref{pVI:Vassalisation Hanover}) to fight revolts
inside his or \pays{hanovre}'s territory.
\bparag Exception: during religious or civil war, any country at war or in
intervention may fight revolts allied to the other side.

\aparag[Revolts and \STAB] At the end of turn, existing revolts cause loss of
\STAB as indicated in~\ruleref{chRedep:Revolts Stability}.

\aparag[Extension of revolts] If one or more revolt still exists at the end of
turn, it extends in the same or adjacent provinces,
see~\ruleref{chRedep:Extension Revolts}.

\aparag[Revolt stacking] Each province may contain up to two revolt counters
(any side) in addition to other military units (with usual stacking).
\bparag If a third counter had to be created in a given province, simply
ignore it but apply all the other effects of this revolt (eg taking the
fortress, adding troops, \ldots)

\aparag[Successful revolts] If they spread too much, revolts may have two
separate but very negative effect on a country.
\bparag If half the national provinces of a country are revolted at end of
turn, then its tyrant is overthrown (executed or exiled) and replaced by a new
benevolent monarch, see~\ruleref{chRedep:ExecutionMonarchByRevolts}.
\bparag If some specific provinces of a given group (eg Ireland) are revolted,
the owner may choose to give independence to the revolted principality rather
than trying to crush it now. See~\ruleref{chSpecific:Peace:Independence
  Revolt} for the list of concerned provinces.



\subsection{Pirates}

% RaW: 31
Pirates appear in \STZ/\CTZ each turn and also due to some events. Each turn
they remain in play, pirates will try to weaken all the commercial fleets,
during the redeployment phase of the turn.

Pirates are represented by the abstract minor ``country'' \pays{pirates},
mostly with \corsaire counters as well as some \LeaderA representing famous
pirates (such as \leaderBlackbeard).

\aparag[Appearance of Pirates] Pirates appear each turn in the \ROTW \STZ
according to the Economic Situation die roll, see~\ruleref{chEvents:Piracy}.
\bparag The presence of named \pays{pirates} \LeaderA increase the risk of
pirate appearing.
\bparag The economical event~\eventref{eco:Piracy} causes several appearances
of pirates.

\aparag[Stacking of pirates] Two \pays{pirates} \corsaire\facemoins in the
same \STZ/\CTZ are immediately exchanged for one \pays{pirates}
\corsaire\faceplus. \pays{pirates} \corsaire\faceplus and \Facemoins can
coexist in a \STZ/\CTZ, without limit.

% Jym: Yeah ! une regle inconnue mais tellement specifique qu'elle en devient
% useless... Paf la baleine !
% \aparag[Revolt and pirates] When all provinces bordering all the sea zones
% attached to a particular \STZ or \CTZ are in revolt (unusual but possible),
% place one \pays{pirates} \corsaire \facemoins in this \STZ/\CTZ at the
% beginning of the Peace phase (during the extension of revolts).

\aparag[Effect of pirates] \pays{pirates} \corsaire (as well as countries
\corsaire) attack the commercial fleet of other countries.
\bparag They may be fought during the military phase by navies.
\bparag At the end of turn, they can cause loss of levels on \TradeFLEET.
\bparag See~\ruleref{chRedep:CorsairAttack}.




\section{The economical system}

\aparag The economical system in \emph{Europa Universalis} tries to reproduce
the constant need to (short term) loans that countries of the epoch endured,
as well as the heavy strain caused by wars, bringing even large and rich
superpowers to bankruptcy and the brink of ruin. Inflation is usually very
high (due to the massive amount of gold and silver coming from America) thus
preventing any country from stockpiling large amount of money (these quickly
lose their value because of inflation).

\aparag The count unit is the \terme{ducat}, written \Ducats.



\subsection{Economic Record Sheets}

\aparag Each player has two sheet, separated in three actual \EcoRS.
\bparag \EcoRS A keeps track of the \terme{Royal Treasure} (\RT) of a
country. This represents the amount of gold stockpiled in the King's
chests\ldots or the amount of debts he has. The final value of the \RT is
carried over from one turn to the other.
\bparag \EcoRS B is used to compute the income and expenses during each
turn. The values are useless after the end of the turn.
\bparag \EcoRS C (below \EcoRS A) is used to record loans. This information
carries over from one turn to the following one.

\aparag Each column of the \EcoRS is used for one turn only. Information for
the following turn should be written in the next column.

\aparag The lines of the \EcoRS are organised in turn order. They should be
filled from top to bottom.
\bparag The turn starts on top of \EcoRS A, there is a ``break'' after line
\ERSshort{RT after Diplomacy} for computation of incomes and expenses on
\EcoRS B.

\aparag The \RT can well be negative (representing debts). This does not cause
extra trouble (as long as the amount of the debt is not ``too big'').



\subsection{A Three stage process}

\aparag The bulk of the economical system works in a three stage process
during each turn.
\bparag First, the \emph{gross income} is computed.
\bparag Then, the \emph{expenses} are computed.
\bparag Lastly, the \emph{Exchequer test} tells how well the taxes were
collected and the monarchs try to find money to fill the gaps.

\NPincludeA4{accounting-rt}{Economic Record sheet A and
  C}{economic-record-sheet-rt} \NPincludeA4{accounting-income}{Economic Record
  sheet B}{economic-record-sheet-income}


\subsubsection{Incomes}
\aparag Income is computed during the Administrative phase. Income comes from
various sources such as:
\bparag Provinces income (basically, taxing the peasants and the artisans).
\bparag Industrial income (European gold mines and manufactures).
\bparag Trade income (\TradeFLEET and trade centres).
\bparag \ROTW income (Colonial establishments and the resources they exploit).

\aparag Lines in \EcoRS B are grouped by kind of income in order to make
partial sums and ease computation of the total.
\bparag Income does not vary much from one turn to another (except for exotic
resources). Hence, most of the time computing income is done by copying the
previous column.

\aparag Gold from the \ROTW is not received as regular income. It must be
physically brought back to Europa and then arrives directly into the \RT (this
is better).

\aparag The total income is called the \terme{Gross income} it is computed on
line \ERS{Gross income B} and copied back on line \ERS{Gross income A}.


\subsubsection{Expenses}
\aparag Expenses come in two kind:
\bparag The administrative expenses are used to maintain troops and buy new
ones and to develop commercially or industrially a country.
\bparag The military expenses are used to move troops during wars. This
quickly becomes \textbf{very} expensive.

\aparag Administrative expenses are written and computed during the
administrative phase.
\bparag They include loan interest and loan refund.

\aparag Military expenses are computed during the military phase.


\subsubsection{The Exchequer test}
\aparag The gross income computed during the administrative phase is only a
rough approximation of what will be available if taxes go well.
\bparag At the end of the turn, each country perform the Exchequer test to
discover how well the taxes went and how much money really made it to the
palace.
\bparag Low stability and wars tend to make taxes go wrong.

\aparag The Exchequer test splits the incomes into three parts: the
\terme{regular} income, the \terme{prestige} income and the \terme{national
  loan} income.
\bparag These incomes are expressed in percentages (of the gross income).
\bparag It is possible (and intended) that these three percentages sum up to
more (or less) than 100\%.

\aparag[The regular income] is used first to cover for the expenses.
\bparag In the rare cases where the regular income is larger than the
expenses, the surplus can be stockpiled into the \RT.

\aparag[The prestige income] may be used to cover for the expenses.
\bparag Any part of it that is not spent to cover for expenses (either surplus
or voluntarily kept) is immediately spent for ``prestige'' expenses such as
building palaces or churches, organising receptions, \ldots
\bparag Prestige expenses provide \VPs at the end of each period.

\aparag[The national loan] income is the maximum amount of money that can be
borrowed from national nobles and burgers.
\bparag National loans are never mandatory and each country always choose how
much to borrow (within the limit of this income).
\bparag Money from national loans can be used to pay for expenses or can be
stockpiled into the \RT.
\bparag Loans have a fixed interest rate of 10\% that must be payed each turn
until the loan is refunded.
\bparag Refunding of national loans is never mandatory. A country may choose
to continue paying interests each turns rather than refunding its
citizens\ldots
\bparag However, having too many loans tends to hamper further Exchequer
tests.

\aparag[International loans] Sometimes, especially when the Exchequer test
went poorly, a country will be in dire need for money and can ask for an
international loan.
\bparag International loans give money that can be used to pay for expenses or
to be stockpiled in the \RT.
\bparag International loans also have a fixed interest rate of 10\%.
\bparag International loans, however, must be refunded within 15 years (3
turns).


\subsubsection{Bankruptcy and inflation}
\aparag[Bankruptcy] During the administrative phase, countries can choose to
go bankrupt.
\bparag Sometimes, the poor economical situation of a country (too many loans
and debts) forces it to go bankrupt.
\bparag Bankruptcies allow a country to erase some or all of its loans and
debts.
\bparag However, it usually cause some agitation (loss of \STAB) and
economical disarray (loss of \TradeFLEET and \MNU) as well as a slight
dishonour (loss of \VPs).

\aparag[Inflation] At the end of each turn, inflation will decrease the amount
of the \RT.
\bparag It is worth noticing than even a negative \RT will suffer from
inflation. Going into debts is not a good way to get ride of inflation.
\bparag Because of inflation, each country needs to get at least some money
into its \RT each turn.

\begin{exemple}[Exchequer test]
  Country A has a gross income of 300\Ducats. It has a total expenses of
  200\Ducats (includes administration, loans interest and refund and military)
  and 40\Ducats in \RT. During the Exchequer test, the results give 50\% in
  regular income, 40\% in prestige income and 20\% in loans (a good
  result). Thus, its regular income is 150\Ducats (50\% of 300), its prestige
  income is 120\Ducats and its maximal national loan amount is 60\Ducats.

  Money must be spent first from the regular income. So the 150\Ducats of the
  regular income are used and 50\Ducats of expenses remains. There are several
  solution to cover this:
  \begin{itemize}
  \item It is possible to use 50\Ducats of the prestige income to pay for
    expenses. The 70\Ducats remaining must be spent for prestige \VPs and
    cannot go into the \RT. No loan is required and so none is contracted and
    inflation has to be payed from the 40\Ducats of \RT.
  \item It is also possible to choose to get more \VPs by spending all the
    120\Ducats of prestige income into \VPs. A new loan of 60\Ducats can be
    contracted to pay for the remaining expenses (50\Ducats) and get 10 extra
    \Ducats into the \RT (to pay for inflation).
  \item Another possibility is to spent all the prestige income in \VPs but
    forgo the loan. The remaining 50\Ducats of expenses is then payed from the
    \RT (thus going to -10\Ducats) and inflation will lower it some more
    (probably not the wisest choice).
  \end{itemize}
  Several other possibilities exists and are up to the player choice. Notice
  that the easiest way to get money back into the \RT (to pay for inflation
  and Diplomacy) is to contract a new loan.
\end{exemple}

\begin{exemple}[The loan trick]
  Loan refund being expenses, they can be payed by the prestige income thus
  giving a process to circumvent the prohibition of putting prestige money in
  the \RT:\par
  A country has 100\Ducats in loans and 200\Ducats in incomes. Its expenses
  are 80\Ducats and it choose to spend an extra 50\Ducats to refund loans
  (this must be chosen before the military phase, so do it carefully while at
  war!) Thus, its total expenses is 130\Ducats.

  The Exchequer test gives 80\Ducats of regular income, 60\Ducats of prestige
  income and 60\Ducats of loans. The 80\Ducats of regular income is spent and
  50\Ducats of expenses remains. These can be covered from the prestige income
  (and 10\Ducats of prestige is spent for \VPs). A new loan of 50\Ducats is
  contracted and all its amount can go into the \RT. The net effect on the
  loans is null and all happened as if prestige income went into the \RT\ldots
  But to do this trick you need to have some loans to refund (and re-contract
  immediately) and you need to carefully estimate the Exchequer test as a bad
  result can cause a huge hole in your \RT. Trying to be too greedy when doing
  this is a good way to force you to take an international loan.
\end{exemple}

\begin{exemple}[When things go wrong: international loans]
  A country has 500\Ducats of gross income. Being at war, and forced to pay
  the interest of previous loans give a total expenses of 700\Ducats (those
  armies and fleet don't move for free). Fortunately, the \RT is still quite
  OK at 10\Ducats.

  The Exchequer test goes badly (as often in wars) and gives a result of
  30\%/20\%/40\%. So the regular income is 150\Ducats, the prestige income is
  100\Ducats and a maximum loan of 200\Ducats.

  The regular income covers a small part of the expenses and the prestige
  income is also quickly swallowed to pay for some troops rather than
  receptions\ldots 450\Ducats still need to be found. The burgers reluctantly
  loan 200\Ducats still leaving a 250\Ducats large ``hole''in the budget.

  Rather than going badly into debt (usually a bad idea), the country choose
  to appeal to Genoese and Dutch bankers and manage to get a 140\Ducats
  loan. Not bad but still not sufficient to cover the expenses. Moreover, the
  international loan must be refunded within 15 years (hopefully this won't
  happen in the middle of another war).

  The 140\Ducats of the international loan cover for part of the expenses but
  the extra 110\Ducats must be taken from the \RT, bringing it to -100\Ducats
  with 340\Ducats more in loans than at the start of the turn. Another similar
  turn and bankruptcy will knock on the door. Maybe now is a good time to try
  and make peace after all\ldots
\end{exemple}




\section{The Great Discoveries}

% RaW: 59, 60
Countries can launch great expeditions in order to discover and explore the
New World as well as find new routes to India and its riches. After the
exploration, they may invest in order to build trading post and increase their
hold on the spice and sugar trade or they may try to colonise the New World to
either exploit gold or populate it.

\aparag[Exploration] During the military phase, countries may send stack in
the \ROTW to try and discover new seas and new lands.
\bparag The presence of an explorer (on sea) or a conquistador (on land)
greatly increases the chance of success of the voyage.
\bparag New discoveries have to be brought back to known areas in order to be
effective.
\bparag See~\ruleref{chMilitary:Discoveries}.



\subsection{Forts}

Forts are small fortifications that bring little protection but are
inexpensive to build and maintain.

\aparag[Building forts] Forts may be built by \LD during the military phase in
the \ROTW only.
\bparag Forts are considered to be fortresses of level 0.

\aparag[Number of forts] The number of forts a country may have in play is
limited by the number of counters only.
\bparag Forts are free to build.
\bparag Each fort costs 1\ducats per turn to maintain.

\aparag[Forts and supply] Forts are supply sources for \LD and \ND only. They
do not provide supply for \ARMY or \FLEET counters.
\bparag Coastal forts are considered as ports for \ND, but not for \FLEET.



\subsection{Colonies and Trading posts}

% RaW: 59, 60
\label{chTime:COL TP}
Colonies (\COL) and Trading Post (\TP) are placed on the map after payment
(and success) of an administrative Colonisation or Trading post implantation
operation.

% If the result on the table for a Colonisation attempt imposes a combat with
% natives and that the latter are not destroyed or do not retreat, the Colony
% cannot be placed.

A newly implanted \COL or \TP begins at level 1, then it progresses according
to actions of Colonisation or Trading post implantation of the owning country
until it reaches the level 6 (maximum).

A \COL generates an income according to its level (1 \ducats per level), and
the wealth of the Area (simple or double income according to the \COL side),
as well as from exotic resources that it can exploit. It may also be
fortified.

% This operation depends especially on the value of the Foreign Trade Index
% (\FTI) of the player and the value of native tolerance of the Area where the
% Trading Post is going to be placed.

A \TP generates an income (1 or 2 \ducats according to its side) and exploits
most exotic resources (except square-shaped resources of \continent{America}
and \continent{Africa}).

\TP are vulnerable to actions of Competition from the other players. They can
also be burnt down during wars. \COL are more resilient and can only be
exchanged as peace condition or in dowry.


\subsubsection{Description}
\aparag[Number of establishments] For each period, the maximum number of \COL
and \TP counters is limited and can never be exceeded
(See~\ruleref{chThePowers:Period Limits}).

\aparag[Placement] New \COL or \TP may only be placed on map via the
colonisation or trading-post administrative action.
\bparag Using a conquistador, missionary or explorer greatly improves the
chances of success.
\bparag \TP may be turned into \COL in some cases as a colonisation action.

\aparag[Level of an establishment] A newly placed \COL or \TP is automatically
level 1, and placed side \Facemoins on the map.
\bparag Each following successful colonisation or trading-post operation
increase its level by 1.
\bparag Up to level 3, the \COL/\TP is placed side \Facemoins on the map.
\bparag From level 4 up, the \COL/\TP is placed side \Faceplus on the map.
\bparag A \COL/\TP may never have more than 6 levels.

\aparag[Establishments and port] Each \COL/\TP located in a coastal province
is considered to be also a port.
\bparag In case of provinces with multiple coasts (eg \granderegion{Panama}),
the principal coast, where the port is located, must be chosen upon creation
of the \COL/\TP.

\aparag[Fortifications] \COL/\TP may be fortified. A \fortress marker can be
built in a \COL/\TP provided the owning country has the required technology.
\bparag Unfortified \COL of level 5 of less and \TP are considered to have
only a fort as a fortification. A level 1 \fortress has to be built before
further fortifications.
\bparag The cost of construction and maintenance of a \fortress is double in
the \ROTW than in Europe.
% Jym, dans EU6: The cost of maintenance is the same (i.e. equal the fortress
% marker value, from 1 to 5).
\bparag In \TP and \COL of level 5 or less, only \fortress of level 1 or 2, or
the special arsenal-\fortress may be built.
\bparag Arsenal may be built in \COL/\TP instead of a \fortress of the same
level. They provide an arsenal (rather than a port) in addition to the
benefits of the \fortress. The limit of arsenals building is the counter
limit.

\aparag[Colony of Level 6] A \COL of level 6 is considered to be a European
province for all military purpose.
\bparag It has an intrinsic \fortress level of 1 even if no \fortress was
built.
\bparag \fortress of any level may be built in \COL of level 6. They cost the
normal (European) price both to build and to maintain.
\bparag It becomes known to every country without need to discover the
province.
\bparag The cost in \MP for entering the province is now computed as if it was
an European province and not a \ROTW one.
\bparag However, its income is still computed as a \COL (including
exploitation of gold or exotic resources).


\subsubsection{Colonial militia}
\aparag Each \COL has an intrinsic colonial militia of 1\LDE of conscripts per
2 levels plus 1\LDE is there is a mission.
% \bparag Exception: \POR has 1\LDE per level of veterans.
% \bparag Exception: \FRA has veteran militia.
\bparag Militias have the same military feature (technology, class, \ldots) as
the player owning the \COL.

\aparag[Utilisation of Militias] These colonial militias can never leave their
Colony of origin. They are never counted in stacking limit.
\bparag Militias can either stay within the fortress or fight in the field (eg
to try and repulse a landing party) at controller's choice.
\bparag In case of combat, just add the force of militias to military units
already present in the Colony (up to 8\LD participating in the battle on each
side).
\bparag If militia are lost, use the generic militia counters (white) to
remember how many \LDE are still present.
\bparag If the \COL is still controlled by its legitimate owner, militias are
automatically reconstituted for free at the end of turn.


\subsubsection{Destruction of TP}
\aparag \TP that are military occupied during wars may be destroyed by the
occupant during the redeployment phase.

\aparag It is possible to do concurrence action on \TP. Each successful
concurrence action reduce the level of the \TP by 1.
\bparag When the \TP reach level 0, remove it from the game.

% Jym: removed, now fort (f0) for all levels.  A Trading Post of level 6
% possesses an intrinsic level fortification 1. No fortress marker is then
% necessary.

% Jym: ?  A fortress marker may only be placed on a Trading Post if the
% fortress level is at most equal to half the level of the Trading Post
% (rounded up).

% Only a Trading Post of level 6 may be fortified up to fortress levels 4
% and/or 5 (if allowed by the technology).



\subsection{Exploitation of Exotic resources}

\label{chThePowers:ResourcesExploitation}
\aparag[Resources] Exotic resources are exploited by Colonies (\COL),
Trading-Posts (\TP) or Manufactures (\MNU). There are ten such resources:
\RES{Cotton}, \RES{Fish}, \RES{Furs}, \RES{Products of America}, \RES{Products
  of Orient}, \RES{Salt}, \RES{Silk}, \RES{Slaves}, \RES{Sugar}, \RES{Spices}.
\bparag \RES{Salt} and \RES{Fish} are the only resources that are also
exploited through \MNU, in Europe. All other resources are only exploited in
the \ROTW.
\bparag The \terme{Exotic Resources Record Sheet} keeps track of the
exploitation of Exotic Resources (globally) while each player has a
\terme{Colonial sheet} to record the exotic resources exploited by his
colonial establishments (and \MNU).
\aparag Each exotic resource has a price, recorded on the prices track (on the
\ROTW map).
\bparag Prices change each turn, partly because of the global economic
situation (one die roll) and partly because of a specific market situation
(one die roll per resource, depending on the exploited
quantity). See~\ref{chExpenses:Variation} for details.
\bparag Exotic resources bring to each country an income equal to the product
of the price of the resource and the quantity exploited by that country.

\aparag Exotic resources are depicted on the \ROTW map by a symbol as well as
a number in a coloured shape. The number indicates how many resources of this
type are available while the shape indicates how the resource can be
exploited.
\bparag The shape of the symbol of the resource indicates how it can be
exploited. See~\ref{chIncomes:Exotic ressources} for details about
exploitation of exotic resources.

\aparag[Monopoly on Exotic Resources]
\label{chThePowers:MonopolyExoticResources}
A country exploiting at least 6 resources of the same kind can claim a
monopoly.
\bparag A partial monopoly is if the country exploits at least as many units
as all other countries together.
\bparag A total monopoly is if the country exploits all exploited units but
two.
\bparag Resources exploited by minor countries are counted in the total, but
see~\ref{chDiplo:AdenOmanExoticResources}.

\aparag[Progressive appearance]\label{chThePowers:LateResources} Some
resources appear only late in the game: \RES{Sugar} (in \continent{Bresil}) in
1560 (turn 15, period III), \RES{Products of America} and \RES{Sugar}
(elsewhere) in 1615 (turn 26, period IV), \RES{Cotton} (in \continent{America}
and \continent{Indonesia}) in 1750 (turn 53, period VII).
\bparag In 1615 and 1750, when the resources appear, they appear at the rate
of exactly one resource per turn per \Area. If there are several possibilities
for a given \Area, the resource is determined at random.

\begin{exemple}
  \granderegionGuyana provides 3 resources that appear in 1615: 2 \RES{Sugar}
  and 1 \RES{Product of America}. Thus, in turn 26 only one of them (chosen at
  random) will be available; on turn 27, a second one (still chosen at random)
  appears and only in 1625 (turn 28) will the three resources be exploitable.

  \granderegionAntilles has 8 \RES{Sugar}. Thus, it will only reach its full
  production capacity by turn 33 (1650).

  In 1615, there are in \granderegionAntilles both colonies of \FRA and \ANG
  with enough levels to exploit 1 \RES{Sugar} each and a colony of \HOL with
  enough levels to exploit 2 \RES{Sugar}. The first \RES{Sugar} appears and 3
  countries can exploit it. Thus an automatic competition is done. The
  resource is finally exploited by \FRA.

  In 1620, a second \RES{Sugar} appear. \FRA still has the right to exploit
  the first one and keeps it, hence \FRA has no more free levels in
  \granderegionAntilles to exploit the new resource. An automatic competition
  in done between \ANG and \HOL only and the resource goes to \HOL.

  In 1625, a third resource appears. Both \ANG and \HOL still have the
  capacity to exploit it (because \HOL has enough levels to exploit 2
  \RES{Sugar}). Hence a new automatic competition is resolved and the resource
  goes again to \HOL.

  In 1630, \FRA managed to raise its \COL and can now exploit 2
  \RES{Sugar}. Thus, both \FRA and \ANG are able to exploit the fourth
  \RES{Sugar}. However, the players agree and \FRA forfeits its claim. The
  resource is exploited by \ANG.

  In 1635, the fifth resource appear and only \FRA can exploit it, thus taking
  it without need for competition.

  In 1700, several other \COL have been settled in \granderegionAntilles thus
  exploiting the 8 \RES{Sugar} there. \HIS wants a part of the trade and
  successfully create a new \COL. Since all the resources are already
  attributed, there is no automatic competition and \HIS will need to do some
  voluntary competition (spending money and action and angering other players)
  in order to exploit some \RES{Sugar}.
\end{exemple}

\aparag[Development of trade in India]\label{chIncomes:TradeIndia}
The trade of exotic resources changes during the game in India. Some events
may change the flux of goods towards south India and Bengal.
\bparag Before the fall of the Kingdom of \pays{vijayanagar} (due to
\eventref{pII:Mughal Expansions}, \eventref{pIII:Fall Vijayanagar},\ldots) one
can exploit only 1 \RES{Products of Orient} and 1 \RES{Spice} in
\granderegion{Karnatika}. A marker is used to show that the resources are
limited.
\bparag After the fall, one can exploit fully the 2 \RES{Products of Orient},
2 \RES{Spices} and 2 \RES{Cotton} (the later only after 1700).
\bparag Before \eventref{pIII:Mughal Akbar}, \granderegion{Bengale} is limited
to 1 resource of each type. Use a marker as reminder.
\bparag After \eventref{pIII:Mughal Akbar}, \granderegion{Bengale} can now
exploit 2 resources of each type. Flip the marker to its ``2'' side.
\bparag When a \COL is built in \ville{Calcutta} and either
\eventref{pIII:Mughal Akbar} has happened or this is period VI or VII, the
full potential of the \Area is reached (3 \RES{Products of Orient}, 3
\RES{Spices} and 3 \RES{Cotton}).

\aparag[Fishing and naval construction] The owner of many fisheries get some
reward from the great numbers of fishers in his country related to the
construction of military naval units (see \ruleref{chExpenses:Effect of Fish
  Monopoly Purchase} and \ruleref{chExpenses:Effect of Fish Purchase}):



\subsection{Trade of Wood}\label{chIncomes:Wood}

\aparag Wood is not an exotic resource. As such, it does not bring any income
coming from a fluctuating price. However, it is exploited in a similar way
(through colonial establishments or manufactures).

\aparag[Wood production] Wood is produced each turn, and cannot be kept. Each
unit of wood can be produced by either a \MNU or a \ROTW establishment.

\aparag[ROTW Wood] A \COL\faceplus or \TP\faceplus in an \Area with a
\RES{Wood} resource can be turned in a Wood factory: it brings the normal
income, and produces 1 unit of wood that can only be used by the owner of the
colony.
\bparag A wood factory \TP or \COL cannot exploit any exotic resources, nor a
gold mine. Several different players can produce wood in the same \Area, but
only one unit of \RES{Wood} can be produced per player per area.
\bparag The exploitation of wood begins with a simple declaration (as a
diplomatic announcement) and can only be stopped by the loss of the
establishment.
\bparag A minor country wood factory does provide the wood to its Patron (in
the case of \pays{portugal}).

\aparag[European Wood] A \RES{Wood} \MNU of level 1 produces 1 unit of Wood
that can only be used by the producer. A \RES{Wood} \MNU of level 2 produces
the same thing, plus 1 unit of Wood that can only be sold to a foreign
country.
\bparag The price is fixed to 10\ducats per unit, that goes in
\lignebudget{Wood and Slaves}.
\bparag The buyer must not be at war with the seller, and either the buyer has
a commercial fleet in the seller's \CTZ, or the seller has a commercial fleet
in the buyer's \CTZ.
\bparag For this purpose, \PRU, \POL and \SUE count the \stz{Baltique} as
their \CTZ, and \POR uses \ctz{Espagne}.

\aparag[Use of wood] Wood raises the naval construction limits and the free
maintenance (see~\ruleref{chExpenses:Effect of Wood Maintenance}
and~\ruleref{chExpenses:Effect of Wood Purchase}).



\subsection{Minors colonial politics}

\aparag Some minor countries did also attempt to colonise the new world with
more or less success.
\bparag These attempts are resolved using~\ref{eco:Rush:Minor colony}.


\subsubsection{Minor establishments}
\aparag[Effects] Minor establishment exploit 1 resource per side (whatever the
usual rules for this resource).

\aparag[Military] A Minor establishment \Facemoins is a fort with 1\LDE of
veteran militia (Latin, class \CAIII).
\bparag A Minor establishment \Faceplus is a fort with 2\LDE of veteran
militia (Latin, class \CAIII).
\bparag The militia fights out of the fort if and only if enemies are landing
in the province (it stays inside if the enemy comes by Land). It and can only
retreat in the fort.
\bparag Minor establishments never trigger native reaction or declaration of
war by \ROTW minors.

\aparag[Destruction] Any country may attack a Minor establishment during any
round.
\bparag Each establishment attacked during the turn costs 1\STAB (declaration
of oversea war against an unspecified country).
\bparag A Minor establishment is immediately destroyed if a country controls
it.
\bparag Minor establishment are never subject to competition, including
automatic competition.


\subsubsection{Pirate haven}
\aparag[Effects] Pirate haven (arsenals) give a malus to fight \pays{pirates}
\corsaire in the \CTZ they are located.

\aparag[Military] A Pirate haven \Facemoins is a fortress of level 1 with 1\LD
of conscript militia (Latin, class \CAIII).
\bparag A Pirate haven \Faceplus is a fortress of level 2 with 2\LD of
conscript militia (Latin, class \CAIII).
\bparag The militia fights out of the fortress if and only if enemies are
landing in the province (it stays inside if the enemy comes by Land). It can
only retreat in the fortress.
\bparag Pirate haven never trigger native reaction or declaration of war by
\ROTW minors.

\aparag[Destruction] Any country may attack a Pirate haven at no cost during
any round.
\bparag A Pirate haven is immediately destroyed if a country other than
\pays{pirates} controls it.

\clearpage




\section{The detailed game sequence}

\aparag Each game turn is composed of several phases, each of the phase is
subdivided into several segments.
\bparag The Military phase is instead composed of several rounds repeating the
same segments. The second segment of the phase is composed of one impulse per
alliance, each alliance performing the same actions (movement and battle) in
order during its impulse.

\aparag Theoretically, each segment must be completed before moving to the
next.
\bparag However, most of the diplomatic discussion and administrative stuff
(incomes and expenses) can be played simultaneously by all the players and do
not require strong synchronisation (it is common to have some players still
discussing while some other are already planning their administrative
actions).
\bparag Even the military phase can be de-synchronised when several distinct
wars are ongoing, but this require a bit more adaptation to deal with the
end-of-phase test.

\begin{designnote}
  The following chapters of the rulebook describe each of the phases and
  segments roughly in turn order.
\end{designnote}

\begin{todo}
  Turn the names into links to the corresponding chapter/section.
\end{todo}

\begin{multicols}{2}
  % RaW: [5,23]
  \GameSequence
\end{multicols}

% Local Variables:
% fill-column: 78
% coding: utf-8-unix
% mode-require-final-newline: t
% mode: flyspell
% ispell-local-dictionary: "british"
% End:

% LocalWords: Galleass
