% -*- mode: LaTeX; -*-
\definechapterbackground{Diplomacy and Revolts events}{events}
\chapter{Diplomacy and Revolts events}

\section{Diplomatic event tables}

\begin{tablehere}\centering
  \begin{tabular}{l|l}
    Roll & Result\\\hline
    1,4,7 & Also test for \xnameref{chEvents:diplomacy:uprising}\\
    1--3 & Catholics \Xcatholique\ (Christians \Xcatholique\Xorthodoxe\ %
    before the Reform)\\
    4--6 & Protestants \Xprotestant\ (Christians \Xcatholique\Xorthodoxe\ %
    before the Reform)\\
    7--9 & Muslims \Xsunnite\Xchiite\ \\
    10 & Other \Xautrereligion\ and a minor will possibly declare a war.
  \end{tabular}
  \caption{Troubled Religion table}\label{table:diplomatic event religion}
\end{tablehere}\par

\begin{tablehere}
  % No aliases, just the internal codes of countries
  \def\paysfid#1{\pays{#1}~(\theminorfid{#1})} \def\labelitemi{10.}
  \begin{enumerate}
  \item {\bf Northern Italy:} %
    1-\paysfid{genes} %
    2-\paysfid{montferrat} %
    3-\paysfid{modene} %
    4-\paysfid{lucca} %
    5-\paysfid{milan}.\\ %
    {\bf Balkans:} %
    6-\paysfid{hongrie} %
    7-\paysfid{moldavie} %
    8-\paysfid{valachie} %
    9-\paysfid{mazovie} %
% [BLP] removed
%    10-\paysfid{transylvanie}. %
  \item {\bf Southern Italy:} %
    1-\paysfid{papaute} %
    2-\paysfid{chevaliers} %
    3-\paysfid{toscane} %
    4-\paysfid{parme} %
    5-\paysfid{venise} %
    6-\paysfid{corse}.\\ %
    {\bf Middle East:} %
    7-\paysfid{arabie} %
    8-\paysfid{irak} %
    9-\paysfid{georgie} %
    10-\paysfid{mamelouks} %
    11-\paysfid{damas}. %
  \item {\bf Spanish road:} %
    1-\paysfid{suisse} %
    2-\paysfid{wurtemberg} %
    3-\paysfid{savoie} %
    4-\paysfid{treves} %
    5-\paysfid{cologne} %
    6-\paysfid{lorraine} %
    7-\paysfid{mayence} %
    8-\paysfid{liege}. %
  \item {\bf Northern \HRE:} %
    1-\paysfid{hollande} %
    2-\paysfid{hanovre} %
    3-\paysfid{hesse} %
    4-\paysfid{palatinat} %
    5-\paysfid{berg} %
    6-\paysfid{oldenburg}.\\ %
    {\bf America:} %
    7-\paysfid{iroquois} %
    8-\paysfid{inca} %
    9-\paysfid{azteque}. %
  \item {\bf Southern \HRE:} %
    1-\paysfid{baviere} %
    2-\paysfid{wurtemberg} %
    3-\paysfid{alsace} %
    4-\paysfid{bade} %
    5-\paysfid{thuringe} %
    6-\paysfid{habsbourg}. %
  \item {\bf Eastern \HRE:} %
    1-\paysfid{boheme} %
    2-\paysfid{brandebourg} %
    3-\paysfid{saxe} %
    4-\paysfid{brunswick} %
    5-\paysfid{pologne} %
    6-\paysfid{Vlithuanie} %
    7-\paysfid{Vpommeranie}.\\ %
    {\bf Asia:} %
    8-\paysfid{chine} %
    9-\paysfid{japon}. %
  \item {\bf Baltic shores:} %
    1-\paysfid{teutoniques1} %
    2-\paysfid{hanse} %
    3-\paysfid{danemark} %
    4-\paysfid{suede} %
    5-\paysfid{Vnorvege} %
    6-\paysfid{Vfinlande} %
    7-\paysfid{Vliflandie} %
    8-\paysfid{Veastprussia} %
    9-\paysfid{courlande} %
    10-\paysfid{pologne}.\\ %
    {\bf Atlantic shores:} %
    11-\paysfid{portugal} %
    12-\paysfid{hollande} %
    13-\paysfid{ecosse} %
    14-\paysfid{Virlande} %
    15-\paysfid{Vbelgique}. %
  \item {\bf Khanates:} %
    1-\paysfid{ryazan} %
    2-\paysfid{pskov} %
    3-\paysfid{steppes} %
    4-\paysfid{cosaquesdon} %
    5-\paysfid{kazan} %
    6-\paysfid{astrakhan} %
    7-\paysfid{crimee} %
    8-\paysfid{ukraine}.\\ %
    {\bf India:} %
    9-\paysfid{gujarat} %
    10-\paysfid{vijayanagar} %
    11-\paysfid{mysore} %
    12-\paysfid{hyderabad}. %
  \item {\bf North Africa:} %
    1-\paysfid{maroc} or (10) %
    2-\paysfid{algerie} %
    3-\paysfid{tunisie} %
    4-\paysfid{tripoli}%
    5-\paysfid{cyrenaique}.\\ %
    {\bf Semi-major countries:} %
    6-\paysfid{suede} %
    7-\paysfid{brandebourg} %
    8-\paysfid{danemark} %
    9-\pays{perse}+\paysfid{ormus} %
    10-\paysfid{portugal} %
    11-\paysfid{pologne}. %
  \item {\bf Eastern Muslims:} %
    1-\pays{perse}+\paysfid{ormus} %
    2-\paysfid{aden} %
    3-\paysfid{oman} %
    4-\paysfid{soudan} %
    5-\paysfid{mogol} %
    6-\paysfid{afghans} %
    7-\paysfid{maroc} or (10) %
    8-\paysfid{algerie} %
    9-\paysfid{tunisie}. %
  \end{enumerate}
  \caption{Diplomatic table}\label{table:diplomatic event}
\end{tablehere}

\clearpage

\section{Revolts tables}
\subsection{Summary of the procedure}
% Rules moved in the corresponding chapter. Only reminder here.

\aparag Roll 2d10 and read the revolted country in the column of the current
period. The target country may be a \MIN or other abstract entity in which
case a pseudo-stability is provided in brackets.
\bparag Decrease this pseudo-stability of minors in the table by \bonus{-1}
if:
\begin{itemize}
\item This is \HOLhol and \SPA perceived the taxes at the preceding turn;
\item This is \PORpor at the turn of \ref{pIII:Portuguese Disaster} or after.
\end{itemize}
\aparag Roll 1d10+the \STAB (or modified pseudo-stability) on the target
country's table. Reroll in the description of groups below if needed.

\aparag Lastly, roll 2d10 in the last column of the table below to find the
strength of the revolt.

% Jym: I prefer to keep the "higher roll, harder revolt" paradigm than to have
% exact translation of the percentages I'd like. The translation of R-G might
% be somewhat harder, so its higher global percentage is OK.

% #faces of R = min rounds needed to crush. Troops make it harder because
% battle may be lost. No more than A- in order to prevent major victories and
% cheap STAB...

% old strength -> new strength 1-3 R-
% (30%) -> (36%) LD (1-2,3%) / A- (3-4,7%) / R- (5-9,26%)
% 4-7 R-G (30%) -> (36%) R-G (11,10%) / R-LD (10,9%) / R- A- (12,9%) / R- A-G
% (13,8%)
% 8 R+ (10%) -> (7%) R+ (14,7%) /
% 9 R+G (10%) -> (11%) R+G (15,6%) / R+ A- (16,5%)
% 0 R+GfLD (10%) -> (10%) R+GfLD (18-20,6%) / R+G A-G (17,4%)

% TODO: replace these by real commands with link to the revolt table rather
% than the specific rules.

\newcommand{\ANGrev}{\ANG} \newcommand{\AUSrev}{\AUS}
\newcommand{\DANrev}{\DAN} \newcommand{\FRArev}{\FRA}
\newcommand{\HISrev}{\HIS} \newcommand{\HOLrev}{\HOL}
\newcommand{\POLrev}{\POL} \newcommand{\PORrev}{\POR}
\newcommand{\PRUrev}{\PRU} \newcommand{\ROTWrev}{COL\xspace}
\newcommand{\RUSrev}{\RUS} \newcommand{\SUErev}{\SUE}
\newcommand{\TURrev}{\TUR} \newcommand{\VENrev}{\VEN}

\begin{designnote}
  The \ROTWrev revolt area is mutually exclusive with both
  \eventref{pIV:Revolt Singala}, \eventref{pV:Slave Revolts WI},
  \eventref{pVI:Slave Revolts WI} and \eventref{pVII:Revolt Indonesia}. If
  using the \ROTWrev revolt area, consider these events as \RD. If not, reroll
  the revolt area whenever \ROTWrev occurs.
\end{designnote}

\subsection{Global revolt table}
\begin{tablehere}\centering\graytabular%
  \begin{tabular}{|c|ccccccc|c|} \hline%
    ~ & I & II & III & IV & V & VI & VII & Strength\\\hline\ghline%
    2 & \SUErev[0] & \PORrev & \FRArev & \FRArev & \PRUrev[0] & \PRUrev[0] &
    \ANGrev & \LD\\\ghline% 1%
    3 & \SUErev[0] & \PORrev & \FRArev & \AUSrev[-1] & \PORrev & \ANGrev &
    \POLrev[-2] & \LD\\\ghline% 2%
    4 & \AUSrev[-1] & \SUErev[-1] & \ANGrev & \PRUrev & \VENrev & \VENrev &
    \PRUrev & \ARMY\facemoins\\\ghline% 3%
    5 & \AUSrev[-1] & \SUErev[-1] & \SUErev & \PORrev & \PRUrev & \PRUrev &
    \ANGrev & \ARMY\facemoins\\\ghline% 4%
    6 & \PORrev & \PRUrev[+3] & \PRUrev[+3] & \HOLrev & \SUErev & \SUErev &
    \AUSrev & \REVOLT\facemoins\\\ghline% 5%
    7 & \ANGrev & \ANGrev & \SUErev & \PORrev[-1] & \POLrev & \POLrev[0] &
    \PRUrev & \REVOLT\facemoins\\\ghline% 6%
    8 & \VENrev & \VENrev & \VENrev & \VENrev[+2] & \AUSrev & \AUSrev &
    \SUErev & \REVOLT\facemoins\\\ghline% 7%
    9 & \FRArev & \HISrev & \HISrev & \HISrev & \HISrev & \HISrev & \HISrev &
    \REVOLT\facemoins\\\ghline% 8%
    10 & \HISrev & \FRArev & \PORrev[-1] & \FRArev & \ANGrev & \ANGrev &
    \POLrev[-2] & \REVOLT\facemoins/\LD\\\ghline% 9%
    11 & \HOLrev[-1] & \HOLrev[-2] & \HOLrev[-3] & \POLrev & \ROTWrev[0] &
    \ROTWrev[0] & \ROTWrev[+3] & \REVOLT\facemoins\LeaderG\\\ghline%10%
    12 & \ANGrev & \ANGrev & \ANGrev & \ANGrev & \RUSrev & \RUSrev &
    \POLrev[-2] & \REVOLT\facemoins/\ARMY\facemoins\\\ghline% 9%
    13 & \RUSrev & \POLrev & \POLrev & \RUSrev & \PORrev & \FRArev & \FRArev &
    \REVOLT\facemoins/\ARMY\facemoins\LeaderG\\\ghline% 8%
    14 & \TURrev & \TURrev & \RUSrev & \SUErev & \POLrev & \POLrev[0] &
    \HOLrev & \REVOLT\faceplus\\\ghline% 7%
    15 & \POLrev & \AUSrev[+1] & \AUSrev[+1] & \TURrev & \TURrev & \TURrev &
    \TURrev & \REVOLT\faceplus\LeaderG\\\ghline% 6%
    16 & \PORrev & \RUSrev & \TURrev & \AUSrev[+1] & \HOLrev & \HOLrev &
    \RUSrev & \REVOLT\faceplus/\ARMY\facemoins\\\ghline% 5%
    17 & \POLrev & \AUSrev[-2] & \AUSrev[-2] & \TURrev & \TURrev & \TURrev &
    \TURrev & \REVOLT\faceplus\LeaderG/\ARMY\facemoins\LeaderG\\\ghline% 4%
    18 & \TURrev & \TURrev & \RUSrev & \ROTWrev[-3] & \FRArev & \PORrev &
    \HOLrev & \REVOLT\faceplus\LeaderG\fortress\LD\\\ghline% 3%
    19 & \VENrev & \VENrev & \VENrev & \ROTWrev[-3] & \FRArev & \FRArev &
    \FRArev & \REVOLT\faceplus\LeaderG\fortress\LD\\\ghline% 2%
    20 & \HISrev & \FRArev & \PORrev[-1] & \AUSrev[-2] & \RUSrev & \RUSrev &
    \PRUrev & \REVOLT\faceplus\LeaderG\fortress\LD\\\hline\ghline% 1%
    % ANG FRA HIS POR SUE HOL AUS VEN TUR RUS POL PRU ROTW
    % _15 __8 _10 _10 __3 _10 __7 __9 _10 __8 _10 __0 ___0
    % _15 _10 __8 __3 __7 _10 _10 __9 _10 __5 __8 __5 ___0
    % _12 __3 __8 _10 _10 _10 _10 __9 __5 _10 __8 __5 ___0
    % __9 _10 __8 _10 __7 __5 __8 __7 _10 __8 _10 __3 ___5
    % __9 __5 __8 _10 __5 __5 __7 __3 _10 _10 _13 __5 __10
    % _11 _10 __8 __3 __5 __5 __7 __3 _10 _10 _13 __5 __10
    % __5 _10 __8 __0 __7 _10 __5 __0 _10 __5 _20 _10 __10
  \end{tabular}\par
  \caption{Revolt table: target area and
    strength}\label{table:alt-revolt-global}
\end{tablehere}

\begin{designnote}
  Here's the percentages of each country being rolled in each period:\\
  \begin{tabular}{|r|rrrrrrrrrrrrr|}
    \hline
    ~ & \ANGrev & \FRArev & \HISrev & \PORrev & \SUErev & \HOLrev & \AUSrev
    & \VENrev & \TURrev & \RUSrev & \POLrev & \PRUrev & \ROTWrev\\
    \hline
    \period{I} & 15 & 8 & 10 & 10 & 3 & 10 & 7 & 9 & 10 & 8 & 10 & 0 & 0\\
    \period{II} & 15 & 10 & 8 & 3 & 7 & 10 & 10 & 9 & 10 & 5 & 8 & 5 & 0\\
    \period{III} & 12 & 3 & 8 & 10 & 10 & 10 & 10 & 9 & 5 & 10 & 8 & 5 & 0\\
    \period{IV} & 9 & 10 & 8 & 10 & 7 & 5 & 8 & 7 & 10 & 8 & 10 & 3 & 5\\
    \period{V} & 9 & 5 & 8 & 10 & 5 & 5 & 7 & 3 & 10 & 10 & 13 & 5 & 10\\
    \period{VI} & 11 & 10 & 8 & 3 & 5 & 5 & 7 & 3 & 10 & 10 & 13 & 5 & 10\\
    \period{VII} & 5 & 10 & 8 & 0 & 7 & 10 & 5 & 0 & 10 & 5 & 20 & 10 & 10\\
    \hline
  \end{tabular}
\end{designnote}

\clearpage

% Jym: Some revolts are much harder to fight. Typically the ROTW one require
% more effort than in Europe.  Those should not be in the 13 line. The 13 line
% should only create "easy" revolts. 1/ Bonus. 2/ +3 STAB can be somewhat
% avoided thus preventing the hard revolts to occur (we don't want MKL abusing
% the system that way).

% I'm basically swapping the 12 and 13 lines where ever I think it should be
% done.




%\section{Countries revolt tables}
%\label{chapter:events:revolt:Country tables}

\subsection{Revolt table for \ANG}

\aparag When a \REVOLT occurs in \ANG, roll on this table, in the column of
the current period.

\begin{tablehere}
  \defgroup{CE}{\group{Central England}}{1.~\provinceKent,
    2--3.~\provinceLincolnshire, 4.~\provinceWessex,
    5--6.~\provinceGloucester, 7--10.~\province{East Anglia}}%
  \defgroup{WE}{\group{Western England}}{1--4.~\provinceCornwall,
    5--8.~\provinceCymru, 9--10.~\provinceMidlands}%
  \defgroup{NE}{\group{Northern England}}{1--3.~\provinceYorkshire,
    4--6.~\provinceCumberland, 7--9.~\provinceDurham,
    10.~\provinceLancashire}%
  \defgroup{PE}{\group{English Provinces}}{1.~\provinceLincolnshire,
    2.~\provinceWessex, 3.~\provinceGloucester, 4.~\provinceCornwall,
    5.~\provinceCymru, 6.~\provinceMidlands, 7.~\provinceYorkshire,
    8.~\provinceCumberland, 9.~\provinceDurham, 10.~\provinceLancashire}%
  \defgroup{II}{\group{Inner Ireland}}{1--5.~\provinceBrega,
    6--10.~\provinceLaighean}%
  \defgroup{OI}{\group{Outer Ireland}}{1--3.~\provinceMumhan,
    4--6.~\provinceConnacht, 7--10.~\provinceUladh}%
  \defgroup{LS}{\group{Low Scotland}}{1--4.~\provinceAyr,
    5--7.~\provinceLothian, 8--10.~\provinceGalloway}%
  \defgroup{HS}{\group{High Scotland}}{1--4.~\provinceHighlands,
    5--7.~\provinceAlba, 8--10.~\provinceMoray}%
  \defgroup{RE}{\group{Europe}}{A random English European province not in
    Great-Britain/Ireland (possibly including \payshanovre); if none,
    \ctz{Angleterre}}%
  \defgroup{Wy}{\provinceCymru}{}%
  \defgroup{Wc}{\provinceCornwall}{}%
  \defgroup{Cl}{\province{East Anglia}}{}%
  \defgroup{Sa}{\ctz{Angleterre}}{}%
  \defgroup{FS}{\group{French Soil}}{1.~\provinceGuyenne,
    2--4.~\provinceFinistere, 5--7.~\provinceArmor, 8--10.~\provincePicardie}%
  \defgroup{AS}{\group{Scotland}}{1--3.~\provinceAyr, 4--5.~\provinceLothian,
    6.~\provinceGalloway, 7--8.~\provinceHighlands, 9.~\provinceAlba,
    10.~\provinceMoray}%
  % \defgroup{Am}{\group{America}}{Any \TP/\COL of \ANG in
  % \continent{America}}%
  % Feedback for American Revolution victory/defeat.  Induced change :
  % replaced a OI by RO in pV,VI (was 4 OI) and a As by RO in pVII. Avoid safe
  % COL/TP elsewhere in America.
  \defgroup{Am}{\group{America}}{A random \COL/\TP (of any nationality) in the
    following area: 1--2.~\granderegionAmerica, 3--4.~\granderegionVirginia,
    5--6.~\granderegionCarolina, 7--8.~\granderegionAntilles,
    9--10.~\granderegion{Terre-Neuve} or \granderegionHudson}
  % America is both a continent and an area...  (should we fix this ?)
  \defgroup{As}{\group{Asia}}{A random \TP/\COL of \ANG not in continent
    \continent{America}}%
  \defgroup{RO}{\group{ROTW}}{A random \TP/\COL of \ANG; if none, \Sa}%
  \centerline{%
    \graytabular\begin{tabular}{|c|ccccc|}\hline%
      Result & I & II & III, IV & V, VI & VII\\\hline\ghline%
      <0& \CE & \Cl & \CE & \CE & \PE \\\ghline%
      0 & \CE & \CE & \NE & \PE & \AS \\\ghline%
      1 & \Wy & \Wy & \WE & \OI & \RE \\\ghline%
      2 & \Sa & \NE & \OI & \HS & \OI \\\ghline%
      3 & \Wc & \Wc & \WE & \LS & \II \\\ghline%
      4 & \CE & \CE & \PE & \II & \RO \\\ghline%
      5 & \NE & \NE & \LS & \OI & \Am \\\ghline%
      6 & \OI & \OI & \OI & \PE & \PE \\\ghline%
      7 & \II & \II & \II & \II & \II \\\ghline%
      8 & \NE & \Sa & \LS & \AS & \As \\\ghline%
      9 & \WE & \WE & \II & \RO & \Am \\\ghline%
      10& \FS & \LS & \OI & \RE & \RE \\\ghline%
      11& \OI & \FS & \HS & \As & \RE \\\ghline%
      12& \LS & \NE & \RO & \Am & \OI \\\ghline%
      13& \Wc & \OI & \OI & \OI & \OI \\\hline\ghline%
      % pI : CE 5>3 WE 3>4 NE 2=2 II 2>1 OI 3>2 LS 0>1 FS 0>1 CTZ 0>1
      % pII: CE 4>3 WE 1>3 NE 1>2 II 3>1 OI 5>2 LS 1>2 FS 0>1 CTZ 0>1
      % pIII:CE 1=1 WE 2=2 NE 0>1 II 2=2 OI 6>5 LS 3=3 HS 1=1
      % pVI: CE 1=1 PE 0>1 II 2=2 OI 6>5 LS 3>2 HS 3>3 RE 0>1
      % pVII:PE 0>1 RE 3=3 II 2=2 OI 5>4 RO 5>4
    \end{tabular}}
  \showgrouplist{Am,As,CE,PE,RE,FS,HS,II,LS,NE,OI,RO,AS,WE}{}
  \caption{Revolt table for \ANG}\label{table:alt-revolt-england}
\end{tablehere}

\clearpage



\subsection{Revolt table for \FRA}

\aparag When a \REVOLT occurs in \FRA, roll on this table, in the column of
the current period.

\smallskip

% small -> catholics area (incl Paris), large -> protestant area.  the roundly
% revolt of FWR only spans if in the correct minor.
\aparag For the roundly revolts caused by~\eventref{pIII:FWR}, always use the
column for period III (even if it occurs during another period).
\bparag Moreover, if \FRA is catholic, \textbf{subtract} its \STAB rather than
adding it to find the localisation of the revolts caused by this event.
% No Tr/Sa in pIII as III-D already span P+ in the CTZ.

\begin{tablehere}
  % Catalogne -> HIS table.
  \defgroup{Pa}{\province{Ile-de-France}}{}%
  \defgroup{BR}{\group{Brittany}}{1--4.~\provinceArmor,
    5--7.~\provinceFinistere, 8--10.~\provinceMorbihan}% FRA
  \defgroup{SE}{\group{Midi}}{1--3.~\provinceCevennes,
    4--6.~\provinceLanguedoc, 7--8.~\provinceDauphine,
    9--10.~\provinceProvence}% Hug+
  \defgroup{SR}{\group{Spanish Road}}{1--2.~\provinceBresse,
    3--4.~\province{Franche-Comte}, 5--6.~\provinceAlsace,
    7--8.~\provincePfalz, 9--10.~\provinceLuxemburg}% out pIV+
  \defgroup{BE}{\group{Belgium}}{1--3.~\provincePicardie,
    4--6.~\provinceArtois, 7--8.~\provinceFlandre,
    9--10.~\provinceHainaut}% mix
  \defgroup{SW}{\group{Aquitaine}}{1--3.~\provinceBearn,
    4--6.~\provincePoitou, 7--8.~\provinceGuyenne,
    9--10.~\provinceQuercy}% Hug
  \defgroup{CF}{\group{Central France}}{1--2.~\provinceLyonnais,
    3--4.~\provinceAuvergne, 5--6.~\provinceLimousin, 7--8.~\provinceTouraine,
    9--10.~\provinceBerry}% mix
  \defgroup{EF}{\group{East}}{1--2.~\provinceBourgogne, 3--4.~\provinceTroyes,
    5--6.~\provinceChampagne, 7--8.~\provinceLorraine,
    9--10.~\provinceAlsace}% Lig
  \defgroup{NW}{\group{North West}}{1--2.~\provinceVendee,
    3--4.~\provinceMaine, 5--6.~\provinceNormandie, 7--8.~\provinceCaux,
    9--10.~\provinceOrleanais}% Lig+
  \defgroup{It}{\group{Italy}}{1--2.~\provinceBresse, 3--4.~\provinceSavoia,
    5--6.~\provinceLombardia, 7--8.~\provinceNice,
    9--10.~\provinceCorsica}% out, pII-,pVII
  \defgroup{Sa}{\ctz{France}}{}%
  \defgroup{Am}{\group{America}}{A random \COL/\TP (of any nationality) in the
    following area: 1--2.~\granderegionQuebec, 3--4.~\granderegion{Grands
      Lacs}, 5--6.~\granderegionMississippi, 7--8.~\granderegionAcadie,
    9--10.~\granderegion{Terre-Neuve} or \granderegionHudson}
  % only specified areas so Caraibes excluded. Nice if SYW is
  % lost and Canada is no more FRA. Bad if SYW is won. Good feedback.
  \defgroup{RO}{\group{\ROTW}}{A random \COL/\TP of \FRA ; \ctz{France} if
    none}%
  \centerline{\graytabular\begin{tabular}{|c|cccccc|}\hline%
      Result & pI,pII & pIII & pIV & pV & pVI & pVII \\\hline\ghline%
      <0 & \Pa & \Pa & \Pa & \Pa & \Pa & \Pa\\\ghline%
      0 & \NW & \EF & \NW & \CF & \CF & \CF\\\ghline%
      1 & \Sa & \NW & \SE & \SW & \SW & \SW\\\ghline%
      2 & \CF & \NW & \CF & \NW & \NW & \NW\\\ghline%
      3 & \CF & \EF & \CF & \CF & \CF & \CF\\\ghline%
      4 & \It & \EF & \NW & \BE & \BE & \BE\\\ghline%
      5 & \SW & \CF & \SW & \SE & \SE & \SE\\\ghline%
      6 & \SE & \CF & \EF & \BR & \SR & \SR\\\ghline%
      7 & \NW & \CF & \CF & \SR & \SW & \SW\\\ghline%
      8 & \SE & \SE & \SE & \SW & \BR & \BR\\\ghline%
      9 & \BR & \BR & \BR & \RO & \Sa & \Sa\\\ghline%
      10 & \EF & \SW & \BE & \SE & \SE & \BE\\\ghline%
      11 & \It & \SE & \RO & \EF & \RO & \RO\\\ghline%
      12 & \BR & \SW & \Am & \Am & \Am & \Am\\\ghline%
      13 & \BE & \SE & \SR & \BR & \EF & \It\\\ghline\hline
    \end{tabular}}
  \showgrouplist{Am,SW,BE,BR,CF,EF,It,SE,NW,RO,SR}{}
  \caption{Revolt table for \FRA}\label{table:alt-revolt-france}
\end{tablehere}

\clearpage



\subsection{Revolt table for \HIS}

\aparag When a \REVOLT occurs in \HIS, roll on this table, in the column of
the current period.

\begin{tablehere}
  \defgroup{HC}{\group{Central Castile}}{1--3.~\province{Castilla La Nueva},
    4--5.~\provinceToledo, 6--7.~\provinceSalamanca, 8.~\provinceLeon,
    9--10.~\province{Castilla La Vieja}}%
  \defgroup{HN}{\group{Northern Castile}}{1--2.~\provinceGaliza,
    3--4.~\provinceAsturias, 5--6.~\provinceVizcaya, 7--8.~\provinceNavarra,
    9--10.~\provinceBearn}%
  \defgroup{HS}{\group{Southern Castile}}{1--2.~\provinceCaceres,
    3--4.~\provinceExtremadura, 5--6.~\provinceHuelva,
    7--9.~\provinceAndalucia, 10.~\provinceGibraltar}%
  \defgroup{HE}{\group{Granada}}{1--4.~\provinceGranada,
    5--7.~\provinceCordoba, 8--9.~\provinceMurcia, 10.~\province{La Mancha}}%
  \defgroup{HA}{\group{Aragon}}{1--4.~\provinceAragon,
    5--8.~\provinceValencia, 9--10.~\province{Illes Balears}}%
  \defgroup{HB}{\group{Catalonia}}{1--5.~\provinceCatalunya,
    6--7.~\provincePirineos, 8--10.~\provinceRosselo}%
  \defgroup{Sa}{\group{Atlantic}}{1--6.~\ctz{Espagne}, 7--8.~\seazoneLion,
    9.~\seazoneCanarias, 10.~\province{Islas Canarias}}%
  \defgroup{Am}{\group{New Spain}}{A random \COL/\TP (of any nationality) in
    the following area: 1--3.~\granderegionAzteca, 4--6.~\granderegionInca,
    7--8.~\granderegionChichimeca, 9.~\granderegionCuba,
    10.~\granderegionGuyana. If some area is empty, it is replaced by \Sa.}%
  \defgroup{As}{\group{Asia}}{A random \TP/\COL of \HIS not in continent
    \continentAmerica; if none, \Sa}%
  \defgroup{RO}{\group{America}}{A random \TP/\COL of \HIS in
    \continentAmerica; if none, \Sa}%
  \defgroup{AA}{\group{Africa}}{1.~\provinceAlgerie, 2--3.~\provinceOran,
    4.~\provinceAnnabah, 5--7.~\provinceTunis, 8.~\provinceIfriqiya,
    9.~\provinceAures, 10.~\provinceAtlas and \provinceKabylie (Revolts
    strength at \bonus{-10}, possibly no revolt if Strength<2)}%
  \defgroup{It}{\group{Italy}}{1.~\provinceMonferrato, 2.~\provinceSavoia,
    3.~\provinceParma, 4.~\provinceLucca, 5.~\provinceToscana,
    6.~\provinceSiena , 7.~\provinceNice, 8.~\provinceLiguria,
    9--10.~\provinceLombardia}%
  \defgroup{IN}{\group{Naples}}{1.~\provinceUmbria, 2.~\provinceLazio,
    3.~\provinceUmbria, 4.~\provinceAbruzzo, 5.~\provincePuglia,
    6.~\provinceBasilicata, 7.~\provinceCalabria, 8--10.~\provinceCampania}%
  \defgroup{II}{\group{Islands}}{1--2.~\provinceCorsica,
    3--4.~\provinceSaldigna, 5--6.~\provincePalermo, 7--8.~\provinceSicilia,
    9--10.~\provinceMalta}%
  \centerline{%
    \graytabular\begin{tabular}{|c|ccccc|}\hline%
      Result & I & II & III, IV & V, VI & VII\\\hline\ghline%
      <0& \HC & \HC & \HC & \HA & \HS \\\ghline%
      0 & \HA & \HA & \HA & \HS & \HN \\\ghline%
      1 & \HS & \HS & \HS & \HN & \IN \\\ghline%
      2 & \HA & \HE & \II & \IN & \Am \\\ghline%
      3 & \HC & \IN & \HN & \Am & \RO \\\ghline%
      4 & \HE & \HE & \HE & \HE & \HE \\\ghline%
      5 & \HB & \HB & \HB & \HB & \HB \\\ghline%
      6 & \Am & \Am & \Am & \II & \II \\\ghline%
      7 & \IN & \IN & \AA & \It & \It \\\ghline%
      8 & \HN & \HN & \HN & \HB & \HB \\\ghline%
      9 & \RO & \RO & \RO & \RO & \RO \\\ghline%
      10& \As & \As & \As & \As & \As \\\ghline%
      11& \II & \II & \IN & \IN & \IN \\\ghline%
      12& \It & \It & \It & \It & \It \\\ghline%
      13& \AA & \AA & \AA & \AA & \AA \\\hline\ghline%
    \end{tabular}}
  \showgrouplist{AA,HA,RO,As,Sa,HB,HC,HE,II,It,IN,Am,HN,HS}{}
  \caption{Revolt table for \HIS}\label{table:alt-revolt-spain}
\end{tablehere}

\clearpage



\subsection{Revolt table for \POR, \SUE and COL}

\aparag When a \REVOLT occurs in \SUE or \POR, roll on this table, in the
column of the correct country and current period.

\bparag If \DANmin or \SUEmin have to fight a revolt, they will raise the Sund
taxes (see \ref{chSpecific:Sund Levies}).

\aparag Decrease the pseudo-stability of \PORpor by \bonus{-1} if
\ref{pIII:Portuguese Disaster} happened (at this turn or a previous one).

\begin{tablehere}
  \defgroup{SC}{\provinceSvealand}{}%
  \defgroup{SN}{\group{Northern Sweden}}{1--3.~\provinceJamtland,
    4--6.~\provinceBergslagen, 7--10.~\provinceGastrikland}%
  \defgroup{SS}{\group{Southern Sweden}}{1--2.~\provinceVastergotland,
    3--4.~\provinceSmaland, 5--7.~\provinceGotland, 8--10.~\provinceSkane}%
  \defgroup{FI}{\group{Finland}}{1--3.~\provinceFinland,
    4--5.~\provinceNyland, 6.~\provinceTavastland, 7--8.~\provinceKarelen,
    9--10.~\provinceKexholm}%
  \defgroup{SB}{\group{Baltic Sweden}}{1.~\seazoneBaltique,
    2--3.~\provinceEstland, 4--5.~\provinceLivonija, 6.~\provinceKurland,
    7.~\provinceDanzig, 8--9.~\provinceHinterpommern,
    10.~\provinceVorpommern}%
  \defgroup{SH}{\group{Hansa}}{1--2.~\seazoneBaltique, 3--4.~\provinceBremen,
    5--6.~\provinceHolstein, 7--8.~\provinceLubeck,
    9--10.~\provinceMecklenburg}%
  \defgroup{DN}{\group{Denmark}}{1--2.~\provinceSjaelland,
    3.~\provinceJylland, 4.~\provinceSlesvig, 5--6.~\provinceOstlandet,
    7.~\provinceVestfold, 8.~\provinceTrondelag, 9--10.~\provinceSkane}%
  \defgroup{RS}{\group{ROTW (\SUE)}}{A random \COL\faceplus of \SUE; if none,
    \stz{Baltique}}% was \ctz RUS but Baltique is worst for SUE.

  \defgroup{RP}{\group{ROTW (\POR)}}{A random \COL\faceplus of \POR; if none,
    \stz{Guinee}}% was \ctz SPA. Could be \stz Canarias
  \defgroup{PR}{\group{Portugal}}{1--4.~\province{Tras-os-Montes},
    5--7.~\provinceAlgarve, 8--10.~\provinceBeira}%
  \defgroup{PC}{\group{Tagus}}{1--5.~\provinceTejo, 6--10.~\provinceAlentejo}%
  \defgroup{PO}{\group{Overseas}}{1--8.~\provinceTanger,
    9--10.~\provinceAcores}%
  \defgroup{PM}{\group{Morocco}}{1--3.~\provinceTanger, 4--6.~\provinceMagrib,
    7.~\provinceGranada, 8.~\ctz{Espagne}, 9.~\provinceSouss,
    10.~\provinceRif}%
  \defgroup{PH}{\group{Spain}}{1--4.~\provinceGaliza, 5--7.~\provinceCaceres,
    8.~\provinceExtremadura, 9--10.~\provinceHuelva}%

  \defgroup{Si}{\group{Singala}}{\REVOLT\facemoins in a random \COL/\TP in
    \granderegionSingala or \granderegionFormose}%
  \defgroup{WI}{\group{Slaves}}{Each power with a \COL in either
    \granderegionCuba, \granderegionHaiti or \granderegionAntilles rolls a
    die. On 7 or more, place a \REVOLT\facemoins (before 1700) or \Faceplus
    (after 1700) in a random \COL of this power in these areas.} %
  \defgroup{In}{\group{Indonesia}}{Place one \REVOLT \facemoins and one
    \REVOLT \faceplus in two randomly chosen \COL/\TP in areas
    \granderegionJava, \granderegionSumatra, \granderegionBorneo and
    \granderegionCelebes. Both \REVOLT can occur in the same place.}
  % STAB: pIV: -3 (Si) ; pV,pVI: 0 (WI) ; pVII: +3 (In).

  \centerline{%
    \graytabular\begin{tabular}{|c|c|c|cc|c|}\hline%
      Result & \POR & \SUE I--II & \SUE III--IV & \SUE V--VII &
      COL\\\hline\ghline%
      <0& \RP & \DN & \SC & \SC & \Si \\\ghline%
      0 & \PC & \SS & \SC & \SN & \Si \\\ghline%
      1 & \PC & \SC & \SN & \SN & \Si\\\ghline%
      2 & \PR & \DN & \DN & \DN & \Si\\\ghline%
      3 & \PR & \SH & \SS & \SH & \WI\\\ghline%
      4 & \PO & \SB & \SB & \SB & \WI\\\ghline%
      5 & \PO & \FI & \FI & \FI & \WI\\\ghline%
      6 & \PM & \DN & \SS & \RS & \WI\\\ghline%
      7 & \PH & \SS & \SS & \SS & \WI\\\ghline%
      8 & \PH & \SH & \SH & \FI & \WI\\\ghline%
      9 & \PH & \SS & \FI & \SB & \In\\\ghline%
      10& \PM & \FI & \FI & \FI & \In\\\ghline%
      11& \PO & \SB & \SB & \SB & \In\\\ghline%
      12& \PR & \RS & \RS & \RS & \In\\\ghline%
      13& \PM & \DN & \DN & \DN & \In\\\hline\ghline%
    \end{tabular}}
  \showgrouplist{PM,PO,PR,RP,PH,PC}{SB,DN,FI,SH,SN,RS,SS}\hrule%
  \showgrouplist{Si,WI,In}{}

  \caption{Revolt table for \POR, \SUE and
    COL}\label{table:alt-revolt-portugal-sweden}
\end{tablehere}

\clearpage



\subsection{Revolt tables for \HOL and \AUS}

\aparag When a \REVOLT occurs in \AUSaus or \HOL, roll on this table, in the
column of the correct country and current period.

\aparag Decrease the pseudo-stability of \HOLhol by \bonus{-1} if \HIS
perceived the taxes last turn.

\begin{tablehere}
  \defgroup{RO}{\group{ROTW}}{A random Dutch \COL; if none, \Sa}%
  \defgroup{NR}{\group{Rhine lands}}{1--4.~\provinceZeeland,
    5--10.~\provinceUtrecht}%
  \defgroup{NN}{\group{North lands}}{1--5.~\provinceFriesland,
    6--10.~\provinceOverijssel, }%
  \defgroup{NO}{\group{Outer lands}}{1--3.~\provinceLimburg,
    4--5.~\provinceBrabant, 6.~\provinceLiege, 7.~\provinceBremen,
    8.~\provinceOldenburg, 9.~\provinceGibraltar, 10.~\provinceBaleares}%
  \defgroup{NE}{\group{Netherlands}}{1--2.~\provinceHolland,
    3--4.~\provinceGelderland, 5.~\provinceZeeland, 6--7.~\provinceUtrecht,
    8--9.~\provinceOverijssel, 10.~\provinceFriesland}%
  \defgroup{NH}{\provinceHolland}{}%
  \defgroup{NG}{\provinceGelderland}{}%
  \defgroup{BF}{\group{Flanders}}{1--5.~\provinceVlaandern,
    6--10.~\provinceFlandre}%
  \defgroup{BB}{\group{Brussels}}{1--5.~\provinceBrabant,
    6--10.~\provinceLimburg}%
  \defgroup{BW}{\group{Wallonia}}{1--3.~\provinceLuxemburg,
    4--6.~\provinceHainaut, 7--10.~\provinceArtois}%
  \defgroup{GW}{\group{Westphalia}}{1--3.~\provinceBerg,
    4--5.~\provinceNassau, 6--8.~\provinceOldenburg,
    9--10.~\provinceOsnabruck}%
  \defgroup{Sa}{\ctz{Hollande}}{}%
  \defgroup{Am}{\group{America}}{A random \TP/\COL of \HOL in
    \continent{America}; if none, \Sa}%
  \defgroup{As}{\group{Asia}}{A random \TP/\COL of \HOL not in
    \continent{America}; if none, \Sa}%
  \defgroup{PW}{\group{Peasants War}}{After \shortref{pI:Reformation}, place
    3 % precise number TBD.
    random \REVOLT in provinces of the \HRE. The Emperor must crush these
    revolts that can extend in all the \HRE and cause loss of \STAB to the
    Emperor. Otherwise, \NN.}

  \defgroup{AA}{\group{Alps}}{1--3.~\provinceTrentino, 4--6.~\provinceTirol,
    7.~\provinceGraubunden, 8--9.~\provinceSchwaben, 10.~\provinceFriuli}%
  \defgroup{AI}{\group{Naples}}{1--2.~\provinceCampania, 3.~\provinceAbruzzo,
    4.~\provincePuglia, 5.~\provinceBasilicata, 6.~\provinceCalabria,
    7.~\provincePalermo, 8.~\provinceSicilia, 9.~\provinceMalta,
    10.~\provinceSaldigna}%
  \defgroup{AP}{\group{Poland}}{1--3.~\provinceBukovina, 4--5.~\provinceWolyn,
    6--7.~\provinceLublin, 8.~\provinceWielkopolska,
    9--10.~\provinceMalopolska}%
  \defgroup{AS}{\group{Slovenia}}{1--2.~\provinceIstria,
    3--5.~\provinceSlovenija, 6--8.~\provinceSteiermark,
    9--10.~\provinceKarnten}%
  \defgroup{AG}{\group{Germany}}{1--3.~\provinceOberPfalz,
    4--7.~\provinceSchwaben, 8--10.~\provinceAnhalt}%
  \defgroup{AD}{\group{Danube}}{1--5.~\provinceOsterreich,
    6--10.~\provinceSalzburg}%
  \defgroup{PM}{\group{Moravia}}{1--3.~\provinceMorava,
    4--6.~\provinceLausitz, 7--10.~\provinceSilesie}%
  \defgroup{PB}{\provinceBoheme}{}%
  \defgroup{HC}{\group{Croatia}}{1--2.~\provinceKapela,
    3--5.~\provinceCroatie, 6--7.~\provinceCarniola,
    8--10.~\provinceDalmacija}%
  \defgroup{HS}{\group{Slovakia}}{1--3.~\provinceSzlovakia,
    4--6.~\provinceBalaton, 7--10.~\provincePecs}%
  \defgroup{HF}{\group{Hungary}}{1--3.~\provinceKarpatok,
    4--5.~\provinceMagyarorszag, 6--8.~\provinceBanat, 9--10.~\provinceBosna}%
  \centerline{%
    \graytabular\begin{tabular}{|c|c|ccc|cc|}\hline%
      Result &\HOL I--II&\HOL III--IV&\HOL V--VI&\HOL VII%
      &\HAB I--VI&\HAB VII\\\hline\ghline%
      <0& \NH & \NH & \As & \NG & \AD & \AD \\\ghline%
      0 & \NR & \RO & \Am & \NN & \AA & \PM \\\ghline%
      1 & \NR & \NR & \NR & \NR & \AI & \AI \\\ghline%
      2 & \NN & \NN & \NN & \As & \PB & \PB \\\ghline%
      3 & \NG & \NG & \NG & \Am & \PM & \PM \\\ghline%
      4 & \PW & \NE & \NE & \NE & \PB & \AP \\\ghline%
      5 & \NO & \NO & \NO & \NO & \AS & \AS \\\ghline%
      6 & \BB & \BB & \BB & \BB & \PM & \PM \\\ghline%
      7 & \BF & \BF & \BF & \BF & \AP & \AP \\\ghline%
      8 & \BW & \BW & \BW & \BW & \AS & \AS \\\ghline%
      9 & \Sa & \NO & \NO & \NO & \AG & \HF \\\ghline%
      10& \GW & \GW & \GW & \Am & \HC & \HC \\\ghline%
      11& \NE & \NE & \Am & \RO & \HS & \HS \\\ghline%
      12& \Sa & \RO & \RO & \As & \PB & \PB \\\ghline%
      13& \BW & \BW & \NE & \GW & \PM & \HF \\\hline\ghline%
    \end{tabular}}
  \showgrouplist{Am,As,BB,BF,NE,NN,NO,PW,NR,RO,BW,GW}{AA,HC,AD,AG,HF,PM,AI,AP,HS,AS}

  \caption{Revolt table for \HOL and
    \AUSaus}\label{table:alt-revolt-holland-austria}
\end{tablehere}

\clearpage



\subsection{Revolt tables for \POL and \PRU}

\aparag When a \REVOLT occurs in \POL or \PRU, roll on this table, in the
column of the correct country and current period.

\begin{tablehere}
  \defgroup{PM}{\group{Capitals}}{1--3.~\provinceMalopolska,
    4--5.~\provinceLietuva (if no union of Lublin; \provinceMalopolska else),
    6--10.~\provinceMazowia. If union with \payssaxe, use rather
    1--3.~\provinceMalopolska, 4--5.~\provinceAnhalt, 6--7.~\provinceSachsen,
    8--10.~\provinceMazowia.}%
  \defgroup{PC}{\group{Central Poland}}{1--3.~\provinceWielkopolska,
    4--6.~\provinceWolyn, 7--10.~\provinceLublin}%
  \defgroup{PB}{\group{Baltic Poland}}{1.~\seazoneBaltique,
    2--3.~\provinceDanzig, 4--5.~\province{West Preussen},
    6--7.~\provinceKurland, 8.~\provinceLivonija, 9.~\provinceMemel,
    10.~\provincePreussen}%
  \defgroup{LS}{\group{Smolensk}}{1--3.~\provinceSmolenska,
    4--5.~\provincePolacak, 6--7.~\provinceSeveria,
    8--10.~\provinceBaltarusija}%
  \defgroup{LL}{\group{Lithuania}}{1--5.~\provinceLietuva,
    6--8.~\provinceZemaitija, 9--10.~\provincePrypec}%
  % Use either GP+GK or GT+GL
  \defgroup{GP}{\group{Prussia}}{1--4.~\provinceMemel,
    5--7.~\provincePreussen, 8--10.~\province{Ost Pommern}}%
  \defgroup{GL}{\group{Livonia}}{1--3.~\provinceKurland,
    4--6.~\provinceEstland, 7--8.~\provinceLivonija, 9--10.~\provinceMemel}%
  \defgroup{GK}{\group{Kurland}}{1--5.~\provinceKurland,
    6--10.~\provinceLivonija}
  \defgroup{GT}{\group{Teutonics}}{1--2.~\provincePreussen,
    3--6.~\province{West Pommern}, 7--10.~\province{Ost Pommern}}%
  \defgroup{UU}{\group{Ukraine}}{1.~\provinceDon, 2.~\provinceDonets,
    3--4.~\provinceZaporozhye, 5-6.~\provincePoltava, 7-8.~\provincePodolie,
    9--10.~\provinceUkraine}%
  \defgroup{RB}{\group{Russia}}{1--2.~\provinceKaluga,
    3--4.~\provinceNovgorod, 5--6.~\provinceNeva, 7--8.~\provincePskov,
    9--10.~\province{Dikoe Pole}}%
  \defgroup{SB}{\group{Carpathians}}{1--5.~\provinceKarpatok,
    6--10.~\provinceBukovina} \defgroup{SC}{\provinceBrandenburg}{}%
  \defgroup{SP}{\group{Prussian Core}}{1--5.~\provinceAltmark,
    6--10.~\provinceNeumark}%
  \defgroup{SM}{\group{Moravia}}{1--5.~\provinceLausitz,
    6--9.~\provinceSilesie, 10.~\provinceMorava}%
  \defgroup{Sb}{\provinceBoheme}{}%
  \defgroup{SG}{\group{Great Prussia}}{1--3.~\provinceBerg,
    4.~\provinceNassau, 5--7.~\province{West Preussen}, 8--9.~\provinceDanzig,
    10.~\provinceWielkopolska}%
  \defgroup{SH}{\group{Hansa}}{1--4.~\provinceMecklenburg,
    5--6.~\provinceLubeck, 7--8.~\provinceHolstein, 9--10.~\provinceBremen}%
  \defgroup{Ss}{\group{Saxony}}{1--7.~\provinceAnhalt,
    8--10.~\provinceSachsen}%
  % Jym : rules allows for COL/TP of POL/PRU, these are safe of revolts.
  \centerline{%
    \graytabular\begin{tabular}{|c|cc|c|}\hline%
      Result & \POL I--IV & \POL V--VII & \PRU \\\hline\ghline%
      <0& \PM & \PM & \SC\\\ghline%
      0 & \PM & \PM & \SC\\\ghline%
      1 & \LL & \LL & \SM\\\ghline%
      2 & \PC & \PC & \SP\\\ghline%
      3 & \LS & \PC & \SP\\\ghline%
      4 & \UU & \UU & \GT\\\ghline%
      5 & \UU & \UU & \GL\\\ghline%
      6 & \PB & \PB & \SG\\\ghline%
      7 & \GT & \GP & \SM\\\ghline%
      8 & \GL & \GK & \Sb\\\ghline%
      9 & \PC & \LS & \Ss\\\ghline%
      10& \RB & \PM & \SP\\\ghline%
      11& \UU & \UU & \SM\\\ghline%
      12& \SB & \RB & \GL\\\ghline%
      13& \RB & \RB & \SH\\\hline\ghline%
    \end{tabular}}
  \showgrouplist{PB,PM,SB,PC,GK,LL,GP,RB,LS,GT,UU}{SG,SH,GL,SM,SP,Ss,GT}

  \caption{Revolt table for \POL and
    \PRU}\label{table:alt-revolt-poland-prussia}
\end{tablehere}

\clearpage



\subsection{Revolt tables for \RUS}

\aparag When a \REVOLT occurs in \RUS, roll on this table, in the column of
the current period.

\bparag If \RUS owns provinces of the \regionPerse, check for
\xnameref{chSpecific:Persia:uprising}.

\begin{tablehere}
  \defgroup{RC}{\group{Capitals}}{1--3.~\provinceMoskva,
    4--10.~\ville{Saint-Petersbourg} (or \provinceMoskva if not built)}%
  \defgroup{RR}{\provinceRyazan}{}%
  \defgroup{RO}{\group{ROTW}}{A random \TP/\COL (any nationality) in
    \continentSiberia.}%
  \defgroup{RN}{\group{Northern Russia}}{1--3.~\provinceLadoga,
    4--6.~\provinceKexholm, 7.~\provinceOnega, 8--10.~\provinceYaroslavl}%
  \defgroup{RW}{\group{Western Russia}}{1--2.~\provinceKaluga,
    3--4.~\provinceRyazan, 5.~\provinceNeva, 6--10.~\provinceNovgorod}%
  \defgroup{WW}{\group{Baltic lands}}{1--5.~\provincePskov,
    6--7.~\provinceKarelen, 8--9.~\provinceEstland, 10.~\provinceLivonija}%
  \defgroup{RU}{\group{Uralic Russia}}{1--2.~\provinceVyatka,
    3--4.~\provinceBolgars, 5--6.~\provinceStep, 7--8.~\provinceBashkiria,
    9--10.~\provinceUral}%
  \defgroup{KK}{\group{Kazan}}{1--2.~\provinceKazan, 3--4.~\provinceTatarstan,
    5--6.~\provinceCheboksary, 7--8.~\provinceMordoviya,
    9--10.~\provinceSamara}%
  \defgroup{KC}{\group{Crimea}}{1--2.~\provinceKhadzhibei,
    3--4.~\provinceZaporozhye, 5--6.~\provinceCrimee, 7--8.~\provinceCaffa,
    9--10.~\provinceAzov}%
  \defgroup{KS}{\group{Caucasus}}{1--2.~\provinceAstragan,
    3--4.~\provinceTerek, 5.~\provinceKuban, 6--7.~\provinceGeorgie,
    8--9.~\provinceDagestan, 10.~\provinceShirvan}%
  \defgroup{KD}{\group{Don}}{1--3.~\province{Dikoe Pole}, 4--7.~\provinceDon,
    8--10.~\provinceDonets}%
  \defgroup{UN}{\group{Northern Ukraine}}{1--5.~\provinceSeveria,
    6--10.~\provincePoltava}%
  \defgroup{UU}{\group{Cossacks}}{1.~\province{Dikoe Pole}, 2.~\provinceDon,
    3.~\provinceDonets, 4--5.~\provinceSeveria, 6.~\provincePoltava,
    7.~\provincePodolie, 8--10.~\provinceUkraine}%
  \defgroup{WS}{\group{Smolensk}}{1--5.~\provinceSmolenska,
    6--8.~\provincePolacak, 9--10.~\provinceBaltarusija}%
  \defgroup{WL}{\group{Lithuania}}{1--2.~\provinceLietuva,
    3--5.~\provinceZemaitija, 6--10.~\provincePrypec}%
  \centerline{%
    \graytabular\begin{tabular}{|c|ccccc|}\hline%
      Result & I--II &III--IV &V & VI & VII\\\hline\ghline%
      <0& \RC & \RO & \RO & \RC & \RO \\\ghline%
      0 & \RC & \RC & \RC & \RO & \RO \\\ghline%
      1 & \RN & \RN & \RN & \RN & \RN \\\ghline%
      2 & \RW & \RW & \RW & \RW & \RW \\\ghline%
      3 & \RU & \RU & \RU & \RU & \RU \\\ghline%
      4 & \WW & \KK & \KK & \KC & \WW \\\ghline%
      5 & \KK & \KK & \KK & \KK & \KK \\\ghline%
      6 & \KS & \KS & \KS & \KC & \KC \\\ghline%
      7 & \UN & \UN & \UU & \UU & \WL \\\ghline%
      8 & \KC & \KC & \UU & \UU & \UU \\\ghline%
      9 & \WS & \WS & \WS & \WS & \WS \\\ghline%
      10& \RR & \WW & \WW & \WW & \WW \\\ghline%
      11& \KD & \KD & \KD & \KD & \RC \\\ghline%
      12& \RW & \RU & \RO & \KS & \RO \\\ghline%
      13& \RU & \RU & \KC & \WL & \KS \\\hline\ghline%
      % pI: Western+Uralic
      % RM RM RN RW WW RU KK KS UU KC WS RR KD RW WW
      % KK KS UU WS RN RU RW
      % Kazan Caucasus Cossacks Smolensk North Uralic Western
      % KC WW KD RM RO
      % Crimea Baltic Don Moskva ROTW
    \end{tabular}}
  \showgrouplist{WW,RC,KS,UU,KC,KD,KK,WL,RN,RO,WS,UN,RU,RW}{}

  \caption{Revolt table for \RUS}\label{table:alt-revolt-russia}
\end{tablehere}

\clearpage



\subsection{Revolt table for \VEN and \TUR}

\aparag When a \REVOLT occurs in \VEN or \TUR, roll on this table, in the
column of the correct country and current period.

\bparag If \TUR owns provinces of the \region{Perse}, check for
\xnameref{chSpecific:Persia:uprising} if a revolt occurs in \TUR (not in
\VEN).

\begin{tablehere}
  \defgroup{VC}{\provinceVeneto}{}%
  \defgroup{VI}{\group{Italy}}{1--2.~\provinceMantova, 3--5.~\provinceRomagna,
    6.~\provinceLombardia, 7.~\provinceModena, 8.~\provinceParma,
    9.~\provinceLucca, 10.~\provinceTrentino}%
  \defgroup{VA}{\group{Adriatic}}{1--2.~\provinceFriuli,
    3--4.~\provinceIstria, 5--6.~\provinceKapela, 7--8.~\provinceDalmacija,
    9--10.~\provinceMontenegro}%
  \defgroup{VZ}{\seazoneAdriatique}{}%
  \defgroup{TI}{\group{Outposts}}{A random Venetian \TP; if none,
    \provinceIzmir}%
  \defgroup{MI}{\group{Islands}}{1--2.~\provinceCyclades,
    3--4.~\provinceCorfu, 5--6.~\provinceKreta, 7--8.~\provinceRhodos,
    9--10.~\provinceChypre}%
  \defgroup{BA}{\group{Balkans}}{1--2.~\provinceMoreas, 3--4.~\provinceHellas,
    5--6.~\provinceMontenegro, 7.~\provinceBosna, 8.~\provinceDalmacija,
    9.~\provinceSerbia, 10.~\provinceAlabania}%
  \defgroup{RO}{\group{ROTW}}{At random between \provinceIzmir, and \COL (any
    side)/\TP\faceplus of \TUR}%
  \defgroup{TX}{\provinceTrakya}{}%
  \defgroup{TA}{\group{Anatolia}}{1.~\provinceAntalya, 2.~\provinceBursa,
    3.~\provinceKocaeli, 4.~\provinceSinop, 5.~\provinceTrabzon,
    6.~\provinceAngora, 7.~\provinceIzmir, 8.~\provinceKonya,
    9.~\provinceAnadolu, 10.~\provinceKilikya}%
  \defgroup{TR}{\group{Romelia}}{1.~\provinceCanakkale, 2.~\provinceMakedonya,
    3.~\provinceRumeli, 4.~\provinceBulgaristan, 5--6.~\provinceValahia,
    7--8.~\provinceBasarabia, 9--10.~\provinceMoldova}%
  \defgroup{TC}{\group{Outer Empire}}{1--2.~\ctz{Turquie},
    3--4.~\provinceMalta, 5--6.~\provinceSzlovakia, 7--8.~\provinceCarniola,
    9--10.~\provinceBalaton}%
  \defgroup{TH}{\group{Hungary}}{1--2.~\provinceMagyarorszag,
    3.~\provinceCroatie, 4.~\provinceKapela, 5.~\provincePecs,
    6.~\provinceBanat, 7.~\provinceErdely, 8.~\provinceKarpatok,
    9.~\provinceBukovina, 10.~\provinceMures}
  \defgroup{TS}{\group{Sultanates}}{1.~\provinceAlep, 2.~\provinceSyrie,
    3.~\provinceLubnan, 4.~\province{Terra Sancta}, 5.~\provinceNil,
    6.~\provinceDelta, 7.~\provinceNubie, 8.~\provinceEgypte,
    9.~\provinceCataractes, 10.~\provinceTobrouk and \provinceSinai (Revolts
    strength at \bonus{-10}, possibly no revolt if Strength<2)}%
  \defgroup{TK}{\group{Caucasus}}{\provinceGeorgie, \provinceKuban,
    \provincePodolie, \provinceHacibey, \provinceUkrainya, \provinceShirvan,
    \provinceDagestan, \provinceCaffa}%
  \defgroup{TM}{\group{Arabs}}{1--2.~\provinceCyrenaique,
    3.~\provinceJordanie, 4--5.~\provinceIrak, 6--7.~\provinceBassorah,
    8.~\provinceNefud, 9--10.~\provinceTripoli}%
  \defgroup{Tp}{\group{Persia}}{1--2.~\provinceAzarbayadjan,
    3--4.~\provinceArmenie, 5--6.~\provinceKordistan, 7.~\provinceTigre,
    8.~\provincePars, 9.~\provinceVan, 10.~\provinceKermanshah}%
  \centerline{%
    \graytabular\begin{tabular}{|c|c|cccc|}\hline%
      Result & \VEN & \TUR I--II & \TUR III--IV & \TUR V--VI & \TUR VII
      \\\hline\ghline%
      <0& \VC & \TX & \TX & \TX & \TA\\\ghline%
      0 & \VC & \TA & \TA & \TA & \TR\\\ghline%
      1 & \VI & \TR & \TR & \TR & \TK\\\ghline%
      2 & \VZ & \TK & \TK & \TK & \TM\\\ghline%
      3 & \VA & \TS & \TS & \TS & \TS\\\ghline%
      4 & \VZ & \TP & \TP & \TH & \TS\\\ghline%
      5 & \VA & \TA & \TS & \TS & \TS\\\ghline%
      6 & \MI & \BA & \TH & \BA & \TH\\\ghline%
      7 & \BA & \BA & \BA & \BA & \BA\\\ghline%
      8 & \MI & \Tp & \Tp & \Tp & \Tp\\\ghline%
      9 & \MI & \TH & \TH & \TH & \TH\\\ghline%
      10& \BA & \TC & \TC & \TC & \TC\\\ghline%
      11& \VI & \MI & \MI & \MI & \MI\\\ghline%
      12& \VZ & \RO & \RO & \RO & \RO\\\ghline%
      13& \TI & \TM & \TM & \TM & \TM\\\hline\ghline%
    \end{tabular}}
  \showgrouplist{VA,BA,MI,VI,TI}{TA,TM,BA,TK,TH,Tp,TC,TR,RO,TS}

  \caption{Revolt table for \VEN and
    \TUR}\label{table:alt-revolt-venice-turkey}\end{tablehere}

% Local Variables:
% fill-column: 78
% coding: utf-8-unix
% mode-require-final-newline: t
% mode: flyspell
% ispell-local-dictionary: "british"
% End:

% LocalWords: se monarchduration malus monarchvalue reroll montferrat modene
% LocalWords: lucca milan hongrie moldavie valachie mazovie transylvanie pV
% LocalWords: papaute toscane parme venise corse arabie irak georgie damas
% LocalWords: mamelouks suisse wurtemberg savoie treves lorraine mayence Ile
% LocalWords: hollande hanovre hesse palatinat oldenburg iroquois inca saxe
% LocalWords: azteque baviere alsace thuringe habsbourg boheme brandebourg
% LocalWords: brunswick pologne Vlithuanie Vpommeranie japon teutoniques de
% LocalWords: hanse danemark Vnorvege Vfinlande Vliflandie courlande ecosse
% LocalWords: portugal Virlande Vbelgique Khanates ryazan pskov cosaquesdon
% LocalWords: kazan crimee ukraine gujarat vijayanagar mysore hyderabad aden
% LocalWords: maroc algerie tunisie tripoli cyrenaique perse ormus oman Tras
% LocalWords: soudan mogol cccccc ccccc os Montes Ost Pommern Illes Balears
% LocalWords: persia Dikoe Sancta ccc politicalevents
