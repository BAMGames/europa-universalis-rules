\sectionJ{\anchorpaysmajeur{Venise}}{\blasonJ{venise}}
\subsection{Italia e San Marco}
\aparag[Enmity with \sectionpays{Genes}.] \VEN can make no diplomacy
upon \pays{Genes}.
\aparag[The Pope in Venice.]\rulelabel{chSpecific:Venice:Pope Venice} If
\ville{Roma} is conquered by \TUR, or if \pays{Papaute} is annexed by
\VEN (see underneath), the Pope is taken in \ville{Venezia}. \VEN gains
a bonus of {\bf +1} to diplomacy attempts on all catholic minor
countries.

\aparag[Policy of Italian Dominance.]
%[TBD keep the optionnal annoucement and penalty in PV ????]
% PB 07/2008: ok, on garde comme cela.
\VEN can declare such a policy at any phase of Diplomacy. It loses 30\PV
and may use the following rules over Italian powers: \pays{Genes},
\pays{Milan}, \pays{Modene}, \pays{Montferrat}, \pays{Naples},
\pays{Papaute}, \pays{Parme}, \pays{Pise}, \pays{Savoie},
\pays{Toscane}.
\bparag These minors can be in \ANNEXION diplomatically (even if it is not
allowed by the diplomatic chart), with a difficulty of 10 (or through wars).
\bparag Other countries can attempt diplomacy on such annexed \MIN. If
this lowers the control of \VEN, this causes a war of revolt instead of
the usual disannexion: the \MIN
declares war unto \VEN, receive reinforcements but no basic forces. The
\MIN controls all the cities in its provinces. The forces of the \MIN
can deploy anywhere in the \MIN and will attack any Venetian force in
the province before the first military round. The \MAJ that manages to
cause this revolt has a \CB against \VEN at this turn to help the \MIN ;
if it uses it, it obtains the \MIN in \EG, else the \MIN is now Neutral.
\bparag Note that a war of revolt can end by reestablishing the \MIN in
\ANNEXION of \VEN.

%\aparag[The Renaissance] [TBD ???]
%At the end of each turn of periods II and III, \VEN may spend 200\ducats
%from its Royal Treasury to score immediately 5\PV. (or more ???) \\
%\textbf{PB 07/2008: va �tre rendu caduque par nouveaux PVs.}


\subsection{A Commercial Empire}
\subsubsection{Relations with Minor Countries}
\aparag \VEN is especially interested in \ruleref{chSpecific:Mamluks} (and more
generally all of \ruleref{chSpecific:Italy}).
\aparag \VEN is also interested in \ruleref{chSpecific:Balkans}
and \ruleref{chSpecific:Crusades}.

\subsubsection{The Salt Monopoly}\label{chSpecific:Venice:Salt Exclusive}
\aparag Because of the large monopoly on the \RES{Salt} \VEN had in the
Mediterranean sea, a \RES{Salt} Manufacture of level 2 in
\province{Veneto} does exploit all \RES{Salt} resources owned by \VEN in
the Mediterranean sea.
\bparag This does not apply to \RES{Salt} sources outside the Mediterranean
sea, especially in \region{Autriche}.

\subsubsection{Naval means}
\aparag[Venetian Galeasses] When obtained \terme{Naval Technology} Galleass,
\VEN can build and have up to 2 \ND of \terme{galeasses} (noted \VGD). One such
\VGD can be built each turn, at a price of 2\NGD (and it uses one full
\ND of the construction limit).
\bparag For most of the rules, a \VGD is a \NGD (movements, stacking and
maintenance).
\bparag In battle against \NGD (not against \NWD or \NTD), having one
\VGD in the force cause full losses obtained in the fire step (and not
half the losses as is the rule for galleys); having the 2 \VGD gives an
additional bonus of {\bf +1} on the die-roll in the Fire step.
\bparag If  \terme{Naval Technology} is \TBAT or higher, because now every \NGD has some
form of Galleass, the only effect is that having at least one \VGD in battle against \NGD
gives the additional bonus of {\bf +1} on the die-roll in the Fire step.
\bparag One \VGD has to be lost (destroyed if possible) if the force
suffers a Major defeat in battle. Else, the repartition of the losses is
left to the player.  A captured \VGD is transformed in a \NGD of the
enemy player.
\aparag[Dutch Fleets] In EU8, \VEN manages the placement of Dutch fleets (see
\ruleref{chSpecific:Holland:Dutch Trading Fleets}).
%\aparag[The \Presidio at Corfu]\label{chSpecific:Venice:Corfu} The
%\province{Corfu} province, as long as it belongs to \pays{Venise}, can
%be used as a \Presidio of level 2 on any fleet that tries to enter or
%exit \seazone{Adriatique}.
\aparag See~\ruleref{chMilitary:Strait Fortifications} for the use of the
\StraitFort at \province{Corfu}.

\subsection{\sectionpays{Venise} as a minor country}
\aparag See \ruleref{chSpecific:Campaign:Transfer Venice} for the conditions of
the transfer proper.
\aparag[Military means] \pays{Venise} has a modifier of {\bf +3} in
reinforcements in period III, and {\bf +1} in periods IV and V.
\bparag \pays{Venise} has one \VGD in its basic forces (that may be in a
\FLEET). It can build another one (or re-build) by using the
reinforcements of 1\ND (or 2\NGD) to build one \VGD.
\aparag \pays{Venise} has trade fleets (and may have the
\CCs{Mediterranee}) and a base \FTI and \DTI of 3, or 4 in periods IV to
VII.  It keeps a commercial fleet action each turn during periods III to
V.

\subsection{\sectionpays{Venise} in play}
\subsubsection{The Doge}
\aparag The Monarch of \VEN is the \anchormonarque{Doge}.
\bparag Use {\bf -2} to determine the length of the reign; \VEN is never
affected by Dynastic Crisis.
\bparag He can be used as an admiral but not as a general.
\bparag The \monarque{Doge} rolls for his monarchs characteristics with
a bonus of {\bf +1} and the minimal value of a given characteristic is
4.
\aparag[\anchormonarque{Barbarigo}] is the Doge in 1492, with
values 8/5/6, that dies at the beginning of turn 3.

\subsubsection{Available counters}
\aparag[Military] 2\ARMY, 2\FLEET, 1\corsaire, 6\LDND, 2\NTD, 4\LDENDE,
2 fortresses 1/2, 5 fortresses 2/3, 3 fortresses 3/4, 1 fortress 4/5, 2 forts.
\aparag[Economical] 1\COL, 4\TP, 6\MNU, 5\TradeFLEET, 2 \ROTW treaty
counters.


% LocalWords: galeasses Autriche Venezia Savoie Italia Modene Pise Mediterranee
% LocalWords: Papaute Montferrat Parme Toscane Venise Egypte Alexandrie Veneto
% LocalWords:  Barbarigo
