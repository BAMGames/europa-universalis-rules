\sectionJ{\anchorpaysmajeur{Angleterre}}{\blason{angleterre}}\label{chSpecific:England}
\subsection{English intervention in wars}
\aparag \ENG has the possiblity of signing offensive limited alliances in any
war that is neither a Civil War nor a Religious War.
\aparag \ENG may use its \corsaire counters in addition to its forces
involved in limited intervention.
\aparag\label{chSpecific:England:Minors at war} \ENG may use the forces of any
minor country in \VASSAL position on its diplomatic track in a limited
intervention. Those forces are dealt with as if in limited intervention
of a minor country at the side of the alliance supported by \ENG,
excepted that they can not be fully involved in the war by the enemy
unless the enemy alliance first declares war to \ENG.
\aparag During any War, \ENG may announce that he will send forces of a
minor country in \VASSAL position on its diplomatic track in the
\ROTW. This declaration costs {\bf 1} \STAB for each vassal that will be
used, and is valid for the rest of the period. The troops of the vassal
are then used in limited or full intervention with the change that they
can go in the \ROTW (and be supplied there by English fleets or colonial
settlements).
\subsubsection{Military leaders}
% Now the general rule.
%\aparag\label{chSpecific:England:Monck Blake}
%\leader{Monck} and \leader{Blake} are both generals and admirals.
%However, contrary to the usual rule, their role for the current turn
%is announced at the deployment phase and fixed for the whole turn.
\aparag\label{rule:RoyalMarines} Royal Marines appear as a general only
during \eventref{pV:WoSS}.
\aparag \leader{Marlborough} does not have the malus of {\bf -1} to its
survival tests in battles (due normally to its '6' values).

\aparag[Fleet in being] [BLP] \anonyme\LeaderA numbered 6, 7, 8 and 9
are only available starting from period \period{V} (1660, turn 35).

\subsection{Overseas and Colonial Policy}
\subsubsection{The Sea Hounds}\label{chSpecific:England:Sea Hounds}
\aparag Leaders \listleadershort{leadersangseahound}
% \leader{Drake} and \leader{Hawkins}, as well as both Explorers
% \leader{Cavendish} and \leader{Frobisher} are
are the \terme{Sea Hounds}. They have a yellow symbol instead of a
black one.
%They have a special mark ``h'' on their counter.
\bparag[Drake] The first time \leader{Drake} is reputed dead due to
battle loss, attrition or exploration, he is in fact unavailable for
the rest of the turn but returns back in play at the beginning of the
following turn.

\aparag[As Privateers]  A \corsaire led by \terme{Sea Hounds} may
attack Convoys or Commercial fleets and/or \COL/\TP of other players,
or even Loot European provinces, without \ENG having to declare war
on that player.
\ENG can however attack only one such player per
turn, and must announce the target country during the military rounds,
at the end of the second round at the latest.
\bparag[Privateer] If commanding a \corsaire unit that was committed to
the attack of commercial fleets not at war, a \terme{Sea Hound} must
stay the rest of the turn with this \corsaire.
\bparag There is no loss of \STAB for England in doing so as there is no
state of war between itself and the attacked country.
\bparag Once the attack is declared the attacked country and its allies
may react with Naval forces (or land forces in \TP/\COL or provinces)
against the units led by the \terme{Sea Hounds}.

\aparag[Exploration with Sea Hounds] \terme{Sea Hounds} with Admiral
symbols may also be used as Explorers for any discoveries to be made by
the English player, including while acting as Privateers.

\subsubsection{English Missionaries and Missions}
\aparag See \ruleref{chSpecific:Missions} for the general rules.

% \aparag \ANG obtains one Missionary each even-numbered turn, beginning
% with period III. It may use 1 Missionary if \terme{Catholic} or
% \terme{Protestant}, and 2 Missionaries if \terme{Anglican}.

\aparag[Religion and availability.]
\bparag \leader{Brewster} is always available. \leader{Penn} and
\leader{Blair} are only available if \ANG is \PROTANG.
\bparag Namely, if either \leader{Penn} or \leader{Blair} is alive
and \ANG is \textbf{not} \PROTANG, this \LeaderMis is immediately
removed from the game and may not come back later, including if \ANG
changes religion.
\bparag Missions are not affected by religion.

\aparag Bonuses given by English Missionaries to \COL/\TP attempts may
be used even on \Faceplus settlements.

\aparag If \ANG is \terme{Protestant}, English Missions give a malus
of {\bf -2} to \COL/\TP placement attempts of any other power in the
same \Area.

\aparag \ANG loses 5\PV each time one of his Missions is destroyed.

\subsubsection{English Sepoys}
\longCipayes{Sepoys}

\aparag[\sectionleader{Clive}] The conquistador \leader{Clive} can use
the table of conquistadors in \continent{India}. Forces stacked with
\leader{Clive} never cause Activation of Natives in \continent{India}.
\bparag If \leader{Clive} is not in play (lost for any reason), the
minimum \LeaderC in period VII can then use the table of conquistadors
in \continent{India} (or a named one if there is one; the leader is
determined at the beginning of the turn and can not change).

\subsubsection{Few acres of snow}
\aparag \ANG may annex all establishments (\COL and \TP) of its enemies
in an Area in \continentAmerica, North of \granderegionChichimeca
(excluded) at peace.
\bparag This count as 1 peace condition, plus 1 per establishment not
controlled by \ANG in the Area at the time of the peace.

\aparag[\anchorconstruction{Gibraltar}]~\\

\begin{todo}
  En fait, on doit pouvoir lisser Gibraltar par :
  \begin{itemize}
  \item  ANG a un arsenal-présidio "Mediterranée".
  \item  Il ne peut être construit que sur un port présidiable qui
    touche la Méditerranée et que à partir de pVI.

    => éventuellement rajouter la présidiabilité de
    Corse/Sardaigne/Palerme, vu ce que ces ports servent ça devrait pas
    changer le jeu mais ça semble pas idiot de laisser ANG s'y accrocher
    si il veut.
  \item Par exception, ANG peut le construire dans une province qu'il
    possède (présidio sur lui-même).
  \item  Par exception, ANG peut l'augmenter en phase admin même sans
    contrôle de la ville.
  \item Si ANG doit céder la province, il garde le présidio ("je te
    laisse Majorque, mais je m'accroche à Minorque")
  \item WoSS ne donne plus Gibraltar, seulement les Baléares. L'anglais
    n'est plus achetable trivialement par HIS mais les Baléares sont
    quand même bien, et ANG doit aller choper Gibraltar si il le veut.
  \item Et pour que Gibraltar/Tanger soient quand même mieux que le
    reste, un truc à peaufiner un peu sur le thème de "une F+ en 74s
    guns dans un arsenal Méditerrannée à Tanger ou Gibraltar a +2 à
    l'interception" (à comparer entre les tables d'interception et de
    détroit fortifié pour trouver la bonne
    condition/modificateur). Seulement en 74s guns car il faut la
    technologie adéquate et seulement avec une F+ car il faut les
    moyens nécessaires.
  \end{itemize}

  Du coup, il y a l'option historique, mais aussi d'autres
  possibilités. Tanger est risqué car le Maroc peut casser le présidio,
  mais ANG peut aussi jouer à Tanger + attaque du Maroc pour lui prendre
  une province et en cas de contre-attaque par event diplo, il perd la
  province au lieu de l'arsenal. Et il y a des possibilités plus loin en
  Méditerranée.

  Et pour symétriser les choses, il faut sans doute donner le même pion
  avec les mêmes règles à HOL qui n'a normalement plus la possibilité de
  se projeter aussi loin en pVI mais ça laisse plus de what if
  raisonnables si HOL réussit bien sa pV-VI. Voire aussi à SUE
  "colonial", parce qu'au point où on en est, c'est pas un pion de plus
  qui va changer quoi que ce soit. (FRA, TUR, HIS n'ont pas besoin de ça
  car sont déjà en Méditerranée et PRU, RUS, AUS n'ont pas de bateaux).

  Et PVs de fin de jeu
  \begin{itemize}
  \item 30 PVs pour ANG/HOL/SUE si leur arsenal est placé.
  \item +20 PVs si c'est à Gibraltar ou Tanger.
  \item 25 PVs pour HIS/TUR si il n'y a pas d'arsenal chez eux ou
    leurs mineurs.
  \item 10 PVs pour FRA/AUS (si elle a hérité de l'Italie du Sud) si
    il n'y a pas d'arsenal chez eux ou leurs mineurs.
  \item -25 PVs pour HIS/TUR si il y a un arsenal sur leur territoire
    national.
  \end{itemize}
\end{todo}

\subsection{\sectionpaysmajeur{Angleterre} in play}
\subsubsection{English Kings, Queens and Ministers}
\aparag[\anchormonarque{Henry VII}] reigns in 1492, with values 7/5/6,
scheduled to die at the beginning of turn 5. His heir is \monarque{Henry
  VIII}.
\aparag[\anchormonarque{Henry VIII}] is the second English king, with
values 6/7/7. His reign is 8 turns long, and he does not test survival
during the first 5 turns. At the end of its reign, apply automatically
\eventrefname{pII:Act Supremacy} as one of the events of the turn.
\aparag[\anchormonarque{Elisabeth I}] arrives through conditions
described in event \eventref{pII:Act Supremacy}. She has values 8/8/6
and her reign will last 8 turns. She does not test survival during the 5
first turns. She cannot be used as a general. During her reign, \ENG
adds a free maintenance of a \FLEET\faceplus and a \corsaire\faceplus.
\aparag[The \anchormonarque{Parliament}.] Because of
\eventref{pIV:English Civil War}, \monarque{Parliament} may rule in
\ENG. It has values 5/8/8 and makes no test of survival. It gives a
bonus of {\bf +2} to the rolls for all administrative actions (except
exceptional taxes, \ruleref{chIncomes:Exceptional Taxes}).
\aparag[\anchormonarque{Cromwell}] may replace the \monarque{Parliament}
following \eventref{pIV:English Civil War}. He has values 8/8/9, is
still a general \leaderdata{Cromwell}.  His Reign is to last the number
of turns remaining for the general (of the initial 5 turns).  He must
test for survival normally. As long as his reign continues, \ENG gains a
free maintenance of one \ARMY\faceplus.
\aparag[\strongmonarque{Willem III} (William III).] Event
\eventref{pV:Glorious Revolution} may put the ruler of \HOL on the
throne of \ENG, if \HOL is ruled by the House of Orange.
\bparag A personal union exists between \HOL and \ENG: that is a
mandatory defensive alliance, and a usual offensive alliance. They make
an immediate mandatory white peace and can not be at war against each
other as long as this lasts. The union ends when the Monarch dies.
\bparag The Monarch from the House of Orange is controlled by \ANG until
the end of \eventref{pV:Glorious Revolution} and by \HOL after. \HOL
makes the survival tests.
\aparag[\anchorministre{Pitt}] may be named minister through
\eventref{pVII:William Pitt}. He has values 9/8/8 and remains a random
number of turns; its values can be used for the next monarch's values
determination if a succession takes place while he is still alive.

\subsection{Available counters}
\aparag[Military] 4\ARMY, 6\FLEET, 3\corsaire, 15\LDND, 5\LD, 4\NTD,
10\LDENDE, 2 fortresses 1/2, 4 fortresses 2/3, 4 fortresses 3/4, 2
fortresses 4/5, 11 forts, 2 Arsenals 2/3, 2 Arsenals 3/4, Arsenal \terme{Gibraltar} 2/3,
2 Missions (1 only if not \terme{Anglican}), 5
\terme{Sepoys} \LD (and 3 \terme{Sepoys} \LDE).
\aparag[Economical] 14\COL, 10\TP, 14\MNU, 18\TradeFLEET, 4\ROTW treaty
counters.
\aparag[Royalists] 3\ARMY, 1\FLEET, 5\LDND, 5\LD, 2\NTD, 5\LDENDE. These
counters are used for English civil wars; the \pays{royalistes} uses
English fortresses counters.

% LocalWords: Sepoys malus pV WoSS pIV Angleterre Monck pIII royalistes
