\sectionJ{\anchorpaysmajeur{Hollande}}{\blason{hollande}}

% Check + More work TODO !!!!

\subsection{Holland as a minor country}

% \subsubsection{Dutch Unification}

\aparag[Before the existence of Holland]
\pays{Hollande} does not exist in 1492 and its provinces are in
\pays{Bourgogne} and \pays{provincesne}.
\bparag See \ruleref{chSpecific:Belgium} and \ruleref{chSpecific:Burgundy}.
\aparag[Spanish Holland] Before \eventref{pIII:Dutch Revolt}, \SPA and
\VEN (in EU9) share the management of the Spanish Holland territories (see
\ruleref{chSpecific:Spain:Spanish Holland}). If \pays{Vhollande} comes in
existence, \VEN continues implanting trade fleets for \pays{Vhollande}
until it becomes independent (by \eventref{pI:Reformation2}).
\aparag As long as \eventref{pIII:Dutch Revolt} does not happen,
\pays{Hollande} is not a \MAJ (and the player continues with playing
\paysmajeur{Venise}). The switch intervenes at the turn of the first
revolt.
\bparag See \ruleref{chSpecific:Campaign:Transfer Venice} for the conditions of
the transfer from \paysmajeur{Venise}.
\bparag See \ruleref{chSpecific:Campaign:Transfer Holland} for the conditions
of the transfer to \paysmajeur{Autriche}.

\aparag[Dutch Trading Fleets]
\label{chSpecific:Holland:Dutch Trading Fleets}
All administrative, commercial and overseas
actions (see afterwards) are resolved by \VEN (in EU8) (even if the \MIN is
allied to another power), or \SPA if \pays{Hollande} is a Special Vassal.
If there is no \VEN, the usual rules are applied (the patron, then the
first preferred country resolves them).
\bparag \pays{Hollande} has commercial fleets and a base \FTI of 3, or 4
in periods IV to VII. It has a \DTI of 4 before \eventref{pIII:Dutch
  Revolt}, and 5 afterwards.

\aparag Until event \eventref{pIII:Dutch Revolt}, 1d10 is rolled at the
end of each administrative phase and its gives a number of levels of
commercial fleets to be placed on the map: 1-2 none; 3-5 one; 6-8 two;
9-10 three levels. Placement of the levels obeys the usual restriction
(discoveries, maximum of 6) and is mandatory.
% TODO: rephrase priority into something understandable.
\bparag In priority, the levels must be placed in an existing
\TradeFLEET where Dutch has not Monopoly or where \HOL has no
\TradeFLEET (max 1 level in new sea per turn). If possible, maximum of
one level per sea zone, and maximum of one level in Mediterranean
Sea. If 3 levels are obtained, one at least should be placed in
Mediterranean Sea.
\bparag Placement is made by \DAN.

\aparag[Commercial and Colonial Expansion]
\bparag  All administrative, commercial and overseas actions (see
afterwards) are resolved by \VEN (even if the \MIN is allied to another
power) or \SPA if \pays{Hollande} is a Special Vassal. If there is no
\VEN, the usual rules are applied (the patron, then the first preferred
country resolves them).
\bparag The actions above are resolved with a medium investment.
\bparag \pays{Hollande} has commercial fleets and a base \FTI of 3, or 4
in periods IV to VII. It has a \DTI of 5.
\bparag Until the end of period V, \pays{Hollande} has one \TP, 1
\COLaction, and two \CONC, two \TradeFLEET actions to be used each turn.
\bparag In periods V and VI, \pays{Hollande} has one \TP or \COLaction
and one \CONC, one \TradeFLEET action to be used each turn.
\bparag \pays{Hollande} has commercial fleets and a base \FTI of 3, or 4
in periods IV to VII. It has a \DTI of 5.

\aparag[Military forces and Discoveries in \ROTW]
Until the end of period V, if at peace or doing limited intervention
only, \pays{Hollande} raises one \FLEET\faceplus and one \ARMY\faceplus
to be used overseas each turn, in discoveries and battles against
Natives; it also has one simple campaign at each round. The named
\LeaderE and \LeaderC of \HOL are used, with a minimum of one \LeaderE
and one \LeaderC to be taken in unnamed counters.

\subsection{Revolt of the United provinces}\label{chSpecific:Holland:First
  Revolt}
\aparag The expanded rules for the revolt are in \eventref{pIII:Dutch
  Revolt}, in which the United Provinces (that became the Netherlands)
revolt against \SPA.
\aparag \HOL begins with all its national provinces
(\theminorprovincesshort{hollande}), its \TradeFLEET as laid out at the
time of the transfer, the military control of \province{Brabant} and
\province{Limburg}.
\bparag All these provinces do revolt, even if they did not belong to
\SPA. If a \MAJ owned one of these, he loses them, but gains a \CB that
may only be used simultaneously against \SPA and \HOL. \SPA loses 5\VP
per province that it did not own by \eventref{pIII:Dutch Revolt}.
\bparag \pays{provincesne} and \pays{Vhollande} are destroyed by the
event if they did exist.
\aparag The initial state of the country is described in the event.
\bparag The event usually gives the Atlantic \terme{Trade Centre} to
\HOL.
\bparag Do not forget that \SPA marks 5\VP per turn during which it does
not acknowledge Dutch sovereignty, and that it still owns
\province{Brabant} and \province{Limburg}.

\aparag[Dutch-Portuguese war] \HOL may harass \pays{portugal} as soon as
it is annexed by \SPA due to \eventref{pIII:Portuguese Annexation}. The
end of this state of war is described in the revolt event.

\subsection{The Dutch Government}
\subsubsection{Choice of government}
\aparag \paysmajeur{Hollande} may have two forms of government: a
\terme{Stadhouder} (aristocratic) government or a \terme{Parliament}
government. The government type can be changed in the following
circumstances:
\bparag After the death of any sovereign, before rolling the next one; instead
of rolling, an available named personalty can be chosen if his type
of government is adopted.
\bparag When a named personalty is available as Monarch for
the \terme{Stadhouder} or the  \terme{Parliament} government
(see the list in \ruleref{chSpecific:Holland:Leaders}), the Government
can be changed to this stance (only once for each personalty).
%\bparag When replacing \monarque{Willem I} by
%\monarque{Oldenbarnevelt} (ruling a \terme{Parliament});
\bparag Immediately after being victim of a declaration of war, to
change for a \terme{Stadhouder};
\bparag At the beginning of the turn following the acknowledgement of
sovereignty by \SPA.
\bparag The first government after \eventref{pIII:Dutch Revolt} is a
\terme{Stadhouder} (ruled by \monarque{Willem I}).
\bparag When changing governement, a new Monarch is rolled
 for (if not given by the event, or by some named personalty).
 If a named personalty is dismissed, the Leader does not disappear
 (and can stille serve as General for instance). However, survival modifications gained as monarch are forgotten.
\begin{designnote}[Dutch zombies]
This precision is here due to misinterpretation of the rules in a game that saw Dutch armies overseen by a zombie Willem of Oranje-Nassau.
\end{designnote}

\subsubsection{The aristocratic government (Stadhouder)}
\aparag Newly rolled-for \terme{Stadhouder} has at least 5 in \MIL.
\aparag \HOL has a \corsaire\faceplus maintained in its \terme{basic
  forces} (or built anew if none are left) during periods III to V (no cost in
  \ducats nor in \ND).
\aparag At the turn of beginning of a full involvement in war (including
if it was the event that provoked the switch to aristocracy), \HOL
receives a free \ARMY\faceplus and a fortress of the highest possible
level to be placed anywhere in Europe.
\aparag The \terme{basic forces} (\FLEET and \GD) are changed (with
an \ARMY\faceplus is in the Maintenance).
\aparag At most one \ARMY counter may be in the \ROTW.
\aparag The minimum \LeaderG is 2 during periods III to VI.

\subsubsection{The Parliament}
\aparag Newly rolled leader of the \terme{Parliament} has at least 5 in \ADM.
\aparag \HOL has one more \TFI and one more \CONC per turn available.
If the VOC has been created, as per \eventref{pIII:VOC}, the \TFI action
is mandatorily in one of the \STZ of the Indian \terme{Trade Centre}.
\aparag The Atlantic \terme{Trade Centre} is worth 150\ducats to \HOL
as long as \eventref{pIV:Act Navigation} is not in effect.
\aparag The \terme{basic forces} (for \ARMY) are changed (no \ARMY
but increased in  \FLEET and \GD).
\aparag At most two \ARMY counters may be in Europe.


\subsection{Military and Overseas rules}
\subsubsection{Naval Construction}
\aparag If a monarch has at least 7 in \ADM and \MIL and naval
technology is not \terme{Ships of the line} of higher, \HOL may forgo
two \TFI actions to gain a free \FLEET\facemoins during the
administrative phase. It does not count in any turn limits.

\subsubsection{Dutch Flood}\label{chSpecific:Holland:Flooding}
\aparag The Dutch player can decide, during the movement of enemy troops
in any of its national province, to flood immediately the province.

\aparag[Effects] Place a Looting/Flooded marker, side \faceplus up, in
each flooded provinces (adjusted only during the turn following the turn
of flooding).
\bparag Enemy units must immediately cease their movement,
retreat to the province of departure, and suffer attrition from an
enemy territory, with the effect of the Looting \faceplus and a {\bf +2}
malus.
\bparag The Dutch player may move in his flooded provinces, but each
count for 6\MP instead of its regular \MP cost.
\bparag No movement from a non-Dutch player or minor country is allowed in a
flooded province.
\bparag The prohibition of movement ceases when the looting marker is
removed.

\aparag[Cost in VP and Stability] Each flooded province costs
immediately 5 VP and -1 in Stability to the Dutch player.  At the
Stability adjustement phase (end of each turn), if Flooded markers
remains, \HOL will lose at least 1 in \STAB per Flooded marker, if
greater than losses due to wars (even if at peace).


\subsubsection{Dutch Indi\"{e}rs}
\longCipayes{Indi\"{e}rs}

\subsubsection{Overseas}
\aparag[VOC conquistadors] Some dutch conquistadors can use the table of
Conquistadors in \granderegion{Java}, \granderegion{Sumatra},
\granderegion{Malacca}, \granderegion{Borneo}, \granderegion{Iles de la
  Sonde}, \granderegion{Iles aux epices} and \granderegion{Celebes}.
They are: \leaderwithdata{Coen}, \leaderwithdata{van Diemen}, and
\leaderwithdata{Maetsuycker}.

\aparag[Dutch Secret on Discoveries] \HOL is not allowed to sell, give
or trade any of his discoveries (except as peace condition), \COL or \TP
with any other player before period VII.

\aparag[Foreign trade index] Once the VOC is
created~(\xnameref{pIII:VOC}), \HOL has a specific \FTI for \ROTW
operations, that is different from its \FTI (see
\ruleref{chAdministration:Special FTI}).

\aparag[Redeploying colonies] Once the VOC is
created~(\xnameref{pIII:VOC}), \HOL may choose to voluntarily destroy
one of its \COL.
\bparag This is decided at the beginning of administrative phase, before
planning of actions.
\bparag Each turn it choose to do so, \HOL may ignore restrictions
of~\ref{chAdministration:Pioneering}.

\begin{playtip}
  Doomed colonies (because of natives attack) may be relocated that way
  and the manpower concentrated into safer and more productive areas.
\end{playtip}

\subsubsection{Few acres of snow}
\aparag \HOL may annex all establishments (\COL and \TP) of its enemies
in an Area in \continentAmerica, North of \granderegionChichimeca
(excluded) at peace.
\bparag This count as 1 peace condition, plus 1 per establishment not
controlled by \HOL in the Area at the time of the peace.


\subsection{\sectionpaysmajeur{Hollande} in play}
\subsubsection{Dutch Leaders}\label{chSpecific:Holland:Leaders}

% From notes PB.  S- William I Stadtholder 9/9/9 [7/9/9] A 322-1 Event
% III-1: t15-(23) mort pr�matur�e (assassin�) en fin t19/d�but t20 -
% Maurice of Nassau (7/7/9 ?) A 455-1 t20-27 (stadtholder au t27) P-
% Johan van Oldenbarnevel 8/7/7 [9/7/7] Land's Advocate of Holland
% t20-26 (ou t21 ?)  S- Frederick Henry, Prince of Orange Stadtholder
% 8/7/7 A 443-1 t27-32 (stadtholder au t28) P- Johan de Witt, Grand
% Pensionary of Holland 9/7/9 Event pV-10: t33-37 S- Willem III of
% Orange, 7/9/7 A 333 Event pV t37-43
\aparag[\anchormonarque{Willem I}] is the first \terme{Stadhouder} of
\HOL. He is scheduled to last 7 turns, and counting from
\eventref{pIII:Dutch Revolt}, he does not roll for death for 3 turns. He
is a monarch 7/9/9. He is also a general \leaderwithdata{Willem I}.

\aparag[\anchormonarque{Oldenbarnevelt}] (Land's Advocate of Holland) is
available as Monarch to a \terme{Parliament} government for turns 19-26
(included). \monarque{Oldenbarnevelt} is a monarch 9/7/7.

\aparag[\anchormonarque{Maurits}] (Prince of Orange) is available as
general \leaderwithdata{M Nassau} %(CHECK: A 455-1 t20-27)
during turns 20 to 27. He is available as Monarch (Stadhouder of Holland
mainly) for a \terme{Stadhouder} government, with values 7/7/9.

\aparag[\anchormonarque{Frederik Hendrik}] (Prince of Orange) is
available as general \leaderwithdata{Frederik Hendrik}
% (CHECK: A 443-1 t27-32)
during turns 27 to 32.  He is availble as Monarch (Stadhouder of Holland
mainly) for \terme{Stadhouder} government, with values 8/7/7.

\aparag[\anchorministre{de Witt}] (Grand Pensionary of Holland) arrives
via event \eventref{pV:de Witt}. He is available as Monarch to a
\terme{Parliament} government, with values 9/7/9.  However, he may
serves as an Excellent Minister in a \terme{Stadhouder} governement.
(Historical dates: turns 33 to 37).
\bparag \HOL adds to its \terme{basic forces} \FLEET\facemoins and
\ARMY\faceplus during every turn if is engaged in a war (Overseas,
limited or full-fledged) during his reign (or Ministry)
\bparag During the last two turns of \shortministre{de Witt}'s term in
office (be it Monarch or Minister), add {\bf +1} to the monarch survival
test.  If the monarch dies during these two turns, \shortministre{de
  Witt} is also removed and this ends the event before the new
monarch is chosen.

\aparag[\anchormonarque{Willem III}] (of Orange-Nassau and England) is a
general \leaderwithdata{Willem III}, availble from Turn 37 to 43. He can
be Monarch of a \terme{Stadhouder} governement, with values 7/9/7.
\bparag \HOL receives a free \ARMY\faceplus added to its \terme{basic
  forces} when he reigns.

\aparag[Personal Union between Holland and England.]
Event \shortref{pV:Glorious Revolution} puts the Orange dynasty on the
throne of England. If there is \terme{Stadhouder} of \HOL, reputed to be
from Orange Dynasty, the two powers share the same Monarch and will be
associated by a Dynastic Alliance.
\bparag At the instant of the event \eventref{pV:Glorious Revolution},
if the governement is \terme{Stadhouder} (e.g., with
\shortmonarque{Willem III}), \HOL is associated to \ANG unless \HOL
declines the offer. In this later case, it immediately reverts to a
\terme{Parliament} government (with a new Monarch) and loses {\bf 2} in
\STAB.
\bparag If the governement is \terme{Parliament}, \HOL may elect to
change to a \terme{Stadhouder} governement (with a new Monarch) and in
this case is associated to \ANG.

\aparag[\anchorministre{Heinsius}] may be named minister through
\eventref{pVI:Heinsius}. He has values 9/8/7 and remains a random
number of turns; its values can be used for the next monarch's values
determination if a succession takes place while he is still alive.

\subsubsection{Available counters}
\aparag[Military] 3\ARMY, 5\FLEET, 2\corsaire, 15\LDND, \LD, 4\NTD,
8\LDENDE, 2 fortresses 1/2, 4 fortresses 2/3, 5 fortresses 3/4, 2
fortresses 4/5, 5 forts, 2 Arsenals 2/3, 2 Arsenals 3/4
and 4 \terme{Indi\"{e}rs} \LD (and 2
\terme{Indi\"{e}rs} \LDE).
\aparag[Economical] 10\COL, 12\TP, 8\MNU, 20\TradeFLEET, 4 \ROTW treaty
counters.


% LocalWords:  provincesne Vhollande hollande pIII Brabant Limburg portugal de
% LocalWords:  Indi Mughals Coen Iles Sonde epices Celebes Stadhouder Venise pI
% LocalWords:  Oldenbarnevelt Autriche pV pVI WoSS pIV
