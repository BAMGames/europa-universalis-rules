\sectionJ{\anchorpaysmajeur{Suede}}{\blasonJ{suede}}
\label{chSpecific:Sweden}

\subsection{\sectionpaysmajeur{Suede} as a Minor Country}
\aparag \pays{Suede} has commercial fleets and a base \FTI of 3, or 4 in
periods IV to VII. It has neither commercial fleet action nor \COL/\TP
colonial.
\aparag[Union of Kalmar]\label{chSpecific:Sweden:Union Kalmar} \pays{Suede} is
linked to \pays{Danemark} in 1492 by the \terme{Union of Kalmar}. No
independent diplomacy is possible on \pays{Suede} (no counter
available).
\bparag If a war is declared upon \pays{Danemark}, \pays{Suede} is
called as an ally of \pays{Danemark} ; if \pays{Danemark} declares a
war, \pays{Suede} makes a limited intervention on the side of
\pays{Danemark}, and the converse is true also (\pays{Danemark} helps
\pays{Suede}).
\bparag Peace is made normally, the two countries being allies.
\bparag \POR always play \pays{Suede} when activated in a war, excepted
if \POR is with the enemy side.
\aparag[End of the Union of Kalmar]
\bparag This alliance is broken when event \eventref{pII:End Kalmar}
occurs.
\bparag \pays{Suede} is Neutral when the Union breaks, and is now
subject to normal diplomacy.
\bparag \pays{Suede} as a \MIN power receives all its reinforcements as
\terme{Veteran}.

\aparag[Transfers.]
\bparag For the transfer from \paysmajeur{Portugal}, see
\ruleref{chSpecific:Campaign:Transfer Portugal}
\bparag For the possible transfer to \paysmajeur{Autriche}, see
\ruleref{chSpecific:Campaign:Transfer Sweden}

\subsection{The Swedish Crown}

\subsubsection{Relations with \sectionpays{Danemark}}
\label{chSpecific:Sweden:Denmark}
\aparag[Claims of \sectionpays{Danemark}] At the time of the transfer,
\pays{Danemark} claims the Swedish Crown. As long as it has not
abandoned its claims, \SUE can attempt no diplomacy on \pays{Danemark}
and has an additional malus of {\bf -2} to make peace with it.
\aparag \pays{Danemark} will abandon its claim to the Swedish Crown by
signing a unfavourable peace with \SUE. This will count as one condition
of the peace won by \SUE.
\bparag When \pays{Danemark} pretends no more to the Swedish Crown, \SUE
can do diplomacy on \pays{Danemark}. 
%but can not achieve a status better than \MA.
\bparag \SUE cannot annex any longer any national province of \pays{Danemark}.
However, \pays{Danemark} is considered to have diplomatic status of \ANNEXION 
achievable by \SUE (value of 10) by normal rules.
\begin{designnote}
This leaves the possibility for a union between the two crowns, be it from a
hazardous dynastics marriage or, more probably, a military imposed solution 
-- however a fragile one has any other player is entitled to break it through diplomacy.
\end{designnote}
\subsubsection{General policy of \sectionpaysmajeur{Suede}}
\aparag At any time during the game, the player of \SUE may announce his
general orientation of the policy of \SUE: either a policy of Domination
of the Baltic sea (\terme{Dominum Maris Baltici}), or a policy of
\terme{Overseas Expansion}. They are exclusive. A declaration is
optional and \SUE can choose to never make one (Note that this would be
almost pointless in periods VI or VII).

\aparag[Domination of the Baltic Sea]\label{chSpecific:Sweden:DMB}
\bparag \SUE gains a third \ARMY counter in periods III, IV and V. It
loses the minimum \LeaderE of period III.
\bparag It may annex any province bordering the \seazone{Baltique}, even
if there is a capital city. This may destroy a minor country.
\bparag By exception to the preceding rule, \ville{Copenhague} may not
be annexed if \pays{Danemark} has at least another province left that is
not in \region{Norvege}. When \ville{Copenhague} is annexed, any
remaining provinces of \pays{Danemark} are associated in a newly created
\pays{Norvege}, which is placed as a \VASSAL of \SUE. 
\bparag It has a maximum of 2 \TP counters and 2 \COL counters in period
IV and afterwards.
\bparag It has a reduced number of \TP/\COL attempts (see tables).

\aparag[Overseas Expansion]
\bparag \SUE has a maximum of 4 \TP counters and 4 \COL counters in
periods IV and afterwards.
\bparag \SUE has an increased number of \TP/\COL attempts (see tables).
\bparag \SUE gains a third \ARMY counter to be used only in \ROTW.
It also gains the use of a 3/4 Arsenal counter.
\bparag \SUE gains a minimum Explorator in pIII, and a minimum Gouvernor
in pV to pVII.
\bparag \SUE ignores restrictions of~\ref{chExpenses:Pioneering}.


\subsubsection{Insufficient demography}
\aparag During periods VI and VII, \SUE has 2 \ARMY counters unless it
owns at least 5 provinces that are neither in \region{Suede}, in
\region{Finlande} nor in \pays{Danemark}, in which case it can use 3
\ARMY counters. 
\bparag \eventref{pVI:Great Northern War} may modify this and give \SUE
permanently 3 \ARMY counters.
\bparag If Overseas expansion was chosen, the third \ARMY can always be
used in \ROTW.
\aparag During periods VI and VII, \SUE can not create new \COL or \TP by
administrative attempts. Its actions can only raise the level of
existing \COL/\TP, and it can gain new \TP/\COL only by conquest or
Dowry.
\aparag \SUE has no \CTZ of its own. It may use up to 6 Commercial
Fleet counters (exception: 10 if strictly Protestant).

\subsection{Swedish Conscription and Military}
\label{chSpecific:Sweden:Conscription}
\aparag All reinforcements purchased under the limit of construction by
\SUE are \terme{Veteran}.
\aparag The \terme{recruitment area} of \SUE comprises
\province{Livonija}, \province{Kurland} and \region{Finlande} as well as
its national provinces.
\aparag \SUE has an added \ARMY\facemoins in its \terme{basic forces}
when it is at war. This is increased to an added \ARMY\faceplus if its
current Monarch has a \MIL of 7 or more.
\aparag \SUE can proceed to exceptional levies
(see~\ruleref{chExpenses:Exceptional Levies}) with no loss of \STAB
or with a loss in \STAB after a normal (not major) defeat in a land battle.
\aparag \SUE has each turn a free major campaign. It is upgraded to 
2 free major campaigns (or one free multiple campaign, player's choice) 
if its current Monarch has a \MIL of 7 or more.
\aparag[Transport Convoy] \SUE has a transport convoy in its
\terme{basic forces}. It can contains up to 4 \NTD (or 2 \NTD if
\facemoins), is freely maintained but \SUE has to pay to recover any
previously lost \NTD. This convoy cannot leave \region{Baltique}. It is
not a \FLEET for attrition, stacking, and so on, but a Convoy.
\aparag[Movements to and from Finland.] \SUE units can move from
\province{Jamtland} or \province{Gastrikland} to \province{Finland} or
\province{Tavastland} at the cost of 12 \MP (and conversely).  It can
use this road for retreat or redeployment.

\subsubsection{Religious Attitude}\label{chSpecific:Sweden:Reformation}
\begin{histoire}
  The kingdom of Sweden was somewhat affected by religious troubles,
  because, even though the population quickly converted to the
  Reformation principles, the Swedish nobility did not follow this
  path. Queen Christina, daughter of Gustav Adolf, was a catholic that
  created a sustained cultural and religious activity in her kingdom
  while Oxenstierna was leading the foreign policy. She finally had to
  step down from her throne due to her religion. She hid her faith until
  her abdication.

  The catholic battle against protestantism is an important part of the
  failed union between Poland and Sweden.
\end{histoire}
\aparag \SUE has to choose its religious stand at the beginning of
period III. It can change afterwards only because of a forced conversion
to Catholicism by a Counter-Reformation \MAJ, or because of some events.
\aparag[Strictly Protestant]
\bparag \POL, if \terme{Catholic}, has a permanent \CB against \SUE in
periods III and IV.
\bparag At the beginning of each war against a Catholic country (such as \POL
but also minor countries) in period III and IV, roll for one Revolt in \SUE.
\bparag \SUE gains a \TFI action each turn, in periods III and IV, and
may use up to 10 \TradeFLEET (instead of 6).
\bparag Some events are affected (TODO: put list here).
\aparag[Tolerant] \SUE is \terme{Protestant} (historical choice).
\aparag[Catholic] \SUE has a \CB against all non-Catholic countries in
periods III and IV.
\bparag Some events are affected (TODO: put list here).


\subsubsection{Union between Poland and Sweden}
\label{chSpecific:Sweden:Polish Union}
\aparag As a consequence of \eventref{pIII:Union Poland Sweden}, \SUE
and \POL can share the same ruler. As long as this is the case:
\bparag \SUE uses the values of the Monarch of \POL. \SUE is considered
\terme{Catholic} during the Union (in every aspect).
\bparag \SUE has a mandatory offensive alliance with \POL in which it is
complied to answer any call.
\bparag \SUE cannot declare war without a \CB or the agreement of \POL.
It cannot declare war against \POL (even with a \CB).
\bparag \POL cannot declare a war against \SUE, except if it has a valid
\CB against it.
\aparag The alliance is contested when the Monarch of \POL dies or if
\POL refuses to answer a call for defensive war (not offensive war), or
if \POL declares a war against \SUE.
\bparag A new monarch is then rolled for \SUE.
\bparag \POL, having still \terme{dynastic claims} over Sweden, can
renew the war to impose its ruler. \POL renews the Union if it wins a
peace of any level against \SUE. As long as the war continues, the Union
exists for matters related to \VP, if not in its consequences.
\aparag[Dynastic Claims.] Even if \eventref{pIII:Union Poland Sweden}
does not result in the Union, \POL may keep \terme{dynastic claims} over
\SUE, at the conditions of the event.
\bparag \POL can renounce these \terme{dynastic claims} by an
announcement at any diplomatic phase, or as a condition for peace in a
losing war against \SUE.
\bparag Each time there is a new monarch in \SUE, \POL has a \CB against
\SUE at this turn to claim its inheritance (see the event).
\bparag In case of \terme{Dynastic Crisis} in \SUE, \POL is a valid
pretender as long as it has \terme{dynastic claims} over Sweden.

\subsection{\sectionpaysmajeur{Suede} in play}
\subsubsection{Monarchs of Sweden}
\aparag[Military skills.]
Add {\bf +1} to the die rolls to determine the values of Fire and Shock
of the Swedish Monarch as general.
\aparag[\anchormonarque{Gustav I}.] If \eventref{pII:End Kalmar} occurs
at the first turn of period III, \SUE has the Monarch \monarque{Gustav
  I}, of values 8/6/7. The length of his reign is rolled for as usual.
\aparag[\anchormonarque{Eric XIV}.] Else, if \eventref{pII:End Kalmar}
happened before, \SUE has the Monarch \monarque{Eric XIV} whose values
are 5/5/7 and should last until the beginning of turn 19.  He has to
roll for survival beginning with turn 17; he has a malus of {\bf +2} to
his survival test.  When he dies, roll for his successor on the 7+
columns (except if there is a \terme{Dynastic Crisis} -- use then usual
rules).
\aparag[\anchormonarque{Charles IX}] may be put on the throne of \SUE by
\eventref{pIII:Union Poland Sweden}. He has values 8/6/6 and the length
of his reign is rolled for as usual. He can not be used as a general
(see~\ref{chSpecific:Sweden:Polish Union}).
\aparag[\anchormonarque{Gustav Adolf}] enters in play during event
\eventref{pIV:TYW}. He has values 9/9/9 and is also a general
\leaderwithdata{Gustav-Adolf}. He will stay for 7 turns (but a death in
battle is possible). As soon as possible, \SUE benefits from a Military
Revolution (see \ruleref{chExpenses:Military Revolutions}) when he enters.
\aparag[\anchormonarque{Charles XII}]\label{chSpecific:Sweden:Charles XII}
\bparag The first Monarch of \SUE after the death of the heir of
\monarque{Gustave Adolphe} who has at least 8 or 9 in \MIL is considered
to be \monarque{Charles XII}.
\bparag Alternatively, after the death of the heir of \monarque{Gustave
  Adolphe}, 1d10 is rolled at the end of each administrative phase if
\SUE is at war against any \MAJ. On a roll of 1--3, \monarque{Charles
  XII} is the heir of the current Monarch. He will last for a length
determined randomly as for a Monarch, plus 2 turns (ignore results baby
or child and re-roll).
\bparag The \MIL value of \monarque{Charles XII} is changed to 9. He is
a general \leaderwithdata{Carl XII}. Other values as a Monarch are
rolled for normally when he becomes King.
\bparag \monarque{Charles XII} makes survival tests only if he is King.
\bparag The first time \monarque{Charles XII} should be killed or
captured in battle, he escapes but comes back only at the very end of
the next turn. During his absence, his values as a ruler are diminished
by 2 (minimum of 3).

\aparag[\anchorministre{Oxenstierna}] may be named minister through
\eventref{pIII:Oxenstierna} or \eventref{pIV:Oxenstierna}. He has values
6/8/8 and remains a random number of turns (three turns more than
usual); its values can be used for the next monarch's values
determination if a succession takes place while he is still alive.

\subsubsection{Available counters}
\aparag[Military] 3\ARMY, 2\FLEET, 1 Transport \FLEET, 1\corsaire,
10\LDND, 5\LD, 2\NTD, 6\LDENDE, 2 fortresses 1/2, 4 fortresses 2/3, 4
fortresses 3/4, 1 fortress 4/5, 4 forts, 2 Arsenals 2/3,
a special Pugatchev \ARMY.
\aparag[Economical] 5\COL, 5\TP, 8\MNU, 10\TradeFLEET (6 normally
usable), 2\ROTW treaty counters.

% LocalWords:  Autriche Kalmar Danemark malus Dominum Baltici Baltique Norvege
% LocalWords:  Copenhague Finlande pII pVI Livonija Kurland Jamtland Tavastland
% LocalWords:  Gastrikland Oxenstierna pIII Vasa
