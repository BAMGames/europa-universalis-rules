% -*- mode: LaTeX; -*-

%\section{The Campaign}

\section{The Great Campaign}
\begin{designnote}
  The Great Campaign is the way this game is meant to be played. It retrace
  history of the world from the European perspective between the discovery of
  the New World and the French Revolution.

  The Great Campaign is currently designed for 9 players. One of them only
  starts to play in period \period{III}, taking the role of the Dutch as they
  revolt against the Spanish empire. Six of the players play each a single
  country while the three other play several countries.

  We start here by describing the setup at the beginning of the game, followed
  by indications on what happens whenever a player has to change and play
  another country. Lastly, we give the indications for who plays which country
  both for the standard nine players game and for variants with fewer
  players.
\end{designnote}


%\begin{designnote}
%  This part is not written yet. Main ideas are summed up there, for 9, 7 or 6
%  players.
%\end{designnote}


\subsection{Placement in 1492 (turn 1)}


\subsubsection{Miscellaneous}
\aparag During round 1 of turn 1, No country may explore a \STZ adjacent to
\continentAmerica, unless the stack is commanded by \leaderColon.
\aparag No monarch survival roll is made for turn 1.
\aparag[Marco Polo] All coastal provinces in the areas between
\granderegionBalouchistan and \granderegionNankin (included) are known by all
players.

\subsubsection{Global markers}
\aparag Inflation is on the 5\% box.
\aparag The prices of all exotic resources are at their respective minimums,
as shown on the track.
\aparag Land Technologies: \terme{Latin} 6, \terme{Orthodox} 4, \terme{Islam}
4, \terme{Asia} 1. Naval Technologies: \terme{Latin} 7, \terme{Orthodox} 4,
\terme{Islam} 5, \terme{Asia} 1.
\bparag Technological goal are on the boxes shown on the counters.
\aparag \granderegionKarnatika and \granderegionBengale only produce 1 of each
resource: put the ``1 resource'' markers there.
\aparag There is no Russian \CTZ. Put a counter there.
\aparag Sund taxes are not raised. Put the ``Free trade'' counter.

\subsubsection{\paysmajeurAngleterre}
\aparag The monarch is \monarque{Henry VII} (6/7/7), followed by
\monarque{Henry VIII}. The \STAB is +1, and \ANG is \terme{Catholic}.

\aparag[Owned provinces.]
\bparag[National territory]: \provinceCumberland, \provinceDurham,
\provinceYorkshire, \provinceLancashire, \provinceCymru, \provinceMidlands,
\provinceLincolnshire, \province{East Anglia}, \provinceGloucester,
\provinceCornwall, \provinceWessex, \provinceKent.
\bparag[Other provinces]: \provinceConnacht, \provinceMumhan,
\provinceLaighean, \provinceBrega, \provinceUladh and \provinceCalais.
\bparag[Known sea zones]: \seazoneAcores.

\aparag[Diplomatic track]:
\begin{modlist}
\item[\VASSAL] \paysEcosse
\end{modlist}

\aparag[Economical situation]:
\bparag \MNU of \RES{Metal} on side\facemoins in \provinceMidlands, \MNU of
\RES{Cloth} on side\facemoins in \province{East Anglia}, \MNU of \RES{Fish} on
side\facemoins in \provinceWessex, \FTI is 2 and \DTI is 1.
\bparag Initial treasury is 50\ducats.
\bparag \TradeFLEET level 2 in \ctz{Angleterre}, level 2 in \stz{Nord}, level
2 in \stz{Baltique} and level 2 in \stz{Canarias}.

\aparag \terme{Land technology} is 5 and \terme{Naval technology} is
9. 1\ARMY\facemoins, 1\LD, 1\FLEET\facemoins (2\NWD/1\NTD), 1\ND, 1\NTD are
raised.


\subsubsection{\paysmajeurFrance}
\aparag The monarch is \monarque{Charles VIII} (5/7/9). The \STAB is +3, and
\FRA is \terme{Catholic}.

\aparag[Owned provinces.]
\bparag[National territory]: \provinceFinistere, \provinceArmor,
\provinceMorbihan, \provinceVendee, \provincePoitou, \provinceLimousin,
\provinceTouraine, \provinceMaine, \provinceNormandie, \provinceCaux,
\province{Ile-de-France}, \provinceOrleanais, \provinceBerry,
\provinceAuvergne, \provinceCevennes, \provinceQuercy, \provinceGuyenne,
\provinceBearn, \provinceLanguedoc, \provinceProvence, \provinceDauphine,
\provinceLyonnais, \provinceBourgogne, \provinceTroyes, \provinceChampagne and
\provincePicardie.
\bparag[Known sea zones]: \seazoneAcores, \seazoneCanarias.

\aparag[Diplomatic track]:
\begin{modlist}
\item[\MA] \payssavoie
\end{modlist}

\aparag[Economical situation]:
\bparag \MNU of \RES{Metal} on side\facemoins in \provinceChampagne, \MNU of
\RES{Wine} on side\facemoins in \provinceGuyenne, \FTI is 1 and \DTI is 2.
\bparag Initial treasury is 100\ducats.
\bparag \TradeFLEET level 2 in \ctz{France}, level 1 in \stz{Lion}, level 1 in
\stz{Ionienne} and level 1 in \stz{Canarias}.

\aparag \terme{Land technology} is 9 and \terme{Naval technology} is
7. 1\ARMY\faceplus, 1\ARMY\facemoins, 1\ND and 2\NGD are raised.


\subsubsection{\paysmajeurEspagne}
\aparag The monarch is \monarque{Isabel and Ferdinand} (6/7/6). The \STAB is
+2, and \SPA is \terme{Catholic}.

\aparag[Owned provinces.]
\bparag[National territory]: \provinceGaliza, \provinceSalamanca,
\provinceExtremadura, \provinceHuelva, \provinceGibraltar, \provinceGranada,
\provinceMurcia, \province{La Mancha}, \provinceToledo, \province{Castilla La
  Nueva}, \province{Castilla La Vieja}, \provinceAsturias, \provinceVizcaya,
\provinceNavarra, \provincePirineos, \provinceCatalunya, \provinceLeon,
\provinceCaceres, \provinceAndalucia, \provinceCordoba, \provinceValencia,
\provinceAragon.
\bparag[Other provinces]: \province{Illes Balears}, \provinceRosselo,
\provinceSaldigna, \provinceSicilia, \provincePalermo, \provinceMalta,
\provinceCanarias.
\bparag[Known sea zones]: \seazoneAcores, \seazoneCanarias.

\aparag[Diplomatic track]:
\begin{modlist}
\item[\SUB] \paysgenes
\item[\RM] \payspapaute and \paysnaples
\end{modlist}

\aparag[Economical situation]:
\bparag \MNU of \RES{Metal} on side\facemoins in \provinceToledo, \FTI is 2
and \DTI is 1. A \MNU of \RES{Cloth} on side\faceplus is placed in
\provinceVlaanderen, and will be available only after \eventref{pI:Burgundy
  Inheritance}.
\bparag Initial treasury is 150\ducats.
\bparag \TradeFLEET level 2 in \ctz{Espagne}, level 2 in \stz{Lion} and level
1 in \stz{Canarias}.

\aparag \terme{Land technology} is 7 and \terme{Naval technology} is
9. 1\ARMY\faceplus, 1\FLEET\facemoins (4\NGD/1\NTD), 3\ND are raised. [BLP] 1
\Presidio of level 1 in \province{Algerie} (Spanish presence on the Peñòn
started during the Middle ages, the island was only fortified in 1510).


\subsubsection{\paysmajeurPologne}
\aparag The monarch is \monarque{John and Alexander} (4/5/4), followed by
\monarque{Zygmunt I}. The \STAB is +2, and \POL is \terme{Catholic}.

\aparag[Owned provinces.]
\bparag[Polish National territory]: \provinceWielkopolska, \province{West
  Preussen}, \provinceDanzig, \provinceLublin, \provinceMalopolska,
\provinceWolyn, \provincePrypec,
\bparag[Lithuanian National territory]: \provinceLietuva, \provinceSmolenska,
\provinceBaltarusija, \provinceSeveria, \provinceZemaitija, \provincePolacak.
\bparag[Other provinces] (These provinces belong to \paysUkraine):
\provinceUkrainya, \provincePodolie, \provincePoltava.

\aparag[Diplomatic track]:
\begin{modlist}
\item[\VASSAL] \paysMazovie and \paysUkraine (special)
\item[\RM] \paysHongrie and \paysBoheme
\end{modlist}

\aparag[Economical situation]:
\bparag \MNU of \RES{Cereals} on side\facemoins in \provinceLietuva, \MNU of
\RES{Metal} on side\facemoins in \provinceWielkopolska, \FTI is 1 and \DTI is
1.
\bparag Initial treasury is 100\ducats.
\bparag No \TradeFLEET.

\aparag \terme{Land technology} is 6 and \terme{Naval technology} is
6. 1\ARMY\faceplus for \paysmajeurPologne, 1\ARMY\faceplus for \paysLithuanie,
1\ARMY\facemoins for \paysUkraine are raised.


\subsubsection{\paysmajeurPortugal}
\aparag The monarch is \monarque{Joao II} (8/6/7), followed by
\monarque{Manuel I} (8/6/8) at the beginning of T2. The \STAB is +3, and \POR
is \terme{Catholic}.

\aparag[Owned provinces.]
\bparag[National territory]: \province{Tras-os-Montes}, \provinceBeira,
\provinceTejo, \provinceAlentejo, \provinceAlgarve.
\bparag[Other provinces]: \provinceTanger, \provinceAcores.
% La Praya increased to level 4 to avoid troubles with pioneering.
\bparag [Already Placed \COL]: \terme{La Praya} (level 4), in
\granderegion{Cabo Verde}, exploits 1 \RES{Fish}. No more malus for
\COLaction.
\bparag[Already placed \TP]: \constructionElmina (level 3) in the western part
of \granderegionCotedor, exploits 3 \RES{Slaves} and two \RES{Gold Mines} (see
\ruleref{chSpecific:Portugal:African Gold}).
\bparag[Already placed mission]: One mission in a coastal province of
\continentAfrica, West of \granderegionCap (excluded) (representing contacts
made by Henry the Navigator with Kongo).
\bparag[Known sea zones]: \seazoneAcores, \seazoneCanarias, \seazoneArguin,
\seazoneGambie, \seazoneGuinee, \seazoneAngola, \seazone{Bonne-Esperance}.
\bparag[Known provinces]: The three provinces of \granderegionCameroun, the
two provinces of \granderegionCotedor, \granderegion{Cabo Verde}, the province
where the mission is placed.

\aparag[Diplomatic track]: Nothing

\aparag[Economical situation]:
\bparag \MNU of \RES{Instruments} on side\facemoins in \provinceTejo, \MNU of
\RES{Wine} on side\facemoins in \province{Tras-os-Montes}, \FTI is 2 (5 for
\ROTW) and \DTI is 3.
\bparag Initial treasury is 400\ducats.
\bparag \TradeFLEET level 3 in \stz{Canarias} and level 1 in \stz{Guinee}.

\aparag \terme{Land technology} is 7 and \terme{Naval technology} is
10. 1\ARMY\facemoins, 1\FLEET\facemoins (2\NWD/1\NTD), 1\LD are raised, all in
Europe.


\subsubsection{\paysmajeurRussie}
\aparag The monarch is \monarque{Ivan III} (6/7/8). The \STAB is +3, and \RUS
is \terme{Orthodox}.

\aparag[Owned provinces.]
\bparag[National territory]: \provinceMoskva, \provinceKaluga,
\provinceNovgorod, \provinceNeva, \provinceOnega, \provinceLadoga,
\provinceYaroslavl, \provinceVyatka.

\aparag[Diplomatic track]:
\begin{modlist}
\item[\MA] \paysKazan
\item[\RM] \paysCrimee
\end{modlist}

\aparag[Economical situation]:
\bparag \MNU of \RES{Cereals} on side\facemoins in \provinceNovgorod, \FTI and
\DTI is 1.
\bparag Initial treasury is 100\ducats.
\bparag No \TradeFLEET.

\aparag \terme{Land technology} and \terme{Naval technology} is
4. 3\ARMY\faceplus are raised.


\subsubsection{\paysmajeurTurquie}
\aparag The monarch is \monarque{Bayezid II} (7/5/6), followed by either
\monarque{Selim I} or \monarqueSuleyman. The \STAB is +2, and \TUR is
\terme{Sunni}.

\aparag[Owned provinces.]
\bparag[National territory]: \provinceTrakya, \provinceCanakkale,
\provinceIzmir, \provinceBursa, \provinceMakedonya, \provinceBulgaristan,
\provinceKosovo, \provinceRumeli, \provinceKocaeli, \provinceTrabzon,
\provinceAngora, \provinceSinop, \provinceAntalya, \provinceKonya,
\provinceAnadolu, \provinceKilikya.
\bparag[Other provinces]: \provinceAlabania, \provinceHellas, \provinceMoreas,
\provinceCaffa.
\bparag[Already place mission] A mission is in \province{Nedj W} (Mecca).
\bparag[Known sea zones]: \seazoneRouge, \seazonePersique.
\bparag[Known provinces]; \province{Nedj W}.

\aparag[Diplomatic track]:
\begin{modlist}
\item[\RM] \paysmoldavie
\item[\VASSAL] \paysvalachie
\end{modlist}

\aparag[Economical situation]:
\bparag \MNU of \RES{Art} on side\facemoins in \provinceTrakya, \FTI is 2 and
\DTI is 3.
\bparag Initial treasury is 400\ducats.
\bparag \TradeFLEET level 2 in \ctz{Turquie}, level 2 in \stz{Noire}.

\aparag \terme{Land technology} is 8 and \terme{Naval technology} is
8. 1\ARMY\facemoins of \Janissaire, 3\ARMY\faceplus of \Timar,
1\FLEET\facemoins (4\NGD/1\NTD) and 5 \Pashas are raised.


\subsubsection{\paysmajeurVenise}
\aparag The monarch is \monarqueBarbarigo (8/5/6). The \STAB is +3, and \VEN
is \terme{Catholic}.

\aparag[Owned provinces.]
\bparag[National territory]: \provinceVeneto, \provinceMantova,
\provinceFriuli, \provinceIstria.
\bparag[Other provinces]: \provinceDalmacija, \provinceCorfu, \provinceKreta,
\provinceCyclades, \provinceChypre.

\aparag[Diplomatic track]:
\begin{modlist}
\item[\SUB] \paysmamelouks
\end{modlist}

\aparag[Economical situation]:
\bparag \MNU of \RES{Salt} on side\faceplus in \provinceVeneto, \MNU of
\RES{Art} on side\facemoins in \provinceVeneto, \MNU of \RES{Wine} on
side\facemoins in \provinceChypre, \FTI is 3 and \DTI is 3.
\bparag Initial treasury is 200\ducats.
\bparag \TradeFLEET level 4 in \ctz{Venise}, level 2 in \ctz{Turquie}, level 3
in \stz{Ionienne} and level 3 in \stz{Noire}. \VEN owns the
\CCs{Mediterranee}.

\aparag \terme{Land technology} is 5 and \terme{Naval technology} is
9. 1\ARMY\facemoins, 1\FLEET\faceplus (8\NGD/1\NTD) are raised. 2 \Presidios
of level 2 are placed in \provinceMoreas and \provinceAlabania and 2 of level
1 are placed in \provinceHellas and \provinceMontenegro.


\subsubsection{Minor countries}
\aparag[Provinces]:
\bparag \provinceTrentino does initially belong to \paysHabsbourg.
\bparag \provinceBresse does initially belong to \paysSavoie.
\bparag \provinceGotland, \provinceVastergotland and \provinceSkane initially
belong to \paysdanemark.
\bparag \provinceMontenegro, \provinceSerbia and \provinceBosna are initially
neutral.

\aparag[Trade Fleets]:
\bparag \paysHollande has a \TradeFLEET lv. 5 in \ctz{Hollande}, lv. 2 in
\ctz{Espagne}, lv. 3 in \ctz{Angleterre}, lv. 5 in \ctz{France}, lv. 4 in
\stz{Baltique}, lv. 4 in \stz{Nord}, lv. 2 in \stz{Lion} and lv. 3 in
\stz{Ionienne}. The \CCs{Atlantic} is in \provinceVlaanderen
\bparag \paysecosse has a \TradeFLEET lv. 3 in \stz{Nord}.
\bparag \paysdanemark has a \TradeFLEET lv. 3 in \stz{Baltique} and lv. 1 in
\stz{Nord}.
\bparag \payshanse has a \TradeFLEET lv. 3 in \stz{Baltique} and lv. 2 in
\stz{Nord}.
\bparag \payssuede has a \TradeFLEET lv. 3 in \stz{Baltique}.
\bparag \paysgenes has a \TradeFLEET lv. 3 in \stz{Lion}, lv. 2 in
\ctz{Espagne} and lv. 2 in \stz{Ionienne}.
\bparag \paysoman has a \TradeFLEET lv. 2 in \stz{Oman}.
\bparag \paysaden has a \TradeFLEET lv. 2 in \stz{Indien}.
\bparag \paysgujarat has a \TradeFLEET lv. 4 in \stz{Oman}, lv. 2 in
\stz{Indien} and lv. 1 in \stz{Tempetes}. The \CCs{Indian} is in \villeDiu.
\bparag \paysjapon has a \TradeFLEET lv. 3 in \stz{Formose}.
\bparag \payschine has a \TradeFLEET lv. 3 in \stz{Formose}.

\aparag[\ROTW]
\bparag \paysOman has a \COL lv. 4 (\constructionOman, 1 \RES{Spices}) in
\province{Oman E} and a \TP lv. 2 (1 \RES{Spices}, 1\RES{Slaves}) in
\provinceZanzibar.
\bparag \paysAden has a \COL lv. 4 (\constructionAden, 1 \RES{Spices}) in
\province{Aden E}.
\bparag \paysgujarat has a \TP\ lv. 3 (2 \RES{PO}, 1 \RES{Spices}) in
\provinceDiu, a \TP\ lv. 1 (1 \RES{PO}) in \province{Malacca S}, a \TP\ lv. 1
(1 \RES{PO}) in \province{Malacca N}, a \TP\ lv. 1 (1 \RES{Slaves}) in
\province{Nyasa N}, a \TP\ lv. 1 (1 \RES{PO}) in \province{Kenya S}, a \TP\
lv. 1 (1 \RES{PO}) in \provinceOrmus, a \TP\ lv. 2 (1 \RES{PO}) in
\provinceMumbai, a \TP\ lv. 3 (1 \RES{PO}, 1 \RES{Spices}) in \provinceGoa, a
\TP\ lv. 3 (2 \RES{Spices}) in \provinceKolikot and a \TP\ lv. 2 (1 \RES{PO},
1 \RES{Spices}) in \province{Malabar S} (Cochin).
\bparag \paysgujarat also owns \granderegionMalacca and
\granderegionGujarat. The \TP in those regions benefit from the natives, and
the town protection in \provinceDiu, if needed.
\bparag \paysvijayanagar owns \granderegionOrissa, \granderegionGondwana,
\granderegionKarnatika, \granderegionMalabar, \granderegionHyderabad and
\granderegionMumbai.
\bparag \paysSiberie has a \TP\ lv. 3 (2 \RES{Fur}) in \province{Siberie S}.

\aparag[Miscellaneous] [BLP]
\bparag The \corsaire of \paysCyrenaique and \paysTunisie are not here. They
will arrive as reinforcement at turn 2.

\subsection{Transfers: New Situations of the Powers}
\aparag Some countries become major powers during the course of the game
(rather than in 1492). Their initial situation is described here.
\bparag The precise moment of change, as well as the player playing these
countries, depends on the number of players in the campaign.

\subsubsection{Becoming \paysmajeurSuede}
%\aparag \SUE becomes a major power at the start of period \period{III}.
%\bparag It is played by the player who previously played \POR.
%\bparag \POR thus becomes a minor power at the end of period \period{II}.
\aparag If event \eventref{pII:End Kalmar} did not occur before \SUE becomes a
\MAJ, it will happen as one event of the turn (as if rolled for) and one less
event is rolled for.
\aparag[Ruling Monarch.] The ruling monarch is \monarque{Gustav I} if the
\terme{union of Kalmar} is still active, \monarque{Eric XIV} else. The \STAB
is +2, minus the number of turns of an ongoing war.
\aparag[Owned provinces:] the ones already owned by \MIN \payssuede before the
transfer, usually \provinceSmaland, \provinceJamtland, \provinceGastrikland,
\provinceBergslagen, \provinceSvealand, \provinceFinland, \provinceTavastland,
\provinceNyland, \provinceKarelen.
\aparag[Diplomatic track] No special rules. Ongoing wars continue.
\aparag[Economical situation]
\bparag 1\MNU of \RES{Wood} on side\faceplus in \provinceSvealand, 1 \MNU of
\RES{Metal} on side\facemoins in \provinceJamtland, \FTI is 2 and \DTI is
3. \TradeFLEET: those that are here (3 levels in \seazoneBaltique in 1492).
\bparag Initial treasury is 150\ducats.
\aparag[Military]
\bparag \terme{Land technology} is placed 3 boxes behind the most advanced
counter (but at least at the level of the \terme{Latin} counter). \terme{Naval
  technology} is 1 box ahead of the \terme{Latin} counter.
\bparag 1\ARMY\faceplus, 2\LD, 1\FLEET\facemoins (2\NWD/1\NTD), the Transport
\FLEET\faceplus (with 4\NTD) and 2 fortress levels are already raised if
\payssuede was at peace and may be placed freely. If already at war, the
equivalent of 3\LD, the Transport \FLEET\faceplus (with 4\NTD) and 2 fortress
levels are raised for free in the owned territory, up to the limit above
(excess forces are lost). Forces already present remain in place.


\subsubsection{Becoming \paysmajeurHollande}
\aparag \HOL becomes a major power when \ref{pIII:Dutch Revolt} occurs. This
usually happens in the first turns of period \period{III}.
\bparag It is played by a new player.
\aparag See \ref{chSpecific:Holland:First Revolt} and \ref{pIII:Dutch Revolt}
for the initial state of \paysmajeurHollande.


\subsubsection{Becoming \paysmajeurAutriche}
\aparag[The Austrian Habsburgs.] If \AUS becomes a major country in period
\period{IV} or earlier, %
% start of pIV or TYW
the monarch is \monarque{Ferdinand II}, whose values and length are obtained
at random (a Dynastic Crisis is not possible).
% for the nine-players version,
%at the beginning of pIV or of \eventref{pIV:TYW} if it happens in pIII.

\aparag[The dissociation of the Habsburgs] (caused by \eventref{pV:WoSS}). %for
%the eight-players version.
If \AUS becomes a major country in period \period{V} or later, the monarch is
\monarque{Ferdinand III}, with values 6/8/7, whose reign length should be
rolled for (a Dynastic Crisis is not possible).

\aparag[General Situation]
\bparag The \STAB is +3, minus the number of turns of an ongoing war, adjusted
with the Major battles of the previous turn (only).
\bparag \HAB is \terme{Catholic/Counter-Reformation}.
\bparag[Owned provinces:] the territory of \payshabsbourg, or the ones decided
by \eventref{pV:WoSS} if the transfer takes place at the time of the
Dissociation.
\aparag[Diplomacy] A white peace or a negotiated peace (but not a formal
peace) may be negotiated and signed immediately in wars, excepted for the ones
that can be aggravated in \eventref{pIV:TYW}.
\bparag The minor powers on the Diplomatic Tracks are now liege of \AUS.

\aparag[Economical situation]
\bparag 1 \MNU in period I or II, 2 \MNU in period III or IV, 3 \MNU in period
V, and 4 if in period VI or VII; one being \facemoins and the rest (if any)
\faceplus. Initial \DTI is 2 in periods I to III, and 3 in period IV and
afterwards.  Initial \FTI is 2 in periods I to IV, and 3 in period V or
afterwards.
\bparag Initial treasury is 400\ducats.
\aparag[military]
\bparag Unless already placed, \terme{Land technology} is placed on the same
box as \SPA.
\bparag The equivalent of 3\ARMY\faceplus and 2 fortress levels are already
raised and may be placed freely. If already at war, the excess forces are
kept, and the missing forces are raised for free, but these will be
\terme{Conscripts}.

\subsubsection{Becoming \paysmajeurPrusse}
\label{chSpecific:Campaign:Becoming Prussia}
\aparag[Kingdom of Prussia] If \eventref{pIV:Great Elector} or
\eventref{pV:Kingdom Prussia} were not played yet, they are considered as the
first political event(s) rolled for this turn. However, the provinces of
\region{Duche de Prusse} still owned by \paysPologne are transferred
immediately to \PRU (no \VPs are won for this transfer).
\aparag[Ruling Monarch.] Before turn 51, the monarch is
\monarque{Friedrich-Wilhelm}. At the beginning of turn 51, the monarch becomes
\monarque{Friedrich II}. The \STAB is +3, minus the number of turns of an
ongoing war, and \PRU is \terme{Protestant}.
\aparag[Owned provinces:] the ones already owned by \paysbrandebourg, plus the
ones of \region{Duche de Prusse}.
\aparag[Diplomatic track] No special rules. Ongoing wars continue.
\aparag[Economical situation]
\bparag 2\MNU on side\faceplus, \FTI and \DTI is 4. No \TradeFLEET.
\bparag Initial treasury is 200\ducats.
\aparag[Military]
\bparag \terme{Land technology} is placed 3 boxes ahead of the \terme{Latin}
counter.
\bparag The equivalent of 2\ARMY\faceplus and 3 fortress levels are already
raised if \paysbrandebourg was at peace and may be placed freely. If already
at war, only the remaining forces of \paysbrandebourg are raised.


\subsubsection{Position of the now Minor country}
\aparag[Military and Economical situation] Unless explicitly mentioned
otherwise, the abandoned country keeps its position of the beginning of the
turn.
\bparag The military forces raised and the fortresses remain and are
maintained as \terme{Veteran} for the turn, in case of an ongoing conflict or
a war beginning at the turn of the transfer.  At the turn following the
transfer, the basic forces of the country become the ones of the \MIN,
excepted for the fortresses: they stay all as they are and are maintained
freely (until destroyed militarily).
\bparag The belongings of the \MAJ (\COL, \TP, provinces, \TradeFLEET) remain
in place. The \TradeFLEET levels serve as the reference level for future trade
operations of the \MIN (minimum levels are given in the annexes).
\bparag[Colonisation] In the case of \paysPortugal and \paysHollande that may
continue their overseas expansion and developing \COL as minor countries, the
\MAJ having them on its diplomatic track (or the first in the preferences,
should the country be \Neutral) manages the \COLaction. The \FTI used is 3
(and 4 from period IV onward), and the investment is a medium one.

\aparag[Diplomatic track of minor countries]\label{chSpecific:Campaign:Minor
  Diplomatic Track}
The countries becoming minor countries keep their diplomatic track and the
\MIN on it. They defend these against diplomatic actions with a medium
investment and a \DIP of 3 (total modifier {\bf +5}).
\bparag In case of war, they defend these countries if attacked, and they
systematically ask for their help in a conflict, if possible.

\aparag[Diplomatic position of the new minor country] If at the time of
transfer, the country was allied with another \MAJ, it is put on the
diplomatic track of this \MAJ: \RM for a dynastic alliance (see
\ruleref{chDiplo:Alliance:Dynastic Alliance}), \SUB for a defensive alliance
(see \ruleref{chDiplo:Alliance:Defensive Alliance}) and \MA for an offensive
alliance (see \ruleref{chDiplo:Alliance:Offensive Alliance}).
\bparag If several \MAJ were allied to the new \MIN, a diplomatic action must
be undertaken by all willing \MAJ (this counts as one of the diplomatic
actions of the turn). This roll is however done as soon as possible (before
the political events of the turn are rolled for even) and in all cases before
the diplomatic reactions.

\aparag[\VP summary] When a country is abandoned, an end of game \VP count has
to be done for this country. This is detailed in \ruleref{chVictories:End
  Game}


\subsection{Countries played by each player}
\subsubsection{The almost 9 players game}
\aparag The game is currently designed to be played by 9 players, one of which
only starting to play at the beginning of period \period{III}.
\bparag The diagram in Figure~\ref{figure:9 players} shows the countries
played by each one (each solid line corresponds to one player).
\bparag Six players only play one country: \ANG, \FRA, \HIS, \TUR, \RUS and
\HOL (from period \period{III} onwards).
\bparag Three players change country mid-game: \POR then \SUE, \VEN then \AUS
and \POL then \PRU.
\bparag The player playing \POR switch to \SUE at the interphase between turns
14 and 15 (end of period \period{II}/beginning of period \period{III}).
\bparag A new player starts playing \HOL as soon as \ref{pIII:Dutch Revolt}
occurs (usually during the first turns of period \period{III}.
\bparag The player playing \VEN starts playing \AUS when \ref{pIV:TYW} occurs
or at the interphase between turns 25 and 26 (end of period
\period{III}/beginning of period \period{IV}), whichever occurs first.
\bparag The player playing \POL starts playing \PRU when the first occurs
among: \ref{pVI:WoPS}, \ref{pVI:WoAS}, interphase between turns 50 and 51.
\begin{figure}\centering
  \begin{tikzpicture}[%
    bigline/.style={line width=3pt,rounded corners=2pt},%
    bigbigline/.style={line width=6pt,rounded corners=2pt}%
    ]
    \EUcampaigncolor{Angleterre}{1,.1,.6}{1,0,0}%
    \EUcampaigncolor{Espagne}{1,1,0}{1,.75,0}%
    \EUcampaigncolor{France}{.1,.1,.6}{.1,.6,.6}%
    \EUcampaigncolor{Autriche}{1,1,0}{0,0,0}%
    \EUcampaigncolor{Hollande}{.5,.5,.9}{.8,.8,.2}%
    \EUcampaigncolor{Portugal}{.9,1,.75}{.75,.95,.5}%
    \EUcampaigncolor{Prusse}{.75,.75,.75}{.25,.25,.25}%
    \EUcampaigncolor{Russie}{.55,.25,.1}{.95,.95,.9}%
    \EUcampaigncolor{Suede}{.7,.85,.9}{0,0,1}%
    \EUcampaigncolor{Turquie}{.5,.6,.5}{0,.6,0}%
    \EUcampaigncolor{Venise}{.75,0,0}{1,.8,0}%
    \EUcampaigncolor{Pologne}{1,.75,.75}{.75,.25,.25}%
    \EUcampaigncolor{Danemark}{1,0,0}{.9,.9,.9}%
    \EUcampaignlinel{France}{0}%
    \EUcampaignlinel{Turquie}{-.5}%
    \EUcampaignlinel{Angleterre}{-1}%
    \EUcampaignlinel{Russie}{-1.5}%
    \EUcampaignlinel{Espagne}{-2}%
    \EUcampaignlines{Portugal}{Suede}{-2.5}{-3}{22mm}%
    \EUcampaignlinelr{Hollande}{-3.5}{27mm}{10cm}{}%
    \node (Hollande) [anchor=base west,xshift=5pt,yshift=-3pt] at (10cm,-3.5)
    {\paysmajeurHollande};%
    \EUcampaignlines{Pologne}{Prusse}{-5}{-5.5}{85mm}%
    \EUcampaignlines{Venise}{Autriche}{-4}{-4.5}{43mm}%
    \begin{pgfonlayer}{background}
      \EUcampaignrepere{0}{-6}{pI}%
      \EUcampaignrepere{10cm}{-6}{VII-5 (2)/T62}%
      \EUcampaignrepere{22mm}{-3}{pIII}%
      \EUcampaignrepere{27mm}{-4}{III-1}%
      \EUcampaignrepere{43mm}{-5}{pIV/IV-A}%
      \EUcampaignrepere{85mm}{-6}{VI-11/VI-13/T51}%
      \EUcampaignrepere{5mm}{-5}{I-A}%
      \EUcampaignrepere{32mm}{-2.5}{III-7}%
      \EUcampaignrepere{49mm}{-2.5}{IV-4}%
      \EUcampaignrepere{70mm}{-2.5}{V-1/VI-7}%
      \EUcampaignlinell{Hollande}{-3.5}{dashed}{\paysprovincesne}%
      \EUcampaignlinell{Autriche}{-4.5}{dashed}{\payshabsbourg}%
      \EUcampaignlinelr{Espagne}{-2.5}{32mm}{49mm}{}%
      \EUcampaignlinelr{Espagne}{-2.5}{53mm}{70mm}{}%
      \EUcampaignlinelr{Espagne}{-4.5}{5mm}{43mm}{}%
      \EUcampaignlinell{Prusse}{-5.5}{dashed}{\paysbrandebourg}%
    \end{pgfonlayer}
  \end{tikzpicture}
  \caption{Standard game}\label{figure:9 players}
\end{figure}
%For the first two periods, nothing is changed and only 8 major powers are
%played. Then, on period III, a ninth player will take care of \HOL for the
%rest of the game.  \VEN remains a \MAJ in period III and transfers to
%\paysmajeurAutriche from period IV on.

\subsubsection{Eight-players game}
\aparag Initial powers are \TUR, \HIS, \FRA, \ANG, \RUS, \VEN, \POR and
\POL. The last three players will change powers during the course of play.

\aparag[Portugal-Sweden] \label{chSpecific:Campaign:Transfer Portugal} The
player of \paysmajeurPortugal abandons this \MAJ during the interphase between
period II and III (turns 14--15, 1560) and becomes \paysmajeurSuede.

\aparag[\paysmajeurVenise-\paysmajeurVenise] \label{chSpecific:Campaign:Transfer
  Venice} As long as \eventref{pIII:Dutch Revolt} does not happen,
\paysmajeurHollande does not exist and the player continues with playing
\paysmajeurVenise.  At the turn of this revolt, the player switches to
\paysmajeurHollande.
\bparag \paysmajeurVenise chooses objectives as if it were to play a complete
period III among the objectives of period II. It will mark (or lose) half of
the objectives value if at least 4 turns are played as \VEN in period III, and
mark it completely if at least 9 turns are played as \VEN in period III.

\aparag[\paysmajeurHollande-\paysmajeurAutriche] \label{chSpecific:Campaign:Transfer
  Holland} The choice of the \HOL-\AUS transfer must be made at the time where
\eventref{pV:WoSS} is \emph{rolled} (not activated) or at the beginning of
period VI, whichever is first.
\bparag \paysmajeurAutriche becomes a new major country played by the former
player of \paysmajeurHollande at the time of the dissociation.
\bparag During the War of Spanish Succession, \AUS also remains the ruler of
\paysmajeurHollande as a major country until the conflict ends (and the player
scores \VP following the general situation of \paysmajeurHollande at this
moment).
\bparag However, if period V is not finished, \paysmajeurHollande remains
played as a \MAJ until the end of the period. While playing the two countries,
\HOL/\AUS is restricted for \HOL to sign only defensive alliances, and cannot
declare war for \HOL without a \CB.

\bparag[Sweden-Austria] \label{chSpecific:Campaign:Transfer Sweden} If \HOL
refuses the transfer to \AUS, \SUE may opt for the transfer instead. The
choice is made just after the refusal by \HOL.  The same conditions apply:
both countries are played during \eventref{pV:WoSS}, \SUE is played as a \MAJ
until the peace or the end of period V (whichever is the latest), etc.

\aparag[Poland-Prussia] \label{chSpecific:Campaign:Transfer Poland} The player
for \paysmajeurPologne takes the control of \paysmajeurPrusse at the beginning
of period VI (at the earliest) and at the beginning of turn 51, according to
the political events rolled for (see~\ruleref{chapter:Events}).
\bparag The events \eventref{pVI:WoPS}, \eventref{pVI:WoAS} and
\eventref{pVII:Seven Years War} trigger the change to \paysmajeurPrusse if in
period VI. If none of those happen, the transfer happens at the beginning of
turn 51.

\begin{figure}\centering
  \begin{tikzpicture}[%
    bigline/.style={line width=3pt,rounded corners=2pt},%
    bigbigline/.style={line width=6pt,rounded corners=2pt}%
    ]
    \EUcampaigncolor{Angleterre}{1,.1,.6}{1,0,0}%
    \EUcampaigncolor{Espagne}{1,1,0}{1,.75,0}%
    \EUcampaigncolor{France}{.1,.1,.6}{.1,.6,.6}%
    \EUcampaigncolor{Autriche}{1,1,0}{0,0,0}%
    \EUcampaigncolor{Hollande}{.5,.5,.9}{.8,.8,.2}%
    \EUcampaigncolor{Portugal}{.9,1,.75}{.75,.95,.5}%
    \EUcampaigncolor{Prusse}{.75,.75,.75}{.25,.25,.25}%
    \EUcampaigncolor{Russie}{.55,.25,.1}{.95,.95,.9}%
    \EUcampaigncolor{Suede}{.7,.85,.9}{0,0,1}%
    \EUcampaigncolor{Turquie}{.5,.6,.5}{0,.6,0}%
    \EUcampaigncolor{Venise}{.75,0,0}{1,.8,0}%
    \EUcampaigncolor{Pologne}{1,.75,.75}{.75,.25,.25}%
    \EUcampaigncolor{Danemark}{1,0,0}{.9,.9,.9}%
    \EUcampaignlinel{France}{0}%
    \EUcampaignlinel{Turquie}{-.5}%
    \EUcampaignlinel{Angleterre}{-1}%
    \EUcampaignlinel{Russie}{-1.5}%
    \EUcampaignlinel{Espagne}{-2}%
    \EUcampaignlines{Portugal}{Suede}{-2.5}{-3}{22mm}%


    \EUcampaignlinet{Venise}{Hollande}{Autriche}{-3.5}{-4}{-4.5}{27mm}{65mm}{50mm}%
    \EUcampaignlines{Pologne}{Prusse}{-5}{-5.5}{85mm}%
    \begin{pgfonlayer}{background}
      \EUcampaignrepere{0}{-5.5}{pI}%
      \EUcampaignrepere{10cm}{-5.5}{VII-5 (2)/T62}%
      \EUcampaignrepere{22mm}{-3}{pIII}%
      \EUcampaignrepere{27mm}{-4}{III-1}%
      \EUcampaignrepere{85mm}{-5.5}{VI-11/VI-13/T51}%
      \EUcampaignrepere{5mm}{-5}{I-A}%
      \EUcampaignrepere{32mm}{-2.5}{III-7}%
      \EUcampaignrepere{49mm}{-2.5}{IV-4}%
      \EUcampaignrepere{74mm}{-2.5}{V-1/VI-7}%
      \EUcampaignrepere{65mm}{-4.5}{V-1}%
      \EUcampaignlinell{Hollande}{-4}{dashed}{\paysprovincesne}%
      % \begin{scope}
      %   \path[clip] (-1,0) rectangle (5,-6);
      %   \EUcampaignlinelr{Hollande}{-4}{0mm}{60mm}{dashed}%
      % \end{scope}
      \EUcampaignlinelr{Espagne}{-2.5}{32mm}{49mm}{}%
      \EUcampaignlinelr{Espagne}{-2.5}{53mm}{74mm}{}%
      \EUcampaignlinell{Prusse}{-5.5}{dashed}{\paysbrandebourg}%
      \EUcampaignlinell{Autriche}{-4.5}{dashed}{\payshabsbourg}%
    \end{pgfonlayer}
  \end{tikzpicture}
  \caption{8 players game}\label{figure:8 players}
\end{figure}

\subsubsection{Nine-players game: full \AUS}
\aparag This is a variant for 9 players all along the game with a major \AUS
from the beginning.
\begin{figure}\centering
  \begin{tikzpicture}[%
    bigline/.style={line width=3pt,rounded corners=2pt},%
    bigbigline/.style={line width=6pt,rounded corners=2pt}%
    ]
    \EUcampaigncolor{Angleterre}{1,.1,.6}{1,0,0}%
    \EUcampaigncolor{Espagne}{1,1,0}{1,.75,0}%
    \EUcampaigncolor{France}{.1,.1,.6}{.1,.6,.6}%
    \EUcampaigncolor{Autriche}{1,1,0}{0,0,0}%
    \EUcampaigncolor{Hollande}{.5,.5,.9}{.8,.8,.2}%
    \EUcampaigncolor{Portugal}{.9,1,.75}{.75,.95,.5}%
    \EUcampaigncolor{Prusse}{.75,.75,.75}{.25,.25,.25}%
    \EUcampaigncolor{Russie}{.55,.25,.1}{.95,.95,.9}%
    \EUcampaigncolor{Suede}{.7,.85,.9}{0,0,1}%
    \EUcampaigncolor{Turquie}{.5,.6,.5}{0,.6,0}%
    \EUcampaigncolor{Venise}{.75,0,0}{1,.8,0}%
    \EUcampaigncolor{Pologne}{1,.75,.75}{.75,.25,.25}%
    \EUcampaigncolor{Danemark}{1,0,0}{.9,.9,.9}%
    \EUcampaignlinel{France}{0}%
    \EUcampaignlinel{Turquie}{-.5}%
    \EUcampaignlinel{Angleterre}{-1}%
    \EUcampaignlinel{Russie}{-1.5}%
    \EUcampaignlinel{Espagne}{-2}%
    \EUcampaignlines{Portugal}{Suede}{-2.5}{-3}{22mm}%
    \EUcampaignlines{Venise}{Hollande}{-3.5}{-4}{27mm}%
    \EUcampaignlines{Pologne}{Prusse}{-5}{-5.5}{85mm}%
    \EUcampaignlinel{Autriche}{-4.5}%
    \begin{pgfonlayer}{background}
      \EUcampaignrepere{0}{-5.5}{pI}%
      \EUcampaignrepere{10cm}{-5.5}{VII-5 (2)/T62}%
      \EUcampaignrepere{22mm}{-3}{pIII}%
      \EUcampaignrepere{27mm}{-4}{III-1}%
      \EUcampaignrepere{85mm}{-5.5}{VI-11/VI-13/T51}%
      \EUcampaignrepere{5mm}{-5}{I-A}%
      \EUcampaignrepere{32mm}{-2.5}{III-7}%
      \EUcampaignrepere{49mm}{-2.5}{IV-4}%
      \EUcampaignrepere{70mm}{-2.5}{V-1/VI-7}%
      \EUcampaignlinell{Hollande}{-4}{dashed}{\paysprovincesne}%
      \EUcampaignlinelr{Espagne}{-2.5}{32mm}{49mm}{}%
      \EUcampaignlinelr{Espagne}{-2.5}{53mm}{70mm}{}%
      \EUcampaignlinell{Prusse}{-5.5}{dashed}{\paysbrandebourg}%
    \end{pgfonlayer}
  \end{tikzpicture}
  \caption{9 players game with full \AUS}\label{figure:9 players AUS}
\end{figure}
\paysmajeurAutriche is added as a \MAJ from period I. There is a mandatory
offensive and defensive alliance with \SPA at all time, that may evolve in an
weak defensive alliance with \SPA after the end of \eventref{pIV:TYW}, and it
disappears in any case with \eventref{pV:WoSS}.

\aparag[The Habsburg Dynastic Alliance]
(see also \ruleref{chSpecific:Habsburg Dynastic Alliance:Major})
\bparag At the beginning, \SPA and \AUS are always linked by a mandatory
alliance, even if they fail to answer it or even at war against one another
(so that they still may answer the alliance against other powers).  They can
do full or limited intervention, both in offensive or defensive stance.
\bparag During that time, \SPA does not lose \STAB\ to use the defensive
alliance to help \AUS.
\bparag However, if they are not using the \CB given by this Alliance, they
are not necessarily allied unless they announce it (and could so make separate
peace at no cost, and so on).
\bparag They may be at war against one another, but only if using a legitimate
\CB to do so.
\bparag They are no limit to money transfer between them.
\bparag At the end of \eventref{pIV:TYW}, if both \SPA and \AUS has achieved
Neutral or Losing positions, the mandatory alliance becomes defensive only and
is weakened in the sense that a limited intervention is sufficient to fulfil
it. The mandatory alliance is not offensive anymore.
\bparag At the beginning of \eventref{pV:WoSS}, there isn't anymore a Dynastic
Alliance. Note however that, depending on the choice of the Heir, there might
be different kinds of Dynastic Ties as described in this event.

%\bparag[ADD Details] for period I and II.

\subsubsection{Nine-players game: \DAN}
\aparag[Alternate version: 9 players with \DAN] \paysmajeur{Danemark} is
played in the first two periods. This setting is not completely thought-out
yet, and may never see the light (\DAN does not interact with enough other
players to the conceptors' taste).
\begin{figure}\centering
  \begin{tikzpicture}[%
    bigline/.style={line width=3pt,rounded corners=2pt},%
    bigbigline/.style={line width=6pt,rounded corners=2pt}%
    ]
    \EUcampaigncolor{Angleterre}{1,.1,.6}{1,0,0}%
    \EUcampaigncolor{Espagne}{1,1,0}{1,.75,0}%
    \EUcampaigncolor{France}{.1,.1,.6}{.1,.6,.6}%
    \EUcampaigncolor{Autriche}{1,1,0}{0,0,0}%
    \EUcampaigncolor{Hollande}{.5,.5,.9}{.8,.8,.2}%
    \EUcampaigncolor{Portugal}{.9,1,.75}{.75,.95,.5}%
    \EUcampaigncolor{Prusse}{.75,.75,.75}{.25,.25,.25}%
    \EUcampaigncolor{Russie}{.55,.25,.1}{.95,.95,.9}%
    \EUcampaigncolor{Suede}{.7,.85,.9}{0,0,1}%
    \EUcampaigncolor{Turquie}{.5,.6,.5}{0,.6,0}%
    \EUcampaigncolor{Venise}{.75,0,0}{1,.8,0}%
    \EUcampaigncolor{Pologne}{1,.75,.75}{.75,.25,.25}%
    \EUcampaigncolor{Danemark}{1,0,0}{.9,.9,.9}%
    \EUcampaignlinel{France}{0}%
    \EUcampaignlinel{Turquie}{-.5}%
    \EUcampaignlinel{Angleterre}{-1}%
    \EUcampaignlinel{Russie}{-1.5}%
    \EUcampaignlinel{Espagne}{-2}%
    \EUcampaignlines{Portugal}{Suede}{-2.5}{-3}{22mm}%
    \EUcampaignlines{Danemark}{Hollande}{-3.5}{-4}{27mm}%
    \EUcampaignlines{Pologne}{Prusse}{-5.5}{-6}{85mm}%
    \EUcampaignlines{Venise}{Autriche}{-4.5}{-5}{43mm}%
    \begin{pgfonlayer}{background}
      \EUcampaignrepere{0}{-6}{pI}%
      \EUcampaignrepere{10cm}{-6}{VII-5 (2)/T62}%
      \EUcampaignrepere{22mm}{-3}{pIII}%
      \EUcampaignrepere{27mm}{-4}{III-1}%
      \EUcampaignrepere{43mm}{-5}{pIV/IV-A}%
      \EUcampaignrepere{85mm}{-6}{VI-11/VI-13/T51}%
      \EUcampaignrepere{5mm}{-5}{I-A}%
      \EUcampaignrepere{15mm}{-3}{II-4}%
      \EUcampaignrepere{32mm}{-2.5}{III-7}%
      \EUcampaignrepere{49mm}{-2.5}{IV-4}%
      \EUcampaignrepere{70mm}{-2.5}{V-1/VI-7}%
      \EUcampaignlinell{Hollande}{-4}{dashed}{\paysprovincesne}%
      \EUcampaignlinell{Autriche}{-5}{dashed}{\payshabsbourg}%
      \EUcampaignlinelr{Espagne}{-2.5}{32mm}{49mm}{}%
      \EUcampaignlinelr{Espagne}{-2.5}{53mm}{70mm}{}%
      \EUcampaignlinelr{Espagne}{-5}{5mm}{43mm}{}%
      \EUcampaignlinell{Prusse}{-6}{dashed}{\paysbrandebourg}%
      \EUcampaignlinelr{Danemark}{-3}{0}{15mm}{}%
    \end{pgfonlayer}
  \end{tikzpicture}
  \caption{9 players game with \DAN}\label{figure:9 players DAN}
\end{figure}

\subsubsection{Seven-players game}
\aparag[Solution 1] \paysmajeurPologne becomes a \MIN. \paysmajeurSuede is
transferred to \paysmajeurPrusse.

\aparag[Solution 2] \paysmajeurSuede becomes a \MIN. \paysmajeurRussie is a
\MIN in periods I-II. \paysmajeurPortugal is transferred to \paysmajeurRussie,
and \paysmajeurPologne is transferred to \paysmajeurPrusse.
\bparag The chapter on \SUE is to ignore/rewrite, but for the limits of
counters and the leaders. \textit{53.27} is resurrected.
\bparag Some events have to be rewritten: III-4 becomes II-2 of Risto, but
\SUE obtains \provinceGotland and \provinceSkane. III-22 is to be rewritten
(\POL may declare war to \SUE only once) ; IV-0 is deleted ; IV-2 (1 and 2),
\SUE uses the \CB ; IV-17, use the one of Risto (\SUE obtains
\provinceVastergotland ; V-1, V-3, V-13 play as written by Risto.


\subsubsection{Six-players game}
\aparag As in the normal game, \paysmajeurPologne, \paysmajeurSuede,
\paysmajeurPrusse and \paysmajeurRussie (in periods I and II) become minor
countries.
\bparag The player playing \paysmajeurPortugal takes the control of
\paysmajeurRussie at the beginning of period \period{III}.
\aparag Political events have to be modified to take all this into
account. One would rather choose to use the Risto events, that are a coherent
and historically plausible set.



% Local Variables:
% fill-column: 78
% coding: utf-8-unix
% mode-require-final-newline: t
% mode: flyspell
% ispell-local-dictionary: "british"
% End:

% LocalWords: pII pI bigbigline Angleterre Espagne Autriche Hollande Prusse
% LocalWords: Turquie Venise Pologne Danemark pIII interphase pV WoSS WoPS de
% LocalWords: TYW xshift yshift Habsburg Risto Kalmar Habsburgs Duche Nord os
% LocalWords: Ile Ionienne Mancha Castilla Nueva Vieja Illes Balears Joao pVI
% LocalWords: Preussen Tras Montes Praya Cabo malus Bonne Esperance Guinee
% LocalWords: Bayezid Noire Mediterranee lv Indien Tempetes Formose Siberie
% LocalWords: bigline Russie WoAS pVII pIV Baltique Canarias Zygmunt
