% -*- mode: LaTeX; -*-
\chapter{Political Events of Period I}
%\section{Period I}
\label{events:pI}

% ref tests :
% \begin{itemize}
% \item no page \nopageref{events:pII}
% \item short \shortref{events:pII}
% \item number \numberref{events:pII}
% \item number long \numberlongref{events:pII}
% \item xname \xnameref{events:pII}
% \item name \nameref{events:pII}
% \end{itemize}

\subsection*{Event Table of Period I}

\begin{eventstable}[Period I events table]
  \tabcolsep=5pt\centering
  \begin{tabular}{|l|*{5}{c}|l|}
    \hline
    1\up{st}\textarrow & 1-4 & 5-6 & 7 & 8 & 9 & 10 \\\hline
    1 & 1  & R2 & 3   & R15 & R16 & \textbullet~1--2 \\
    2 & 1  & 3  & R11 & R14 & 3   &  +1 then \\
    3 & 1  & 10 & R12 & 4   & R11 & \nameref{events:pII}\\
    4 & 3  & 12 & R4  & 7   & R15 & \textbullet~3--10: \\
    5 & 5  & 13 & R8  & 11  & R4  & \nameref{events:pII} \\
    6 & R6 & 4  & R4  & R6  & R8  & \\
    7 & R2 & R6 & R5  & R8  & R3  & \\
    8 & 7  & 9  & 8   & 9   & R16 & \\
    9 & 11 & 13 & 3   & 10  & R7  & \\\hline
    10 & \multicolumn{5}{l}{\nameref{events:pII}} & \\\hline
  \end{tabular}
\end{eventstable}
\begin{eventstablespec}[General modifiers for the period]
  % \oldref{Rule 53.23.A, modified} \\
  After \continentAmerica has been discovered and until \ref{pI:Tordesillas}
  is rolled-for the first time, use the following modifiers for both dice each
  turn when rolling for events (a result less than 1 is 1):
  \begin{modlist}
  \item[\bonus{-1}] If \SPA is \CATHCR or \ref{pI:Reformation2} has not
    occurred;
  \item[\bonus{-1}] If new \COL/\TP counters were placed in \continentAmerica
    last turn;
  \item[\bonus{-1}] If \SPA or \POR control \payspapaute.
  \end{modlist}
\end{eventstablespec}
% grep '^.evnt' engEvnt1.tex|cut -f2 -d\[|cut -f1 -d\]|sed -e 's/$/|,%/g'#$

\eventssummary{%
  pI:Tordesillas|,%
  pI:Emperor Election|,%
  pI:War Italy Napoli|,%
  pI:War Italy Milano|,%
  pI:Hungarian Freedom|,%
  pI:Bohemian Alliance|,%
  pI:Hungarian Alliance|,%
  pI:Milanese Alliance|,%
  pI:Habsburg Dynasty|E/E,%
  pI:Revolt Comuneros|,%
  pI:Reformation|,%
  pI:Reformation2|,%
  pI:Reformation3|,%
  pI:Turkish Diplomatic Pressure|S{pI:TD:Barbaross
    brothers}/S{pI:TD:Vassalisation of Algeria}/S{pI:TD:Alignment
    Barbaresques}/E/E/E
,%
  pI:War Scotland|,%
  pI:End Golden Horde|,%
  pI:Pskov Ryazan|,%
} \eventssummary{%
  pI:War Russia Poland|,%
  pI:War Roads Spices|S{pI:WRS:War Indian Sea}/S{pI:WRS:Veneto-Turkish
    Commercial Dispute},%
  pI:Resistance Muslim Traders|,%
  pI:Chinese Expeditions|,%
  pI:Barbaros Brothers|,%
  O|,%
  pI:Habsburg Alliance|,%
  pI:Burgundy Inheritance|,%
  pI:Habsburg Bohemia|,%
  pI:Habsburg Hungary|,%
  pI:Fall Hungary|,%
  pI:Habsburg Milano|,%
  pI:Spanish Milano|,%
  pI:Fall Teutonic|,%
  pI:Spanish Naples|,%
}

\newpage\startevents



\event{pI:Tordesillas}{I-1}{Treaty of Tordesillas}{1}{RistoMod}

\history{1494}
\dure{end of Period III, or until \ref{pIII:Portuguese Annexation}, whichever
  comes first}

\condition{}
\aparag Re-roll and do not mark off if \continentAmerica has not been
discovered.
% or apply War in Italy (Napoli) ??
\aparag Both \SPA and \POR have to accept this event for it to take
effect. Otherwise it is marked off, but can occur again.

\phevnt
\aparag \FRA and \ENG receive a temporary \CB for this turn to declare war
against both \SPA and \POR.
\aparag \SPA and \POR receive each 50\ducats.

\effetlong
\aparag From now on \SPA and \POR have specific areas for overseas expansions:
\bparag The exclusive area of \POR contains \continent{Middle East},
\continentSiberia, \continentAsia (except \granderegionPhilippines,
\continent{Extreme Orient}), \continentAfrica, \continentBrazil.
\bparag The exclusive area of \SPA contains \continentAmerica except
\continentBrazil, \granderegionAmazonia, and \granderegion{Minas Gerais}.
\bparag The regions \granderegionAmazonia, \granderegion{Minas Gerais},
\granderegionPhilippines, and \continent{Extreme Orient} are shared.
\aparag[Effects on \SPA and \POR]
\bparag All markers of \SPA or \POR currently on map in the exclusive area of
the other \MAJ are immediately destroyed, or may be replaced by the other \MAJ
by equivalent markers of its own if there are some available, and the \MAJ
fulfils the conditions to place such a marker here (especially discoveries).
\bparag Their movements are limited to their areas, the sea zones bordering
them, and sea zones that borders only islands. \POR may also go in sea zones
\seazone{Horn} and \seazone{Chili}. \SPA may also go in
\seazone{Bonne-Esperance} and \seazone{Tempetes}.
\bparag For each stack of the country violating this restriction, this country
loses {\bf 1} \STAB per restricted province or sea zone trespassed in. All
units of \SPA or \POR in prohibited zones when the Treaty is signed must
immediately return home as per normal peace procedure.
\bparag Until the end of period II, \SPA and \POR have free overseas
\terme{CB} against any Minor country in his area, and against any European
country (Catholic or not) trespassing their area. The free overseas \terme{CB}
might be used in reaction at the end of a round where a trespassing occurs, or
at the beginning of the next turn.
\bparag Until the end of period \period{III}, \SPA and \POR may attack
\terme{Minor establishments} in their area at no cost in \STAB.
\bparag Until the end of period III, \SPA and \POR have the capability to burn
down European \COL installed in their area (same condition as for burning
\TP).
\bparag Spanish Missionaries and Missions can only go in the exclusive area of
\SPA until the end of the Treaty (not in the shared area).
\aparag[Shared Area]
\bparag The regions that can be disputed between \POR and \SPA can be explored
by both countries and they can settle \COL and \TP without penalty.
\bparag If \POR has a \TP or \COL in \granderegionPhilippines,
\granderegionAmazonia or \granderegion{Minas Gerais}, \SPA gains an Overseas
\CB against \POR.
\bparag If \SPA has a \TP or \COL in \granderegionPhilippines or
\continent{Extreme Orient}, \POR has an Overseas \CB against \SPA.
\aparag[Effects on other countries] Until the end of period II all \COL/\TP
placement attempts (successful or not), and any movement of units from the
European Map into \ROTW (not in the other way) by Catholic players other than
\SPA or \POR entail to this \MAJ a malus of \bonus{-2} to \STAB improvement
action and the loss of control of \payspapaute. This malus is applied at most
once per turn.
\aparag The \terme{Treaty of Tordesillas} can be declared void by \POR or \SPA
as a Diplomatic Announcement. The Treaty is at an end, the other power gains
an immediate free \terme{CB} against the announcer; the announcer loses
diplomatic control of \payspapaute.



\event{pI:Emperor Election}{I-2}{Election of the \HRE Emperor}{1}{RistoMod}

\history{1519}
\dure{until the Emperor is Habsburg.}

\phevnt
\aparag Election of the new Emperor has to be conducted. The pretenders are
the monarchs of \SPA, \FRA, \ENG, and \POL if they are Catholic.  Each
Pretender makes a secret bid of Ducats (a multiple of 10\ducats) for the
title, and the country with the highest bid will win. In case of draw, those
Pretenders bid again secretly an additional bid. All bids are lost.
\aparag A candidate from one (unspecified) minor country makes a bid of (1d10
$\times$ 10)+30 \ducats (rolled for after the initial bids of the players ; in
case of ties, this candidate bids in addition of 1d10 times 10\ducats). If the
winner of the bid, and so the Emperor is a from a Minor Power, he will live
1d10 turns before a new election takes place. The event is still marked.
\aparag Each Minor Country that has the Electoral Dignity: \paysCologne,
\paysPalatinat, \paysSaxe, \paysTreves, \paysMayence, \paysBrandebourg,
\paysBoheme gives a free bid of 10\ducats to the Pretender secretly decided at
the bidding time by the Major Country having Diplomatic alliance with the
Electorate. If they are Neutral, they give their bid to the pretender of the
minor country
% Jym, 05/2013, following Pierre's notes 2007 :
\aparag At the first election, the House of Fugger may provide an immediate
international loan of 50\ducats for this election
% (interest rate of 5\% for a duration of 5 turns)
to \SPA, or 100\ducats if the monarch of \SPA is \monarque{Charles V}, that
are directly put in the RT.
\aparag The winner of the Crown gains 75\ducats and 10 VP.
\aparag The new Emperor has now the benefice pertaining to the \HRE,
see~\ruleref{chSpecific:HRE}.
\aparag If the Emperor is Spanish and \ref{pI:Habsburg Alliance} is in effect,
\SPA gains the possibility to involve \HAB in all wars in which \SPA is
currently involved (both as attacker and defender), but not conversely. This
is made with a free \CB.
\aparag If the Emperor is \FRA, \ENG or \POL, \SPA may declare a war (with no
\CB) against the new Emperor and \AUSaus will help in a offensive alliance. A
valid victory condition for \SPA is to cause immediately a new election where
the losing power can not be a candidate.

\effetlong
\aparag When the elected emperor dies or converts to another religion, a new
election occurs with the same system during the very next Event Phase.
\aparag However, if the dying Emperor is Spanish, the event terminates
permanently and no elections are held. The title of emperor reverts back to
\AUSaus and all effects of the event are cancelled. Furthermore, the
\dynasticaction{C}{2} is activated now if possible.
% C2=Spanish Milanese
\bparag Exception: If \monarque{Charles V} has not been reigning yet, then the
emperor stay \HIS.



\event{pI:War Italy Napoli}{I-3 (1)}{Wars in Italy (Napoli)}{1}{RistoMod}

\history{1494-1504, 1508}

\condition{Mandatory War.}
\aparag If \FRA is Protestant, mark off the event but apply \RD with the
\REVOLT in \FRA.
\aparag If \paysNaples exists no more, mark off the event, then apply and mark
off the second event.
\aparag The second event can not take place if the first one is not
finished. In that case re-roll and do not mark off. \phevnt
\aparag \FRA has a Mandatory \CB against \paysNaples. This \CB has to be used
this turn or the next, at the phase of Declaration of War. If the \CB is used,
the controller of \paysNaples may abandon the minor country with no cost, even
if it is own \VASSAL (because of valid Dynastic Claims of the French King).
\aparag If \FRA is already at war against this country, the war is linked to
this event at this turn and that fulfils the Mandatory \CB.

\phdipl
\aparag[Refusing the event]
\bparag At the very beginning of the Declarations Phase, \FRA may refuse the
event.
\bparag If \FRA refuses the event, it loses {\bf 2} \STAB and the rest of the
event is ignored.
\aparag[Entry in War of the Italian countries]
\bparag The following countries may be involved by themselves in the war:
\paysGenes, \paysMilan, \payspapaute, \paysSavoie, The following tests are
made each turn of the war (excepted if the \MIN was already forced out of the
war by a separate peace).
\bparag Those countries in the list that are allied to a \MAJ involved in the
war, make a mandatory test of Entry in the War as per the usual rules
\ruleref{chDiplo:Entry War Minor}, excepted that the \MAJ has no choice here
and this test is made even if the \MIN is not in \EG; if the \MIN is not in
\EG at least, use \bonus{-2} to the die roll in the test and a failure does
not change the diplomatic status of the \MIN.
\bparag Those countries in the list that are \Neutral, may join the following
\MAJ according to the roll of 1d10: 1 \FRA, 2-3 \HAB, 4 \VEN, 5-6 enters war
by itself, 7-10 stays \Neutral. A country joins a \MAJ only if it is involved
in the war ; it is then put in \EG of this \MAJ, and declares war of the
enemies of this \MAJ. If the \MAJ is not involved in the war, the \MIN stays
\Neutral.
\bparag If a \MIN enters war by itself, it declares war to all involved
countries then it asks help of the preferred country in its list that is not
one of its enemies.
\aparag[Diplomatic effects of the wars] \FRA has a bonus of \bonus{+2} for its
diplomacy on \paysToscane and \bonus{-1} for \payspapaute and \paysParme
during the event.
\aparag[The Serenissima in the Wars in Italy]
\bparag \VEN has a \CB against \FRA and/or \paysNaples, as long as the war is
not finished.
\bparag During this war also, \VEN may make limited intervention at the side
of any involved alliance each turn. Such limited intervention can begin at any
turn (not only the first) and \VEN can change side between turns. \VEN may
force any Italian \MIN in limited intervention for the enemy alliance, to be
fully involved in the war at no cost.
\bparag Conversely, \FRA and \HAB both have a free \CB against \VEN, to be
used at any turn of the war or on the turn following its conclusion.

\phmil
\aparag[First turn of the war] \FRA has the right of free access and supply in
all Italian minor countries not engaged in this war. Supply is not given by a
province if its city is besieged by country hostile to this city.
\aparag[Restricted War Field]
\bparag The war is restricted to \regionItalie if no side broadens the zone of
war.
\bparag The war is no more restricted if the side of \FRA invades a province
outside \regionItalie of the other side. \FRA loses immediately {\bf 1} \STAB
and 20 \VP, and if the invasion was not due to \FRA, the Major Power
responsible for it loses also {\bf 1} \STAB and 20 \VP. However, if dynastic
actions \shortdynasticaction{A}{1} and \shortdynasticaction{A}{2} have both
been played, the penalty in \VP is void.
\bparag The war is also no more restricted if the side enemy of \FRA invades a
province of the side of \FRA outside \regionItalie and that stack does not
draw its supply from \regionItalie.
\aparag At the time a stack of \FRA invades \provinceCampania, \FRA, \SPA and
\HAB gain free access in, but only supply across, Italian minor countries not
engaged in this war. Supply across a province is impossible if its city is
under siege by an enemy of this city.

\phpaix
\aparag During this war, \FRA may annex \provinceCampania as a regular
province, even if it's a capital.

\phinter
\aparag If \FRA does not manage the military conquest of \ville{Napoli} at any
time of this war, it loses 10 \VP at the end of the event.
\aparag If, on the contrary, \FRA annexes \provinceCampania, it gains 10\VP.
\aparag[Spanish reaction] \SPA has to choose to do \dynasticaction{A}{3} as
one of its diplomatic action on the turn following the beginning of the war
(this will use a Diplomatic action, with no cost and automatic success, but
\SPA is allowed another Dynastic Action this turn), thus activating event
\ref{pI:Spanish Naples} or renounces to its Inheritance: it then loses {\bf 3}
\STAB, and \dynasticaction{A}{3} is considered played for no effect.

\effetlong
\aparag If at any time of this war \FRA manages the military conquest of
\ville{Napoli}, it gains a \CB against \TUR for the rest of the period.
Moreover, \FRA may now annex \provinceTrakya until the end of the period.
\aparag Until the end of the current period, \FRA has a permanent \CB against
the owner of \provinceCampania.



\event{pI:War Italy Milano}{I-3 (2)}{Wars in Italy (Milano)}{1}{RistoMod}

\history{1510-1511 / 1513-1515}
\dure{Until the end of the war caused by this event.}

\condition{Mandatory War.}
\aparag If \FRA is Protestant, marked off the event but apply \RD with the
\REVOLT in \FRA.
\aparag The second event can not take place if the first one is not
finished. In that case re-roll and do not mark off.

\phevnt
\aparag \FRA has a Mandatory \CB against the owner of \provinceLombardia. This
\CB has to be used this turn or the next, at the phase of Declaration of
War. If \FRA is \CATHCR after \ref{pI:Reformation}, the \CB is free.
\aparag If \FRA is already at war against this country, the war has to become
the war linked to this event at this turn or the following (the choice is made
by \FRA during the Declarations of War) and that fulfils the Mandatory \CB.
\aparag If \FRA owns \provinceLombardia, any former owner of this province has
a free \CB against \FRA.

\phdipl
\aparag[Refusing the event]
\bparag At the very beginning of the Declarations Phase, \FRA or the owner of
\provinceLombardia may refuse the event.
\bparag If \FRA refuses the event, it loses {\bf 2} \STAB and the rest of the
event is ignored.
\bparag If the owner of \provinceLombardia refuses the event, it loses {\bf 3}
\STAB and gives \provinceLombardia to \FRA (or its former controller if it was
\FRA that refused the event). Then the rest of the event is ignored. If this
province is owned by the \HAB, \SPA may refuse the event (and lose the \STAB).
\aparag[Milan as a Minor country] If \provinceLombardia is owned by the Minor
country \paysMilan, \HAB have a free \CB in reaction to a Declaration of War
of \FRA against this country. \paysMilan is moved up to \EG on the diplomacy
track of \HAB if it was not already on a higher position.
\aparag[The Papacy and the war] If \payspapaute is allied to a \MAJ involved
in the war, each turn make a mandatory test of Entry in the War is made as per
the usual rules \ruleref{chDiplo:EW Effects}, excepted that the \MAJ has no
choice here and this test is made even if the \MIN is not in \EG; if the
\paysPapaute is not in \EG at least, use \bonus{-2} to the die roll in the
test and a failure does not change its diplomatic status. Exception: if
\payspapaute was forced out of this war, it does not enter back in it.
\aparag[Diplomatic effects of the wars] \FRA has a bonus of \bonus{+2} for its
diplomacy on \paysToscane and \bonus{-1} for \payspapaute and \paysParme
during the event.
\aparag[The Serenissima in the Wars in Italy]
\bparag \VEN has a \CB against \FRA and/or the owner of \provinceLombardia, as
long as the war is not finished.
\bparag During this war, \VEN may make limited intervention at the side of any
involved alliance each turn. Such limited intervention can begin at any turn
(not only the first) and \VEN can change side between turns.  \VEN may force
any Italian \MIN in limited intervention for the enemy alliance, to be fully
involved in the war at no cost.
\bparag Conversely, \FRA and \HAB both have a (normal) \CB against \VEN, to be
used at any turn of the war.
\aparag[Swiss Mercenaries] If \paysMilan is (or was) a vassal of \HAB
(according to \ref{pI:Habsburg Milano}), \HAB may spend one Diplomatic action
to automatically gain \paysSuisse in \CE (no money is spent).

\phmil
\aparag[Restricted War Field]
\bparag The war is restricted to \regionItalie if no side broadens the zone of
war.
\bparag The war is no more restricted if the side of \FRA invades a province
outside \regionItalie of the other side. \FRA loses immediately {\bf 1} \STAB
and 20 \VP and if the invasion was not due to \FRA, the Major Power
responsible for it loses also {\bf 1} \STAB and 20 \VP. However, if dynastic
actions \shortdynasticaction{A}{1} and \shortdynasticaction{A}{2} have both
been played, the penalty in \VP is void.
\bparag The war is also no more restricted if the side enemy of \FRA invades a
province of the side of \FRA outside \regionItalie and that stack does not
draw its supply from \regionItalie.
\aparag \paysSavoie gives free access and supply in its province to \FRA
during the first turn of the war, if it stays neutral in this war. Supply from
or across a province is impossible if its city is under siege by an enemy of
this city.
\aparag If \ref{pI:Habsburg Milano} was not played and \FRA besieges the city
of \provinceLombardia with at least one \ARMY\faceplus, it takes the city
without resolving the siege and annexes immediately the province; \FRA may
destroy the Minor country \paysMilan by this way.

\effetlong
\aparag[Passing through \paysSavoie]
\bparag At the instant \FRA annexes \provinceLombardia during the war, it
gains from \paysSavoie free access and supply through its provinces (but no
stopping in, or supply from) when at peace with \FRA. Supply across a province
is impossible if its city is under siege by an enemy of this city.
\bparag This right is void if/when \FRA is at war against \paysSavoie, and is
permanently lost if \FRA loses \provinceLombardia.
\bparag Enemies of \FRA gain the same right when at war with \FRA.
\aparag At the end of this event, if the Minor country \paysMilan still
exists, \dynasticaction{B}{2} is played then \HAB annexe all its provinces and
the minor country disappears.
\aparag Until the end of the current period, \FRA has a \CB against the owner
of \provinceLombardia.



\event{pI:Hungarian Freedom}{I-4 (1)}{Declaration of Hungarian
  Freedom}{1}{RistoMod}

\history{1505}

\condition{If \ref{pI:Habsburg Hungary} or \ref{pI:Fall Hungary} has already
  been activated, mark off but play \RD.}

\phevnt
\aparag The Hungarian Inheritance (\ref{pI:Habsburg Hungary}) that might be
pending is now impossible.
\aparag \POL has the immediate choice of supporting a Jagiellon dynasty in
\paysHongrie. If it does, it gains \paysHongrie in \MR at once, makes a white
peace with it if necessary, and gains a temporary \CB against any countries at
war against \paysHongrie.
\bparag Else, \paysHongrie becomes \Neutral.
\aparag \HAB has a temporary \CB against \paysHongrie. See also \ref{pI:Fall
  Hungary} that might happen.

\effetlong The \dynasticaction{C}{1}, the events \xref{pI:Hungarian Alliance}
and \xref{pI:Habsburg Hungary} are no more possible and will be ignored.



\event{pI:Bohemian Alliance}{I-4 (2)}{Dynastic Alliance with
  \paysBoheme}{1}{RistoMod}

\history{1526}

\condition{If \ref{pI:Habsburg Bohemia} has already been played, mark off and
  play \RD.}

\phevnt
\aparag The \dynasticaction{B}{1} is played, and it activates \ref{pI:Habsburg
  Bohemia}.



\event{pI:Hungarian Alliance}{I-5}{Dynastic Alliance with
  \payshongrie}{1}{RistoMod}

\history{1491, not activated}
\dure{until the activation of \ref{pI:Habsburg Hungary} or \ref{pI:Habsburg
    Bohemia}, or the \ref{pI:Hungarian Freedom}}

\condition{If \shortdynasticaction{C}{1} or \ref{pI:Hungarian Freedom} has
  already been played, mark off and play \RD.}

\phevnt
\aparag The \dynasticaction{C}{1} is played, and consequently \ref{pI:Habsburg
  Hungary} is pending.
\aparag \POL gains a temporary \CB against \paysHongrie.

\phdipl
\aparag At the beginning of each diplomatic phase, the diplomatic status of
\paysHongrie moves one level toward the track of \HAB, up to \EG. This ends if
\ref{pI:Fall Hungary}, \ref{pI:Hungarian Freedom} or \ref{pI:Habsburg Hungary}
happens.



\event{pI:Milanese Alliance}{I-6}{Dynastic Alliance with Milano}{1}{Risto}

\begin{todo}
  Remove (happens to early). Replace with something else. Maybe something for
  \DAN? In the meantime, mark off and reroll.
\end{todo}

\condition{If \shortdynasticaction{C}{1} or \ref{pI:Hungarian Freedom} has
  already been played, mark off and re-roll.}

\aparag The \dynasticaction{B}{2} is played, and it activates \ref{pI:Habsburg
  Milano}.



\event{pI:Habsburg Dynasty}{I-7 (1)}{Habsburg Dynastic Action}{2}{PBNew}

\phevnt
\aparag \SPA may immediately play one dynastic action of its choice, without
test nor cost.
\bparag This action may be an annexation of one of the Provinces of the
North-East, if applicable.



\event{pI:Revolt Comuneros}{I-7 (2)}{Revolt of the Comuneros}{1}{PBNew}

\history{1520-1522}

\phevnt
\aparag Place one \REVOLT\facemoins in \provinceToledo, one Rebel
\ARMY\facemoins, \LD with a minor \LeaderG. The rebels control the fortress
(reduced to level 2 max if need be).
\aparag Draw at random 2 other provinces where a \REVOLT\facemoins is placed,
by rolling 1d10: 1-2 \province{La Mancha}, 3-4 \province{Castilla La Nueva},
5-6 \provinceSalamanca, 7-8 \provinceLeon, 9-10 \province{Castilla La Vieja}
\aparag The Rebels are controlled by \RUS (the most remote player designers
could think of).  They will receive no reinforcement (excepted through \REVOLT
extension).



\event{pI:Reformation}{I-8 (1)}{Reformation}{1}{RistoMod}

\history{1517-1560}

\tour{Turn 1}

\phevnt
\aparag[Luther's 95 Thesis] \paysDanemark, \paysSuede, \paysBerg, \paysSuisse,
\paysHanse, \paysprovincesne, \paysHesse, \paysSaxe, \paysHanovre,
\paysOldenburg, \paysBrunswick, and \paysBoheme become Protestant.
\aparag \terme{Religious enmities} begin between Protestant and Catholic
countries. They will end when \ref{pIV:TYW} is terminated, or at the beginning
of period IV if this event ended before, or at the end of period IV if the
event is not yet finished.
\aparag[Orthodoxes in Poland] \POL has to decide of its attitude regarding
Orthodoxy: Conversion, Tolerance or Support.
\bparag The lasting effects are mainly described in
\ruleref{chSpecific:Poland:Orthodoxy}.
\bparag If \POL chooses Support of Orthodoxes, it loses {\bf 2} \STAB and
rolls for 2 \REVOLT on its table.
\bparag If \POL chooses Tolerance of Orthodoxes, it loses {\bf 1} \STAB and
rolls for 1 \REVOLT on its table.
\aparag[Russian Religious Attitude] \RUS has to decide its behaviour regarding
Religions: Championship of Orthodoxy or Religious Tolerance.
\bparag The lasting effects are mainly described in
\ruleref{chSpecific:Russia:Orthodoxy}.
\bparag If \RUS chooses Religious Tolerance, it loses {\bf 2} \STAB and rolls
for 1 \REVOLT on its table.

\tour{Turn 2}

\phevnt
\aparag \paysBrandebourg becomes Protestant. Play \ref{pI:Fall Teutonic} as a
supplementary event this turn.



\event{pI:Reformation2}{I-8 (2)}{Growth of the Reformation}{1}{RistoMod}

\history{1517-1560}

\phevnt
\aparag \FRA, \SPA, \ENG and \POL must choose between \CATHCR, \CATHCO or
Protestantism (forbidden to \SPA). The choice is made simultaneously and
secretly at the beginning of the Phase of Declarations. It cannot be
voluntarily changed later except by events. If \POL has chosen Support of
Orthodoxes, he is complied to choose \CATHCO now.

\phase{Consequence:}{Each country is affected by the following general
  consequences, added to specific effects for each country, described
  afterwards.}
\aparag[\CATHCR]
\bparag If only one of the eligible players chooses \CATHCR, he is permanent
\SDCF and receives 20 \VP.
\bparag If several players choose \CATHCR, the \SDCF is determined according
normal procedure but between them only.
\bparag If none of the eligible players chooses \CATHCR, all of them lose {\bf
  1} additional \STAB.
\bparag A bonus of \bonus{+1} is received for diplomacy on all Catholic
countries until the end of \terme{Religious Enmities}.
\aparag[\CATHCO]
\bparag {\bf 1} \STAB is lost.
\bparag One \REVOLT is rolled in the player country.
\bparag An additional Diplomatic Action is gained and a \bonus{+2} bonus is
received for diplomacy on all Protestant countries until the end of
\terme{Religious Enmities}.
\aparag[Protestantism]
\bparag No Diplomacy (support included) with \payspapaute until the end of the
current period. Control of \payspapaute is lost.
\bparag {\bf 2} \STAB are lost.
\bparag Two \REVOLT are rolled in the player country.

\begin{digressions}[Specific effects]


  \digression[pI:Reformation:France]{\sc France}
  \aparag[Independent \paysVhollande] If \paysVhollande is or comes into play
  before the \xnameref{pV:WoSS}, immediately apply \ref{pIII:Dutch Revolt}.
  \aparag[\CATHCR]
  \bparag Some events (especially \xnameref{pIII:FWR}, \xnameref{pV:Expulsion
    French Protestants}) are modified.
  \aparag[\CATHCO]
  \bparag \bonus{+1} bonus to \STAB improvement attempts this turn and the two
  following ones.
  \aparag[Protestantism]
  \bparag No Diplomacy (support included) with \payspapaute until the end of
  period III.
  \bparag Some events (\xnameref{pI:War Italy Napoli}, \xnameref{pI:War Italy
    Milano}, \xnameref{pII:War Italy}, \xnameref{pIV:La Rochelle},
  \xnameref{pIII:FWR} \xnameref{pV:Expulsion French Protestants},
  \xnameref{pV:Colbertian Mercantilism}) are modified.
  \bparag The turn and period limits of \FRA are changed. \FRA receives an
  explorer for one turn as per~\ref{eco:Explorer}.


  \digression[pI:Reformation:Spain]{\sc Spain}

  \aparag[\CATHCR]
  \bparag Permanent bonus \bonus{+2} for diplomacy on \payspapaute.
  \bparag \SPA gains the possibility of forcing Restoration of Catholicism in
  Protestant countries, with the relevant bonuses.
  \aparag[\CATHCO]
  \bparag A further {\bf -1} in \STAB is applied.
  \bparag A malus of \bonus{-2} to \STAB improvement attempts for the rest of
  the period and the following one
  \bparag Restoration of Catholicism in Protestant countries gives no bonuses.
  \bparag Dynastic actions are no more allowed, except when permitted or
  required by an event.
  \begin{designnote} Future option: modifications of some events: [temporary
    list II-9, III-1, III-7, III-8, III-11, IV-1 and V-8]. As this choice
    might largely change the course of the game, especially for the player of
    \VEN, it is good policy to have part of an agreement with this player
    before choosing this attitude.
  \end{designnote}


  \digression[pI:Reformation:England]{\sc England}

  \aparag[\CATHCR]
  \bparag The turn and period limits of \ENG are changed.
  \aparag[\CATHCO]
  \bparag \bonus{+1} bonus to \STAB improvement attempts this period and the
  following one.
  \aparag[Protestantism]
  \bparag \ANG is automatically \PROTPUR.
  \bparag The turn and period limits of \ENG are changed. \ENG receives an
  explorer for one turn as per~\ref{eco:Explorer}.
  \bparag Each time \ENG is rolled-for in the Revolt Country chart, the number
  of \REVOLT is doubled. This continues until the end of period III.
  \aparag The Religious and Civil Wars of \ENG (\xnameref{pII:Act Supremacy},
  \xnameref{pIV:English Civil War}, \xnameref{pV:Glorious Revolution} and
  \xnameref{pVI:Jacobite Rebellion}) depend on its Religious choice.


  \digression[pI:Reformation:Poland]{\sc Poland}

  \aparag[\CATHCR]
  \bparag Some events (\xnameref{pI:Fall Teutonic}, \xnameref{pIII:Union
    Poland Sweden}, \xnameref{pIV:TYW}, \xnameref{pV:Saxon King Poland}) are
  modified.
  \bparag \POL gain a \CB against all Protestant countries until the end of
  period III, and the right to convert them to Catholicism.
  \aparag[\CATHCO]
  \bparag \bonus{+1} bonus to \STAB improvement attempts this turn and the two
  following ones.
  \aparag[Protestantism]
  \bparag The Union of Lublin (see \xnameref{pII:Union Lublin} or
  \xnameref{pIII:Union Lublin}) is broken and will not be possible. Some other
  events (\xnameref{pIV:Bohemian Revolt}) are modified.
  \bparag The turn and period limits of \POL are changed.
\end{digressions}



\event{pI:Reformation3}{I-8 (3)}{Intensification of the
  Reformation}{1}{RistoMod}

\history{1517-1560}

\phevnt
\aparag[Calvin] \paysPalatinat, \paysThuringe and \paysecosse become
Protestant.



% \event{pI:Turkish Diplomatic Pressure}{I-9}{Turkish Diplomatic
%   Pressure}{2}{Risto}

% \history{no specific date}

% \condition{If \leaderBarbaros is in play, \TUR may choose to apply
%   \ref{pII:Algeria Vassalisation} instead.  If \leaderDragut is in play, \TUR
%   may choose to apply \ref{pII:Alignment of Barbaresques} instead.}

% \phdipl
% \aparag \TUR receives a bonus of \bonus{+3} for a Muslim minor of its
% choice. Choice has to be made secretly during the negotiations step.

\event{pI:Turkish Diplomatic Pressure}{I-9}{Turkish
  Dynamism}{*}{RistoMod/PBnew/Jym [BLP]}
% \history{Capture of Algiers: 1516, Recapture of Algiers: 1529,
%   Conquest of Tunis: 1534.}

\phevnt
\aparag \TUR chooses, when all events of this turn have been rolled, to apply
one of the following cases:
\bparag If \leader{Oruc Reis} is alive, \TUR may choose~\ref{pI:TD:Barbaross
  brothers}. This may only occur once per game.
\bparag If \leader{Barbaros2} is alive, \TUR may
choose~\ref{pI:TD:Vassalisation of Algeria}. This may only occur once per
game.
\bparag If This is period \period{II} or later, \TUR may
choose~\ref{pI:TD:Alignment Barbaresques}. This may only occur once per game.
\bparag \TUR may always choose~\ref{pI:TD:Diplomatic pressures}. This may
happens any number of time.

\subevent[pI:TD:Barbaross brothers]{Barbaross brothers}
\history{Capture of Algiers by Aruj and Hayreddin Barbarossa: 1516}
\phevnt
\aparag \TUR immediately chooses one \Presidio in \paysAlgerie which is
destroyed.

\aparag If not controlled by \TUR, \paysAlgerie becomes immediately Neutral.

\aparag On this turn, the Algerian \corsaire is raised \faceplus (even if it
was not in play).

\subevent[pI:TD:Vassalisation of Algeria]{Vassalisation of Algeria}
\history{Recapture of Algiers by Hayreddin Barbarossa, and formal sovereignty
  of Soliman: 1529}
\phevnt
\aparag \paysAlgerie is immediately placed on \VASSAL of \TUR.

\aparag \leaderBarbaros is now also a Turkish leader, and as long as he is
alive, \paysAlgerie is permanent Vassal of \TUR not subject to diplomacy.

\aparag At the death of \leaderBarbaros, the {\bf -3} malus for \TUR to all
diplomacy attempts against all \terme{Barbaresque} countries is cancelled.


\subevent[pI:TD:Alignment Barbaresques]{Alignment of the Barbaresques}
\history{Ottoman conquest of Tunis: 1534, alignment: around 1540}
\phevnt
\aparag From now on, the {\bf -3} malus for \TUR to all diplomacy attempts
against all \terme{Barbaresque} countries is cancelled.

\aparag \paysTunisie is immediately placed on \VASSAL of \TUR if \leaderDragut
is alive.

\subevent[pI:TD:Diplomatic pressures]{Turkish Diplomatic Pressures}
\history{No precise date}
\phdipl
\aparag \TUR receives a bonus of \bonus{+3} for a Muslim minor of its
choice. Choice has to be made secretly during the negotiations step.


\event{pI:War Scotland}{I-10}{War with Scotland}{1}{Risto}
\history{1513-1514}

\condition{}
\aparag Occurs only if \paysecosse is at present inactive. Otherwise re-roll.
\aparag \ENG can refuse this event (mark as played) by losing {\bf 2} \STAB
and 20 \VP. It also loses the control of \paysecosse and can then make no
diplomacy on it until the end of period.

\phevnt
\aparag \paysecosse declares war against \ENG, which loses the control of
\paysecosse.
\aparag Allies can be called for this war as per normal rules.
\aparag Control of \paysecosse is offered to the first country in the list:
\bparag Any current enemy of \ENG (follow the normal preferences to decide
which).
\bparag The current controller of \paysecosse or, failing that, another power,
according to the usual rules.

\phadm
\aparag For the duration of the event, \paysecosse receives reinforcements in
offensive attitude.



\event{pI:End Golden Horde}{I-11 (1)}{The End of the Golden Horde}{1}{PB}

\history{1502}

\condition{}
\aparag If \paysCrimee exist no more, mark off and play \RD instead.

\phevnt
\aparag \paysCrimee declares war to \paysSteppes. The war is not played.
\aparag Both countries make mandatory White Peaces in existing wars.

\phdipl
\aparag Diplomacy, Call to Allies or Limited intervention is forbidden for
these two countries for the duration of the turn, and neither exterior
involvement in this war is allowed.

\phpaix
\aparag The Khanate of the Golden Horde is defeated by \paysCrimee at the end
of turn. From now on, the minor country \paysSteppes has reduced military
forces and stop helping other Khanates when attacked.



\event{pI:Pskov Ryazan}{I-11 (2)}{Russian Annexation of Pskov and
  Ryazan}{1}{PB}

\history{1510 and 1517}

\phevnt
\aparag The provinces \provincePskov and \provinceRyazan become Russian
National provinces.
\aparag \RUS can annex immediately one the two countries \paysPskov or
\paysRyazan of its choice.
\aparag A \MAJ having the annexed country on its track has a \CB against \RUS
at this turn.
\aparag \POL has a \CB against \RUS at this turn.



\event{pI:War Russia Poland}{I-12}{War between Russia and Poland}{1}{PB}

\history{1507-1522 / 1534-1537}

\condition{If \RUS and \POL are already at war against each other, mark off
  the case and play \RD instead.}

\phevnt
\aparag \RUS has a temporary free \CB against \POL and \POL has a temporary
free \CB against \RUS. Those \CB may be used this turn or the following
turn. If no power uses it, both lose 1 \STAB on the second turn.



\event{pI:War Roads Spices}{I-13}{Wars on the Roads of Spices}{2}{PBMod}

\history{1508-09/non historic}

\condition{}
\aparag If there is a \TP/\COL producing a \POSPICE belonging to any European
country, apply the \ref{pI:WRS:War Indian Sea}. It can happen only once.
\aparag Otherwise, apply \ref{pI:WRS:Veneto-Turkish Commercial Dispute}.  This
event can also happen only once.
\aparag If the second event happened and the first is not possible, do not
mark off and re-roll.


\subevent[pI:WRS:War Indian Sea]{War in Indian Sea}

\phevnt
\aparag \paysEgypte and \paysGujerat allies themselves.  They declare an
overseas war to any European country having a \TP/\COL in \continentAfrica
north-east of \granderegionNatal (included) or \continentAsia west of
\granderegionMalaisie and \granderegionSumatra (both included). They naturally
break diplomatic relations with countries they declare war to.
\bparag If \paysEgypte exists no more, \TUR gains an \dipAT with \paysGujerat.
\aparag A Major country having \terme{Treaty} with the \paysGujerat or any
diplomatic status with \paysEgypte has an oversea \CB at this turn against all
the countries aimed by the event (all at once).

\phdipl
\aparag From now on, \VEN can make diplomacy to \paysAden, \paysOman and
\paysGujerat, even through it does not know adjacent sea zones or have
\TP/\COL adjacent. However if \VEN is at war against the owner of the
\CCs{Grand Orient} it can make no diplomacy on these countries, and any
\terme{Treaty} it might have is inactive during the war; still it can resist
diplomatic attempts from other Major powers.

\phadm
\aparag \paysEgypte gains the discoveries of all seas from \seazoneErythree to
\seazoneMalaisie, bordering coastal zones (and \seazoneIndien excepted). From
now on, \paysEgypte has only one \ARMY counter, but has also one \FLEET
counter (but no navy in basic forces) and can use all its detachments as \LD
or \ND, and gains 2 counters \LDENDE. Its basic forces are changed.
\bparag If \paysEgypte exists no more, \TUR gains the discoveries of
\seazoneMascate and \seazoneKutch (only).
\aparag In the first turn of war induced by the event \paysEgypte chooses
Naval reinforcement.
\aparag On the first turn of war caused by the event, \paysGujerat raises an
additional \FLEET\facemoins (even if it is beyond its basic forces; it keeps
these warships until the end of the war).

\phpaix
\aparag At the end of the first turn of the war (only the first), the two
minor countries do not automatically accept a White Peace as usual in Overseas
Wars.  A formal peace has to be obtained.


\subevent[pI:WRS:Veneto-Turkish Commercial Dispute]{Veneto-Turkish Commercial
  Dispute}

\phevnt
\aparag As long as the \CCs{Grand Orient} is in \paysEgypte, \TUR can not, by
any means, receive part of its income.

\phdipl
\aparag \TUR gains a temporary free \OCB against \VEN.
\aparag \TUR gains a \CB vs \paysSyria and \paysMamelouks.
\aparag At any following turn, \VEN can nullify the event by announcing it at
the beginning of the Declaration phases.  \VEN loses {\bf 1 } \STAB and \TUR
regains rights to part of the income of \CCs{Grand Orient} if it controls
\paysdamas. \TUR loses the \CB given by the event whereas \VEN gains a \CB
against \TUR, valid once before the end of the current period.
\aparag If \TUR makes a winning peace of level 2 or more against \VEN in a war
(oversea or regular), it can ask for its right on the \CCs{Grand Orient}
instead of one peace condition.



\event{pI:Resistance Muslim Traders}{I-14}{Resistance of Muslim
  Traders}{1}{PBNew}

\history{Non historic}

\condition{}
\aparag If the country \paysGujerat is destroyed, all European \TP in
\granderegionGujarat, \granderegionMalacca, \granderegionSumatra,
\granderegionJava, and \granderegion{Iles aux epices} will be attacked by
Natives during the turn.
\aparag If the country \paysGujerat still exists, use the following events.

\phevnt
\aparag All undestroyed \TP of \paysGujerat regain their initial level.  All
European \TP in the same Region will suffer a \CONC attempt at this turn from
\paysGujerat (Medium Investment).
\aparag In all provinces were \TP of \paysGujerat have been destroyed before,
European \TP will be attacked by Natives during the turn.



\event{pI:Chinese Expeditions}{I-15}{Chinese Expeditions}{1}{PBNew}

\history{Abandoned before 1492}

\phevnt
\aparag \paysChine gains three \TP of level {\bf 3} in the following
provinces: \provinceCalicut, \province{Malacca S}, \provinceMadras, replacing
existing \TP from \paysGujerat, and 2 \DT on each \TP.
\aparag However, if an European country has already discovered a sea zone
adjacent to the postulated position of those \TP, the Chinese \TP is not
placed here but in one province (determined randomly among those free of
\TP/\COL) in the following Regions (in this order, 1 by region):
\granderegionJava, \granderegionCelebes, \granderegionSumatra,
\granderegion{Iles aux epices} (if there is not enough unoccupied provinces in
those, the remaining \TP are lost).  Those \TP only have one \DT and level 1.
\aparag The Chinese \TP take the exploitation of resources (\RES{Products of
  Orient} first then \RES{Spices}) without concurrence; a Major Power will
have to make proper \CONC to take them back.
\aparag From now on, \paysChine has increased basic forces.  Added to the 2
\ARMY\faceplus in mainland \paysChine, it has garrisons of 1 \LD per \TP (or 2
\LD if they remain from the event), one \FLEET\faceplus and one Admiral (use
one from the minor pool, with the added capacity to go in the \ROTW) that can
move freely in the \ROTW when at war. Its reinforcements are one
\ARMY\faceplus in mainland, and a \LD, a \ND for the garrisons.
\aparag \paysChine has a \FTI of 2. The Chinese \TradeFLEET in \stz{Chine}
is increased to level 4.
\aparag \paysChine is considered to have discovered all land regions of
\continentAsia (including islands but \granderegionOceania and
\granderegionPacifique excepted) and those of \continentAfrica north and east
of \granderegionNatal included. It also has discovered all sea zones bordering
those territories.



\event{pI:Barbaros Brothers}{I-16}{Barbaros Brothers}{1}{PBNew}
\condition{[BLP] Apply~\ref{pI:Turkish Diplomatic Pressure}}

% \history{1516}

% \condition{Takes place only if leader \leaderBarbaros is not yet in play or no
%   more alive.  If \leaderBarbaros is in play, apply \xnameref{pII:Algeria
%     Vassalisation}}

% \phevnt
% \aparag If not controlled by Turkish, \paysAlgerie becomes immediately in
% Neutral.
% \aparag On this turn, the Algerian \corsaire is raised \faceplus and is
% supposed to have an admiral of Manoeuvre 3 leading it (another Brother).



\event{pI:Habsburg Alliance}{I-A}{Dynastic Alliance of the Habsburg}{1}{PB}

\history[Philip the Handsome, Habsburg heir, marries Juana the Mad, heiress of
Spain.]{1496}

\activation{Activated by \dynasticaction{A}{1}}

\phevnt
\aparag \SPA and \HAB are now allied in a specific way as described
in~\ruleref{chSpecific:Habsburg Dynastic Alliance}.  The diplomatic counter of
\HAB is placed in \EG of \SPA.
\aparag \SPA has now the right to annex the \paysprovincesne through war (it
has a \CB for such a war) or diplomatic actions.
\aparag \SPA has a temporary \CB at this turn or the following against any
country possessing any province that was part of \paysBourgogne in 1492.

\effetlong
\aparag[The Habsburg] The special alliance is now enforced between \SPA and
\HAB as per \ruleref{chSpecific:Habsburg Dynastic Alliance}, until broken by
\ref{pV:WoSS}.



\event{pI:Burgundy Inheritance}{I-B}{Burgundy Inheritance}{1}{PB}

\history[Spain takes full political control of Burgundian heirdom.]{1506}

\activation{Activated by \dynasticaction{A}{2}}

\phevnt
\aparag \SPA annexes all provinces of \paysBourgogne and this country exists
no more.  \SPA has a \CB (this turn and the following one) against any country
possessing any province owned by \paysBourgogne in 1492.
\aparag A Spanish \MNU of \RES{Cloth} with 2 levels is set in
\provinceVlaandern.
% \aparag At the instant where \paysHollande exists (by \ref{pI:Habsburg
%   Alliance}), \provinceZeeland is transferred from \SPA to \paysHollande.
\aparag If \provinceZeeland is still owned by \pays{provincesne}, it is
immediately annexed by \HIS with no \VPs gained.
\aparag When \ref{pI:Habsburg Alliance} has been played as well as the current
event, \paysLiege can now be \VASSAL or in \ANNEXION of the owner of Spanish
Flanders, \SPA now (and possibly \FRA, \ENG or \AUS later).

\effetlong
\aparag[Holland before its revolt]
\bparag The minor country \paysHollande is created by this event.  It will
consist of all provinces of \paysprovincesne that \SPA has gained, and
\ref{pI:Burgundy Inheritance} gives additional provinces from \paysBourgogne,
that is all national provinces of \paysHollande. This minor country is
permanent \VASSAL of \SPA, not subject to diplomacy, until it revolts by
\ref{pIII:Dutch Revolt}. It has no military forces, and any war against it has
to be declared as a war against \SPA. \SPA can not raise forces in
\paysHollande.
\bparag The commercial system of \paysHollande contributes to \SPA: its
\TradeFLEET are added to those of \SPA in order to find who has the different
\CC.
\bparag \SPA does not receive income for the provinces of \paysHollande.
Instead, it can impose a \terme{Tax} on \paysHollande that amounts to
40\ducats plus 10 \ducats for each province in \paysHollande.
\bparag Event \xref{pIII:Dutch Revolt} will free \paysHollande and change the
previous rules. Each turn of \terme{Taxes} will liken the Revolt.



\event{pI:Habsburg Bohemia}{I-C}{Habsburg Bohemia}{1}{PB}

\history{1526}

\activation{Activated by \dynasticaction{B}{1}, or by events~\xref{pI:Bohemian
    Alliance} or \xref{pI:Habsburg Hungary}}

\phevnt
\aparag \HAB annexes all provinces of \paysBoheme and this country exists no
more.  The power that has \paysBoheme on its diplomatic track has a temporary
\CB against \HAB.
\aparag \HAB has a free \CB (this turn and the following one) against any
country possessing any province owned by \paysBoheme in 1492; \SPA decides if
\HAB uses it or not.
\aparag If \paysBoheme was at war, \HAB is substituted to this country for the
on-going war. \HAB offers its enemies the immediate possibility to sign a
White Peace.

\effetlong
\aparag \paysBoheme may reappear as a ``liege'' country of \HAB or \SPA (see
\ruleref{chSpecific:Spain:Autonomous States}) or by means of \ref{pIV:Bohemian
  Revolt}.



\event{pI:Habsburg Hungary}{I-D}{Habsburg Inheritance of Hungary}{1}{PB}

\history{Never activated}

\activation{The first \RD event beginning with turn 8 activates this Event
  instead of its normal effect if either \ref{pI:Hungarian Alliance} or
  \dynasticaction{C}{1} has been played, and \ref{pI:Hungarian Freedom} has
  not.}

\condition{}
\aparag Play the \ref{pI:Habsburg Bohemia} if was not already played.
\aparag If \paysHongrie exists no more, ignore the rest of the event.

\phevnt
\aparag All provinces of \paysHongrie are annexed by \HAB and the country is
destroyed.
\aparag If \paysHongrie was at war, \HAB is substituted to this country for
the on-going war. \HAB offers its enemies the immediate possibility to sign a
White Peace.
\aparag Event \xref{pI:Habsburg Bohemia} is activated at this turn.

\effetlong
\aparag The basic forces of \HAB are increased by an \ARMY\faceplus.
\aparag \paysHongrie may reappear as a ``liege'' country of \HAB or \SPA (see
\ruleref{chSpecific:Spain:Autonomous States}).
\aparag If \HAB controls at least 5 provinces of \paysHongrie, it may use the
counters of \paysHongrie.
\aparag All future Hungarian leaders are now given to \HAB.
\aparag If \TUR annexes \villeBuda before the end of period II, lasting
effects of \ref{pI:Fall Hungary} are applied instead, and this event is
supposed to have happened for the rest of the rules (Victory conditions and so
on).



\event{pI:Fall Hungary}{I-E}{Downfall of Hungary}{1}{PB/Jym [BLP]}

\history{1526}

\activation{}
\aparag Activated by~\ref{chSpecific:Hungary} on the turn following either
\begin{modlist}
\item a major victory of \TUR against a stack with a least one \ARMY counter
  of \paysHongrie, if \TUR chooses to activate it;
\item[OR] occupation of \villeBuda by \TUR;
\item[OR] Turkish control of at least 5 provinces owned by \paysHongrie.
\end{modlist}
\bparag The moment the condition is met, \POL can make a limited intervention
at the side of \paysHongrie and \HAB may make a limited intervention or enter
war at the side of \paysHongrie. These are not declarations of war, no \STAB
is lost and no reinforcements are rolled.
\bparag Once the condition is met, \TUR may not sign peace with \paysHongrie
this turn.
\bparag On the next turn, this event is considered to be the first event
rolled.

% \bparag After the collapse and for the remaining of the turn, \HAB and/or \POL
% can make a limited intervention as an ally of collapsing \paysHongrie, with no
% declaration of war and no cost in \STAB. This intervention can be made with
% all the available forces of the power at that time (no reinforcements),
% without the usual limit for limited intervention.
% \bparag If the collapse begins on the last round of a turn, or would begin on
% the next turn, the resolution of the event is postponed to the end of the next
% turn.
% [BLP] removed
% \aparag If \ref{pI:Hungarian Freedom} has been played, military control of
% \villeBuda by \HAB in a war against \paysHongrie activates the same event,
% during any Period.
% \bparag \TUR may then make a limited intervention per the same rules as above.

\phpaix
\aparag Note that this happens the turn the event is resolved, \emph{i.e.} one
turn after \TUR causes the Downfall. Thus, there is always at least one full
turn during witch \HAB and \POL may try and defend \paysHongrie.

\aparag \paysHongrie is destroyed. Its remaining provinces are given as
follows:
\bparag \provincePecs, \provinceCroatie, \provinceHongrie, \provinceKarpatok,
\provinceBukovina are annexed by whoever controls militarily the province
among \TUR, \HAB and \POL (the presence of stack with \ARMY\faceplus in a
province with fortresses of an allied collapsing \paysHongrie gives control to
the leader of this stack). Those controlled by \paysHongrie at the end are
annexed by \HAB. (These provinces have no extra shield)
\bparag \provinceSzlovakia, \provinceBalaton, \provinceKranj and
\provinceKapela are annexed by \HAB (and nobody gains the \VP). (These
provinces have a blurred Austrian shield reminder)
\bparag \provinceBanat, \provinceSerbia and \provinceBosnia (if owned by
\paysHongrie or Neutral) are annexed by \TUR. (These provinces have a blurred
Turkish shield reminder)
\bparag A minor country \paysTransylvanie is created, composed from the
remaining provinces of \paysHongrie: likely, \provinceErdely and
\provinceMures (These provinces have a blurred Transylvanian shield) plus any
province that \paysHongrie may have annexed since the beginning of the
game. This country is created as a special \VASSAL of whoever got
\provinceHongrie during the partition.
\bparag Excepted for some provinces annexed by \HAB, the usual \VP are given.

\aparag If a power controls provinces given to another power, it may declare
now a war with a \CB, or its troops withdraw (as per peace evacuation).

\aparag[\paysTransylvanie] [BLP] For the rest of the game, \paysTransylvanie
is a special \VASSAL of the owner of \provinceHongrie.
\bparag As soon as this province changes owner, the new owner immediately
becomes the Diplomatic patron of \paysTransylvanie.
\bparag No diplomacy is allowed on \paysTransylvanie. It is not subject to
Diplomatic events.

\aparag The limited interventions of \HAB and \POL (if any) end immediately.
\bparag However, if \HAB chose to enter war, a formal peace treaty must be
obtained at this turn or another one, as usual.

% \aparag The basic forces of \HAB are increased by an \ARMY\facemoins. All
% future Hungarian leaders are now given to \HAB.

% \aparag Whenever the \HAB annexes \provinceHongrie, the counters of
% \paysHongrie (2 \ARMY and 4 \DT) are permanently added to those available to
% \HAB.

% \aparag \paysHongrie may reappear as a ``liege'' country of \HAB (but not
% \SPA, see \ruleref{chSpecific:Spain:Autonomous States}). %or by means of
% event \ref{pIII:War HRE}.

\effetlong
\aparag The basic forces of \HAB are increased by an \ARMY\facemoins.
\aparag \paysHongrie may reappear as a ``liege'' country of \HAB or \SPA (see
\ruleref{chSpecific:Spain:Autonomous States}). %or by means of event
% \ref{pIII:War HRE}.
\aparag If \HAB controls at least 7 provinces of \paysHongrie, it may use the
counters of \paysHongrie.
\aparag All future Hungarian leaders are now given to \HAB.

\aparag[] [BLP] \ref{chSpecific:Little war} is now active.


\event{pI:Habsburg Milano}{I-F}{Habsburg Control of Milano}{1}{RistoMod}

\history{around 1520}

\activation{Activated by \ref{pI:Habsburg Dynasty} or \dynasticaction{B}{2}}
\aparag If \paysMilan is now at war, \HAB have a free \CB to join the war on
the side of \paysMilan. The rest of the event is activated when the \CB is
used.
\aparag If \HAB is not allied yet to \SPA, it uses the \CB of the event as
soon as it is not active elsewhere.

\phevnt
\aparag \paysMilan becomes a permanent \VASSAL of \HAB. \paysMilan and \HAB
are from now on one entity for wars and peaces.
\aparag If the province \provinceLombardia is french, a \REVOLT \facemoins is
placed herein and \HAB have a free \CB this or the following turn against
\FRA.

\effetlong
\aparag \dynasticaction{C}{2} is now possible.



\event{pI:Spanish Milano}{I-G}{Spanish Milano}{1}{RistoMod}

\history{around 1560}

\activation{Activated by \dynasticaction{C}{2}}

\phevnt
\aparag \SPA annexes \provinceLombardia if this province is in \paysMilan
(whether a permanent \VASSAL of \HAB or not) or owned by \HAB. The minor
country \paysMilan exists no more.
\aparag If \provinceLombardia is owned by another country, a \REVOLT
\facemoins is placed herein and \SPA and \HAB have free \CB this or the
following turn against this country.

\effetlong
\aparag \SPA can now raise troops in \provinceLombardia if it controls it,
with normal cost.
\aparag \SPA can recreate \paysMilan as a ``liege'' country (see
\ruleref{chSpecific:Spain:Autonomous States}).



\event{pI:Fall Teutonic}{I-H}{Secularisation of \pays{Teutoniques2}}{1}{PB}

\history{1525}

\activation{Activated by \ref{pI:Reformation2} or \ref{pIII:Northern
    Secularisation}, whichever occurs first}

\phevnt
\aparag \pays{Teutoniques2}, part of minor country \pays{Teutoniques1} become
Protestant.  All units from any country in \provincePreussen, \province{Ost
  Pommern} and \province{West Pommern} have to retreat when those provinces
are annexed by another country.
\aparag The province \provincePreussen become part of \region{Duche de
  Prusse}.
\bparag If \POL is \CATHCO, it annexes the province if owned by
\pays{Teutoniques1}, or has a \CB against its owner until the end of the
Period.
\bparag Else, \region{Duche de Prusse} is annexed by \paysBrandebourg, and
\provincePreussen become part of \paysBrandebourg. If this province is owned
by any other country than \pays{Teutoniques1}, this country has a \CB against
\paysBrandebourg.
\aparag The provinces \province{Ost Pommern} and \province{West Pommern} are
annexed by \paysHanse if owned by \pays{Teutoniques1}.
\bparag If one or the two provinces are owned by any country except \POL,
\paysHanse declares war to this country. \POL may (his choice) have \paysHanse
placed in \AM before usual calls for help is made, in which case \paysHanse
calls it to his help. Else, usual rules are used.
\bparag Else, if one or the two provinces are owned by \POL, \paysHanse
declares war to \POL and \paysBrandebourg too, allied with \paysHanse.
\bparag If a war results of this event, only \paysHanse can annex the 2
provinces.
\aparag The minor country \pays{Teutoniques1} (now Livonian Brothers of the
sword) loses one \ARMY counter, and its basic forces are diminished by one
\ARMY\faceplus.

\phpaix
\aparag If a war is prosecuted between minor countries only, it lasts one turn
and the side of \paysBrandebourg wins (gaining the provinces).



\event{pI:Spanish Naples}{I-I}{Spanish Naples}{1}{PB}

\history{1497 -- The Spanish rulers decide to take direct control of the
  kingdom of Naples}

\activation{Activated by \dynasticaction{A}{3}, or at the turn following
  \ref{pI:War Italy Napoli}, whichever occurs first.}

\phevnt
\aparag \SPA gains a permanent \CB against \paysNaples (even if on his own
diplomatic track), and also against any owner of a national province of
\paysNaples.

\phdipl
\aparag When \SPA declares a war against \paysNaples, \FRA has a \CB at this
turn only in a reaction to declare a war jointly to \SPA and \paysNaples.
\aparag \SPA may also annex the country by diplomatic means.

\phpaix
\aparag Any province of \paysNaples controlled by \SPA at the end of a turn is
immediately annexed without need for peace. If it was the last province of
\paysNaples, the country is destroyed. When \villeNaples is annexed by \SPA,
remaining provinces of \paysNaples surrender now, are annexed by \SPA and the
country is destroyed.
\aparag In period II, if \SPA has \paysNaples in diplomatic \ANNEXION, the
minor country is destroyed and permanently annexed by \SPA.
\aparag \SPA loses the \CB given by this event as soon as it owns every
national province of \paysNaples.

\effetlong
\aparag As long as \SPA owns \provinceCampania, it gains a free maintenance of
one \FLEET\facemoins, in period II, III and IV.
\aparag \SPA can recreate \paysNaples as a ``liege'' country (see
\ruleref{chSpecific:Spain:Autonomous States}).

% Local Variables:
% fill-column: 78
% coding: utf-8-unix
% mode-require-final-newline: t
% mode: flyspell
% ispell-local-dictionary: "british"
% End:

% LocalWords: pII pI Tordesillas Napoli Milano Habsburg Comuneros Pskov HRE
% LocalWords: Ryazan Barbaros RistoMod Minas Gerais Bonne Esperance Tempetes
% LocalWords: malus Serenissima EW Risto PBNew Mancha Castilla Nueva Vieja
% LocalWords: pV WoSS pIII FWR pIV Colbertian pVI TYW Lublin Khanate PBMod
% LocalWords: Khanates Malaisie Indien Mascate Veneto Iles epices Formose de
% LocalWords: undestroyed Vassalisation heirdom Teutoniques Ost Pommern WRS
% LocalWords: Duche Prusse Livonian
