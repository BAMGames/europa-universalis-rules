% -*- mode: LaTeX; -*-

\definechapterbackground{Playing the game}{chess}
\chapter{Playing the game}\label{chapter:Playing}

\begin{todo}
  Technical details and advice: how to sort the counters, build the player
  aids, use the generic record sheets (TF, exotic resources, \ldots) and
  generally how to hold a game session.
\end{todo}

\section{Building the Game}

\section{Playing the Game}

\section{Teaching the Game}
This Section explains how to hold a Tutorial session to teach the game (on 1
or 2 days). It's a more a checklist of things I say when teaching the game
than anything else. Hence, it is very terse and is intended for people who
already know the game and want to teach it (not for people who don't know the
game and want to learn it). Note that due to all explanations and sequential
solving of actions (everybody want to see what's happening to understand a
bit), Turn 1 might well last for 4 to 6 hours even without any real
Diplomacy\ldots

Most of the tutorial happens on the European map. Thus it is very well
possible to play with only one map (takes less space). In this case, it is
advised to not play \POR, and abstract the \ROTW for \HIS (playing it only on
the minimap). You'll still need a \STAB track and the first line of the
technology track (draw them on some paper). Prices of exotic resources won't
change, but that's not an issue. You also need a European Diplomatic track.

It is possible to play with more or less than 9 players. If playing with more
than 9, make teams. If playing with less, the focus should be around the
Mediterranean: \TUR, \VEN, \FRA, \HIS. Next, add \POL and \RUS (if 2 more
players). \POR has a mostly solo play but is interesting to teach some \ROTW
aspect (together with \HIS). \ANG and \DAN can be left alone without troubles.

\subsection{Before starting}
\aparag Global organisation of tutorial session:
\bparag Quick go through material, aids and rules
\bparag First turn is scripted and stuff will be detailed when playing it.
\bparag Some things done during T1 are stupid but serve pedagogical
purpose\ldots
\bparag Many special rules are skipped for tutorial.
\bparag Turn 2 is not scripted, but I will take many complex decisions for you
(in order to do so instantaneously).

\subsection{Explain material}
\aparag Explain map
\bparag Provinces: income, fortresses of level 1 or 2
\bparag \ROTW: Areas with provinces sharing stuff
\bparag Sea zones and \CTZ/\STZ

\aparag \STAB
\bparag Several tables on \ROTW map, notably \STAB.
\bparag \STAB is super important for incomes and peace.
\bparag \STAB is \textbf{super important}.

\aparag Counters
\bparag Explain \faceplus/\facemoins concept. Mostly for economical counters
(\COL, \TP, \TradeFLEET), but also for \ARMY and \FLEET.
\bparag Show leaders, do not explain values.
\bparag Skip the rest.

\aparag Specific player aid
\bparag Country player aid: period limit table, turn limit table. Only the
first line of these (period \period{I}) is used for tutorial
\bparag Troops and navies cost. Only the first two lines are used for tutorial.
\bparag Recall of Specific rules: ignored for tutorial.
\bparag Do not explain what each column of the table means, just that these
are related to how many counters you may have in play, or how much your army
cost, but that's it.

\aparag Monarch sheet
\bparag Monarch values, initiative
\bparag Period and turn limit reminders
\bparag Actions planners (pre-filled for T1, do not explain yet what they
mean).

\aparag ERS
\bparag 3 ERS on 2 sheets: RT, income and expenses, loans

\aparag Colonial sheet
\bparag Colonies, \TP, \TradeFLEET are recorded here.

\aparag Common player aids
\bparag sorted in turn order
\bparag many, many tables, skip them now
\bparag go to page 8 with game sequence

\subsection{Going through the game turn}
\aparag Using the Game sequence, go through game turn and explain a bit what
happens there.
\bparag Only explain the concepts, not the precise rules and resolution.
\bparag \emph{e.g.} for diplomatic actions, say that you can spend money to
influence minor countries and this is resolved by dice, but do not explain all
the modifiers and stuff at this point.

\aparag Event phase.
\bparag Check if monarchs die
\bparag Random economical event (1 per country)
\bparag Random political events. They do give the game tempo.

\aparag Diplomatic phase
\bparag Alliances between \MAJ.
\bparag Declaration of war (cost \STAB, and \STAB is super important).
\bparag Diplomatic actions on minors.

\aparag Income phase
\bparag 4 sources of income: land, industry, trade and \ROTW.
\bparag Just explain what each category corresponds to globally, no details.
\bparag Gross income is not real income, there will be an Exchequer test.

\aparag Administrative phase
\bparag Economical development of your country.
\bparag Buying new troops.

\aparag Military phase.
\bparag Rounds, impulses.
\bparag Impulse: move all stacks, resolve all battles.
\bparag Movement may cause attrition.
\bparag Interceptions and counter-interceptions are possible.
\bparag Sieges are resolved after all impulse.
\bparag This is a long phase (several rounds) and a complex one (many
decisions to take). After all, the game is largely a wargame\ldots

\aparag Redeployment phase.
\bparag Mostly cleanup of Military phase.

\aparag Peace and Budget phase.
\bparag Signing peaces (\STAB is important as it decide level of peace!)
\bparag Exchequer test.
\bparag Explain the concept of cutting the income in three.
\bparag Explain the idea of the Prestige income that must be spent for VPs.
\bparag Explain that part of income has to be borrowed (loan).
\bparag The economical system works better with loans.
\bparag Exchequer test is complicated and is a complex and meaningful decision
to take each turn.
\bparag \STAB is important for Exchequer test!
\bparag \STAB improvement. Paying money is the best way to get \STAB.

\aparag Interphase
\bparag Cleanup for the turn.

\subsection{Turn 1 - Event and Diplomacy}
\aparag This is tutorial
\bparag Some Segments are skipped, will be added later.
\bparag All actions and decisions are already scripted.

\subsubsection{Event phase}
\aparag Only political events.

\aparag Do not read the full events, just give an overview of the decisions.

\aparag \ref{pI:War Italy Napoli}
\bparag \FRA has claim on \pays{naples} and attacks it. \HIS has family there
and is not happy.

\aparag \ref{pI:Habsburg Dynasty}
\bparag Dynastic actions are a specific thing for \HIS.
\bparag \HIS does \ref{pI:Habsburg Alliance}. Check box on player's aid.
\bparag \HIS is now ally with \paysHabsbourg.

\aparag \ref{pI:War Scotland}
\bparag \paysEcosse attacks \ANG.
\bparag If \ANG is not played, still do this event but explain that it Will
have no effect\ldots

\aparag Revolt
\bparag \REVOLT\facemoins in \provinceCatalogne.
\bparag Do not speak of the Diplomatic event that normally comes with it.

\subsubsection{Diplomatic phase (Majors)}
\aparag Wars due to event
\bparag \FRA attacks \paysNaples. \HIS join in Limited
Intervention. \paysSavoie joins \FRA because of ``specific rules of the
event''.
\bparag Lower \STAB accordingly. Recall the \STAB is super important.

\aparag Discussions: Explain that normally players can discuss privately and
make agreements but that is already decided. Explain the plans for each
country for the turn:
\bparag \ANG vs \paysEcosse.
\bparag \FRA in Italy, needs to take \villeNaples for \VPs due to event.
\bparag \HIS crushes \REVOLT, plus maybe tries to help \paysNaples. Plus goes
to \continentAmerica.
\bparag \POL and \RUS ally and try to destroy \paysDon.
\bparag \POR explores and tries to go to \continentIndia.
\bparag \TUR attacks \paysGeorgie and tries to moves in the \regionBalkans
(explain it's free for all).
\bparag \VEN tries to take \villeRagusa (explain it gives a commercial bonus,
without trying to explain it).

\aparag Announcements.
\bparag \RUS and \POL sign an offensive alliance.
\bparag Explain Defensive and Offensive alliances.

\aparag Declarations of War (1).
\bparag \RUS attacks \paysDon, \POL joins the war.
\bparag Lower \STAB (\RUS has no \CB, \POL has due to alliance).
\bparag Explain the deal: the first annexed province goes to \RUS, the second
to \paysUkraine, the third to \RUS. That's why \RUS is declaring the war
first.

\aparag Declaration of War (2).
\bparag \TUR attacks \paysGeorgie.
\bparag Explain that \TUR has \CB due to religion. Lower \STAB.

\subsubsection{Diplomatic phase (Minors)}
\aparag Explain the correct way to do it:
\bparag Secret planning of action and writing (target + investment + final
modifier).
\bparag Rolling dice, writing results.
\bparag Changing game state (moving markers).

\aparag For tutorial, all actions are written. Be sure to resolve them in the
order indicated here.
\bparag Before resolving the first action, explain the principle: 1d10+mod vs
2d10, gaining progression point.
\bparag Before resolving each action, make the player compute the modifier for
it. Take the time it need to find the correct modifier.
\bparag After rolling the dice for each action, make the player write down the
result.
\bparag After rolling (and noting) for the first action (and if it's a
success\ldots) grab the diplomatic counter and explain how to make the
progression.
\bparag If reaching \SUB or \RM, explain the choice and let the player
choose.
\bparag Let the players try to do the progression themselves, but check that
it is correct.

\aparag \POL on \paysBrandebourg.
\bparag Explain the principle. Compute the modifier. Check that it is the same
as written.
\bparag Talk about various investments and say that it is usually better to
keep low investment for most actions.
\bparag Roll dices. Write result.
\bparag If success, grab the counter, explain that the values are progressions
points for the various levels and do the progression.

\aparag \POR on \paysMaroc and \VEN on \paysPerse
\bparag Show how the religious bonus/malus is important (compare with previous
action).

\aparag \RUS and \TUR on \paysKazan.
\bparag Explain what happens when two countries attempt diplomacy on the same
minor and resolve it.
\bparag Explain that because of that you need to secretly plan everything,
then announce everything (and find conflicts), then start resolving.

\aparag \TUR on \paysAlgerie and \paysDamas.
\bparag At this point, players should be able to resolve action on their
own. Let them do the steps while simply checking that everything is done
correctly.

\aparag \HIS on \paysPapaute, \paysSuisse and \paysLorraine.
\bparag Let \HIS do its stuff, just tell them not to do the last written
'action'.
\bparag Explain that \paysSuisse has no religion.

\aparag \FRA on \paysGenes.
\bparag Explain that this is an attack on \HIS minor and that \HIS can (and
will) react. Because this is a reaction, \HIS cannot gain levels even if
winning.
\bparag Resolve action.

\aparag \FRA on \paysToscane.
\bparag There is special a \bonus{+2} due to the event.


% Local Variables:
% fill-column: 78
% coding: utf-8-unix
% mode-require-final-newline: t
% mode: flyspell
% ispell-local-dictionary: "british"
% End: