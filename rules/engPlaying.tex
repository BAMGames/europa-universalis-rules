% -*- mode: LaTeX; -*-

\definechapterbackground{Playing the game}{chess}
\chapter{Playing the game}\label{chapter:Playing}

\begin{todo}
  Technical details and advice: how to sort the counters, build the player
  aids, use the generic record sheets (TF, exotic resources, \ldots) and
  generally how to hold a game session.
\end{todo}

\section{Building the Game}

\section{Playing the Game}

\section{Teaching the Game}
This Section explains how to hold a Tutorial session to teach the game (on 1
or 2 days). It's a more a checklist of things I say when teaching the game
than anything else. Hence, it is very terse and is intended for people who
already know the game and want to teach it (not for people who don't know the
game and want to learn it). Be sure to read this Section fully before the
Tutorial (so you know what to say and not say) and use it as a checklist
during the session. Note that due to all explanations and sequential solving
of actions (everybody want to see what's happening to understand a bit), Turn
1 might well last for 4 to 6 hours even without any real Diplomacy\ldots

Most of the tutorial happens on the European map. Thus it is very well
possible to play with only one map (takes less space). In this case, it is
advised to not play \POR, and abstract the \ROTW for \HIS (playing it only on
the minimap). You'll still need a \STAB track and the first line of the
technology track (draw them on some paper). Prices of exotic resources won't
change, but that's not an issue (that is, they should change but you won't be
able to track that so you'll play with fixed prices, which is good enough for
teaching the game). You'll also need a European Diplomatic track.

It is possible to play with more or less than 9 players. If playing with more
than 9, make teams. If playing with less, the focus should be around the
Mediterranean: \TUR, \VEN, \FRA, \HIS. Next, add \POL and \RUS (if 2 more
players). \POR has a mostly solo play but is interesting to teach some \ROTW
aspects (together with \HIS). \ANG and \DAN can be left out without troubles.

\subsection{Before the Tutorial}
\aparag Make sure to print everything needed\ldots
\bparag As far as possible, get the counters glued and cut in advance.
\bparag The \TradeFLEET Incomes (on the Colonial record sheets) need to be
manually filled (they can't be pre-filled because they're likely to
change). Do it before the Tutorial.
\bparag Same goes with the global record sheets (\TradeFLEET and Exotic
Resources).
\bparag Set up the game before starting to explain, if needed, do the setup
before the rest of the players arrive\ldots There are quite few counters in
play on Turn 1.

\aparag Initial placement
\bparag Make sure that \FRA has enough troops in \provinceProvence to reach
2\ARMY\faceplus there after recruiting (that's 12\MP to \villeNaples).
\bparag Do not place \HIS troops in \provinceCatalogne (to avoid speaking
about troops in revolted provinces) but still close to it (so it can move
there on first round).

\subsubsection{Guideline}
The idea is to introduce stuff only when needed. The first
approaches (explaining material, going through the turn phases) should stay at
a very high level (\emph{e.g.} say that provinces' colour is terrain but do
not try to explain the differences between each kind of terrain as it would be
pointless at that point). Later, when actually playing through Turn 1 and Turn
2, detailed explanations become needed and start to make sense.

Of course, if players are curious and ask questions, answer them. But try to
avoid getting into details too early as you'll likely lose them. That is a
perfect answer is ``this is resolved by die roll and reading a table, we'll
see that in details later when we actually do it.''

As the teacher, do not be shy to take some decisions instead of the players
(\emph{e.g.} undermining or assault?) as it can be difficult for newcomers to
fully understand the implications while the choice looks obvious for an
experienced player. Taking decision in their place will avoid too much
Analysis Paralysis. You may even take ``bad'' decisions at some point to show
why it is bad (``assaulting a fortress of level 3 in \TMED? Sure, let's do it
and figure out why that is silly.'') Be sure to warn the players in advance
(``This is a bad choice, let's do it to see what happen''); and don't be shy
to cheat on some rolls (``suppose you had rolled 8 instead, \ldots'') or
rollback some choices afterwards (``OK, that was silly, let's go back and take
another option'').

The goal of the Tutorial is only to learn the game. There is no winner in
it. Everything that happens during the Tutorial will be erased if you start a
real game. Thus, it's better to see a variety of situations rather than
optimise your choices.

\subsection{Initial Explanations}
\subsubsection{Start of the Tutorial}
\aparag Global organisation of tutorial session:
\bparag Quick go through material, aids and rules
\bparag First turn is scripted and stuff will be detailed when playing it.
\bparag Some things done during T1 may be stupid but serve pedagogical
purpose\ldots
\bparag Many special rules are skipped for tutorial.
\bparag Turn 2 is not scripted, but the teacher will make some complex
decisions instead of the players (in order to do so instantaneously).

\subsubsection{Explain material}
\aparag Explain map
\bparag Provinces: income, fortresses of level 1 or 2, terrain.
\bparag \ROTW: Areas with provinces sharing stuff (coloured boxes and
borders).
\bparag Sea zones and \CTZ/\STZ

\aparag \STAB
\bparag Several tables on \ROTW map, notably \STAB.
\bparag \STAB is super important for incomes and peace.
\bparag \STAB is \textbf{super important}.

\aparag Counters
\bparag Explain \Faceplus/\facemoins concept. Mostly for economical counters
(\COL, \TP, \TradeFLEET), but also for \ARMY and \FLEET.
\bparag Show leaders, do not explain values.
\bparag Skip the rest.

\aparag Specific player aid
\bparag Country player aid: period limit table, turn limit table. Only the
first line of these (period \period{I}) is used for tutorial
\bparag Troops and navies cost. Only the first two lines are used for tutorial.
\bparag Recall of Specific rules: mostly ignored for tutorial.
\bparag Do not explain what each column of the table means, just that these
are related to how many counters you may have in play, or how much your army
cost, but that's it.

\aparag Monarch sheet
\bparag Monarch values, initiative
\bparag Period and turn limit reminders
\bparag Actions planners (pre-filled for T1, do not explain yet what they
mean).

\aparag ERS
\bparag 3 ERS on 2 sheets: RT, income and expenses, loans

\aparag Colonial sheet
\bparag Colonies, \TP, \TradeFLEET are recorded here.

\aparag Common player aids
\bparag sorted in turn order
\bparag many, many tables, skip them now
\bparag go to page 8 with game sequence

\subsection{Going through the game turn}
\aparag Using the Game sequence, go through game turn and explain a bit what
happens there.
\bparag Only explain the concepts, not the precise rules and resolution.
\bparag \emph{e.g.} for diplomatic actions, say that you can spend money to
influence minor countries and this is resolved by dice, but do not explain all
the modifiers and stuff at this point. As a rule of thumb, the explanations
here should stay high level enough that actually looking at any of the tables
is useless.

\aparag Event phase.
\bparag Check if monarchs die
\bparag Random economical event (1 per country)
\bparag Random political events. They do give the game tempo.

\aparag Diplomatic phase
\bparag Alliances between \MAJ.
\bparag Declaration of war (cost \STAB, and \STAB is super important).
\bparag Diplomatic actions on minors.
\bparag Discussions can take a lot of time before players reach an agreement,
Diplomacy is an important part of the game\ldots

\aparag Income phase
\bparag 4 sources of income: land, industry, trade and \ROTW.
\bparag Just explain what each category corresponds to globally, no details.
\bparag Gross income is not real income, there will be an Exchequer test.

\aparag Administrative phase
\bparag Economical development of your country.
\bparag Buying new troops.

\aparag Military phase.
\bparag Rounds, impulses (in initiative order).
\bparag Impulse: move all stacks, resolve all battles.
\bparag Movement may cause attrition.
\bparag Interceptions and counter-interceptions are possible.
\bparag Sieges are resolved after all impulses.
\bparag This is a long phase (several rounds) and a complex one (many
decisions to take). After all, the game is largely a wargame\ldots

\aparag Redeployment phase.
\bparag Mostly cleanup of Military phase.

\aparag Peace and Budget phase.
\bparag Signing peaces (\STAB is important as it decides level of peace!)
\bparag Exchequer test.
\bparag Explain the concept of cutting the income in three.
\bparag Explain the idea of the Prestige income that must be spent for VPs.
\bparag Explain that part of income has to be borrowed (loan) and you may go
bankrupt if you don't want to refund it.
\bparag The economical system works better with loans.
\bparag Exchequer test is complicated and is a complex and meaningful decision
to take each turn.
\bparag \STAB is important for Exchequer test!
\bparag \STAB improvement. Paying money is the best way to get \STAB.

\aparag Interphase
\bparag Cleanup for the turn.

\subsection{Turn 1 -- Event and Diplomacy}
\aparag This is tutorial
\bparag Some Segments are skipped, will be added later.
\bparag All actions and decisions are already scripted.

\aparag Whenever you need to resolve an action or compute a value, go to the
correct table in the aid and explain the process in details.
\bparag But there is no need to read all the modifiers of complex tables
(\emph{e.g.}, when rolling for land attrition read the modifiers headers and
skip the parts that only deal with naval attrition).
\bparag Be sure that the players understand each computation of modifiers and
reading of tables. Go slowly the first time. Let them do it after that (always
checking that they do it correctly).

\subsubsection{Event phase}
\aparag Only political events, and they are pre-rolled.

\aparag Do not read the full events, just give an overview of the decisions.
\bparag Think about ticking off boxes in the event table that can produce
these results\ldots

\aparag \ref{pI:War Italy Napoli}
\bparag \FRA has claim on \pays{naples} and attacks it. \HIS has family there
and is not happy.

\aparag \ref{pI:Habsburg Dynasty}
\bparag Dynastic actions are a specific thing for \HIS.
\bparag \HIS does \ref{pI:Habsburg Alliance}. Check box on player's
aid. Explain that player's aid hold spaces to remember some country specific
stuff.
\bparag \HIS is now ally with \paysHabsbourg (aka Austria).

\aparag \ref{pI:War Scotland}
\bparag \paysEcosse attacks \ANG.
\bparag If \ANG is not played, still do this event but explain that it will
have no effect\ldots

\aparag Revolt
\bparag \REVOLT\facemoins in \provinceCatalogne.
\bparag Do not speak of the Diplomatic event that normally comes with it. Do
not show the revolt tables, just say that it has been randomly decided.

\subsubsection{Diplomatic phase (Majors)}
\aparag Wars due to event
\bparag \FRA attacks \paysNaples. \HIS join in Limited
Intervention. \paysSavoie joins \FRA because of ``specific rules of the
event'' (do not make the checks for the other Italian countries, do not speak
of Venetian possibilities).
\bparag Lower \STAB accordingly. Recall that \STAB is super important.

\aparag Discussions: Explain that normally players can discuss privately and
make agreements but that is already decided. Explain the plans for each
country for the turn:
\bparag \ANG vs \paysEcosse.
\bparag \FRA in Italy, needs to take \villeNaples for \VPs due to event.
\bparag \HIS crushes \REVOLT, plus maybe tries to help \paysNaples. Plus goes
to \continentAmerica.
\bparag \POL and \RUS ally and try to destroy \paysDon.
\bparag \POR explores and tries to go to \continentIndia.
\bparag \TUR attacks \paysGeorgie and tries to moves in the \regionBalkans
(explain it's free for all).
\bparag \VEN tries to take \villeRagusa (explain it gives a commercial bonus,
without trying to explain it).

\aparag Announcements.
\bparag \RUS and \POL sign an offensive alliance.
\bparag Explain Defensive and Offensive alliances.

\aparag Declarations of War (1).
\bparag \RUS attacks \paysDon, \POL joins the war.
\bparag Lower \STAB (\RUS has no \CB, \POL has due to alliance).
\bparag Explain the deal: the first annexed province goes to \RUS, the second
to \paysUkraine, the third to \RUS. That's why \RUS is declaring the war
first (and looses more \STAB).

\aparag Declaration of War (2).
\bparag \TUR attacks \paysGeorgie.
\bparag Explain that \TUR has \CB due to religion. Lower \STAB.

\subsubsection{Diplomatic phase (Minors)}
\aparag Explain the correct way to do it:
\bparag Secret planning of action and writing (target + investment + final
modifier).
\bparag Rolling dice, writing results (for all actions).
\bparag Changing game state (moving markers).

\aparag For tutorial, all actions are written. Be sure to resolve them in the
order indicated here.
\bparag Before resolving the first action, explain the principle: 1d10+mod vs
2d10, gaining progression points.
\bparag Before resolving each action, make the player compute the modifier for
it. Take the time it need to find the correct modifier.
\bparag After rolling the dice for each action, make the player write down the
result.
\bparag After rolling (and noting) for the first action (and if it's a
success\ldots) grab the diplomatic counter and explain how to make the
progression. Don't hesitate to cheat on the first action so that it becomes a
success and you can explain (``let's say the minor only rolled 3 and 6,
\ldots'')
\bparag If reaching \SUB or \RM, explain the choice and let the player
choose (they may stop at a lower box or let the minor stay Neutral).
\bparag Let the players try to do the progression themselves, but check that
it is correct.

\aparag \POL on \paysBrandebourg.
\bparag Explain the principle. Compute the modifier. Check that it is the same
as written.
\bparag Talk about various investments and say that it is usually better to
keep low investment for most actions.
\bparag Roll dices. Write result.
\bparag If success, grab the counter, explain that the values are progressions
points for the various levels and do the progression.

\aparag \POR on \paysMaroc and \VEN on \paysPerse
\bparag Show how the religious bonus/malus is important (compare with previous
action).

\aparag \RUS and \TUR on \paysKazan.
\bparag Explain what happens when two countries attempt diplomacy on the same
minor and resolve it.
\bparag Explain that because of that you need to secretly plan everything,
then announce everything (and find conflicts), then start resolving.

\aparag \TUR on \paysAlgerie and \paysDamas.
\bparag At this point, players should be able to resolve actions on their
own. Let them do the steps while simply checking that everything is done
correctly.

\aparag \HIS on \paysPapaute, \paysSuisse and \paysLorraine.
\bparag Let \HIS do its stuff, just tell them not to do the last written
'action'.
\bparag Explain that \paysSuisse has no religion (in game).

\aparag \FRA on \paysGenes.
\bparag Explain that this is an attack on \HIS minor and that \HIS can (and
will) react. Because this is a reaction, \HIS cannot gain levels even if
winning.
\bparag Resolve action.

\aparag \FRA on \paysToscane.
\bparag There is special a \bonus{+2} due to the event.

\aparag Don't forget to update \EcoRS A according to results. Make sure every
player has correctly filled its \lignebudget{RT after Diplomacy}.
\bparag It is unlikely that someone gains a new \VASSAL but if it happens
remember to handle that when computing Income.

\subsection{Turn 1 -- Incomes and Expenses}
\subsubsection{Income phase}
\aparag Go through \EcoRS B Section by Section (and line by line). After
explaining each, give the players enough time to compute if themselves (except
maybe for Provinces Income which can be difficult\ldots)

\aparag[Land Income]
\bparag \lignebudgetlong{Provinces income}: can be difficult to compute but
does not changes often (so just copy old value). Do not forget \provinceAcores
and \province{Islas Canarias}.
\bparag \lignebudgetlong{Vassal provinces income}: Do not forget to update if
there's new \VASSAL after Diplomacy.
\bparag \lignebudgetlong{Occupation, Pillages, Revolts}: subtracting these
allow to not change \lignebudget{Provinces income} very often.

\aparag[Industrial Income]
% \bparag \lignebudgetlong{Manufactures}
\bparag \lignebudgetlong{European mines}: \POR gets 40\ducats for
\construction{Elmina}.

\aparag[Trade Income.] Many of these computation are not very easy. Take your
time to have everybody managing to find the good value.
\bparag \lignebudgetlong{Domestic trade income}: explain \DTI (and \FTI),
special power of Cereals \MNU.
\bparag \lignebudgetlong{Foreign trade income}: special power of Wine and
Clothes \MNU.
\bparag \lignebudgetlong{STZ+CTZ level income}: difference between \STZ and
\CTZ.
\bparag \lignebudgetlong{STZ+CTZ monopoly income}: what is a Monopoly in
\STZ/\CTZ.
\bparag \lignebudgetlong{Partial/Total monopolies (trade)}: recorded for \VPs
purpose.
\bparag \lignebudgetlong{Trade centres income}: \CC{Grand Orient} has variable
value (do not explain computation, just give value) and is not related to
\TradeFLEET. Explain that it's shared between \paysEgypte and \paysDamas and
in turn they sometimes give it to \VEN and \TUR.

\aparag[\ROTW income]
\bparag \lignebudgetlong{Colonies}: Explain a bit the 3 values of an \Area
\bparag \lignebudgetlong{Exotic resources}: in the \ROTW, they are shared for
an \Area. In Europe, Fish and Salt \MNU produce some. Special Salt \MNU of
\VEN.

\aparag[Gross income] Immediately copy it back in \lignebudgetlong{Gross
  income A}.

\subsubsection{Administrative actions}
\aparag Again, explain the correct way to do it:
\bparag Compute and write everything (action, investment, column, modifier).
\bparag Roll for all, simply writing result.
\bparag Once all rolls are done, go and change state.

\aparag And be sure to resolve it that way\ldots
\bparag Have one country make all its rolls, noting results.
\bparag Only them show how to implement the results.

\aparag Explain the general principle of administrative actions.
\bparag One table to resolve them all. Column and DRM.
\bparag Ways to compute the column and DRM vary widely for each action.

\aparag As for Diplomacy, go slowly, explain how to compute (that is, how to
read the table). Be sure that every player manages to compute its own columns
and DRMs and find the correct values.
\bparag Skip special cases (critical failures in \ROTW actions).
\bparag It's not an easy table to read the first few times, and it is full of
abbreviations. So give the players the time and help they need to figure it
out.

\aparag \POR is a good country to start with as it does a bit of everything.
\bparag Explain the Special \FTI.

\aparag Most actions are done with small investment.
\bparag This maximise long term chances of success.

\aparag Some actions (Technology) are done with medium or strong investment.
\bparag It is actions where there can be an huge immediate gain (new
technology) and thus a short term boost is important.
\bparag Show how a small investment would mean way less chances of immediate
success.
\bparag Sometimes, it's just worth paying a lot.

\aparag Don't forget about conversion cost for new technology, but don't
forget either that it has to be paid for the newly recruited troops.

\aparag When implementing the result of the action, make sure to immediately
change the \EcoRS \textbf{B} for next turn.
\bparag Insist that this is one of the most important trick in keeping the
Income phase manageable.
\bparag Empty boxes mean ``same as previous turn'' and are easy to copy. As
soon as something changing Income happens, it should be recorded in \EcoRS
\textbf{B} so it's not forgotten.

\subsubsection{Logistic}


% Local Variables:
% fill-column: 78
% coding: utf-8-unix
% mode-require-final-newline: t
% mode: flyspell
% ispell-local-dictionary: "british"
% End: