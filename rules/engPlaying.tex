% -*- mode: LaTeX; -*-

\definechapterbackground{Playing the game}{chess}
\chapter{Playing the game}\label{chapter:Playing}

\begin{todo}
  Technical details and advices: how to sort the counter, build the player
  aid, use the generic record (TF, exotic resources, \ldots) and generally how
  to hold a game session.
\end{todo}

\section{Tutorial}
Stuff I say to people when teaching the game in a tutorial session (1 or 2
days).
\subsection{Before starting}
\aparag Global organisation of tutorial
\bparag Quick go through raids and rules
\bparag First turn is scripted and stuff will be detailed when playing it.
\bparag Some stuff done during T1 are stupid but serve pedagogical
purpose\ldots
\bparag Many special rules are skipped for tutorial.

\subsection{Explain material}
\aparag Explain map
\bparag Provinces: income, fortresses of level 1 or 2
\bparag \ROTW: Areas with provinces sharing stuff
\bparag Sea zones and \CTZ/\STZ

\aparag Stability
\bparag Several tables on \ROTW map.
\bparag Stability is super important for incomes and peace.
\bparag Stability is \textbf{super important}.

\aparag Counters
\bparag Explain \faceplus/\facemoins concept
\bparag Skip the rest

\aparag Specific player aid
\bparag Country player aid: period limit table, turn limit table. Only the
first line of these (period \period{I}) is used for tutorial
\bparag Troops and navies cost. Only the first two lines is used for tutorial.
\bparag Recall of Specific rules: ignored for tutorial.
\bparag Do not explain what each column of the table means, just that these
are related to how many counters you may have in play, or how much your army
cost, but that's it.

\aparag Monarch sheet
\bparag Monarch values, initiative
\bparag Period and turn limit reminders
\bparag Actions planners (pre-filled for T1).

\aparag ERS
\bparag 3 ERS on 2 sheets: RT, income and expenses, loans

\aparag Colonial sheet
\bparag Colonies, \TP, \TradeFLEET are recorded here.

\aparag Common player aids
\bparag sorted in turn order
\bparag many, many tables, skip them now
\bparag go to page 8 with game sequence

\subsection{Going through the game turn}
\aparag Using the Game sequence, go through game turn and explain a bit what
happens there.
\bparag Only explain the concepts, not the precise rules and resolution.
\bparag \emph{e.g.} for diplomatic actions, say that you can spend money to
influence minor countries and this is resolved by dice, but do not explain all
the modifiers and stuff at this point.

\aparag Event phase.
\bparag Check if monarchs die
\bparag Random economical event (1 per country)
\bparag Random political events. They do give the game tempo.

\aparag Diplomatic phase
\bparag Alliances between \MAJ.
\bparag Declaration of war (cost Stability, and Stability is super important).
\bparag Diplomatic actions on minors.

\aparag Income phase
\bparag 4 sources of income: land, industry, trade and \ROTW.
\bparag Just explain what each category corresponds to globally, no details.
\bparag Gross income is not real income, there will be an Exchequer test.

\aparag Administrative phase
\bparag Economical development of your country.
\bparag Buying new troops.

\aparag Military phase.
\bparag Rounds, impulses.
\bparag Impulse: move all stacks, resolve all battles.
\bparag Movement may cause attrition.
\bparag Interceptions and counter-interceptions are possible.
\bparag Sieges are resolved after all impulse.

\aparag Redeployment phase.
\bparag Mostly cleanup of Military phase.

\aparag Peace and Budget phase.
\bparag Signing peaces (Stability is important !)
\bparag Exchequer test.
\bparag Explain the concept of cutting the income in three.
\bparag Explain the idea of the Prestige income that must be spent for VPs.
\bparag Explain that part of income has to be borrowed (loan).
\bparag The economical system works better with loans.
\bparag Exchequer test is complicated.
\bparag Stability is important for Exchequer test!

\aparag Interphase
\bparag Cleanup for the turn.

\subsection{Turn 1}
\aparag This is tutorial
\bparag Some Segments are skipped, will be added later.
\bparag All actions and decisions are already scripted.

\subsubsection{Event phase}
\aparag Only political events.

\aparag Do not read the full events, just give an overview of the decisions.

\aparag \ref{pI:War Italy Napoli}
\bparag \FRA has claim on \pays{naples} and attacks it. \HIS has family there
and is not happy.

\aparag \ref{pI:Habsburg Dynasty}
\bparag Dynastic actions are a specific thing for \HIS.
\bparag \HIS does \ref{pI:Habsburg Alliance}. Check box on player's aid.
\bparag \HIS is now ally with \paysHabsbourg.

\aparag \ref{pI:War Scotland}
\bparag \paysEcosse attacks \ANG.

\aparag Revolt
\bparag \REVOLT\facemoins in \provinceCatalogne.
\bparag Do not speak of the Diplomatic event that normally comes with it.

\subsubsection{Diplomatic phase (Majors)}
\aparag Wars due to event
\bparag \FRA attacks \paysNaples. \HIS join in Limited Intervention.
\bparag Lower \STAB accordingly. Recall the Stability is super important.

\aparag Discussions (Explain the plans for each country for the turn).
\bparag \ANG vs \paysEcosse.
\bparag \FRA in Italy, needs to take \villeNaples for \VPs due to event.
\bparag \HIS crushes \REVOLT, plus maybe try to help \paysNaples. Plus goes to
\continentAmerica.
\bparag \POL and \RUS ally and try to destroy \paysDon.
\bparag \POR explores and try to go to \continentIndia.
\bparag \TUR attack \paysGeorgie and tries to moves in the \regionBalkans
(explain it's free for all).
\bparag \VEN tries to take \villeRagusa (explain it gives a commercial bonus,
without trying to explain it).

\aparag Announcements.
\bparag \RUS and \POL sign an offensive alliance.
\bparag Explain Defensive and Offensive alliances.

\aparag Declarations of War (1).
\bparag \RUS attack \paysDon, \POL joins the war.
\bparag Lower \STAB (\RUS has no \CB, \POL has due to alliance).
\bparag Explain the deal: the first annexed province goes to \RUS, the second
to \paysUkraine, the third to \RUS. That's why \RUS is declaring the war
first.

\aparag Declaration of War (2).
\bparag \TUR attacks \paysGeorgie.
\bparag Explain that \TUR has \CB due to religion.

\subsubsection{Diplomatic phase (Minors)}


% Local Variables:
% fill-column: 78
% coding: utf-8-unix
% mode-require-final-newline: t
% mode: flyspell
% ispell-local-dictionary: "british"
% End: