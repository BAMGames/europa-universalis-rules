% -*- mode: LaTeX; -*-

\definechapterbackground{Events}{events}
\chapter{Events}\label{chapter:Events}

% \section{The events phase}\label{chEvents:Events:Events Phase}

\aparag[Overview.] During the event phase, all players simultaneously
check for monarch survival. If the previous monarch is dead (either
because he died at the die roll or he was scheduled to die), a new one
is rolled for. Each player then makes a roll for one economical event. A
die is rolled for the economical situation and then 4 (sometimes 5)
political events are rolled.
\bparag Economical events are always rolled on the same table while
political events are rolled to the table corresponding to the current
period.
\bparag Political events or other circumstances can call for a \REVOLT
event (either anywhere or on some specific country table) as well to
create some diplomatic disarray among minor countries.

\aparag[Sequence.]
\EventsDetails




\section{Monarch survival}\label{chEvents:Survival}

\aparag Monarchs that are scheduled to die at the current turn
die. Their players roll for a new monarch.
\aparag Players whose monarchs are not scheduled to die at the current
turn roll for monarch survival.
\bparag Some historical monarch are exempted from survival rolls during
the first few turns of their reign. Check the specific rules of each
country for details. If such a monarch is currently ruling a country,
the corresponding player does not roll for monarch survival.
\aparag Some events give a modifier for survival tests.
\bparag Countries with negative stability have a \bonus{+1} modifier for
survival tests.
\bparag Monarch with \terme{fragile health} have a \bonus{+1} modifier
for survival tests.
\bparag Turkish sultans with less than 7 in \MIL also have a \bonus{+1}
modifier for survival tests.
\aparag The result of the survival test can be found in the ``survival''
column of the table~\ref{table:Reign}.
\bparag If the net result is 1, the monarch will rule 1 more turn than
initially scheduled. This can only happen twice for each monarch (more
``1'' results are ignored). Mark this on the monarch sheet.
\bparag If the net result is 10 or more, the monarch dies immediately.
The player has to roll for a new monarch.
\aparag \label{chEvents:Survival:TUR revolt}For \TUR only, if the result
is 9 or more, a \REVOLT in Turkey is rolled as
per~\ref{chEvents:Revolts}.
\bparag In addition, for the Turkish player, if the result is 11 or more
there is an automatic dynastic crisis when rolling for the new sultan.

\GTtable{monarchduration}



\subsection{New monarchs}\label{chEvents:Monarchs}

\aparag Players whose monarch died roll for a new monarch.
\aparag[Reign length] First, a die is rolled in~\ref{table:Reign} to
check for dynastic crisis as well as reign length.
\bparag Some countries have die roll modifier for this die. Check the
specific rules of each country.
\aparag[Dynastic Crisis]\label{chEvents:Dynastic Crisis} If the result
is 1 or less, a dynastic crisis occurs.
\bparag The country immediately loses 2 \STAB and the values of the new
monarch will be halved at the first turn (minimum value remains 3).
\bparag In addition, if the country is at war or had dynastic ties with
another major country, a succession war may occur as per
\ref{chSpecific:War of Succession}.
\bparag A new die as to be rolled to determine reign length. Neither
dynastic crisis nor fragile health may occur this second time.
\aparag If the result is 10, the new monarch will have \terme{fragile
  health}.
\bparag A monarch suffering of \terme{fragile health} has a \bonus{+1}
modifier for all its survival tests.
\bparag A new die has to be rolled to determine reign length. Neither
dynastic crisis nor fragile health may occur this second time.
\aparag The scheduled reign duration of the new monarch is found in the
``reign length'' column of the table.
\bparag On a result of 8 or more, the new monarch is young and will have
limited capacities until he reaches adult age.
\bparag On a result of 8, the monarch is only a teenager. He will have a
\bonus{-1} to all characteristics on his first turn of reign.
\bparag On a result of 9, the new monarch is still a child. He cannot be
used as a general and he will have a malus of \bonus{-2} to all
characteristic on his first turn of reign. He will become a teenager on
the second turn (\bonus{-1} to all characteristics) and an adult on the
third.
\bparag On a result of 10, the new monarch is just a newborn. On its
first turn of reign, he will have \bonus{-3} to all characteristics and
cannot be used as a general. He will become a child on the second turn
of reign, a teenager on the third and an adult on the fourth.
\bparag These maluses can never drop the characteristic below 3.
\aparag Note: the extreme results (1 or 10) for duration of the monarch
may only occur as the second die roll in case of a \DC or fragile
health.

\GTtable{monarchvalue}

\aparag[Characteristics] For each of the three characteristics, roll for
a new value in table~\ref{table:Successors Values}.
\bparag The base column for each country is indicated on the side of the
table.
\bparag If the deceased monarch had a characteristic higher than the
base column of his country, roll in the column immediately to the right
of the base column for his successor (for this characteristic).
\bparag If the deceased monarch had a characteristic lower than the base
column of his country, roll in the column immediately to the left of the
base column for his successor (for this characteristic).
\bparag If a dynastic crisis occurs this turn, the values of the new
monarchs are always rolled on the base column of the country.

\aparag[Military average] Unless specified for some named monarchs, the
monarch can also be used as a general (or admiral for \VEN) and lead
troops in battle. However, the exact values of monarchs as military
leaders are not known precisely before they actually fight a
battle. Instead, only a ``military average'' is known which give a
global indication whether the monarch will be a good or bad leader, but
surprises may arise.
\bparag Once the characteristic of a monarch are known, roll 1d10 and
cross-reference the result with the \MIL of the monarch
in~\ref{table:Monarchs Military Skills} to find the military
average. Report this value on the monarch sheet.
\bparag When needed, and only when needed, the precise values for
\terme{manoeuvre}, \terme{shock} and \terme{fire} are determined. This
usually happens at the first battle involving the monarch, except for
\terme{manoeuvre} which may be needed earlier due to attrition.
\bparag For each of the value needed, roll a die on the bottom
of~\ref{table:Monarchs Military Skills} to obtain a modifier between
\bonus{-2} and \bonus{+2} and add this modifier to the military average
to get the exact value.
\bparag These final values can never be less than 0 or more than 6.

\aparag[Siege] Only monarchs whose military average comport a
\textetoile may have a siege value. All other have a siege value of 0.
\bparag If a monarch with a \textetoile is implied in a siege, consider
his base value for siege as \bonus{1}, modified as the other
characteristics by a roll on~\ref{table:Monarchs Military Skills}
(minimum 0).

\begin{exemple}[New monarch]
  It is the beginning of turn 3. The Doge, \monarqueBarbarigo is
  scheduled to die now, so \VEN has to roll for another one. The values
  of \monarqueBarbarigo are 8/5/6 (for \ADM/\DIP/\MIL). Since there is
  no specific monarch in the rules for the succession of
  \monarqueBarbarigo, the regular procedure for new monarchs is used.

  Firstly, \VEN rolls for reign length. He rolls a 7 which is modified
  by \bonus{-2} per specific rules for \VEN (see the modifier on the
  right of~\ref{table:Reign}) for a result of 5, hence Loredano will
  last for 7 turns. He becomes Doge on turn 3 and is thus scheduled to
  die at the beginning of turn 10. (historically, he will fail a
  survival roll on turn 7).

  Then \VEN rolls for the characteristics. Looking on the right
  of~\ref{table:Successors Values}, there is no special case for \VEN,
  so the base value if 6. Since the \ADM of \monarqueBarbarigo is 8,
  larger than the base value, \VEN gets a 1 column shift bonus for this
  characteristic. He rolls a 4, cross-referencing it in column 7 gives a
  new \ADM of 6. Then, \VEN rolls for \DIP and gets a 6,
  cross-referencing it in column 5 (\bonus{-1} column for the less than
  base value of the predecessor), he gets 5. Lastly, he rolls a 8 for
  \MIL which, in column 6 (predecessor has the same as base value),
  gives 7.

  So, Loredano is 6/5/7, rather a good monarch.

  \textbf{Remark:} The die rolls for determining the values of the Doge
  should all be modified by \bonus{+1} as per specific Venetian
  rules. This has been omitted here to simplify the example. Consider
  that the actual rolls were 3, 5 and 7 to obtain the same results.
  % VEN chosen for example because it is the only country with no
  % scheduled successor for the initial monarch (or almost scheduled as
  % FRA/HIS) *and* with initial monarch with values both below and above
  % the base value...

  Then, \VEN rolls for military average and gets a 9! Cross-referencing
  it with the \MIL of 7 in~\ref{table:Monarchs Military Skills} gives a
  military average of 3. The new Doge could even be a military genius.

  Later, on turn 5, \HIS, \FRA and \paysPapaute ally themseves in the
  \terme{League of Cambrai} and attack \VEN! The player decides to send
  his Doge as an admiral to try and repulse the enemy fleet. When the
  Venetian galleys go at sea, \VEN needs to known the \terme{manoeuvre}
  of his Doge. He rolls a die and gets 6, this means no modification
  from the military average of 3. A bit later, the Venetian fleet engage
  the Spanish and the \terme{fire} and \terme{shock} of the Doge must be
  known. \VEN rolls two dice, getting respectively 2 and 10 for
  modifiers of \bonus{-1} and \bonus{+2} from the military average.

  So, as an admiral, the Doge is a leader 325. Quite a good new for a
  galleys admiral in this age of boarding.
\end{exemple}

\GTtable{monarchmilitary}

\aparag[Excellent Ministers]\label{chEvents:Excellent Ministers} Some
events can give \terme{Excellent Ministers} that enhance the
characteristics of the monarch for some time.
\bparag Unless specified otherwise, the value of the Minister may be
used instead of the value of the Monarch when rolling for a new Monarch.
\bparag Unless specified otherwise, a Minister retires after the ``new
Monarch'' segment of the event phase (during the turn he is scheduled to
leave office).




\section{Economical events}\label{chEvents:eco-events}

\aparag Each player rolls for an economical event during the
``Economical events'' segment of the Event phase.
\aparag To roll for an economical event, each player rolls two dices and
look for the result on~\tableref{table:Random economical events}.
\bparag The first die gives a column. The second die gives a line. By
crossing the column and line, the number of an economical event is
found.
\bparag The effect of the event applies
immediately. Check~\ref{chapter:Events:Eco} for the description.
\aparag Unless explicitly specified in the description, each economical
event only affects the player rolling it.




\section{Economic situation}

\label{chEvents:Economic situation}
\aparag A die is rolled for both the global economic situation and raise
of piracy.
\bparag Combined with the \ref{eco:Piracy}, it is now possible to place
\emph{Piracy} on the map.
\bparag The result of the die is kept. It will be used during the
Administrative phase to handle the variation of prices of exotic
resources. See~\ref{chExpenses:ExoticResourcesPrices}.




\section{Piracy}\label{chEvents:Piracy}



\subsection{Raise of the Piracy}

\aparag The roll for economic situation, read in \tableref{table:Random
  Piracy and Economy Roll}, may cause the apparition of some \PIRATE
(\pays{pirates} \corsaire). The result is partly modified if there is a
named pirate in play at this turn, see~\ref{chEvents:Named Pirates}.
\bparag A \PIRATE \corsaire\facemoins or \PIRATE \corsaire\faceplus is
placed in the target \STZ according to the table, unless a result of 1
or 10 was obtained, in which case the procedure below is used.
\bparag If ``Inflation'' is obtained, increase the level of inflation by
1 (without exceeding the maximum level). This is the same effect as
\ref{eco:Inflation} except that it can happen the same turn as another
\ref{eco:Inflation} or \ref{eco:Deflation}.
% Jym : not sure about the cumulativeness of this with the eco events...

\aparag If the economical situation roll was 1 or 10 (or 3 or 6 and a
named \pays{pirates} \LeaderA is alive) or if~\ref{eco:Piracy} was
obtained this turn, several \PIRATE may appear. Determine the target of
piracy for the turn.
\bparag If there are two or more causes for piracy, then the target is
``Everywhere''.
\bparag If the only cause for piracy is the economic situation, the
target is indicated in the table.
\bparag If the only cause for piracy is a single economical event, then
the target is rolled at random: 1--5: America ; 6--10: Asia.

\aparag For each \STZ in the target, in the order indicated below (this
is relevant for named pirates), roll on die. If the die is higher than
the appearance threshold of the \STZ, place a \PIRATE\facemoins there.
\bparag In a \STZ with several \PIRATE\faceplus, or with a named pirate,
check whether they stay at see or try and loot an establishment.

\aparag[Targets of piracy] Depending on the target, roll for the \STZ
indicated, in order.
\bparag Everywhere: \seazoneCaraibes, \seazone{Atlantique W},
\seazoneIndien, \seazoneOman, \seazoneGuinee, \seazoneRecife,
\seazonePerou, \seazoneFormose, \seazonePatagonie, \seazoneTempetes,
\stz{Canarias}.
\bparag America: \seazoneCaraibes, \seazone{Atlantique W},
\seazoneGuinee, \seazoneRecife, \seazonePatagonie, \seazoneTempetes,
\stz{Canarias}.
\bparag Asia: \seazoneIndien, \seazoneOman, \seazonePerou,
\seazoneFormose, \stz{Canarias}.
\bparag Yes, \stz{Canarias} is both for piracy in America and in
Asia. Yes \seazoneGuinee and \seazoneTempetes are only for America. Yes,
\seazonePerou is for Asia.

\aparag[Appearance threshold]\label{chEvents:Piracy Level}
The appearance threshold for \PIRATE for each \ROTW \STZ is written on
the right-hand-side of the \STZ symbol.
\bparag Some \STZ have several thresholds. Use the higher one if only
one country has \COL/\TP bordering the \STZ; use the second number if
two countries have \COL/\TP bordering the \STZ and use the lowest number
if three or more countries have \COL/\TP bordering the \STZ.
\bparag Remember that the \STZ contains the seazone where the symbol is
located as well as all adjacent seazones. Thus, an establishment borders
the \STZ if it is coastal and touch any of these zones.

\aparag[Pirate Placement] \label{chEvents:PiracyPlacement} When placing
Piracy counters on the map:
\bparag No \PIRATE counter may be placed if there is no commercial fleet
in the target \STZ.
\bparag Two \PIRATE\facemoins are immediately exchanged for a single
\PIRATE\faceplus.
\bparag There are no limits to the number of \corsaire counters that can
coexist in a single \STZ.



\subsection{Named Pirates}\label{chEvents:Named Pirates}

\aparag Some famous pirates appear as named \pays{pirates} \LeaderA and
are treated like other named leaders.
\bparag If a named pirate is alive and not yet on the map, he will take
command of the first \PIRATE that appears in his area of action
(\continent{America}, \continent{Asia}, or \ROTW).
\bparag Named pirates may never be placed in Europe, even if a \PIRATE
can be placed in Europe due to various events.
\bparag If more than one named pirate is in play, place them in
decreasing order of Rank. It is possible that one or more named pirate
are not placed at a given turn. It is possible that a pirate admiral is
placed in the same \STZ that another (existing from previous turn).

\aparag \PIRATE with named admiral always test to check if they loot a
\TP or \COL, with a modifier to the die roll equal to the \terme{shock}
of the admiral.

\aparag At the end of a turn, if there is a \PIRATE\facemoins with a
named admiral, it becomes a \PIRATE\faceplus. \PIRATE\faceplus, even
with admiral, remain unchanged.

\aparag As any privateer, named pirates use their \terme{manoeuvre} to
protect the \PIRATE they are stacked with and check for survival if
their \PIRATE is destroyed.



\subsection{Sea or land?}\label{chEvents:PiracyTarget}

\aparag If there are no named pirate and no more than one
\PIRATE\faceplus in a given \STZ, then all \PIRATE in this \STZ will
attack \TradeFLEET.
\bparag If there is a named pirate, check if he stays at sea or tries to
loot an establishment.
\bparag Whenever a second \PIRATE\faceplus is placed in a \STZ,
immediately check if he stays at sea or tries to loot an establishment.
\aparag Each \PIRATE that may loot is tested by one die, modified by +1
for each (other) \PIRATE\facemoins and +2 for each (other)
\PIRATE\faceplus present in the \STZ.
\bparag Named pirates also add their \terme{shock} value to this test.
\bparag A result of 10 or more means a looting, otherwise the pirate
will attack the commercial fleets.
\bparag The target of the looting is chosen at random among all \TP or
\COL bordering the \STZ. Move the counter to the target (it is
considered to disembark at the beginning of the first round of the
military phase).




\section{Historical/Political events}\label{chEvents:political-events}

\aparag Political events are grouped by period. To each period
corresponds both a set of events and a table to roll for these. Events
are rolled using two consecutive die rolls. Each result is either the
number of an event or a 'R' followed by the number of an event.

\aparag Each turn, 4 political events are rolled. It may happen that a
fifth event is required. No more than 5 political events can happen on
the same turn.
\aparag To roll for an event, roll two dices.
\aparag If the first result is 10, the second die indicates what to do
according to the ``10'' column in the table.
\bparag A result of ``+1'' (event) indicates that a fifth event will be
rolled this turn.
\bparag A period number (either the previous or next one) indicates that
this event will be rolled on the table of the given period instead of
the current one.
\aparag If the second result is 10, the last line of the table indicates
what to do.
\bparag Usually, this event will have to be redrawn in either the next
or previous period. A third roll might be needed to decide between next
and previous period.
\aparag In no case can more than 5 events occur in the same turn. If the
``+1'' result is obtained more than once, this only means that the
corresponding event will be rolled on another period table, as
indicated.
\aparag In no case events of a period other than the current, next or
previous one can occur.
\bparag In such cases (double-next or double-previous), restart from the
current period table.
% \bparag \emph{e.g.} in period I, the result indicates that the event
% must be rolled in period II, where another $10$ indicates that the
% event should be rolled in period III. The event is instead rolled in
% period I.
\aparag If none of the die roll is '10', proceed as follows:
\bparag The first die gives a column in the table.
\bparag The second die determines the precise event.
\bparag Results of the table are counted from the top of the column.
\bparag Already marked off results are skipped.
\bparag If the bottom of the column is reached, then the count resumes
in the next column (wrap around to the first column if the last column
with events of the table was already reached).
\bparag When the count reaches the second die, the result reached by the
count will be applied.
\begin{designnote}
  This counting down is slightly more complicated than a simple
  cross-referencing of two numbers (as done for the economical
  events). It has, however, interesting properties with respect to the
  probabilities of each event occurring.

  The first columns of each event table are usually more likely to be
  rolled than the last ones, hence events there are more likely to
  occur. However, marking off an event has the effect of ``reducing''
  the size of its column, thus events at the top of the following column
  become more probable.

  Hence, in a given period, events of the first columns are likely to
  occur early in the period while events in the last columns are
  unlikely to occur early but become more and more likely to occur as
  time goes.
\end{designnote}
\aparag Once the result is found, if the corresponding event can be
applied, mark the result off in the table as it must be skipped in
future counts.
\bparag Some events cannot occur if certain conditions are not met, in
which case an event must be re-rolled with or without marking off the
corresponding result in the table. Check the precise event description
for details.
\bparag Most events can only happen a given number of time (usually,
only once). If an event is drawn after already occurring its maximum
number of time, mark off the result in the table but play \RD instead.
\bparag Some events share the same number in the events list for the
period, such as~\ref{pI:End Golden Horde} and~\ref{pI:Pskov Ryazan}. If
this number is obtained in the table, apply these events in order. It is
possible that all the events with the same number occur in the same turn
if this number is rolled several time in the table.
% (Jym). Clarification: \REVOLT is only the revolt.  \RD is the event.
% 'R*' results cause \RD if no previous \RD.  Rerolling a one-time event
% causes \RD.  Events causing revolts (civil wars, ...) cause \REVOLT,
% not \RD, hence don't prevent further \RD.
\aparag[Revolts] If the result if 'R' followed by a number, a \RD
(\REVOLT/Diplomacy event) may occur.
\bparag If no \RD occurred this turn (either by another 'R' result or
because a specific event description told to apply a \RD event), then a
\REVOLT/Diplomacy event (\RD) happens as per~\ruleref{chEvents:Revolts}
and~\ruleref{chEvents:diplomacy}.
\bparag If one or more revolts already occurred this turn, then the
event given by the number occurs.
\bparag Events creating revolts without explicitly applying a \RD event
(eg \ref{pI:Revolt Comuneros}) are not considered as a \RD and thus do
not prevent further 'R' result to be treated as a \RD rather than a
number.
\aparag If the result is a number, the corresponding event occurs
\bparag Check the event description for effects and applicability.

\begin{exemple}[Rolling for events]
  It is the first turn of the first period. Political events are thus
  rolled on \ref{events:pI}. The first die roll is 9. The second is
  6. In the column 9 of the event table for period I, we look at the
  sixth non marked off number. It is ``R6''. Since no revolt has been
  rolled this turn, the first event will be \RD and this box of the
  table is marked off.

  Then, for the second event, the rolls are 10 and 1. This indicates
  that there will be 5 events this turn.

  Then, we reroll for the second event. The rolls are 7 and 9. In the
  column ``7-8'', the ninth non marked number is 3. Since \xref{pI:War
    Italy Napoli} can happen (as explained in the event description),
  the box is marked off and the event will happen.

  Then, we roll for the third event. The die are 3 and 4. The fourth
  number in column ``1-4'' is 1. However, \xref{pI:Tordesillas} cannot
  happen at the first turn (America has not been discovered). Hence, as
  per event description, the event is not marked off and the dice are
  rerolled.

  Rolling again for third event, we obtain 10 and 2. Since there are
  already 5 events scheduled for this turn, there cannot be 6. Hence,
  according to what is written in the last column, the event has to be
  rolled in period II.

  Rolling for third event in period II, the dice are 10 and 5. It should
  be an event in period III. However, in period I, no event of period
  III may happen. Hence, we go back to period I instead\ldots

  Rolling again for third event, the dice are 8 and 2. The second non
  marked box in column ``7-8'' is ``R11''. Since there already was a
  revolt this turn (as first event), the 11 is applied. The box is
  marked off and \xref{pI:End Golden Horde} will happen.

  Then, we roll for fourth event. The dice are 7 and 5. In column
  ``7-8'', we looked for the fifth non marked off box. Since the second
  box (``R11'') has been marked off at the third event, it is
  skipped. The fifth non marked box is in the sixth line, ``R4''. Since
  there already was a revolt this turn, \xref{pI:Hungarian Freedom} is
  applied (and the box is marked off).

  Lastly, we roll for the fifth event. The dice are 5 and 10, indicating
  that we must roll on the next period.

  We roll for fifth event on period II. The dice are 8 and 6.Looking for
  the sixth non marked box in column 8, the result is 7. The box is
  marked off and \xref{pII:War Poland Turkey} will happen.

  So, to sum up, there will be 5 events this turn (instead of 4) and
  these will be:\par
  R/D, \ref{pI:War Italy Napoli}, \ref{pI:End Golden Horde},
  \ref{pI:Hungarian Freedom}, \ref{pII:War Poland Turkey}.

  The revolt is resolved as explained below. The other events are
  resolved as per their specific description.
\end{exemple}

\begin{playtip}
  One player, usually one knowing the game quite well, should be
  designed to take care of political events. This ``events-keeper'' ask
  other players to roll the dice and take care of counting the boxes and
  marking off the results as needed, making note of which events do
  occur each turn.

  It is easier to start by rolling all four events (with the
  event-keeper quickly checking that the event can occur). Then, once
  they are known, the event-keeper should read (aloud) the events
  descriptions for all to hear.

  Since the precise order in which the events are resolved is not an
  issue, the events-keeper can read them in an order different from the
  one they were rolled. Especially, it is advised to first announce all
  the events names and then read the descriptions. It is also often
  preferable to start reading small events with few specific rules and
  finish with the big stuff.

  Then while players start to discuss the new diplomatic situation
  created by the events, the events-keeper can ask another player to
  roll for the revolts and diplomatic events. Due to their many
  switching between tables, revolts are best rolled by one other
  player. Diplomatic events, on the other hand, require a lot of reading
  on the same table and are best rolled by the event-keeper with another
  player near the diplomatic track to implement the results.
\end{playtip}




\section{Diplomatic events}\label{chEvents:diplomacy}

\begin{designnote}
  Description and meaning of diplomatic status of minors can be found
  in~\ref{chapter:Diplo}.
\end{designnote}

\aparag Every time a \RD occurs, both a \REVOLT and a Diplomacy event
occur.
\bparag Diplomacy events do not occur when revolts are rolled as per a
specific event description or special rule. Only the \RD event (result
'R' in the table) triggers a Diplomacy event.
\bparag In some conditions an event cannot be played (\emph{e.g.}
because it was already played) and is resolved as a \RD result
instead. Hence there may be several Diplomacy events in a given turn.
\aparag If Diplomacy event occurs, a first die is rolled to know which
religion suffers from troubles.
\bparag Even if several Diplomacy events occur, only one religion will
suffer from troubles.
\bparag The troubled religion is rolled in \ref{table:diplomatic event
  religion}.
\bparag On some results, an additional check for
\ref{chEvents:diplomacy:uprising} will be made.
\aparag Then, for each Diplomacy event, roll for a group of minors to
test in \ref{table:diplomatic event}.
\bparag For each minor country in the group, add 2 dice (+3 if the minor
is of the troubled religion).
\bparag If the result is higher than the fidelity of the minor, the
diplomatic status of the minor is lowered by one for each extra point
(the marker goes one box to the left toward \Neutral).
\bparag For \ROTW minors, if the roll is 2 or more than the fidelity,
then all \dipFR are broken to \dipNR while \dipAT go to \dipFR ; if the
result is 5 or more than the fidelity, all \dipAT and \dipFR are broken
to \dipNR.
\bparag If the fidelity of the minor is higher than the result, nothing
happens.
\bparag Countries that either do not exist (yet or any more) or are
still/already a major country are not tested even if they do appear in
the table.

\aparag[Uprising of a conquered minor
province] \label{chEvents:diplomacy:uprising} If a result of 1, 4 or 7
was obtained for the first die (troubled religion), check for the
autonomous uprising of a province of a Minor country that would be owned
by another power.
\bparag The group affected is chosen by first rolling for a group of
countries in \ref{table:diplomatic event}. Then, in order of the
group(s) obtained, find the first (still existing) minor country that
does not own all its National territory or that has a \Presidio in its
territory. A group is defined by a name in bold (2 groups are sometimes
obtained on a given result).
% (Jym). Seem to be easier that way. Could be uprising of natives in an
% area previously owned.
\bparag \ROTW countries are never subject of uprising.% [TDB?].
\bparag This minor immediately takes back the ownership of one of these
territories, or destroy a \Presidio (chosen at random) and a state of
war is considered to exist between this Minor and the previous owner of
the province (or a state of overseas war if a \Presidio was destroyed
this way). This is not a declaration of wars \emph{per se} (more like a
revolt) so there is no proper Call to allies (only as if continuing an
existing war, the diplomatic patron of a minor involved can choose to
% (Jym). Don't leave small minors (especially barbaresques) alone
% without a chance to get major help or they likely end up crushed
% again.
enter the war (with a \CB) or stay out (with no penalty)) and there can
be no generalisation of war this turn if this is an overseas war.

\aparag[Minor declares a war] If a result of 10 was obtained for the
first die, a random minor declares war to a random neighbour.
\bparag The minor is chosen by first rolling for a group of countries in
the table above and then randomly choosing one existing minor of this
group.
\bparag The chosen minor declares war to one of its neighbours, chosen
randomly.
\bparag Usual calls for allies happen during Diplomacy phase, for all
countries involved in this war.
\bparag If the minor is a \ROTW country, its neighbours are other minors
countries owning areas adjacent to its own areas, as well as countries
having a \TP or \COL in its areas or adjacent ones.
\bparag If the minor is a \ROTW country, the war is an overseas war.
\bparag In case of \paysPerse/\paysOrmus, it is considered as a single
country with neighbours both in Europa and in the \ROTW but will only
declare war (regular or overseas) to one of its neighbours.




\section{Revolt events}\label{chEvents:Revolts}

\aparag \REVOLT may occur either when a \RD is rolled in the event table
(either a result 'R' in the table, an event already occurred its maximum
number of times or specific event condition) or because a specific event
needs to roll for a \REVOLT in a given country (\emph{e.g.} the survival
test for \TUR may require to roll for a \REVOLT in \paysmajeurTurquie).
\aparag If the \REVOLT was rolled as an event (\RD), a Diplomacy event
also happens as per~\ruleref{chEvents:diplomacy}.
\aparag For each \REVOLT , dies are rolled in order to determine:
\bparag The country where the \REVOLT occurs (unless already specified
elsewhere).
\bparag The province where the \REVOLT happens.
\bparag The strength of the \REVOLT .

\aparag Roll 2d10 and read the revolted country in the column of the
current period on \ref{table:alt-revolt-global}. The target country may
be a \MIN or other abstract entity in which case a pseudo-stability is
provided in brackets.
\bparag Decrease this pseudo-stability of minors in the table by
\bonus{-1} if:
\begin{itemize}
\item This is \HOLhol and \SPA perceived the taxes at the preceding
  turn;
\item This is \PORpor at the turn of \ref{pIII:Portuguese Disaster} or
  after.
\end{itemize}
\bparag Even if the \REVOLT was caused by an event from another period,
always use the column of the current period for determining the target
country or area.

\aparag Roll 1d10+the \STAB of the target (or modified pseudo-stability)
on the target country's table (\ref{chapter:events:revolt:Country
  tables}) and read the result in the column corresponding to the
current period (for some countries, there is only one column to use for
all periods).
\bparag Even if the \REVOLT was caused by an event from another period,
always use the column of the current period for determining the revolted
province.
\bparag \textbf{Exception:} For \REVOLT is \FRA cause by \ref{pIII:FWR},
follow the instruction on top of the table.

\aparag[Groups] Often, the result will be a group of provinces; a
further roll is required to choose the resulting province.
\bparag In some groups, the number of targets may vary: roll with equal
probability between all possible choices.
\bparag Some groups bear the same name but do not have the same
content. Use the definitions attached to the table.

\aparag The revolt is against the owner of the province. This may be a
different country than the one whose table was used (\emph{e.g.} a
revolt rolled on the table for \FRA may well occur in \SPA in which case
the revolt is against \SPA for all purposes).
\bparag Revolts inside minors countries are automatically suppressed if
the minor stays inactive for one full turn. Otherwise, they do expand as
usual and hamper supply or income as well as \STAB of the diplomatic
patron.

\aparag Lastly, roll 2d10 in the last column of
\ref{table:alt-revolt-global} to find the strength of the revolt.
\bparag There may be any of the following: \REVOLT counter (either\
\facemoins or \Faceplus), troops (\LD or \ARMY\facemoins), \LeaderG and
sometimes seizing the \fortress (simply noted ``\fortress'' in the
Table).
\bparag If there are troops, they are considered to have the same
characteristics (size, artillery, \ldots) as the country in which the
revolt occurs. Their technology is the technology this country had at
the beginning of the turn. They may not move, are supplied within the
revolted province and can only retreat in the \fortress after battle
(they are thus destroyed if forced to retreat and they do not hold the
\fortress). They may cause a major battle with the usual loss of
\STAB. They do besiege the \fortress, even if there are less \LD than
the level of the \fortress (exception to the normal siege rules).
\bparag \LeaderG sometimes leads the revolt, sometimes the troops,
sometimes there are two leaders, one for each. If the \LeaderG is
written after the \REVOLT, he leads it, if he's after the troops, he
leads them.
\bparag If the \fortress is seized, the \LD is inside it. The \LeaderG
leads the \REVOLT and retreat inside the \fortress once the revolt is
crushed (if the rebels still hold the \fortress).
\bparag A \CTZ or \STZ may be rolled as a revolted province. Roll for
the strength as usual but put in a \PIRATE\facemoins or \PIRATE\faceplus
instead of a \REVOLT. Ignore the troops or capture of the city if any
and use a \LeaderA instead of a \LeaderG. The \PIRATE attacks all
\TradeFLEET in the \CTZ/\STZ.




\section{Technology adjustment}\label{chEvents:Technology}

\aparag If the technology levels of some major or cultural group were
moved during the event phase:
\bparag Adjust the cultural groups markers
per~\ref{chExpenses:Technology:Cultural Adjustment} ;
\bparag adjust the goal markers per~\ref{chExpenses:Technology:Goals
  Adjustment}.
