\definechapterbackground{Military}{military}
\chapter{Military}\label{chapter:military}

\section{Preparations}

\subsection{Leadership}\label{chMilitary:Leadership}
% Put here by JC, to be moved where deemed necessary, mainly for the
% label
\subsubsection{Double-sided Leaders}\label{chMilitary:Double Sided Leaders}
\aparag Some leaders have two different sides for the same country and the
same turns but with two different roles, and can be used as either one or the
other of their roles.
\bparag The counter has a {\textetoile} written on one of the sides,
indicating in which limit the counter counts (independently of the role
for which it is used).
\bparag Kings that can also be something else do not count in the limits
as soon as they are kings.
\bparag For most countries, the role has to be determined at the
beginning of the round, effective for the whole round. \POR is an
exception (see \ruleref{chSpecific:Portugal:Explorers}).

\aparag Some leaders have two sides but no {\textetoile}.
\bparag The side they're used (and the category they're considered) has
to be determined at the beginning of each turn and is effective for the
whole turn.
\bparag In most cases, the choice is restricted because each side
denotes a change of state of the leader (e.g. change of nationality,
crowning, \ldots) and thus only one of them is available at a given
time.
\bparag Especially, generic monarchs of minor countries
(e.g. \leaderShah) are chosen at random when the country goes at war and
cannot be changed before total peace.

\aparag Beware! Some double-sided leaders do not have the same turns of
activity on each side. Thus during certain turns only one side will be
usable.

\subsubsection{Leaders of Multi-national stacks}\label{chMilitary:multi national} 
\aparag[On land]
\bparag If there is a Leader with a monarch symbol (Monarch, Turkish Vizier or
heir allowed to lead troops) in the stack, he must takes command.
\bparag Otherwise, the leader with the most troop of its country (or troops he
is allowed to command) takes command (this may be a replacement leader if the
country with the most troops has no leader in the stack).
\bparag In case of tie, highest ranking tied leader takes command.
\bparag If tied again, players choice (at random in case of disagreement).
\bparag The country of the commanding leader pays for campaigns and win/loss
\STAB in case of Major Battle.

\aparag[At sea]
\bparag If there is a Monarch, he take command.
\bparag Otherwise, the highest ranking leader among those which can command at
least one \FLEET counter takes command.
\bparag Otherwise, highest ranking leader takes command.
\bparag In case of tie, players choice (at random in case of disagreement).
\bparag The country of the commanding leader pays for campaigns and win/loss
\STAB in case of Major Battle.

\subsubsection{Deployment of leaders}% [to be moved in ch. VI]}
% Jym :
% Ca devrait pas plutot etre dans l'interphase ? (placement des nouveaux chefs)

% Jym : Heu, c'est fait la, non ?
%TODO + Hiérarchie.

\aparag[Placement in the \ROTW]
\bparag \LeaderGov and \LeaderG (restricted to the \ROTW) must be deployed on
existing \COL/\TP.
\bparag \LeaderMis must be deployed in Europe
\bparag \LeaderE and \LeaderC may be deployed either in Europe, on an existing
\COL/\TP or in a discovered province where a leader of the same type
(including \anonyme) was present at the end of the previous turn.

\aparag[Placement of unnammed leaders]
\bparag At the beginning of each turn, all \MAJ \anonyme\ leaders are removed,
except those besieged.
\bparag Then, if a \MAJ has less leaders in a category that the minimum limit
indicated in his country table, he draws at random as many \anonyme\ leaders as
necessary to reach the limit.
\bparag All new leaders (both named and unammed) arriving at a given turn are
then deployed at once, respecting the hierarchy.
\begin{exemple}
  At the beginning of turn 1, \POR has one \LeaderE (\leader{Dias}). Since the
  minimum limit for \POR in period I lists 1 \LeaderE, \POR does not get any
  \anonyme\LeaderE. On the other hand, \POR has no \LeaderC but a minimum
  limit of 1 \LeaderC. So he gets a random \anonyme\LeaderC. 

  At turn 1, \FRA has one \LeaderG (\leader{Foix}) and an mimimum limit of 2
  \LeaderG. So he gets an extra \anonyme\LeaderG in order to reach the limit.
\end{exemple}

\aparag[Replacement of unammed leaders]
\bparag If, during military rounds, one player falls below the minimum limit
of leaders for one category (due to death or injury) he gets as many random
\anonyme\ as necessary to reach the limit again.
\bparag The new leader arrive at the beginning of next round, in the same
place (\LeaderE and \LeaderC may also be placed in Europe or in any
\COL/\TP).
\bparag This may break the hierarchy in which case the player must try to
restore it.
\bparag When a wounded leader comes back, the lowest ranking \anonyme\ leader
of the same category is removed and the wounded may take command of any stack
without breaking the hierarchy.

\begin{exemple}
  At the beginning of turn 1, \FRA has two \LeaderG : \leader{Foix} (rank A)
  and \anonyme 2 (rank F). During the military campaign in Italy \leader{Foix}
  get ambushed by perfidious Spaniards near Napoli. He barely escaped with
  several sword wounds and must rest for many months.

  Since \FRA has now only one \LeaderG, he pick at random a \anonyme\LeaderG
  and gets \anonyme 1 (rank E).

  A few rounds later, \leader{Foix} comes back and can take command of any
  stack. Since \anonyme 2 has the lowest rank (F), he is relieved from command
  and removed from the game.

  \smallskip

  During turn 2, \POR has two \LeaderE (\leader{Dias}, if not dead during turn
  1, and \leader{Cabral}). \leader{Dias} boldly tries to circumnavigate
  America but dies, his ship crushed in the ice at Cape Horn. Since \POR still
  has 1 \LeaderE, which is larger or equal to his limit for period I, he does
  not get any \anonyme\LeaderE in replacement.
\end{exemple}

% Jym :
% Pas a sa place ? Je vire.
%\aparag[\bf E3 chefs gouverneurs]
%(note: Certains généraux ou C sont devenus des gouverneurs.)

%Ils agissent comme des C mais entrent dans une limite
%de chefs spécifiques. Ils obéissent  aux règles
%suivantes : \\
% Jym :
% Contredit le point precedent sur le placement des Gov (COL/TP only). Je
% laisse le precedent.
%- ils doivent être placés sur un TP/COL du joueur ou
%qu'il occupe militairement en début de tour, ou en
%métropole. Si en métropole, ils doivent débarquer dans
%rotw sur TP/COL du joueur (ou contrôlé militairement). \\
% Jym :
% A mettre avec les actions administratives, pas ici.
%- ils donnent un bonus de +man aux implantations dans
%leur province, et de +1 dans les autres provinces de
%la grande région. \\
%- ils sont limités aux mouvements dans la grande région
%où ils sont placés en début de tour (ou celle où ils
%débarquent) et les grandes régions adjacentes.

\aparag[Admirals in the \ROTW]
An admiral temporary gets the possibility to go in the \ROTW if both the
following conditions are fulfilled:
\bparag The country has no naval leader allowed in the \ROTW (either \LeaderE
or a \LeaderA with the \ROTW capacity).
\bparag It is period V or later, or the country as at least 3 \COL/\TP in
\continent{America}.
\bparag The \LeaderA allowed to go in the \ROTW (for this turn) is the lowest
ranking \LeaderA. He may not go in seazones with a malus. If he arrives this
turn, he must be placed in Europe.


\aparag[Conquistador table]
The Conquistador table may be used only:
\bparag in \continent{America} and \continent{Africa}, by any \LeaderC and
\LeaderE (half values) ;
\bparag in \continent{Indonesia} by \leader{Coen}, \leader{van Diemen} and
\leader{Maetsuycker} only ;
\bparag in \continent{India} by all \LeaderC restricted to \continent{Asia}
(@). Namely, \leader{Clive}, \leader{Dupleix}, \leader{Bussy} and the minimum
\LeaderC@ of \FRA and \ANG in period VII.

\subsection{Stacking}\label{chMilitary:Stacking}


\section{Deployment [to be moved from ch. V]}


\section{On campaigns [12]}

\section{On movements [13,14,15,reactions]}
\subsection{Special Movements}
% Put here by JC, to be moved where deemed necessary, mainly for the
% label
\aparag[Provinces with several coasts]\label{chMilitary:Movement:Port Multiple
  Coasts} Movements that imply entering a port and going out of a port
may allow a naval stack to go out through a different sea zone than the
one used to enter.
\bparag It is not possible if this means to go through land (if the
province has multiple coasts as defined in \ruleref{chBasics:Multiple
  Coasts}).
\bparag This is possible only if the naval stack owns the port. A \COL
or \TP is required for a \FLEET, a fort is sufficient otherwise
(including convoys).
\bparag[Portugal] It is possible for \SPA to go through \province{Cabo
  Verde} (or any other portuguese settlement) if it has \pays{Portugal}
as a special vassal.
\bparag[Cape Horn] As a special exception, it is not possible to go out
through a different sea zone if it avoids \seazone{Horn}, unless the
naval stack ends its movement there (and goes out at the next round).
\begin{exemple}It is possible, with a \TP placed in \ville{Kyoto}, to
  enter in the same move with 1\NWD coming from \seazone{Japon} and
  going out from \seazone{Pacifique W}, without going through
  \seazone{Coree} nor \seazone{Pacifique NW}. However, it is not
  possible with a \FLEET, nor if the \TP is not owned by the naval
  stack.
\end{exemple}
% Put here by JC, to be moved where deemed necessary, mainly for the
% label
\aparag[Wasteland]\label{chMilitary:Movement:Wasteland} Movement in the
Wasteland area (see~\ruleref{chBasics:Wasteland}) (for all purposes,
including \LoS length computation) is doubled until the end of the
Wasteland (see~\ruleref{chSpecific:End Wasteland}).
% Put here by JC, to be moved where deemed necessary, mainly for the
% label
\aparag[Blockading with several coasts]\label{chMilitary:Movement:Blockading
  Multiple Coasts} A naval stack may blockade a port from any sea zone
adjacent to the port, unless there are multiple coasts as defined in
\ruleref{chBasics:Multiple Coasts}.
\bparag In this special case, there is a \terme{main coast} which is the
one that must be blockaded (usually where the anchor is drawn).
%\bparag Galleys cannot blockade ports that are adjacent to sea zones
%that they cannot reach (\province{Gibraltar}, \province{Tanger},
%\province{Jylland}, \province{Ostlandet}).

\section{On discoveries [58]}\label{chMilitary:Discoveries}
\aparag One can try to discover several sea or provinces at the same time, using the most difficult
sea and adding +1 for each sea eone/province beyond the first. On Land, themovement capacity
limits how many provinces can be discovered.

\aparag As long as the forces doingthe discoveries have not came back to an establishment
existing at the beginning of the military phase, the discovery are not yet usable by other forces.
Neither is 'rendez-vous' authorized between stacks having made independent discoveries (no
stacking).

\aparag[Diffusion of discoveries]
MODIFICATION 07/2007.
\bparag[On sea] At the beginning of period IV, all discovered sea zones of Atlantique
Ocean are known to everyone else at the beginning of the military phase of the turn following the discovery. Other discovered sea zones have a bonus of {\bf -2} for discoveries by other players.
\bparag [On land] At the beginning of period IV, all provinces containing a \COL or \TP are known to 
everyone else at the beginning of the military phase of the turn following the discovery.



\section{On battles [17]}
% Put here by JC, to be moved where deemed necessary, mainly for the label

\subsection{Convoys}\label{chMilitary:Convoys}
Do not use if using the experimental system for Revolts, \corsaire and Natives.

\aparag[Convoy movements and Pirates/Privateers] 
A convoy (or a naval stack carrying Gold)
entering a sea zone of \STZ or \CTZ is attacked on
\tableref{table:Pirates Natives Raids} by the pirates, and each
privateer allowed to attack the owner of the convoy present in the \STZ
(even if not in the right sea zone) if it has the right to attack the
power (see \ruleref{chRedep:CorsairAttack}).
\bparag Only \corsaire\faceplus may attack (be they pirates or
privateers).
\bparag The pirates attack first (one only, with leading
named Pirate if any), then the privateers in order of initiative. The
attack is resolved before regular naval interceptions.
\bparag Only one attack for all the pirates in a given \STZ or \CTZ; and
one attack per \corsaire is allowed per move.
%\bparag  This kind of attack may be performed by Privateer Admirals with a naval stack
%with 1\ND/2\ND as if \Facemoins/\Faceplus Privateers.

\aparag[Attack Procedure]
% For Rerefence: in \attackconvoy
\bparag Roll for naval interception. Pirates with no leader use 2
as Manoeuvre.  If successful, reduce the \corsaire to \Facemoins and proceed with the attack, 
else test for the next interception.
\bparag Before the attack, an accompanying fleet may try to disperse and
reduce the pirates or privateers by making a roll on the corresponding
table.  If successful, the \corsaire is not reduced but the attack is aborted.
\bparag If not aborted, resolve the attack on the  Pirate/Privater raid table.
\bparag Each level in the column \TradeFLEET\faceplus corresponds to one \NTD
captured (with 15\ducats).
\bparag Afterwards, \corsaire goes at port and are finished for the
turn. However Pirates stay in the \CTZ or \STZ and will attack normally Trade
Fleets .
\bparag The \terme{Barbaresque} corsairs cannot attack a Convoy if it is
not in \region{Mediterranee}. \\
%\attackconvoy
\aparag[Convoy in battles] If a battle occurs between two naval forces,
one of them containing a convoy or Transports, the convoy does not take
part in the battle, nor incurs losses during it.
\bparag However, at the end of the battle, the pursuit \textetoilex\ 
result may apply to the convoy or the Transports if the winner decides
so.
\bparag Each \textetoilex\ captures 2 \NTD with corresponding transports
points sunk if loaded with troops, or 10 \ducats captured and 5 \ducats
sunk for \NTD loaded of Gold.
\bparag The rest of the convoy is kept by the loser. 

\aparag[Flota de Oro] \label{chMilitary:FlotaDeOroMovement} As soon as the
\terme{Flota de Oro} (and only this convoy) is sunk or reaches Europe, it
reappears in a Spanish port on the Atlantic coast. %TODO: To be MOVED.

% Jym :
% Deplacement du label dans le morceau correct de "unsorted rules" ci-dessous.
%
% 1 Subsection de moins a ecrire ! Yes !
%
%\subsection{Bazaar}
% Put here by JC, to be moved where deemed necessary, mainly for the label
%\aparag[Effect of Forests]\label{chMilitary:Battle:Forests}

\section{Siegecraft [18]}

\section{End-of-phase [19]}

\section{Redeployment [20]}

\section{Unsorted rules}
\begin{designnote}
  This Section consists in a bunch of unrelated rules relevant to the Military
  Phase. These rules should be properly grouped and dispatched in the correct
  place of this Chapter. This will be done when the military rules will be
  written (aka in a long time\ldots)

  Rules presented here are sometimes barely more than a summary rather than a
  proper rule written in a proper way.
\end{designnote}

\subsection{About logistic}
% Jym :
% Deja present a sa place en V.6.2.3 A.1
% Contradiction avec V.6.2.3 A.1 qui met 1DN/P et non 1DN par face.
% 1DN par face me semble plus logique. Je corrige V.6.2.3 A.1
%\aparag [Lever/Entretenir des corsaires] en plus du coût, il
%faut y dédier 1DN/2DN de la limite de construction du pays.

\aparag[Campaigns for \MIN]
\bparag limited intervention: 1 simple campaign each turn. 1 passive campaign
each round. More can be paid by the \MAJ (pay only the difference in cost of
campaign, not the full campaign).
\bparag full intervention: 1 simple campaign each round. Multiple campaigns
may be obtaind by reinforcement. More can be paid by the \MAJ (as above).


\subsection{Military campaigns}
\aparag Interception is allowed according to the last campaign paid.
\bparag For player without initiative, this is the campaign of the previous
round.
\bparag During first round, players without initiative may intercept (before
their first move) as if they had done a passive campaign.

\aparag When moving both at sea and on land, the cost of both campaigns is
computed separately and only the maximum cost is paid.
\begin{exemple}
  A Major campaign allows to both:
  \begin{itemize}
  \item attack with one naval stack of 3\FLEET ;
  \item move without attacking (exploration possible) with as many naval
    stacks as wanted (non-aggressive movement is not restricted) ;
  \item maintain as many blocus and fight against \corsaire as wanted (only
    movement is restricted) ;
  \item attack with as many small ($\leq$ 5\LD) land stacks as wanted (the
    reason for which the campaign is Major needs not to be the same at sea and
    on land) ;
  \item move without attacking as many large land stacks as wanted
    (non-aggressive movement is not restricted) ;
  \item maintain as many sieges and fights against revolts with large stacks as
    wanted (only movement is restricted).
  \end{itemize}
\end{exemple}

\aparag[None] 0\ducats: No action, no movement, no exploration, no siege,
\ldots allowed (troops may retreat before battle and will fight back if
attacked). No interception allowed.

\aparag[Passive] 10\ducats: 
\bparag Interception allowed only in friendly provinces.
\bparag On land: Moving in friendly provinces ; maintaining sieges and fights
against revolts ; moving \LeaderG (and \LeaderC) to retablish hierarchy.
% Jym :
% "reorganiser stacking", j'imagine que c'est que les chefs, sinon je sens la
% faille format San Andreas...
%       \bparag terre: mvt dans ami seulement; continuer sièges; réorganiser
%      stacking
\bparag At sea: Moving stacks of 1\FLEET maximum. No attack.
\bparag Naval actions: friendly-to-friendly transport, maintain fight against
\corsaire, exploration, maintain blocus.
      
\aparag[Active (aka Simple)] 20\ducats: All allowed by Passive plus
\bparag Any interception.
\bparag On land: one stack of $\leq$ 5 \LD + 1 \Pasha without restriction
[TBD: or +2 pashas ?]
% Jym :
% 2 Pashas garantient presque 3LD, donc +1 au choc. Ca me semble trop
% bourrin. Avec 1 pasha, il n'a en a qu'un a 3LD.
\bparag At sea: one stack with at most 1\FLEET counter without restriction.

% Jym :
% Meme si c'ast plus en cout croissant, ca me semble plus logique de le mettre
% apres pour pouvoir dire "/me too".
\aparag[Active/No Logistic] 10\ducats: Same as Active but
% Jym :
% Diffenrece suivante subtile et prompte a l'erreur (je l'ignorais avant de
% traduire)
% Est-elle vraiment necessaire ? Je ne vois pas trop le genre d'abus faisable
% sans elle...
\bparag At sea: one stack \textbf{without} \FLEET without restriction.
\bparag On land: all stacks $\geq$ 3\LD roll for attrition (even if not moving).

\aparag[Major] 50\ducats: All allowed by passive plus
\bparag On land: either one stack without restriction (neither size nor acton)
or all stacks $\leq$ 5 \LD + 1 \Pasha without restriction  [TBD: or +2 pashas]
% Jym :
% cf simple.
\bparag At sea: either one stack without restriction (neither size nor acton)
or all stacks with at most 1\FLEET counter without restriction.
% Jym :
% Si pas en faveur, je commente...
%            [TBD: ne pas autoriser 2\FLEET sauf \NGD ??? Pas en faveur de ça actuellement]

\aparag[Multiple] 100\ducats: all stacks may act without restriction.


\subsection{Supply, Attrition, Sieges}
\subsubsection{Sources of Supply, Lines of Supply}

\aparag[Source of Supply - Land]
\bparag Source of Supply on Land: any controled city; \TP or \COL.  Exception:
neither owned nor allied: gives weak supply.  Fortresses in desert: gives full
supply in the province, only weak supply further.
\bparag Forts: are Sources of Supply on Land for \LD or \LDE only.
\bparag \Presidios are Sources of Supply only for forces inside the fortress.

\aparag[Supply by naval forces]
Naval forces may provide SoS to Land forces in coastal provinces.
\bparag \de only: can supply up to 1\LD (and 2\LDE) and blocade only fort.
% Jym :
% Pourquoi est-ce redige "blocus *ou ravitaillement*" pour les f ? On n'a pas
% besoin de ravitailler les fortifications, non ?
%ravitaille jusqu'à 1\DT (+ \LDE) et blocus ou ravitaillement de fort (f0) seulement
\bparag \ND counters: supply up to 3\LD (without \ARMY) and blocade up to
\fortress level 1.
%ravitaille jusqu'à 3\DT (sans \ARMY) et blocus ou ravitaillement de f0 ou f1
\bparag One \FLEET counter and at least 2 \ND in the stack: may supply up to
5\LD (including \ARMY) and blocade up to \fortress 3.
%\bparag un pion \FLEET et au moins 2 \ND ravitaille jusqu'à 5\DT (avec \ARMY
%possible) et blocus ou ravitaillement jusqu'à f3
% Jym :
% la f4 ne peut apparaitre qu'en pIV (Baroque) et les seuls F- 1/1 en pIV sont
% POL.
% Donc a part POL le blocus d'une f4 par une F+ comporte toujours 3DN (sauf si
% 2DN+de fait passer + ? de mémoire on ne met pas les de dans les F). Je pense
% qu'on peut sans trop de risque virer la restriction "au moins 3DN"...
%
% Hum, ou alors c'est si plusieurs NTD ? Je suis d'avis qu'une flotte anglaise
% de pVII avec 1ND et 2NTD devrait etre F-, NTD (et encore...) et pas F+ mais
% c'est p'tet trop penible de rediger les regles pour forcer ca. ("F =
% conteneur, on prend celui qui contient les ND et on a le droit de rajouter
% des NTD" ?)
\bparag \FLEET\faceplus with at least 3\ND in the stack: may supply any stack
and blocade any \fortress.
%\bparag un pion \FLEET\faceplus et au moins 3 \ND (Convoy ne comptent pas): ravitaille pile
%de taille quelconque et blocus ou ravitaillement jusqu'à f5
\bparag Convoys and are never taken into account for supply and blocade.

\aparag[Source of Supply - Sea]
Arsenals are SoS for all naval forces; other ports of city, \COL or \TP are SoS for 
stacks with at most one \FLEET; 
\bparag Forts (not of \TP) are SoS only for stack with at most one \ND 
(and possibly \NDE).
\bparag \Presidios are SoS for naval forces without \FLEET;
however, a naval force containing up to one \FLEET may enter a 
\Presidio to supply it (if besieged) or bring forces.
\bparag[Stacking:] Arsenals contain any size of force;
Normal ports can have at most one \FLEET inside;
forts may contain only \ND, \NDE (no \FLEET).

\aparag[Line of Supply - Land] LoS goes from SoS to troops.
\bparag In Wasteland, any non Wasteland native country double the cost in \MP
for LoS until construction of \ville{Saint-Petersbourg} or \eventref{pVI:Great
  Northern War} (whichever occurs first).
\bparag In non-nationnal desert, double the cost in \MP for LoS.
\bparag {\bf When supplied by naval forces} Length of LoS is 3\MP (6\MP in
Wasteland or Desert) plus 1 per sea crossed from a SoS able to supply the
naval stack.
% Jym, precisions :
\bparag Note that the seazone with the fleet is \textbf{not} crossed by the
LoS (only entered to turn the fleet into a SoS for the troops), hence troops
supplied by ships adjacent to a port have a LoS of length 3\MP only.
\bparag Note also that only the 3\MP of ``supply by sea'' is doubled if
required, not the extra \MP for extra seas.
%La LoS a une longueur de 3 MP (ou
%6 MP en Waste) plus 1 MP par mer traversée pour rejoindre une SoS amie qui
%peut ravitailler la flotte.

\subsubsection{When does Attrition occur?}
\aparag[Supply Segment] (Before movement). Land stacks (only) roll for
attrition if at least one of the following case occurs. If several cases
occur, each above the first gives a \textbf{+2} malus to the roll (``Double
cause'').
% Jym :
% Je sépare No LoS de besieged car no LoS peat causer la désintégration de la
% troupe en fin de round, pas besieged.
% En fait, beseged s'aparente a du weak supply plutot qu'a une absence de LoS
% imho.
\bparag No LoS ;
\bparag weak Supply, namely:
\begin{modlist}
\item LoS of 6 or more \MP (except \LD/\LDE in \ROTW)
\item LoS through non-national desert (including last province)
\item SoS not owned by alliance (only controlled)
\item Supplied by a fleet not adjacent to its own SoS
% Jym, ajout pour explo
(except for \LD/\LDE in the \ROTW).
\item Besieged (siege attition)
\end{modlist}
\bparag Force in \terme{Cold area} in an uncontrolled province after Winter
round (including in case of Summer/Summer transition and end of turn) (in the
\ROTW, add the malus of the area) ;
\bparag \Timar after Winter round (as above) (Special,
see~\ruleref{chSpecific:Turkey:Yearly Campaigning})

\aparag[Movement Segment, land] Land stacks roll for attrition at the end of
movement (before battle) if at least one of the following case occurs. If
several cases occur, each above the first gives a \textbf{+2} malus to the
roll (``Double cause'').
% Jym :
% 5LD+pacha attrite ?
\bparag Large stack ($\geq$ 6 \LD, or $\geq$ 3\LD if no logistic) ;
% Jym :
% Je sépare les 2 de mvt. C'est 2 causes separees qui provoquent double cause
% si besoin.
\bparag moving 6\MP or more ;
\bparag moving 3\MP or more during \terme{bad weather} ;
% Jym : 
% etait seulement ici. Il me semble qu'on fait tjr attr avant btaille
% (fourager, ...) J'ai mis plus haut.
%[fait en fin du mvt, avant cmbt] \\
\bparag if embarking \textbf{or} disembarking not in friendly port.

\aparag[Attrition at sea] Naval stacks always roll for attrition except when
staying at port the whole round.

\aparag[Siege Attrition] (during Supply or Siege Segment)
\bparag Besieged during Supply Segment.
\bparag Besieger if the siege is impossible (not enough troops or no LoS) or
if requested by the siege roll.
% Jym :
% Pas a sa place ici. Je vire
%test.  Si le 1d10 de siège est < distance en mvt à la Supply Source,
%faire attrition. Mod.:
%- nbre de DT ennemis dans la forteresse ; \\
%+4 si 1er round de siège (i.e. pas de test de siège par cette alliance
%au round précédent de ce tour ; tjs le cas au 1er round)
%; \\
%-2 si siège continué du tour d'avant. \\
%-S valeur de siège des assiégés

\aparag[After battle]
\bparag On land, any non-winning troop (use specific table).
\bparag At sea, any moving stack (retreat or following to port).

\aparag[End of round (or turn)/Redeployement] In the following cases, a stack
must move and roll for attrition at the end of round or turn. Usual causes of
attrition for movement occur and cause maluses.
\bparag If no LoS during Supply Segment and still no LoS at end of round:
forced redeployement (and attrition). If no way out (naval not allowed), the
stack is destroyed.
\bparag Siege not maintained at end of turn (no Siegework\faceplus).
\bparag Fleet staying at sea at end of turn.
\bparag Fleet going to port at end of turn.
\bparag Peace evacuation

% Jym :
% beacoup de la fin de cette section semble etre de la redite des points
% precedents. Je vire violemment.

%\aparag[Zones de Grand Froid] 
%Elle recouvre Norvège, Suède (sauf Skane), Finlande. + grand froid \ROTW
%dans les régions qui produisent de la neige.
%On fait un test de ravitaillement si dans une province non controllée
%après un round d'hiver. En \ROTW, rajouter un malus égal au nombre de
%neige de la région.

%\aparag[Zones Sous-Cultivée] (indiquée sur la carte). Les coûts doublés
%pour mvts et LOS des pays n'étant pas initialement dans la zone (plus
%Crimée et Cosaques d'Ukraine) ; valable quel que soit le propriétaire
%réel.  Cesse par construction de StPetersbourg ou lors de VI-1 (ce qui
%arrive en premier).

%\aparag[Zones de désert] Ces zones ne procurent qu'un ravaitaillement
%tenu (sauf à son possesseur légitime dans la province uniquement; même
%pas à ses alliés) ; coût doublé pour établir LOS.  NB: c'est un terrain
%accidenté.

%\aparag[Retraite par manque de LOS] si aucune LOS ne peut être établie
%(une LOS ayant au plus 12 MPs, ou nécessitant une flotte de soutien pour
%passer par mer): faire test d'attrition [à +2] en début de round, et si
%pas rétabli à la fin du round : redéploiement forcé (avec test
%d'attrition à +2 à nouveau, qq soit la longueur du mouvement ; mouvement
%par mer ou à travers provinces avec forces ennemies autres que
%forteresse interdit -- si déploimement impossible, force détruite).

%\aparag[Impossibilité de tenir un siège] 
%Ceci est regardé au début de la phase de siège (nbre de DT >= niveau) ;
%si impossible, mvt de rédéploiement forcé vers chez soi \\
%- en fin de tour: si pas Usure\faceplus, redéploiement forcé.

%\aparag[Fin de tour] 
%- Rester en mer en fin de tour: permis avec test d'attrition. \\
%- Mouvement de redéploiement forcé vers chez soi
%(port ou province le + proche) si siège pas établi, 
%ou pour les flottes en mer ne voulant pas y rester. \\
%- Mvt de redéploiement des forces sur terre après paix
%pour évacuer les provinces [attrition à -2, pas de
%limite de MP, vers province la + proche si au-delà de 12MP]

%\aparag[Mvt de redéploiement]  toujours gratuit.
%Une force faisant un tel mvt ne peut faire de mvt régulier
%à ce round.

\subsubsection{Attrition results}
\aparag The effect of the result \textbf{P} in the attrition table depends on
the technology of the stack. In case of mixed stacks, take the worst technology.
% Jym :
% On joue en placant le pillage dans n'importe quelle province traversée. Je
% redige conformement a l'usage.
\bparag Until \TARQ: 1\LD lost during movement \textbf{and} one side of
\PILLAGE in any non-neutral province entered or left.
%Jusqu'à Arquebuse : 1 perte pendant le mouvement et pillage (-)  dans
%a province où est la force après le combat.
%\bparag Mousquet, Baroque, Mouvement : prendre une perte pendant le mouvement ou fourrager,
%(i.e. -1 aux dés du 1er jour de combat si bataille) et poser un pillage (-)
%dans la province où est la force après le combat.
%\bparag  Dentelles : prendre une perte pendant le mouvement ou poser un pillage (-)
%dans la province où est la force après le combat.
\bparag \TMUS, \TBAR, \TMAN: either 1\LD lost during movement or both
\terme{foraging} (-1drm during 1st day of battle) and one side of \PILLAGE in
any non-neutral province entered or left.
\bparag \TL: either 1\LD lost during movement or one side of \PILLAGE in any
non-neutral province entered or left.
\bparag Besieged troops cannot pillage and thus must lose 1\LD.

\subsection{Autour du mouvement}

\aparag[Empilement]
\bparag sur terre, [3 pions et 8 DT] + 2 pachas
\bparag  sur mer [3 pions] + 2 Tr
\bparag de et dc ne sont pas comptés, mais au plus 2 dans une pile.
\bparag[Arsenaux et ports] Ports réguliers ne peuvent contenir plus d'un pion \FLEET.
Seuls les arsenaux (en Europe: indiqués sur la carte; \ROTW (et cas particuliers) : indiqué
sur le pion forteresse) peuvent accueillir 2 ou 3 \FLEET.
Les forts ne peuvent accueillir que des \DN ou \NDE.

\aparag[Pour l'interception]
\bparag Noter que c'est l'intercepteur qui attaque toujours.
et que la bataille est résolue tout de suite entre la pile qui intercepte et la force interceptée.
Il peut choisir de regrouper dans sa pile les forces immobiles amies dans la province.
\bparag un intercepteur peut être intercepté à son tour [les forces qui
interceptent viennent en renfort de la pile qui a été la 1e interceptée] ;
c'est tout de même le joueur qui n'est pas en phase qui est l'attaquant.
\bparag Si une force intercepte dans une zone où il y a déjà une force ennemie,
cette for peut intercepter comme par la paragraphe précédent pour se joindre à
la défense [remarque : il peut alors y avoir plus de 8 DT, les forces supplémentaire
n'ont simplement aucun effet sur le combat].
\bparag  Si un combat d'interception a lieu dans une province avec des forces non impliquée
dans le combat, ces forces suivent la retraite éventuelle de la pile allée engagée en
bataille (mais sans perte). [TBD]
\bparag une pile qui a combattu pendant une phase de mouvement et n'a
pas gagné (perdu ou ex-aequo) ne peut plus bouger ni intercepter.
\bparag malus aux interceptions : \\
	intercepter à travers passe de montagne -2 \\
	intercepter depuis ou vers un marais -1 \\
	intercepter à travers détroit : impossible \\


\aparag[Transport maritime]
\bparag Considéré comme l'action de la flotte.
\bparag Les forces terrestres doivent partir soit d'une côté quelconque,
soit d'un port/arsenal pouvant contenir la pile navale. A destination
nécessairement d'un port/arsenal ami (contrôlé, allié,...) pouvant
contenir la pile navale. La force navale doit stopper son mouvement là.
\bparag Les forces terrestres ont dépensé 2 MP si de port ami à port ami
(3 en \ROTW), 3 MP sinon (6 MP en \ROTW), et peuvent rester dans la
forteresse  ou continuer le mouvement comme province si la forteresse n'est pas assiégée.
Elles peuvent être intercepté dans la province si ils sortent et ont alors le
malus débarquement pour la bataille [TBD ?].
\bparag Jet d'attrition de mouvement pour force terrestre si a embarqué
ailleurs qua dans un port/arsenal.
\bparag Exception avec C ou Gouv ou Expl dans la pile (combinée),
l'embarquement et débarquement en \ROTW est tjs considéré dans un port
ami si il n'est pas opposé (pas de ville, \COL, \TP ou forces d'un
ennemi en guerre contre soi).

\aparag[Invasion navale] Force terrestre : part d'un port ou arsenal
contrôlé.  La flotte doit pouvoir entrer dans le port.
\bparag Il coûte 3 MP à la force terrestre pour être laissé sur une côte
sans port contrôlé (6 MP en \ROTW).
\bparag Jet d'attrition de mouvement pour force terrestre si a débarqué 
ailleurs que dans un port/arsenal.

\aparag[combiner mvt terre/mer]
Une force terrestre doit commencer dans la province côtière. 

\aparag[Mouvement le long d'un rivière en \ROTW] 
se qualifie si un même fleuve ou lac est adjacent
aux deux provinces. Ajouter le coût de traversée du
fleuve le cas échéant.
\bparag Ne sert pas au mvt de pions A; sert pour mvt de LD, LDE
et au ravitaillement. 

\aparag[Combat d'ecrasement (Overrun)]
En cas de disproportion des forces en présence, un combat d'écrasement est possible
pendant le mouvement qui ne l'arrête pas (et l'attrition n'est testée que plus tard);
inversement, si une force attaque un adversaire en surnombre, le défenseur
peut déclarer un combat d'écrasement immédiat.
\bparag Si 4 \LD vs. 1 \LD ou moins :  résoudre le combat et si la force la
plus nombreuse gagne, elle peut continuer son mouvement (ou continuer des
interceptions si ce n'était pas sa phase).
\bparag Si 8 \LD vs. 1 \LD ou moins : la force la plus faible est éliminée automatiquement
sans combat (et son chef ne fait pas de test de perte).
\bparag Dans les 2 cas, le défenseur peut dire qu'il se retire dans la forteresse
de la province avant le combat d'écrasement.


\subsection{règles navales}

\aparag[Ravitaillement des flottes et ports d'attache]
Arsenals are SoS for all naval forces; other ports of city, \COL or \TP: for 
stacks with at most one \FLEET; forts: only for stack with at most one \ND 
(and possibliy \NDE).

\aparag[Taille des forces navales] pour le ravitaillement terrestre, des
forteresses et le blocus.
\bparag \de seuls : ravitaille jusqu'à 1\DT (+ \LDE) et blocus ou ravitaillement de fort (f0) seulement
\bparag pions \DN : ravitaille jusqu'à 3\DT (sans \ARMY) et blocus ou ravitaillement de f0 ou f1
\bparag un pion \FLEET et au moins 2 \ND ravitaille jusqu'à 5\DT (avec \ARMY possible) et blocus ou ravitaillement jusqu'à f3
\bparag un pion \FLEET\faceplus et au moins 3 \ND (Convoy ne comptent pas): ravitaille pile
de taille quelconque et blocus ou ravitaillement jusqu'à f5
\bparag Les pions Convoys ne comptent pas.

\aparag[Rappel de tout ce qui vaut pour une action de la flotte]

\bparag[Exploration] -- résolu pendant le mouvement de la pile
\bparag[Transport naval ou invasion navale]
(c'est-à-dire embarquement, débarquement, y compris ravitaillement et 
éventuel blocus de la province d'arrivée) : achève le mouvement naval. \\
Le joueur doit anoncer en entrant dans la mer son intention de débarquer
des forces et dans quelle province. \\
Exception: une pile qui débarque avec un C - ou un E - ne compte pas comme une action. \\
Noter : après bataille navale, interdiction de débuter une transport naval (oui invasion) au
même round (on peut seulement finir celui-ci, si victorieux)
\bparag[Blocus d'un port et/ou ravitaillement maritime] d'un force sur une côte
\bparag[Ravitaillement martitime]
\bparag [Ravitaillement d'un port sous blocus]  il faut entrer dans le port en étant passé
dans un autre port (peut s'accompagner d'un débarquement de
troupes dans le port) et avoir au moisn autant de forces navales que ce qui serait
nécessaire pour le blocus.
\bparag attaque -- à la fin du mouvement de la pile
\bparag lutte contre pirates et corsaires -- résolu à la fin du round

\aparag Actions passives
\bparag Mouvements
\bparag Interception : une flotte d'un joueur inactif (pas en phase)
peut tenter d'intercepter chaque pile qui bouge dans sa mer ou une mer
adjacente (autant de tentatives que d'opportunités, mais un pays/une
alliance ne peut faire qu'une seule tentative par mer).
Si au port : dans une des mer adjacente.

\aparag[Blocus]
Il faut avoir au moins la force navale voulue selon le niveau de forteresse (voir table ou supra).
\bparag Coupe le bonus de -3 au test de ravitaillement des assiégés ;
\bparag Flotte qui veut sortir ou entrer : doit faire un test pour échapper au blocus
(ou attaquer la flotte en blocus)

\aparag[Rappel des mod. d'interception] VOIR TABLE
\interceptiona
\bparag Pour les flottes faisant Invasion/Naval Transport : le bonus +2 remplace
le malus de -3 au port ou le bonus de +1(même mer) si intention de
débarquée a été indiquée. Pour une flotte en mer: +2 si c'est dans la
même zone; flotte dans arsenal : +2 dans les mers qui bordent l'arsenal;
flotte dans port: +2 si c'est dans la province du port.

%
%\aparag[Flottes indiennes, Mamelouks, Chinoises et Japonaises]
%elles ont un malus de -1 à l'avantage du vent par rapport aux navires
%européens ; leur technologie sera cependant en augmentation
%de la même manière que leur tech terrestre (mais comme ils
%n'ont que peu de F, ça change pas grand chose).
% PB: enlevé depuis qq temps des tables


\subsection{Effet d'un presidio (COMPLETER avec 53.8)}\label{chMilitary:Presidios}

\aparag[Presidios and Blockade]
\bparag The port is considered as blockaded by this fortress, even if the 
country that thus exerts the blockade is not at war with the owner of 
the blockaded port.

\bparag Any exit from or entry into this port by units (privateers, Dn
or F) may trigger an reaction by the fortress. This reaction is decided
by the owner of the Presidio. This a declaration of war (with the usual
CB cost) if the interception is against any unit except privateers.

\bparag The reaction is resolved as a fire by the Presidio on the following table:
\interceptionb

\aparag[As Source of Supply]
\bparag \Presidios are Sources of Supply only for forces inside the fortress.
\bparag \Presidios are SoS Sources of Supply for naval forces without \FLEET;
however, a naval force containing up to one \FLEET may enter temporarily a 
\Presidio to supply it (if besieged) or bring forces.

%%% détroits fortifiés par Jym. Modifiez à volonté.
\aparag[Strait fortifications]\label{chMilitary:Strait Fortifications}Certain
straits are marked with a red naval frontier and a tower symbol near the
province controlling them. These are the strait between Italy and
Sicily (controlled by \ville{Messine}), the entrance to
\seazone{Adriatique} (controlled by \province{Corfou}), the Dardanelles
(\province{Dardanelles}) and the Bosporus (\province{Trace}) in Europe;
and the entrance to Saint-Laurent river (Louisbourg, on Cape Breton
Island), entrance to \seazone{Mer Rouge} (\ville{Socotra}), entrance to
\seazone{Persian Sea} (\ville{Ormus}), the Malacca strait (\ville{Malacca})
and the Sunda strait (\ville{Jakarta}) in the \ROTW.
\bparag In Europe, they act as a \Presidio of level 2 against any fleet
trying to cross the red lines. Using them against any unit but \corsaire
gives a free \CB to the owner of the intercepted stack for the next
turn.
\bparag If a power has a \Presidio on the \province{Dardanelles}, it
negates the effect of the Strait Fortifications for this power.
\bparag In the \ROTW, they act as a \Presidio of level half the level of
the fortress in the province (rounded down). Using them against any unit
but \corsaire give a free \CB (normal or oversea, offended player's
choice)) to the owner of the intercepted stack.
\bparag For the Sunda strait, the city of \ville{Jakarta} must also by
owned, usually by placing a \COL there.
\bparag Minor countries (usually \pays{venise} in Europe and
\pays{Gujerat} for Malacca (sometimes \pays{Chine})) will always use
them against power at war with them. If they are at peace, their
controller chose whether to use it or not. If they are neutral, they
will always use them against \corsaire and never against other naval
units.

\subsection{Les sièges}
\aparag Pour la sape, effet du terrain (non cumulatif)
\bparag           -2      Port sans blocus, terrain clair
\bparag           -3      Port sans blocus, terrain autre que clair
\bparag -2 Terrain accidenté (montagne, marais, forêt, désert) sans port
ou blocus

\aparag[TBD] si un assaut a causé au moins 1 perte (sans modif de
taille ni bonus ``grosse armée'' dans le tour : +1 à la sape et à
l'assaut (max +1, non cumulatif avec le +2 de brèche).


\aparag Les tables sont à jour !

\aparag[Expérimental]
Un assaut qui a obtenu au moins 1 pertes (sans compter les bonus
de Janissaires, \RUS, \POL) sans prendre la forteresse donnera un 
bonus de {\bf +1} aux jets de sappe et aux assauts suivants du tour.

\aparag[Port Siegeworks] 
Ports that are besieged with at least one level of Siegework are
submitted to a fire from the siegework that works the same way as the
Presidios, with a {\bf +1} per counter Siegework\faceplus.  {\bf But the
 port is not blockaded.}

\aparag[Impossibilité de tenir un siège] 
Ceci est regardé au début de la phase de siège (nbre de DT >= niveau) ;
si impossible, mvt de rédéploiement forcé vers chez soi \\
- en fin de tour: si pas Usure\faceplus, redéploiement forcé.


\subsection{Terrains}

\aparag[Effet du terrain sur mouvement et combats]
\bparag Plaine: 1 PM si ami, 2 PM sinon (2 et 4 si hors-Europe) ;
\bparag Accidenté en Europe : 2 PM, sauf 3 PM en Montagne ennemi ;
\bparag Accident en ROTW: 4 PM si ou mvt de forces d'un pays mineur de
ROTW ; 6 pm si ennemi ;
\bparag Rivière, passe, détroit, arrivée ou départ en marais: +1 PM (et
+2 PM HE)
\bparag Déplacement naval: 3 PM (indépendamment du terrain de départ ou
d'arrivée, y compris marais), sauf si de port ami à port ami 2 PM. 6/3
PM en rotw.

\aparag[Les différentes zones de forêts]
\bparag  forêts nordiques : suède+Finlande+côte baltique actuellement orientale
\bparag forêts orientales : celles actuelles (sauf dessus) et Prussia et
adjacent, Lovonie, Podolie.


\aparag[Effet sur le combat]\label{chMilitary:Battle:Forests} REVOIR : tables à jour
\bparag  Modificateurs feu et choc \\
	en marais, forêt ou désert -1 \\
	en montagne pour l'attaquant (sauf s'il a intercepté) -1 \\
	force traversant un fleuve ou une passe de montagne -1 \\
		(1er round, et sauf si il a intercepté) 
\bparag Modificateurs feu \\
	force débarque ou traverse un détroit -2 [1er round]
\bparag Modificateurs choc \\
	force débarque ou traverse un détroit -3 [1er round]
\bparag Modificateurs poursuite\\
	en marais, forêt, désert ou montagne -1 \\
	vainqueur a traversé fleuve, passe, ou détroit ou débarque -1 \\
	retraite du perdant à travers passe, fleuve, détroit ou réembarquement +1

\aparag Si plusieurs piles se rejoignent dans une même province pour
une bataille (2 forces qui convergent ou interception), on prend le plus
défavorable effet de terrain de frontière.

\aparag[Finlande-Suède] Un mouvement de retraite (après bataille ou
redéploiment forcé) est autorisé entre les provinces au
nord de ces deux zones. Le mouvement prend toute la capacité de mouvement
restante (donc il faut faire un test d'attrition car les 12 MP sont dépensés).
C'est la seule forme de mouvement autorisée par ce chemin.

\subsection{Les batailles}
\aparag[Victoire majeure]
Elle est obtenue aux conditions suivantes
\bparag Sur terre en Europe : déroute du perdant et différence des
pertes égale à 3\LD ou plus 
\bparag Sur terre en \ROTW : déroute du perdant, perdant avait au moins
un pion \ARMY européen et différence des pertes égale à 3\LD ou plus
\bparag Sur mer : déroute du perdant et différence des pertes
d'au moins 5\ND ou 8 DGa.

\subsubsection{Les batailles terrestres}
\aparag[Organisation des armées et cavaleries] Valable si au moins un pion armée
de la classe en question.
\bparag[\terme{Sipahi}]  TUR (avant réforme M-2) a +1 en choc et poursuite plaine/désert
\bparag[iim]  bonus +1 au choc en I-IV en plaine/forêt orientale
\bparag[tercios] toutes les autres armées ont un malus -1 en choc contre eux sauf
       i, im, ii et iim en I-V
\bparag[iiim] bonus +1 au choc en IV-V en plaine et forêts occidentales,
\bparag[SUE] bonus de +1 au choc en II-VI en forêts nordiques
\bparag[iv] bonus +1 en III-V en plaines et forêts occidentales

\aparag[Test de survie des généraux]
En Europe, sur terre, on ne teste pas le général d'un camp si son adversaire
n'a pas au moins 3 \LD.

\subsubsection{Les batailles navales}
\aparag Deux jours au maximum en cumulant les pertes jusqu'à la fuite (volontaire
ou obligatoire) d'un camp.
\bparag Fuite obligatoire 
       si le moral arrive à 0, ou
       si le nombre de pertes reçues est > nbre de D de la flotte
               (tenir compte ici des modificateurs finaux aux pertes).
     Dans les deux cas c'est une déroute (avec poursuite, etc.).
\bparag Fin 2e jour si egalité en moral ;
peuvent choisir de retourner à un port (au choix) ou rester en mer (attaquant
d'abord). 
\bparag Les flottes continue leur action si elles ont gagné la bataille, sinon
elles ont fini pour le tour (devant soit retourner auport, soit choisir de rester en mer
en cas d'égalité mais sans rien faire de plus).
Exception : on ne peut débuter un transport maritime après une bataille, même gagnée.

\aparag[Effet de la différence de taille des forces] Modificateurs au dé de bataille
\bparag Si la flotte est de taille >= à (taille+1) adverse, +1 au choc
\bparag Si la flotte est de taille >= à (taille+3) adverse, +1 aux feu, choc
\bparag Si la flotte est de taille >= à (taille+5) adverse, +1 aux feu, choc et poursuite
\bparag Si la flotte est de taille >= à (taille+7) adverse, +1 aux feu,  +2 choc et +1 poursuite
\bparag Si le moral perdu est > moral adverse perdu, -1 aux feu et choc

\aparag[Effet de la taille des forces]
Appliquer variation des pertes:
\bparag Si moins de 6 \ND : réduction des pertes \\ 
Si plus de 6 \ND (ne pas compter les \NDE):
line +1 if 7 to 12\DN; +2 if 13 to 18\DN; +3 if 19+\DN 
de la table 'Size Comparison'.

\bparag Si l'adversaire a dérouté, les pertes sont minimales sont 1.
\bparag Le max de pertes que peut faire une flotte est le double de sa taille
       (1 si 'de' seul).


\aparag[Répartition des pertes]
\bparag  Integer {\bf losses split} evenly in \terme{Damaged}, \terme{Destroyed} and \terme{At port}, in units of \ND.
\bparag {Winner:} 1st \ND\ lost  \terme{Damaged}, 2nd \terme{Destroyed} and 3rd refitted (then loop over).
\bparag {Loser (or equality.):} 1st \ND\ lost  \terme{Damaged}, 2nd \terme{Destroyed} and 3rd \terme{Damaged}.
\bparag Fractions (\tu or \td) vs. \NGD or \NTD are  rounded up. Fractions vs. \ND are applied as \NDE or 2\NDE  in the next category of loss.
\bparag Examples 2: 3\td losses against losing \ND: 2\ND+2\NDE {Damaged}, 1 \terme{Destroyed}\bparag Examples 2: 4\tu losses against winning \NGD = 5 losses: 2  \terme{Damaged}, 2 \terme{Destroyed} (one immediately refitted fo no effect).


\aparag[Pertes en poursuite] En plus des pertes normales, elles permettent
de capturer ou attaquer les transports. Le niveau de capture est égal
au nombre d'étoiles. 
\bparag Capture de navires de guerre = le gagnant peut capturer un DN ou
2 DGa par * en poursuite        (pris d'abord sur les Imm, puis les End, puis les autres)
\bparag Couler Transports = 2 DTr coulé par * de poursuite dédiée à ceci
       une force terrestre transportée au minimum égale à ce que
       ces DTr perdus transportent doivent être détruits.
\bparag Capturer Or = 2 DTr par * de poursuite dédiée à ceci avec 5 ors perdus, 10 \ducats capturés.
Les transports sont gardés avec la flotte (et peuvent être repris, attaqué par pirates etc)
jusqu'à un port de la métropole où ils disparaissent (et or dans le RT).

\aparag[Damaged ships] \terme{Damaged} \ND are written down globally by naval zones:
Mediterranean Sea, Atlantic in Europe, Atlantic in \ROTW, Indian, Asian and East Pacific.
They are refitted for usage:
\bparag cost = 0.5*coût achat DN à un round suivant pour les remettre en état.
               Effet = remet tout de suite en jeu les DN voulus.
\bparag gratuit au début du tour suivant si on entretient la flotte;
\bparag on peut la garder \terme{Damaged} pour un coût d'entretien divisé par 2 ;
\bparag On les remet en priorité dans un Arsenal de la zone, sinon dans un
port capable de les contenir.

\subsection{En ROTW}

\aparag[Indigènes et combat]
Ils attaquent des forces à chaque round normalement ; en cas
de défaite avec déroute, ils n'attaquent pas ce joueur au
round suivant, seulement celui d'après. En cas de victoire
ou de défaite normale, ils attaquent dès le round suivant.
Ils font le siège des forts/forteresses, mais jamais d'assaut
[l'assaut est représenté en fin de round par l'attaque des indigènes]

\aparag Une ville dans une région qui n'est à aucun pays mineur peut
être attaquée sans déclaration préalable de guerre. La déclaration
doit se faire à la phase des combats, les indigènes de la zone 
forment l'armée qui défend la ville et le pays européen peut
ensuite, si il les défait, mettre le siège ou faire l'assaut.
On ne peut installer une COL dans une telle zone qu'en ayant
pris la ville au tour d'avant.

\aparag Un TP établi ou une Mission a un fort. En revanche, une présence militaire
autre qu'une forteresse peut causer l'activation des indigènes.

\aparag Une COL n'a pas cet avantage (mais on peut en construire) ;
cependant dans une COL établie, la présence de forces armées 
n'entraine plus de réaction des indigènes, mais
seulement d'un pays mineur ayant la région, ou lors des 
résultats E* à une colonisation.

\aparag[Pillages]
\bparag Sans A, les pillages en \ROTW sont au plus \Facemoins.
\bparag L'or à terre est capturé à moitié si pillage \Facemoins et en
entier si pillage \Faceplus.


-- NE PAS PRENDRE EN COMPTE --
-- NE PAS PRENDRE EN COMPTE --
-- NE PAS PRENDRE EN COMPTE --
-- NE PAS PRENDRE EN COMPTE --
-- NE PAS PRENDRE EN COMPTE --
-- NE PAS PRENDRE EN COMPTE --
-- NE PAS PRENDRE EN COMPTE --

\begin{designnote}
The following (until end of chapter) is an old set of rule for fast
battle. These are not up-to-date (far from it).

Do not read this. This is work in progress. Part or even all of it may be
entierly wrong.

The tables are up-to-date. You can use them if you want.

A summary of the fast battle system (in English) can be found at
\url{http://old.bamgames.org/Europa/EU8/dev/Fast-battle-en.txt}.

If you need further details, please feel free to ask us on the EU mailing list
at Yahoo groups (in English) or at the forum
\url{http://europa-universalis.frbb.net/forum.htm} (in French, but we'll
answer in English if needed)\ldots
\end{designnote}

\section{Le combat rapide revisité -- Version initiale du PPI -- en chantier -- BEWARE}

En cours de réécriture et adaptation aux évolutions.

\subsection{Présentation}

Ce système de combat terrestre rapide est repris de celui qui doit être présenté 
dans la 2\up{e} extension d'Europa Universalis toujours à paraître. Il doit 
permettre de ne plus prendre en considération les contenances d'armées 
pour aucun aspect du jeu, même hors europe (usure et combat contre les 
indigènes). Les règles du combat rapide, laissant de côté l'attrition, le 
problème de la taille des armées, les spécificités des campagnes hors europe, 
etc, des propositions de complément sont données ici. 


Les règles écrites par Ph. Thibaut sont utilisés sauf pour les ajustements
que nous proposons ; les majeurs sont indiqués
en \textit{italique} ci-dessous.
Cette nouvelle écriture des règles est complète (à la différence de 
\texttt{eu8combat.pdf} qui ne donne que les changements proposés aux règles
de combat rapide).

Enfin, comme le restant de nos modifications ajoutent deux joueurs et des puissances
majeurs potentielles (Pologne, Suède, Prusse, Russie en périodes I et II), 
les tableaux font apparaître ces nations.

\subsection{Autres règles militaires}
-- NE PAS PRENDRE EN COMPTE --

\subsubsection{Phase de jeu}
-- NE PAS PRENDRE EN COMPTE --

On propose comme clarification et modification de la phase militaire de prendre la suivante.
Le premier camp, ayant l'initiative, est celui qui a son monarque
dont la somme des valeurs est la plus forte. Les alliés doivent bouger
ensemble (ceci rend possible une coopération militaire) et prennent alors l'initiative
du moins bon monarque. En cas d'égalité entre deux camps, tirer au hasard pour tout le tour
au début du premier round.
\paragraph{Séquence du round militaire}
\begin{itemize}
\item Test de fin du tour après le round qui commence. \\
\textbf{Modification :} il n'y a pas de test au 2\up{e} round -- ainsi le tour
comporte au moins 3 rounds.
\item Phases du camp 1 : \\
choix de la campagne, mouvements et découvertes ; interceptions 
(et batailles immédiates) 
possibles par le camp 2 
pendant les mouvements, puis 
usure et batailles.
\item Phases du camp 2 : \\
choix de la campagne, mouvements et découvertes ; interceptions
(et batailles immédiates) possibles par le camp 1
pendant les mouvements, puis
usure et batailles.
\item Phases finales : \\
dans l'ordre d'initiative, sièges des deux camps (sape et/ou assaut), 
lutte contre les révoltes et les pirates.
\end{itemize}

\subsubsection{Mouvement, empilement, attrition}
-- NE PAS PRENDRE EN COMPTE --

\paragraph{b. Empilement, usure}
L'empilement maximum est de trois pions dans une même zone terrestre pour
un camp donné avec une limite de 8 équivalents détachement d'un
même camp et deux pachas. Cette limite doit
être respectée à la fin du déplacement de chaque pile.

L'usure des mouvements terrestre  est jouée
sur la table d'attrition donnée ci-dessous dès que la pile
fait au moins 6 PM. Elle se fait sur la table ci-dessous en Europe
(voir plus pour hors-Europe). Les forces enfermées dans des
fortifications subissent aussi une attrition si le siège a été
mis depuis au moins le round précédent.


\paragraph{c. Réorganisation des forces.}
Pendant la phase de mouvement d'un camp, il lui est possible d'intégrer des détachements
dans des armées face -. Il faut deux détachements pour passer une armée - en armée +. 
Il est aussi possible de séparer des pions en armée sans que le nombre équivalent de
détachement soit modifié. Ces ajustements peuvent se faire au cours du mouvement.

En revanche, il est interdit de faire apparaître un nouveau pion armée ; ainsi 2 DT
ne peuvent devenir une A- ou une A+ ne peut se couper en 2 A-. On peut si on veut
éliminer un pion armée dans les réorganisations.

Les forces qui interceptent un ennemi peuvent se réorganiser à la fois dans
la zone de départ (pour laisser des forces en arrière) et dans celle d'arrivée
(pour intégrer des forces déjà sur place). Le général qui a servi à l'interception
doit suivre les forces interceptantes. Celles qui sont interceptées ne
se réorganisent qu'après le combat. L'intercepteur peut donc gagner la
bataille et se retrouver dans la même province qu'une pile ennemie : il
aura le droit de tenter de l'intercepter si elle cherche à sortir de la province,
et si les deux piles sont toujours au même endroit après les mouvements,
une bataille sera résolue, la pile ayant interceptée comme attaquant.

Le niveau d'expérience des forces amalgamées (en une A+) doit être noté sur un papier.
Il faut noter le nombre de DT équivalent vétéran (le reste est conscrit).
Il faut aussi noter (pour la détermination du moral en cas d'égalité aux détachements)
le moral du pion armée.

\paragraph{d. Hiérarchie des chefs.}
La hiérarchie des généraux doit être respectées à chaque round. 
Si la hiérarchie n'est pas respectée en début de round, des forces
terrestres ne peuvent être intégrées à une pile ou laissées que si cela
permet de satisfaire à nouveau la hiérarchie (si plusieurs
mouvements sont nécessaires, ils sont autorisés dans n'importe
quel ordre du moment qu'à la fin des mouvements la hiérarchie est
correcte). Sinon aucune armée ne peut modifier sa composition.
Une pile terrestre ne peut pas ramasser ou laisser des forces en
violant la hiérarchie.

Les piles multi-nationales sont commandées comme il suit : \\
- le monarques passe devant tous les autres généraux (y compris
d'un autre pays) pour commander une pile, \\
- la majorité des troupes décide sinon de quel général commande ;
en cas d'égalité du nombre de DT les joueurs choisissent (ou tirent su hasard
si ils ne se mettent pas d'accord) ; si le pays majoritaire n'a pas de général présent,
on utilise un général de remplacement de ce pays,  \\
- pour le général qui commande, toute la pile est comptée au regard
de la hiérarchie ; pour les généraux des autres pays qui ne
sont pas le commandant en chef, ils n'ont que leur contingent
sous leurs ordres pour déterminer le respect de la hiérarchie.

Ces mêmes règles s'appliquent pour les forces maritimes, avec
comme seule diffférence que la taille de chaque contingent est
comptée comme pour les forces terrestres : un pion F- compte comme
2 D, et un F+ comme 4. Les DGa comptent comme les DNav et les
DC comme un demi.

\paragraph{e. Interception}
Une pile ne peut tenter d'intercepter qu'une seule fois pendant
le mouvement d'une pile ennemie. Une même pile peut
cependant être victime de plusieurs tentatives d'interception
par différentes piles ennemies durant son mouvement, au
plus une interception par province dans laquelle elle entre.

Lors d'une interception, la pile interceptante peut laisser des
forces en arrière et se réorganiser après le mouvement avec une
force déjà sur place (pour respecter la limite de pions
dans la région). Cette pile du camp en réaction est l'attaquant
lors du combat qui est immédiatement résolu (avant tout autre 
mouvement) ; si l'interception est  dans une province contenant des forces terrestres
amies de l'intercepteur, c'est la force en mouvement qui est l'attaquant.
Personne ne peut tenter de refuser le combat 
d'une interception réussie. 

La force interceptante doit faire un jet d'attrition si elle entre dans
une province ennemie non occupée par des troupes amies (qui 
feraient le siège). Celle interceptée teste l'attrition si elle a déjà
bougé d'au moins 6 PM. 

Après le combat, si la force interceptée a gagné le combat, elle peut
poursuivre son mouvement (mais n'aura plus à tester l'attrition si
elle a déjà dû le faire).


\paragraph{f. Passer outre l'ennemi.}
Une force de 2 A+ entrant dans une province ne contenant qu'un détachement
peut déclarer un combat d'écrasement. Le combat est immédiatement
résolu (sans résoudre maintenant l'attrition) et si la force active gagne, 
elle peut continuer son mouvement. Si elle perd, elle retraite et test après le
combat l'attrition éventuelle si elle avait fait au moins 6 PM avant la
bataille.

Une force entrant dans une province contenant une forteresse hostile
non assiégée doit s'y arrêter sauf si elle laisse une force suffisante
pour mettre le siège (1 DT par niveau de la forteresse), le restant
pouvant continuer le mouvement. L'attrition éventuelle portera
en priorité sur les troupes qui ont poursuivi le mouvement.


\subsubsection{Logistique des guerres}

\paragraph{a. Les campagnes.}
Le fonctionnement des campagnes est modifié. Il existe 
4 types de campagnes. \\


%-- campagne passive, 10d \\
%-- campagne active, 20d (une pile active d'au plus 5 DT) \\
%-- campagne majeure, 50d (une pile active non limitée, ou
%plusieurs piles actives d'au plus 5 DT) \\
%-- campagne multiples, 100d (aucune limite) 

Un pacha compte pour 1 DT dans l'empilement pour cette règle et
on peut en mettre un seul dans les piles des campagnes actives ou
majeures limitées à 5DT. Quand il n'y a pas cette limite, l'empilement
maximum est de 8DT, 3 pions plus 2 pachas (règle du chapitre II).

La campagne d'une pile multinationales peut être payée par 
n'importe quel pays majeur ayant des troupes dans la pile. Un
pays mineur ne peut prendre à sa charge que les piles
qu'il commande.

\paragraph{b. Achat de troupes.}
L'achat de troupe se fait uniquement par pion entier. Les limites d'achat par période 
ont été adaptées dans le tableau final ;
elles sont exprimées en équivalent détachement. 

Le prix d'achat d'un détachement terrestre est maintenant toujours égal
à la moitié (arrondie au supérieur) du prix d'une A-.
Il faut remarquer que les pions
armées doivent être achetés comme tels et qu'il est impossible d'en créer
au cours des réorganisations. À la phase de logistique les réorganisations sont
possibles autant avant qu'après l'entretien et le placement des nouvelles unités.


\paragraph{c. Entretien des troupes}
\textit{
Pour rendre compte de la difficulté de maintenir une armée levée 
à cette époque, et à la payer, les coûts d'entretien sont majorés en
temps de paix (y compris pendant seulement des guerres
maritimes). Le prix n'est pas modifié quand le pays est en guerre
car l'armée peut vivre partiellement sur le pays et parfois les soldats
sont réquisitionnés et ne peuvent repartir même en l'absence de
solde régulière.}

Le coût d'entretien des troupes n'est pas le même si le
pays majeur est en guerre ou en paix. Les tableaux d'entretien
indiquent les nouveaux coûts (pour vétérans et conscrits).
Lors d'une guerre commerciale ou une intervention limitée (défensive
ou offensive), le pays utilise le coût d'entretien en paix. Le
coût en guerre est réservé à une guerre complète. Noter que
certaines interventions dans des guerres provoquées par
des événements (guerres civiles en particulier), sont des
guerre complètes (du moment que l'intervention n'est pas
qualifiée de limitée).


\paragraph{d. Levée exceptionnelle.}
\textit{
L'épuisement des forces armées étant trop élévé, on donne la 
possibilité de lever des troupes durant la phase militaire.}
En cas de pénurie de troupes durant la phase militaire, un
pays peut procéder à une levée exceptionnel de forces
terrestres si il vient de subir une DEFAITE MAJEURE. 
Décider d'une levée exceptionnel se fait en toute 
fin d'un round (dans l'ordre l'initiative) et fait perdre 1 en stabilité
(sauf à la Suède et à la Prusse).

Cette levée est la poursuite de la conscription
à la phase logistique et respecte donc les limites d'achat de troupe
(et les coûts correspondant) en ajoutant les nouvelles troupes
à celles levées au début du tour. Une fois la levée exceptionnelle
décrétée, le pays peut continuer de lever des troupes à la fin de chaque
round (dans les limites du recrutement maximum) sans perdre
de stabilité supplémentaire.

\paragraph{f. Prêts exceptionnels.}
Pour faire face aux dépenses imprévues des rounds militaires,
il est possible de souscrire des prêts durant les phases militaires.
Ils se font aux mêmes conditions que les prêts de la phase logistique,
obéissent aux mêmes restrictions si ce n'est qu'on peut en faire
un à ce moment même si un premier a été souscrit pendant la
logistique. Un prêt demandé durant les phases militaires ou
en fin de tour ne peut en revanche pas être refusé, quelles que soient
les conditions obtenues.


\subsection{Le combat rapide}
-- NE PAS PRENDRE EN COMPTE --

Les forces armées terrestres sont dans ce système toujours évaluées en équivalent détachement 
pour toutes les fonctions du jeu~: combat mais aussi logistique, empilement
et attrition, en Europe comme dans le reste du monde.

Faisons un survol du combat rapide : il se déroule en 1 ou 2 round, sans avoir
à calculer des facteurs de Feu et de Choc. À 
chaque round, chaque joueur lance 1 dé pour le feu puis un pour le choc (\textit{si 
personne n'a craqué au moral après le feu}) sur la nouvelle table de combat ; la colonne y est 
determinée par la technologie (tables des aides de jeu). Les dommages sont 
évalués en nombre équivalent de détachements perdus par l'adversaire et en 
points de moral. 

Si personne ne craque à la fin du 1er round ni ne tente
de retraiter pendant la bataille (règle \ref{RetraiteEnBataille}), un second round 
a lieu de la même manière mais avec -1 aux jets de dés pour les deux protagonistes. 
L'armée vaincue subit une poursuite (toujours en colonne E)
\textit{même si elle n'a pas craqué au moral}.

On calcule ensuite le nombre de pertes effectivement subies par chaque côté 
(\textit{voir plus bas les modificateurs de taille d'armée à utiliser, \ref{PertesVariables}}). 
Le perdant peut subir des pertes supplémentaires en faisant un test sur la 
table de retraite. La totalité des pertes sont alors appliquées aux armées.

\subsubsection{Description des armées, moral, technologie}
-- NE PAS PRENDRE EN COMPTE --

Toute force militaire est maintenant ramenée à son contenu en détachement selon 
l'équivalence : 1 A- = 2D ; 1 A+ = 4D.
Les pachas turcs valent de 0 à 3 détachements selon ce qui est indiqué.
Les pions armées ont quand même une particularité importante : ils sont
les seuls à contenir de l'artillerie.

Le \textbf{moral} d'une pile est celui de la majorité des forces, comptée en équivalent détachement. 
En cas d'égalité on prend d'abord le moral de la majorité des pions armées. Une égalité
à nouveau donne un moral conscrit.
Cette procédure est aussi utilisée pour connaître la technologie
militaire d'une pile. La moins bonne est utilisée en cas d'égalité.

Chaque pion appartient à une certaine \textbf{classe} militaire 
qui décrit en quelque sorte l'évolution des diverses doctrines et tailles
des armées de l'époque et
décide de sa taille (de 0 à 7) en fonction de la période en cours.

La \textbf{taille} d'une pile militaire est donnée par la moyenne,
arrondie à l'inférieure, des tailles de chaque unité. Cette moyenne
est comptabilisée en équivalents détachements (donc une A+
compte pour 4 fois plus qu'un DT) et arrondie strictement à l'inférieur.

\textbf{Exception:} Si des pachas accompagnent l'armée, le moral est forcément
conscrit. Le restant (taille et technologie) est déterminé selon la règle normale.

\subsubsection{Séquence de la bataille}
-- NE PAS PRENDRE EN COMPTE --

\noindent
La bataille commence après les tentatives éventuelles d'interception, de
retraite avant combat, et les jets d'attrition des forces qui ont 
bougé. \\
{\bf A.} Les rounds de combat, simultanés. 
Si à la fin d'un des 4 rounds, une armée a craqué au moral (aussi appelé déroute, 
c'est-à-dire est arrivé à 0 ou moins au moral), \textit{passer directement en C.1} \\
1. Premier round de feu~: chaque camp lance un dé modifié sous la colonne de feu. On retient les
pertes faites par les deux camps. \\
2. Premier round de choc~: chaque camp lance un dé modifié sous la colonne de choc. On ajoute les
résultat aux pertes faites par chaque camp. \\
{\bf B.} Possibilité de rompre le combat, défenseur puis attaquant. Si les
pertes sont à ce moment suffisantes pour que, une fois modifiées par les pertes variables,
un camp soit éliminé, le combat cesse et on passe au C. \\
3. Second round de feu~: chaque camp lance un dé modifié sous la colonne de feu, avec -1 au dé. 
On ajoute le résultat aux pertes faites par chaque camp. \\
4. Second round de choc~: chaque camp lance un dé modifié sous la colonne de choc, avec -1 au dé. 
On ajoute le résultat aux pertes faites par chaque camp. \\
{\bf C.1} Si une armée a craqué au moral et pas l'autre, effectuer un jet de poursuite (colonne E). \\
{\bf C.2} Si une armée a moins de moral restant que l'autre mais n'est pas en déroute, elle perd le combat. 
\textit{Le vainqueur fait une poursuite qui peut causer une déroute}. \\
{\bf C.3} Si les deux armées ont le même moral final ou que les deux ont craqué au moral, 
chaque camp retourne d'où il vient, on ne fait pas de poursuite et personne ne gagne \\
{\bf D.} On totalise les pertes de chaque camp (des 4 rounds et la poursuite) qui sont modifiées 
en fonction de la taille de l'armée causant les pertes, 
\textit{ensuite de sa classe comparée à celle de
l'armée prenant les pertes}. \\
{\bf E.} Le perdant du combat (qui doit retraiter) fait un test d'attrition au cours de la retraite
qui peut accroître ses pertes de $1/2$ ou 1. La man{\oe}uvre du général est utilisée 
si la force n'est pas en déroute. \\
{\bf F.} Les pertes sont arrondies à l'entier inférieur (sauf 1/2 qui devient 1).\\
{\bf G.} Les tests de perte des généraux sont faits (règle usuelle). On regarde si
il y a eu bataille majeure. \\ 

\subsubsection{Pendant les rounds de combat}
-- NE PAS PRENDRE EN COMPTE --


\paragraph{1. Technologie et feu.}
- Une armée en Médiéval ne lance pas de jet de feu. \\
%- \textbf{version de Bertrand :} Une force en Renaissance utilise toujours la table de feu, même
%si il n'y a que des DT (selon la clarification des Q\&A). Elle ne fait que les pertes au moral \\
%- \textbf{version de Pierre :}  Une armée en Renaissance ne fait
%que les pertes de moral indiqués par le feu ; si il n'y a que des D, pas de test de feu
%(selon la règle de combat rapide de P. Thibaut).  \\
- En Renaissance, une force utilise le table de feu si elle contient des pions armées, ou bien
si elle n'a que des DT, quand elle combat des indigènes ou pays non européens en Médiéval.
Dans les deux cas elle ne fait que les pertes au moral. \\
- En Arquebuse, les pertes obtenues sur la table doivent être divisées par deux 
(arrondies à l'inférieur). \\
- Pour toutes les technologies après Arquebuse,  les pertes sont celles indiquées.

\paragraph{2. Modificateurs.}
Ils sont indiqués à côté de la table de combat (terrain; -1 au second round; 
effet des généraux).


\paragraph{3. Avantage de cavalerie.}
Chaque pile contient un nombre de cavalerie qui dépend de sa taille et 
de la quantité de cavalerie par équivalent détachement. Une valeur \textit{qui dépend 
de la nationalité de l'armée en question (voir ci-dessous)} est multipliée par 
le nombre d'équivalent de détachement et donne ainsi la quantité de cavalerie 
dans la force. Les DT contiennent autant de cavalerie  par détachement que les A.

Si un camp a au moins deux fois plus de cavalerie que son adversaire,
il a +1 au dé pour le choc et la poursuite si la bataille est en plaine 
(qu'il soit défenseur ou attaquant), désert, ou dans les forêts orientales (voir
\ref{foretOrientale}, pour certaines technologies seulement).


\paragraph{4. Poursuite.}
Les jets de poursuite sont affectés par le différentiel de choc,
le terrain, l'avantage de cavalerie et la condition à la fin du combat : \\
+1 si l'adversaire a craqué au feu, \\
+2 si l'adversaire a craqué à un des 2 premiers rounds (cumulable avec le précédent).

\paragraph{5. Rompre le combat.}\label{RetraiteEnBataille}
Lors du segment B. de la bataille, entre les 2 premiers rounds de feu puis choc et les
deux derniers, une armée peut décider de rompre le combat. 
Le défenseur a la possibilité de le faire et, si il décline ou échoue, l'attaquant
peut le tenter.

Un jet de dé inférieur à la man{\oe}uvre du général plus le moral restant à l'armée
permet de finir la bataille tout de suite ;
celui qui rompt le combat est le perdant. On finit le combat par le segment  C.2
et les suivants. Si le test est échoué, le combat continuera et l'adversaire a un
bonus de +2 à son jet de feu subséquent.


\subsubsection{Variation des pertes}
-- NE PAS PRENDRE EN COMPTE --

\label{PertesVariables}
Le résultat des pertes est le total de ce qui est fait aux différents tests de feu,
choc et éventuellement poursuite (mais sans la retraite) donnant un nombre d'équivalent 
détachement encaissé par 
l'armée adversaire. Cependant la table est prévue pour donner le nombre de 
pertes faites par une pile de 2 A+ à une armée de même taille. Notez 
qu'avoir plus de 8 DT dans une pile (pour la Turquie avec les pachas) ne 
donne aucun avantage : pas de pertes supplémentaires (à la différence
du traitement des indigènes, voir \textit{infra}).

{\bf 1.} Pour tenir compte de la taille réelle de l'armée, on consulte la table des 
pertes variables (voir tables de combat) qui indique combien de perte enlever pour obtenir le nombre 
de pertes final. \textbf{On applique une limitation importante à ce stade :
le total des 
perte ne peut être supérieur à la taille de l'armée causant les pertes, comptée 
en équivalent détachement}. 

{\bf 2.}
\textit{On compare ensuite le type de chaque armée.} Il y a 5 groupe d'armée qui sont 
les suivants, leur taille étant indiquée pour les sept périodes (la répartition précise est
indiquée dans les annexes pour les mineurs et sur les tableaux des majeurs) :


Le tableau suivant (à droite) permet alors de déterminer le différentiel selon la taille de 
chaque armée. On a mis en caractère gras les lignes et colonnes qui servent 
usuellement, en caractères normaux celles où des armées
de tailles différentes sont mélangées (colonne 1, 5 et 6).
%Le différentiel est variable selon les périodes ; après ce chiffre 
%sont indiquées les périodes de validité du modificateur. 
L'armée qui subit les 
dommages est prise en ordonnée sur une colonne, celle qui les inflige sur une
ligne ; le tableau est 
symétrique avec un changement de signe par rapport à la diagonale.


\textit{Algorithme : diviser la différence de taille entre l'armée la plus grande et la plus
petite par 3 et arrondir au modificateur le plus proche pour obtenir le +?
accordé à l'armée de taille plus grande.} 


{\bf 3.}
Les pertes véritablement infligées sont alors celles données par le tableau 
ci-dessus, la ligne 0 correspondant au nombre de perte calculé à l'étape A, 
avant le modificateur 
dû à la comparaison des classes d'armée.


{\bf 4.}
On ajoute à la valeur obtenue le nombre de pertes données par la table de 
retraite (qui n'est pas modifié donc par le point {\bf 3}).
Les pertes obtenue sont arrondies à l'unité inférieure (sauf \f\ qui donne 1)
et donnent la valeur en équivalent détachement du nombre de pertes effectuées.


\subsubsection{Qui gagne le combat}
-- NE PAS PRENDRE EN COMPTE --


Les différentes issues du combat sont données dans la séquence des batailles et
détaillées ici.
%\newpage\null%\newpage

\textbf{Le vainqueur de la bataille.} 

{\bf C.1} Si une armée seule armée craque au moral (arrive à 0 ou moins) et pas l'autre à la fin
d'un round, l'adversaire gagne le combat. Il effectue une poursuite, on ajuste les
pertes. Le perdant recule dans une zone amie adjacente et fait un test d'attrition
sans soustraire la man{\oe}uvre du général. Les PVs normaux sont accordés.

{\bf C.2} Si aucune armée n'a craqué au moral après les 4 rounds, l'armée qui a le moins de moral restant perd 
le combat. \textit{Le vainqueur fait une poursuite.} Les pertes dues à la poursuite 
peuvent entraîner une déroute du perdant, en quel cas la fin de la procédure est la même
que C.1. Autrement, le perdant recule dans une zone amie adjacente et fait un test d'attrition
modifié par la man{\oe}uvre du général. Des PVs réduits de moitié sont accordés.

{\bf C.3} Si les deux armées ont le même moral final, ou si les deux ont dérouté, 
chaque camp retourne d'où il vient. C'est-à-dire qu'un siège continue à être maintenu, qu'une
armée qui vient de se déplacer ou d'intercepter retourne dans la zone où elle était juste
avant le combat. Il n'y a pas de poursuite; IL Y A ATTRITION de retraite et personne ne gagne
ni ne marque de PV.\\   

Une \textbf{victoire majeure} est accordée si le perdant a effectivement perdu 
\textbf{3} détachements
de plus que le vainqueur (après modifications de la classe, retraite et arrondis),
ou \textbf{4} DT si le perdant avait un modificateur de comparaison de taille
égal à -2.

Les pertes sont réparties par celui qui les subit comme il le veut parmi ses forces.
Il peut détruire des pions armées (une A+ = 4D) par exemple ou faire tout son
possible pour en garder (par exemple 2 A+ subissant 4 pertes peuvent rester sous
forme de 1 A+ ou 2 A- ou encore 4D -- ce qui poserait des problèmes d'empilement...)

\subsection{Les sièges}
-- NE PAS PRENDRE EN COMPTE --

%Les règles ne sont quasiment pas changées par rapport au combat rapide de la 2nde extension.

\subsubsection{L'assaut}
-- NE PAS PRENDRE EN COMPTE --


\paragraph{Les rounds d'assaut.}
L'assaut se fait en deux jets, un de feu puis un de choc sauf que 
le choc n'est pas fait par un camp qui a craqué au moral .
Les tables de combat montrent une colonne spécifique à l'assiégé et une pour l'assiégeant.
Noter que l'assiégé fait une perte en moins au feu et au choc si le combat est 
suite à une \textbf{brèche}.


\paragraph{Modificateurs.}
L'assiégeant ajoute 1 si le défenseur est médiéval, soustrait 1 si le défenseur est en arquebuse 
ou mieux, à son feu et son choc.
L'assiégeant soustrait aussi le niveau de la forteresse aux deux si il n'y a pas eu de
brèche. Enfin, l'artillerie ajoute  +1 en assaut si l'assiégeant a au moins
4 fois le niveau de la forteresse en artillerie (sauf contre un fort).

\paragraph{Les ajustements aux pertes.} 
\begin{itemize}
\item 1- si l'assiégeant n'a pas 2 A+, le tableau des pertes variables réduit
ce qu'il inflige ;
\item 2- la Turquie et la Russie jusqu'en 1614, et la Pologne jusqu'en 1559 augmentent
les pertes faites en assaut de 1/2 par A+ présente ;
\item 3- l'assiégeant prend une demie-perte en plus si il a craqué au moral.
\item 4- les pertes de l'assiégeant sont limitées au nombre de DT dans la
fortification plus 2 fois la résistance de
la forteresse (ajustée par la brèche).  
\end{itemize}

\paragraph{Résistance de la forteresse.}
Les pertes faites à l'assiégé sont d'abord prises sur les unités enfermées dans
la forteresse, puis sur la résistance de celle-ci. Cette résistance est égale à
son niveau, mais est réduite en cas de brèche. Elle revient à son niveau
maximum après chaque assaut.

\paragraph{La victoire.} Elle revient au camp selon l'ordre de priorité suivant~:
\begin{enumerate}
\item Assiégé, si l'assiégeant est éliminé ;
\item Assiégeant, si les troupes à l'intérieur sont éliminées et la résistance atteint 0, 
ou si l'assiégé craque au moral 
(même si l'assiégeant déroute) ;
\item Assiégé, si l'assiégeant seul ou si personne ne craque au moral.
\end{enumerate}


\subsubsection{La sape}
-- NE PAS PRENDRE EN COMPTE --

\paragraph{Mettre le siège.}
%\begin{minipage}[b]{0.5\linewidth}
Le siège par usure n'est presque pas modifié. Un pion armée  contient toujours 
un nombre d'artillerie égal à celui de la contenance maximum de sa nation à la période en cours.
Une armée sur la face - contient l'artillerie de l'armée + divisée par 2 et arrondie
à l'inférieur.

Il faut pour maintenir le siège devant une forteresse disposer d'au moins autant 
d'équivalent détachement que le niveau de la forteresse.
Si l'assiégeant ne peut maintenir le siège en fin de round (après un assaut ou une
bataille), il doit immédiatement retraiter dans une province amie (avant de pouvoir
piller) et jouer l'attrition. 
Si il choisit de maintenir le siège, il doit soit lancer un assaut, soit faire
un test dur la table de sape (qui peut être suivi d'un assaut en cas de brèche).

Le propriétaire de la forteresse peut laisser des troupes dans celle-ci.
L'empilement dans une forteresse est d'au plus 2DT par niveau de la
forteresse, ou d'un DT dans les forts. Ces forces subissent une attrition
à chaque fin de phase de mouvement si le siège est déjà
établi. Une fois enfermés dans une forteresse, une force ne peut 
en sortir qu'en fin de siège (victorieux ou non) et n'a pas le
droit s'attaquer les assiégeants.

\paragraph{Résolution de la sape.} On utilise la table des annexes,
avec les modificateurs indiqués.

Les pertes assiégeantes obtenues sur la table des sièges se résolvent
en lançant 1d10, diminué des valeurs en siège des généraux 
et augmenté de 1 par DT (ou équivalent) en défense dans la forteresse. 
Si le résultat est inférieur (strictement) au nombre de round
de siège écoulé, l'assiégeant doit faire un test d'attrition sur la table 
adéquante (Europe ou non) avec les modificateurs indiqués. 

\subsubsection{Prise des forteresse}
Une forteresse qui tombe par assaut ou sape perd 2 niveaux de
fortification (avec la valeur mise sur la carte en tant que minimum),
sauf si le nouvel occupant décide immédiatement de mettre un
garnison. Il doit pour cela utiliser un DT qui est perdu (le DT peut
provenir de la séparation d'un pion armée).

\subsubsection{Généraux ingénieurs}
-- NE PAS PRENDRE EN COMPTE --

Un petit nombre de généraux sont en fait des ingénieurs militaires :
Vauban, Coehoorn et Dahlberg. Ils n'ont pas à respecter la
hiérarchie militaire si le joueur le souhaite (et peuvent ainsi autant
aider à un siège que rester seuls dans une forteresse pour
la défendre) mais sont alors restreints aux actions de siège.
Il peuvent être utilisés comme des généraux mais entrent
alors dans la hiérarchie.

%\end{minipage}
%\begin{minipage}[t]{0.5\linewidth}
%\artillerie
%\end{minipage}


\subsubsection{La cavalerie et les forêts orientales}\label{foretOrientale}
-- NE PAS PRENDRE EN COMPTE --

\textit{
L'usage de la cavalerie en Europe orientale n'est pas entièrement
simulée dans le jeu. En effet celle-ci domina pendant la période
1500-1600, en particulier la cavalerie polonaise (hussards), grâce à ses
capacités de choc, sa mobilité et surtout la possibilité de contrôler
les larges espaces de la région et de s'y ravitailler alors que le
peuplement n'était qu'éparse. Les changements suivants sont proposés.}

\paragraph{a. Étendue des forêts orientales.}
Ce sont toutes les forêts à l'est de la ligne Neumark, Lausitz,
\"Osterreich, Steienmark (provinces comprises), au nord de
Transylvania, Marosz, Moldavia (provinces comprises aussi)
puis ligne plein est jusqu'à la carte. Les provinces de la Finlande
ne sont pas dans la zone, ni celles de la Scandinavie.


\paragraph{b. Supériorité de cavalerie.}
Le bonus de supériorité de cavalerie en combat est rétabli
dans les zones de forêt d'europe orientale pour certaines 
technologies.

Les pays de technologie latine ou orthodoxe peuvent  en bénéficier dans ces forêts si 
leur technologie est Arquebuse, Mousquet ou Baroque.
Si l'adversaire est de technologie Man\oe uvre ou Dentelles, tout avantage
de cavalerie est annulé.

Les pays musulmans n'ont jamais cet avantage du fait de leur cavalerie
plus légère. Leur propre avantage réside dans le nombre de cavalerie
présent dans leur armée.

\paragraph{c. Siège.} 
Une force pouvant bénéficier du bonus précédent et
qui assiège en forêt orientale avec un nombre 
de cavalerie supérieur ou égal à 8,
réduit le modificateur de mauvais terrain à -1 au lieu de -2. 
De plus elle n'a pas le malus de terrain non clair pour
tester le résultat "pertes assiégeantes".

\subsubsection{Le militaire et la mer Baltique}
-- NE PAS PRENDRE EN COMPTE --


\paragraph{a. Les galères.}
\textit{
Les galères qui agissent en mer Baltique n'eurent jamais une domination
nette face aux navires de ligne comme cela put être en méditerranée.
En fait les galères sont plus souvent des navires légers ne pouvant pas
tenir des sorties océaniques longues, sans être techniquement des galères.
Cela a permis la constitution de flottes pas trop chères et d'efficacité
convenable à certains pays mais pas plus.}

Avant 1615, les galères en mer Baltique ne bénéficient pas du bonus en combat 
contre les vaisseaux de ligne de la \textit{règle 53.7}.

\paragraph{b. Gel de la mer Baltique.}
En cas d'événement de mauvais temps, si le jet de dé est 1 le détroit du Sund est
gelé pour le round et autorise le passage d'armées entre la province Danmark et 
les provinces de Skäne, Schleswig,
Lübeck et Jutland. Le passage est considéré un mouvement en terrain difficile mais
sans détroit (y compris pour l'effet sur le combat).
La province maritime "Sund" est impassable en cas de gel
(et les flottes qui y sont présentent restent bloquées) ;
une flotte à Copenhague ne peut sortir que en mer Baltique.

 
\subsection{Diverses règles militaires}
-- NE PAS PRENDRE EN COMPTE --


\paragraph{a. \textit{53.22} Les monarques au combat.}

Le tableau de la règle \textit{53.22} est remplacé la procédure
qui suit. On tire d'abord la valeur moyenne comme général du souverain
sur la 1\up{ère} partie de la table. Seule cette valeur moyenne est initialement
connue. À la première bataille que commande le souverain, on modifie 
séparément chaque valeur
par le jet de 1d10 sur la table en-dessous ; avant cela on utilise
la valeur moyenne comme man{\oe}uvre.  
La compétence en siège est de 0, ou 1 si la valeur moyenne est de 4 ou 5, modifiée par 
la seconde étape qui est tirée lors de sa première action de sape. 
Le modificateur pour un roi adolescent s'ajoute après la détermination des valeurs. 

\paragraph{b. \textit{53.21} Inondation des provinces hollandaises.}
L'inondation touche les provinces de Friesland, Utrecht, Holland,
Overijssel, Gelderland. 
Elle est annoncée à la phase de mouvement de l'ennemi et toute armée hostile
dans la zone doit retraiter (qu'elle vienne d'entrer ou non).


\paragraph{c. Les Croisades.}
Lors d'une croisade (guerre à l'appel du pape), deux pions armées Croisés
sont disponibles qui peuvent être constitués de détachement de n'importe
quels pays en guerre. Ils ont la contenance générique des pays d'occident (la
même que ceux d'Italie et du Saint-Empire, classe \CAIII) et de technologie
des latins.
La composition exacte de ces armées doit etre notée. Leur utilisation est
entièrement confiée au chef de la croisade (y compris pour la
répartition des pertes au combat).

\paragraph{d. Les amiraux corsaires.}
Les corsaires (dont le symbole est en blanc) de rang A, B ou C
sont en fait aussi des amiraux et peuvent commander de flottes.
Si ils ne sont pas mis avec
des pions corsaires, ils font partie de la hiérarchie des amiraux.
Les corsaires de rang inférieurs ne peuvent commander que des
pions corsaires et des DN.
\textit{Les nouveaux pions remplacent tous les corsaires affectés
par des amiraux corsaires et permettent de se passer de cette règle.}

\paragraph{e. Les alliés indigènes.}\label{chMilitary:Movement:Indian forces}
Les alliés indigènes et les troupes du mineur Iroquois ont les capacités particulières
suivantes :
\begin{itemize}
\item ils sont toujours ravitaillés dans leur province d'origine et jusqu'à 12 PM de 
celle-ci;
\item on considère qu'ils donnent une man{\oe}uvre de 5 pour le mouvement à la pile
qu'ils accompagnent ;
\item si ils sont seuls en attaque, ils ne tiennent pas compte du terrain.
\end{itemize}

\paragraph{f. \textit{53.14} Les Cosaques.}
Quand la Russie bénéficie de l'aide des Cosaques, elle applique les deux
effets suivants :
\begin{itemize}
\item entretien gratuit de 2DT conscrits en Sibérie ;
\item achat gratuit à chaque tour d'un DT en Sibérie ou dans une province des Cosaques.
\end{itemize}


\subsection{Partie navale}
-- NE PAS PRENDRE EN COMPTE --


\emph{La gestion des forces navales est maintenant dans ces règles simplifiée
de façon à ne comptabiliser que des détachements navals
et des flottes (et non des navires individuels) et à utliser le système
de combat accéléré.}

%\textbf{Ces ajustements en sont à une phase de test seulement,
%et remplacent certains des paragraphes précédents : la table d'attrition
%de 2.4.2, les paragraphes 2.4.4 et 2.6
%en entier.}

\subsubsection{Description des unités navales}
-- NE PAS PRENDRE EN COMPTE --

Les forces maritimes sont représentées par des \textbf{détachements navals},
notés \textbf{pions DN}, et appelés dans les règles
\textbf{DNav} si ils contiennent des navires de guerre à voile, 
{\bf DGa} si ils
contiennent des galères. Pour Venise, {\bf DGal} indiquent spécifiquement des
galéasses. Elle dispose en effet à partir de 1550 d'au plus 
2 DGal (soit dans des flottes, soit
représentées par un pion DN) qui se comportent comme des galères, sauf
en combat.
Les pions détachements navals peuvent ainsi représenter des navires
à voile ou des galères et ceci est à noter sur la feuille de marque. 
\medskip 

Un nouveau type de pions, notés
{\bf DTr}, représentent des \textbf{vaisseaux de
transport}. Les pays mineurs (autres que ceux qui peuvent devenir ou ont été majeur)
disposent tous d'un pion DTr qui peut leur servir à représenter
un détachement de transports seulement en plus de ceux dans les flottes.
Les pays majeurs en ont 4 (ESP, POR, FRA, ANG, HOL, TUR)
ou 2 (les autres).

La flottes de l'Or, le convoi de Smyrne et le convoi des Indes Orientales
sont tous des flottes de transport : ils contiennent chacun
5 DTr. Les convois de Smyrne et des Indes portent 10d par DTr, et
la Flota de Oro voie l'Or transportée dans sa capacité de transport
d'Or illimitée répartie également entre les 5 DTr.
\medskip

Les explorations 
maritimes hors de l'Europe peut se faire avec des escadres réduites
qui sont appelées de \textbf{détachements d'exploration}, ou {\bf DE}, qui sont
en gros un tiers ou un demi DN (selon le cas)\footnote{
Les DE sont comptés comme un demi en général, sauf dans les tables
d'attrition et de pertes en combat où il sont un tiers des DN, et
pour reformer un DN à partir de DE sur la carte où l'équivalence est
de 3 DE pour un DN}. Ils servent aussi pour fractionner l'attrition
et les pertes en combat des navires de guerre et galères.  
Il est ainsi rajouté des pions DE dont le nombre maximal dépend du pays (les
mineurs ont 2 DE au plus sauf Oman et Aden qui en ont 4 ; 
pour les pays majeurs voir la table
des limites navales). Les pions DE contiennent en général des navires
à voile (pour les explorations maritimes), sauf mention explicite sur la feuille 
de marque quand il s'agit de galères (après attrition ou combat) \footnote{Les
pions DE seront faits avec deux faces : l'une représent les DE, l'autre
sert aux DC, voir plus bas le paragraphe sur les explorations.}. 

\textbf{Consolidation -} Si 3 DE d'un même pays sont dans la même pile navale (après 
un combat ou des pertes par attrition, même si ils sont dans des pions F différents), 
ils sont immédiatement fondus en un DN si
un pion est disponible. Il n'est pas possible de volontairement
couper un DN en détachements d'exploration (mais on pourra acheter
des DE séparément).
\medskip

Tous ces détachements peuvent être réunis dans des \textbf{flottes}, pions notés F, 
dont la contenance
(face - ou +) varie selon le pays et la période (de 2DN/1DTr à 7DN/2DTr).
Un DGa (ou DGal) ou un DE n'occupe dans une flotte la place que d'un demi-DN dans
les pions de flotte F.
Les détachements qui sont dans des pions de flottes
n'utilisent pas de pions. Il faut noter sur la feuille militaire ce que contient
exactement la flotte comme on note la nature des pions DN (vaisseaux,
galères, transports, galéasses ou détachements d'exploration). 
Voir la \textbf{table de contenance des pions flottes} selon le pays.
\medskip

La \textbf{limite d'empilement} dans chaque pile navale 
est de deux pions plus un DE à tout moment
la phase de mouvement (avant les combats). Des piles
navales séparées d'un même camp peuvent  être dans la même
mer (pour ainsi faire des
actions séparées). Un port ne peut contenir qu'un seule pile.

%\contenanceflottes
%TABLE des contenances des flottes

\subsubsection{Logistique des forces navales}
-- NE PAS PRENDRE EN COMPTE --

Les forces navales sont entretenues par pion DN, F- ou F+. L'entretien 
a un coût indiqué dans les tableaux de chaque pays. On peut
utiliser l'équivalence suivante pour la force gratuite de base : D entretient 2 DE et
2D permettent l'entretien de F-, 4D de F+ (l'équivalence inverse n'est pas possible).
En revanche un pion posé doit être entretenu sans lui appliquer cette 
équivalence (F+ ne peut être compté comme 2 F-). Les DTr s'entretiennent comme des
DN (de même que DNav, DGa ou DGal s'entretiennent de la même façon).

L'achat se fait par DN, les coûts étant différents entre les DNav,
les DGal et les DTr. Il est aussi possible d'acheter des flottes entières F+
ou remplies à moitié F-, au coût indiqué (en général réduit). Enfin
des DE peuvent être achetées au prix de la moitié d'un DN du type
de navire choisi.

On verra ci-dessous la \textbf{table des limites d'achat par tour et du maximum
de DN par période}. Les DTr ne comptent pas dans ces deux limites ; les DGa si mais
comme 1/2 DN seulement.
Pour le nombre de navires, les DE dans les flottes sont arrondis au DN (ou DGa) 
supérieur, tandis que les DE qui ne sont pas dans les flottes sont limités par
le nombre de pions disponibles. À l'achat, un DE compte pour 1/2 DN (pour
le coût et la limite d'achat par tour).

\paragraph{Renforts des pays mineurs.}
La table des renforts navals est modifiée de manière à convertir
les navires reçus en DN ou DE. 2 ou 3N correspondent à DE ;
5 ou 10N à DN ; 15N à F- ; 20N à 3DN et DTr ; 30N à 4DN et DTr.


\subsubsection{Clarification sur les mouvements}
Une pile navale dispose d'un potentiel de mouvement illimité 
pendant un round et peut ramasser ou laisser des détachements en
route librement. Des forces laissées ou ramassées ne peuvent faire ou avoir
fait d'autres mouvements du round. Une pile est restreinte de deux 
manières durant la phase militaire. Premièrement elle ne peut faire
que l'une des activités suivantes par round : transport naval 
(c'est-à-dire débarquement de troupes, accompagné 
d'un blocus de la province de débarquement et
du soutien de ravitaillement des forces débarquées -- on peut
embarquer librement des troupes pour les transporter et faire une
des actions décrites ici librement si on ne les décharge pas ;
voir plus bas l'exception des conquistadors), mise en
blocus d'un port (qui peut ne s'annoncer qu'à la phase des
sièges, avant tout test de siège), lutte contre les pirates d'une zone commerciale,
exploration d'une nouvelle mer, attaque d'une pile navale ennemie en mer.

Deuxièment elle doit faire un test d'attrition à la fin de son mouvement
ainsi qu'à chaque combat ou au moment de débarquer des troupes, 
ce qui cesse même temporairement son mouvement. La flotte
peut donc être conduite à faire plusieurs attritions dans un même tour 
et le nombre de provinces traversées est celui entre deux de ces tests
d'attrition.

L'empilement est limitée dans chaque pile navale à deux pions plus un DE,
et non dans chaque mer. Cependant seule une pile navale d'un camp est
autorisée à faire une action donnée dans une mer (c'est-à-dire lutter contre
pirates et corsaires, ou attaquer une pile ennemie spécifique, ou encore
mettre le blocus à un port précisé). Deux piles maritimes peuvent tout
de même mettre un blocus de manière indépendante à deux ports sur la 
même zone.

\paragraph{Interception.}
Durant les mouvements, les flottes ennemies peuvent tenter d'intercepter dans
leur mer ou la mer adjacente la pile maritime qui bouge, sans limite du
nombre de tentative. La force en mouvement peut tenter d'intercepter
une fois (et une seule) toute pile maritime en mer en étant dans sa zone, afin de
forcer le combat tout de suite. On résout d'abord les interceptions
de forces inactives avant de tenter celles de la pile active.
En cas de réussite, on fait immédiatement les tests
d'attrition des flottes (celle active, et celle inactive si c'est elle
qui intercepte). La flotte active pourra continuer son mouvement
après interception si elle gagne la bataille. Une flotte défaite ne
peut plus intercepter pendant jusqu'à début de sa prochaine phase de
mouvement.

De toute façon, et sans test d'interception,
le combat devient possible et automatique contre une force maritime
dans la même mer après les déplacements (mais ceci compte alors comme
l'action de la force navale pour le round).

\subsubsection{Transport maritime}
-- NE PAS PRENDRE EN COMPTE --

La \textbf{table de transport maritime}
indique le nombre de points de transport nécessaires pour embarquer un DT
(il en faut 2) ou une A- (de 3 à 6 selon la classe d'armée). Une A+
compte comme deux A-. Les DN (DNav, DGa ou DGal)  transportent 1 point
chacun et les DTr transportent 3 points (c'est même
leur unique fonction !). Les DE transportent
seulement 1/2 points. Les DC (conquistadors) ne demandent eux que 1/2 point  
de transport.

Autre limite, un pion naval ne peut en général transporter qu'un seul pion
terrestre. Pour être plus précis, un pion F de navires peut transporter jusqu'à une A+
(soit l'équivalent de 4 DT, éventuellement en 2 pions, mais pas 3) ; 
si le transport est assuré par des DGa et des DTr dans F, la limite 
passe à 2A+ en 3 pions. Un détachement
naval (de toute nature : DTr, DN, DE) ne peut contenir plus 
d'un pion (donc DC, DT ou A- pour les armées les plus petites).

Une force navale doit faire un test d'attrition au moment où elle débarque
des forces. Si elle obtient des pertes par attrition, le
même pourcentage de perte est appliqué aux DT ou armées transportées
en consultant la table des forces restantes après attrition. En combat maritime,
les pertes n'affectent les armées que quand il n'y a plus assez de navires 
intacts pour transporter les troupes (les forces terrestres en trop sont détruites
tout de suite).

%\transportflottes

\subsubsection section{Attrition en mer}
-- NE PAS PRENDRE EN COMPTE --

Les flottes qui se déplacent ou restent en mer encourent un risque d'attrition.
La procédure est peu changée par rapport à la règle normale.

Le test d'attrition consiste à lancer 1d10 auquel on ajoute la valeur
de risque la plus haute des mers pénétrées par la force navale.
Les valeurs de risque sont diminuées de 2 si un port ami est
sur la zone. On ajoute de plus la valeur des modifcateurs de
toutes les mers difficiles traversées (valeur en bleu), un malus
de +1 par groupe (plein) de 4 zones de mouvement (on ignore les
fractions), et
un malus éventuel d'un événement (+2 si mauvais
temps). On soustrait d'un autre côté
la man\oe uvre d'un amiral (ou un explorateur si le mouvement 
commence ou se termine hors Europe). Le résultat est comparé à 11 :
chaque point du résultat au-delà de 11 induit 10\% 
d'attrition. Ce résultat peut être lu sur la table d'exploration et attrition hors-Europe.

%
%on fait un test sous la valeur de risque la plus haute de celles
%des mers (ou entrées). Les valeurs de risque sont \textbf{diminuées de 2
%si un port ami} est sur la mer.  Le dé est modifié de +1 si les navires
%traversent 2 ou plus région ayant cette valeur de risque, et des 
%modificateurs en bleu pour certaines mers au passage difficile. La
%man\oe uvre d'un amiral (ou si le mouvement commence ou fini hors-europe
%d'un explorateur) est soustraite du dé.

Ensuite, le pourcentage obtenu est converti en pertes à l'aide de
la \textbf{table des attritions pour l'exploration et le mouvement maritime}.
Ce tableau indique, selon la perte et le nombre de détachements présents,
ce qu'il reste dans la pile après les pertes d'attrition. Un résultat * (dans la table 
d'attrition ou le tableau de ce qui reste) indique une chance sur deux de perdre un DE
(sur mer) ou DC (sur terre).

Les transports ne disposent pas de pions fractionnaires DE comme les DN.
Si un détachement de transports doit prendre au moins DE de perte, 
il est détruit en entier.


\subsubsection{Bataille navales rapides}
-- NE PAS PRENDRE EN COMPTE --

\paragraph{a. Séquence de bataille.}
Les batailles navales sont résolues sur un système de combat accéléré
qui utilise la table des résultats des batailles rapides. Le combat peut
se poursuivre sur plusieurs jours selon la séquence suivante.
\begin{enumerate}
\item Décider du type de navires (Nav, Ga ou Tr) mis en avant (et en déduire le moral,
les colonnes utilisées, et les modificateurs pour l'avantage du vent) ; ils
doivent constituer au moins 1/4 des détachements de la pile.
\item Déterminer l'avantage du vent (sauf dans un combat de galères
contre galères).
\item Segment de feu ; noter les pertes. Elles sont réduites de moitié
si des Ga sont en 1e ligne. Si un camp craque au moral,
aller directement en 7.
\item Retraite optionnel du camp sous le vent, sans poursuite (mais avec 
suivi et attrition). 
\item Segment d'abordage ; noter les pertes.
\item Ajouter les pertes de 3 et 5 et les ajuster en fonction du nombre de DN 
présents. Les retirer des forces navales.
\item Si un camp est à 0 en moral, il rompt le combat et se réfugie au port
(un des ports amis les plus proches), avec poursuite si l'autre camp n'a
pas craqué au moral, et suivi éventuel pour établir un blocus.
\item Si personne n'a craqué au moral, les deux joueurs choisissent en secret
de rester ou de se replier au port. Si seule une force se replie, l'autre peut suivre
au port pour mettre le blocus. Si les deux forces restent, reprendre une journée
de combat au segment 2 en utilisant les valeurs de moral restant
(contrairement aux batailles terrestres, le combat n'est pas limité à deux
journées, le modificateur -1 s'appliquant à partir du 2\up{e} jour).
Si jamais les DN du type de navires mis en avant sont tous détruits, il
faut choisir un nouveau type et le moral est le minimum entre celui après la journée
de combat et celui du nouveau type de navire en 1e ligne.
\end{enumerate}

\paragraph{b. Effet des pertes.}
Les pertes obtenues en 3 et 5 sont ajoutées et modifiées alors
par un pourcentage dépendant du nombre de DN dans la flotte
qui fait les pertes (les DE comptent comme une moitié de DN mais
l'arrondi est fait vers le bas). Voir le \textbf{tableau des modificateurs des
taille de flottes}. On arrondi les fractions du résultat au demi supérieur,
sauf si le résultat est $\le 0,3$ qui est réduit à 0. 

Ensuite ces pertes sont réparties entre des détachements coulés,
immobilisés (reviennent au tour suivant dans un port ami à
décider immédiatement) et endommagés (reviennent au round suivant dans cette
flotte) en consultant la \textbf{table de répartition des pertes navales}.
Si les pertes sont plus que 3 D, utiliser plusieurs fois la table pour chaque
tranche de 3 D et le reste. 
Les DE se combinent ici selon l'équivalence 1DN=3DE. 

Les pertes affectent d'abord les détachements des navires mis en avant,
ensuite sur les autres navires.


\paragraph{c. Poursuite ou suivi au port.} La poursuite est un jet de perte
sur la colonne E. Les pertes sont modifiées par la taille de la force navale
qui poursuit (après les pertes du combat) et se répartissent pour moitié (arrondie
à l'inférieur) comme des vaisseaux capturés et endommagés (reviennent rou
round suivant dans un port ami au choix), et le restant comme des vaisseaux
coulés.

Ensuite une force navale est parfois autorisée à suivre son adversaire qui
retraite pour mettre le blocus devant le port atteint. Elle peut ne suivre qu'avec
une partie de la flotte (qui cesse son mouvement en blocus, après attrition
pour le déplacement) tandis que le restant de la flotte continue son
mouvement si le combat résultait d'une interception, ou fera une autre
action navale dans la mer ensuite (blocus, lutte contre des pirates, etc).

\paragraph{d. Galères et galéasses.} Les DGa et DGal comptent en empilement
dans les pions F comme un demi-DN. Ainsi les flottes de galères sont en général
plus importantes. Cependant elle font des pertes réduites de moitié à l'étape 3
de feu (arrondir les fractions 1/4 vers le bas), ceci avant de les modifier par la taille.

Venise peut avoir jusqu'à deux détachements de galéasses. La présence
d'un détachement de galéasses en combat contre des galères (uniquement) permet d'utiliser 
les pertes complètes au feu ; la présence de 2 DGal fait que le feu se fait
en plus avec +1 au dé de feu.
Contre des vaisseaux, les galéasses combattent comme des galères.

\paragraph{e. Transports en combat.}
Les DTr ne comptent pas du tout en combat maritime et prennent des pertes si les 
navires de guerre qui les accompagnent sont déjà perdus.
Des DTr mis en avant au début d'un jour de combat rompent toujours au moral
après le feu quel que soit le résultat adverse et ne font pas de pertes (même si des
navires de guerre sont en retrait).

%%%%%%%%%%%%%%

\subsection{Explorations et découvertes}
-- NE PAS PRENDRE EN COMPTE --

\subsubsection{Les détachements de conquistadors}
Les règles précédentes s'appliquent entièrement aux unités qui sont sur la carte hors europe,
sauf quelques ajustements pour les détachements se déplaçant seuls hors-europe et pour les indigènes.
L'ajustement principal est qu'il existe des \textbf{détachements de conquistadors},
notés {\bf DC}, utilisables
uniquement hors europe. Les DC sont les versos des pions DE.

Un DC s'entretient au coût de la moitié d'un DT conscrit (arrondi au supérieur) et
occupe 1/2 DT en transport maritime. Il compte comme un pion pour
l'empilement dans une case. Si 3 DC sont empilés ensemble, ils peuvent
être regroupés en un DT (conscrit). Les DC sont toujours comptés comme des troupes
conscrites \textit{(simule le faible nombre de cavalerie en général)}.  
\textbf{Exceptions :}
les DC de tercios espagnols et les DC portugais qui sont toujours vétérans et
se combinent donc en DT vétérans.

Si des DC apparaissent dans un transport maritime du fait de l'attrition, on arrondit
lors d'un débarquement en Europe 2DC à 1DT et DC à rien immédiatement. Si le débarquement 
est dans les colonies, il n'y a pas d'arrondi (mais il faut veiller à respecter les limites
d'empilement).


\subsubsection{Mouvement, exploration, attrition hors europe}
-- NE PAS PRENDRE EN COMPTE --

La même table sert à résoudre attritions et explorations et une flotte
ne teste pas l'attrition en plus du test d'exploration. Cependant,
les explorations ne nécessitent pas de rajouter la difficulté des
provinces traversées (on lit sur la première colonne, qui correspond en
fait à une difficulté de 10), mais on ajoute tout de même 1 par groupe de 
4 mers pénétrées pour l'exploration maritime.
Les attritions dues aux explorations 
sont transcrites en pertes en équivalent détachement à satisfaire. La table
est la même que pour l'attrition maritime, voir \textbf{table des restes après
attrittion pour exploration}. Les ``de'' indiqués indiquent bien sûr des DC.

L'attrition hors europe sans faire d'exploration est jouée sur la même table
que l'attrition en mer, avec une difficulté des provinces de 8 si la pile ne passe
que par des provinces explorées, ou 6 si toutes les provinces (de celle du
départ à celle d'arrivée) sont explorées et amies (présence d'un établissement
colonial ou un fort dans la province).
Comme pour l'attrition en mer, on ne compte que les résultats d'attrition en 
pourcentage.
Si une exploration est faite en fin du mouvement terrestre, cela remplace le jet d'attrition.

Le débarquement de troupes hors-europe ne compte pas comme une action
pour une force navale si ce débarquement est accompagné par un
conquistadors (ou un explorateur qui sert de conquistador) et ne contient pas
de pion armée. Ainsi une force navale peut explorer une mer et débarquer
des conquistadors à la découverte des terres adjacentes dans le même round.
Il faut tout de même résoudre l'attrition maritime avant le débarquement si
la flotte a bougé avant le débarquement (sans qu'une exploration résolve
l'attrition de ce mouvement).

\subsubsection{Les combats hors europe}
-- NE PAS PRENDRE EN COMPTE --

Les pertes hors europe ne sont pas arrondies : une demie perte fait perdre deux DC,
et donc laisse un DC sur un DT. Les pertes infligées par DC ou 2 DC en combats sont
réduites de 2 sur la tables des pertes variables, puis limitées à 1 perte maximum.
Des DC en combat avec des DT ou armées comptent pour 1DT  si il y en a 2,
pour rien si il n'y en a qu'un.

Les DC ne contiennent pas de cavalerie donc il faut au moins un DT
pour obtenir un avantage de cavalerie sur des indigènes. Faire attention
que les indigènes d'Asie ont 2 ou 3 cavaleries par équivalent DT.

Les pertes causées aux indigènes sont de 15 points par détachement complet 
à perdre et 5 points  pour une demie-perte (qui n'est pas alors arrondie).


\subsubsection{Les indigènes}
-- NE PAS PRENDRE EN COMPTE --

En combat les forces indigènes sont transcrites en équivalent détachements selon
le rapport : 15 points d'indigène = 1 DT ; 5 points = 1 DC. 
Ils sont de classe terrestre Asie, Afrique, Amériques, soit \CAA.

Si une force indigène compte au final plus de 8 équivalent DT, elle inflige une fois
des pertes par fraction de 8 DT complète et une de plus pour la fraction restante
en lançant plusieurs dés sur la table de combat (un par 8 DT ou fraction).
Cette dernière fraction fera des dégats diminués par la table des pertes variables.

Les armées indigènes (Chine, Japon, etc) ne font pas apparaître de DC 
puisqu'ils n'ont pas de pions de ce type (leurs pertes sont donc arrondies 
comme les combats en Europe).

\subsubsection{Forts, missions, milices}
-- NE PAS PRENDRE EN COMPTE --

Un fort est une forteresse de niveau zéro pour l'action de sape.
Il suffit d'un détachement même réduit pour y mettre le siège.
Lors d'un assaut, sa résistance est 1/2 sauf si une brèche fut obtenue en
quel cas sa résistance est 0 : elle chute automatiquement sans combat
si il n'y a pas de troupe enfermée dans le fort. On peut mettre
un DT en défense dans un fort.

Une COL dispose d'une milice d'un DC par fraction de 2 niveaux, et donc 
d'un DT complet aux niveaux 5 et 6. Elles sont en général conscrites. Par 
exception la France a des milices vétérans. Les colonies du Portugal ont 
des milices vétérans et au nombre de 1 DC par niveau de colonie.

Tout comptoir possède de base un fort en défense qu'il faut prendre
militairement pour avoir le contrôle du comptoir (et pouvoir
alors le réduire à la phase de redéploiement). 
Des troupes en défense dans un fort de comptoir
n'entrainent pas de réaction indigène. Celles qui font le siège,
ou qui dépassent 1 DT, comptent. Si une forteresse de niveau supérieur est
construite, elle remplace ce fort et la présence de plus d'un DT peut
causer une réaction. Dans ce cas les troupes peuvent
s'enfermer dans la fortification (et elles réduisent les dommages
causés par la réaction indigène comme indiqué sur la
table).

Une mission compte dans une colonie comme un fort qui peut donc accueillir un DT
et doit être prise pour assurer le controle militaire de l'établissement. 
Les milices peuvent s'enfermer dans les missions.

\subsubsection{Achats dans les colonies}
-- NE PAS PRENDRE EN COMPTE --

L'achat est limité à 1 DC par province hors-Europe avec une COL ou un COM, ou bien
1 DT au coût double, et comptant double dans les limites d'achat.
On ne peut pas normalement pas acheter de pion armée dans les colonies et comptoirs.
Les forces navales achetées dans les colonies le sont au coût double et
comptent aussi double dans la limite d'achat du tour.

Une exception pour les colonies de niveau 6 : 
on peut y acheter par tour juqu'à 2 DT au coût normal, ou bien une
A- au coût double. Tous ces achats entrent dans la limite de construction normale.
Des navires peuvent y être achetés exactement comme en Europe (coût normal).

Les pêcheries permettent d'acheter un DE par deux ressources de pêche exploitées
au coût normal et en dehors de la limite des navires construits par tour.

-- NE PAS PRENDRE EN COMPTE --
-- NE PAS PRENDRE EN COMPTE --
-- NE PAS PRENDRE EN COMPTE --
-- NE PAS PRENDRE EN COMPTE --
-- NE PAS PRENDRE EN COMPTE --
-- NE PAS PRENDRE EN COMPTE --
-- NE PAS PRENDRE EN COMPTE --







% LocalWords:  condotta pikemen condottiere condotierri Landsknechts tercios
