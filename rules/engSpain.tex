\sectionJ{\anchorpaysmajeur{Espagne}}{\blason{espagne}}\label{chSpecific:Spain}



\subsection{Habsburg dynastic actions}\label{chSpecific:Spain:Dynastic Actions}

\begin{histoire}
  In 1492, the Spain sovereigns had not yet access to the resources of
  Burgundy or Dutch holdings of the Habsburg family. The dynastic bonds
  were woven bit by bit through weddings and inheritance. This rule
  allows to recreate the formation of this European Empire.
\end{histoire}


\subsubsection{The nature of dynastic actions}
\aparag Each turn, \SPA can use one (and only one) diplomatic action to
do a Habsburg diplomatic action. Each action has a difficulty, and a
score of at least this difficulty must be reached with 2d10 to have a
success.

\aparag The cost of the Habsburg diplomatic action is the one of a usual
diplomatic action
\bparag The usual modifiers due to investment (0, +2, +5) do apply to
the dice.
\bparag No other modifiers is possible, and no diplomatic support may
take place.

\aparag The actions are split in three classes (A, B and C). All the
actions of class A must have been successful to try an action of class
B. All the actions of class B must have been successful to try an action
of class C.

\aparag It is not possible to attempt a Dynastic Action at the turn
following a successful one (be it because of events or of diplomatic
action). Exception: there is no limit to attempt a diplomatic annexion
of a province of \pays{provincesne}.

\aparag It is no more possible to do diplomatic actions if the Habsburg
of Austria and Spain are dissociated (as per \eventref{pV:WoSS}).

\aparag Some events have as a consequence the success of a Habsburg
dynastic action. These actions do not cost anything to \SPA and are
always successful.
\bparag The effect of some of those actions is usually to activate
certain events (some of those events cannot take place without them).
\bparag A dynastic action may also allow be used to annex a province of
\pays{provincesne}, in which case this dynastic action is not counted
for the sake of Habsburg endogamy (see \ruleref{chSpecific:Belgium:Diplomatic
  Annexation}).


\subsubsection{List of dynastic actions}
\aparag Class A of dynastic actions:
\bparag[Habsburg wedding]
Difficulty 7. Activates \eventref{pI:Habsburg Alliance}.
\bparag[Burgundy inheritance] Difficulty 7. Activates
\eventref{pI:Burgundy Inheritance}.
\bparag[Neapolitan inheritance]
Difficulty 8. Activates \eventref{pI:Spanish Naples}.
\aparag Class B of dynastic actions:
\bparag[Bohemian wedding] Difficulty 8. Activates \eventref{pI:Habsburg
  Bohemia}.
\bparag[Milanese wedding] Difficulty 9. Activates \eventref{pI:Habsburg
  Milano}.
\aparag Class C of dynastic actions:
\bparag[Hungarian wedding] Difficulty 12. Activates
\eventref{pI:Habsburg Hungary}.
\bparag[Cession of \sectionprovince{Lombardia} to Spain] Difficulty
10. Activates \eventref{pI:Spanish Milano}.
\bparag[Portuguese wedding] Difficulty 11. When
event~\eventref{pIII:Portuguese Disaster} happens,
event~\eventref{pIII:Portuguese Annexation} is also applied immediately.
\bparag[Bavarian Wedding] Difficulty 9. \HAB (or \SPA as long as \SPA
and \HAB are not dissociated) has a diplomatic bonus of +1 on
\pays{Baviere}.


\subsubsection{Habsburg endogamy}
\begin{histoire}
  The Habsburg family often practised intra-familial weddings. Combined
  with the frequent violent deaths, this reduced the number of family
  members of high rank and increased the risk of congenital
  illnesses. Only a large crisis such as the Spanish War of Succession
  managed to inject some new blood in the royal family of Spain.
\end{histoire}
\aparag Each dynastic action increases the problems related to the
Habsburg endogamy for the Spanish sovereigns.
\bparag A special malus is applied to the dice throw of reign
duration. The malus does apply only to know if there is a dynastic
crisis. If there is no dynastic crisis (net result larger than 1), the
malus does not apply to determine the length of the reign.
\bparag The same malus is subtracted from 6 to determine the column
under which are read the characteristics of the new sovereign.
\dynasticactionsrecapfalse \GTtable{habsburgendogamy}

\aparag[War of Spanish Succession] Any dynastic crisis in period V
starts immediately \eventref{pV:WoSS} as one of the events of this turn.
\bparag The endogamy malus is no more applied if \SPA and \HAB are
dissociated due to \eventref{pV:WoSS}.


\subsubsection{Spanish Annexations}
\aparag The annexations that increased the Habsburg territory or Spanish
territory are set by events (such as \eventref{pIII:Portuguese
  Annexation} or dynastic actions that themselves trigger events.
\aparag[Italy] The annexation of \pays{Naples} is made after
\eventref{pI:Spanish Naples} either by conquest or diplomatic
annexation. The annexation of \province{Lombardia} is made after
\eventref{pI:Habsburg Milano} and \eventref{pI:Spanish Milano}.
\aparag[Bohemia] The annexation of \pays{Boheme} is made through
\eventref{pI:Habsburg Bohemia}.
\aparag[Hungary] The \pays{Hongrie} is quite sensitive to the
instability of the \region{Balkans} (\ruleref{chSpecific:Balkans}). Then
several events lead to \eventref{pI:Fall Hungary}, which splits the
Hungarian kingdom among \POL, \TUR and \HAB.
\aparag[Low countries] The Dutch provinces have to be either conquered
or annexed through dynastic actions. The remainder of the Burgundy
inheritance is given through \eventref{pI:Burgundy Inheritance}. See
also \ruleref{chSpecific:Belgium}.
\bparag If \HAB does not control all the provinces of \pays{provincesne}
and \pays{hollande} when \eventref{pIII:Dutch Revolt} occurs, \HAB loses
5\VP per uncontrolled revolted province.



\subsection{Autonomous Habsburg States}\label{chSpecific:Spain:Autonomous
  States}

\begin{designnote}
  \AUS and \SPA may choose to grant a greater autonomy to the cadet
  branches of their estates, losing the income provided by those lands
  in exchange for free \terme{basic forces} maintained by those states.
\end{designnote}


\subsubsection{General Conditions of Autonomy}
\aparag The autonomy is declared during the diplomatic announces
phase. The Habsburg country doing the declaration (\SPA or \AUS --- the
``owner'' hereafter) loses 1 \STAB.
\aparag The autonomy is granted to a whole group of provinces, none of
them can be retained.
\bparag Any revolt in the newly-autonomous country is automatically
removed.
\bparag The group must have at least three provinces to be declared
autonomous, except for \pays{HMilan} (one province only).
\aparag The income of the autonomous kingdom is no more perceived, but
the country granting autonomy keeps the ownership of the provinces.
\bparag The sum of the land income of the autonomous kingdom is neither
added to the \terme{Blocked Trade}, nor to the Vassal or main land
income.
\aparag No military forces can be raised in an autonomous kingdom but
the ones of the autonomous kingdom itself.
\bparag The autonomous kingdom has \terme{basic forces} freely
maintained, but does not receive any reinforcements. His armies have the
characteristics mentioned in the appendix (\ruleref{chAppendix:Habsburg vassal
  kingdoms}).
\bparag Their owner may rebuild forces if they were destroyed.
\bparag In some provinces, a fortress can be built (at the expense of
the owner), that is then maintained for free by the autonomous kingdom.
%% Muf ? forteresses constructibles ailleurs ?
% règles normales pour cela, la réponse est donc : Non.
\bparag The forces of the autonomous kingdom have a limited range.
\bparag The military campaigns are included in those of the \HAB having
granted Autonomy, and paid by him.
\aparag The autonomous kingdom is not subject to diplomacy. It is in
automatic \EW of the owner, never makes any separate peace and has its
provinces subjected to the peace agreements of its owner.
\bparag \pays{HBoheme} and \pays{HHongrie} are automatically put on the
\AUS diplomatic track; \pays{HNaples} is automatically put on the \SPA
diplomatic track; \pays{HMilan} is put on the track of its owner
(depending on whether \eventref{pI:Spanish Milano} has been played or
not).
\bparag For all other countries, the provinces of these autonomous
kingdoms are still assimilated to provinces of their owner (for peace
levels, etc.).


\subsubsection{\sectionpays{HNaples}}
\aparag \pays{HNaples} is constituted by the provinces of
\province{Campania}, \province{Basilicata}, \province{Abruzzo},
\province{Puglia}, \province{Calabria}.  If activated by \AUS or \hab
only, add the three following provinces: \province{Sicilia},
\province{Palermo}, \province{Saldigna}
\aparag \pays{HNaples} has an \ARMY\facemoins (\CAIII, Latin, \TTER if
\SPA is) and a \FLEET\facemoins (choose between galleys or warships).
\bparag A fortress may be maintained for free in \province{Campania}.
\bparag The \ARMY may act in the whole kingdom and in \region{Italie},
the \FLEET may act in Mediterranean.


\subsubsection{\sectionpays{HMilan}}
\aparag \pays{HMilan} is constituted by the sole province of
\province{Lombardia}. \SPA may grant autonomy only after
\eventref{pI:Spanish Milano} has been played.
\aparag \pays{HNaples} has an \ARMY\faceplus (\CAIII, Latin, \TTER if
\SPA is).
\bparag A fortress may be maintained for free in \province{Lombardia}.
\bparag The \ARMY may act in \paysmajeur{Espagne}, in \region{Italie},
in \paysmajeur{Autriche}.


\subsubsection{\sectionpays{HBoheme}}
\aparag \pays{HBoheme} is constituted by the provinces of
\province{Boheme}, \province{Lausitz}, \province{Silesie},
\province{Morava}, and is part of the \HRE.
\aparag \pays{HBoheme} has an \ARMY\faceplus (\CAIII, Latin).
\bparag A fortress may be maintained for free in \province{Boheme}.
\bparag The \ARMY may act in the whole \HRE, in \paysmajeur{Pologne},
\pays{Hongrie}, \paysmajeur{Autriche} (or minor \pays{habsbourg}).
% \aparag If \pays{HBoheme} is invaded, it receives immediately (during
% the invasion movement) 2\LD of militia (\terme{Conscripts} when they
% appear), and this, once per war. The maintenance of this militia is
% free until the end of the war. They cannot go out of \pays{HBoheme}
% and are destroyed in the peace phase. They may be integrated in an
% \ARMY, but the special conditions remain.


\subsubsection{\sectionpays{HHongrie}}
\aparag \pays{HHongrie} is constituted by the provinces of
\province{Szlovakia}, \province{Karpatok}, \province{Bukovina},
\province{Balaton}, \province{Pecs}, \province{Erdely},
\province{Mures}, \province{Kranj}, \province{Croatie},
\province{Banat}, \province{Kapela}, \province{Hongrie}.

\bparag It may be granted autonomy by \SPA only if \eventref{pI:Habsburg
  Hungary} was played (not \eventref{pI:Fall Hungary}).
\bparag If autonomy is granted by \AUS, \AUS may no more use the
military counters given by \eventrefshort{pI:Habsburg Hungary} or
\eventrefshort{pI:Fall Hungary}, nor the augmentation of \terme{basic
  forces} due to \pays{Hongrie}.

\aparag \pays{HHongrie} has an 2\ARMY\faceplus (\CAIIM, Latin).
\bparag If \pays{HHongrie} is reduced to 4 provinces or less, the forces
are reduced to 1\ARMY\faceplus.
\bparag Fortresses may be maintained for free in any province.
\bparag The \ARMY may act in any potential province of \pays{HHongrie},
in \paysmajeur{Pologne}, \paysmajeur{Turquie}, \pays{HBoheme},
\pays{Hongrie}, \paysmajeur{Autriche} (or minor \pays{habsbourg}) and
provinces of the Balkans (listed in~\ruleref{chSpecific:Balkans}).
% \aparag If \pays{HHongrie} is invaded, it receives immediately (during
% the invasion movement) 4\LD of militia (\terme{Conscripts} when they
% appear), and this, once per war. The maintenance of this militia is
% free until the end of the war. They cannot go out of \pays{HBoheme}
% and are destroyed in the peace phase. They may be integrated in an
% \ARMY, but the special conditions remain.
% \bparag The militia drops to 2\LD if \pays{HHongrie} is reduced to
% less than 4 provinces.


\subsubsection{Autonomous States and Events}
\aparag Like in any minor country, the revolts in an autonomous state at
peace are automatically subdued.
\aparag During \eventref{pIV:Bohemian Revolt}, the Kingdom of Bohemia
may revolt, and thus \pays{HBoheme} ceases to be an autonomous
state. \AUS may redeclare autonomy (if desired) after the end of the
event.
\aparag At the time of dissociation, \HAB may decide to take anew the
control of \pays{HBoheme} or \pays{HHongrie}. It is made by a simple
announce. It is complied to give the autonomy to \pays{HNaples} when it
is in its control, at the first diplomacy phase where \AUS is at
peace. This autonomy cannot be deactivated.
\bparag The same does apply to \pays{HMilan} if \AUS obtains this
territory at the end of \eventref{pV:WoSS}.



\subsection{Spanish economy}


\subsubsection{Spanish Colonial Policy}
% 53.28, paragraphs A, B and C.

\aparag[Viceroys] The Spanish player is allowed one extra colonisation
action (as compared to the turn limit), free of charge and of investment
low, each turn and for a specific Area, provided that \SPA has named a
Viceroy in the Area.
\bparag A Viceroy is a Spanish Conquistador that is publicly announced
to be a Viceroy during the Administrative phase. This Conquistador is
not allowed to leave the Area anymore.
\bparag When a Spanish Conquistador captures the \pays{Azteque} capital
city of \ville{Tenochtitlan}, or the \pays{Inca} capital city of
\ville{Cuzco}, this Conquistador is named viceroy of those respective
Areas (as well as the adjacent Area of \granderegion{Chichimeca} in the
case of the \pays{Azteque} Area conqueror) and is not allowed to leave
them anymore until removed from play.
\bparag A Viceroy acts as a Governor for the bonuses in \COL
attempts. It is a Conquistador for discoveries and the use of the table
of Conquistadors.
\bparag If more than one Area has a Viceroy, only one each turn gains
the free colonisation action (player's choice).

\aparag[El Dorado] The Spanish player may only attempt to place \COL in
Areas in \continent{America} that contain at least one gold mine site,
or in Areas adjacent to such gold mine Areas, or also in
\continent{Caraibes}.
% \granderegion{Haiti}, \granderegion{Cuba} and \granderegion{Antilles}
% Areas.
\bparag This restriction is lifted from 1615 (turn 26, period IV)
onward.
\bparag \textbf{Exception:} Starting from 1560 (turn 15, period III),
\HIS may also attempt to place \COL in \granderegionPhilippines.

\aparag[Foreign trade index] \SPA has a specific \FTI for \COL
operations, that is different from its \FTI (see
\ruleref{chAdministration:Special FTI}).
\bparag This \FTI is also used for Portuguese \COL operations while
\pays{Portugal} is in annexation.

\subsubsection{New Spain}
\aparag \HIS may annex all establishments (\COL and \TP) of its enemies
in an Area in the El dorado (as defined above).
\bparag This count as 1 peace condition, plus 1 per establishment not
controlled by \SPA in the Area at the time of the peace.

\subsubsection{Spanish Missionaries}
\aparag See \ruleref{chSpecific:Missions} for the general rules.

\aparag[Unnamed \LeaderMis]
\bparag \SPA has four unnamed \LeaderMis: two \leader{Dominicos} and
two \leader{Franciscanos}.
\bparag Between turn 2 and 25 included (periods \period{I} to
\period{III}), if there is no new named Spanish \LeaderMis scheduled
this turn, \SPA receives an anonymous \LeaderMis (if one is
available).
\bparag From period \period{IV} onward, \SPA only receives an
anonymous \LeaderMis on even-numbered turns (they have a \Xcatholique
symbol on the turn track).
\bparag If an anonymous \LeaderMis dies (battle, exploration,
attrition, \ldots), he is returned to the counter pool.
\bparag However, if a \LeaderMis is used to build a mission, its
counter is permanently removed from game as with regular \LeaderMis.

\aparag Spanish missions add 1\LDE of colonial militia and give the
\terme{Veteran} status to the Colonial Militia in the same province.

\aparag \SPA loses 5\PV each time one of its Mission is destroyed.

\aparag \SPA should place one Mission in each Area where it has a \COL
(nor necessarily for \TP). For each colonised Area without such a
Mission \SPA loses 5\PV at the end of the period.
\bparag Only one Mission is needed for \continent{Caraibes}
(in any one Area).

\subsubsection{The Gold Flow}
\aparag As soon as the gold mines bring at least 40\ducats per turn to
\SPA, there is a permanent malus of {\bf -2} for \MNU construction and
\FTI or \DTI augmentation.

\subsubsection{The American Empire}
\aparag On turns 21 and 22, \leader{Antonelli} allows the free building
of one level of \terme{arsenal} or of \terme{fortress} in the \ROTW
province where he is during the \terme{expense phase}.


\subsubsection{The Flota de Oro}
\aparag \SPA can use each turn two convoys: the Flota de Oro and the
Flota del Peru. They are transports fleets each containing 5 \NTD and
allowed to carry gold only (thus up to 75\ducats).
\bparag Those fleets can be placed automatically full and back on the
\ROTW map, in a \COL port belonging to \SPA, at the beginning of each
military round upon reaching Europe or being destroyed (sunk or
captured).
\bparag The Flota de Oro is placed in any Spanish port in
\continent{America} on the Atlantic Ocean, and the Flota del Peru in any
port in \continent{America} on the Pacific Ocean.

\aparag The counters are considered to be naval units but count in the
stacking limit as a small counter (a \DT, and not a fleet).
\bparag Therefore it must roll for attrition as any other naval unit. If
this unit is intercepted while alone (not escorted), all the gold it
currently carries is captured.
\bparag If attacked when escorted and if the escort loses the battle,
the attacking player receives any gold transported on Transport that
would be captured during a pursuit.
\bparag Losses due to attrition are of 15\ducats per sunken \NTD.

\aparag[] [BLP] During periods \period{II} to \period{VI} included,
\HIS gains an extra \anonyme\LeaderA out of limit each turn.
\bparag This \anonyme\LeaderA is always stacked with the Flota de
oro. It gains the capacity to go in the ROTW if the counter does not
have it.
\bparag Every time the Flota de Oro is moved back to
\continent{America} (after reaching Europe or being sunk), its admiral
is changed (discard the previous \anonyme\LeaderA and draw a new one
at random among the Spanish ones).

\subsubsection{The Spanish Holland}\label{chSpecific:Spain:Spanish Holland}
\aparag See also \ruleref{chSpecific:Belgium} for the state of Holland before
annexation by \SPA, and \ruleref{chSpecific:Holland:First Revolt} for what
happens after \eventref{pIII:Dutch Revolt}.

\aparag[The Spanish Tax] \label{chSpecific:Spain:Dutch Tax} The Spanish
Holland is the set of all provinces belonging to \SPA in the limits of
the national territory of \paysmajeur{Hollande}
(\theminorprovincesshort{hollande}). \SPA does not raise directly income
from these provinces, but may choose to tax those after
\eventref{pI:Habsburg Alliance}.
\bparag The income is of 40\ducats, plus 10\ducats per province owned,
for a maximum of 100\ducats, to be added in \lignebudget{Special
  income}.
\bparag These provinces are counted as foreign for
\ruleref{chIncomes:Trade Income}.

\aparag[Trade implantation] See rule \ruleref{chSpecific:Holland:Dutch Trading
  Fleets}.
\bparag The Dutch \TradeFLEET are counted as Spanish for
\ruleref{chIncomes:Trade Centres} after \eventref{pI:Habsburg Alliance} and
\eventref{pI:Burgundy Inheritance} have been both played. The Atlantic
\terme{Trade Centre} is initially set in \province{Vlaanderen} (and
counts for \pays{provincesne} and \pays{bourgogne}).
% (Jym, 06/2013) : to who ? against whom ?
% plus, this disagree with the tables where the \OCB of SPA is valid
% only before the revolt of HOL...
% Removing.
%
% \bparag However, when \SPA has the Atlantic \terme{Trade Centre} because
% of this rule, this does not gives an Overseas \CB.

\aparag All the preceding rules do not apply any more if
\pays{Vhollande} comes into existence (through
\ruleref{chPeace:Peace:Independence Revolt}). This may lead to applying
the effects of \eventref{pIII:Dutch Revolt} before period III.


\subsubsection{The Flanders Factories}\label{chSpecific:Spain:Cloth}
\aparag The \RES{Cloth} \MNU that is available following
\eventref{pI:Burgundy Inheritance} must remain in
\province{Vlaanderen}. It is destroyed if \province{Vlaanderen} ceases
to be Spanish.
\bparag It is also destroyed also the first time that \ENG, \HOL and
\FRA each have a \RES{Cloth} \MNU and if \SPA is
\terme{Counter-Reformation}.
\bparag Before \eventref{pIV:Olivares}, this \RES{Cloth} \MNU can only
be re-built in provinces of Flanders and Holland.


\subsubsection{Expulsion of the Jews and the Moriscos}
\label{chSpecific:Spain:Expulsion}
\begin{histoire}
  The Alhambra Decree was issued in 1492 by the Monarchs of Spain,
  following the final triumph over the Moors after the fall of
  Granada. The decree ordered the expulsion of all Jews from Spain.

  After the fall of Granada in 1492, the Muslim population was promised
  religious freedom by the Treaty of Granada, but that promise was
  short-lived. The persecutions led to an uprising in 1500. This was
  suppressed, and the Spanish authorities took that as a pretext to void
  the rights and obligations in the surrender treaty.

  The Moriscos, or converted Muslims, still lived in Spain, especially
  in the Granada and Valence areas. They were, however, persecuted by
  the inquisition and the population, leading to the uprising of Granada
  in 1568. Several edicts of expulsion were tried in various part of
  Spain until the final expulsion of the Moriscos from all the kingdom
  in 1614.
\end{histoire}

\aparag  Nouvelle règle d'expulsion des Juifs et des Moresques de l'Espagne : \\
- choix au moment de I-8 (1) : Politique d'expulsion ou non \\
- si CR : pas d'expulsion = -50 PV \\
- si Conc. : pas de pénalité \\

- Politique d'expulsion : \\
a- limite en FTI/DTI \\
b- ajoute 10\% à la capacité d'emprunt national \\
c- bonus +1 aux tests de banqueroute \\
- Ces effets se terminent d'un manière ou de l'autre : \\
1) Expulsion finale après III-10 : annule les effets b et c \\
2) Revenir sur la politique d'expulsions : annule tous les effets, \\
coûte 25 PVs ou gratuit lors de IV-2 (1) Olivares \\

% --- SUITE A ENLEVER SI COMPTABILITE v2 ---

% \aparag At the beginning of any diplomatic phase, during periods I to
% III, \SPA may proclaim the expulsion of the Jews and the Moriscos.
% \bparag \SPA automatically receives 500\ducats in its royal
% treasury. All the national loans running can be cancelled without any
% penalty.
% \bparag However, the \FTI (non-\COL part) and \DTI of \SPA remain
% limited to 2 until the end of period V or \eventref{pIV:Olivares} is
% active.
% \bparag If Spain is the \SDCF and has not proceeded to the expulsion by
% the end of period III, it loses 75\VP.
% \\
% --- FIN: SUITE A ENLEVER SI COMPTABILITE v2 ---

\bparag This effect is cancelled if \SPA chooses to be conciliatory.


\subsubsection{Asiento and Exclusivity}\label{chSpecific:Spain:Asiento}
\aparag \SPA applies a commercial policy of exclusivity, conceding the
right to trade with the Spanish colonies (the \terme{Asiento}) to a very
limited number of merchants. There are three different possible
statuses:
\bparag Exclusive Asiento to Spain;
\bparag Weakened Asiento (allows for some contraband and partial trade
with foreigners);
\bparag Asiento conceded to another \MAJ.
\aparag[Exclusive Asiento.]
\bparag \SPA cannot grant the right to trade for \STZ where it has \COL
to other countries.
\bparag \SPA cannot use \RES{Slaves} from any other establishment than
its own and those of its vassals.
\bparag If \pays{Portugal} is in annexation, the same policy applies to
the \COL of \pays{Portugal}.
\bparag \SPA has a free \CONC of high investment against a \TradeFLEET
in a \STZ bordering a Spanish (or Portuguese if annexed) \COL.

\aparag[Weakened Asiento]
\bparag From 1615 (turn 26, period IV) onward, \SPA may choose each
turn to redefine its commercial policy. The first time it goes to
\terme{Weakened Asiento}, \SPA loses 20\VP and 1\STAB. Changing later
does not cost anything.
\bparag \SPA can now use \RES{Slaves} from the contraband, or buy
\RES{Slaves} to a \MAJ having a \TradeFLEET in a \STZ bordering a
Spanish \COL.

\bparag \SPA can grant the right to put a \TradeFLEET in a \STZ
bordering a Spanish \COL, paying 10\VP per country and per \STZ.
\bparag When \SPA returns to \terme{Exclusive Asiento}, all the
countries with a \TradeFLEET in a \STZ bordering a Spanish \COL have an
Overseas \CB for this turn only.

\aparag[Asiento conceded to a country] The Asiento is a right that can
be conceded to someone else due to wars after 1665 (turn 36, period V)
(Asiento is equivalent to one province in terms of peace condition and
can be taken also by an oversea war). The former owner of the
\terme{Asiento} (if not \SPA) gains an Overseas or normal \CB this turn
or the next one (to be chosen by the victim).
\bparag A country imposing a peace to \SPA can ask for the
\terme{Asiento} instead of a province. \SPA cannot oppose this. \SPA has
then a permanent Overseas \CB to retake the \terme{Asiento} right.
\bparag If the \terme{Asiento} is given, the \terme{Weakened Asiento}
effects are applied (with losses of \VP and \STAB for the first time).
\bparag \SPA must use the \RES{Slaves} of only the owner of the
\terme{Asiento}. \SPA may request between 0 and 4 \RES{Slaves} each
turn, to be given for free. More may be sold, but there is no
obligation. As long as the \terme{Asiento} owner can give the requested
\RES{Slaves} that \SPA requests, he keeps the \terme{Asiento}. He loses
the \terme{Asiento} after 3 consecutive turns of not providing the
requested \RES{Slaves}.
\bparag \SPA may use the \RES{Slaves} contraband.
\bparag[VP of the Asiento] A \MAJ with the \terme{Asiento} accumulates
20\VP, plus 1 per turn where he can meet the \RES{Slaves} request. Those
\VP are stored, and received at the end of the game or if
\terme{Asiento} is retaken following a war. The \VP are lost if the
\terme{Asiento} is lost due to not giving the requested \RES{Slaves} 3
consecutive turns.



\subsection{Military means of an empire}


\subsubsection{The Tercios}
\aparag \TTER is a technology specific to the Spanish forces, obtained
when \SPA Land technology marker reaches the \TTER box.
\bparag The autonomous kingdoms of \pays{HNaples} and \pays{HMilan} are
also \TTER during periods I and II if \SPA is \TTER.
\bparag The technology \TTER cannot be obtained before 1530. If this
happens, \SPA will get the \TTER technology in 1530 (turn 9).
\bparag \TTER units have a basic morale of 3 during \TREN and \TARQ
(contrarily to other units, that have only 2).
\bparag The units opposed to \TTER units receive a malus of {\bf -1} to
Shock unless in classes \CAI, \CAIM, \CAII, \CAIIM\ during periods I to
V.

\aparag The \TTER status is lost as soon as a major battle is lost by a
stack containing at least one \ARMY\faceplus with the \TTER advantage
against a stack with technology \TBAR.
% \aparag The \TTER status is lost as soon as a battle is won on a force
% counting at least one \ARMY\faceplus with the \TTER advantage is
% beaten by a country with technology \TBAR.
\bparag It is also lost as soon as a power obtains the \TMAN technology.


\subsubsection{Spanish Recruitment Area}\label{chSpecific:Spain:Recruitment
  Area}
% Jym, 05/2011
% recruitment now limited only in Catalogne (reluctant cortes).
%\aparag The \terme{Recruitment Area} of \SPA is limited to its capital
%province \province{Castilla La Nueva}, \province{Andalucia} and
%\province{Campania} (only if Spanish).
\aparag[Reluctant Catalogne] Recruitment of \HIS inside provinces of
Catalogne in its National territory (\provinceCatalunya,
\provincePirineos) costs double the normal price.
\aparag[Spanish Lombardia] After \eventref{pI:Spanish Milano},
\province{Lombardia} is added to \SPA's \terme{Recruitment Area}.
\aparag[Spanish Road] If \SPA controls minor countries or provinces
forming a continuous road from \province{Lombardia} to any province of
\pays{provincesne} or \pays{hollande}, all the provinces of the Low
Countries (\paysProvincesne, \paysHollande and \paysBourgogne) are added
to its \terme{Recruitment Area}. It can raise forces there at normal
cost. The control can be any diplomatic status (starting at \RM).
\begin{histoire}
  The cortes (assembly) of Catalogne was usually reluctant to the rising
  of new troops ordered by the central power in Castille.

  The historic ``Spanish Road'' went through Savoy, Franche-Comt\'e,
  Lorraine and Alsace, Luxembourg and shifted toward Switzerland when
  the Bresse became French.
\end{histoire}


\subsubsection{The Italian Fleet}\label{chSpecific:Spain:Italian Fleet}
In periods II, III and IV, \SPA adds \FLEET\facemoins to its
\terme{basic forces} if it owns \province{Campania} (without having
given its autonomy to \pays{HNaples}).


\subsubsection{Flemish sailors}
\aparag \SPA may raise a \corsaire in any of the following provinces:
\province{Calais}, \province{Flandre}, \province{Vlaanderen},
\province{Zeeland} if it owns the province in question.
\bparag This \corsaire may only be used on the European map and not in
the Mediterranean Sea.
\aparag The second \corsaire can only be raised after
\ministre{Olivares} or \ministre{Alberoni} and cannot go in \STZ of the
\CCs{Mediterranean}.

\subsubsection{At sea}
\aparag[Fleet in being] [BLP] \anonyme\LeaderA numbered 6, 7, 8 and 9
are only available starting from period \period{V} (1660, turn 35).

\subsection{Other political rules for Spain}


\subsubsection{Grouped annexions in Italy}
\aparag \SPA may consider \province{Palermo} and \province{Sicilia} as
one province when signing a winning peace, so as to take them as one
Peace condition.
\aparag \SPA may consider 2 provinces among \province{Campania},
\province{Basilicata}, \province{Abruzzo}, \province{Puglia},
\province{Calabria} as one province when signing a winning peace, so as
to take them as one Peace condition.
\aparag \SPA may consider all the provinces \province{Campania},
\province{Basilicata}, \province{Abruzzo}, \province{Puglia},
\province{Calabria} as being two provinces when signing a winning peace,
so as to take them as two Peace conditions.


\subsubsection{Minor countries dependent on Spain}
\aparag[Knights] \SPA is interested in the rules about \pays{chevaliers}
(\ruleref{chSpecific:Knights}), especially by \ruleref{chSpecific:Knights:Transfer}.
\aparag[Low Countries] \SPA is interested in the rules about
\pays{provincesne} and \pays{hollande} (\ruleref{chSpecific:Belgium})
\aparag[Burgundy] \SPA is interested in the rules about \pays{Bourgogne}
(\ruleref{chSpecific:Burgundy}) and \pays{Liege} (\ruleref{chSpecific:Liege}).
\aparag[Austria] Finally, \SPA should read the rules about Austria and
Habsburg (\ruleref{chSpecific:Austria}).


\subsubsection{The Defence of the Catholic Faith}
\aparag[Catholic Faith] \SPA is interested in \ruleref{chSpecific:Catholic
  Faith}, \ruleref{chSpecific:Crusades} and \ruleref{chSpecific:Fall Vienna}.
\bparag See also the rules about \ruleref{chSpecific:Papacy}.



\subsection{\sectionpaysmajeur{Espagne} in play}


\subsubsection{Spanish Monarchs and Ministers}
\aparag[\anchormonarque{Isabel and Ferdinand}] are the monarchs in 1492,
with values 6/7/6, scheduled to die at the beginning of turn 6.
\aparag[\anchormonarque{Charles V}] is the first Spanish monarch to
access the throne after \dynasticaction{A}{1} (and \eventref{pI:Habsburg
  Alliance}). He has values 6/9/8 and lasts 8 turns. He does not roll
for survival for the first five turns of his reign. He is also a general
\leaderwithdata{Carlos I}.
\bparag When \monarque{Charles V} is sovereign, \HAB has a special
alliance with \SPA. %\SPA gains the gold mines income of \HAB.
\HAB can be activated without any test, nor losing any \STAB, when \SPA
declares war.
\bparag \SPA has a bonus for \eventref{pI:Emperor Election}.

\aparag[\anchormonarque{Felipe II}] is the heir to \monarque{Charles
  V}. His values are 6/7/6, and is reign lasts 9 turns. He does not roll
for survival for the first five turns of his reign. He is not a general.
\bparag When \monarque{Felipe II} is sovereign, \HAB has a special
alliance with \SPA. % \SPA gains the gold mines income of \HAB.
\bparag During his reign, \HAB has a special bonus of {\bf +2} to all
Bankruptcy tests.
\aparag[\anchorministre{Olivares}] may be named minister through
\eventref{pIV:Olivares}. He has values 8/9/7 and remains a random number
of turns; its values can be used for the next monarch's values
determination if a succession takes place while he is still alive.
\aparag[\anchorministre{Alberoni}] may be named minister through
something, probably.
% \eventref{pIV:Olivares}. He has values 8/9/7 and remains a random
% number of turns; its values can be used for the next monarch's values
% determination if a succession takes place while he is still alive.


\subsubsection{Available counters}
\aparag[Military] 5\ARMY, 4\FLEET, 2\corsaire, 10\LDND, 10\LD, 4\NTD,
10\LDENDE, 6 fortresses 1/2, 4 fortresses 2/3, 4 fortresses 3/4, 3
fortresses 4/5, 10 forts, 2 Arsenals 2/3, 2 Arsenals 3/4, 15 Missions.
\aparag[Economical] 32\COL, 7\TP, 9\MNU, 13\TradeFLEET, 2\ROTW treaty
counters.

% LocalWords: Habsburg endogamy malus pI Milano pIII HBoheme HHongrie
% HNaples LocalWords: HMilan provincesne Autriche Lombardia Campania
% Espagne Abruzzo LocalWords: Basilicata Puglia Calabria Boheme Lausitz
% Silesie Morava Pologne LocalWords: Hongrie habsbourg Turquie pIV
% redeclare TYW WoSS pV Szlovakia de LocalWords: Bukovina Erdely Croatie
% Kapela Banat Sicilia Saldigna Interphase LocalWords: Azteque Dorado
% tercios intra Baviere Karpatok Mures Kranj Flota LocalWords: hollande
% Franche Comt Bresse Flandre Zeeland Oro Nueva Pacifico LocalWords:
% Moriscos Asiento Italie Andalucia Vlaanderen bourgogne
