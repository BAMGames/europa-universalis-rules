% -*- mode: LaTeX; -*-

\section{Military forces}

We describe here the different kinds of military forces: troops, navies and
fortifications. Troops and navies work in similar ways, especially with the
notion of \terme{detachment}, but with small differences.

All these forces are in limited amount. The number of counters provided in the
game is an absolute limit on what is usable. Different countries have
different number of counters of each kind.

Exception: \pays{pirates} \corsaire, \pays{natives} troops, neutral fortresses
and revolted troops are not limited. If you need more than provided by the
game, you may use whatever you wish to represent them.



\subsection{Land forces}


\subsubsection{Troops}\label{chExpenses:Logistic:Troops definition}
\aparag[Troops] are represented by three different kinds of counters
corresponding to various size of land forces: \terme{Land Detachment} (\LD),
\terme{Army} (\ARMY) and \terme{Land Detachment of Exploration} (\LDE).
\bparag The basic count unit is one \LD.

\aparag[Detachments.] One \LD represent some infantry and cavalry. The precise
number of them depends on the country and the period. \LD are abstract
representation of small field forces and consists roughly in 1000 to 5000
soldiers.

\aparag[Armies.] \ARMY counters have two sides. An \ARMY\facemoins is two \LD
plus some field and siege artillery. An \ARMY\faceplus is four \LD plus more
artillery.

\aparag[Breaking armies.] An \ARMY counter can be broken into an equivalent
number of \LD (2 or 4) of the same country at any time in the game. Note that
artillery is lost in the process.
\bparag Similarly, an \ARMY\faceplus can be turn into one \ARMY\facemoins and
two \LD at any time.
\bparag Especially, \ARMY can be broken during movement or to satisfy losses
(whether combat or attrition). If one \ARMY\faceplus suffers a 1 \LD loss,
there is one \ARMY\facemoins and 1 \LD remaining.
\bparag However, if there are not enough \LD counters to satisfy the loss,
heavier loss are suffered. If one \ARMY\facemoins suffers a 1 \LD loss but
there are no more unused \LD of the same nationality available, then the
entire \ARMY\facemoins is annihilated.

\aparag[Creating and reinforcing armies.] The only way to create a new \ARMY
counter is to buy it during the Administrative phase (logistic segment).
\bparag Especially, it is never possible to ``merge'' two \LD into an
\ARMY\facemoins.
\bparag On the other hand, it is possible to reinforce an \ARMY\facemoins with
two \LD (in one stack) and turn it into an \ARMY\faceplus. This can be done at
any time in the game.
\bparag It is also possible to merge two \ARMY\facemoins into one
\ARMY\faceplus.

\aparag[Special armies.] The armies of \pays{saint-empire} and \pays{croises}
act as containers. Each may contain up to 4\LD of some nationality and can be
created at any point during the turn. The precise contents of these armies
must be written done in order to give back the \LD to their owners when the
army is broken.

\aparag[Detachments of Exploration.] One \LDE represents roughly one third of
a \LD.
\bparag \LDE can only exists on the \ROTW map (including European provinces on
the \ROTW map). As soon as one \LDE enters the European map, it is immediately
destroyed.
\bparag One \LD can be split in 3 \LDE at any time (especially to satisfy
losses in the \ROTW). 3 \LDE stacked together can be turned into 1\LD at any
time.
\bparag For maintenance and purchase, 1\LDE costs as much as half a \LD.
\bparag \LDE are never counted in stacking and supply limits.

\aparag[Natives.] Each \ROTW \Area holds a certain number of natives per
province. They are written on the \ROTW map in number of \LD.
\bparag Counters are provided to remember losses of natives in each
province. You may use \ARMY\faceplus/\facemoins to represent 4/2 \LD of
natives but this is for convenience only: they are not considered as \ARMY for
game purposes. These counters are in unlimited quantity.


\subsubsection{Military doctrine}
\aparag Each country has an \terme{Army Class} written in roman numerals on
its counters.
\bparag The class of a country determines three factors: its \terme{Size}, its
\terme{Cavalry} and its \terme{Artillery}.
\bparag Some countries (mostly majors) belong to one army class but have
special cases for artillery and cavalry.
\bparag The army class of each country can be read in~\ref{table:Army
  Classes}. There is one line per class with its number and name on the left
and the list of countries belonging to it on the right.
\bparag Most minor countries are grouped according to their cultural groups.

\GTtable{armyclasses}

\aparag[Size.] The army size of each country, per period, can be read
in~\ref{table:Army Classes} by cross-referencing the army class of the country
(or its name) with the current period.
\bparag The result is a number between 0 and 7 representing an abstract
measure of the typical size of forces fielded by this country during that
period.
\bparag A larger size means that the country usually fielded more men in
battles. However, this is an abstract measure and there is no direct
correspondence between the size and an actual number of soldiers. Moreover,
these numbers are relative (to other countries). A decreasing size does not
mean that the country had smaller armies, but rather than its neighbours
started having larger ones.
\bparag Countries with larger size do more damage in battle when facing
countries with smaller size.

\aparag[Cavalry] is abstractly represented by giving a small bonus in battle
to certain classes of armies during certain periods of the game.

\aparag[Artillery.] Each \ARMY\facemoins and \ARMY\faceplus contains a certain
number of artillery. This number is an abstract representation (rather than an
actual number of guns and howitzers) of the amount and efficiency of field and
siege artillery.
\bparag The number of artillery per \ARMY\faceplus can be read
in~\ref{table:Artillery Per Army} by cross-referencing the country (or class)
with the current period.
\bparag An \ARMY\facemoins always contains half the number of artillery of an
\ARMY\faceplus (rounded down).

\GTtable{artilleryvalue}

\aparag[Artillery of stacks.] When two (or more) \ARMY are stacked together,
their artillery numbers do not simply add. Instead, use the following
computation:
\bparag Take the artillery value of one \ARMY in the stack (the larger the
better); add \bonus{+2} if there is another \ARMY with 2 or more artillery
otherwise, add \bonus{+1} if there is another \ARMY with 1 artillery.

\begin{exemple}
  \FRA is of class \CAIV (``majors''). In periods \period{I} to \period{IV},
  it has a size of 2, then 3 in period \period{V} and 4 afterwards.

  In period \period{II}, \FRA has 3 artillery per \ARMY\faceplus. Thus, it has
  only 1 artillery per \ARMY\facemoins (3/2, rounded down). A stack with
  2\ARMY\faceplus of \FRA is thus considered to have 3 (first \ARMY) + 2
  (second \ARMY with 2 or more artillery) = 5 artillery for all game purposes
  (battles and sieges). A stack of \ARMY\faceplus \ARMY\facemoins has 3 + 1
  (second \ARMY with only 1 artillery) = 4 artillery. Lastly, a stack of 3
  \ARMY\facemoins of \FRA only has 1 (first \ARMY) + 1 (second \ARMY with 1
  artillery) = 2 artillery (\emph{i.e.} the third \ARMY counter does not add
  any artillery to the stack).
\end{exemple}



\subsection{The Navy}

\aparag[Naval forces] are represented by three different kinds of counters
corresponding to various sizes of naval forces: \terme{Naval Detachment}
(\ND), \terme{Fleet} (\FLEET) and \terme{Naval Detachment of Exploration}
(\NDE).
\bparag The basic count unit is one \ND. However, there are several kind of
\ND corresponding to various type of ships.

\aparag[Warships and Galleys.] \ND can represent different kinds of
ships. Mostly warships, galleys or transports.
\bparag Thus, there are several kind of \ND: \terme{Naval Warship Detachment}
(\NWD), \terme{Naval Galley Detachment} (\NGD) (also the Venetian
\terme{Galleass}, written \VGD) and \terme{Naval Transport Detachment} (\NTD).
\bparag All those naval detachments are treated differently, but some rules
apply to all. In this case, the generic term \ND will be used.
\bparag \VGD are considered \NGD when the case apply (\emph{i.e.} whenever
there are rules for \NGD without special cases for \VGD, these rules apply).
\bparag \NGD can only exists in the \regionMediterrannee and the
\regionBaltique.

\aparag[Detachments.] One \ND represents roughly 2 to 6 first category ships
(galleys, galleons, man-o-war, \ldots) plus various second category ships. The
precise number depends on the country, the period and the kind of ships
involved.
\bparag \NTD only contains transport ships. They do not participate in battles
but can be used to transport troops or gold.

\aparag[Fleets.] \FLEET counters are containers. They may hold a certain
number of \NWD (or \NGD) plus some \NTD.
\bparag Unlike \ARMY, the exact content of a \FLEET counter depends both on
countries and period (representing evolution of the naval doctrines).
\bparag The (maximal) content of the fleets is detailed in the
\tableref{table:Countenance of Fleets}. It can contain a number of \NWD (a
\NGD counts for half a \NWD) and a number of \NTD. This number depends of the
period and the country involved.
\bparag A \FLEET is put on the side \Faceplus only if there is not enough room
in a \FLEET\facemoins to accommodate all the \ND. The counter is turned as
necessary.
\bparag Since the exact content of \FLEET counters is not fixed, it must be
written down. There is space for this on the colonial record sheet of each
country.

\GTtable{fleetsize}

\aparag[Creating and breaking fleets.] A \FLEET may be broken into several \ND
(depending of its content) at any time.
\bparag Similarly, several \ND can be merged into a \FLEET (or incorporated
into an existing one) at any time. \FLEET counters may be created this way.
\bparag Even if it does not provides a direct military advantage (such as the
artillery for \ARMY), using \FLEET rather than \ND usually decrease
maintenance cost and allows for more concentration of forces (because of
stacking limits).

\aparag[Detachments of Exploration.] One \NDE represents roughly one or two
warships (one third of a \NWD).
\bparag \NDE can exists both on the \ROTW and European maps.
\bparag One \NWD (only) can be split in 3 \NDE at any time (especially to
satisfy losses). 3 \NDE stacked together can be turned into 1\NWD at any time.
\bparag For maintenance and purchase, 1\NDE costs as much as half a \NWD.
\bparag \NDE are never counted in stacking and supply limits.

\aparag[Pirates.] The last naval forces are pirates and privateers. They
represent independent sailors that attack trade fleets. Privateers (\corsaire)
work for one country; pirates are represented by the (abstract) minor country
\pays{pirates} (who mostly has \corsaire units). \corsaire have to be
maintained or bought.

\aparag[Trade fleets] (\TradeFLEET) are not naval forces. They only represent
trade activity (not specific ships), do not move and can only be attacked by
pirates and privateers.

\begin{exemple}
  In period \period{III}, the size of English fleets is ``2/1:5/1''. Thus, a
  \FLEET\Facemoins of \ANG may contain up to 2\NWD and 1\NTD while a
  \FLEET\Faceplus may contain up to 5\NWD and 1\NTD.

  If \ANG wishes to group together 3\NWD (and no \NTD), it must use a
  \FLEET\Faceplus (and pay the maintenance cost for one) because this cannot
  fit within one \FLEET\Facemoins.

  In period \period{III}, \TUR has also a fleet size of ``2/1:5/1''. However,
  since \NGD only count as half a \ND in fleet countenance, one
  \FLEET\Faceplus of \TUR may hold up to 10\NGD and 1\NTD.
\end{exemple}



\subsection{Fortifications}

\aparag Fortifications are immobile forces used to defend provinces. There are
two kinds of fortifications: fortresses and forts. In Europe, fortifications
represent the whole defence system of the province thus including several
actual fortresses, citadels, fortified towns, \ldots
\bparag Fortifications are also supply sources for both land and naval troops.

\aparag[Fortresses] have a level between 1 and 5.
\bparag Each European province, as well as some \ROTW provinces, has a basic
fortress of level either 1 or 2 drawn of the map.
\bparag Fortresses of higher level may be built provided the country has a
sufficiently high technology.
\bparag Fortresses may lose levels due to sieges. If this puts the fortress
below its basic level, use the white level 1 counters to denote it. In no case
can the fortress of an European province go below 1.
\bparag Note that fortresses counters are double-sided. Thus, building a
fortress prevents a country from building the one on the back of the
counter. It is always possible to switch one fortress counter for another (of
the same level (and country)) if the need arise.

\aparag[Forts] are sometimes referred to as ``level 0 fortresses''. They may
only exist in the \ROTW.
\bparag All colonial establishments (\COL and \TP), as well as missions
automatically have a fort.
\bparag Other forts may be built during the military phase by land forces.
\bparag A \COL of level 6 is considered to be an European province. Thus, it
gains for free a basic fortress of level 1. Use white level 1 counters to
denote it.

\aparag[\Presidios] are small fortifications built in enemy territory to try
and control access to the sea rather than the land itself.
\bparag In European provinces where there is a circled anchor (whatever its
colour), a foreign country may build a
\Presidio. See~\ref{chRedep:Presidios} for building it
and~\ref{chMilitary:Presidios} for its effects.

\aparag[Arsenals.] Some countries have fortresses counters with a gold anchor
on them. These are \terme{arsenals}.
\bparag Arsenals can only be built in the \ROTW (exception:
\construction{Gibraltar}, \construction{Sebastopol} and
\construction{Saint-Petersbourg}).

\aparag[Land Supply.]
\bparag Forts may only supply detachments (\LD or \LDE).
\bparag Other fortresses can supply any number of land troops, whatever the
level of the fortress.
\bparag \COL and \TP, although they only have a fort, are supply sources for
any number of land troops (that is, the establishment has more supply capacity
that its fortification level).

\aparag[Naval Supply.]
\bparag Forts may only supply detachments (\ND or \NDE).
\bparag Regular (non-arsenal) ports can supply any number of naval stacks
containing at most one \FLEET counter (each), whatever the level of the
fortress.
\bparag Arsenal can supply any number of naval forces of any size.
\bparag \COL and \TP, although they only have a fort, are supply sources as
regular ports: each can supply any number of naval stacks containing at most
one \FLEET counter (that is, the establishment has more supply capacity that
its fortification level).

\begin{designnote}
  Note that supply limits are cumulative. That is, a single fortress may
  supply as many stacks (land or naval) as wanted, as long as it can supply
  each of them individually. There is no ``using up'' of the supply capacity.

  The ``extra supply capacity'' of \COL or \TP (with respect to their
  fortification level) is reminded is the size of the counter: they use big
  counters because they have a lot of food.
\end{designnote}



\subsection{Veteran and Conscripts}

\aparag[Veterans and Conscripts.] All land forces can be either
\terme{Veteran} or \terme{Conscripts}. A \terme{Veteran} army has seen more
battles than conscripts, are better trained, and less likely to flee in the
presence of the enemy. A \terme{Conscript} army is formed of newer soldier and
paid less.
\bparag \terme{Veteran} have a bonus in battle (better moral). However, their
maintenance cost is also higher.
% forces have a morale higher by 1 point compared to
% \terme{Conscripts}. However, the maintenance cost of \terme{Conscripts}
% units is lesser than the \terme{Veteran} ones.

\aparag[Who is Veteran?] If the country is at peace (being only engaged in
\terme{Overseas Wars} (see \ref{chDiplo:Overseas Wars}) and limited
interventions (see \ref{chDiplo:InterventionLimitee}) counts as being at
peace), all land forces are maintained as \terme{Veteran} forces, using the
\terme{Peace maintenance} price.
\bparag If the country is at war, then all land forces already existing at the
beginning of turn can be maintained (unit by unit) as either \terme{Veteran}
or \terme{Conscript}. Newly recruited units are \terme{Conscripts}.

\aparag[Mixed stacks.] A force formed by stacking or merging several units is
\terme{Veteran} only if more than half of the \LD composing the units are
\terme{Veteran}.
\begin{exemple}
  An \ARMY\faceplus composed by the merging of 1 \terme{Veteran} \LD and 1
  \terme{Veteran} \ARMY\facemoins and 1 \terme{Conscript} \LD is considered to
  be a \terme{Veteran} unit. But if this \ARMY\faceplus is stacked with an
  \ARMY\facemoins of \terme{Conscripts}, this stack is considered
  \terme{Conscripts} (since there are as many \terme{Conscripts} as there are
  \terme{Veteran}). However, if one \LD of this stack is destroyed (due to
  battle or attrition), one \LD of \terme{Conscripts} will be removed (leaving
  either 2\ARMY\facemoins and 1\LD or 1\ARMY\faceplus and 1\LD), and the stack
  as a whole will be \terme{Veteran}.
\end{exemple}

\begin{designnote}
  Think twice before upkeeping troops as \terme{Conscripts}. The extra moral
  will make a huge difference in battles.
\end{designnote}

\aparag[Navy] Naval forces are \terme{Veterans} if they are maintained from a
previous turn, or \terme{Conscripts} if there are newly recruited this turn.
\bparag The only difference will be for Navy of technology \terme{Vessel} or
\terme{Three-decker}.% \terme{Veteran} naval forces have morale 4 instead of 3.

%%%%%%%%%%%%%%%%%%%%%%%%%%%%%%%%%%%%%%%%%%%%%%%%%%%%%%%%%%%%%%%%%%%%%%




\section{Maintenance}\label{chExpenses:Maintenance}

\aparag Each turn, forces existing from a previous turn must be maintained or
disbanded. Maintaining troops costs money.



\subsection{Basic forces}

\aparag Depending on the period, each country is entitled to some \terme{basic
  forces}. These forces are maintained for free. \terme{Basic forces} can be
found in the player's aids.
\bparag The units maintained as part of the \terme{basic forces} are
maintained as \terme{Veteran}.

\aparag The basic forces comes in three kinds: specific land forces (such as
\ARMY\faceplus, \ARMY\facemoins, 3\LD), specific naval forces (such as
\FLEET\faceplus, etc.) or generic detachments (\GD) that can either serve as
\LD or \ND.
\bparag In some cases (see below), it is possible to convert basic forces from
one denomination to another. For these conversions (only), use the
equivalences: 1\ARMY\Facemoins=2\LD, 1\ARMY\Faceplus=2\ARMY\Facemoins=4\LD ;
1\FLEET\Facemoins=2\ND, 1\FLEET\Faceplus=2\FLEET\Facemoins=4\ND. Note that for
naval forces this can be sensibly different from the actual content of \FLEET.

\aparag If possible, the \terme{basic forces} must be used to maintain
counters that are the same size or larger than them.
% (by combination), using 2\LD=\ARMY\facemoins and
% 2\ARMY\facemoins=\ARMY\faceplus.

\aparag If all \terme{basic forces} cannot be used to maintain counters that
are the same size or larger than them, then the rest can be converted to \LD
(or \ND) and used to maintain forces of any size.

\aparag At most one land counter and one naval counter may be partially
maintained with the \terme{basic forces} (due to the fact that already
deployed forces are larger than the \terme{basic forces}).
\bparag The units partially maintained with \terme{basic forces} will have to
be maintained with a \terme{Veteran} maintenance.

\aparag Some \terme{basic forces} include \corsaire counters. These forces
cannot be converted to anything else. If no \corsaire is used, the basic force
is lost.

\aparag[Effect of Wood.]\label{chExpenses:Effect of Wood Maintenance} Each
\RES{Wood} resource, that can be either bought or produced by a \MAJ,
increases the \terme{basic forces} by 1 \ND.
\bparag A \MAJ may use up to 3 \RES{Wood} each turn (thus gaining up to 3\ND
of basic force).



\subsection{Extra Maintenance}

\aparag All the units that are not maintained by \terme{basic forces} must be
payed for in order to be kept.
\bparag If a unit is partially maintained, the part which is not maintained is
destroyed.
\bparag Maintenance costs depends both on the technology and the country. They
can be found in the player's aids.

\aparag The maintenance cost for land units also depends on the state of war
of the \MAJ.
\bparag A country at peace (including if it is only engaged in \terme{Overseas
  Wars} (see \ref{chDiplo:Overseas Wars}) and limited interventions (see
\ref{chDiplo:InterventionLimitee})) uses the \terme{Peace maintenance} and its
land forces are automatically \terme{Veterans}.
\bparag A country at war must choose, for each counter, whether it is
maintained as \terme{Veteran} or \terme{Conscript} and use the corresponding
price.
\bparag A counter cannot be maintained partially as \terme{Veteran} and
partially as \terme{Conscripts}.

\aparag the maintenance price is found by cross-referencing the country's
technology with the kind of maintenance used (size of counter and
Veterans/Conscripts status).

\aparag A counter must be maintained as a whole. That is, it is not possible
to maintain one \ARMY\Facemoins for the price of two \LD.
\bparag However, it is possible to partially maintain a counter and have the
rest destroyed. That is, one \ARMY\Facemoins may be broken before maintenance
and only one of the two \LD maintained (and the other is destroyed).

\aparag For troops partially maintained by \terme{basic force}, use the
conversion of \terme{basic forces} to determine what is left to be payed.
\bparag Example: if a \FLEET\Faceplus is partially maintained by 3\ND
(=\FLEET\Facemoins, \ND) of \terme{basic force}, only 1\ND is missing and must
be payed as extra maintenance (whatever the actual content of the \FLEET).

\aparag \corsaire that are not included in \terme{basic forces} may not be
maintained. They can, however, be recruited anew.


\aparag the sum of all extra maintenance costs is written in \lignebudget{Unit
  maintenance}.

\begin{exemple}[Maintenance of forces]
  \SPA is with technologies \TREN\ and \TGLN, at war, during period II. It has
  a basic force in period II of 2\ARMY\faceplus and 3\GD. Its existing troops
  are 2\ARMY\faceplus, 1\FLEET\faceplus and 4\LD and it wants to maintain all
  of them.

  Since basic forces must first be used to maintain unit of the same or larger
  size, the 2\ARMY\Faceplus of basic forces must be used to maintain the
  2\ARMY\Faceplus counters (\emph{i.e.} they may not be used to maintain the
  \LD). However, the 3\GD may be used either to maintain some \LD and/or (part
  of) the \FLEET.

  \SPA may choose to use the 3\GD to maintain 3\LD. This leaves 1\LD and
  1\FLEET\Faceplus to pay for. The \LD may be maintained as either Veteran
  (7\ducats) or Conscript (4\ducats) while the \FLEET\Faceplus costs 80\ducats
  to maintain.

  The less expensive solution is to use the 3\GD to partially maintain the
  \FLEET. Since, for maintenance purpose, 1\FLEET\Faceplus is considered to be
  4\ND, this leaves only 1\ND to maintain (17\ducats) and the 4\LD must also
  be maintained (for 4 or 7 \ducats each, depending on their status).

  It is also possible to use 2\GD to maintain \FLEET\Facemoins and the third
  to maintain one \LD. This leave 3\LD and \FLEET\Facemoins to maintain.

  \smallskip

  If \SPA only has 1\ARMY\Faceplus, 4\LD and 1\FLEET\Faceplus, then it is
  possible to use the second \ARMY\faceplus of basic force to maintain the the
  4\LD (because there is no counter of same or larger size, thus the remaining
  \ARMY\Faceplus of basic forces is turned into 4\LD).

  \smallskip

  If \SPA only has 2\ARMY\Faceplus and 3 \ARMY\Facemoins (and no naval
  forces), then the 2\ARMY\Faceplus of basic forces must be used to maintain
  the 2\ARMY\Faceplus of actual troops. The, the 3\ARMY\Facemoins will use the
  3\GD. However, since at most 1 land counter may be partially maintained by
  basic forces, it is not possible to use each \GD to partially maintain
  1\ARMY\Facemoins (leaving 3\LD to pay). Thus, \SPA must use 2\GD to maintain
  1\ARMY\Facemoins, then 1\GD to partially maintain the second
  \ARMY\Facemoins. This leaves 1\LD (7\ducats, since it is a partially
  maintained counter, it must be maintained as Veteran) and 1\ARMY\Facemoins
  (12 or 8 \ducats) to pay for.
\end{exemple}

\begin{designnote}
  Since all administrative actions must be planned (and payed) before any is
  resolved, troops are always maintained using the cost for the technology
  that the country had at the beginning of the turn. Indeed, maintenance is
  planned and payed at the same time as administrative actions, thus it is not
  known whether the new technology will be reached or not.
\end{designnote}



\subsection{Maintenance of fortresses}

\aparag All fortresses that are not at their basic level have to be maintained
in activity (including \Presidios). The cost of maintaining a fortress is
indicated in the last column of~\ref{table:Cost of Fortresses}.
\bparag The maintenance cost of a fortress is payed by its controller.
\bparag The maintenance cost of a fortress in Europe is its level.
\bparag Maintenance is doubled in the \ROTW. A fort requires a maintenance of
1\ducats.
\bparag Maintenance is doubled for fortresses of level 3 before obtaining the
technology \TARQ (representing the spreading of the ``\emph{trace italienne}''
during the Wars in Italy).
\bparag Maintenance is doubled for fortresses of level 4 before Turn 40
(representing the spreading of the ideas of \leader{Vauban}).
\bparag The free forts given by \TP or \COL do not need any maintenance. The
ones given by missions do, even if the local fort/fortress is of a higher
level. The free level 1 fortress of a level 6 \COL remains free also.
\bparag The level of a fortress can be lowered by its owner if he controls
it. A lowered fort is destroyed.

\aparag[Arsenals] An arsenal is maintained at the same cost as a fortress of
the same level.

\aparag the sum of all maintenance costs of fortifications is written in
\lignebudget{Fort. and presidios maintenance}.

\GTtable{fortresses}



\subsection{Maintenance of Minor Powers}
\label{chLogistic:Maintenance of minors}
\aparag[At peace] a \MIN maintains up to its \terme{basic forces}. Extra
forces (troops, navy, fortifications) are destroyed.
% \bparag If a \MIN is at peace during one full turn, its \terme{basic forces}
% are automatically recruited to their maximum.
\bparag Exception: a \MAJ may pay to maintain fortresses of a \VASSAL.
\bparag Exception: former \MAJ (\paysPortugal, \paysVenise and \paysPologne)
maintain all their fortresses for free (unless otherwise specified, typically
for non-absolutist \paysPologne).

\aparag[At war] a \MIN maintains up to its \terme{basic forces}.
\bparag The diplomatic Patron of a \MIN fully involved in a war, may maintain
any or all forces above the \terme{basic forces} of the \MIN, up to its
counter allowance.
\bparag The costs are the same as those of the \MAJ for the Technology of the
\MIN.
\bparag A \MAJ may pay to maintain fortresses of a \VASSAL.
\bparag Other minors (non \VASSAL) at war maintains all their fortifications
in addition to their \terme{basic forces}.
\bparag Minors fully at war without \MAJ allies maintain all the forces they
have.

\aparag[Moral] Troops maintained by minors are always \terme{Veteran}.
\bparag Troops maintained by the diplomatic patron are either \terme{Veteran}
or \terme{Conscript}, depending on the cost payed.

\begin{designnote}[Cost of maintenance] The cost may vary according to the
  technologies and to the countries, but the cost per \LD is usually higher
  for the \ARMY\facemoins, then the \LD, then the \ARMY\faceplus. That is, the
  cost for 1\ARMY\Facemoins is usually more than twice the cost for 1\LD while
  the cost for 1\ARMY\Faceplus is usually less than the cost of 4\LD. To
  achieve the cheapest maintenance (but it may not be always the best due to
  the \terme{Veteran} distinction), it is better to try and use basic forces
  to maintain those in this order.

  Similarly, the cost per \ND (on the basis of 1\FLEET\facemoins=2\ND) is
  usually \FLEET\facemoins, \ND, \FLEET\faceplus. However, a \FLEET\facemoins
  contains more than 2\ND (sometimes a lot more), so fleets are usually a
  better way to maintain the naval forces (if they are regrouped enough). As
  for the \GD, they are usually best used as \ND (costing much more than \LD
  to pay), unless the naval forces are few enough to be covered by the naval
  allowance.
\end{designnote}

%%%%%%%%%%%%%%%%%%%%%%%%%%%%%%%%%%%%%%%%%%%%%%%%%%%%%%%%%%%%%%%%%%%%%%




\section{Recruitment}\label{chExpenses:Recruitment}



\subsection{Land forces}


\subsubsection{Land recruitment in Europe}
\aparag[Limit]
\bparag Each country has a recruitment \terme{limit}, expressed in \LD,
varying by periods (and some specific conditions).
\bparag It can be found in the player's aid of each country, in the column
``Troops purchase''.
\bparag It is also summarised in~\ref{table:Recruitment per Country} (first
line for each country).
\bparag Each turn, it is possible to recruit up to 3 times this limit.

\aparag[Costs]
\bparag Each country has a recruitment cost, for \LD and \ARMY\Facemoins,
varying with its current technology.
\bparag It can be found in the player's aid of each country, in the columns
``Land Purchase''.
\bparag The cost for one \ARMY\Facemoins is usually the cost of 2\LD
(exception: \RUS).
\bparag In order to buy an \ARMY\Faceplus, a country usually buys
1\ARMY\Facemoins and 2\LD and immediately (during the administrative phase)
turn them into 1\ARMY\Faceplus. It is also possible to do so by buying
2\ARMY\Facemoins but it is usually more expensive and requires two \ARMY
counters instead of one.

\aparag[Recruitment area.] Each country has a \terme{Recruitment area}. Unless
specified in its specific rules, it is all the provinces in its national
territory.
\bparag Exceptions: \HIS, \TUR, \RUS and \SUE.

\aparag[Recruitment.] Each country decides how many troops it wants to
purchase, where it wants them recruited and under which form (counters). Then
it computes the cost for these.
\bparag Recruitment can only takes place in owned, controlled, not besieged
and not revolted provinces.
\bparag The cost is written in \lignebudget{Units purchase} (together with the
cost of newly brought navies).
\bparag The new units are put on the map when resolving administrative
actions.

\aparag[Multipliers.] Counters are brought in order chosen by the player. The
total number of \LD recruited so far is tallied and compared with the
\terme{limit} to compute the exact price of the counter.
\bparag Any counter that can fully be recruited under the limit is paid at the
cost listed in the tables.
\bparag Any counter that cannot be recruited under the limit but can be
recruited under twice the limit is paid at twice the price listed.
\bparag Any counter that cannot be recruited under twice the limit but can be
recruited under thrice the limit is paid at thrice the price listed.
\bparag Any counter that cannot be recruited under thrice the limit cannot be
brought.
\bparag Any counter recruited out of the \terme{Recruitment area} has its cost
doubled.
\bparag These multipliers are cumulative.

\begin{playtip}
  The order of recruitment can be important in some cases. Because of the
  multipliers, it can change the price one pays for the counters (see the
  examples below).

  Do not put new counters immediately on map. Recruitment is supposed to be
  simultaneous, that is all countries plan which troops they want to buy and
  then put them on the map. If you put your troops on the map before your
  opponent has finished planning his actions, don't complain that he changes
  his mind and decide to recruit more (or less) troops\ldots

  When buying \ARMY\Faceplus or reinforcing \ARMY\Facemoins by buying 2\LD for
  them, you can directly put the \ARMY\Faceplus counter on the map (rather
  than placing 1\ARMY\Facemoins, 2\LD and immediately turning them into
  1\ARMY\Faceplus). Especially, stacking limits are not enforced between the
  purchase and the conversion.
\end{playtip}

\aparag[Moral.] The newly recruited troops are always \terme{Conscripts}
(except for \SUE and \PRU).

\GTtable{recruitment}

\begin{exemple}[Under the limit]
  \HIS is \TARQ in period \period{III} (recruitment of 5\LD), and wishes to
  recruit 1\ARMY\Faceplus in its recruitment area. Since 1\ARMY\Faceplus is
  4\LD, this is less than the limit and can be recruited at simple cost. \HIS
  recruits 1\ARMY\Facemoins and 2\LD for 60\ducats and immediately turn them
  into 1\ARMY\Faceplus. Actually, it is easier to directly put the
  \ARMY\Faceplus in play to save time and manipulations.
\end{exemple}

\begin{exemple}[Limit and whole counters]
  \HIS wants to buy 3\ARMY\facemoins with its 5\LD recruitment limit.  The
  first two correspond to 4\LD total and can thus be recruited at simple cost
  for 30 + 30\ducats. The third one, however, makes the total goes to 6\LD,
  over the limit but under twice the limit. So, it must be recruited at double
  cost (for another 2\textmultiply 30\ducats). Even if 1\LD of the third
  \ARMY\facemoins is still within the limit, the whole counter price is
  doubled because it makes the total number of \LD recruited go over the
  limit.

  \HIS, however, could choose to buy 2\ARMY\facemoins and 2\LD. In this case,
  both \ARMY\facemoins and the first \LD are within the limit and only the
  last \LD is payed at double cost. This, however, produce fewer \ARMY
  counters and hence might not be the wisest solution.
\end{exemple}

\begin{exemple}[Small limits and big counters]
  In period \period{II}, \POR has 3\LD of limit and is at technology \TREN. It
  wants to recruits 1\ARMY\Faceplus in \provinceTejo (in the recruitment
  area). \POR does so by recruiting 1\ARMY\Facemoins and 1\LD under the limit
  for 24 + 12 \ducats and another \LD above the limit for 2 \textmultiply
  12\ducats. The total is thus 60\ducats and \POR can directly put the
  \ARMY\Faceplus in play.

  Note that if \POR wanted to recruit its \ARMY\Faceplus by merging
  2\ARMY\Facemoins instead of 1\ARMY\Facemoins and 2\LD, then the second
  \ARMY\Facemoins does not fit within the limit and thus has to be paid
  entirely at double cost bringing the total to 24 + 2 \textmultiply 24 =
  72\ducats. Moreover, this is simply impossible because it would require
  2\ARMY counters and \POR has only one (even if the \ARMY\Faceplus can be
  directly put on the map, the two \ARMY\Facemoins are virtually here at some
  point during the process).
\end{exemple}

\begin{exemple}[When order matters]
  In period \period{I}, \POR has 2\LD of limit and is at technology \TMED. It
  wants to recruits 2 \LD in \provinceTejo (in the recruitment area) and 2 in
  \provinceTanger (out of the recruitment area).

  \POR may chose to first recruit the 2\LD in \provinceTejo. Since they are
  below the limit and in recruitment area, the cost is not multiplied and they
  cost 10 + 10 = 20\ducats. Then , \POR recruits the 2\LD in
  \provinceTanger. Since they are above the limit, their cost is
  doubled. Since they are recruited out of the recruitment area, their cost is
  doubled a second time. Thus, they cost 2 \textmultiply 2 \textmultiply (10 +
  10) = 80\ducats. The total cost is 100\ducats.

  On the other hand, \POR could first recruit the \LD in
  \provinceTanger. Thus, they are below the limit and the cost is only doubled
  once (for being recruited out of the area) for 2 \textmultiply 20 =
  40\ducats. Then \POR recruits the \LD in \provinceTejo. Since they are above
  the limit, their cost is doubled for 40\ducats. But the total is only
  80\ducats.

  Because multipliers are cumulative, the order in which troops are recruited
  may change the final price.
\end{exemple}

\begin{exemple}[Big computation]
  \HIS is \TARQ in period \period{III} (recruitment of 5\LD), and wishes to
  recruit 2\ARMY\facemoins, 1\ARMY\faceplus and 3\LD, 2 of which are not in
  its \terme{Recruitment Area}. The \ARMY\faceplus is bought with
  1\ARMY\facemoins and 2\LD, which brings the total to 3\ARMY\Facemoins and
  5\LD.  The cheapest way to purchase this is to purchase 2\ARMY and 1\LD (out
  of \terme{Recruitment Area}) under the limit (for the cost of
  30+30+2\textmultiply 15), 1 \ARMY, the second double-cost \LD and two other
  \LD under twice the limit (for 2\textmultiply30+4\textmultiply
  15+2\textmultiply 15+2\textmultiply 15) and the last \LD for 3\textmultiply
  15, which brings the total to 315\ducats. This is big, even for the Spanish
  treasury!
\end{exemple}

\begin{playtip}
  Even if it was frequent in the examples, recruiting above the limit is
  uncommon and recruiting at thrice the price is a very rare
  occurrence. Especially, when there are very few troops above the limit, it
  is often better to simply recruit a bit less and wait for next
  turn. Typically, in the last example, the 11th \LD cost 45\ducats alone and
  it would probably be better to simply not recruit it right now.

  This means that recruitment sometimes has to be planned a bit in
  advance. Especially for countries with small limits (typically, \ANG). If
  you plan to go on war, you may want to recruit one turn in
  advance. Obviously, that would cost the maintenance for one turn, but that
  will lower the recruitment cost at the crucial time and save money for
  campaigning (maintenance is lower than recruitment, especially at
  war). Moreover, that allows more troops to become veteran.

  On the other hand, sometimes you just suffer several defeats and lose many
  troops and need to raise them asap if the war goes on. Recruiting above the
  limit when at peace is rarely a good idea. Similarly, maintaining a large
  army at peace is very costly and it is cheaper to demobilise it. But raising
  it again will require a bit of planning if one wants all its troop at the
  right time\ldots
\end{playtip}


\subsubsection{Land recruitment in the \ROTW}
\aparag Recruitment in the \ROTW is even more restricted than in European
provinces that are not part of the recruitment area.
\bparag European provinces in the \ROTW are considered as European provinces.
\bparag Even if they are usually considered as European provinces, recruitment
in level 6 \COL is also restricted.
\bparag Troops recruited in the \ROTW are tallied against the limit, and their
price might be doubled or tripled, just like other troops. This is the same
limit as in Europe: both recruitment in Europe and in the \ROTW are added to
know if the limit has been reached.

\aparag[\COL level 6.] In a \COL of level 6, it is possible to recruit each
turn up to 2\LD (at normal cost) or 1\ARMY\Facemoins at double cost.

\aparag[Other Establishments.] In other \COL/\TP, it is possible to recruit
either 1\LDE at normal cost or 1\LD at double cost.
\bparag It is not possible to recruit in a mission or a fort.

\aparag[Exploration.] A \LDE count as half a \LD for recruitment purposes: its
price is half the price of 1\LD, rounded up ; and it is considered as half a
\LD in the recruitment limit.

\begin{exemple}[Recruitment in the \ROTW]
  During period \period{I}, \POR is \TREN and has a recruitment limit of
  2\LD. \POR wants to recruit 1\ARMY\Facemoins in \construction{Goa} (a level
  6 \COL), 2\LD in \provinceTejo (in the recruitment area) and 1\LD in
  \continentBrazil (in level 2 \COL in the \ROTW).

  \POR can choose to recruit first the \LD in Europe for 12 + 12 = 24\ducats,
  then the \ARMY\Facemoins at quadruple cost (doubled because it is an
  \ARMY\Facemoins in a \COL of level 6 and doubled because it is above the
  limit) for 2 \textmultiply 2 \textmultiply 24 = 96\ducats, and lastly the
  \LD in \continentBrazil at 6 times the cost (thrice for being above twice
  the limit and doubled for recruiting 1\LD in the \ROTW) for 3 \textmultiply
  2 \textmultiply 12 = 72\ducats for a grand total of 24 + 96 + 72 =
  192\ducats.

  By recruiting first the \ARMY\Facemoins, then the Brazilian \LD and lastly
  the European \LD, \POR would have paid only 156\ducats.

  If \POR wants to recruit 1\LD in Europe and 3\LDE in 3 different \ROTW
  establishments, the cheapest way is to recruit first the \LD for 12\ducats,
  then the first 2\LDE at normal cost for (12/2) + (12/2) = 12\ducats and
  lastly the third \LDE is above the limit (even if a \LDE represent a third
  of a \LD, it takes half of an \LD in recruitment limit), thus at double cost
  for another 12\ducats. The total is thus 36\ducats. If \POR starts by
  recruiting the 3 \LDE at simple cost, then part of the \LD is above the
  limit and it must be payed at double cost for a total of 42\ducats.

  Note that it is not allowed, during a single turn, to recruit several \LDE
  at the same place ; or to recruit 3\LD or more (including 1\ARMY\Faceplus)
  in a \COL of level 6.
\end{exemple}



\subsection{Purchasing fortresses}

\aparag[Generalities.] Fortresses can be raised above the level indicated on
the map, up to level 5.
\bparag Each turn, the level of each fortress may only increase by 1.
\bparag Fortresses of high level may only be built in the late game when the
corresponding land technology is reached.
\bparag A fortress may only be built in a controlled, not besieged and not
revolted province. Note that ownership of the province is not required.
\bparag \Presidios are not built as other fortresses.
\bparag The ``fortress'' counters can be exchanged at will (they are two-sided
counters, and not always equivalent). A combination of counters with the
desired levels has to exist to be allowed to build fortresses.

\aparag[Technology.] In order to raise a fortress to a given level, a country
must have at least the land technology indicated in the ``Required
Technology'' column of~\ref{table:Cost of Fortresses}. Note that since all
administrative actions are planned before any is resolved, that means that one
cannot increase a fortress on the same turn it reaches the required
technology.
\bparag In addition, no fortress of level 5 may be built before turn 40.

\aparag[Cost.] The cost for each level of fortress is indicated in the
``Cost'' column of~\ref{table:Cost of Fortresses}.
\bparag The first number is the cost to build a fortress of this level in
Europe (usually, 25\ducats per starting level of the fortress). The second is
the cost to build a fortress of this level in the \ROTW (usually, twice more).
\bparag The cost of all fortresses of level 3 is doubled for countries that do
not have the land technology \TARQ (representing the spreading of the
``\emph{trace italienne}'' during the Wars in Italy).
\bparag the cost of all fortresses of level 4 is doubled before turn 40
(representing the spreading of the ideas of \leader{Vauban}).
\bparag The total cost for building fortresses is recorded in
\lignebudget{Fort. purchase}.

\aparag[\ROTW.] \COL of level 6 are treated like European provinces for
building fortresses.
\bparag In other provinces, it is not possible to build a fortress of level
higher than 2, unless it is an arsenal.
\bparag Fortresses of level 1 can be build in any \COL or \TP.

\aparag[Arsenals]\label{chExpenses:Build Arsenals} are built instead of
a fortress of the same level. That is, instead of building a fortress of a
given level, one can build an arsenal of the same level. All conditions (and
price) for building this level of fortress must be met (or paid).
\bparag Except for the named arsenals of \construction{Gibraltar},
\construction{Sebastopol} and \construction{Saint-Petersbourg}, arsenals may
only be built in a coastal \TP or \COL (including \COL of level 6).
\bparag The named arsenal \construction{Brazilie} can only be built in
\continent{Brazil}.
\bparag The named arsenal \construction{Gibraltar} can only be built in
\province{Gibraltar} or \province{Tanger}.
\bparag The named arsenal \construction{Sebastopol} can only be built in a
province bordering the \regionNoire.
\bparag The city \ville{Saint-Petersbourg} is also an arsenal
\construction{Saint-Petersbourg} with specific rules attached to its
construction (see~\ruleref{chSpecific:Russia:St-Petersburg}).

\aparag[Forts] are built during the military or redeployment phases.
\bparag A \LD is required and the construction takes 2 rounds.
\bparag A \LeaderMis can be transformed in a Mission during the redeployment
phase.

\aparag[Wasteland]\label{chExpenses:Fortresses Wasteland} In the Wasteland (see
\ruleref{chBasics:Wasteland}), until 1615 (periods \period{I}--\period{III}),
fortresses may not be more than 1 level higher than the basic level one map
(\emph{i.e.} the maximum level is 2 on provinces with a basic level of 1 and 3
in provinces with a basic level of 2).
\bparag After 1615 (periods \period{IV}--\period{VII}) and until the
construction of \ville{Saint-Petersbourg}, the limit becomes 2 levels higher
than the one on the map.
\bparag After the construction of \ville{Saint-Petersbourg}, all limits are
removed.
\bparag This does not preclude other conditions on fortresses level (such as
Land Technology level).

\aparag[\Presidios] are built only according to rule \ref{chRedep:Presidios},
during the redeployment phase of the turn. The maximum level of any \Presidio
is 3.
% \bparag
% Yet, an existing \Presidio (not a Blockading \Presidio) can not be increased
% at the Logistic Phase; it can only be increased if at war and besieging or
% controling the enemy fortress, by means of rule
% \ref{chRedep:Presidios}.
\bparag As an exception to fortress building, \Presidio can be constructed
directly at any level. The cost is the sum of cost for all intermediary
levels.
% \bparag It is not possible to recruit forces in a \Presidio.

\begin{exemple}[Building fortresses]
  At turn 3, \FRA has \TREN and, being at war against \HIS, wants to fortify
  its Southern border. \FRA would like to build fortresses of level 3 both in
  \provinceBearn and \provinceLanguedoc. However, since the current fortress
  of \provinceLanguedoc is only of level 1 (the default level for this
  province), it is not possible to go directly to level 3.

  So, \FRA decides to increase by one level the fortresses of \provinceBearn
  and \provinceLanguedoc. In \provinceLanguedoc, \FRA builds a fortress of
  level 2 (on top of the existing level 1) for 25 \ducats. In \provinceBearn,
  \FRA build a fortress of level 3. Since \FRA is not \TARQ yet, the cost is
  doubled for 100\ducats. The total is 125\ducats, to be written in
  \lignebudget{Fort. purchase}. \FRA now has to find the proper counters in
  its counters mix to put on the map (this should be easy at this point).

  In turn 4, the war is still going on, so \FRA wants to increase the fortress
  of \provinceLanguedoc. Since there is already a fortress of level 2, it is
  possible to build a level 3 here, for 100 \ducats (since \FRA is still not
  \TARQ).
\end{exemple}

\subsection{Naval forces}\label{chExpenses:Naval Purchase}
\subsubsection{Naval recruitment in Europe}
\aparag[Limit]
\bparag Each country has a recruitment \terme{limit}, expressed in \ND,
varying by periods (and some specific conditions).
\bparag It can be found in the player's aid of each country, in the column
``Troops purchase''.
\bparag It is also summarised in~\ref{table:Recruitment per Country} (second
line, first number).
\bparag Each turn, it is possible to recruit up to this limit. Contrary to
land recruitment, it is not possible to recruit more than the naval
limit.
\bparag \NGD and \NDE count as half a \ND for this limit. \NTD and \VGD count
as a full \ND.

\aparag[Increasing the limit]
\label{chExpenses:Effect of Wood Purchase}
\label{chExpenses:Effect of Fish Monopoly Purchase}
\bparag Each \RES{Wood} brought or produced increase the naval recruitment
limit by 1\ND, up to a maximum augmentation of 3\ND.
\bparag In addition, a country having a partial or total monopoly on
\RES{Fish} adds 1\ND to its naval recruitment limit.

\aparag[Navy size]
\bparag Each country has a maximum number of \ND allowed on map at the same
time, varying by periods. This counts both \ND counters and \ND in \FLEET
counters.
\bparag It can be found in the player's aid of each country, in the column
``Max. \ND''.
\bparag It is also summarised in~\ref{table:Recruitment per Country} (second
line, second number).
\bparag \NGD and \NDE count as half a \ND toward this limit. \NTD and \VGD
count as a full \ND.

\aparag[Costs]
\bparag Each country has a recruitment cost, for \NWD, \NTD, \FLEET\Facemoins,
and sometimes also for \NGD and \FLEET\Facemoins of \NGD, varying with its
current technology.
\bparag It can be found in the player's aid of each country, in the columns
``Navy Purchase'' and ``Purchase (other)''.
\bparag The cost for one \FLEET\Facemoins is the cost for a full counter, up
to its countenance.
\bparag Beware! Countries with small recruitment limit (\emph{e.g.} \POL or
\RUS) may not recruit a \FLEET\Facemoins in one turn unless they first
increase their limit. That is, the existence of a price for a counter does not
remove other conditions for buying it.
\bparag Even if they do not benefit from the technologies, the cost of \NGD
varies with them.
\bparag \NDE cost half the price of a \ND, rounded up.

\aparag[Recruitment area.] There is no specific recruitment area for
navies. They can be brought in any European province with a port, including
European provinces in the \ROTW and (coastal) \COL of level 6.

\aparag[Recruitment.] Each country decides how many ships it wants to
purchase, where it wants them recruited and under which form (counters). Then
it computes the cost for these.
\bparag The cost is written in \lignebudget{Units purchase} (together with the
cost of newly brought armies).
\bparag The new units are put on the map when resolving administrative
actions.
\bparag Since \FLEET counters are containers, it is possible to recruit some
\ND directly ``inside'' them (if there is still room left for them) without
physically putting the \ND counter on the map.

\begin{exemple}[Naval recruitment]
  In Period~\period{IV}, the recruitment limit of \SUE is 4\ND. Since its
  \FLEET\Faceplus contains 5\ND and 2\NTD, \SUE may not buy one of them in one
  turn.

  If the naval technology of \SUE is \TBAT, \SUE can in on turn buy one
  \FLEET\Facemoins (containing 2\ND and 1\NTD) for 150\ducats, plus an
  additional \NWD at 55\ducats. This makes a total of 4\ND, the recruitment
  limit for a given turn. The \NWD may be directly incorporated within the
  \FLEET, since a \FLEET\Facemoins is too small to contain 3\ND, the counter
  is turned ~\faceplus (and will require maintenance of a full
  \FLEET\Faceplus, thus it might be way cheaper to keep this stack as
  1\FLEET\Facemoins and 1\ND). On the next turn, \SUE may buy another
  \FLEET\Facemoins and merge all of this into a full \FLEET\Faceplus.
\end{exemple}

\begin{exemple}[Wood]
  Continuing the previous example, suppose that \SUE has one \RES{Wood} \MNU
  of level 1 and also buys a second \RES{Wood} from \ANG. Each of these
  \RES{Wood} increases its limit by 1\ND to a total of 6\ND. Thus, \SUE may
  now recruit 2\FLEET\Facemoins on the same turn (but still not a full
  \FLEET\Faceplus).
  
  If \SUE exploits \RES{Wood} in the \ROTW and has the possibility to buy
  another \RES{Wood} from \POL, that would make a total of 4 \RES{Wood}
  available. However, the limit may only be increased by 3\ND. That is, the
  fourth (and subsequent) \RES{Wood} is useless and, in this case, buying it
  is a waste of money.
\end{exemple}

\begin{exemple}[Galleys]
  In Period~\period{III}, \TUR may ``only'' recruit 9\ND per turn. However,
  \NGD count as half, so \TUR may recruit up to 18\NGD per turn! Its
  \FLEET\Faceplus can hold up to 5\ND and 1\NTD, that is 10\NGD and
  1\NTD. Thus, \TUR can largely buy 1\FLEET\Faceplus and 1\FLEET\Facemoins
  each turn\ldots 

  This is, typically, what happened after Lepanto where the Turkish navy was
  crushed but rebuilt in a couple of years. Given the high cost of both
  building and upkeeping navies, it is very rare to buy that much \ND during a
  given turn and things are usually more evenly spread over time.
\end{exemple}

\subsubsection{Naval Recruitment in the ROTW}
\aparag[Level 6 \COL] are considered as European provinces and follow the
normal rules for naval recruitment.

\aparag[Other establishments.] Each \ND (or \NTD) built in a \COL (of level
5 or less) or \TP costs twice the normal price and counts as 2\ND in the
recruitment limit.
\bparag When building \FLEET\Facemoins in the \ROTW, make sure that the
recruitment limit is high enough!
\bparag It is not possible to build navies in forts or missions (alone).

\aparag[Fisheries]\label{chExpenses:Effect of Fish Purchase} In a \COL
(including of level 6) where \RES{Fish} is exploited, up to 1\NDE per two
levels of \RES{Fish} exploited can be built outside of the construction limit,
at normal cost.

\begin{exemple}
  In Period~\period{IV}, \FRA has a \COL of level 4 in
  \granderegion{Terre-Neuve}, exploiting 4 \RES{Fish}. Thus, it may build
  there 2\NDE that do not count toward the recruitment limit (of 5\ND). If
  \FRA wants to build 2\ND there, that is 3\NDE plus 1\ND, the first two \NDE
  do not count toward the limit, but the third counts twice (thus, as a full
  \ND) and the \ND also counts twice, so this takes 3\ND of recruitment.

  If the technology of \FRA is \TBAT, this would cost 23\ducats (22.5 rounded
  up) for each of the first two \NDE, then the cost of the rest is
  doubled. This brings the total to 23 + 23 + 2 \textmultiply 23 + 2
  \textmultiply 55 = 167\ducats.
\end{exemple}

\subsubsection{Privateers}\label{chExpenses:Recruiting Privateers}
\aparag \corsaire may be recruited in any controlled and owned port, including
in the \ROTW.
\bparag Each side cost 10\ducats and counts as 1\ND toward recruitment limit.
\bparag When \corsaire are included in \terme{basic forces}, they are rebuilt
for free (both cost and limit) if destroyed (up to the basic forces).
\bparag Some \corsaire are also obtained via specific rules (see rules dealing
with each power, especially~\ref{chSpecific:France:Privateers}
and~\ref{chSpecific:Ragusa}). These are free of the costs mentioned here (both
in \ducats and construction limit) but the specific rules may entail specific
costs.

%%%%%%%%%%%%%%%%%%%%%%%%%%%%%%%%%%%%%%%%%%%%%%%%%%%%%%%%%%%%%%%%%%%%%%


\subsection{Exceptional Levies [should be moved in chMilitary]}
\label{chExpenses:Exceptional Levies}
% placeholder label for correct references.
\label{chMilitary:Exceptional Levies}

% TODO -- Check one more time...  PB 07/2008: seems ok...

\aparag[Declaring Exceptional Levies]
\bparag A country which is fully at war may declare \terme{Exceptional levies}
at the end of any round during which it suffered a major defeat in a land
battle. This is possible only during regular full wars (\emph{i.e.} not during
civil, religious or overseas wars and not during limited or foreign
interventions).
\bparag The country immediately loses 1 \STAB. Levies may be declared by a
country already at -3 \STAB, at no additional cost.
\bparag Once declared, levies are available for the rest of the turn. It is
not possible to declare Exceptional Levies several times during the same
turn.
\bparag[Exception:] \SUE and \PRU may declare levies after any defeat in a land
battle (not necessarily major). Moreover, one of them may declare levies after
a \textbf{major} defeat without paying 1 \STAB.
\bparag[Exception:] \POR may declare Exceptional Levies during \terme{Overseas
  Wars}, if it has a Vice Roy alive. There are specific conditions for these
levies, see~\ref{chSpecific:Portugal:Viceroys}.

\aparag[Recruitment during Exceptional Levies]
\bparag Once Exceptional Levies are declared, the Country may recruit land
forces at the end of each round of the turn, except the last one. This is done
during the End of turn segment, after the continuation roll.
\bparag The recruitment limit is halved (rounded up) and the number of \LD
recruited this turn is reseted to 0 when levies are declared.
\bparag Recruitment due to exceptional levies follows the normal recruitment
rules, with this new limit.
\bparag The cost of recruitment is written in \lignebudget{Exceptional
  recruitments}.
\bparag \SUE and \PRU do not recruit \terme{Veteran} troops with Exceptional
Levies.

\begin{exemple}
  During Period \period{IV}, \HIS has a land recruitment limit of
  5\LD. Exhausted by the Thirty Years War, it suddenly has to face \FRA in
  addition to the Protestants! Thus, it decides to recruit 6\LD (one at double
  cost) during the administrative phase. \HIS has a \STAB of only 0, due to
  the already long war. Its technology is \TBAR.
  
  Alas! At Rocroy, the Spaniards are hopelessly crushed by \leader{Grand
    Conde}, a major defeat. \HIS loses 1\STAB because of the defeat (thus
  going to -1) and then decides that there are way too many blue counters on
  the map and that adding some yellow ones is required. Thus, it declares
  exceptional levies. This brings the \STAB down to -2.

  Now, \HIS may recruit troops anew. The limit is halved (to 3\LD), but the
  number of troops recruited so far is reseted. \HIS decides to recruit an
  \ARMY\Faceplus, that is 4\LD. Since the limit is 3\LD, the fourth is doubled
  for a cost of 50 + 25 + 2 \textmultiply 25 = 125\ducats. 

  \smallskip

  On the next round, \HIS manages to avoid the main French armies and wage a
  war of attrition, thus suffering no new major defeat but still loosing some
  troops in skirmishes. Since levies are declared for the full turn, it may
  still recruit troop at the end of this round. The treasure fleet made its
  way safely from \continentAmerica, thus \HIS decides to raise another
  \ARMY\Faceplus, that is 4 new \LD. Since it has already recruited 4\LD last
  round with exceptional levies, and the limit is only 3\LD, that means that
  2\LD are below twice the limit and the last 2 are at triple price, for a
  cost of 2 \textmultiply 50 + 3 \textmultiply 50 = 250\ducats! Hopefully,
  disagreement within the French nobility will prevent this war from lasting
  too long\ldots

  \smallskip

  Later this turn \HIS suffers yet another major defeat at
  N\"{o}rdlingen. Since 8\LD were already recruited with Exceptional Levies,
  it is possible to recruit only one more. Thus, \HIS would very much like to
  declare Exceptional Levies a second time to reset the count of recruited
  troops. However, this is not possible and \HIS has to hold for the rest of
  the turn with what is left of its armies.
\end{exemple}

% Jym, 12/12:
% I don't think this is necessary anymore. RT can be <0, so there is no need
% to raise taxes in order to do levies (above).
% Moreover, this is a lot of exceptional rules (esp. with respect to Domestic
% operations) to express this correctly. Thus, I remove.

% \aparag[Exceptional Money Taxes]
% In addition to Loans or Gold transported during the round, a power may raise
% in a limited way money at the end of each round.
% \bparag If it lost a Major Defeat in a battle (land or at sea) during the
% round, or has declared Exceptional Levies, and if the Power has not taken
% \terme{Exceptional taxes} during the Administrative Phase, it may proceed to
% raise \terme{Exceptional taxes} during the turn. The rules followed are
% exactly those of~\ref{chExpenses:Exceptional Taxes}; the Stability of the
% Power can not be at -3 level; the Power immediately loses 1 \STAB level, note
% the current modifier and will receive money based on the test described
% in~\ref{chExpenses:Exceptional Taxes} at the "États au vrai" segment.
% \bparag Only one \terme{Exceptional taxes} may be raised during a given turn
% (be that during the Income Phase or the Military Phase).  If
% \terme{Exceptional taxes} are raised during the Military Phase, it would
% preclude any further Domestic operations on the next turn.

%%%%%%%%%%%%%%%


\subsection{Recruitment of Minor Powers}
\label{chLogistic:Recruitment of minors}
\subsubsection{Minors fully at peace}
\aparag A minor which is fully at peace (no war, overseas war or Limited
intervention) recruits up to its basic forces.
\bparag In addition, during periods \period{V}-\period{VII}, each minor
country with an \terme{Income} of 16 or more gets one extra level of fortress
in its basic forces.
\bparag Remember that troops in excess of the basic forces are disbanded
(except some fortresses), see~\ref{chLogistic:Maintenance of minors}.
\bparag Thus, it is not necessarily to put these counters on map. Once a minor
is at peace during the Administrative phase, all its counters (except some
fortresses) may be removed from the map. The next time the minor goes to war,
it will receive its basic forces.

\subsubsection{Minors in Limited Intervention or Overseas wars}
\aparag[Maximum recruitment]
\bparag A \MIN which is doing a \terme{Limited Intervention} or is involved in
an \terme{Overseas War} (and is not fully at war in some other war) recruits
troop if it has less than its \terme{Basic forces}.
\bparag Troops recruited during \terme{Limited Interventions} or
\terme{Overseas Wars} may not raise the total number of troops above the
\terme{Basic Forces}.
\bparag If the allowed reinforcement would bring the total number of troops
above the \terme{Basic Forces} of the country, then it only recruits up to its
\terme{Basic Forces} and excess reinforcement is lost.

\aparag[Recruitment of non-\VASSAL]
\bparag A \MIN which is either \Neutral, \RM, \SUB or \MA may recruit its
\terme{Basic Reinforcement} (indicated in the country description in the
Appendices).
\bparag A \MIN in \CE or \EW may recruit its \terme{Basic Reinforcement} plus
one \LD or \ND (controller's choice).
\bparag These recruitment do not cost anything to anybody. There are
considered as payed by the minor, whatever its actual income may be.

\aparag[Recruitment of \VASSAL]
\bparag A \MIN in \VASSAL gains no free reinforcement each turn.
\bparag Instead, the Patron may pay for reinforcements, on his own treasury,
to raise troops up to the basic forces of the country.
\bparag The cost are those of the Controller, with the technology of the
minor.
\bparag These troops are not counted toward the recruitment limit of the
major.
\bparag The maximal reinforcements so raised are the \terme{basic
  reinforcements}, plus 2 detachments (\LD or \ND).

\aparag[Moral]
\bparag All land reinforcements of \MIN are \terme{Conscripts}, except:
\bparag \paysSuede recruits all its new forces as \terme{Veteran};
\bparag \paysSuisse recruits its new forces as \terme{Veteran} if its Land
Technology is \TMUS or less;
\bparag \paysPerse recruits half of its new forces as \terme{Veteran} (round
down).

\aparag[Campaigns]
\bparag \Neutral minors in overseas wars or interventions have 1 active campaign
each round.
\bparag Countries that are neither \Neutral nor \VASSAL receive 1 passive
campaign each round, plus one active campaign for the turn. The controller may
pay for larger campaigns (paying the difference between the chosen campaign
and the passive one).
\bparag \VASSAL in overseas wars or interventions have no campaign. All their
campaign cost must be payed by their diplomatic patron. The patron may either
chose to move the minor's troops with its own campaign or pay a whole new
campaign only for the minor (in addition to the one used for its troops).

\subsubsection{Full Wars}
\aparag[Generalities]
\bparag Minor fully at war, whatever their diplomatic status, receive
reinforcements according to a \terme{Reinforcement roll}.
\bparag These roll are made during the \textbf{Administrative action of minors
  (incl. recruitment)} segment.
\bparag In case the order is relevant, each \MAJ, in decreasing order of
initiative, roll for reinforcements of its minors.
\bparag The reinforcement roll provides troops, campaigns, fortress levels and
leaders.

\aparag[Attitude]
\bparag Before rolling for reinforcements, the controlling player chooses an
\terme{Attitude} for each minor.
\bparag The choice is usually free (but may be constrained by
events). Typically, a minors that was declared war upon can choose an
\terme{Offensive} attitude.
\bparag The attitude chosen may change at each turn of the war.
\bparag Some attitudes entail constraints on the moves the country will be
allowed to do during the turn.

\aparag[List of attitudes]
\bparag[Offensive:] this attitude gives more troops and campaigns
% (Jym) too complicated to force players to do this without creating loopholes
% everywhere.
%
%  the minor country has to move at least a stack into enemy
% territory (i.e. the enemy player territory or the territory of one of the
% controlled minors of the assailant), if the concerned enemy territory is
% adjacent to the minor country territory, or within possible reach of same
% (i.e. maximum 12 MP away). The minor's move must be effective during the 1st
% or the 2nd round of the Military phase.
\bparag[Defensive:] the troops of the minor country may only move in provinces
that it owns or owned at some point in the game, as well as provinces adjacent
to the ones it currently owns. This is the best attitude for getting
fortresses.
\bparag[Naval:] this attitude may only be chosen for a minor country that has
naval counters at its disposal. It is the only attitude that gives naval
forces.
% (Jym) too complicated to force players to do this without creating loopholes
% everywhere.
%
% the minor must move at least one naval stack to undertake an
% attack against an enemy naval stack; or to establish a blockade or also to
% undertake transportation and/or landing of its land units into enemy
% territory.

\GTtable{minorforcesaw}

%\ref{table:Minor Reinforcements}

\aparag[Reinforcement roll]
\bparag Roll 1d10, add some modifiers as indicated on the right of the table
and cross reference the result in \ref{table:Minor Reinforcements} with the
attitude chosen.
\bparag This die roll gives troops, fortress levels, campaigns and a leader
value (in the last column).
\bparag[Political] There is a Political modifier specific to some minor
countries and periods (or events). These modifiers are indicated on the right
of \ref{table:Minor Reinforcements}, and also in the country's description in
the Appendix.
\bparag[Incomes] There is also an Economical modifier depending on the income
of the country. This modifier is used for all minors and is cumulative with
the Political modifier. It is based on the income of the provinces that are
owned and controlled by the minor, and neither besieged, revolted or pillaged
at the time the roll is made (\emph{i.e.} the provinces that would count in
the \terme{Land income} if this was a major country).

\aparag[Troops]
\bparag The reinforcement roll can give some \LD and \ND to the minor. The new
troops must be placed in owned and controlled provinces that are neither
besieged nor revolted.
\bparag \LD may be freely converted into \ARMY as the usual rate of
2\LD=1\ARMY\facemoins, 4\LD=1\ARMY\faceplus.
\bparag \ND may be used for either 1 \NWD, 1 \NTD, 1 \VGD or 2 \NGD.
\bparag Naval forces can be included into \FLEET, according to the fleet size
of the minor.
\bparag There is no limit to the amount of recruited troops other than the
counter limit for the country.

\aparag[Moral]
\bparag All land reinforcements of \MIN are \terme{Conscripts}, except:
\bparag \paysSuede recruits all its new forces as \terme{Veteran};
\bparag \paysSuisse recruits its new forces as \terme{Veteran} if its Land
Technology is \TMUS or less;
\bparag \paysPerse recruits half of its new forces as \terme{Veteran} (round
down).

\aparag[Fortresses.] Some levels of fortresses are obtained by the
Reinforcement Roll. The new levels must be placed in controlled provinces that
are neither besieged nor revolted.
\bparag In addition, during periods \period{V}-\period{VII}, each minor
country with an \terme{Income} of 16 or more gets one extra level of fortress
in its \terme{Basic Forces}.
\bparag A given fortress can not be improved by Reinforcements by more than
one level at a given turn. This rule does not constrain the fortresses that
are in the \terme{Basic Forces}.
\bparag Before \TARQ, a fortress of level 3 can only be placed by minor
countries at the cost of 2 levels of fortress on top of an existing level 2.
\bparag Before turn 40, a fortress of level 4 can only be placed by minor
countries at the cost of 2 levels of fortress on top of an existing level 3.

\aparag[Campaigns]
\bparag Each minor fully at war gets 1 active campaign each round.
\bparag In addition, it may receive multiple campaigns (MC) per reinforcement
roll.
\bparag The diplomatic patron may pay for more campaigns.

\aparag[Military Leaders of Minors: basic forces]
\bparag If there is a living named leader of the country, he automatically
comes into play.
\bparag Some minors have military leaders in their \terme{basic forces}. If
there are not enough named leaders to reach this limit, the minor receive
\anonyme\ leaders. If possible, take these among the minor country pool,
otherwise among the generic grey leaders (of ``country'' \textsc{Quidam}).
\bparag Contrary to major countries, \anonyme\ leaders of minors are not
changed each turn. They are only removed when the country is at peace. If, due
to death, the country falls below its basic forces (in number of leaders), it
immediately receive a new one.

\aparag[Military Leaders of Minors: reinforcements]
\bparag By cross-referencing the (modified) reinforcement roll with the last
column in~\ref{table:Minor Reinforcements}, one gets a \terme{Leader value}.
\bparag 1d10 is rolled. If less or equal than this value, the minor receive a
leader for the duration of the war. This does not change its basic forces
(that is, no replacement if the leader is killed).
\bparag If the attitude of the minor is \terme{Naval}, then it receives either
a \anonyme\LeaderA or a \anonyme\LeaderG. Otherwise, it receives a
\anonyme\LeaderG.
\bparag This leader is taken among those of the minor, if some exist and among
the generic grey ones otherwise.
\bparag Like other \anonyme\ leaders of minor, the leader will be available
for this minor until it is fully at peace.

\aparag[Military Leaders of Minors: double-sided monarchs]
\bparag Several Minors have generic double-sided monarchs. These leaders are
usually in the basic forces of the minor.
\bparag These are treated like \anonyme\ leaders. That is, one of the side is
chosen (at random) when the minor is activated and it is kept until the minor
is fully at peace (or until the death of the leader).
\bparag List of concerned leaders (and countries): \leader{Caliph}
(\paysMamelouks), \leader{Giray} (\paysCrimee) and \leader{Shah}
(\paysPerse).
\bparag Note that \leader{Grand Maitre} (\paysChevaliers) is not concerned as
it is the same leader that may serve either as \LeaderG or \LeaderA.

\aparag[Military Leaders of Minors: named and generic monarchs]
\bparag Some minors have two (or more) different counters for their monarch,
usually an unnamed one (as above) and a named one.
\bparag The named one replace the unnamed one when he is alive. He enters game
either at a given turn or following certain rules or event.
\bparag As long as a named monarch is available for a minor, the unnamed one
is not available and may not enter game (even through reinforcements).
\bparag List of concerned leaders (and countries): \leader{La Valette}
replaces \leader{Grand Maitre} (\paysChevaliers), \leader{Abbas Shah} and
\leader{Nadir Shah} replace \leader{Shah} (\paysPerse) and \leader{Akbar}
replaces \leader{Great Mughal} (\paysMogol).
\bparag Note that the \anonyme\LeaderG of \paysDanemark and \paysUSA are not
concerned. They are always available for these countries as reinforcement
leaders even if there are some named leaders alive.

\begin{exemple}[Minor reinforcements]
  At turn 10, \paysAlgerie is at peace. Thus, whatever it had left on previous
  turn, it rebuilds and keeps only its basic forces of \ARMY\facemoins and
  \FLEET\facemoins (the \corsaire is left out of this example). There is no
  need to keep these counters on map. Since it is a \VASSAL of \TUR, \TUR also
  chooses to pay for maintaining a fortress of level 2 in
  \provinceAlgerie. This counter has to be put on map as a reminder of the
  existence of this fortress. The technology of \paysAlgerie is \TREN.

  At turn 11, \ref{pIII:Revolt Grenade} occurs (earlier than historically) and
  \TUR decides to go to war, together with its ally \paysAlgerie.

  \textbf{Basic forces.} First, the basic forces of \paysAlgerie are put on
  map. \TUR puts the \ARMY\facemoins and the \FLEET\facemoins both in
  \provinceAlgerie. Then, since \leader{Barbaros2} is still alive, he is also
  put in play (or, probably, kept).

  \textbf{Reinforcements.} Next, \TUR rolls for reinforcements of its minor
  (after all majors have finished their administrative actions, especially
  after \TUR and \HIS both have brought their own troops). Since \paysAlgerie
  is one of the \Barbaresques, it gets a \bonus{+1} political DRM (in periods
  \period{I}-\period{III}). Since its total income is 17\ducats (it still own
  all its original provinces and no more), it also gets a \bonus{+1}
  economical DRM (for income between 16 and 30\ducats).

  \TUR chooses a \terme{Naval} attitude for \paysAlgerie as it wants to try
  and invade Spain. The roll is 5, modified to 7. Thus, \paysAlgerie gets
  1\LD, 1 level of fortress and 1\ND. The leader value is 2. \TUR rerolls on
  die for the leader and gets a 2, smaller than the leader value, thus
  \paysAlgerie gets an extra leader for the duration of the war. Since it
  already has an \LeaderA (\leader{Barbaros2}), \TUR chooses to take a
  \LeaderG for \paysAlgerie (among the grey \anonyme\LeaderG). \TUR cannot
  raise the fortress of \provinceAlgerie to level 3, indeed at \TREN this
  would cost 2 levels of fortresses and \paysAlgerie received only one. Since
  it it not possible either to ``keep'' the level for a further turn, \TUR
  rather chooses to put it in \provinceOran (and place a grey level 2 fortress
  there). As for the troops, \TUR chooses to take 2\NGD instead of 1\ND and
  put them directly in the \FLEET (and note on its record sheet the exact
  content of the Algerian \FLEET). The \LD is put in \provinceAlgerie, waiting
  to board for an invasion of Spain\ldots
\end{exemple}

% (Jym) all this is said at the correct place earlier in this chapter.

% \section{Deployment [to be moved in chMilitary]}

% \aparag All purchased units must be deployed at the end of the recruitment
% phase, before reinforcements of \MIN. All non-maintained units must be removed
% from the map.
% \bparag The purchases are deemed to be simultaneous. In case of ambiguity, it
% is better to write down all purchases locations.
% \bparag Units that are maintained as \terme{Veteran} must be marked as such,
% including units that are only partially \terme{Veteran} (and mark how many \LD
% in it are \terme{Veteran}).
% \bparag This deployment segment is accompanied by a reorganisation segment in
% which leaders can move any distance (unless encircled or under siege) to
% maintain hierarchy (see \ruleref{chMilitary:Leadership}), units can be
% combined.
% \bparag Nations with a limited recruitment area (\SPA, \TUR and \RUS) have no
% stacking limits in this phase (but stacking must be respected by the end of
% the first military round). Stacking limits must otherwise be respected (see
% \ruleref{chMilitary:Stacking}).
% \bparag There is no limited recruitment area for sea units, unless specified
% otherwise. However, the new naval units must be built in a port in owned
% territory in Europe (see above for construction in the \ROTW).



% \subsection{Deployment of leaders [to be moved in ch. VI]}
%
% TODO + Hiérarchie.

%%%%%%%%%%%%%%%%%%%%%%%%%%%%%%%%%%%%%%%%%%%%%%%%%%%%%%%%%%%%%%%%%%%%%%
% \section{Logistics in French}
% {\bf Dans V.4}

% {\bf E1 placement rotw}

% Les E et C peuvent être placés en début de tour en métropole, ou dans une
% COL/TP établie, ou dans une région découverte où se trouvait un chef du même
% type à  la fin du tour précédent.  Les Gouv et G(rotw) doivent être mis sur
% des COL/TP déjà  établis.  Les missionnaires restent en place sans
% redéploiement, et apparaissent toujours en métropole.

% {\bf E2 placement des non nommés}

% Les chefs ? sont tirés au début de chaque tour pour combler si en dessous de
% la limite et tous les chefs sont placés en même temps en respectant la
% hiérarchie (seuls les chefs pris dans un siège n'ayant pas de port sans
% blocus restent nécessairement sur place).

% Si pendant les rounds, on tombe à  nouveau en-dessous des limites, un chef ?
% de la bonne catégorie est tiré pour remplacer le dernier de ce type tué ou
% blessé au round précédent. Il doit être mis au même endroit pour les A et G;
% pour les E et C, on les met en métropole ou sur un TP/COL; ensuite seulement
% on doit règler les pb de hiérarchie comme on peut. Quand un blessé revient,
% le chef non nommé de sa catégorie de rang le plus bas est enlevé, et le chef
% revenant doit prendre commandement d'une pile en respectant la hiérarchie.

% {\bf E3 chefs gouverneurs / changement de C}

% Certains généraux ou C sont en fait des gouverneurs.

% Ils agissent comme des C mais entrent dans une limite de chefs
% spécifiques. Ils obéissent de plus aux règles suivantes : - ils doivent être
% placés sur un TP/COL du joueur ou qu'il occupe militairement en début de
% tour, ou en métropole. Si en métropole, ils doivent débarquer dans rotw sur
% TP/COL du joueur (ou contrôlé militairement).  - ils donnent un bonus de
% +man aux implantations dans leur province, et de +1 dans les autres
% provinces de la grande région.  - ils sont limités aux mouvements dans la
% grande région où ils sont placés en début de tour (ou celle où ils
% débarquent) et les grandes régions adjacentes.

% Local Variables:
% fill-column: 78
% coding: utf-8-unix
% mode-require-final-newline: t
% mode: flyspell
% ispell-local-dictionary: "british"
% End:

% LocalWords: Carrack Tercios Galleass Boyars
