\sectionJ{\anchorpaysmajeur{France}}{\blasonJ{france}}\label{chSpecific:France}

\subsection{Military assets Overseas}
\subsubsection{French Privateers}\label{chSpecific:France:Privateers}
\aparag \FRA can use only one \corsaire counter if not using the
following rules.
\aparag[French Buccaneers.]
\bparag From period II onward, a second \corsaire counter can be raised
\Facemoins and placed in \stz{Caraibes}, and \Faceplus from period III
onward. This \corsaire may only be placed in discovered seas.
\bparag This is not possible if \FRA has a \COL\faceplus on any sea in
the \STZ, or if \eventref{pV:Colbertian Mercantilism} already happened,
or if \monarque{Louis XIV} is or was once king of \FRA.
\aparag[Licensed Privateers.]
\bparag Beginning with the reign of \monarque{Louis XIV}, or event
\eventref{pV:Colbertian Mercantilism}, \FRA can grant licenses to raise
more than one \corsaire.
\bparag Each license gives right to raise one more \corsaire counter,
but lowers the number of \ND that \FRA can recruit this turn by 2 \ND
(instead of the usual one \ND needed for a \corsaire)
and uses one \FLEET\facemoins of the \terme{basic forces}. Up to 3
licenses can be given.
\bparag Each license given allows \FRA to draw one Privateer Admiral
from those available at this turn. If none are, one unnamed Privateer
Admiral (of hierarchical rank X) can be used (at most one, even if more
than one License is accorded).
\bparag The named Privateer Admirals are \leader{Bart},
\leader{Esnambuc}, \leader{Estrees}, \leader{Forbin},
\leader{Duguay-Trouin}, \leader{Cassard} and \leader{Suffren}.

\subsubsection{French Missionaries and Missions}
\aparag See \ruleref{chSpecific:Missions} for the general rules.
\aparag \FRA receives Missionaries, one every even turn, beginning with
period IV if \terme{Catholic}, and period III if \terme{Protestant}. It
has a maximum of 4 Missionaries if \terme{Catholic}, and 2 if
\terme{Protestant}.
\aparag French Missions give a bonus of {\bf +2} to the diplomacy on
minor countries in the \ROTW, or to raise French Indian allies (see
below), when used as emissaries (thus the global bonus is {\bf +5}
instead of {\bf +3}).

\subsubsection{French Cipayes}
\longCipayes{Cipayes}
\aparag[\sectionleader{Dupleix} and \sectionleader{Bussy}]
\bparag Leaders \leader{Dupleix} and \leader{Bussy} can use the table of
conquistadors (\tableref{table:Conquistadors Effects}) in
\continent{India}. Forces with those 2 leaders never cause Activation of
Natives in \continent{India}.
\bparag \terme{Cipayes} raised or maintained in the province where
\leader{Dupleix} is are always Veterans (even if only 1\ducats is paid).
\bparag The minimum \LeaderC in period VII can also use the table of
conquistadors in \continent{India}.
\bparag If neither \leader{Dupleix} nor \leader{Bussy} is in game, 
the minimum \LeaderC$\!$@ in period VII can then use the table of 
conquistadors in \continent{India} up to Turn 58 (included).


\subsubsection{French Indian allies}
\aparag[How to raise them] French Indian allies may be obtained only in
\granderegion{Quebec} or \granderegion{Grands Lacs}. At the end of each
administrative phase, \FRA may roll 1d10 for each \Area, add {\bf +2} if
there is a Mission therein, and {\bf + ?} the \Man of an emissary in the
\Area (\LeaderC, \LeaderGov, \LeaderMis or Mission), and substract {\bf
  -1} for each \TP\faceplus of \COL (any level) of other powers in the
\Area.  On a result of 7 or more, an Indian ally \LD is placed in any
\TP or \COL of the \Area.
\bparag If \FRA did eliminate any Indian in the \Area, it can raise no
more Indian Ally therein (note that in case of Activation of Natives, it
can choose not to defend itself, even with colonial militia).
\bparag Indian allies can not go outside \granderegion{Quebec} or
\granderegion{Grands Lacs} or adjacent regions. They can not be
incorporated in army counters.
\bparag Indian allies are always withdrawn at the end of the turn. They
may come back on the following turn by the same mechanism.
\aparag[Military advantages]
\bparag They are \LD of \FRA, sharing its \terme{Land Technology}. They
never cause reactions of Natives or of minor countries in the
\ROTW. They are not counted for checking the conditions of reactions.
\bparag Any stack in which they are have a \Man of 5 (or 6 if the leader
already has 6).
\bparag If they are alone in attack, they are not adversely affected by
terrain.
\bparag After any battle, Indian allies are withdrawn from the map and
replaced at the end of the next round in any french \COL or \TP in
\granderegion{Quebec} or \granderegion{Grands Lacs} (their region) that
is free of enemy.

\subsubsection{French Colonial Militia}
\aparag French colonial militia (one \LDE for each 2 levels of \COL --
round up) are \terme{Veteran}.

\subsubsection{Few acres of snow}
\aparag \FRA may annex all establishments (\COL and \TP) of its enemies
in an Area in \continentAmerica, North of \granderegionChichimeca
(excluded) at peace.
\bparag This count as 1 peace condition, plus 1 per establishment not
controlled by \FRA in the Area at the time of the peace.

\subsection{\sectionpaysmajeur{France} in play}

\subsubsection{Monarchs of France}
\aparag[\anchormonarque{Charles VIII}] is the monarch in 1492 is
\monarque{Charles VIII}, with values 5/7/9, scheduled to die at the
beginning of turn 4.
\aparag[\anchormonarque{Francois I}] is the first French Monarch after
\eventref{pI:War Italy Napoli}. He has values 5/8/9 and is a general
whose military values are rolled as usual. His reign will last at least
4 turns (if less is rolled, consider it is 4 turns).
\aparag[\anchormonarque{Henri IV}.] At the end of \eventref{pIII:FWR},
the French Monarch will be either \anchormonarque{Henri de Navarre},
\monarque{Henri IV} or \anchormonarque{Henri de Guise}. See especially
\subeventref{pIII:FWR:Designation Heir} for all details. The heir is
either 6/9/7 or 9/9/9.
\aparag[\anchormonarque{Louis XIV}] is the first French king after
\eventref{pIV:Richelieu} or \eventref{pIV:Fronde}. He has values 7/6/9
but he is a Baby at the beginning, with a length of reign of 12 turns.
He will make no test of survival during the 5 first turns. He can not be
used as a general. When he is adult, \FRA adds a free maintenance of one
\ARMY\faceplus and one \ND.
\aparag[Revolution.] Event \eventref{pVII:French Revolution} may
overthrow the French king and replace him with a somewhat republican
government. This governement is represented by either
\anchormonarque{Convention} (values 3/6/7) or \anchormonarque{Terror}
(values 5/6/9). None of them roll for survival, neither can they be used
as generals.
\subsubsection{Ministers of France}
\aparag[\anchorministre{Richelieu}] may be named minister through
\eventref{pIV:Richelieu}. He has values 9/8/7 and remains a random
number of turns; the successor of the current monarch will be
\monarque{Louis XIV}.
\aparag[\anchorministre{Mazarin}] may be named minister through
\eventref{pIV:Fronde}. He has values 7/8/7 and remains till
\monarque{Louis XIV} becomes an adult.
\aparag[\anchorministre{Colbert}] may be named minister through
\eventref{pV:Colbertian Mercantilism}. He has values 8/9/8 and remains a
random number of turns.
\subsubsection{Versailles}
\aparag During %
% (Jym)
%the reign ->
the adulthood of \monarque{Louis XIV}, all expenses put in Prestige are
multiplied by 150\% for the construction of Versailles
% (Jym)
% Je commente nos discussions internes. Je laisse juste la version compta v2.
%\\
%---- WAS:
%NOTE :  beaucoup de PV assez faciles. Passer � 200\ducats par tour ???
%\\
%PB: En fait sere repris a la lumiere des nouvelles regles de tresor et de PV.
%\aparag \FRA may build Versailles in \province{Ile-de-France} from the
%reign of \monarque{Louis XIV} (when adult) or event
%\eventref{pV:Colbertian Mercantilism}.
%\aparag The construction costs 150\ducats per turn, during 5 turns.
%\aparag If \province{Ile-de-France} is pillaged during the construction,
%it will stop temporary as long as the marker remains. If
%\province{Ile-de-France} is military controlled by an enemy during the
%construction, it is stopped completely. The construction will have to
%resume at the beginning.
%\aparag If Versailles is finished at the end of period V, \FRA wins
%100\PV. If Versailles is finished in period VI or VII, \FRA wins only
%50\PV. \\
%------

\subsubsection{Available counters}
\aparag[Military] 6\ARMY, 5\FLEET, 4\corsaire (only 1 available at
start), 15\LDND, 5\LD, 4\NTD, 8\LDENDE, 5 fortresses 1/2, 5 fortresses
2/3, 6 fortresses 3/4, 4 fortresses 4/5, 11 forts, 2 Arsenals 2/3, 2 Arsenals 3/4, 
4 Missions (2 only if \terme{Protestant}), 5 \terme{Cipayes} \LD (and 3 \terme{Cipayes} \LDE),
2 \terme{Indian Allies} \LD counters (and 4 \terme{Indian Allies} \LDE).
\aparag[Economical] 14\COL, 10\TP, 14\MNU, 18\TradeFLEET, 4\ROTW treaty
counters.

% LocalWords:  Caraibes pV Colbertian Esnambuc Estrees Forbin Duguay Trouin Ier
% LocalWords:  Cassard Suffren Cipayes Dupleix Bussy cois pI Napoli pIII FWR de
% LocalWords:  pIV Ile
