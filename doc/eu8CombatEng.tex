\chapter{The new rapid battle system}

\section{Presentation}

This new system for land battle is a modification of the one which must be
presented in the 2nd extension of Europa Universalis still to come. He
must allowed to completly avoid the army size for all the game aspects
(execpt for little thing on the RotW map).

The rules written by Ph. thibaut are used except for some ajustements we
propose which are indicated in {\it italic} above.

Those modifications (as all our propositions) add two new players and some
new potential major powers (Poland, Sweden, Prussia, Russia in period I
and II), those nations appear in all the tables.

\section{Rapid battle}

In this system, all land forces are evaluated in equivalent number of
detachement for all game purpose : combat, but also logistic, stacking and
attrition, in Europa as in RotW.

Let do a sum up of the combat : it take place in 1 or 2 round, whitout to
have to calculated the fire and shock factors. In each round, the two
players roll a dice for the fire and a dice for the shock ({\it if nobody 
break morale after the fire}), on the new CRT, colomns are determined by
technologies (the usual way). The losses are in equivalent number of  
detachment and in moral points.

If nobody break moral at the end of the first round nor try to retreat 
during the battle, a second round take place on the same way but with a
$-1$ modificator to the dices for the two players. The losing army must
undergo a pursuit {\it even if it didn't break morale}.

The losses suffered by each side are then calculated ({\it see above for
all modificators of army size and type}\ref{PertesVariables}). The looser can suffered
more losses du to retreat (calculated by a roll on the retreat table).
All the losses are then applicated to the armies.

\subsection{Description of Armies, morale, and technologies}

The \textbf{moral} of a stack is the one of the majority of the forces, 
compted in number of detachement. In case of equality, the moral of the 
majority of
army counters are considerated first. A new equality indicate a conscript
morale. The same procedure is used for technology determination, the   
worst is used in case of equality.

Each token belong to an army \textbf{class} (called \CAI to CAIV, with some particularities called \CAIM, \CAIIM, \CAIVM and 
\CAA) which describe in a certain way the evolution of size and doctrine of the armies and determined its size (from 0 to 
7) function of the period.

The \textbf{size} of a stack is the average (round down) of the size of each token. This average is calculated with the number of 
detachment as coeficient (an A+ worth 4 times more than a LD), and round down.

{\bf Exception :} If pachas are in the stack, the moral is always
concript, technology and army class are determined following the normal
procedure.

\subsection{Battle sequence}
\noindent
The battle begin after the possible roll for interception, retreat
before battle and attrition.\\
{\bf A.} Combat rounds are simultaneous. If at the end of one of the 4 
rounds, one
army break morale (also call rout, this means a morale of 0 or less), {\it
then directly go to C.1}.\\
1. First round of fire : each side roll a modified dice on the fire
colomn. Losses take by the 2 sides are keeping.\\
2. First round of shock : each side roll a modified dice on ths shock
column. Losses take by the 2 sides are added to the ones of the first   
round of fire.\\
{\bf B.} Possibility for each side to retreat, defender then attacker.
If at this time, there's enough losses to eliminated one side (after 
modification by variable losses), the combat ends, and 
directly
go to C.\\
3. Second round of fire : each side roll a modified dice with a -1
modificator. Losses are added. \\
4. Second round of shock : each side rolle a modified dice with a -1
modificator. Losses are added. \\   
{\bf C.1} If an army break morale and not the other, then do a pursuit (E
column). \\
{\bf C.2} If an army as a lesser moral than the other, but is not in rout, it
losses the battle {\it the winner do a pursuit with can lead to a rout}. \\
{\bf C.3} If the two armies have the same final moral, or if the two break
morale, then its a draw, each camp go back where it come from, and there's
no pursuit nor retreat.\\
{\bf D.} Losses of each sides are totalised (from the 4 rounds, and the pursuit)
which are modificated by the size of the army which make the losses,
{\it and then by the comparison of its class and the class of the other
army}.\\
{\bf E.} The loser of the battle (which must retreat), do an attrition test for
the retreat qui can add 1/2 or 1  to its losses. The general maneuver are
used for this test if the army is not in rout. \\
{\bf F.}  Losses are round down except for 1/2 which become 1.\\
{\bf G.} General losses test are doing according to the usual rule. possibility
of major battle.

\subsection{During the battle rounds}

\paragraph{1. Technologie and fire.}
- An army with medieval technology do not roll for fire.\\
- \textbf{Bertrand's version :} An army with renaissance technology always roll
for fire but just do moral losses.\\
- \textbf{Pierre's version :} An army with renaissance technology just do
moral losses for fire, but if there's no army counter, then it do not roll  
for fire. \\
- In arquebuse, the losses of fire roll are halfed (round down). \\
- After, the losses are the one indicated

\paragraph{2. Modificators.}
they are indicated on the table juste near the CRT.

\paragraph{3. Cavalery advantage.}
Each stack contain a number of cavalry which depend on the syze and the
quantity of cavalery per equivalent detachment. A value {\it which depend
on the nationality of the army (see above)}, is multipicated by the number
of equivalent detachment, and then lead to the total number of cavalry in
the stack. LD contain the same number of cavalry per detachment than the 
A.

If a side has at least 2 times more cavalry of the other, it as a
+1 modificator to the shock and the pursuit if the battle is in plain (if
it is defenser or attacker).

\cavalerie

\paragraph{4. Pursuit.}
Pursuit dice roll are affected by the differential of maneuvre (possibly
negative), the terrain, the cavalry avantadge, and the way the battle ends 
\\: +1 is the losser break in fire, \\

\paragraph{5. Retreating in the middle of battle\label{RetraiteEnBataille}}
During segment B. of the battle, between the first round of shock and the
second round of fire, a side can decide to retreat. The defender, as the  
possibility to try, and if it decline, the attacker can do it.

A dice roll lesser than the general maneuver plus the left moral of the   
stack allowed to end battle now ; the one which retreat is the looser. the
end of the combat is resolving by segment C.2 and the followers. If the  
test is a failure, the battle continue and the other side as a bonus of +2
to its following fire dice roll.

\subsection{Losses variation}
\label{PertesVariables}

The result of losses is the sum of all the result of the roll of fire,
shock and pursuit (but whitout retreat), leading to a number of detachment
taking by the other side. But this table is calculated for doing the
number of losses doing by a stack of 2 A+ to an army of the same syze and
type. Noticed that to have more than 8DT in a stack (for Turk with Pachas)
do not provide any advantage : no suplementary loss (this is a difference
with the rules for natives, see above).

{\bf 1.} In order to take the real side of the army into account, the table of
variable losses are consulted (see combat tables) which indicate how much 
losses must be substract to obtain the final number of losses. {\bf
Another limitation also applied now : the number of losses doing by a
stack can't be more than the size of the stack in equivalent detachement.}

{\bf 2.}{\it the type of each army are now compared}. There's army groups
which are the followings :

%\classes
\classesnew

The following table indicate the differential du to the type of each army.
Lines and columns which usally served are put in bolt font , the ones which served
only when different armies are are in normal font 
The army which takes the losses is in horizontal, the other is in vertical ; 
this table is symetric to the diagonal with a sign change.

%\matricetaille 
\comparaisonnew

\textit{Algorithm : divided the sums of the two armies by 3 and round to the nearest to obtain the positive modificateur
accorded to the greastest army.}

\hspace*{-0cm}
\noindent
\pertetaillenew



{\bf 3.}
C. The real losses which are take into account are the ones indicated in
the following table, ligne 0 corresponding to the number of losses
calculated in A. before the modification du the the comparison between
army type : \\

{\bf 4.}
The number of losses doing by the retreat dice roll is added to the
number calculated in C. the losses a round down execpt for 1/2 which is
round to 1, and is the number of equivalent detachment loss by the
losing side.  

\subsection{Who win the battle}

Differents ends of the battle are explain in the battle sequence and 
detailled in the following page.

\newpage\null\newpage

\textbf{The winner of the battle.}

{\bf C.1} If a side break morale (morale fall to 0 or less), and not the 
orther at the end of a round, it lose the battle, and the
other side win it. a pursuit is doing by the winner, losses are adjusted. 
The losser retreat in an adjacent friend province and do 
$attrition test without the help of the maneuver of its general.\\

{\bf C.2} If no side break moral after the 4 rounds of combat, the side 
with the less morale left lose the battle. {\it The winner
do a pursuit}. The losses du to pursuit can lead to a rout, in this case, 
the procedure is the same than in C.1. If there's no rout $the losser 
retreat in a friend adjacent province and do an attrition test modifed by 
the maneuver of its general.\\

{\bf C.3} If the two sides have the same moral at the end of the battle, 
or break moral at the same time, each side go back where
they come from. This mean a siege can continue, a stack which just move or 
intercept come back in the province where it was just
before battle. There's no pursuit, no attrition, it's a draw and there's 
no winner.\\

A {\bg major victory} is granted is the looser effectively lose   
\textbf{3} detachment more than the winner (after all 
modifications), or \textbf{4} LD if the loser has a size modificator of -2.

Repartition of losses are doing by the player who take them whatever he 
want. He can destroy army token (one A+=4D) or doing all 
is possible for keep them (for exemple 2 A+ taken 4 losses can become 2 A-,
1 A+, or 1 A- and 2D).

\section{Besiege}
The rules are praticaly unchange from the rules of 2nd extension.

\subsection{The assault}

\paragraph{Assault rounds.}
The assault is doing in two rools, one for the fire, and one for shock, 
except that the shock is not doing by a side which break
moral after fire. CRT shows one specific column for besieger and one for
besieged.

\paragraph{Modificators.}
Besieger add 1 if defender is medieval, substarct 1 if defender is  
arquebuse or better, to its fire and its shock.
Besiger substract the level of the fortres to the two dice rolls if
there's no breach, the artillery can reduce this malus for
the fire uniquely :
\begin{itemize}
\item substract 1 to the malus of the forteresse if besieger as at least 4 times this level in artillery.
\item substract 2 to the malus of the forteresse if besieger as at least 6 times this level in artillery.
\end{itemize}

\paragraph{Losses adjustment.}
\begin{itemize}
\item Besieger take 1/2 more losse if it break morale.
\item Besieged do one 1 losses less to the fire and the shock if the
assault follow a \textbf{breach}.
\item If the besieger doesn't have 2 A+, the table of variable losses
shows the modificator to apply to losses.
\item Turquie and Russia until 1614, and the poland until 1559 do 1/2 more 
losse if they do an assault with 1 A+, and 1 more losse
with 2 A+.
\end{itemize}

\paragraph{Fortress resistance.}
The losses do to the besieged are firstly taken by unit enclosed in the 
fortress and secondly to the resistance of the fortress.
This resistance is equal to the level of the fortress, and is reduce in 
cas of
breach. The resistance recover its full level after each
assault.

\paragraph{The victory.} The victory side is determined by the following  
priority order :
\begin{enumerate}
\item Besieged, if the besieger is eliminated;
\item Besieger, if the resistance fall to 0, or if the besieged break morale (even if besieger rout) ;
\item Besieged, in any other case.
\end{enumerate}

A capture fortress is replaced by a fortress of one level less (except if 
it is the original fortress printed on the map)

\artillerie

\subsection{Siegework action}
\noindent
The siege by sape is not modified at all. An army counter on its + side 
is always supposed to contain a number of artillerie 
equal to the one of the maximum size of its nation in the considereted 
period (see the table). An army on the - side contain the 
half  (round down) of the artillery contain in an army on its + side.

To maintain the siege on a fortress, a number of equivalent detachment 
equal to the level of the fortress is needed. If the 
besieger can't maintain the siege at the end of a round (after an assault 
or a battle), he must immediatly retreat in a friendly province
(before he can loot).

%\end{minipage}
%\begin{minipage}[t]{0.5\linewidth}
%\artillerie
%\end{minipage}

\section{Mouvement's Rules}

\subsection{Stacking}
The maximum stacking is 3 counters by province for one side, there is
another limit of 8 equivalent detachement of a single side. Those limits
must be respected at the end of the mouvement of each stack.

\subsection{Attrition}

For the Attrition, the retreat table is used with
multiplicating the results by 2. \\
{\bf Exception} : A stack of one or two equivalent detachment use the
table whitout doubling the losses. An half losses correspond to nothing in
Europe and two 5 strengh point on the Rest Of The World mal (see above).

\attrition

\subsection{Reorganisation of stack}
During the mouvement phase of a side, he has the possiblility to
integrated detachments army counters on the - side. Two detachements
transform - side army two  a + side army. It is possible to split army
counters whitout midificating the number of equivalent detachement. those
adjustements can be doing at any time during the mouvement.

It is forbidden to create a new army counter during the mouvement. 2  
Detachment can't become an A-, and an A+ can't split into two A-. It is
possible two destroy army counter during reorganisations.

Forces which are intercepting an ennemy can reorganised in the beginning
area(to let forces in the province) and in the ending area (to integrated
forces already there). The general who made the interception must follow
intercepting forces.

The moral level of amalgamed force (in a A+) must be noted on a paper,  
the number of veteran equivalent detachament (the rest is conscript). The
moral of army counters must also be noted.


\subsection{Naval Transport}
A fixed number of transport points is needed in a fleet to transport an
army. This number is indicated in the following table, this is the number
af transport point needed to transport an A-. Detachement always needed 10
transport point.

\maritime

There's another limitation, one fleet counter can never transport more
than 4 equivalent detachment, or 8 equivalent detachment if the fleet
counter contain galley instead of warship.

\subsection{Overrun}
A stack of 2 A+  entering a province which contain only one detachement 
can declared an overun battle. The combat is immediattly solve and if the
active stack win, it can continue its mouvement.

A force entering into a province contening a unbesieged ennemy fortress
Must stop here or left here enough force to conduct the siege (1
detachment per level).


\section{Puchasing troops}
Troops are purchased uniquely in entire counters. The purchased limits   
perperiod have been adapted in the final table ; they are expressed in
number of detachment.

The purchase cost of a detachment is now always the half (round down) of
the purchased cost of an A-. The other prices of purchase and maintenance
are unchange. Never forget that army counter must be purchase as army
counter and cannot be created during the reorganisation. during the
logistic phase reorganisation are possible before and after maintenance
and purchase.

\recrutement

\section{Rest Of The World}

All the preceding rules are in application to the unit on the rest of the
worl map, except for several adjustement for the detachements moving alone
and for natives. the principal adjustement is that a detachemnt on the rest
of the world map take losses by 1/10 of detachemnt : it contain 10 strengh
points.

\subsection{Natives}
In battle, natives forces are converted into equivalent detachment by the
way : 15 points of native=1 equivalent detachment. The total of detachment
is round up. They are classe i troops.

If a native forces has more than 8 equivalent detachment, it roll one dice
for each entire fraction of 8 detachment it counts, and one for the last
incomplete fraction. Don't forget to applied variable losses to the last
fraction which in general do not count 8 equivalent detachment.

Natives losses are 15 strengh point by losse and 5 for an half losse
(which is not round).

\subsection{Exploration}
Attirtion losses du to exploration by a force contain more than two
detachment or an army are transcript into equivalent detachement losse to   
satisfy. A losse of 0.5 detachement indicated that one detachment is
reduce of 5 strengh points. When a stack contain less than 20 strengh
points (in general 2 Detachment and no army), the attrition is given in
percent for mouvement and exploration on the original table.

\hspace*{-10mm}
\exploration

\subsection{Forts}

A Fort is a level zero fortress for siegework action. Any forces is
sufficiant to do the siege. Durine an assault, his resistance is 1/2   
except if there's a breach, in this case theresistance is 0 : the assault
is always a succes whitout any combat if there's no troups in the fort.


\subsection{Land detachments}

Attrition of land detachment do to usure, exploratio (but not mouvement)
or the losses of combat are counted in strengh forces if it is in the     
restof the worl map. A land detachement contin at  the beginning 10 
strengh points. The losses, indicated in percent, normaly applied. Losses
indicated in detachment (in combat and usure) are of 5 points for each 1/2
losse taken and are not round when its a LD which must take it.


\subsubsection{Logistic}

It is not possible to purchase detachment with less than 10 points or to
completed the one which have less than 10 strengh point. Uncomplteted
detachment can be join in a unique detacment in the logistic phase. The
maintenance of an uncompleted detachment is the same than complet one. 

an uncomplete detachment can be completed by paying the cost of the entire
one. There's is no need to pay the maintenance in this case. the advantage
is that if the counters contain a least 5 strengh point, the new complete
counter is a veteran one.

an incomplete detachment which come back in Europe must be completed by 
this way at the logistic phase or destroy, it can't be just maintain.     

The purchase is limited to 1 LD in each rest of the world province, at   
double cost.  Army counters can't be purchased in colonies or trading
posts. there's an execption for level 6 colonies : 2 LD can be purchased
here at normal cost or 1 A- at double cost.

\subsubsection{In Battle}
A stack in the rest of the world containing uncomplete detachement count
for 1 LD for 10 strengh point, 1LD if the force has less than 10 strengh
point. The maximum number of losses that a rest of the world stacking do
if it not contain an army counter is the numerb of strengh point it
contain. this apply for the combat against natives.

Against native, cavalry advantage is not granted if the european country 
has less than one LD and a half (this mean 15 strengh points), whitout
included colonial militia.

\subsubsection{\textit{53.14}Cossacks}
When Russia benefit of cossacks help, following effect are applied :    
\begin{itemize}
\item free maintenance of 2 LD conscript in siberia
\item free purchased of a LD in siberia or cossacks provinces each turn. 
\end{itemize}

%\end{document}
